\documentclass{amsart}

\usepackage{amssymb}
\usepackage[all]{xy}
\usepackage{verbatim}
\usepackage{ifthen}
\usepackage{xargs}
\usepackage{bussproofs}
\usepackage{turnstile}

\renewcommand{\turnstile}[6][s]
    {\ifthenelse{\equal{#1}{d}}
        {\sbox{\first}{$\displaystyle{#4}$}
        \sbox{\second}{$\displaystyle{#5}$}}{}
    \ifthenelse{\equal{#1}{t}}
        {\sbox{\first}{$\textstyle{#4}$}
        \sbox{\second}{$\textstyle{#5}$}}{}
    \ifthenelse{\equal{#1}{s}}
        {\sbox{\first}{$\scriptstyle{#4}$}
        \sbox{\second}{$\scriptstyle{#5}$}}{}
    \ifthenelse{\equal{#1}{ss}}
        {\sbox{\first}{$\scriptscriptstyle{#4}$}
        \sbox{\second}{$\scriptscriptstyle{#5}$}}{}
    \setlength{\dashthickness}{0.111ex}
    \setlength{\ddashthickness}{0.35ex}
    \setlength{\leasturnstilewidth}{2em}
    \setlength{\extrawidth}{0.2em}
    \ifthenelse{%
      \equal{#3}{n}}{\setlength{\tinyverdistance}{0ex}}{}
    \ifthenelse{%
      \equal{#3}{s}}{\setlength{\tinyverdistance}{0.5\dashthickness}}{}
    \ifthenelse{%
      \equal{#3}{d}}{\setlength{\tinyverdistance}{0.5\ddashthickness}
        \addtolength{\tinyverdistance}{\dashthickness}}{}
    \ifthenelse{%
      \equal{#3}{t}}{\setlength{\tinyverdistance}{1.5\dashthickness}
        \addtolength{\tinyverdistance}{\ddashthickness}}{}
        \setlength{\verdistance}{0.4ex}
        \settoheight{\lengthvar}{\usebox{\first}}
        \setlength{\raisedown}{-\lengthvar}
        \addtolength{\raisedown}{-\tinyverdistance}
        \addtolength{\raisedown}{-\verdistance}
        \settodepth{\raiseup}{\usebox{\second}}
        \addtolength{\raiseup}{\tinyverdistance}
        \addtolength{\raiseup}{\verdistance}
        \setlength{\lift}{0.8ex}
        \settowidth{\firstwidth}{\usebox{\first}}
        \settowidth{\secondwidth}{\usebox{\second}}
        \ifthenelse{\lengthtest{\firstwidth = 0ex}
            \and
            \lengthtest{\secondwidth = 0ex}}
                {\setlength{\turnstilewidth}{\leasturnstilewidth}}
                {\setlength{\turnstilewidth}{2\extrawidth}
        \ifthenelse{\lengthtest{\firstwidth < \secondwidth}}
            {\addtolength{\turnstilewidth}{\secondwidth}}
            {\addtolength{\turnstilewidth}{\firstwidth}}}
        \ifthenelse{\lengthtest{\turnstilewidth < \leasturnstilewidth}}{\setlength{\turnstilewidth}{\leasturnstilewidth}}{}
    \setlength{\turnstileheight}{1.5ex}
    \sbox{\turnstilebox}
    {\raisebox{\lift}{\ensuremath{
        \makever{#2}{\dashthickness}{\turnstileheight}{\ddashthickness}
        \makehor{#3}{\dashthickness}{\turnstilewidth}{\ddashthickness}
        \hspace{-\turnstilewidth}
        \raisebox{\raisedown}
        {\makebox[\turnstilewidth]{\usebox{\first}}}
            \hspace{-\turnstilewidth}
            \raisebox{\raiseup}
            {\makebox[\turnstilewidth]{\usebox{\second}}}
        \makever{#6}{\dashthickness}{\turnstileheight}{\ddashthickness}}}}
        \mathrel{\usebox{\turnstilebox}}}

% \providecommand\WarningsAreErrors{false}
% \ifthenelse{\equal{\WarningsAreErrors}{true}}{\renewcommand{\GenericWarning}[2]{\GenericError{#1}{#2}{}{This warning has been turned into a fatal error.}}}{}

\newcommand{\newref}[4][]{
\ifthenelse{\equal{#1}{}}{\newtheorem{h#2}[hthm]{#4}}{\newtheorem{h#2}{#4}[#1]}
\expandafter\newcommand\csname r#2\endcsname[1]{\ref{#2:##1}}
\expandafter\newcommand\csname R#2\endcsname[1]{#4~\ref{#2:##1}}
\newenvironmentx{#2}[2][1=,2=]{
\ifthenelse{\equal{##2}{}}{\begin{h#2}}{\begin{h#2}[##2]}
\ifthenelse{\equal{##1}{}}{}{\label{#2:##1}}
}{\end{h#2}}
}

\newref[section]{thm}{theorem}{Theorem}
\newref{lem}{lemma}{Lemma}
\newref{prop}{proposition}{Proposition}
\newref{cor}{corollary}{Corollary}

\theoremstyle{definition}
\newref{defn}{definition}{Definition}
\newref{example}{example}{Example}

\theoremstyle{remark}
\newref{remark}{remark}{Remark}

\newcommand{\red}{\Rightarrow}
\newcommand{\deq}{\Leftrightarrow}
\renewcommand{\ll}{\llbracket}
\newcommand{\rr}{\rrbracket}
\newcommand{\cat}[1]{\mathbf{#1}}
\newcommand{\C}{\cat{C}}
\newcommand{\Set}{\cat{Set}}
\newcommand{\ccat}{\cat{CCat}}
\newcommand{\syntt}{\cat{SynTT}}
\newcommand{\algtt}{\cat{AlgTT}}
\newcommand{\Mod}{\text{-}\cat{Mod}}
\newcommand{\Sig}{\cat{Sig}}
\newcommand{\Th}{\cat{Th}}
\newcommand{\ttvdash}{\vartriangleright}

\numberwithin{figure}{section}

\newcommand{\pb}[1][dr]{\save*!/#1-1.2pc/#1:(-1,1)@^{|-}\restore}
\newcommand{\po}[1][dr]{\save*!/#1+1.2pc/#1:(1,-1)@^{|-}\restore}

\begin{document}

\title{Algebraic and Syntactic Presentations of Type Theories}

\author{Valery Isaev}

% \begin{abstract}
% Abstract
% \end{abstract}

\maketitle

\section{Introduction}

\begin{comment}
\section{Contextual categories}

Contextual categories were defined by Cartmell \cite{GAT}.
An equivalent definition was given by Voevodsky in \cite{c-systems}.
In this section we will give another equivalent definition which is just a description of models of the initial algebraic type theory as we will see later.
\end{comment}

\section{Essentially algebraic theories}

There are several equaivalent ways of defining essentially algebraic theories (\cite{LPC}, \cite{GAT}, \cite{PHT}, \cite[D 1.3.4]{elephant}).
We will use approach introduced in \cite{PHT} under the name of partial horn theories since it seems to be the most general.
In this section we will review necessary for our development parts of the theory of partial horn theories and define a notion of morphisms between them suitable for our purposes.

A signature $\Sigma = (\mathcal{S}, \mathcal{F}, \mathcal{P})$ consists of a set $\mathcal{S}$ of sorts, a set $\mathcal{F}$ of function symbols, and a set $\mathcal{P}$ of predicate symbols.
Also $\Sigma$ assigns a signature to each $f \in \mathcal{F}$ which is written as $f : s_1 \times \ldots \times s_n \to s$ where $s_1$, \ldots $s_n$, $s$ are sorts.
Finally $\Sigma$ assigns a signature to each $R \in \mathcal{P}$ which is written as $R : s_1 \times \ldots \times s_n$ where $s_1$, \ldots $s_n$ are sorts.
For each $V \in \Set^\mathcal{S}$ we can define a set $Term_\Sigma(V)_s$ of terms of sort $s$ inductively:
\begin{itemize}
\item If $x \in V_s$, then $x \in Term_\Sigma(V)_s$.
\item If $f : s_1 \times \ldots \times s_n \to s$ and $t_i \in Term_\Sigma(V)_{s_i}$, then $f(t_1, \ldots t_n) \in Term_\Sigma(V)_s$.
\end{itemize}
Then $Term_\Sigma(V)$ is actually a functor $Term_\Sigma : \Set^\mathcal{S} \to \Set^\mathcal{S}$.
Moreover, there is a natural monad structure on $Term_\Sigma$.

Let $\Sigma = (\mathcal{S}, \mathcal{F}_\Sigma, \mathcal{P})$ and $\Sigma' = (\mathcal{S}, \mathcal{F}_{\Sigma'}, \mathcal{P})$ be a pair of signatures with the same sets of sorts and predicate symbols.
Then we define morphisms of these signatures as morphisms between monads $Term_\Sigma$ and $Term_{\Sigma'}$.
Since $Term_\Sigma$ is a free monad there is a bijection between such morphisms and functions that to each $f \in \mathcal{F}_\Sigma$ assign a term of $\Sigma'$ with the same signature as $f$.
We will write such functions $F$ as $F(f(x_1, \ldots x_n)) = t$ where $f \in \mathcal{F}_\Sigma$, $f : s_1 \times \ldots \times s_n \to s$ and $t \in Term_{\Sigma'}(\{ x_1 : s_1, \ldots x_n : s_n \})_s$.
The category of sinatures will be denoted by $\Sig$.

An \emph{atomic formula} over $\Sigma$ with free variables in $V$ is an expression either of the form $t_1 = t_2$ where $t_1, t_2 \in Term_\Sigma(V)_s$ for some sort $s$ or of the form $R(t_1, \ldots t_n)$ where $R \in \mathcal{P}$, $R : s_1 \times \ldots \times s_n$ and $t_i \in Term_\Sigma(V)_{s_i}$.
A \emph{Horn sequent} over $\Sigma$ is an expression of the form $\varphi \sststile{}{V} \psi$ where $\varphi$ and $\psi$ are finite sets of atomic formulae over $\Sigma$ with free variables in $V$.
A \emph{Horn theory} in a signature $\Sigma$ is a set of Horn sequents over $\Sigma$.
In \cite{PHT} were described the inference rules of the partial Horn logic.
If $\mathbb{T}$ is a Horn theory, then a \emph{theorem} of $\mathbb{T}$ is a sequent derivable from $\mathbb{T}$ in this logic.

Let $F$ be a morphism of signatures $\Sigma$ and $\Sigma'$.
Then for each formula $\varphi$ over $\Sigma$ we can define formula $F(\varphi)$ over $\Sigma'$ just by applying $F$ to every term in $\varphi$.
If $\varphi \sststile{}{V} \psi$ is a sequent over $\Sigma$, then $F(\varphi) \sststile{}{V} F(\psi)$ is a sequent over $\Sigma'$.
Let $\mathbb{T}$ and $\mathbb{T}'$ be Horn theories in signatures $\Sigma$ and $\Sigma'$ respectively.
Then we will say that $F$ \emph{respects} a sequent $\varphi \sststile{}{V} \psi$ if $F(\varphi) \sststile{}{V} F(\psi)$ is a theorem of $\mathbb{T}'$.
Note that if $F$ respects all axioms of $\mathbb{T}$, then it respects all of its theorems too.
We will say that morphisms $F$ and $G$ are \emph{equivalent} if for all $f \in \mathcal{F}_\Sigma$, $f : s_1 \times \ldots \times s_n \to s$ sequent $\sststile{}{x_1, \ldots x_n} F(f(x_1, \ldots x_n)) = G(f(x_1, \ldots x_n))$ is a theorem of $\mathbb{T}'$.
Note that if $F$ and $G$ are equivalent then $\sststile{}{V} F(t) = G(t)$ is a theorem of $\mathbb{T}'$ for each $t \in Term_\Sigma(V)_s$.

Morphisms of theories $\mathbb{T}$ and $\mathbb{T'}$ are equivalence classes of morphisms of signatures $\Sigma$ and $\Sigma$ which respect all axioms of $\mathbb{T}$.
The composition of morphisms of signatures respects the equivalence relation; hence this defines a category of theories which will be denoted by $\Th$.
Note that $\Sig$ is a full subcategory of $\Th$; indeed, every signature can be considered as a theory with an empty set of axioms.

\label{sec:T1}
\section{Theory $T_1$}

In this section we will describe a quasi-equational theory $T_1$ and prove that the category of models of $T_1$ is equivalent to the category of contextual categories.
Later we will use this theory to define algebraic type theories.

We will consider quasi-equational theories in signatures with the set of sorts $\mathcal{C} = \{ Ctx_n\ |\ n \in \mathbb{N} \} \cup \{ Tm_n\ |\ n \in \mathbb{N} \}$.
Let $T_0$ be the theory with the set of function symbols $\{ * \} \cup \{ ft_n : Ctx_{n+1} \to Ctx_n\ |\ n \in \mathbb{N} \} \cup \{ ty_n : Tm_n \to Ctx_{n+1}\ |\ n \in \mathbb{N} \}$ and the following axioms:
\begin{align*}
& \top \sststile{}{A} ft_n(A) \downarrow \\
& \top \sststile{}{a} ty_n(a) \downarrow \\
& \top \sststile{}{} * \downarrow \\
& \top \sststile{}{A} A = *
\end{align*}
Let $ft^i_n : Ctx_{n+i} \to Ctx_n$ be the following derived operation:
\begin{align*}
& ft^0_n(A) = A \\
& ft^{i+1}_n(A) = ft^i_n(ft_{n+i}(A))
\end{align*}

Now we describe theory $T_1$ which contains.
The set of function symbols of $T_1$ consists of the symbols of $T_0$ and the following symbols:
\begin{align*}
v_{n,i}     & : Ctx_n \to Tm_n \text{, } 0 \leq i < n \\
Subst_{n,k} & : Ctx_n \times Ctx_{k+1} \times Tm_n^k \to Ctx_{n+1} \\
subst_{n,k} & : Ctx_n \times Tm_k \times Tm_n^k \to Tm_n
\end{align*}

Auxiliary predicates $Hom_{n,k} : Ctx_n \times Ctx_k \times Tm_n^k$ are defined as follows: $Hom_{n,k}(B, A, a_1, \ldots a_k)$ holds if and only if
\[ ty_n(a_i) = Subst_{n,i-1}(B, ft^{k-i}_i(A), a_1, \ldots a_{i-1}) \text{ for each } 1 \leq i \leq k \]
The idea is that a tuple of terms should represent a morphism in a contextual category.
So $Hom_{n,k}(B, A, a_1, \ldots a_k)$ holds if and only if $(a_1, \ldots a_k)$ is a morphism with domain $A$ and codomain $B$.
Note that if $Hom_{n,k}(B, A, a_1, \ldots a_k)$, then $ft_n(ty_n(a_i)) = B$.

The set of axioms of $T_1$ consists of the axioms of $T_0$ and the axioms we list below.
The following axioms describe when functions are defined:
\begin{align}
\label{ax:def-var}
                                             & \sststile{}{A}           v_{n,i}(A) \downarrow \\
\label{ax:def-Subst}
Hom_{n,k}(B, ft_k(A), a_1, \ldots a_k)       & \ssststile{}{B, A, a_i}  Subst_{n,k}(B, A, a_1, \ldots a_k) \downarrow \\
\label{ax:def-subst}
Hom_{n,k}(B, ft_k(ty_k(a)), a_1, \ldots a_k) & \ssststile{}{B, a, a_i}  subst_{n,k}(B, a, a_1, \ldots a_k) \downarrow
\end{align}

The following axioms describe the ``typization'' of the constructions we have:
\begin{align}
\label{ax:type-var}
& \sststile{}{A}         ty_n(v_{n,i}(A)) = Subst_{n,n-i-1}(A, ft^i_{n-i}(A), v_{n,n-1}(A), \ldots v_{n,i+1}(A)) \\
\label{ax:type-Subst}
& \sststile{}{B, A, a_i} ft_n(Subst_{n,k}(B, A, a_1, \ldots a_k)) \leftrightharpoons B \\
\label{ax:type-subst}
& \sststile{}{B, a, a_i} ty_n(subst_{n,k}(B, a, a_1, \ldots a_k)) \leftrightharpoons Subst_{n,k}(B, ty_k(a), a_1, \ldots a_k)
\end{align}

The following axioms prescribe how substitution ($Subst_{n,k}$ and $subst_{n,k}$) must be defined on indices ($v_{n,i}$):
\begin{align}
\label{ax:Subst-var}
& \sststile{}{A}         Subst_{n,n}(ft_n(A), A, v_{n,n-1}(ft_n(A)), \ldots v_{n,0}(ft_n(A))) = A \\
\label{ax:subst-var}
& \sststile{}{a}         subst_{n,n}(ft_n(ty_n(a)), a, v_{n,n-1}(ft_n(ty_n(a))), \ldots v_{n,0}(ft_n(ty_n(a)))) = a \\
\label{ax:var-subst}
& Hom_{n,k}(B, A, a_1, \ldots a_k) \sststile{}{B, a_i, A} subst_{n,k}(B, v_{k,i}(A), a_1, \ldots a_k) = a_{k-i}
\end{align}

The following axioms say that substitution must be ``associative'':
\begin{align}
\label{ax:Subst-Subst}
& Hom_{n,k}(C, B, b_1, \ldots b_k) \land Hom_{k,m}(B, ft_m(A), a_1, \ldots a_m) \sststile{}{C, b_i, B, a_i, A} \\ \notag
& Subst_{n,k}(C, Subst_{k,m}(B, A, a_1, \ldots a_m), b_1, \ldots b_k) = \\ \notag
& Subst_{n,m}(C, A, subst_{n,k}(C, a_1, b_1, \ldots b_k), \ldots subst_{n,k}(C, a_m, b_1, \ldots b_k)) \\
\label{ax:subst-subst}
& Hom_{n,k}(C, B, b_1, \ldots b_k) \land Hom_{k,m}(B, ft_m(ty_m(a)), a_1, \ldots a_m) \sststile{}{C, b_i, B, a_i, a} \\ \notag
& subst_{n,k}(C, subst_{k,m}(B, a, a_1, \ldots a_m), b_1, \ldots b_k) = \\ \notag
& subst_{n,m}(C, a, subst_{n,k}(C, a_1, b_1, \ldots b_k), \ldots subst_{n,k}(C, a_m, b_1, \ldots b_k))
\end{align}

Now, we want to show that the category of models of $T_1$ is equivalent to the category of contextual categories.
First, we construct a functor $F : T_1\Mod \to \ccat$.
Let $M$ be a model of $T_1$.
Then the set of objects of level $n$ of $F(M)$ is $M(Ctx_n)$.
For each $A \in M(Ctx_n)$, $B \in M(Ctx_k)$ morphisms from $A$ to $B$ are tuples $(a_1, \ldots a_k)$ such that $a_i \in M(Tm_n)$ and $Hom_{n,k}(A, B, a_1, \ldots a_k)$.

For each $0 \leq i \leq n$ axiom~\eqref{ax:type-var} implies
\[ \sststile{}{A} Hom_{n,n-i}(A, ft^i_{n-i}(A), v_{n,n-1}(A), \ldots v_{n,i}(A)). \]
For each $A \in M(Ctx_n)$ we define $id_A : A \to A$ as tuple
\[ (v_{n,n-1}(A), \ldots v_{n,0}(A)) \]
and $p_A : A \to ft(A)$ as tuple
\[ (v_{n,n-1}(A), \ldots v_{n,1}(A)). \]

Now, we introduce some notation.
If $B \in M(Ctx_n)$, $A \in M(Ctx_{k+1})$, and $f = (a_1, \ldots a_k) : B \to ft_k(A)$ is a morphism, then we define $A[f] \in M(Ctx_{n+1})$ as $Subst_{n,k}(B, A, a_1, \ldots a_k)$.
If $a \in M(Tm_k)$ and $ty_k(a) = A$, then we define $a[f] \in M(Tm_n)$ as $subst_{n,k}(B, a, a_1, \ldots a_k)$.
By axioms \eqref{ax:def-Subst} and \eqref{ax:def-subst} these constructions are total.

If $A \in M(Ctx_n)$, $B \in M(Ctx_k)$, $C \in M(Ctx_m)$, $f : A \to B$, and $(c_1, \ldots c_m) : B \to C$, then we define composition $(c_1, \ldots c_m) \circ f$ as $(c_1[f], \ldots c_m[f])$.
The following sequence of equations shows that $(c_1, \ldots c_m) \circ f : A \to C$.
\begin{align*}
ty_n(c_i[f]) & = \text{(by axiom~\eqref{ax:type-subst})} \\
ty_k(c_i)[f] & = \text{(since $Hom_{k,m}(c_1, \ldots c_m)$)} \\
ft^{m-i}_i(C)[c_1, \ldots c_{i-1}][f] & = \text{(by axiom~\eqref{ax:Subst-Subst})} \\
ft^{m-i}_i(C)[c_1[f], \ldots c_{i-1}[f]] &
\end{align*}

With these notations we can rewrite axioms \eqref{ax:type-subst}, \eqref{ax:Subst-var}, \eqref{ax:subst-var}, \eqref{ax:Subst-Subst}, \eqref{ax:subst-subst} as follows:
\begin{align}
\setcounter{equation}{\ref{ax:type-subst}}
\addtocounter{equation}{-1}
ty_n(a[f]) & = A[f] \\ \notag
\text{ for each } f : B \to ft_k(A) & \text{ where } A = ty_k(a) \\
A[id_{ft_n(A)}] & = A \\
a[id_{ft_n(ty_n(a))}] & = a \\
\setcounter{equation}{\ref{ax:Subst-Subst}}
\addtocounter{equation}{-1}
A[g][f] & = A[g \circ f] \\ \notag
\text{ for each } f : C \to B \text{ and } & g : B \to ft_m(A) \\
a[g][f] & = a[g \circ f] \\ \notag
\text{ for each } f : C \to B \text{ and } & g : B \to ft_m(ty_m(a))
\end{align}

Associativity of the composition follows from axiom~\eqref{ax:subst-subst}, and the fact that $id$ is identity for it follows from axioms \eqref{ax:subst-var} and \eqref{ax:var-subst}.

For every $A \in M(Ctx_{k+1})$ there is a bijection $\varphi$ between the set of $a \in M(Tm_k)$ such that $ty_k(a) = A$ and the set of morphisms $f : ft_k(A) \to A$ such that $p_A \circ f = id_{ft_k(A)}$.
For every such $a \in M(Tm_k)$ we define $\varphi(a)$ as
\[ (v_{k,k-1}(ft_k(A)), \ldots v_{k,0}(ft_k(A)), a). \]
Note that if $(a_1, \ldots a_{k+1}) : B \to A$ is a morphism, then axiom~\eqref{ax:var-subst} implies that $p_A \circ (a_1, \ldots a_{k+1})$ equals to $(a_1, \ldots a_k)$.
Thus $\varphi(a)$ is a section of $p_A$.
Clearly, $\varphi$ is injective.
Let $f : ft_k(A) \to A$ be a section of $p_A$; then first $k$ components of $f$ must be identity on $ft_k(A)$.
So if $a$ is the last component of $f$, then $\varphi(a)$ equals to $f$.
Hence $\varphi$ is bijective.

If $A \in M(Ctx_{k+1})$, $B \in M(Ctx_n)$, and $f = (a_1, \ldots a_k) : B \to ft_k(A)$, then we define $f^*(A)$ as $A[f] = Subst_{n,k}(B, A, a_1, \ldots a_k)$.
Map $q(f,B)$ defined as the tuple with $i$-th component equals to
\[ \left\{
  \begin{array}{lr}
    a_i[v_{n+1,n}(A[f]), \ldots v_{n+1,1}(A[f])] & \text{ if } 1 \leq i \leq k \\
    v_{n+1,0}(A[f])                              & \text{ if } i = k+1
  \end{array}
\right. \]
Now we have the following commutative square:
\[ \xymatrix{ A[f] \ar[r]^-{q(f,A)} \ar[d]_{p_{A[f]}} & A \ar[d]^{p_A} \\
              B \ar[r]_-f                             & ft_k(A)
            } \]
We need to prove that this square is cartesian.
By proposition~2.3 of \cite{c-systems} it is enough to construct a section $s_{f'} : B \to A[f]$ of $p_{A[f]}$ for each $f' = (a_1, \ldots a_k, a_{k+1}) : B \to A$ and prove a few properties of $s_{f'}$.
We define $s_{f'}$ to be equal to $\varphi(a_{k+1})$.
Axioms \eqref{ax:var-subst} and \eqref{ax:subst-subst} implies that $q(f, B) \circ s_{f'} = f$.
To complete the proof that the square above is cartesian we need for every $g : ft_k(A) \to ft_m(C)$ and $A = C[g]$ prove that $s_{f'} = s_{q(g,C) \circ f'}$.
The last component of $q(g,C) \circ f'$ equals to $v_{n+1,0}(C[g])[f'] = a_{k+1}$.
Thus the last components of $q(g,C) \circ f'$ and $f'$ coincide, hence $s_{f'} = s_{q(g,C) \circ f'}$.

We are left to prove that operations $A[f]$ and $q(f,A)$ are functorial.
Equations $A[id_{ft_k(A)}] = A$ and $A[f \circ g] = A[f][g]$ are precisely axioms \eqref{ax:Subst-var} and \eqref{ax:Subst-Subst}.
The fact that $q(id_{ft_k(A)}, A) = id_A$ follows from axiom~\ref{ax:var-subst}.
Now let $g : C \to B$ and $f : B \to ft_k(A)$ be morphisms; we need to show that $q(f \circ g, A) = q(f,A) \circ q(g,A[f])$.
The last component of $q(f,A) \circ q(g,A[f])$ equals to $v_{n+1,0}(A[f])[q(g,A[f])] = v_{m+1,0}(A[f][g])$ which equals to the last component of $q(f \circ g, A)$, namely $v_{m+1,0}(A[f \circ g])$.
If $1 \leq i \leq k$, then $i$-th component of $q(f,A) \circ q(g,A[f])$ equals to
\[ a_i[v_{n+1,n}(A[f]), \ldots v_{n+1,1}(A[f])][q(g,A[f])] = \]
\[ a_i[b_1[v_{m+1,m}(A[f][g]), \ldots v_{m+1,1}(A[f][g])], \ldots b_n[v_{m+1,m}(A[f][g]), \ldots v_{m+1,1}(A[f][g])]] \]
where $a_i$ is $i$-th component of $f$ and $b_i$ is $i$-th component of $g$.
$i$-th component of $q(f \circ g, A)$ equals to
\[ a_i[g][v_{m+1,m}(A[f \circ g]), \ldots v_{m+1,1}(A[f \circ g])] = \]
\[ a_i[b_1[v_{m+1,m}(A[f \circ g]), \ldots v_{m+1,1}(A[f \circ g])], \ldots b_n[v_{m+1,m}(A[f \circ g]), \ldots v_{m+1,1}(A[f \circ g])]]. \]
Thus $q(f \circ g, A) = q(f,A) \circ q(g,A[f])$.
This completes the construction of contextual category $F(M)$.

\begin{prop}
Mapping $F$ is functorial, and functor $F : T_1\Mod \to \ccat$ is an equivalence of categories.
\end{prop}
\begin{proof}
Given a map of $T_1$ models $\alpha : M \to N$, we define a map of contextual categories $F(\alpha) : F(M) \to F(N)$.
$F(\alpha)$ is already defined on objects.
Let $f = (a_1, \ldots a_k) \in Hom_{n,k}(B,A)$.
We define $F(\alpha)(f)$ as $(\alpha(a_1), \ldots \alpha(a_k)) \in Hom_{n,k}(\alpha(B), \alpha(A))$.
$F(\alpha)$ preserves identity morphisms, compositions, $f^*(A)$, and $q(f,A)$ since all of these operations are defined in terms of $T_1$ operations.
Clearly, $F$ preserves identity maps and compositions of maps of $T_1$ models.
Thus $F$ is a functor.

First, note that if $a \in M(Tm_k)$ and $\alpha : M \to N$, then $F(\alpha)(\varphi(a)) = \varphi(\alpha(a))$.
Indeed, consider the following sequence of equations:
\begin{align*}
F(\alpha)(\varphi(a)) & = \\
F(\alpha)(v_{k,k-1}(ft_k(ty_k(a))), \ldots v_{k,0}(ft_k(ty_k(a))), a) & = \\
(v_{k,k-1}(ft_k(ty_k(\alpha(a)))), \ldots v_{k,0}(ft_k(ty_k(\alpha(a)))), \alpha(a)) & = \\
\varphi(\alpha(a)) & .
\end{align*}

Now, we prove that $F$ is faithful.
Let $\alpha,\beta : M \to N$ be a pair of maps of $T_1$ models such that $F(\alpha) = F(\beta)$.
Then $\alpha$ and $\beta$ coincide on contexts.
Given $a \in M(Tm_n)$ we have the following equation: $\alpha(a) = \varphi^{-1}(F(\alpha)(\varphi(a))) = \varphi^{-1}(F(\beta)(\varphi(a))) = \beta(a)$.

Now, we prove that $F$ is full.
Let $\alpha : F(M) \to F(N)$ be a map of contextual categories.
Then we need to define $\beta : M \to N$ such that $F(\beta) = \alpha$.
If $A \in M(Ctx_n)$, then we let $\beta(A) = \alpha(A)$.
Note that if $f : ft_n(A) \to A$ is a section of $p_A$, then $\alpha(f)$ is a section of $\alpha(A)$.
If $a \in M(Tm_n)$, then we let $\beta(a) = \varphi^{-1}(\alpha(\varphi(a)))$.

Maps $F(\beta)$ and $\alpha$ agree on contexts.
We prove by induction on $k$ that they coincide on morphisms $f = (a_1, \ldots a_k) \in M(Hom_{n,k})(B,A)$.
If $k = 0$, then $F(A)$ is terminal objects, hence $F(\beta) = \alpha$.
Suppose $k > 0$ and consider the following equation: $f = q((a_1, \ldots a_{k-1}), A) \circ \varphi(a_k)$.
By induction hypothesis we know that $F(\beta)(q((a_1, \ldots a_{k-1}), A)) = \alpha(q((a_1, \ldots a_{k-1}), A))$.
Thus we only need to prove that $F(\beta)(\varphi(a_k)) = \alpha(\varphi(a_k))$.
But $F(\beta)(\varphi(a_k)) = \varphi(\beta(a_k)) = \varphi(\varphi^{-1}(\alpha(\varphi(a_k)))) = \alpha(\varphi(a_k))$.

Finally, we prove that $F$ is essentially surjective on objects.
Given contextual category $C$ we define $T_1$ model $M$.
Let $M(Ctx_n)$ be equal to $Ob_n(C)$ and $M(Tm_n)$ be the set of pairs of objects $A \in Ob_{n+1}(C)$ and sections of $p_A : A \to ft_n(A)$.
Let $ty_n$ be the obvious projection.
We will usually identify $a \in M(Tm_n)$ with the section $ft_n(ty_n(a)) \to ty_n(a)$.

For each $n,k \in \mathbb{N}$ we define partial function
\[ Subst_{n,k} : M(Ctx_n) \times M(Ctx_{k+1}) \times M(Tm_n)^k \to M(Ctx_{n+1}) \]
such that $ft_n(Subst_{n,k}(B, A, a_1, \ldots a_k)) = B$.
We also define morphism
\[ q_{n,k} \in Hom_{n+1,k}(Subst_{n,k}(B, A, a_1, \ldots a_k), A) \]
whenever $Subst_{n,k}(B, A, a_1, \ldots a_k)$ is defined.
We define $Subst_{n,k}$ and $q_{n,k}$ by induction on $k$.
Let $Subst_{n,0}(B,A) = !_B^*(A)$ and $q_{n,0} = q(!_B,A)$ where $!_B : B \to Ob_0(C)$ is the unique morphism.
\[ \xymatrix{ Subst_{n,0}(B,A) \ar[r]^-{q_{n,0}} \ar[d] \pb & A \ar[d]^{p_A} \\
              B \ar[r]_{!_B} & 1
            } \]
Let $Subst_{n,k+1}(B, A, a_1, \ldots a_{k+1})$ be defined whenever $Subst_{n,k}(B, ft_k(A), a_1, \ldots a_k)$ is defined and $ty_n(a_{k+1}) = Subst_{n,k}(B, ft_k(A), a_1, \ldots a_k)$.
In this case we let $Subst_{n,k+1}(B, A, a_1, \ldots a_{k+1}) = f^*(A)$ and $q_{n,k+1} = q(f,A)$ where $f$ is the composition of $a_{k+1}$ and $q_{n,k}$.
\[ \xymatrix{ Subst_{n,k+1}(B, A, a_1, \ldots a_{k+1}) \ar[rr]^-{q_{n,k+1}} \ar[d] \pb & & A \ar[d]^{p_A} \\
              B \ar[r]_-{a_{k+1}} & Subst_{n,k}(B, ft_k(A), a_1, \ldots a_k) \ar[r]_-{q_{n,k}} & ft_k(A)
            } \]
It is easy to see by induction on $k$ that axiom~\eqref{ax:def-Subst} holds.
Axiom~\eqref{ax:type-Subst} holds by definition of $Subst_{n,k}$.

The definition of predicates $Hom_{n,k}$ makes sense in $M$ now.
Thus we can define as before the set $Hom^M_{n,k}(B,A)$ of morphisms in $M$ as the set of tuples $(a_1, \ldots a_k)$ such that $Hom_{n,k}(B, A, a_1, \ldots a_k)$.
There is a bijection $\alpha : Hom^M_{n,k}(B,A) \to Hom_{n,k}(B,A)$ such that $Subst_{n,k}(B, A, a_1, \ldots a_k) = \alpha(a_1, \ldots a_k)^*(A)$ and $q_{n,k} = q(\alpha(a_1, \ldots a_k), A)$.
We define $\alpha$ by induction on $k$.
Both $Hom^M_{n,0}(B,A)$ and $Hom_{n,0}(B,A)$ are singletons, so there is a unique bijection between them.
If $(a_1, \ldots a_k) \in Hom^M_{n,k}(B,ft_k(A))$, then there is a bijection between $f \in Hom_{n,k+1}(B,A)$ such that $p_A \circ f = \alpha(a_1, \ldots a_k)$ and sections of $p_{\alpha(a_1, \ldots a_k)^*(A)}$.
By induction hypothesis these sections are just sections of $p_{Subst_{n,k}(B, A, a_1, \ldots a_k)}$.
This gives us a bijection between $Hom^M_{n,k+1}(B,A)$ and $Hom_{n,k+1}(B,A)$, namely $\alpha(a_1, \ldots a_{k+1}) = q(\alpha(a_1, \ldots a_k), A) \circ a_{k+1}$.
Then equations $Subst_{n,k+1}(B, A, a_1, \ldots a_{k+1}) = \alpha(a_1, \ldots a_{k+1})^*(A)$ and $q_{n,k+1} = q(\alpha(a_1, \ldots a_k), A)$ hold by definition.

Now we define total functions $v_{n,i} : M(Ctx_n) \to M(Tm_n)$.
Let $v_{n,i}(A) = (p^{i+1}(A)^*(ft^i_{n-i}(A)), s_{p^i_A})$.
\[ \xymatrix{ p^{i+1}(A)^*(ft^i_{n-i}(A)) \ar[r] \ar[d] \pb & ft^i_{n-i}(A) \ar[d]^{p_{ft^i_{n-i}(A)}} \\
              A \ar[r]_{p^{i+1}(A)} \ar@/^1pc/[u]^{s_{p^i_A}} \ar[ur]_{p^i_A} & ft^{i+1}_{n-i-1}(A)
            } \]
Axiom~\eqref{ax:def-var} holds by definition.
By induction on $n - i$ it is easy to see that $\alpha(v_{n,n-1}(A), \ldots v_{n,i}(A))$ equals to $p_A^i : A \to ft^i_{n-i}(A)$.
Axiom~\eqref{ax:type-var} follows from the following sequence of equations:
\begin{align*}
Subst_{n,n-i-1}(A, ft^i_{n-i}(A), v_{n,n-1}(A), \ldots v_{n,i+1}(A)) & = \\
\alpha(v_{n,n-1}(A), \ldots v_{n,i+1}(A))^*(ft^i_{n-i}(A)) & = \\
p^{i+1}(A)^*(ft^i_{n-i}(A)) & = \\
ty_n(v_{n,i}(A)) & .
\end{align*}
Axiom~\eqref{ax:Subst-var} follows from the facts that $\alpha(v_{n,n-1}(ft_n(A)), \ldots v_{n,0}(ft_n(A))) = id_{ft_n(A)}$ and $id_{ft_n(A)}^*(A) = A$.

Now we define partial functions $subst_{n,k} : M(Ctx_n) \times M(Tm_k) \times M(Tm_n)^k \to M(Tm_n)$.
$subst_{n,k}(B, a, a_1, \ldots a_k)$ is defined whenever $Hom_{n,k}(B, ft_k(ty_k(a)), a_1, \ldots a_k)$ holds.
In this case we let $subst_{n,k}(B, a, a_1, \ldots a_k) = a[\alpha(a_1, \ldots a_k)]$ where $a[f] = s_{a \circ f}$.
Axioms \eqref{ax:def-subst} and \eqref{ax:type-subst} hold by definition.
Axiom~\eqref{ax:subst-var} follows from the fact that $id_{ft_n(ty_n(a))}^*(a) = a$.

To prove axiom~\eqref{ax:var-subst} note that $p_A \circ \alpha(a_1, \ldots a_{k+1}) = \alpha(a_1, \ldots a_k)$ by definition of $\alpha$.
Hence $p^i(A) \circ \alpha(a_1, \ldots a_k) = \alpha(a_1, \ldots a_{k-i})$.
Also note that $s_{\alpha(a_1, \ldots a_k)} = a_k$.
Now the axiom follows from the following equations:
\begin{align*}
subst_{n,k}(B, v_{k,i}(A), a_1, \ldots a_k) & = \\
s_{v_{k,i}(A) \circ \alpha(a_1, \ldots a_k)} & = \\
s_{q(p^{i+1}(A), ft^i_{n-i}(A)) \circ v_{k,i}(A) \circ \alpha(a_1, \ldots a_k)} & = \\
s_{p^i(A) \circ \alpha(a_1, \ldots a_k)} & = \\
s_{\alpha(a_1, \ldots a_{k-i})} & = \\
a_{k-i} & .
\end{align*}

Now we prove that $\alpha$ preserves compositions.
To do this we need to show that $\alpha(a_1, \ldots a_k) \circ f = \alpha(a_1[f], \ldots a_k[f])$.
We do this by induction on $k$.
For $k = 0$ it is trivial and for $k > 0$ we have the following sequence of equations:
\begin{align*}
\alpha(a_1, \ldots a_k) \circ f & = \\
q(\alpha(a_1, \ldots a_{k-1}), A) \circ a_k \circ f & = \\
q(\alpha(a_1, \ldots a_{k-1}), A) \circ q(f, B[\alpha(a_1, \ldots a_k)]) \circ a_k[f] & = \\
q(\alpha(a_1, \ldots a_{k-1}) \circ f, A) \circ a_k[f] & = \\
q(\alpha(a_1[f], \ldots a_{k-1}[f]), A) \circ a_k[f] & = \\
\alpha(a_1[f], \ldots a_k[f]) & .
\end{align*}

Now axioms \eqref{ax:Subst-Subst} and \eqref{ax:subst-subst} follow from the facts that $\alpha$ preserves compositions and $(f \circ g)^*(A) = f^*(g^*(A))$.
This completes the construction of $T_1$ model $M$ from a contextual category $C$.
To finish the proof we need to show that $F(M)$ is isomorphic to $C$.
The isomorphism is given by bijection $\alpha$.
We already saw that $\alpha$ preserves the structure of contextual categories.
Thus $\alpha$ is a morphism of contextual categories, and it is easy to see that $\alpha^{-1}$ also preserves the structure.
Hence $\alpha$ is isomorphism and $F$ is an equivalence.
\end{proof}

\section{Algebraic presentations of type theories}

In this section we will describe an algebraic approach to defining type theories.
We will consider a particular kind of algebraic type theories which we call \emph{regular}.
Finally, we will define categories $\algtt$ and $\algtt_{reg}$ of algebraic and regular algebraic type theories.

\begin{defn}
An \emph{algebraic type theory} is a quasi-equational theory in a signature with the set of sorts $\mathcal{C}$ together with a map from $T_1$.
Category $\algtt$ of algebraic type theory is the under category $T_1 / \cat{Th}_\mathcal{C}$.
\end{defn}

Usually in type theories all of the function symbols are available in every context.
We call such theories \emph{regular}.
Let $\mathcal{F}$ be a set of function symbols such that every $\sigma \in \mathcal{F}$ has a signature of the form $(p_1,q_1) \times \ldots \times (p_n,q_n) \to (p,0)$.
Then we define $reg(\mathcal{F})$ as the set of symbols of $T_1$ together with $\{ \sigma_k : (ctx, k) \times (p_1, q_1 + k) \times \ldots \times (p_n, q_n + k) \to (p, k) \ |\ \sigma \in \mathcal{F}, \sigma : (p_1,q_1) \times \ldots \times (p_n,q_n) \to (p,0), k \in \mathbb{N} \}$.

Now, for every $t \in Term_{reg(\mathcal{F})}(x_1 : (p_1,q_1), \ldots x_n : (p_n,q_n))_{(p,q)}$ we define a term $L(t) \in Term_{reg(\mathcal{F})}(x_1 : (p_1,q_1+1), \ldots x_n : (p_n,q_n+1))_{(p,q+1)}$ by induction on $t$:
\begin{align*}
& L(x_i) = x_i \\
& L(ft_n(A)) = ft_{n+1}(L(A)) \\
& L(ty_n(a)) = ty_{n+1}(L(a)) \\
& L(v_{n,i}(A)) = v_{n+1,i}(L(A)) \\
& L(subst_{p,n,k}(B, A, a_1, \ldots a_k)) = subst_{p,n+1,k}(L(B), A, L(a_1), \ldots L(a_k)) \\
& L(\sigma_k(A, a_1, \ldots a_n)) = \sigma_{k+1}(L(A), L(a_1), \ldots L(a_n))
\end{align*}

% For each $\sigma \in \mathcal{F}$, $\sigma : (p_1,q_1) \times \ldots \times (p_n,q_n) \to (p,0)$ we introduce a set of axioms $A(\sigma)$:

In general, we cannot construct an algebraic type theory from a syntactic one.
So we consider a full subcategory of syntactic type theories for which we can do that.
We call such theories \emph{regular}.

We say that a function symbol $f \in \Sigma$ is \emph{closed} if it has signature $s_1 \times \ldots \times s_n \to (p,0)$ for some $s_1$, \ldots $s_n$, and $p$.
We say that a theory is \emph{closed} if every its function symbol is closed.

We say that a judgement defines variable $x$ if it is either of the form $\Gamma \vdash x$ or of the form $\Gamma \vdash x : A$.
We say that a judgement defines symbol $(f : s_1 \times \ldots s_n \to (p,q)) \in \Sigma$ if it is either
of the form $\Gamma \vdash f(x_1, \ldots x_n)$ or of the form $\Gamma \vdash f(x_1, \ldots x_n) : A$ where $x_i$ are variables.
We say that an inference rule defines symbol $f$ if its conclusion defines it.
We say that such inference rule is \emph{regular} if the length of $\Gamma$ is $q$ and
if all variables in the list $x_1$, \ldots $x_n$ are distinct, this list contains all free variables of the rule,
and every $x_i$ defined by a single premise with free variables in $x_1$, \ldots $x_{i-1}$.
A syntactic type theory $T = (\Sigma, \mathcal{I})$ is \emph{regular} if the following conditions hold:
\begin{itemize}
\item $T$ is closed.
\item Every function symbol in $\Sigma$ is defined by a single regular inference rule in $\mathcal{I}$,
\item There is a well-ordering on $\Sigma$ such that the rule that defines $f$ uses in premises only symbols that are less then $f$.
\item Every inference rule in $\mathcal{I}$ is either defines some function symbol or has conclusion of the form $a \deq b$.
\end{itemize}

Now for each regular syntactic type theory $T = (\Sigma, \mathcal{I})$ we define an algebraic type theory $E(T)$.
There is a function from the set of sorts of syntactic theories to the set of sorts of algebraic theories.
Sort $(tm,n)$ is mapped to $Tm_n$ and $(ty,n)$ is mapped to $Ctx_{n+1}$.
We will usually omit this function and use sorts $(tm,n)$ and $(ty,n)$ directly.
The signature $E(\Sigma)$ of $E(T)$ consists of symbols $v_{n,i}$, $subst_{n,k}$, $Subst_{n,k}$ (with the signatures as in $T_1$), and
$f_k : Ctx_k \times (p_1,q_1+k) \times \ldots \times (p_n,q_n+k) \to (p,k)$
for each $k \in \mathbb{N}$ and $(f : (p_1,q_1) \times \ldots \times (p_n,q_n) \to (p,0)) \in \Sigma$.

Now we define functions
\[ \alpha_{U,V,p,n,k} : Term_{E(\Sigma)}(V)_{Ctx_{n+k}} \times Term_{\Sigma}(U)_{(p,k)} \times Env_{U,V,p,n} \to Term_{E(\Sigma)}(V)_{(p,n+k)} \]
and
\[ \beta_{U,V,n,k} : Term_{E(\Sigma)}(V)_{Ctx_n} \times Ctx_\Sigma(U)_k \times Env_{U,V,ty,n} \to Term_{E(\Sigma)}(V)_{Ctx_{n+k}} \]
where $Ctx_\Sigma(U)_k$ is the set of contexts of length $k$ and $Env_{U,V,p,n}$ is the set of functions $U_{(p,m)} \to Term_{E(\Sigma)}(V)_{(p,n+m)}$ for all $m \in \mathbb{N}$.
\begin{itemize}
\item $\beta_{U,V,n,0}(\Gamma, (), \rho) = \Gamma$.
\item $\beta_{U,V,n,k+1}(\Gamma, (\Delta,A), \rho) = \alpha_{U,V,ty,n,k}(\beta_{U,V,n,k}(\Gamma, \Delta, \rho), A, \rho)$.
\item $\alpha_{U,V,p,n,k}(\Gamma, v_{k,i}, \rho) = v_{n+k,i}(\Gamma)$.
\item $\alpha_{U,V,p,n,k}(\Gamma, x, \rho) = \rho(x)$ where $x$ is a variable.
\item $\alpha_{U,V,p,n,k}(\Gamma, subst_{ty,k,m}(x, a_1, \ldots a_m)) = Subst_{n+k,n+m}(\Gamma, \rho(x), a_1', \ldots a_{n+m}')$
    where $a_i' = v_{n+k,n+k-i}(\Gamma)$ if $i \leq n$ and $a_i' = \alpha_{U,V,tm,n,k}(\Gamma, a_{i-n}, \rho)$ if $i > n$.
\item $\alpha_{U,V,p,n,k}(\Gamma, subst_{tm,k,m}(x, a_1, \ldots a_m)) = subst_{n+k,n+m}(\Gamma, \rho(x), a_1', \ldots a_{n+m}')$
    where $a_i' = v_{n+k,n+k-i}(\Gamma)$ if $i \leq n$ and $a_i' = \alpha_{U,V,tm,n,k}(\Gamma, a_{i-n}, \rho)$ if $i > n$.
\item $\alpha_{U,V,p,n,k}(\Gamma, f_k(a_1, \ldots a_m)) = f_{n+k}(\Gamma, a_1', \ldots a_m'))$
    where $f : (p_1,q_1) \times \ldots \times (p_m,q_m) \to (p,0)$,
    $f(x_1, \ldots x_m)$ is defined by a rule $R$ in which $x_i$ is defined by a judgement with context $\Delta_i$, and
    \[ a_i' = \alpha_{U,V,p_i,n,k+q_i}(\beta_{\{ x_1, \ldots x_{i-1} \},V,n+k,q_i}(\Gamma, \Delta_i, [x_j \mapsto a_j']), a_i, \rho). \]
\end{itemize}
Conditions in the definition of regularity guarantee that these functions are well-defined.

Let $n \in \mathbb{N}$ and $\Delta \in V_{Ctx_n}$.
Let $i : V_{(p,m)} \to V_{(p,n+m)}$ be any injective function and $\rho : V_{(p,m)} \to Term_{E(\Sigma)}(V)_{(p,n+m)}$ be the function such that $\rho(x) = i(x)$ for each $x$.
Then for each judgement $J$ of the form $\Gamma \vdash A$ we define an equation $A(J)$:
\[ ft(\alpha_{V,V,ty,n,k}(\beta_{V,V,n,k}(\Delta, \Gamma, \rho), A, \rho)) = \beta_{V,V,n,k}(\Delta, \Gamma, \rho). \]
For each judgement $J$ of the form $\Gamma \vdash a : A$ we define an equation $A(J)$:
\[ ty(\alpha_{V,V,tm,n,k}(\beta_{V,V,n,k}(\Delta, \Gamma, \rho), a, \rho)) = \alpha_{V,V,ty,n,k}(\beta_{V,V,n,k}(\Delta, \Gamma, \rho), A, \rho). \]
For each judgement $J$ of the form $a \deq a'$ we define an equation $A(J)$:
\[ \alpha_{V,V,p,n,0}(\Delta, a, \rho) = \alpha_{V,V,p,n,0}(\Delta, a', \rho). \]

For each inference rule $R$ with premises $J_1$, \ldots $J_m$ and conclusion $J$ we define an axiom $A(R)$:
\[ A(J_1), \ldots A(J_m) \sststile{}{FV(A(J_1)) \cup \ldots \cup FV(A(J_m)) \cup FV(A(J))} A(J) \]
For each function symbol $f(x_1, \ldots x_k)$ which is defined by a rule with premises $J_1$, \ldots $J_m$ we define an axiom $D(f)$:
\[ A(J_1), \ldots A(J_m) \ssststile{}{i(x_1), \ldots i(x_k)} f(\rho(x_1), \ldots \rho(x_k)) \downarrow \]
For each function symbol $f : (p_1,q_1) \times \ldots \times (p_m,q_m) \to (p,0)$ such that $f(x_1, \ldots x_m)$ is defined by a rule with premises $J_1$, \ldots $J_m$ we define an axiom $S(f)$:
\[ \sststile{}{\Gamma, \Delta, a_1, \ldots a_m, b_1, \ldots b_k} subst_{p,n,k}(\Delta, f_k(\Gamma, a_1, \ldots a_m), b_1, \ldots b_k) \leftrightharpoons f_n(\Delta, a_1', \ldots a_m') \]
where
\[ a_i' = subst_{p_i,n+q_i,k+q_i}(\Delta_i', a_i, b_{i,1}', \ldots b_{i,k}', v_{n+q_i,q_i-1}(\Delta_i'), \ldots v_{n+q_i,0}(\Delta_i')), \]
\[ b_{i,j}' = subst_{tm,n+q_i,n}(\Delta_i', b_j, v_{n+q_i,n+q_i-1}(\Delta_i'), \ldots v_{n+q_i,q_i}(\Delta')), \]
\[ \Delta_i' = \beta_{\{ x_1, \ldots x_{i-1} \},V,n,q_i}(\Delta, \Delta_i, [x_j \mapsto a_j']) \text{, and} \]
$x_i$ is defined by a judgement with context $\Delta_i$.

Axioms of $E(\Sigma, \mathcal{I})$ are the axioms of $T_1$, $D(f)$ and $S(f)$ for each $f \in \Sigma$, and $A(R)$ for each $R \in \mathcal{I}$.

\section{Syntactic presentations of type theories}

In this section we will describe a syntactic approach to defining type theories.
We will define category $\syntt$ of syntactically presented type theories.
Finally, we will give a few examples of such theories.

Let $\mathcal{K} = \{ tm, ty \}$ and let $\mathcal{S} = \mathcal{K} \times \mathbb{N}$ be the set of sorts.
Let $l : \mathcal{S} \to \mathbb{N}$ be the function $l(k,n) = n$.
Let $\Sigma$ be a set of function symbols together with a signature, that is expressions of the form
\[ f : s_1 \times \ldots \times s_n \to s \]
where $s_1, \ldots s_n, s \in \mathcal{S}$.
Then let $\overline{\Sigma}$ be the following set of function symbols:
\[ \{ f_k\ |\ f \in \Sigma, k \in \mathbb{N} \} \cup \{ v_{n,i}\ |\ n,i \in \mathbb{N}, 0 \leq i < n \} \cup \{ subst_{p,n,k}\ |\ p \in \mathcal{K}, n,k \in \mathbb{N} \}. \]
Signatures of these symbols are defined as follows:
\begin{align*}
v_{n,i} & : (tm,n) \\
subst_{p,n,k} & : (p,k) \times (tm,n)^k \to (p,n) \\
f_k\ & : (p_1,q_1+k) \times \ldots \times (p_n,q_n+k) \to (p,q+k)
\end{align*}
where $f : (p_1,q_1) \times \ldots \times (p_n,q_n) \to (p,q)$.
We will omit subscripts $k$ in $f_k$ and $n$ in $v_{n,i}$ if it is clear from the context.
If $V$ is a set, then let $Term_\Sigma(V)$ be the set of terms in signature $\overline{\Sigma}$.

Let $T_\Sigma$ be the algebraic theory in signature $\overline{\Sigma}$ with the following set of axioms:
\begin{align*}
subst_{p,n,n}(a, v_{n,n-1}, \ldots v_{n,0}) & = a \\
subst_{tm,n,k}(v_{k,i}, a_1, \ldots a_k) & = a_{k-i} \\
subst_{p,m,n}(subst_{p,n,k}(a, b_1, \ldots b_k), c_1, \ldots c_n) & = subst_{p,m,k}(a, b_1', \ldots b_k')
\end{align*}
where $b_i' = subst_{tm,m,n}(b_i, c_1, \ldots c_n)$, and
\[ subst_{p,n+q,k+q}(f_k(a_1, \ldots a_m), b_1'', \ldots b_k'', v_{n+q,q-1}, \ldots v_{n+1,0}) = f_n(a_1', \ldots a_m') \]
for each $f : (p_1,q_1) \times \ldots \times (p_n,q_n) \to (p,q)$ and $b_j : (tm,n)$ where
\[ b_j'' = subst_{tm,n+q,n} (b_j, v_{n+q,n+q-1}, \ldots v_{n+q,q}), \]
\[ a_i' = subst_{p_i,n+q_i,k+q_i}(a_i, b_{i,1}', \ldots b_{i,k}', v_{n+q_i,q_i-1}, \ldots v_{n+q_i,0}) \text{, and} \]
\[ b_{i,j}' = subst_{tm,n+q_i,n}(b_j, v_{n+q_i,n+q_i-1}, \ldots v_{n+q_i,q_i}). \]
Let $T_\Sigma : \Set^\mathcal{S} \to \Set^\mathcal{S}$ be a monad corresponding to the theory $T_\Sigma$.
Explicitly, $T_{\Sigma}(V)_s = Term_{\Sigma}(V)_s/\sim$ where $\sim$ is the congruence generated by the axioms above.
Elements of $T_{\Sigma}(V)_s$ we will call terms of sort $s$ with free variables in $V$.
A morphism of signatures $\Sigma$ and $\Sigma'$ is a function that to each $(f : s_1 \times \ldots s_k \to s) \in \Sigma$ assigns
a term of sort $s$ in signature $\Sigma'$ with free variables $x_1$, \ldots $x_k$ of sorts $s_1$, \ldots $s_k$ respectively.
Such a function $F$ extends to functions $\overline{F} : T_\Sigma(V)_s \to T_{\Sigma'}(V)_s$ in the obvious way.
Compositions of morphisms $F : \Sigma \to \Sigma'$ and $G : \Sigma' \to \Sigma''$ is defined as follows: $(G \circ F)(f) = \overline{G}(F(f))$.
An identity morphism assigns to each $f$ the term $f(x_1, \ldots x_k)$.

If $V \in \Set^{\mathcal{S}}$, then a \emph{context} of length $n$ with free varibles in $V$ is a sequence $A_0$, \ldots $A_{n-1}$ where $A_i \in Term_\Sigma(V)_(ty,i)$.
A \emph{judgement} with free variables in $V$ is simply a predicate of a certain form.
We will consider judgements of the forms $\Gamma \vdash$, $\Gamma \vdash A$, $A \deq A'$, $\Gamma \vdash a : A$, and $a \deq a'$
where $\Gamma$ is context of length $n$ (for some $n \in \mathbb{N}$), $A, A' \in Term_\Sigma(V)_{(ty,n)}$, and $a, a' \in Term_\Sigma(V)_{(tm,n)}$.
An inference rule with free variables in $V$ consists of a finite set of judgements $J_1$, \ldots $J_n$ (with free variables in $V$) called premises and a judgement $J$ (with free variables in $V$) called conclusion.
An inference rule is usually written as
\begin{center}
\AxiomC{$J_1 \quad \ldots \quad J_n$}
\UnaryInfC{$J$}
\DisplayProof
\end{center}
If $\mathcal{I}$ is a set of inference rules, then the set $D(\mathcal{I})$ of derived rules is defined in the usual way.

\begin{defn}
A \emph{syntactic type theory} consists of a set $\Sigma$ of function symbols and a set $\mathcal{I}$ of inference rules.
\end{defn}
Let $F$ be a morphisms of signatures $\Sigma$ and $\Sigma'$.
If $J$ is a judgement of $\Sigma$, then we can define a judgement $\overline{F}(J)$ of $\Sigma'$ by applying $\overline{F}$ to every term in $J$.
If $R$ is an inference rule of $\Sigma$, then we can define an inference rule $\overline{F}(R)$ of $\Sigma'$ by applying $\overline{F}$ to every judgement in $R$.
A morphism of theories $T = (\Sigma_T, \mathcal{I}_T)$ and $S = (\Sigma_S, \mathcal{I}_S)$ is a morphism $F$ of signatures $\Sigma$ and $\Sigma'$
such that $\overline{F}(R)$ is a derived inference rules of $S$ for every $R \in \mathcal{I}_T$.
This defines a cateogry $\syntt$ of syntactic type theories.

Now we introduce a few auxiliary constructions.
If $b$ is a term of sort $(p,n+k)$ and $a_1$, \ldots $a_k$ are terms of sort $(tm,n)$, then we write $b[a_1, \ldots a_k]$ for
\[ subst_{p,n,n+k}(b, v_{n,n-1}, \ldots v_{n,0}, a_1, \ldots a_k). \]
If $b$ is a term of sort $(p,n)$, then we write $b\!\uparrow$ for
\[ subst_{p,n+1,n}(b, v_{n+1,n}, \ldots v_{n+1,1}). \]

\begin{example}
The theory of unit types with eta rules has function symbols $\top : (ty,0)$ and $unit : (tm,0)$ and the following inference rules:
\medskip
\begin{center}
\AxiomC{}
\UnaryInfC{$\vdash \top$}
\DisplayProof
\quad
\AxiomC{}
\UnaryInfC{$\vdash unit : \top$}
\DisplayProof
\quad
\AxiomC{$\vdash t : \top$}
\UnaryInfC{$t \deq unit$}
\DisplayProof
\end{center}
\end{example}

\begin{example}
The theory of unit types without eta rules has function symbols $\top : (ty,0)$, $unit : (tm,0)$ and $\top\text{-}elim : (ty,1) \times (tm,0) \times (tm,0) \to (tm,0)$
and the following inference rules:
\medskip
\begin{center}
\AxiomC{}
\UnaryInfC{$\vdash \top$}
\DisplayProof
\quad
\AxiomC{}
\UnaryInfC{$\vdash unit : \top$}
\DisplayProof
\quad
\AxiomC{$\top \vdash D$}
\AxiomC{$\vdash d : D[unit]$}
\AxiomC{$\vdash t : \top$}
\TrinaryInfC{$\vdash \top\text{-}elim(D, d, t) : D[t]$}
\DisplayProof
\end{center}

\medskip
\begin{center}
\AxiomC{$\top \vdash D$}
\AxiomC{$\vdash d : D[unit]$}
\BinaryInfC{$\top\text{-}elim(D, d, unit) \deq d$}
\DisplayProof
\end{center}
\end{example}

\begin{example}
The theory of $\Sigma$ types with eta rules has function symbols
$\Sigma : (ty,0) \times (ty,1) \to (ty,0)$, $pair : (ty,0) \times (ty,1) \times (tm,0) \times (tm,0) \to (tm,0)$,
$proj_1 : (ty,0) \times (ty,1) \times (tm,0) \to (tm,0)$ and $proj_2 : (ty,0) \times (ty,1) \times (tm,0) \to (tm,0)$
and the following inference rules:
\medskip
\begin{center}
\AxiomC{$A \vdash B$}
\UnaryInfC{$\vdash \Sigma(A, B)$}
\DisplayProof
\quad
\AxiomC{$A \vdash B$}
\AxiomC{$\vdash a : A$}
\AxiomC{$\vdash b : B[a]$}
\TrinaryInfC{$\vdash pair(A, B, a, b)$}
\DisplayProof
\end{center}

\medskip
\begin{center}
\AxiomC{$\vdash p : \Sigma(A, B)$}
\UnaryInfC{$\vdash proj_1(A, B, p) : A$}
\DisplayProof
\quad
\AxiomC{$\vdash p : \Sigma(A, B)$}
\UnaryInfC{$\vdash proj_2(A, B, p) : B[proj_1(A, B, p)]$}
\DisplayProof
\end{center}

\medskip
\begin{center}
\AxiomC{$A \vdash B$}
\AxiomC{$\vdash a : A$}
\AxiomC{$\vdash b : B[a]$}
\TrinaryInfC{$proj_1(A, B, pair(A, B, a, b)) \deq a$}
\DisplayProof
\end{center}

\medskip
\begin{center}
\AxiomC{$A \vdash B$}
\AxiomC{$\vdash a : A$}
\AxiomC{$\vdash b : B[a]$}
\TrinaryInfC{$proj_2(A, B, pair(A, B, a, b)) \deq b$}
\DisplayProof
\end{center}

\medskip
\begin{center}
\AxiomC{$\vdash p : \Sigma(A, B)$}
\UnaryInfC{$pair(A, B, proj_1(A, B, p), proj_2(A, B, p)) \deq p$}
\DisplayProof
\end{center}
\end{example}

\begin{example}
The theory of $\Sigma$ types without eta rules.
\end{example}

\begin{example}
The theory of $\Pi$ types with eta rules.
\end{example}

\begin{example}
The theory of $\Pi$ types without eta rules.
\end{example}

\begin{example}
The theory of identity types has function symbols $Id : (ty,0) \times (tm,0) \times (tm,0) \to (ty,0)$,
$refl : (ty,0) \times (tm,0) \to (tm,0)$ and $J : (ty,0) \times (ty,3) \times (tm,1) \times (tm,0) \times (tm,0) \times (tm,0) \to (tm,0)$
and the following inference rules:
\medskip
\begin{center}
\AxiomC{$\vdash a : A$}
\AxiomC{$\vdash a' : A$}
\BinaryInfC{$\vdash Id(A, a, a')$}
\DisplayProof
\quad
\AxiomC{$\vdash a : A$}
\UnaryInfC{$\vdash refl(A, a) : Id(A, a, a)$}
\DisplayProof
\end{center}

\medskip
\begin{center}
\AxiomC{$A, A\!\uparrow, Id(A\!\uparrow\uparrow, v_1, v_0) \vdash D$}
\AxiomC{$A \vdash d : D[v_0, v_0, refl(A\!\uparrow, v_0)]$}
\AxiomC{$\vdash p : Id(A, a, a')$}
\TrinaryInfC{$\vdash J(A, D, d, a, a', p) : D[a, a', p]$}
\DisplayProof
\end{center}

\medskip
\begin{center}
\AxiomC{$A, A\!\uparrow, Id(A\!\uparrow\uparrow, v_1, v_0) \vdash D$}
\AxiomC{$A \vdash d : D[v_0, v_0, refl(A\!\uparrow, v_0)]$}
\AxiomC{$\vdash a : A$}
\TrinaryInfC{$J(A, D, d, a, a, refl(A, a)) \deq d[a]$}
\DisplayProof
\end{center}
\end{example}

\begin{example}
The theory a universe has function symbols $Type : (ty,0)$ and $El : (tm,0) \to (ty,0)$ and the following inference rules:
\medskip
\begin{center}
\AxiomC{}
\UnaryInfC{$\vdash Type$}
\DisplayProof
\quad
\AxiomC{$\vdash A : Type$}
\UnaryInfC{$\vdash El(A)$}
\DisplayProof
\end{center}
\end{example}

% Now, usually universes are assumed to be closed under different type constructions.

\section{Initial models of syntactic type theories}

Results of section~\ref{sec:T1} show that models of algebraic type theories can be described as contextual categories with additional structure.
Initial models are of particular interest.
In this section we show that initial models of those theories that come from (regular) syntactic theories can be constructed from fully annotated lambda terms.
Then we show that in some cases we can drop annotations on lambda terms.

\bibliographystyle{amsplain}
\bibliography{ref}

\end{document}
