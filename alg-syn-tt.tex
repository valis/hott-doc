\documentclass{amsart}

\usepackage{amssymb}
\usepackage[all]{xy}
\usepackage{verbatim}
\usepackage{ifthen}
\usepackage{xargs}
\usepackage{bussproofs}
\usepackage{turnstile}

\renewcommand{\turnstile}[6][s]
    {\ifthenelse{\equal{#1}{d}}
        {\sbox{\first}{$\displaystyle{#4}$}
        \sbox{\second}{$\displaystyle{#5}$}}{}
    \ifthenelse{\equal{#1}{t}}
        {\sbox{\first}{$\textstyle{#4}$}
        \sbox{\second}{$\textstyle{#5}$}}{}
    \ifthenelse{\equal{#1}{s}}
        {\sbox{\first}{$\scriptstyle{#4}$}
        \sbox{\second}{$\scriptstyle{#5}$}}{}
    \ifthenelse{\equal{#1}{ss}}
        {\sbox{\first}{$\scriptscriptstyle{#4}$}
        \sbox{\second}{$\scriptscriptstyle{#5}$}}{}
    \setlength{\dashthickness}{0.111ex}
    \setlength{\ddashthickness}{0.35ex}
    \setlength{\leasturnstilewidth}{2em}
    \setlength{\extrawidth}{0.2em}
    \ifthenelse{%
      \equal{#3}{n}}{\setlength{\tinyverdistance}{0ex}}{}
    \ifthenelse{%
      \equal{#3}{s}}{\setlength{\tinyverdistance}{0.5\dashthickness}}{}
    \ifthenelse{%
      \equal{#3}{d}}{\setlength{\tinyverdistance}{0.5\ddashthickness}
        \addtolength{\tinyverdistance}{\dashthickness}}{}
    \ifthenelse{%
      \equal{#3}{t}}{\setlength{\tinyverdistance}{1.5\dashthickness}
        \addtolength{\tinyverdistance}{\ddashthickness}}{}
        \setlength{\verdistance}{0.4ex}
        \settoheight{\lengthvar}{\usebox{\first}}
        \setlength{\raisedown}{-\lengthvar}
        \addtolength{\raisedown}{-\tinyverdistance}
        \addtolength{\raisedown}{-\verdistance}
        \settodepth{\raiseup}{\usebox{\second}}
        \addtolength{\raiseup}{\tinyverdistance}
        \addtolength{\raiseup}{\verdistance}
        \setlength{\lift}{0.8ex}
        \settowidth{\firstwidth}{\usebox{\first}}
        \settowidth{\secondwidth}{\usebox{\second}}
        \ifthenelse{\lengthtest{\firstwidth = 0ex}
            \and
            \lengthtest{\secondwidth = 0ex}}
                {\setlength{\turnstilewidth}{\leasturnstilewidth}}
                {\setlength{\turnstilewidth}{2\extrawidth}
        \ifthenelse{\lengthtest{\firstwidth < \secondwidth}}
            {\addtolength{\turnstilewidth}{\secondwidth}}
            {\addtolength{\turnstilewidth}{\firstwidth}}}
        \ifthenelse{\lengthtest{\turnstilewidth < \leasturnstilewidth}}{\setlength{\turnstilewidth}{\leasturnstilewidth}}{}
    \setlength{\turnstileheight}{1.5ex}
    \sbox{\turnstilebox}
    {\raisebox{\lift}{\ensuremath{
        \makever{#2}{\dashthickness}{\turnstileheight}{\ddashthickness}
        \makehor{#3}{\dashthickness}{\turnstilewidth}{\ddashthickness}
        \hspace{-\turnstilewidth}
        \raisebox{\raisedown}
        {\makebox[\turnstilewidth]{\usebox{\first}}}
            \hspace{-\turnstilewidth}
            \raisebox{\raiseup}
            {\makebox[\turnstilewidth]{\usebox{\second}}}
        \makever{#6}{\dashthickness}{\turnstileheight}{\ddashthickness}}}}
        \mathrel{\usebox{\turnstilebox}}}

% \providecommand\WarningsAreErrors{false}
% \ifthenelse{\equal{\WarningsAreErrors}{true}}{\renewcommand{\GenericWarning}[2]{\GenericError{#1}{#2}{}{This warning has been turned into a fatal error.}}}{}

\newcommand{\newref}[4][]{
\ifthenelse{\equal{#1}{}}{\newtheorem{h#2}[hthm]{#4}}{\newtheorem{h#2}{#4}[#1]}
\expandafter\newcommand\csname r#2\endcsname[1]{\ref{#2:##1}}
\expandafter\newcommand\csname R#2\endcsname[1]{#4~\ref{#2:##1}}
\newenvironmentx{#2}[2][1=,2=]{
\ifthenelse{\equal{##2}{}}{\begin{h#2}}{\begin{h#2}[##2]}
\ifthenelse{\equal{##1}{}}{}{\label{#2:##1}}
}{\end{h#2}}
}

\newref[section]{thm}{theorem}{Theorem}
\newref{lem}{lemma}{Lemma}
\newref{prop}{proposition}{Proposition}
\newref{cor}{corollary}{Corollary}

\theoremstyle{definition}
\newref{defn}{definition}{Definition}
\newref{example}{example}{Example}

\theoremstyle{remark}
\newref{remark}{remark}{Remark}

\newcommand{\red}{\Rightarrow}
\newcommand{\deq}{\Leftrightarrow}
\renewcommand{\ll}{\llbracket}
\newcommand{\rr}{\rrbracket}
\newcommand{\cat}[1]{\mathbf{#1}}
\newcommand{\C}{\cat{C}}
\newcommand{\Set}{\cat{Set}}
\newcommand{\ccat}{\cat{CCat}}
\newcommand{\syntt}{\cat{SynTT}}
\newcommand{\algtt}{\cat{AlgTT}}
\newcommand{\ttvdash}{\vartriangleright}

\numberwithin{figure}{section}

\newcommand{\pb}[1][dr]{\save*!/#1-1.2pc/#1:(-1,1)@^{|-}\restore}
\newcommand{\po}[1][dr]{\save*!/#1+1.2pc/#1:(1,-1)@^{|-}\restore}

\begin{document}

\title{Algebraic and Syntactic Presentations of Type Theories}

\author{Valery Isaev}

% \begin{abstract}
% Abstract
% \end{abstract}

\maketitle

\section{Introduction}

\begin{comment}
\section{Contextual categories}

Contextual categories were defined by Cartmell \cite{GAT}.
An equivalent definition was given by Voevodsky in \cite{c-systems}.
In this section we will give another equivalent definition which is just a description of models of the initial algebraic type theory as we will see later.
\end{comment}

\section{Syntactic presentations of type theories}

In this section we will describe a syntactic approach to defining type theories.
We will define category $\syntt$ of syntactically presented type theories.
Finally, we will give a few examples of such theories.

% One of the main operation in type theories is the operation of substitution.

We will consider partial horn theories in signatures $\Sigma$ satisfying the following conditions:
\begin{itemize}
\item The set of sorts $\Sigma$ is the set $\{ Ty, Tm \}$.
\item The set of function symbols of $\Sigma$ contains symbols
\begin{align*}
v_n     & : Tm \\
Subst_n & : Ty \times Tm^n \to Ty \\
subst_n & : Tm^{n+1} \to Tm
\end{align*}
for each $n \in \mathbb{N}$.
\item The set of predicate symbols of $\Sigma$ contains symbols
\begin{align*}
Context_n & : Ty^n \\
Type_n & : Ty^{n+1} \\
Term_n & : Ty^n \times Tm \times Ty \\
Eq_n & : Ty^{n+2} \\
eq_n & : Ty^n \times Tm^2 \times Ty
\end{align*}
for each $n \in \mathbb{N}$.
\end{itemize}

We use uppercase latin letters for variables of sort $Ty$ and lowercase latin letters for variables of sort $Tm$.
We use uppercase greek letters for finite sequences of variables of sort $Ty$.
We use the following standard notations:
\begin{align*}
B[a_1, \ldots a_n] & \text{ means } Subst_n(B, a_1, \ldots a_n) \\
b[a_1, \ldots a_n] & \text{ means } subst_n(b, a_1, \ldots a_n) \\
A_1, \ldots A_n \ttvdash & \text{ means } Context_n(A_1, \ldots A_n) \\
A_1, \ldots A_n \ttvdash A & \text{ means } Type_n(A_1, \ldots A_n, A) \\
A_1, \ldots A_n \ttvdash a : A & \text{ means } Term_n(A_1, \ldots A_n, a, A) \\
A_1, \ldots A_n \ttvdash A \deq B & \text{ means } Eq_n(A_1, \ldots A_n, A, B) \\
A_1, \ldots A_n \ttvdash a \deq b : A & \text{ means } eq_n(A_1, \ldots A_n, a, b, A)
\end{align*}

A syntactically presented type theory is a horn theory in a signature satisfying the above conditions and such that the following axioms hold in it:
\begin{align*}
\top & \sststile[d]{}{x_1, \ldots x_n} f(x_1, \ldots x_n) \downarrow \text{ for each function symbol $f$} \\
\top & \sststile[d]{}{} \ \ttvdash \\
\Gamma \ttvdash A & \ssststile{}{\Gamma, A} \Gamma, A \ttvdash \\
\Gamma \ttvdash A & \sststile{}{\Gamma, A} \Gamma \\
\Gamma \ttvdash a : A & \sststile{}{\Gamma, a, A} \Gamma \ttvdash A \\
\Gamma \ttvdash A & \sststile{}{\Gamma, A} \Gamma, A \ttvdash v_0 : A[v_n, \ldots v_1] \text{ if } |\Gamma| = n \\
\Gamma \ttvdash,\ A_1, \ldots A_n \ttvdash B,\ \Gamma \ttvdash a_i : A_i[a_1, \ldots a_{i-1}] & \sststile{}{\Gamma, A_1, \ldots A_n, a_1, \ldots a_n, B} \Gamma \ttvdash B[a_1, \ldots a_n] \\
\Gamma \ttvdash,\ A_1, \ldots A_n \ttvdash b : B,\ \Gamma \ttvdash a_i : A_i[a_1, \ldots a_{i-1}] & \sststile{}{\Gamma, A_1, \ldots A_n, a_1, \ldots a_n, B, b} \Gamma \ttvdash b[a_1, \ldots a_n] : B[a_1, \ldots a_n]
\end{align*}

\section{Algebraic presentations of type theories}

In this section we will describe an algebraic approach to defining type theories.
We will define category $\algtt$ of algebraic type theories.
Finally, we will give a few examples of such theories.

We will consider quasi-equational theories in signatures $\Sigma$ which have the set of sorts $\{ Ctx_n\ |\ n \in \mathbb{N} \} \cup \{ Tm_n\ |\ n \in \mathbb{N} \}$.
Let $T_0$ be the theory with the set of function symbols $\{ * \} \cup \{ ft_n : Ctx_{n+1} \to Ctx_n\ |\ n \in \mathbb{N} \} \cup \{ ty_n : Tm_n \to Ctx_{n+1}\ |\ n \in \mathbb{N} \}$ and the following axioms:
\begin{align*}
& \top \sststile{}{A} ft(A) \downarrow \\
& \top \sststile{}{a} ty(a) \downarrow \\
& \top \sststile{}{} * \downarrow \\
& \top \sststile{}{A} A = *
\end{align*}



\bibliographystyle{amsplain}
\bibliography{ref}

\end{document}
