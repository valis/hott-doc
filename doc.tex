\documentclass{amsart}

\usepackage{amssymb}
\usepackage[all]{xy}
\usepackage{verbatim}
\usepackage{ifthen}
\usepackage{xargs}
\usepackage{bussproofs}
\usepackage{listings}

\providecommand\WarningsAreErrors{false}
\ifthenelse{\equal{\WarningsAreErrors}{true}}{\renewcommand{\GenericWarning}[2]{\GenericError{#1}{#2}{}{This warning has been turned into a fatal error.}}}{}

\newcommand{\newref}[4][]{
\ifthenelse{\equal{#1}{}}{\newtheorem{h#2}[hthm]{#4}}{\newtheorem{h#2}{#4}[#1]}
\expandafter\newcommand\csname r#2\endcsname[1]{#3~\ref{#2:##1}}
\expandafter\newcommand\csname R#2\endcsname[1]{#4~\ref{#2:##1}}
\newenvironmentx{#2}[2][1=,2=]{
\ifthenelse{\equal{##2}{}}{\begin{h#2}}{\begin{h#2}[##2]}
\ifthenelse{\equal{##1}{}}{}{\label{#2:##1}}
}{\end{h#2}}
}

\newref[section]{thm}{theorem}{Theorem}
\newref{lem}{lemma}{Lemma}
\newref{prop}{proposition}{Proposition}
\newref{cor}{corollary}{Corollary}

\theoremstyle{definition}
\newref{defn}{definition}{Definition}
\newref{example}{example}{Example}

\theoremstyle{remark}
\newref{remark}{remark}{Remark}

\newcommand{\cat}[1]{\mathbf{#1}}
\newcommand{\C}{\cat{C}}
\newcommand{\sSet}{\cat{sSet}}
\newcommand{\bs}{\beta\sigma}
\newcommand{\red}{\Rightarrow}
\newcommand{\deq}{\Leftrightarrow}

\numberwithin{figure}{section}

\lstset{escapeinside={\#}{\#}}

\begin{document}

\title{Homotopy Type Theory with an interval type}

\author{Valery Isaev}

% \begin{abstract}
% Abstract
% \end{abstract}

\maketitle

\section{Introduction}

We define a version of homotopy type theory based on a notion of the interval type.
We give a simple formulation of the univalence axiom and describe a few extenisons of the system.

The first extension is called data types with conditions.
Data types with conditions generalize ordinary data types, allowing us to define not only coproducts of types, but also pushouts of certain functions (which we can call cofibrations).
Using data types with condtions, we will show how to define higher inductive types which have a simple description of elimination principles.

The second extension is called records with conditions and it is dual to the first one.
Using this extensions, we can define path types for which the usual elimination principle J holds.

\section{Syntax}

\centerAlignProof

\begin{table}

\medskip
\begin{center}
\AxiomC{}
\UnaryInfC{$\varnothing \vdash$}
\DisplayProof
\quad
\AxiomC{$\Gamma \vdash A$}
\RightLabel{, $x \notin \Gamma$}
\UnaryInfC{$\Gamma, x : A \vdash$}
\DisplayProof
\quad
\AxiomC{$\Gamma \vdash$}
\RightLabel{, $x : A \in \Gamma$}
\UnaryInfC{$\Gamma \vdash x : A$}
\DisplayProof
\end{center}

\medskip
\begin{center}
\AxiomC{$\Gamma \vdash a : A$}
\AxiomC{$\Gamma \vdash B$}
\RightLabel{, $A \deq B$}
\BinaryInfC{$\Gamma \vdash a : B$}
\DisplayProof
\end{center}

\medskip
\begin{center}
\AxiomC{$\Gamma \vdash A$}
\AxiomC{$\Gamma, x : A \vdash B$}
\BinaryInfC{$\Gamma \vdash (x : A) \to B$}
\DisplayProof
\end{center}

\medskip
\begin{center}
\AxiomC{$\Gamma, x : A \vdash b : B$}
\UnaryInfC{$\Gamma \vdash \lambda x \red b : (x : A) \to B$}
\DisplayProof
\quad
\AxiomC{$\Gamma \vdash f : (x : A) \to B$}
\AxiomC{$\Gamma \vdash a : A$}
\BinaryInfC{$\Gamma \vdash f\ a : B[x := a]$}
\DisplayProof
\end{center}

\medskip
\begin{center}
\AxiomC{$\Gamma \vdash$}
\UnaryInfC{$\Gamma \vdash I$}
\DisplayProof
\quad
\AxiomC{$\Gamma \vdash$}
\UnaryInfC{$\Gamma \vdash left : I$}
\DisplayProof
\quad
\AxiomC{$\Gamma \vdash$}
\UnaryInfC{$\Gamma \vdash right : I$}
\DisplayProof
\end{center}

\medskip
\begin{center}
\AxiomC{$\Gamma, x : I \vdash A$}
\AxiomC{$\Gamma \vdash i : I$}
\AxiomC{$\Gamma \vdash a : A[x := i]$}
\AxiomC{$\Gamma \vdash j : I$}
\QuaternaryInfC{$\Gamma \vdash coe_{\lambda x \red A}\ i\ a\ j : A[x := j]$}
\DisplayProof
\end{center}

\medskip
\begin{center}
\AxiomC{$\Gamma, x : I \vdash A$}
\AxiomC{$\Gamma \vdash a : A[x := left]$}
\AxiomC{$\Gamma \vdash a' : A[x := right]$}
\TrinaryInfC{$\Gamma \vdash Path\ (\lambda x \red A)\ a\ a'$}
\DisplayProof
\end{center}

\medskip
\begin{center}
\AxiomC{$\Gamma, x : I \vdash a : A$}
\UnaryInfC{$\Gamma \vdash path\ (\lambda x \red a) : Path\ (\lambda x \red A)\ a[x := left]\ a[x := right]$}
\DisplayProof
\end{center}

\medskip
\begin{center}
\AxiomC{$\Gamma \vdash p : Path\ (\lambda x \red A)\ a\ a'$}
\AxiomC{$\Gamma \vdash i : I$}
\BinaryInfC{$\Gamma \vdash p\ @_{a,a'}\ i : A[x := i]$}
\DisplayProof
\end{center}

\begin{align*}
& (\lambda x \red b)\ a \red_\beta b[x := a] \\
& coe_{\lambda k \red A}\ i\ a\ j \red_\beta a \text{ if } i \deq j \\
& coe_{\lambda k \red A}\ i\ a\ j \red_\sigma a \text{ if } k \notin FV(A) \\
& path\ (\lambda x \red t)\ @_{a,a'}\ i \red_\beta t[x := i] \\
& p\ @_{a,a'}\ left \red_\beta a \\
& p\ @_{a,a'}\ right \red_\beta a' \\
\end{align*}

\caption{Inference and reduction rules.}
\label{table:inf-rules}
\end{table}

In this section, we will describe the inference rules of the basic system.
It has $\Pi$-types, path types, and also the interval type $I$, which has two constructors ($left$ and $right$) and one elimination rule ($coe$).
The behaviour of $coe$ can be described as follows:
given a fibration $\lambda x \red A$ over $I$ and a point $a$ in the fibre over some point $i : I$, $coe_{\lambda x \red A}\ i\ a$ gives us a section of this fibration.

Path types are heterogeneous, that is we can construct paths between different types (which are itself connected by a path).
We write $a =_A a'$ (or simply $a = a'$) for usual homogeneous paths, which are defined as $Path (\lambda i \red A)\ a\ a'$.
Also path types have unusual constructors ($path$) and eliminators ($@$).
But we still can define usual functions for path types using these constructions:
\begin{lstlisting}
idp : (x : A) #$\to$# x = x
idp = #$\lambda$#x #$\red$# path (#$\lambda$#i #$\red$# x)

pmap : (f : A #$\to$# B) #$\to$# (a a' : A) #$\to$# a = a' #$\to$# f a = f a'
pmap = #$\lambda$#f a a' p #$\red$# path (#$\lambda$#i #$\red$# f (p @ i))

funExt : (f g : (a : A) #$\to$# B[x := a]) #$\to$#
    ((a : A) #$\to$# f a = g a) #$\to$# f = g
funExt = #$\lambda$#f g p #$\red$# path (#$\lambda$#i a #$\red$# p a @ i)

transport : (a a' : A) #$\to$# a = a' #$\to$# B[x := a] #$\to$# B[x := a']
transport = #$\lambda$#a a' p x #$\red$#
    coe (#$\lambda$#i #$\red$# B[x := p @ i]) left x right
\end{lstlisting}

The definition of $J$ is a little bit more complicated.
First, we need to define a function $squeeze$, satisfying the following rules:
\begin{lstlisting}
squeeze : I #$\to$# I #$\to$# I
squueze left j #$\red$# left
squueze right j #$\red$# j
squueze i left #$\red$# left
squueze i right #$\red$# i
\end{lstlisting}

We can define the following function:
\begin{lstlisting}
squeeze' : I #$\to$# I #$\to$# I
squeeze' = #$\lambda$#i j #$\red$# coe (#$\lambda$#k #$\red$# left = k) left
    (path (#$\lambda$#k #$\red$# left)) j @ i
\end{lstlisting}
It satisfies almost all of the rules except for the last one.
But we can construct $squeeze$ which satisfies all of the rules using $squeeze'$:
\begin{lstlisting}
squeeze : I #$\to$# I #$\to$# I
squeeze = #$\lambda$#i j #$\red$# coe (#$\lambda$#i #$\red$# Path (#$\lambda$#j #$\red$# left = squeeze' i j)
    (path (#$\lambda$#k #$\red$# left)) (path (#$\lambda$#j #$\red$# squeeze' i j))) left
    (path (#$\lambda$#k #$\red$# path (#$\lambda$#k #$\red$# left))) right @ i @ j
\end{lstlisting}

Now, we can define $J$:
\begin{lstlisting}
J : ((a : A) #$\to$# B[x := a, y := a, p := idp a]) #$\to$#
    (a a' : A) (p : a = a') #$\to$# B[x := a, y := a']
J = #$\lambda$#d a a' p #$\red$# coe (#$\lambda$#i #$\red$# B[x := a, y := p @ i,
    p := path (#$\lambda$#j #$\red$# p @ squeeze i j)]) left (d a) right
\end{lstlisting}

\subsection{Univalent universes}
Now, we show how to add a univalent universe to the system.

\medskip
\begin{center}
\AxiomC{$\Gamma \vdash$}
\UnaryInfC{$\Gamma \vdash Type$}
\DisplayProof
\quad
\AxiomC{$\Gamma \vdash$}
\UnaryInfC{$\Gamma \vdash I : Type$}
\DisplayProof
\quad
\AxiomC{$\Gamma \vdash A : Type$}
\AxiomC{$\Gamma, x : A \vdash B : Type$}
\BinaryInfC{$\Gamma \vdash (x : A) \to B : Type$}
\DisplayProof
\end{center}

\medskip
\begin{center}
\Axiom$\fCenter \Gamma \vdash A : Type$
\noLine
\UnaryInf$\fCenter \Gamma \vdash B : Type$
\def\extraVskip{1pt}
\Axiom$\fCenter \Gamma \vdash f : A \to B$
\noLine
\UnaryInf$\fCenter \Gamma \vdash g : B \to A$
\Axiom$\fCenter \Gamma \vdash p : (a : A) \to Path\ A\ (g\ (f\ a))\ a$
\noLine
\UnaryInf$\fCenter \Gamma \vdash q : (b : B) \to Path\ B\ (f\ (g\ b))\ b$
\def\extraVskip{2pt}
\AxiomC{$\Gamma \vdash i : I$}
\QuaternaryInfC{$\Gamma \vdash iso\ A\ B\ f\ g\ p\ q\ i : Type$}
\DisplayProof
\end{center}
\medskip

We also add the following reduction rules:
\begin{itemize}
\item $iso\ A\ B\ f\ g\ p\ q\ left \red_\beta A$
\item $iso\ A\ B\ f\ g\ p\ q\ right \red_\beta B$
\item $coe_{\lambda k \red iso\,A\,B\,f\,g\,p\,q\,k}\ left\ a\ right \red_\beta f\ a$ if $k \notin FV(A\ B\ f\ g\ p\ q)$
\item $coe_{\lambda k \red iso\,A\,B\,f\,g\,p\,q\,k}\ right\ b\ left \red_\beta g\ b$ if $k \notin FV(A\ B\ f\ g\ p\ q)$
\end{itemize}

Usually, univalence is stated in the form of an axiom as follows:
\begin{lstlisting}
isContr A = #$\Sigma$#(a : A) ((a' : A) #$\to$# a = a')

isEquiv f = (b : B) #$\to$# isContr (#$\Sigma$#(a : A) (f a = b))

pte : (A B : Type) #$\to$# A = B #$\to$# #$\Sigma$#(f : A #$\to$# B) (isEquiv f)
pte = #$\lambda$#A B p #$\red$# (transport A B p, ...)

univalence = (A B : Type) #$\to$# isEquiv (pte A B)
\end{lstlisting}
Instead of dots there should be a proof that $transport\ A\ B\ p$ is an equivalence, which we omit because of its length.

We can show that $univalence$ holds using $iso$.
First, we construct the following function
\begin{lstlisting}
etp : (A B : Type) #$\to$# #$\Sigma$#(f : A #$\to$# B) (isEquiv f) #$\to$# A = B
etp = #$\lambda$#A B p #$\red$# path (#$\lambda$#i #$\red$# iso A B f _ _ _ i)
\end{lstlisting}
Omitted terms can be constructed using a proof of $isEquiv\ f$.

To prove that $pte\ A\ B$ is an equivalence it is enough to show that $etp\ A\ B$ is its inverse.
The first components of pairs $pte\ A\ B\ (etp\ A\ B (f, e))$ and $(f, e)$ are equal definitionally, and its second components are equal since $isEquiv\ f$ is a proposition.
Finally, we need to show that for each $p : A = B$ we have a path $etp\ A\ B\ (pte\ A\ B\ p) = p$.
We can do this using $J$

\subsection{Data types with conditions}

An ordinary data type $D$ with parameters $a_1 : A_1, \ldots a_n : A_n$ is described by a list of its constructors:
\[ c_1 : (x_1 : X_1) \ldots (x_{k_1} : X_{k_1}) \to D\ a_1 \ldots a_n \]
\[ \vdots \]
\[ c_m : (z_1 : Z_1) \ldots (z_{k_m} : Z_{k_m}) \to D\ a_1 \ldots a_n \]

A data type with conditions also allows us to put a condition on a constructor.
A condition on a constructor is simply a partial definition of a function with the type of this constructor.
Syntactically, this means that constructors now can be evaluated if its arguments match one of the clauses of this partially defined function.
Also, this means that when we define a function over such a data type we need to check that this definition respects conditions.

Semantically, ordinary data types allow to define coproducts of types, and data types with conditions allow us to define pushouts.
One of the maps in the pushout diagram must be given by a collection of constructors of a data type, which means it is (often can be interpreted as) a cofibration.
This implies that this pushout is allways a correct homotopy pushout.

Let us give an example. We can define integers as the data type with two constructors:
\[ negative : \mathbb{N} \to \mathbb{Z} \]
\[ positive : \mathbb{N} \to \mathbb{Z}. \]

The problem is that we get two different zeros: $negative\ 0$ and $positive\ 0$.
Of course, we can define $-1$ to be $negative\ 0$, $-2$ to be $negative\ 1$, and so one, but this can easily lead to a confusion.
It is better to identify the positive and the negative zeros, that is to define $\mathbb{Z}$ as the pushout of $1 \overset{0}\to \mathbb{N}$ with itself.
So, we add the following condition to the definition of $\mathbb{Z}$:
\[ negative\ 0 = positive\ 0. \]
Now, when we define a function $f$ over $\mathbb{Z}$, we need to check that $f\ (negative\ 0) \deq f\ (positive\ 0)$.

Using data types with conditions, we can define higher inductive types.
For example, the circle $S^1$ can be defined as follows: it has two constructors
\[ base : S^1 \]
\[ loop : I \to S^1 \]
and two conditions
\[ loop\ left = base \]
\[ loop\ right = base. \]

After we add path types, it will be posible to describe elimination rules for such data types,
but it is more convenient to simply define functions over them.

\subsection{Records with conditions}

Records with conditions are dual to data types with conditions.
Ordinary records allow us to define product of types, and records with conditions allow us to define pullbacks.
A record is given by a list of fields, and a record with conditions can additionally put conditions on some fields.
A condition on a field is a partially defined function with the type of this field.
Semantically, such a record is given by a pullback of a function $(B \to X) \to (A \to X)$ for some cofibration $A \to B$, which means that it is still fibrant.

Let us show how to define path types as a record with conditions.
It will be a record $Path$ with three parameters $A : I \to Type$, $a : A\ left$, and $a' : A\ right$.
We will define a ``heterogeneous path type'' because it seems impossible to define useful elimination rules for higher inductive types using only homogeneous path types.
We abbreviate $Path\ (\lambda i \red A)\ a\ a'$ to $a = a'$.
The record $Path\ (A : I \to Type)\ (a : A\ left)\ (a' : A\ right)$ has one constructor
\[ path : ((i : I) \to A\ i) \to Path\ (A : I \to Type)\ (a : A\ left)\ (a' : A\ right), \]
one field
\[ at : (i : I) \to A\ i, \]
and two conditions
\[ at\ left = a \]
\[ at\ right = a'. \]

When we define an element of a record with conditions, we need to check that it satisfies them.
And when we access fields with conitions, they might evaluate, if arguments to it match some condition on it.
For example, if $p : a = a'$, then $p.at\ left$ evaluates to $a$, $p.at\ right$ evaluates to $a'$,
and if we want to consruct a path of type $a = a'$, then we need to specify a function $f : I \to A$ such that $f\ left \deq a$ and $f\ right \deq a'$.

We can define the $J$ rule as follows:
\begin{itemize}
\item[] $idp : (A : Type) (a : A) \to Path\ (\lambda i \red A)\ a\ a$
\item[] $idp\ A\ a = path (\lambda i \red a)$
\item[]
\item[] $J : (A : Type) (C : (x\ y : A) \to x = y \to Type) \to$
\item[] \qquad $((a : A) \to C\ a\ a\ (idp\ A\ a)) \to (a\ a' : A) (p : a = a') \to C\ a\ a'\ p$
\item[] $J\ A\ C\ d\ a\ a'\ p =\ coe_{\lambda i \red \,C\,a\,(p.at\,i)\,(path\,(\lambda j \red \,p.at\,(squeeze\,i\,j)))}\ left\ (d\ a)\ right$
\end{itemize}

Now, we can prove the univalence axiom in its usual form, which states that a certain function $F$ is equivalence.
Function $F$ maps path types $A = B$ to the type of functions $A \to B$ which are equivalences.
We can construct an inverse $G$ to function $F$: given an equivalence $A \to B$, $G$ construct a path $A = B$ using $iso$.
If $f : A \to B$ is an equivalence, then $F\ (G\ f)$ definitionally equals to $f$ because of the compuation rule for $iso$.
If $p : A = B$, then we can show that there is a path between $G\ (F\ p)$ and $p$ using $J$.
Actually, we can't use $J$ directly because of the size issues, but we can expand its definition, and then it works fine.

\section{A model of data types and records with conditions}

In this section, we will show how to interpret data types and records with conditions in simplicial sets.

\subsection{Data types with conditions}

Each constructor of a data type gives us an accessible functor $C_i : \sSet \to \sSet$.

% If a data type without conditions has $n$ constructors, then we can interpret it as an algebra for endofunctor $\coprod_{1 \leq i \leq n} C_i$.

\subsection{Records with conditions}

% \bibliographystyle{amsplain}
% \bibliography{ref}

\end{document}
