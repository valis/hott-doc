\documentclass[reqno]{amsart}

\usepackage[russian]{babel}
\usepackage[utf8]{inputenc}
\usepackage{mathtools}
\usepackage{cmap}
\usepackage{amsfonts}
\usepackage{upgreek}
\usepackage{xargs}
\usepackage{ifthen}
\usepackage[all]{xy}
\usepackage{hyperref}
\usepackage{etex}
\usepackage{bussproofs}
\usepackage{turnstile}

\hypersetup{colorlinks=true,linkcolor=blue}

\renewcommand{\turnstile}[6][s]
    {\ifthenelse{\equal{#1}{d}}
        {\sbox{\first}{$\displaystyle{#4}$}
        \sbox{\second}{$\displaystyle{#5}$}}{}
    \ifthenelse{\equal{#1}{t}}
        {\sbox{\first}{$\textstyle{#4}$}
        \sbox{\second}{$\textstyle{#5}$}}{}
    \ifthenelse{\equal{#1}{s}}
        {\sbox{\first}{$\scriptstyle{#4}$}
        \sbox{\second}{$\scriptstyle{#5}$}}{}
    \ifthenelse{\equal{#1}{ss}}
        {\sbox{\first}{$\scriptscriptstyle{#4}$}
        \sbox{\second}{$\scriptscriptstyle{#5}$}}{}
    \setlength{\dashthickness}{0.111ex}
    \setlength{\ddashthickness}{0.35ex}
    \setlength{\leasturnstilewidth}{2em}
    \setlength{\extrawidth}{0.2em}
    \ifthenelse{%
      \equal{#3}{n}}{\setlength{\tinyverdistance}{0ex}}{}
    \ifthenelse{%
      \equal{#3}{s}}{\setlength{\tinyverdistance}{0.5\dashthickness}}{}
    \ifthenelse{%
      \equal{#3}{d}}{\setlength{\tinyverdistance}{0.5\ddashthickness}
        \addtolength{\tinyverdistance}{\dashthickness}}{}
    \ifthenelse{%
      \equal{#3}{t}}{\setlength{\tinyverdistance}{1.5\dashthickness}
        \addtolength{\tinyverdistance}{\ddashthickness}}{}
        \setlength{\verdistance}{0.4ex}
        \settoheight{\lengthvar}{\usebox{\first}}
        \setlength{\raisedown}{-\lengthvar}
        \addtolength{\raisedown}{-\tinyverdistance}
        \addtolength{\raisedown}{-\verdistance}
        \settodepth{\raiseup}{\usebox{\second}}
        \addtolength{\raiseup}{\tinyverdistance}
        \addtolength{\raiseup}{\verdistance}
        \setlength{\lift}{0.8ex}
        \settowidth{\firstwidth}{\usebox{\first}}
        \settowidth{\secondwidth}{\usebox{\second}}
        \ifthenelse{\lengthtest{\firstwidth = 0ex}
            \and
            \lengthtest{\secondwidth = 0ex}}
                {\setlength{\turnstilewidth}{\leasturnstilewidth}}
                {\setlength{\turnstilewidth}{2\extrawidth}
        \ifthenelse{\lengthtest{\firstwidth < \secondwidth}}
            {\addtolength{\turnstilewidth}{\secondwidth}}
            {\addtolength{\turnstilewidth}{\firstwidth}}}
        \ifthenelse{\lengthtest{\turnstilewidth < \leasturnstilewidth}}{\setlength{\turnstilewidth}{\leasturnstilewidth}}{}
    \setlength{\turnstileheight}{1.5ex}
    \sbox{\turnstilebox}
    {\raisebox{\lift}{\ensuremath{
        \makever{#2}{\dashthickness}{\turnstileheight}{\ddashthickness}
        \makehor{#3}{\dashthickness}{\turnstilewidth}{\ddashthickness}
        \hspace{-\turnstilewidth}
        \raisebox{\raisedown}
        {\makebox[\turnstilewidth]{\usebox{\first}}}
            \hspace{-\turnstilewidth}
            \raisebox{\raiseup}
            {\makebox[\turnstilewidth]{\usebox{\second}}}
        \makever{#6}{\dashthickness}{\turnstileheight}{\ddashthickness}}}}
        \mathrel{\usebox{\turnstilebox}}}

\newcommand{\axlabel}[1]{(#1) \phantomsection \label{ax:#1}}
\newcommand{\axtag}[1]{\label{ax:#1} \tag{#1}}
\newcommand{\axref}[1]{(\hyperref[ax:#1]{#1})}

\newcommand{\newref}[6][]{
\ifthenelse{\equal{#1}{}}{\newtheorem{h#2}[hthm]{#4}}{\newtheorem{h#2}{#4}[#1]}
\expandafter\newcommand\csname r#2\endcsname[1]{#3~\ref{#2:##1}}
\expandafter\newcommand\csname R#2\endcsname[1]{#4~\ref{#2:##1}}
\expandafter\newcommand\csname d#2\endcsname[1]{#5~\ref{#2:##1}}
\expandafter\newcommand\csname p#2\endcsname[1]{#6~\ref{#2:##1}}
\expandafter\newcommand\csname n#2\endcsname[1]{\ref{#2:##1}}
\newenvironmentx{#2}[2][1=,2=]{
\ifthenelse{\equal{##2}{}}{\begin{h#2}}{\begin{h#2}[##2]}
\ifthenelse{\equal{##1}{}}{}{\label{#2:##1}}
}{\end{h#2}}
}

\newref[section]{thm}{теорема}{Теорема}{теореме}{теореме}
\newref{lem}{лемма}{Лемма}{лемме}{лемме}
\newref{prop}{утверждение}{Утверждение}{утверждению}{утверждении}
\newref{cor}{следствие}{Следствие}{следствию}{следствии}

\theoremstyle{definition}
\newref{defn}{определение}{Определение}{определению}{определении}
\newref{example}{пример}{Пример}{примеру}{примере}

\theoremstyle{remark}
\newref{remark}{замечание}{Замечание}{замечанию}{замечании}

\newcommand{\bcat}[1]{\mathbf{#1}}
\newcommand{\cat}[1]{\mathcal{#1}}
\newcommand{\Set}{\bcat{Set}}
\newcommand{\fs}[1]{\mathrm{#1}}

\newenvironment{tolerant}[1]{\par\tolerance=#1\relax}{\par}

\newcommand{\pb}[1][dr]{\save*!/#1-1.2pc/#1:(-1,1)@^{|-}\restore}

\begin{document}

\title{Title}

\author{Valery Isaev}

\maketitle

\section{Введение}

\section{Частичные хорновские теории}

В данной диссертации мы будем работать с определенным классом логических теорий, которые известны под разными названиями, но наиболее распространенный термин -- это \emph{существенно алгебраические теории}.
Существует несколько эквивалентных способов определить такие теории:
\begin{enumerate}
\item Декартовы теории \cite[Definition~D1.3.4]{elephant} -- это специальный вид теорий в логике первого порядка, где единственные логческие связки -- это конъюнкция и квантор существования.
Кроме того, множество аксиом должно удовлетворять определенному условию.
Мы не будем использовать это понятие, поэтому точное определение нам не понадобится.
\item Обобщенные алгебраические теории \cite{GAT} могут содержать сорта, зависящие от других сортов.
\item Существенно алгебраические теории \cite[Definition~3.34]{LPC} -- это теории, в которых некоторые функциональные символы могут интерпретироваться как \emph{частичные} функции.
Мы не будем использовать это понятие, поэтому точное определение нам не понадобится.
\item Частичные хорновские теории \cite{PHL} являются обобщением существенно алгебраических теорий.
\item Категорное определение теорий является самым простым.
Согласно нему существенно алгебраическая теория -- это просто конечно полная малая категория.
Модель такой теории $\cat{C}$ -- это просто функтор $\cat{C} \to \Set$, сохраняющий конечные пределы.
Другое преимущество этого определения заключается в том, что легко определить структуру категории (и даже 2-категории) на классе теорий.
\end{enumerate}

Мы будем работать с частичными хорновскими теориями, определение которых мы приведем в следующем подразделе.
Кроме того, в этом разделе мы зададим структуру 2-категории на них и докажем, что она эквивалентна 2-категории конечно полных малых категорий.

\subsection{Определение}

В этом подразделе мы преведем основные определения из \cite{PHL}.
\emph{Сигнатура} -- это тройка $(\mathcal{S},\mathcal{F},\mathcal{P})$, где $\mathcal{S}$ -- множество сортов, $\mathcal{F}$ -- множество функциональных символов, и $\mathcal{P}$ -- множество предикатных символов.
Каждому функциональному символу $\sigma$ сопоставляется непустой список сортов $s_1$, \ldots $s_n$, $s$, что записывается как $\sigma : s_1 \times \ldots \times s_n \to s$.
Каждому предикатному символу $R$ сопоставляется список сортов $s_1$, \ldots $s_n$, что записывается как $R : s_1 \times \ldots \times s_n$.

Мы как обычно для каждого сорта фиксируем счетное множество переменных этого сорта.
Множество \emph{термов} определяется как обычно индуктивно:
\begin{enumerate}
\item Любая переменная сорта $s$ является термом этого сорта.
\item Если $\sigma : s_1 \times \ldots \times s_n \to s$ -- функциональный символ, и $t_i$ -- терм сорта $s_i$ для всех $1 \leq i \leq n$, то $\sigma(t_1, \ldots t_n)$ -- терм сорта $s$.
\end{enumerate}

\emph{Атомарная формула} -- это выражение вида $t_1 = t_2$, где $t_1$ и $t_2$ -- термы одного и того же сорта, либо вида $R(t_1, \ldots t_n)$, где $R : s_1 \times \ldots \times s_n$ -- предикатный символ, а $t_i$ -- терм сорта $s_i$ для любого $1 \leq i \leq n$.
\emph{(Хорновская) формула} -- это выражение вида $\varphi_1 \land \ldots \land \varphi_n$, где $\varphi_i$ -- атомарные формулы.
Конъюнкция пустого множества формул будет обозначаться как $\top$.
Выражение $t\!\downarrow$ является сокращением для $t = t$.
Функциональные символы интерпретируются как частичные функции.
Формула $t\!\downarrow$ означает, что все подвыражения в $t$ определены.
\emph{Секвенция} -- это выражение вида $\varphi \sststile{}{x_1, \ldots x_n} \psi$, где $\varphi$ и $\psi$ -- формулы, такие что все переменные, встречающиеся в них, принадлежат множеству $\{ x_1, \ldots x_n \}$.
Множество переменных, встречающихся в формуле или терме $\varphi$ будет обозначаться как $\mathrm{FV}(\varphi)$.
Вместо $\varphi_1 \land \ldots \land \varphi_n \sststile{}{x_1, \ldots x_n} \psi$ мы будем часто писать $\varphi_1, \ldots \varphi_n \sststile{}{x_1, \ldots x_n} \psi$.

Мы будем использовать следующие сокращения:
\begin{align*}
\varphi \sststile{}{V} t \cong s & \Longleftrightarrow \varphi \land t\!\downarrow\,\sststile{}{V} t = s \text{ и } \varphi \land s\!\downarrow\,\sststile{}{V} t = s \\
\varphi \ssststile{}{V} \psi & \Longleftrightarrow \varphi \sststile{}{V} \psi \text{ и } \psi \sststile{}{V} \varphi
\end{align*}

\begin{defn}
\emph{Частичная хорновская теория} -- это четверка $(\mathcal{S},\mathcal{F},\mathcal{P},\mathcal{A})$, где $(\mathcal{S},\mathcal{F},\mathcal{P})$ -- это сигнатура, а $\mathcal{A}$ -- множество секвенций.
Элементы $\mathcal{A}$ мы будем называть аксиомами.
\end{defn}

\begin{remark}
Так как частичные хорновские теории -- это единственный вид теорий, с которым мы будем работать, то мы их будем называть просто теориями.
\end{remark}

Если $\mathcal{S}$ -- множество сортов, то $\mathcal{S}$-множество $M$ -- это коллекция множеств $\{ M_s \}_{s \in \mathcal{S}}$.
\emph{Интерпретация} сигнатуры $(\mathcal{S},\mathcal{F},\mathcal{P})$ -- это $\mathcal{S}$-множество $M$
вместе с коллекцией частичных функций $M(\sigma) : M_{s_1} \times \ldots \times M_{s_n} \to M_s$ для каждого функционального символа $\sigma : s_1 \times \ldots \times s_n \to s$
и коллекцией отношений $M(R) \subseteq M_{s_1} \times \ldots \times M_{s_n}$ для каждого предикатного символа $R : s_1 \times \ldots \times s_n$.

Пусть $V = \{ x_1 : s_1, \ldots x_n : s_n \}$ -- некоторое конечное множество переменных.
Если $t$ -- терм сорта $s$ в сигнатуре $\Sigma$ такой, что $\mathrm{FV}(t) \subseteq V$, и $M$ -- интерпретация $\Sigma$, то мы можем определить частичную функцию $M(t) : M_{s_1} \times \ldots \times M_{s_n} \to M_s$ рекурсией по структуре $t$.
Если $t = x_i$, то $M(t)$ -- это просто проекция на $i$-ую координату.
Если $t = \sigma(t_1, \ldots t_k)$, то $M(t)(a_1, \ldots a_n) = M(\sigma)(M(t_1)(a_1, \ldots a_n), \ldots M(t_k)(a_1, \ldots a_n))$.
Если $\varphi$ -- формула в сигнатуре $\Sigma$ такая, что $\mathrm{FV}(t) \subseteq V$, и $M$ -- интерпретация $\Sigma$, то мы можем определить отношение $M(\varphi) \subseteq M_{s_1} \times \ldots \times M_{s_n}$.
Если $\varphi$ -- формула вида $t_1 = t_2$, то $M(\varphi)$ состоит из таких $(a_1, \ldots a_n)$, что $M(t_1)$ и $M(t_2)$ определены на этих значениях и $M(t_1)(a_1, \ldots a_n) = M(t_2)(a_1, \ldots a_n)$.
Если $\varphi$ -- формула вида $R(t_1, \ldots t_k)$, то $M(\varphi)$ состоит из таких $(a_1, \ldots a_n)$, что функция $M(t_i)$ определена на этих значениях для всех $1 \leq i \leq k$ и $(M(t_1)(a_1, \ldots a_n), \ldots M(t_k)(a_1, \ldots a_n)) \in M(R)$.
Секвенция $\varphi \sststile{}{x_1, \ldots x_n} \psi$ верна в интерпретации $M$ если $M(\varphi) \subseteq M(\psi)$.

\begin{defn}
\emph{Модель} теории -- это интерпретация $M$ ее сигнатуры такая, что все аксиомы теории верны в этой интерпретации.
\end{defn}

\begin{example}
Теория категорий состоит из двух сортов $\fs{ob}$ и $\fs{hom}$, функциональных символов $d,c : \fs{hom} \to \fs{ob}$, $\fs{id} : \fs{ob} \to \fs{hom}$, $\circ : \fs{hom} \times \fs{hom} \to \fs{hom}$ и следующих аксиом:
\begin{align*}
& \sststile{}{f} d(f)\!\downarrow \land c(f)\!\downarrow \\
& \sststile{}{x} d(\fs{id}(x)) = x \land c(\fs{id}(x)) = x \\
c(f) = d(g) & \ssststile{}{f,g} \circ(g,f)\!\downarrow \\
c(f) = d(g) & \sststile{}{f,g} d(\circ(g,f)) = d(f) \land c(\circ(g,f)) = c(g) \\
& \sststile{}{f} \circ(\fs{id}(c(f)),f) = f \land \circ(f,\fs{id}(d(f))) = f \\
c(f) = d(g) \land c(g) = d(h) & \sststile{}{f,g,h} \circ(\circ(h,g),f) = \circ(h,\circ(g,f))
\end{align*}
Первые две аксиомы говорят, что функции $c$, $d$ и $\fs{id}$ тотальны и описывают домен и кодомен морфизма $\fs{id}(x)$.
Третья аксиома говорит, что функция $\circ(g,f)$ определена тогда и только тогда, когда домен $g$ совпадает с кодоменом $f$.
Четвертая аксиома описывает домен и кодомен морфизма $\circ(g,f)$.
Последние две аксиомы говорят, что $\circ$ ассоциативна и $\fs{id}$ является единицей для $\circ$.
Модели этой теории -- это в точности малые категории.
\end{example}

\begin{example}[fc-cats]
Теория конечно полных категорий является расширением теории категорий.
Мы добавляем функциональные символы $1 : \fs{ob}$, $! : \fs{ob} \to \fs{hom}$, $\pi_1,\pi_2 : \fs{hom} \times \fs{hom} \to \fs{hom}$, $\fs{pair} : \fs{hom} \times \fs{hom} \times \fs{hom} \times \fs{hom} \to \fs{hom}$ и следующие аксиомы:
\begin{align*}
& \sststile{}{x} d(!(x)) = x \land c(!(x)) = 1 \\
c(f) = 1 & \sststile{}{f} f = !(d(f)) \\
c(f) = c(g) & \ssststile{}{f,g} \pi_1(f,g)\!\downarrow \\
c(f) = c(g) & \ssststile{}{f,g} \pi_2(f,g)\!\downarrow \\
c(f) = c(g) & \sststile{}{f,g} c(\pi_1(f,g)) = d(f) \land c(\pi_2(f,g)) = d(g) \\
c(f) = c(g) & \sststile{}{f,g} d(\pi_1(f,g)) = d(\pi_2(f,g)) \\
\fs{pair}(a,b,f,g)\!\downarrow & \ssststile{}{f,g,a,b} d(a) = d(b) \land c(a) = d(f) \land c(b) = d(g) \land c(f) = c(g) \\
\fs{pair}(a,b,f,g)\!\downarrow & \sststile{}{f,g,a,b} \circ(\pi_1(f,g),\fs{pair}(a,b,f,g)) = a \\
\fs{pair}(a,b,f,g)\!\downarrow & \sststile{}{f,g,a,b} \circ(\pi_2(f,g),\fs{pair}(a,b,f,g)) = b \\
c(h) = d(\pi_1(f,g)) & \sststile{}{f,g,h} h = \fs{pair}(\circ(\pi_1(f,g),h),\circ(\pi_2(f,g),h),f,g)
\end{align*}
Модели этой теории -- это в точности конечно полные малые категории.
\end{example}

Правила вывода \emph{частичной хорновской логики} приведены ниже.
\emph{Теорема} теории -- это секвенция, выводимая из аксиом этой теории при помощи этих правил вывода.
Мы будем писать $\varphi \sststile{T}{V} \psi$ для обозначения того факта, что $\varphi \sststile{}{V} \psi$ является теоремой теории $T$.

\begin{center}
$\varphi \sststile{}{V} \varphi$ \axlabel{b1}
\qquad
\AxiomC{$\varphi \sststile{}{V} \psi$}
\AxiomC{$\psi \sststile{}{V} \chi$}
\RightLabel{\axlabel{b2}}
\BinaryInfC{$\varphi \sststile{}{V} \chi$}
\DisplayProof
\qquad
$\varphi \sststile{}{V} \top$ \axlabel{b3}
\end{center}

\medskip
\begin{center}
$\varphi \land \psi \sststile{}{V} \varphi$ \axlabel{b4}
\qquad
$\varphi \land \psi \sststile{}{V} \psi$ \axlabel{b5}
\qquad
\AxiomC{$\varphi \sststile{}{V} \psi$}
\AxiomC{$\varphi \sststile{}{V} \chi$}
\RightLabel{\axlabel{b6}}
\BinaryInfC{$\varphi \sststile{}{V} \psi \land \chi$}
\DisplayProof
\end{center}

\medskip
\begin{center}
$\sststile{}{V} x\!\downarrow$ \axlabel{a1}
\qquad
$x = y \land \varphi \sststile{}{V} \varphi[y/x]$ \axlabel{a2}
\end{center}

\medskip
\begin{center}
\AxiomC{$\varphi \sststile{}{V,x} \psi$}
\RightLabel{, $x \in FV(\varphi)$ \axlabel{a3}}
\UnaryInfC{$\varphi[t/x] \sststile{}{V,V'} \psi[t/x]$}
\DisplayProof
\end{center}
\medskip

Эти правила немного отличаются от тех, что приведены в \cite{PHL}.
Правило вывода \axref{a3} там заменено на следующие правила вывода:
\begin{align*}
R(t_1, \ldots t_k) & \sststile{}{V} t_i\!\downarrow \axtag{a4} \\
t_1 = t_2 & \sststile{}{V} t_i\!\downarrow \axtag{a4'} \\
\sigma(t_1, \ldots t_k)\!\downarrow & \sststile{}{V} t_i\!\downarrow \axtag{a5}
\end{align*}

\medskip
\begin{center}
\AxiomC{$\varphi \sststile{}{x_1, \ldots x_n} \psi$}
\RightLabel{\axlabel{a3'}}
\UnaryInfC{$t_1\!\downarrow \land \ldots \land t_n\!\downarrow \land \varphi[t_1/x_1, \ldots t_n/x_n] \sststile{}{V} \psi[t_1/x_1, \ldots t_n/x_n]$}
\DisplayProof
\end{center}
\medskip

\begin{prop}
В присутствии остальных правил вывода, правило \axref{a3} эквивалентно правилам \axref{a3'}, \axref{a4}, \axref{a4'} и \axref{a5}.
\end{prop}
\begin{proof}
Так как секвенции $R(x_1, \ldots x_k) \sststile{}{V} x_i\!\downarrow$, $x_1 = x_2 \sststile{}{V} x_i\!\downarrow$ и $\sigma(x_1, \ldots x_k)\!\downarrow \sststile{}{V} x_i\!\downarrow$ выводимы из \axref{b2}, \axref{b3} и \axref{a1},
то \axref{a3} влечет \axref{a4}, \axref{a4'} и \axref{a5}.
Мы можем доказать, что \axref{a3'} выводимо индукцией по $n$.
Для этого достаточно показать, что следующее правило выводимо:
\medskip
\begin{center}
\AxiomC{$\varphi \sststile{}{V,x} \psi$}
\UnaryInfC{$t\!\downarrow \land \varphi[t/x] \sststile{}{V} \psi[t/x]$}
\DisplayProof
\end{center}
\medskip
Для этого в правиле \axref{a3} достаточно в качестве $\varphi$ взять $x\!\downarrow \land \varphi$.

Теперь покажем, что \axref{a3} выводимо из правил \axref{a3'}, \axref{a4}, \axref{a4'} и \axref{a5}.
По \axref{a3'} из $\varphi \sststile{}{x_1, \ldots x_n, x} \psi$ выводится $x_1\!\downarrow \land \ldots \land x_n\!\downarrow \land t\!\downarrow \land \varphi[t/x] \sststile{}{x_1, \ldots x_n, V'} \psi[t/x]$.
По \axref{b2} и \axref{b6} нам достаточно показать, что $\varphi[t/x] \sststile{}{x_1, \ldots x_n, V'} x_i\!\downarrow$, $\varphi[t/x] \sststile{}{x_1, \ldots x_n, V'} t\!\downarrow$ и $\varphi[t/x] \sststile{}{x_1, \ldots x_n, V'} \varphi[t/x]$.
Первая секвенция следует из \axref{a1}, \axref{b2} и \axref{b3}, а последняя из \axref{b1}.
Докажем, что вторая секвенция выводится.
Если $\varphi$ -- это формула вида $R(t_1, \ldots t_k)$, то $x \in \mathrm{FV}(t_i)$ для некоторого $1 \leq i \leq k$.
По \axref{a4} верно, что $R(t_1[t/x], \ldots t_k[t/x]) \sststile{}{x_1, \ldots x_n, V'} t_i[t/x]\!\downarrow$
Если $\varphi$ -- это формула вида $t_1 = t_2$, то $x \in \mathrm{FV}(t_i)$ для некоторого $1 \leq i \leq 2$.
По \axref{a4'} верно, что $t_1[t/x] = t_2[t/x] \sststile{}{x_1, \ldots x_n, V'} t_i[t/x]\!\downarrow$.
По \axref{b2} достаточно доказать, что секвенция $t'[t/x]\!\downarrow\ \sststile{}{V} t\!\downarrow$ выводима для любого терма $t'$ такого, что $x \in \mathrm{FV}(t')$.
Это легко сделать индукцией по $t'$.
Если $t' = x$, то это верно по \axref{b1}.
Если $t' = \sigma(t_1, \ldots t_k)$, то $x \in \mathrm{FV}(t_i)$ для некоторого $1 \leq i \leq k$.
По \axref{a5} секвенция $\sigma(t_1[t/x], \ldots t_k[t/x])\!\downarrow\ \sststile{}{V} t_i[t/x]\!\downarrow$ выводима.
По индукционной гипотезе $t_i[t/x]\!\downarrow\ \sststile{}{V} t\!\downarrow$.
\end{proof}

\subsection{Естественный вывод}

Позже мы встретим несколько утверждений, которые доказываются индукцией по выводу секвенции.
Мы будем работать с секвенциями, левая сторона которых обладает некоторым свойством, но в выводе секвенции в частичной хорновской логике левая сторона может варьироваться произвольным образом.
Таким образом, нам нужно описать эквивалентный набор правил вывода, в котором левая формула не менялась бы.
Мы будем называть эти правила \emph{естественным выводом}.
В этой системе правая сторона секвенций всегда является атомарной формулой.

\begin{center}
\AxiomC{}
\RightLabel{\axlabel{nv}}
\UnaryInfC{$\varphi \sststile{}{V} x\!\downarrow$}
\DisplayProof
\qquad
\AxiomC{$\varphi \sststile{}{V} t_1 = t_2$}
\RightLabel{\axlabel{ns}}
\UnaryInfC{$\varphi \sststile{}{V} t_2 = t_1$}
\DisplayProof
\end{center}
\medskip

\begin{center}
\AxiomC{}
\RightLabel{\axlabel{nh}}
\UnaryInfC{$\varphi_1 \land \ldots \land \varphi_n \sststile{}{V} \varphi_i$}
\DisplayProof
\qquad
\AxiomC{$\varphi \sststile{}{V} t_1 = t_2$}
\AxiomC{$\varphi \sststile{}{V} \psi[t_1/x]$}
\RightLabel{\axlabel{nl}}
\BinaryInfC{$\varphi \sststile{}{V} \psi[t_2/x]$}
\DisplayProof
\end{center}
\medskip

\begin{center}
\AxiomC{$\varphi \sststile{}{V} R(t_1, \ldots t_n)$}
\RightLabel{\axlabel{np}}
\UnaryInfC{$\varphi \sststile{}{V} t_i\!\downarrow$}
\DisplayProof
\qquad
\AxiomC{$\varphi \sststile{}{V} \sigma(t_1, \ldots t_n)\!\downarrow$}
\RightLabel{\axlabel{nf}}
\UnaryInfC{$\varphi \sststile{}{V} t_i\!\downarrow$}
\DisplayProof
\end{center}
где $R$ -- это предикатный символ теории, а $\sigma$ -- функциональный символ.

Наконец, для каждой аксиомы $\psi_1 \land \ldots \land \psi_n \sststile{}{x_1 : s_1, \ldots x_k : s_k} \chi_1 \land \ldots \land \chi_m$
и всех термов $t_1 : s_1$, \ldots $t_k : s_k$ мы добавляем следующее правило для всех $1 \leq j \leq m$:
\begin{center}
\AxiomC{$\varphi \sststile{}{V} t_i\!\downarrow$, $1 \leq i \leq k$}
\AxiomC{$\varphi \sststile{}{V} \psi_i[t_1/x_1, \ldots t_k/x_k]$, $1 \leq i \leq n$}
\RightLabel{\axlabel{na}}
\BinaryInfC{$\varphi \sststile{}{V} \chi_j[t_1/x_1, \ldots t_k/x_k]$}
\DisplayProof
\end{center}

\begin{prop}
Секвенция $\varphi \sststile{}{V} \psi_1 \land \ldots \land \psi_n$ выводима из правил \axref{b1}-\axref{b6}, \axref{a1}-\axref{a3} тогда и только тогда, когда
секвенции $\varphi \sststile{}{V} \psi_1$, \ldots $\varphi \sststile{}{V} \psi_n$ выводимы из правил естественного вывода.
\end{prop}
\begin{proof}
Легко доказать ``только тогда'' направление.
Наоборот, правила \axref{b1}, \axref{b4} и \axref{b5} следуют из \axref{nh},
правила \axref{b3} и \axref{b6} верны тривиально,
правило \axref{a1} следует из \axref{nv},
правило \axref{a2} следует из \axref{nl} и \axref{nh},
и аксиомы выводимы по \axref{na}.

Чтобы доказать правило \axref{b2}, нам достаточно показать, что если секвенции $\varphi \sststile{}{V} \psi_1$, \ldots $\varphi \sststile{}{V} \psi_n$
и $\psi_1 \land \ldots \land \psi_n \sststile{}{V} \chi$ выводимы в естественном выводе, то $\varphi \sststile{}{V} \chi$ также выводима.
Мы можем сконструировать дерево вывода для этой секвенции как дерево вывода для $\psi_1 \land \ldots \land \psi_n \sststile{}{V} \chi$,
в котором левая сторона каждой секвенции заменена на $\varphi$ и правила \axref{nh} заменены на деревья вывода для $\varphi \sststile{}{V} \psi_i$.

Чтобы доказать правило \axref{a3}, рассмотрим дерево вывода для секвенции $\varphi \sststile{}{V} \psi$.
Чтобы сконструировать дерево вывода для $\varphi[t/x] \sststile{}{V,V'} \psi[t/x]$, нам достаточно применить подстановку $t/x$ к каждой секвенции в этом дереве вывода.
Единственное правило, которое не замкнуто относительно подстановки, -- это \axref{nv}.
По предположению $x \in FV(\varphi)$.
Это влечет, что $\varphi[t/x] \sststile{}{V,V'} t\!\downarrow$ выводима из \axref{np}, \axref{nf} и следующих правил:
\begin{center}
\AxiomC{$\varphi \sststile{}{V} t_1 = t_2$}
\RightLabel{\axlabel{ne1}}
\UnaryInfC{$\varphi \sststile{}{V} t_1\!\downarrow$}
\DisplayProof
\qquad
\AxiomC{$\varphi \sststile{}{V} t_1 = t_2$}
\RightLabel{\axlabel{ne2}}
\UnaryInfC{$\varphi \sststile{}{V} t_2\!\downarrow$}
\DisplayProof
\end{center}
Правило \axref{ne2} следует из \axref{nl} если мы возьмем $\psi(x)$ равным $x = b$.
Правило \axref{ne1} следует из \axref{ne2} и \axref{ns}.
\end{proof}

\subsection{Синтаксическая категория теории}

В этом подразделе для каждой теории $T$ мы определим конечно полную категорию $\cat{C}_T$, которая называется \emph{синтаксической категорией} этой теории.

\emph{Производный сорт} теории -- это пара, состоящая из упорядоченного конечного множества переменных $V = \{ x_1 : s_1, \ldots x_n : s_n \}$ и формулы $\varphi$ такой, что $\mathrm{FV}(\varphi) \subseteq V$.
Мы будем записывать сорт, соответствующий такой паре как $\{ \overline{x} : \overline{s} \mid \varphi \}$ или более коротко как $\{ \overline{x} \mid \varphi \}$.

Каждой теории $T$ можно сопоставить ее \emph{синтаксическую категорию} $\cat{C}_T$.
Объекты этой категории -- это производные сорта $T$.
Морфизм между объектами $\{ (x_1, \ldots x_n) \mid \varphi \}$ и $\{ (y_1, \ldots y_k) : \overline{s} \mid \psi \}$ -- это класс эквивалентности списков термов $(t_1, \ldots t_k)$ таких,
что $t_i$ имеет сорт $s_i$, $\mathrm{FV}(t_i) \subseteq \{ x_1, \ldots x_n \}$ и $\varphi \sststile{T}{\overline{x}} \psi[\overline{t}/\overline{y}] \land t_1\!\downarrow \land \ldots \land t_k\!\downarrow$.
Два списка $(t_1, \ldots t_k)$ и $(t_1', \ldots t_k')$ эквивалентны, если $\varphi \sststile{T}{\overline{x}} t_1 = t_1' \land \ldots \land t_k = t_k'$.

Тождественный морфизм на объекте $\{ \overline{x} \mid \varphi \}$ -- это список $\overline{x}$.
Композиция морфизмов $(t_1, \ldots t_k) : \{ \overline{x} \mid \varphi \} \to \{ \overline{y} \mid \psi \}$
и $(t_1', \ldots t_m') : \{ \overline{y} \mid \psi \} \to \{ \overline{z} \mid \chi \}$ -- это список $(t_1'[\rho], \ldots t_m'[\rho])$, где $\rho(y_i) = t_i$.
Легко видеть, что это определение корректно и действительно задает категорию.

\begin{remark}
Можно модифицировать определение $\cat{C}_T$, объединив объекты, эквивалентные с точностью до переименования переменных и эквивалентности формул в теории $T$.
Легко видеть, что такая категория будет эквивалентна $\cat{C}_T$.
Иногда мы будем отождествлять эквивалентные объекты $\cat{C}_T$.
\end{remark}

В \cite{PHL} приводится другое определение синтаксической категории.
Она определяется как начальный объект теории $\mathrm{Cart} \overline{\omega} T$.
Эта теория является расширением теории, описанной в \pexample{fc-cats}.
Для каждого сорта $s$ теории $T$ мы добавляем константу $\upgamma^s : \fs{ob}$ и аксиому $\sststile{}{} \upgamma^s\!\downarrow$.
Для каждого предикатного символа $R : s_1 \times \ldots \times s_n$ теории $T$ мы добавляем константу $\upgamma^R : \fs{hom}$ и аксиому $\sststile{}{} c(\upgamma^R) = \upgamma^{s_1} \times \ldots \times \upgamma^{s_n} \land \fs{Mon}(\upgamma^R)$,
где $X \times Y$ -- декартово произведение объектов, которое определяется очевидным образом в теории конечно полных категорий, и $\fs{Mon}(f)$ -- предикат, утверждающий, что $f$ является мономорфизмом (это верно тогда и только тогда, когда $\pi_1(f,f) = \pi_2(f,f)$).
Для каждого функционального символа $\sigma : s_1 \times \ldots \times s_n \to s$ теории $T$ мы добавляем константы $\upgamma^\sigma_d, \upgamma^\sigma_m : \fs{hom}$
и аксиому $\sststile{}{} c(\upgamma^\sigma_m) = \upgamma^{s_1} \times \ldots \times \upgamma^{s_n} \land c(\upgamma^\sigma_d) = \upgamma^s \land d(\upgamma^\sigma_d) = d(\upgamma^\sigma_m) \land \fs{Mon}(\upgamma^\sigma_m)$.
Идея заключается в том, что $\upgamma^\sigma_m$ задает некоторый подобъект домена $\sigma$, а $\upgamma^\sigma_d$ является морфизмом из этого подобъекта в кодомен $\sigma$.
То есть такая пара морфизмов -- это в точности частичный морфизм из домена $\sigma$ в его кодомен.

В \cite[Section~8]{PHL} описана категориальная семантика частичных хорновских теорий.
В частности, она описывает интерпретацию термов и формул.
Если $V = \{ x_1 : s_1, \ldots x_n : s_n \}$ -- упорядоченное множество переменных, а $t$ -- терм сорта $s$ такой, что $\mathrm{FV}(t) \subseteq V$,
то категориальная семантика дает нам пару морфизмов $\upgamma^t_d$ и $\upgamma^t_m$, задающих частичный морфизм из $\upgamma^{s_1} \times \ldots \times \upgamma^{s_n} \to \upgamma^s$.
Если $\varphi$ -- формула такая, что $\mathrm{FV}(\varphi) \subseteq V$, то по категориальной семантике мы получаем мономорфизм $\gamma^\varphi$, задающий подобъект $\upgamma^{s_1} \times \ldots \times \upgamma^{s_n}$.
Нам понадобятся следующие леммы, доказанные в \cite[Lemma~39]{PHL} и \cite[Lemma~40]{PHL}:

\begin{lem}[term-subst]
Если $V = \{ x_1 : s_1, \ldots x_n : s_n \}$ и $V' = \{ y_1 : s_1', \ldots y_k : s_k' \}$ -- два упорядоченных множества переменных, $t'$ -- терм такой, что $\mathrm{FV}(t') \subseteq V'$,
и $(t_1, \ldots t_k)$ -- список термов таких, что $\mathrm{FV}(t_i) \subseteq V$ и $t_i$ имеет сорт $s_i'$.
Тогда ограничение частичного морфизма $\upgamma^{t'[t_1/y_1, \ldots t_k/y_k] \land t_1 \land \ldots \land t_k}$ на пересечение подобъектов $\upgamma^{t_i}_m$ изоморфно композиции частичных морфизмов $\gamma^{\langle t_1, \ldots t_k \rangle}$ и $\gamma^{t'}$,
где $\gamma^{\langle t_1, \ldots t_k \rangle}$ -- это морфизм из пересечения подобъектов $\upgamma^{t_i}_m$, который задается как $\langle t_1, \ldots t_k \rangle$.
\end{lem}

\begin{lem}[form-subst]
Если $V = \{ x_1 : s_1, \ldots x_n : s_n \}$ и $V' = \{ y_1 : s_1', \ldots y_k : s_k' \}$ -- два упорядоченных множества переменных, $\psi$ -- формула такая, что $\mathrm{FV}(\psi) \subseteq V'$,
и $(t_1, \ldots t_k)$ -- список термов таких, что $\mathrm{FV}(t_i) \subseteq V$ и $t_i$ имеет сорт $s_i'$.
Тогда подобъект $\upgamma^{\psi[t_1/y_1, \ldots t_k/y_k] \land t_1 \land \ldots \land t_k}$ изоморфен $j \circ p$, где $j$ -- пересечение подобъектов $\upgamma^{t_i}_m$, а $p$ -- прообраз $\upgamma^\psi$ вдоль $\langle t_1, \ldots t_k \rangle$.
\end{lem}

Для каждой аксиомы $a$ вида $\varphi \sststile{}{V} \psi$ мы добавляем константу $\upgamma^a : \fs{hom}$ и аксиому $\sststile{}{} \upgamma^\varphi = \upgamma^\psi \circ \upgamma^a$.
Это завершает определение теории $\mathrm{Cart} \overline{\omega} T$.
Начальную модель этой теории мы будем обозначать $\cat{C}_T'$.
Мы покажем, что эта категория эквивалентна $\cat{C}_T$.

\begin{prop}
Категории $\cat{C}_T$ и $\cat{C}_T'$ эквивалентны.
\end{prop}
\begin{proof}
Во-первых, покажем, что $\cat{C}_T$ является моделью $\mathrm{Cart} \overline{\omega} T$.
Эта категория конечно полная.
Действительно, $\{ () \mid \top \}$ является терминальным объектом.
Если $(t_1, \ldots t_k) : \{ \overline{x} \mid \varphi \} \to \{ \overline{z} \mid \chi \}$ и $(t_1', \ldots t_k') : \{ \overline{y} \mid \psi \} \to \{ \overline{z} \mid \chi \}$ -- два морфизма в $\cat{C}_T$,
то их послойное произведение можно определить как $\{ \overline{x}, \overline{y} \mid \varphi \land \psi \land t_1 = t_1' \land \ldots \land t_k = t_k' \}$.
Легко видеть, что этот объект обладает необходимым универсальным свойством.

Константа $\upgamma^s$ интерпретируется как $\{ x : s \mid \top \}$.
Константа $\upgamma^R$ интерпретируется как $(x_1, \ldots x_n) : \{ (x_1, \ldots x_n) \mid R(x_1, \ldots x_n) \} \to \{ (x_1, \ldots x_n) \mid \top \}$.
Константа $\upgamma^\sigma_m$ интерпретируется как $(x_1, \ldots x_n) : \{ (x_1, \ldots x_n) \mid \sigma(x_1, \ldots x_n)\!\downarrow \} \to \{ (x_1, \ldots x_n) \mid \top \}$.
Константа $\upgamma^\sigma_d$ интерпретируется как $\sigma(x_1, \ldots x_n) : \{ (x_1, \ldots x_n) \mid \sigma(x_1, \ldots x_n)\!\downarrow \} \to \{ y \mid \top \}$.
Чтобы проинтерпретировать константу $\upgamma^a$, соответствующую аксиоме $\varphi \sststile{}{V} \psi$, достаточно показать, что подобъект $\upgamma^\varphi$ вкладывается в $\upgamma^\psi$.
Для этого достаточно показать, что для любой формулы $\varphi$ подобъекты $\overline{x} : \{ \overline{x} \mid \varphi \} \to \{ \overline{x} \mid \top \}$ и $\upgamma^\varphi$ эквивалентны.
Легко индукцией по структуре терма $t$ показать, что подобъекты $\upgamma^t_m$ и $\overline{x} : \{ \overline{x} \mid t\!\downarrow \} \to \{ \overline{x} \mid \top \}$ эквивалентны
и $\upgamma^t_d$ соответствует морфизму $t : \{ \overline{x} \mid t\!\downarrow \} \to \{ y \mid \top \}$.
Используя этот факт, легко показать необходимое свойство формул.

Таким образом, $\cat{C}_T$ действительно является моделью $\mathrm{Cart} \overline{\omega} T$.
Следовательно существует уникальный функтор $F : \cat{C}_T' \to \cat{C}_T$, являющийся морфизмом моделей.
Постороим функтор $G$ в обратную сторону.
Объект $\{ \overline{x} \mid \varphi \}$ отображается в $d(\gamma^\varphi)$.
Если $(t_1, \ldots t_k) : \{ \overline{x} \mid \varphi \} \to \{ \overline{y} \mid \psi \}$ -- морфизм, то у нас есть стрелка $d(\upgamma^\varphi) \to d(\psi[t_1/y_1, \ldots t_k/y_k] \land t_1\!\downarrow \land \ldots \land t_k\!\downarrow)$.
По \dlem{form-subst} у нас есть стрелка $d(\psi[t_1/y_1, \ldots t_k/y_k] \land t_1\!\downarrow \land \ldots \land t_k\!\downarrow) \to d(\psi)$.
Мы определяем $G(t_1, \ldots t_k)$ как композицию этих двух стрелок.
Легко видеть, что $G$ сохраняет тождественные морфизмы.
\Rlem{term-subst} влечет, что $G$ сохраняет композицию.

Мы не можем воспользоваться универсальным свойством $\cat{C}_T'$, чтобы доказать, что $G \circ F = \fs{Id}$, так как $G$ не является морфизмом моделей $\mathrm{Cart} \overline{\omega} T$.
Вместо этого мы построим естественный изоморфизм между этими функторами.
Для этого мы определим частичную функцию $\alpha$ на замкнутых термах $t$ теории $\mathrm{Cart} \overline{\omega} T$ таких, что $\sststile{T}{} t\!\downarrow$.
Эта функция будет удовлетворять следующим условиям:
\begin{itemize}
\item Если $t$ имеет сорт $\fs{ob}$ и $\alpha_t$ определена, то $\alpha_t$ -- это изоморфизм $GF(t) \to t$, то есть терм сорта $\fs{hom}$ такой, что $\sststile{T}{} d(\alpha_t) = GF(t) \land c(\alpha_t) = t$ и что существует обратный к нему морфизм.
\item Если $t$ имеет сорт $\fs{hom}$ и $\alpha_t$ определена, то $\alpha_t$ -- это пара изоморфизмов $\alpha_t^d : GF(d(t)) \to d(t)$ и $\alpha_t^c : GF(c(t)) \to c(t)$.
\end{itemize}

Функция $\alpha_t$ определяется рекурсией по $t$.
TODO

% сопоставляет терм $\alpha_t$ сорта $\fs{hom}$ такой, что, если $t$ имеет сорт $\fs{ob}$, то $\sststile{T}{} d(\alpha_t) = GF(t) \land c(\alpha_t) = t$, и, если $t$ имеет сорт $\fs{hom}$, то $\sststile{T}{} 1$.
% Мы покажем, что если $x$ имеет сорт $\fs{ob}$, то $\sststile{T}{} d(\alpha_x) = GF(x) \land c(\alpha_x) = x$, и если $f$ имеет сорт $\fs{hom}$, то $\sststile{T}{} \alpha_f = \circ(f,\alpha_{d(f)}) \land \alpha_f = \circ(\alpha_{c(f)}, GF(f))$.
% Мы определим $\alpha_t$ рекурсией по $t$ одновременно с доказательством факта, приведенного выше.
% Значения $\alpha_{d(f)}$ и $\alpha_{c(f)}$ мы определим рекурсией по $f$.
% Другими словами, для каждого терма $f$ сорта $\fs{hom}$ нам достаточно определить пару термов $\alpha_{d(f)}$ и $\alpha_{c(f)}$ таких, что $\sststile{T}{} \circ(f,\alpha_{d(f)} = \circ(\alpha_{c(f)}, GF(f))$.

% Для этого мы сначала определим частичную функцию $\alpha$, которая терму $x$ теории $\mathrm{Cart} \overline{\omega} T$ сорта $\fs{ob}$ сопоставляет терм $\alpha_x$ сорта $\fs{hom}$,
% а терму $f$ сорта $\fs{hom}$ сопоставляет пару термов $\alpha_f^d$ и $\alpha_f^c$ этого же сорта.
% Идея в том, что каждому объекту $x$ эта функция сопоставляется изоморфизм $\alpha_x : GF(x) \to x$, а каждому морфизму $f : x \to y$ она сопоставляет пару изоморфизмов $\alpha_x$ и $\alpha_y$.
% Истинность этого утверждения мы докажем позже.

Нам осталось доказать, что $F \circ G$ изоморфен $\fs{Id}$.
Пусть $X = \{ \overline{x} : \overline{s} \mid \varphi \}$ -- объект $\cat{C}_T$.
Тогда $FG(X) = F(d(\upgamma_\varphi))$.
Мы уже использовали тот факт, что этот объект изоморфен исходному объекту $X$.
Легко видеть, что этот изоморфизм естественен.
\end{proof}

\subsection{Категория теорий}

В этом подразделе мы дадим первое определение категории теорий.
В следующем подразделе мы дадим более явное описание морфизмов этой категории.

\bibliographystyle{amsplain}
\bibliography{ref}

\end{document}
