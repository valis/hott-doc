\documentclass[reqno]{amsart}

\usepackage[russian]{babel}
\usepackage[utf8]{inputenc}
\usepackage{mathtools}
\usepackage{cmap}
\usepackage{amsfonts}
\usepackage{upgreek}
\usepackage{xargs}
\usepackage{ifthen}
\usepackage[all]{xy}
\usepackage{hyperref}
\usepackage{etex}
\usepackage{bussproofs}
\usepackage{turnstile}
\usepackage{amssymb}
\usepackage{verbatim}

\hypersetup{colorlinks=true,linkcolor=blue}

\renewcommand{\turnstile}[6][s]
    {\ifthenelse{\equal{#1}{d}}
        {\sbox{\first}{$\displaystyle{#4}$}
        \sbox{\second}{$\displaystyle{#5}$}}{}
    \ifthenelse{\equal{#1}{t}}
        {\sbox{\first}{$\textstyle{#4}$}
        \sbox{\second}{$\textstyle{#5}$}}{}
    \ifthenelse{\equal{#1}{s}}
        {\sbox{\first}{$\scriptstyle{#4}$}
        \sbox{\second}{$\scriptstyle{#5}$}}{}
    \ifthenelse{\equal{#1}{ss}}
        {\sbox{\first}{$\scriptscriptstyle{#4}$}
        \sbox{\second}{$\scriptscriptstyle{#5}$}}{}
    \setlength{\dashthickness}{0.111ex}
    \setlength{\ddashthickness}{0.35ex}
    \setlength{\leasturnstilewidth}{2em}
    \setlength{\extrawidth}{0.2em}
    \ifthenelse{%
      \equal{#3}{n}}{\setlength{\tinyverdistance}{0ex}}{}
    \ifthenelse{%
      \equal{#3}{s}}{\setlength{\tinyverdistance}{0.5\dashthickness}}{}
    \ifthenelse{%
      \equal{#3}{d}}{\setlength{\tinyverdistance}{0.5\ddashthickness}
        \addtolength{\tinyverdistance}{\dashthickness}}{}
    \ifthenelse{%
      \equal{#3}{t}}{\setlength{\tinyverdistance}{1.5\dashthickness}
        \addtolength{\tinyverdistance}{\ddashthickness}}{}
        \setlength{\verdistance}{0.4ex}
        \settoheight{\lengthvar}{\usebox{\first}}
        \setlength{\raisedown}{-\lengthvar}
        \addtolength{\raisedown}{-\tinyverdistance}
        \addtolength{\raisedown}{-\verdistance}
        \settodepth{\raiseup}{\usebox{\second}}
        \addtolength{\raiseup}{\tinyverdistance}
        \addtolength{\raiseup}{\verdistance}
        \setlength{\lift}{0.8ex}
        \settowidth{\firstwidth}{\usebox{\first}}
        \settowidth{\secondwidth}{\usebox{\second}}
        \ifthenelse{\lengthtest{\firstwidth = 0ex}
            \and
            \lengthtest{\secondwidth = 0ex}}
                {\setlength{\turnstilewidth}{\leasturnstilewidth}}
                {\setlength{\turnstilewidth}{2\extrawidth}
        \ifthenelse{\lengthtest{\firstwidth < \secondwidth}}
            {\addtolength{\turnstilewidth}{\secondwidth}}
            {\addtolength{\turnstilewidth}{\firstwidth}}}
        \ifthenelse{\lengthtest{\turnstilewidth < \leasturnstilewidth}}{\setlength{\turnstilewidth}{\leasturnstilewidth}}{}
    \setlength{\turnstileheight}{1.5ex}
    \sbox{\turnstilebox}
    {\raisebox{\lift}{\ensuremath{
        \makever{#2}{\dashthickness}{\turnstileheight}{\ddashthickness}
        \makehor{#3}{\dashthickness}{\turnstilewidth}{\ddashthickness}
        \hspace{-\turnstilewidth}
        \raisebox{\raisedown}
        {\makebox[\turnstilewidth]{\usebox{\first}}}
            \hspace{-\turnstilewidth}
            \raisebox{\raiseup}
            {\makebox[\turnstilewidth]{\usebox{\second}}}
        \makever{#6}{\dashthickness}{\turnstileheight}{\ddashthickness}}}}
        \mathrel{\usebox{\turnstilebox}}}

\newcommand{\axlabel}[1]{(#1) \phantomsection \label{ax:#1}}
\newcommand{\axtag}[1]{\label{ax:#1} \tag{#1}}
\newcommand{\axref}[1]{(\hyperref[ax:#1]{#1})}

\newcommand{\newref}[6][]{
\ifthenelse{\equal{#1}{}}{\newtheorem{h#2}[hthm]{#4}}{\newtheorem{h#2}{#4}[#1]}
\expandafter\newcommand\csname r#2\endcsname[1]{#3~\ref{#2:##1}}
\expandafter\newcommand\csname R#2\endcsname[1]{#4~\ref{#2:##1}}
\expandafter\newcommand\csname d#2\endcsname[1]{#5~\ref{#2:##1}}
\expandafter\newcommand\csname p#2\endcsname[1]{#6~\ref{#2:##1}}
\expandafter\newcommand\csname n#2\endcsname[1]{\ref{#2:##1}}
\newenvironmentx{#2}[2][1=,2=]{
\ifthenelse{\equal{##2}{}}{\begin{h#2}}{\begin{h#2}[##2]}
\ifthenelse{\equal{##1}{}}{}{\label{#2:##1}}
}{\end{h#2}}
}

\newref[section]{thm}{теорема}{Теорема}{теореме}{теореме}
\newref{lem}{лемма}{Лемма}{лемме}{лемме}
\newref{prop}{утверждение}{Утверждение}{утверждению}{утверждении}
\newref{cor}{следствие}{Следствие}{следствию}{следствии}

\theoremstyle{definition}
\newref{defn}{определение}{Определение}{определению}{определении}
\newref{example}{пример}{Пример}{примеру}{примере}

\theoremstyle{remark}
\newref{remark}{замечание}{Замечание}{замечанию}{замечании}

\newcommand{\bcat}[1]{\mathbf{#1}}
\newcommand{\cat}[1]{\mathcal{#1}}
\newcommand{\Set}{\bcat{Set}}
\newcommand{\fs}[1]{\mathrm{#1}}
\newcommand{\nf}{\mathrm{nf}}
\newcommand{\FV}{\fs{FV}}
\newcommand{\repl}{:=}
\newcommand{\Term}{\mathrm{Term}}

\newenvironment{tolerant}[1]{\par\tolerance=#1\relax}{\par}

\newcommand{\pb}[1][dr]{\save*!/#1-1.2pc/#1:(-1,1)@^{|-}\restore}

\begin{document}

\title{Title}

\author{Valery Isaev}

\maketitle

\section{Введение}

\section{Частичные хорновские теории}

В данной диссертации мы будем работать с определенным классом логических теорий, которые известны под разными названиями, но наиболее распространенный термин -- это \emph{существенно алгебраические теории}.
Существует несколько эквивалентных способов определить такие теории:
\begin{enumerate}
\item Декартовы теории \cite[Definition~D1.3.4]{elephant} -- это специальный вид теорий в логике первого порядка, где единственные логческие связки -- это конъюнкция и квантор существования.
Кроме того, множество аксиом должно удовлетворять определенному условию.
Мы не будем использовать это понятие, поэтому точное определение нам не понадобится.
\item Обобщенные алгебраические теории \cite{GAT} могут содержать сорта, зависящие от других сортов.
\item Существенно алгебраические теории \cite[Definition~3.34]{LPC} -- это теории, в которых некоторые функциональные символы могут интерпретироваться как \emph{частичные} функции.
Мы не будем использовать это понятие, поэтому точное определение нам не понадобится.
\item Частичные хорновские теории \cite{PHL} являются обобщением существенно алгебраических теорий.
\item Категорное определение теорий является самым простым.
Согласно нему существенно алгебраическая теория -- это просто конечно полная малая категория.
Модель такой теории $\cat{C}$ -- это просто функтор $\cat{C} \to \Set$, сохраняющий конечные пределы.
Другое преимущество этого определения заключается в том, что легко определить структуру категории (и даже 2-категории) на классе теорий.
\end{enumerate}

Мы будем работать с частичными хорновскими теориями, определение которых мы приведем в следующем подразделе.
Часто аксиомы теории пораждаются отношением, которое обладает различными хорошими свойствами такими, как сильная нормализация и конфлюэнтность.
Мы изучим такие теории в этом разделе.
Кроме того, мы зададим структуру 2-категории на них и докажем, что она эквивалентна 2-категории конечно полных малых категорий.

\subsection{Определение}

В этом подразделе мы приведем основные определения из \cite{PHL}.
Мы дадим более общее определение инфинитарных теорий.
\emph{Сигнатура} -- это тройка $(\mathcal{S},\mathcal{F},\mathcal{P})$, где $\mathcal{S}$ -- множество сортов, $\mathcal{F}$ -- множество функциональных символов, и $\mathcal{P}$ -- множество предикатных символов.
Каждому функциональному символу $\sigma$ сопоставляется его сигнатура, то есть множество $I$, функция $s : I \to \mathcal{S}$ и сорт $s$, что записывается как $\sigma : \prod_{i \in I} s_i \to s$.
Каждому предикатному символу $R$ сопоставляется множество $I$ и функция $s : I \to \mathcal{S}$, что записывается как $R : \prod_{i \in I} s_i$.
Если $I = \{ 1, \ldots n \}$ -- конечное множество, то сигнатуры $\sigma$ и $R$ мы будем записывать как $\sigma : s_1 \times \ldots \times s_n \to s$ и $R : s_1 \times \ldots \times s_n$.
Если $\lambda$ -- регулярный кардинал, то мы будем говорить, что сигнатура \emph{$\lambda$-достижима}, если кардинальность всех множеств $I$ меньше $\lambda$.
Мы будем говорить, что сигнатура \emph{финитарна}, если она $\omega$-достижима.

Пусть $V$ -- $\mathcal{S}$-множество, то есть коллекция множеств, индексированная $\mathcal{S}$.
Тогда мы можем определить $\mathcal{S}$-множество термов $\Term(V)$ с переменными в $V$ индуктивным образом:
\begin{enumerate}
\item Любая переменная сорта $s$ является термом этого сорта.
\item Если $\sigma : \prod_{i \in I} s_i \to s$ -- функциональный символ, и $t_i$ -- терм сорта $s_i$ для всех $i \in I$, то $\sigma(\{ t_i \}_i)$ -- терм сорта $s$.
\end{enumerate}

\emph{Атомарная формула} -- это выражение вида $t_1 = t_2$, где $t_1$ и $t_2$ -- термы одного и того же сорта,
либо вида $R(\{ t_i \}_i)$, где $R : \prod_{i \in I} s_i$ -- предикатный символ, а $t_i$ -- терм сорта $s_i$ для всех $i \in I$.
\emph{(Хорновская) формула} -- это выражение вида $\bigwedge_{i \in I} \varphi_i$, где $\varphi_i$ -- атомарные формулы.
Конъюнкция пустого множества формул будет обозначаться как $\top$.
Выражение $t\!\downarrow$ является сокращением для $t = t$.
Функциональные символы интерпретируются как частичные функции.
Формула $t\!\downarrow$ означает, что все подвыражения в $t$ определены.
\emph{Секвенция} -- это выражение вида $\varphi \sststile{}{V} \psi$, где $V$ -- $\mathcal{S}$-множество, а $\varphi$ и $\psi$ -- формулы с переменными в $V$.
Множество переменных, встречающихся в формуле или терме $\varphi$ будет обозначаться как $\FV(\varphi)$.
Если $I = \{ 1, \ldots n \}$, то мы будем записывать $\bigwedge_{i \in I} \varphi_i$ как $\varphi_1 \land \ldots \land \varphi_n$.
Вместо $\varphi_1 \land \ldots \land \varphi_n \sststile{}{V} \psi$ мы будем часто писать $\varphi_1, \ldots \varphi_n \sststile{}{V} \psi$.

Мы будем использовать следующие сокращения:
\begin{align*}
\varphi \sststile{}{V} t \cong s & \Longleftrightarrow \varphi \land t\!\downarrow\,\sststile{}{V} t = s \text{ и } \varphi \land s\!\downarrow\,\sststile{}{V} t = s \\
\varphi \ssststile{}{V} \psi & \Longleftrightarrow \varphi \sststile{}{V} \psi \text{ и } \psi \sststile{}{V} \varphi
\end{align*}

\begin{defn}
\emph{Частичная хорновская теория} -- это четверка $(\mathcal{S},\mathcal{F},\mathcal{P},\mathcal{A})$, где $(\mathcal{S},\mathcal{F},\mathcal{P})$ -- это сигнатура, а $\mathcal{A}$ -- множество секвенций.
Элементы $\mathcal{A}$ мы будем называть аксиомами.
Если $\lambda$ -- регулярный кардинал, то мы будем говорить, что теория \emph{$\lambda$-достижима}, если подлежащая сигнатура $\lambda$-достижима и для каждой аксиомы $\bigwedge_{i \in I} \varphi_i \sststile{}{V} \bigwedge_{j \in J} \psi_j$
верно, что $|\coprod_{s \in \mathcal{S}} V_s| < \lambda$, $|I| < \lambda$ и $|J| < \lambda$.
Теория \emph{финитарна}, если она $\omega$-достижима.
\end{defn}

\begin{remark}
Так как частичные хорновские теории -- это единственный вид теорий, с которым мы будем работать, то мы их будем называть просто теориями.
\end{remark}

\emph{Интерпретация} сигнатуры $(\mathcal{S},\mathcal{F},\mathcal{P})$ -- это $\mathcal{S}$-множество $M$
вместе с коллекцией частичных функций $M(\sigma) : \prod_{i \in I} M_{s_i} \to M_s$ для каждого функционального символа $\sigma : \prod_{i \in I} s_i \to s$
и коллекцией отношений $M(R) \subseteq \prod_{i \in I} M_{s_i}$ для каждого предикатного символа $R : \prod_{i \in I} s_i$.

Пусть $V$ -- некоторое $\mathcal{S}$-множество.
Если $t$ -- терм сорта $s$ в сигнатуре $\Sigma$ с переменными в $V$, и $M$ -- интерпретация $\Sigma$, то мы можем определить частичную функцию $M(t) : \prod_{s \in \mathcal{S}} M_s^{V_s} \to M_s$ рекурсией по структуре $t$.
Если $t = x \in V_s$, то $M(t)(f) = f_s(x)$.
Если $t = \sigma(\{ t_i \}_i)$, то $M(t)(f) = M(\sigma)(\{ M(t_i)(f) \}_i)$.
Если $\varphi$ -- формула в сигнатуре $\Sigma$ с переменными в $V$, и $M$ -- интерпретация $\Sigma$, то мы можем определить отношение $M(\varphi) \subseteq \prod_{s \in \mathcal{S}} M_s^{V_s}$.
Если $\varphi$ -- формула вида $t_1 = t_2$, то $M(\varphi)$ состоит из таких $f$, что $M(t_1)(f)$ и $M(t_2)(f)$ определены и $M(t_1)(f) = M(t_2)(f)$.
Если $\varphi$ -- формула вида $R(\{ t_i \}_i)$, то $M(\varphi)$ состоит из таких $f$, что функции $M(t_i)(f)$ определены для всех $i \in I$ и $\{ M(t_i)(f) \}_i \in M(R)$.
Если $\varphi = \bigwedge_{i \in I} \varphi_i$, то $M(\varphi) = \bigcap_{i \in I} M(\varphi_i)$.
Секвенция $\varphi \sststile{}{V} \psi$ верна в интерпретации $M$ если $M(\varphi) \subseteq M(\psi)$.

\begin{defn}
\emph{Модель} теории -- это интерпретация ее сигнатуры такая, что все аксиомы теории верны в этой интерпретации.
\end{defn}

\begin{example}
Теория категорий -- это финитарная теория, состоящая из двух сортов $\fs{ob}$ и $\fs{hom}$, функциональных символов $d,c : \fs{hom} \to \fs{ob}$, $\fs{id} : \fs{ob} \to \fs{hom}$, $\circ : \fs{hom} \times \fs{hom} \to \fs{hom}$ и следующих аксиом:
\begin{align*}
& \sststile{}{f} d(f)\!\downarrow \land c(f)\!\downarrow \\
& \sststile{}{x} d(\fs{id}(x)) = x \land c(\fs{id}(x)) = x \\
c(f) = d(g) & \ssststile{}{f,g} \circ(g,f)\!\downarrow \\
c(f) = d(g) & \sststile{}{f,g} d(\circ(g,f)) = d(f) \land c(\circ(g,f)) = c(g) \\
& \sststile{}{f} \circ(\fs{id}(c(f)),f) = f \land \circ(f,\fs{id}(d(f))) = f \\
c(f) = d(g) \land c(g) = d(h) & \sststile{}{f,g,h} \circ(\circ(h,g),f) = \circ(h,\circ(g,f))
\end{align*}
Первые две аксиомы говорят, что функции $c$, $d$ и $\fs{id}$ тотальны и описывают домен и кодомен морфизма $\fs{id}(x)$.
Третья аксиома говорит, что функция $\circ(g,f)$ определена тогда и только тогда, когда домен $g$ совпадает с кодоменом $f$.
Четвертая аксиома описывает домен и кодомен морфизма $\circ(g,f)$.
Последние две аксиомы говорят, что $\circ$ ассоциативна и $\fs{id}$ является единицей для $\circ$.
Модели этой теории -- это в точности малые категории.
\end{example}

\begin{example}[fc-cats]
Теория конечно полных категорий является расширением теории категорий.
Мы добавляем функциональные символы $1 : \fs{ob}$, $! : \fs{ob} \to \fs{hom}$, $\pi_1,\pi_2 : \fs{hom} \times \fs{hom} \to \fs{hom}$, $\fs{pair} : \fs{hom} \times \fs{hom} \times \fs{hom} \times \fs{hom} \to \fs{hom}$ и следующие аксиомы:
\begin{align*}
& \sststile{}{x} d(!(x)) = x \land c(!(x)) = 1 \\
c(f) = 1 & \sststile{}{f} f =\ !(d(f)) \\
c(f) = c(g) & \ssststile{}{f,g} \pi_1(f,g)\!\downarrow \\
c(f) = c(g) & \ssststile{}{f,g} \pi_2(f,g)\!\downarrow \\
c(f) = c(g) & \sststile{}{f,g} c(\pi_1(f,g)) = d(f) \land c(\pi_2(f,g)) = d(g) \\
c(f) = c(g) & \sststile{}{f,g} d(\pi_1(f,g)) = d(\pi_2(f,g)) \\
\fs{pair}(a,b,f,g)\!\downarrow & \ssststile{}{f,g,a,b} d(a) = d(b) \land c(a) = d(f) \land c(b) = d(g) \land c(f) = c(g) \\
\fs{pair}(a,b,f,g)\!\downarrow & \sststile{}{f,g,a,b} \circ(\pi_1(f,g),\fs{pair}(a,b,f,g)) = a \\
\fs{pair}(a,b,f,g)\!\downarrow & \sststile{}{f,g,a,b} \circ(\pi_2(f,g),\fs{pair}(a,b,f,g)) = b \\
c(h) = d(\pi_1(f,g)) & \sststile{}{f,g,h} \fs{pair}(\circ(\pi_1(f,g),h),\circ(\pi_2(f,g),h),f,g) = h
\end{align*}
Модели этой теории -- это в точности конечно полные малые категории.
\end{example}

Правила вывода \emph{частичной хорновской логики} приведены ниже.
В этом подразделе мы приведем правила только для финитарных теорий, правила для произвольных будут приведены в следующем.
\emph{Теорема} теории -- это секвенция, выводимая из аксиом этой теории при помощи этих правил вывода.
Мы будем писать $\varphi \sststile{T}{V} \psi$ для обозначения того факта, что $\varphi \sststile{}{V} \psi$ является теоремой теории $T$.

\begin{center}
$\varphi \sststile{}{V} \varphi$ \axlabel{b1}
\qquad
\AxiomC{$\varphi \sststile{}{V} \psi$}
\AxiomC{$\psi \sststile{}{V} \chi$}
\RightLabel{\axlabel{b2}}
\BinaryInfC{$\varphi \sststile{}{V} \chi$}
\DisplayProof
\qquad
$\varphi \sststile{}{V} \top$ \axlabel{b3}
\end{center}

\medskip
\begin{center}
$\varphi \land \psi \sststile{}{V} \varphi$ \axlabel{b4}
\qquad
$\varphi \land \psi \sststile{}{V} \psi$ \axlabel{b5}
\qquad
\AxiomC{$\varphi \sststile{}{V} \psi$}
\AxiomC{$\varphi \sststile{}{V} \chi$}
\RightLabel{\axlabel{b6}}
\BinaryInfC{$\varphi \sststile{}{V} \psi \land \chi$}
\DisplayProof
\end{center}

\medskip
\begin{center}
$\sststile{}{V} x\!\downarrow$ \axlabel{a1}
\qquad
$x = y \land \varphi \sststile{}{V} \varphi[y/x]$ \axlabel{a2}
\end{center}

\medskip
\begin{center}
\AxiomC{$\varphi \sststile{}{V,x} \psi$}
\RightLabel{, $x \in \FV(\varphi)$ \axlabel{a3}}
\UnaryInfC{$\varphi[t/x] \sststile{}{V,V'} \psi[t/x]$}
\DisplayProof
\end{center}
\medskip

Эти правила немного отличаются от тех, что приведены в \cite{PHL}.
Правило вывода \axref{a3} там заменено на следующие правила вывода:
\begin{align*}
R(t_1, \ldots t_k) & \sststile{}{V} t_i\!\downarrow \axtag{a4} \\
t_1 = t_2 & \sststile{}{V} t_i\!\downarrow \axtag{a4'} \\
\sigma(t_1, \ldots t_k)\!\downarrow & \sststile{}{V} t_i\!\downarrow \axtag{a5}
\end{align*}

\medskip
\begin{center}
\AxiomC{$\varphi \sststile{}{x_1, \ldots x_n} \psi$}
\RightLabel{\axlabel{a3'}}
\UnaryInfC{$t_1\!\downarrow \land \ldots \land t_n\!\downarrow \land \varphi[t_1/x_1, \ldots t_n/x_n] \sststile{}{V} \psi[t_1/x_1, \ldots t_n/x_n]$}
\DisplayProof
\end{center}
\medskip

\begin{prop}
В присутствии остальных правил вывода, правило \axref{a3} эквивалентно правилам \axref{a3'}, \axref{a4}, \axref{a4'} и \axref{a5}.
\end{prop}
\begin{proof}
Так как секвенции $R(x_1, \ldots x_k) \sststile{}{V} x_i\!\downarrow$, $x_1 = x_2 \sststile{}{V} x_i\!\downarrow$ и $\sigma(x_1, \ldots x_k)\!\downarrow \sststile{}{V} x_i\!\downarrow$ выводимы из \axref{b2}, \axref{b3} и \axref{a1},
то \axref{a3} влечет \axref{a4}, \axref{a4'} и \axref{a5}.
Мы можем доказать, что \axref{a3'} выводимо индукцией по $n$.
Для этого достаточно показать, что следующее правило выводимо:
\medskip
\begin{center}
\AxiomC{$\varphi \sststile{}{V,x} \psi$}
\UnaryInfC{$t\!\downarrow \land \varphi[t/x] \sststile{}{V} \psi[t/x]$}
\DisplayProof
\end{center}
\medskip
Для этого в правиле \axref{a3} достаточно в качестве $\varphi$ взять $x\!\downarrow \land \varphi$.

Теперь покажем, что \axref{a3} выводимо из правил \axref{a3'}, \axref{a4}, \axref{a4'} и \axref{a5}.
По \axref{a3'} из $\varphi \sststile{}{x_1, \ldots x_n, x} \psi$ выводится $x_1\!\downarrow \land \ldots \land x_n\!\downarrow \land t\!\downarrow \land \varphi[t/x] \sststile{}{x_1, \ldots x_n, V'} \psi[t/x]$.
По \axref{b2} и \axref{b6} нам достаточно показать, что $\varphi[t/x] \sststile{}{x_1, \ldots x_n, V'} x_i\!\downarrow$, $\varphi[t/x] \sststile{}{x_1, \ldots x_n, V'} t\!\downarrow$ и $\varphi[t/x] \sststile{}{x_1, \ldots x_n, V'} \varphi[t/x]$.
Первая секвенция следует из \axref{a1}, \axref{b2} и \axref{b3}, а последняя из \axref{b1}.
Докажем, что вторая секвенция выводится.
Если $\varphi$ -- это формула вида $R(t_1, \ldots t_k)$, то $x \in \FV(t_i)$ для некоторого $1 \leq i \leq k$.
По \axref{a4} верно, что $R(t_1[t/x], \ldots t_k[t/x]) \sststile{}{x_1, \ldots x_n, V'} t_i[t/x]\!\downarrow$
Если $\varphi$ -- это формула вида $t_1 = t_2$, то $x \in \FV(t_i)$ для некоторого $1 \leq i \leq 2$.
По \axref{a4'} верно, что $t_1[t/x] = t_2[t/x] \sststile{}{x_1, \ldots x_n, V'} t_i[t/x]\!\downarrow$.
По \axref{b2} достаточно доказать, что секвенция $t'[t/x]\!\downarrow\ \sststile{}{V} t\!\downarrow$ выводима для любого терма $t'$ такого, что $x \in \FV(t')$.
Это легко сделать индукцией по $t'$.
Если $t' = x$, то это верно по \axref{b1}.
Если $t' = \sigma(t_1, \ldots t_k)$, то $x \in \FV(t_i)$ для некоторого $1 \leq i \leq k$.
По \axref{a5} секвенция $\sigma(t_1[t/x], \ldots t_k[t/x])\!\downarrow\ \sststile{}{V} t_i[t/x]\!\downarrow$ выводима.
По индукционной гипотезе $t_i[t/x]\!\downarrow\ \sststile{}{V} t\!\downarrow$.
\end{proof}

Позже нам понадобится следующая лемма:

\begin{lem}[mcf]
Секвенция $\varphi \sststile{}{x_1, \ldots x_n} \psi$ доказуема в финитарной теории $T$ тогда и только тогда,
когда секвенция $\sststile{}{} \psi[c_1/x_1, \ldots c_n/x_n]$ доказуема в теории $T \cup \{ \sststile{}{} c_i\!\downarrow\ \mid 1 \leq i \leq n \} \cup \{ \sststile{}{} \varphi[c_1/x_1, \ldots c_n/x_n] \}$, где $c_1$, \ldots $c_n$ -- новые константы.
\end{lem}
\begin{proof}
Это следует из \cite[Theorem~10, Theorem~11]{PHL}.
\end{proof}

Мы будем говорить, что две теории \emph{эквивалентны}, если их множества функциональных символов, предикатных символов и теорем совпадают.

\subsection{Естественный вывод}

Позже мы встретим несколько утверждений, которые доказываются индукцией по выводу секвенции.
Мы будем работать с секвенциями, левая сторона которых обладает некоторым свойством, но в выводе секвенции в частичной хорновской логике левая сторона может варьироваться произвольным образом.
Таким образом, нам нужно описать эквивалентный набор правил вывода, в котором левая формула не менялась бы.
Мы будем называть эти правила \emph{естественным выводом}.
В этой системе правая сторона секвенций всегда является атомарной формулой.

\begin{center}
\AxiomC{}
\RightLabel{\axlabel{nv}}
\UnaryInfC{$\varphi \sststile{}{V} x\!\downarrow$}
\DisplayProof
\qquad
\AxiomC{$\varphi \sststile{}{V} t_1 = t_2$}
\RightLabel{\axlabel{ns}}
\UnaryInfC{$\varphi \sststile{}{V} t_2 = t_1$}
\DisplayProof
\end{center}
\medskip

\begin{center}
\AxiomC{}
\RightLabel{\axlabel{nh}}
\UnaryInfC{$\varphi_1 \land \ldots \land \varphi_n \sststile{}{V} \varphi_i$}
\DisplayProof
\qquad
\AxiomC{$\varphi \sststile{}{V} t_1 = t_2$}
\AxiomC{$\varphi \sststile{}{V} \psi[t_1/x]$}
\RightLabel{\axlabel{nl}}
\BinaryInfC{$\varphi \sststile{}{V} \psi[t_2/x]$}
\DisplayProof
\end{center}
\medskip

\begin{center}
\AxiomC{$\varphi \sststile{}{V} R(t_1, \ldots t_n)$}
\RightLabel{\axlabel{np}}
\UnaryInfC{$\varphi \sststile{}{V} t_i\!\downarrow$}
\DisplayProof
\qquad
\AxiomC{$\varphi \sststile{}{V} \sigma(t_1, \ldots t_n)\!\downarrow$}
\RightLabel{\axlabel{nf}}
\UnaryInfC{$\varphi \sststile{}{V} t_i\!\downarrow$}
\DisplayProof
\end{center}
где $R$ -- это предикатный символ теории, а $\sigma$ -- функциональный символ.

Наконец, для каждой аксиомы $\psi_1 \land \ldots \land \psi_n \sststile{}{x_1 : s_1, \ldots x_k : s_k} \chi_1 \land \ldots \land \chi_m$
и всех термов $t_1 : s_1$, \ldots $t_k : s_k$ мы добавляем следующее правило для всех $1 \leq j \leq m$:
\begin{center}
\AxiomC{$\varphi \sststile{}{V} t_i\!\downarrow$, $1 \leq i \leq k$}
\AxiomC{$\varphi \sststile{}{V} \psi_i[t_1/x_1, \ldots t_k/x_k]$, $1 \leq i \leq n$}
\RightLabel{\axlabel{na}}
\BinaryInfC{$\varphi \sststile{}{V} \chi_j[t_1/x_1, \ldots t_k/x_k]$}
\DisplayProof
\end{center}

\begin{prop}
Секвенция $\varphi \sststile{}{V} \psi_1 \land \ldots \land \psi_n$ выводима из правил \axref{b1}-\axref{b6}, \axref{a1}-\axref{a3} тогда и только тогда, когда
секвенции $\varphi \sststile{}{V} \psi_1$, \ldots $\varphi \sststile{}{V} \psi_n$ выводимы из правил естественного вывода.
\end{prop}
\begin{proof}
Легко доказать ``только тогда'' направление.
Наоборот, правила \axref{b1}, \axref{b4} и \axref{b5} следуют из \axref{nh},
правила \axref{b3} и \axref{b6} верны тривиально,
правило \axref{a1} следует из \axref{nv},
правило \axref{a2} следует из \axref{nl} и \axref{nh},
и аксиомы выводимы по \axref{na}.

Чтобы доказать правило \axref{b2}, нам достаточно показать, что если секвенции $\varphi \sststile{}{V} \psi_1$, \ldots $\varphi \sststile{}{V} \psi_n$
и $\psi_1 \land \ldots \land \psi_n \sststile{}{V} \chi$ выводимы в естественном выводе, то $\varphi \sststile{}{V} \chi$ также выводима.
Мы можем сконструировать дерево вывода для этой секвенции как дерево вывода для $\psi_1 \land \ldots \land \psi_n \sststile{}{V} \chi$,
в котором левая сторона каждой секвенции заменена на $\varphi$ и правила \axref{nh} заменены на деревья вывода для $\varphi \sststile{}{V} \psi_i$.

Чтобы доказать правило \axref{a3}, рассмотрим дерево вывода для секвенции $\varphi \sststile{}{V} \psi$.
Чтобы сконструировать дерево вывода для $\varphi[t/x] \sststile{}{V,V'} \psi[t/x]$, нам достаточно применить подстановку $t/x$ к каждой секвенции в этом дереве вывода.
Единственное правило, которое не замкнуто относительно подстановки, -- это \axref{nv}.
По предположению $x \in \FV(\varphi)$.
Это влечет, что $\varphi[t/x] \sststile{}{V,V'} t\!\downarrow$ выводима из \axref{np}, \axref{nf} и следующих правил:
\begin{center}
\AxiomC{$\varphi \sststile{}{V} t_1 = t_2$}
\RightLabel{\axlabel{ne1}}
\UnaryInfC{$\varphi \sststile{}{V} t_1\!\downarrow$}
\DisplayProof
\qquad
\AxiomC{$\varphi \sststile{}{V} t_1 = t_2$}
\RightLabel{\axlabel{ne2}}
\UnaryInfC{$\varphi \sststile{}{V} t_2\!\downarrow$}
\DisplayProof
\end{center}
Правило \axref{ne2} следует из \axref{nl} если мы возьмем $\psi(x)$ равным $x = b$.
Правило \axref{ne1} следует из \axref{ne2} и \axref{ns}.
\end{proof}

Теперь мы можем привести правила вывода для инфинитарных теорий.
Большинство правил вывода остаются прежними с очевидными поправками:

\begin{center}
\AxiomC{}
\RightLabel{\axlabel{iv}}
\UnaryInfC{$\varphi \sststile{}{V} x\!\downarrow$}
\DisplayProof
\qquad
\AxiomC{$\varphi \sststile{}{V} t_1 = t_2$}
\RightLabel{\axlabel{is}}
\UnaryInfC{$\varphi \sststile{}{V} t_2 = t_1$}
\DisplayProof
\qquad
\AxiomC{}
\RightLabel{\axlabel{ih}}
\UnaryInfC{$\bigwedge_{i \in I} \varphi_i \sststile{}{V} \varphi_i$}
\DisplayProof
\end{center}
\medskip

\begin{center}
\AxiomC{$\varphi \sststile{}{V} R(\{ t_i \}_i)$}
\RightLabel{\axlabel{ip}}
\UnaryInfC{$\varphi \sststile{}{V} t_i\!\downarrow$}
\DisplayProof
\qquad
\AxiomC{$\varphi \sststile{}{V} \sigma(\{ t_i \}_i)\!\downarrow$}
\RightLabel{\axlabel{if}}
\UnaryInfC{$\varphi \sststile{}{V} t_i\!\downarrow$}
\DisplayProof
\end{center}
где $R$ -- это предикатный символ теории, а $\sigma$ -- функциональный символ.

Для каждой аксиомы $\bigwedge_{i \in I} \psi_i \sststile{}{V} \bigwedge_{j \in J} \chi_j$
и всех термов $\{ t_i \}_{i \in V}$ мы добавляем следующее правило для всех $j \in J$:
\begin{center}
\AxiomC{$\varphi \sststile{}{V'} t_x\!\downarrow$, $x \in V$}
\AxiomC{$\varphi \sststile{}{V'} \psi_i[t_x/x,]$, $i \in I$}
\RightLabel{\axlabel{ia}}
\BinaryInfC{$\varphi \sststile{}{V'} \chi_j[t_x/x]$}
\DisplayProof
\end{center}

Правило \axref{nl} удобно разбить на несколько правил:

\begin{center}
\AxiomC{$\varphi \sststile{}{V} t_1 = t_2$}
\AxiomC{$\varphi \sststile{}{V} t_2 = t_3$}
\RightLabel{\axlabel{it}}
\BinaryInfC{$\varphi \sststile{}{V} t_1 = t_3$}
\DisplayProof
\end{center}
\medskip

\begin{center}
\AxiomC{$\varphi \sststile{}{V} a_i = b_i$, $i \in I$}
\AxiomC{$\varphi \sststile{}{V} \sigma(\{ a_i \}_i)\!\downarrow$}
\RightLabel{\axlabel{ic}}
\BinaryInfC{$\varphi \sststile{}{V} \sigma(\{ a_i \}_i) = \sigma(\{ b_i \}_i)$}
\DisplayProof
\end{center}
\medskip

\begin{center}
\AxiomC{$\varphi \sststile{}{V} a_i = b_i$, $i \in I$}
\AxiomC{$\varphi \sststile{}{V} R(\{ a_i \}_i)$}
\RightLabel{\axlabel{ii}}
\BinaryInfC{$\varphi \sststile{}{V} R(\{ b_i \}_i)$}
\DisplayProof
\end{center}
\medskip

Легко видеть, что в финитарном случае эти правила эквивалентны \axref{nl}.

\subsection{Синтаксическая категория теории}

В этом подразделе для каждой $\lambda$-достижимой теории $T$ мы определим конечно полную категорию $\cat{C}_T$, которая называется \emph{синтаксической категорией} этой теории.

\emph{Производный сорт} $\lambda$-достижимой теории -- это пара, состоящая из $\mathcal{S}$-множества $V$ такого, что $|\coprod_{s \in \mathcal{S}} X_s| < \lambda$, и формулы $\varphi$ с переменными в $V$.
Мы будем записывать сорт, соответствующий такой паре как $\{ f : \prod V \mid \varphi(f) \}$ или более коротко как $\{ f \mid \varphi(f) \}$.

Каждой $\lambda$-достижимой теории $T$ можно сопоставить ее \emph{синтаксическую категорию} $\cat{C}_T$.
Объекты этой категории -- это производные сорта $T$.
Морфизм между объектами $\{ f : \prod V \mid \varphi(f) \}$ и $\{ g : \prod W \mid \psi(g) \}$ -- это класс эквивалентности коллекций термов $\{ t_{s,i} \}_{s \in \mathcal{S}, i \in W_s}$ таких,
что $t_{s,i}$ имеет сорт $s$, $\FV(t_{s,i}) \subseteq V$ и $\varphi \sststile{T}{V} \psi[t_{s,i}/f_{s,i}] \land \prod_{s \in \mathcal{S}, i \in W_s} t_{s,i}\!\downarrow$.
Две коллекции $\{ t_{s,i} \}_{s,i}$ и $\{ t'_{s,i} \}_{s,i}$ эквивалентны, если $\varphi \sststile{T}{V} \prod_{s \in \mathcal{S}, i \in W_s} t_{s,i} = t_{s,i}'$.

Тождественный морфизм на объекте $\{ p \mid \varphi(p) \}$ -- это коллекция $\{ p_{s,i} \}_{s,i}$.
Композиция морфизмов $\{ t_{s,i} \}_{s,i} : \{ p \mid \varphi(p) \} \to \{ q \mid \psi(q) \}$
и $\{ t'_{s,j} \}_{s,j} : \{ q \mid \psi(q) \} \to \{ r \mid \chi(r) \}$ -- это коллекция $\{ t'_{s,j}[\rho] \}_{s,j}$, где $\rho(i) = t_{s,i}$.
Легко видеть, что это определение корректно и действительно задает категорию.

\begin{remark}
Можно модифицировать определение $\cat{C}_T$, объединив объекты, эквивалентные с точностью до переименования переменных и эквивалентности формул в теории $T$.
Легко видеть, что такая категория будет эквивалентна $\cat{C}_T$.
Иногда мы будем отождествлять эквивалентные объекты $\cat{C}_T$.
\end{remark}

\begin{remark}
Определение категории $\cat{C}_T$ зависит не только от $T$, но и от кардинала $\lambda$.
Например, верно, что в $\cat{C}_T$ существуют все $\lambda$-малые пределы.
\end{remark}

Для финитарных теорий мы можем упростить нотацию.
Производный сорт -- это пара из упорядоченного множества $V = \{ x_1 : s_1, \ldots x_n : s_n \}$ и формулы $\varphi$ такой, что $\FV(\varphi) \subset V$.
Мы будем записывать такой сорт как $\{ \overline{x} : \overline{s} \mid \varphi \}$ или более коротко как $\{ \overline{x} \mid \varphi \}$.
Морфизм между объектами $\{ (x_1, \ldots x_n) \mid \varphi \}$ и $\{ (y_1, \ldots y_k) : \overline{s} \mid \psi \}$ -- это класс эквивалентности списков термов $(t_1, \ldots t_k)$ таких,
что $t_i$ имеет сорт $s_i$, $\FV(t_i) \subseteq \{ x_1, \ldots x_n \}$ и $\varphi \sststile{T}{\overline{x}} \psi[\overline{t}/\overline{y}] \land t_1\!\downarrow \land \ldots \land t_k\!\downarrow$.
Два списка $(t_1, \ldots t_k)$ и $(t_1', \ldots t_k')$ эквивалентны, если $\varphi \sststile{T}{\overline{x}} t_1 = t_1' \land \ldots \land t_k = t_k'$.
Тождественный морфизм на объекте $\{ \overline{x} \mid \varphi \}$ -- это список $\overline{x}$.
Композиция морфизмов $(t_1, \ldots t_k) : \{ \overline{x} \mid \varphi \} \to \{ \overline{y} \mid \psi \}$
и $(t_1', \ldots t_m') : \{ \overline{y} \mid \psi \} \to \{ \overline{z} \mid \chi \}$ -- это список $(t_1'[\rho], \ldots t_m'[\rho])$, где $\rho(y_i) = t_i$.

В \cite{PHL} приводится другое определение синтаксической категории для финитарных теорий.
Она определяется как начальный объект теории $\mathrm{Cart} \overline{\omega} T$.
Эта теория является расширением теории, описанной в \pexample{fc-cats}.
Для каждого сорта $s$ теории $T$ мы добавляем константу $\upgamma^s : \fs{ob}$ и аксиому $\sststile{}{} \upgamma^s\!\downarrow$.
Для каждого предикатного символа $R : s_1 \times \ldots \times s_n$ теории $T$ мы добавляем константу $\upgamma^R : \fs{hom}$
и аксиому $\sststile{}{} c(\upgamma^R) = \upgamma^{s_1} \times \ldots \times \upgamma^{s_n} \land \fs{Mon}(\upgamma^R)$,
где $X \times Y$ -- декартово произведение объектов, которое определяется очевидным образом в теории конечно полных категорий, и $\fs{Mon}(f)$ -- предикат,
утверждающий, что $f$ является мономорфизмом (это верно тогда и только тогда, когда $\pi_1(f,f) = \pi_2(f,f)$).
Для каждого функционального символа $\sigma : s_1 \times \ldots \times s_n \to s$ теории $T$ мы добавляем константы $\upgamma^\sigma_d, \upgamma^\sigma_m : \fs{hom}$ и аксиому
$\sststile{}{} c(\upgamma^\sigma_m) = \upgamma^{s_1} \times \ldots \times \upgamma^{s_n} \land c(\upgamma^\sigma_d) = \upgamma^s \land d(\upgamma^\sigma_d) = d(\upgamma^\sigma_m) \land \fs{Mon}(\upgamma^\sigma_m)$.
Идея заключается в том, что $\upgamma^\sigma_m$ задает некоторый подобъект домена $\sigma$, а $\upgamma^\sigma_d$ является морфизмом из этого подобъекта в кодомен $\sigma$.
То есть такая пара морфизмов -- это в точности частичный морфизм из домена $\sigma$ в его кодомен.

В \cite[Section~8]{PHL} описана категориальная семантика частичных хорновских теорий.
В частности, она описывает интерпретацию термов и формул.
Если $V = \{ x_1 : s_1, \ldots x_n : s_n \}$ -- упорядоченное множество переменных, а $t$ -- терм сорта $s$ такой, что $\FV(t) \subseteq V$,
то категориальная семантика дает нам пару морфизмов $\upgamma^t_d$ и $\upgamma^t_m$, задающих частичный морфизм из $\upgamma^{s_1} \times \ldots \times \upgamma^{s_n} \to \upgamma^s$.
Если $\varphi$ -- формула такая, что $\FV(\varphi) \subseteq V$, то по категориальной семантике мы получаем мономорфизм $\gamma^\varphi$, задающий подобъект $\upgamma^{s_1} \times \ldots \times \upgamma^{s_n}$.
Нам понадобятся следующие леммы, доказанные в \cite[Lemma~39]{PHL} и \cite[Lemma~40]{PHL}:

\begin{lem}[term-subst]
Если $V = \{ x_1 : s_1, \ldots x_n : s_n \}$ и $V' = \{ y_1 : s_1', \ldots y_k : s_k' \}$ -- два упорядоченных множества переменных, $t'$ -- терм такой, что $\FV(t') \subseteq V'$,
и $(t_1, \ldots t_k)$ -- список термов таких, что $\FV(t_i) \subseteq V$ и $t_i$ имеет сорт $s_i'$.
Тогда ограничение частичного морфизма $\upgamma^{t'[t_1/y_1, \ldots t_k/y_k] \land t_1 \land \ldots \land t_k}$ на пересечение подобъектов $\upgamma^{t_i}_m$ изоморфно композиции частичных морфизмов $\gamma^{\langle t_1, \ldots t_k \rangle}$ и $\gamma^{t'}$,
где $\gamma^{\langle t_1, \ldots t_k \rangle}$ -- это морфизм из пересечения подобъектов $\upgamma^{t_i}_m$, который задается как $\langle t_1, \ldots t_k \rangle$.
\end{lem}

\begin{lem}[form-subst]
Если $V = \{ x_1 : s_1, \ldots x_n : s_n \}$ и $V' = \{ y_1 : s_1', \ldots y_k : s_k' \}$ -- два упорядоченных множества переменных, $\psi$ -- формула такая, что $\FV(\psi) \subseteq V'$,
и $(t_1, \ldots t_k)$ -- список термов таких, что $\FV(t_i) \subseteq V$ и $t_i$ имеет сорт $s_i'$.
Тогда подобъект $\upgamma^{\psi[t_1/y_1, \ldots t_k/y_k] \land t_1 \land \ldots \land t_k}$ изоморфен $j \circ p$, где $j$ -- пересечение подобъектов $\upgamma^{t_i}_m$, а $p$ -- прообраз $\upgamma^\psi$ вдоль $\langle t_1, \ldots t_k \rangle$.
\end{lem}

Для каждой аксиомы $a$ вида $\varphi \sststile{}{V} \psi$ мы добавляем константу $\upgamma^a : \fs{hom}$ и аксиому $\sststile{}{} \upgamma^\varphi = \circ(\upgamma^\psi,\upgamma^a)$.
Это завершает определение теории $\mathrm{Cart} \overline{\omega} T$.
Начальную модель этой теории мы будем обозначать $\cat{C}_T'$.

\begin{prop}
Для любой финитарной теории $T$ категории $\cat{C}_T$ и $\cat{C}_T'$ эквивалентны.
\end{prop}
\begin{proof}
Во-первых, покажем, что $\cat{C}_T$ является моделью $\mathrm{Cart} \overline{\omega} T$.
Эта категория конечно полная.
Действительно, $\{ () \mid \top \}$ является терминальным объектом.
Если $(t_1, \ldots t_k) : \{ \overline{x} \mid \varphi \} \to \{ \overline{z} \mid \chi \}$ и $(t_1', \ldots t_k') : \{ \overline{y} \mid \psi \} \to \{ \overline{z} \mid \chi \}$ -- два морфизма в $\cat{C}_T$,
то их послойное произведение можно определить как $\{ \overline{x}, \overline{y} \mid \varphi \land \psi \land t_1 = t_1' \land \ldots \land t_k = t_k' \}$.
Легко видеть, что этот объект обладает необходимым универсальным свойством.

Константа $\upgamma^s$ интерпретируется как $\{ x : s \mid \top \}$.
Константа $\upgamma^R$ интерпретируется как $(x_1, \ldots x_n) : \{ (x_1, \ldots x_n) \mid R(x_1, \ldots x_n) \} \to \{ (x_1, \ldots x_n) \mid \top \}$.
Константа $\upgamma^\sigma_m$ интерпретируется как $(x_1, \ldots x_n) : \{ (x_1, \ldots x_n) \mid \sigma(x_1, \ldots x_n)\!\downarrow \} \to \{ (x_1, \ldots x_n) \mid \top \}$.
Константа $\upgamma^\sigma_d$ интерпретируется как $\sigma(x_1, \ldots x_n) : \{ (x_1, \ldots x_n) \mid \sigma(x_1, \ldots x_n)\!\downarrow \} \to \{ y \mid \top \}$.
Чтобы проинтерпретировать константу $\upgamma^a$, соответствующую аксиоме $\varphi \sststile{}{V} \psi$, достаточно показать, что подобъект $\upgamma^\varphi$ вкладывается в $\upgamma^\psi$.
Для этого достаточно показать, что для любой формулы $\varphi$ подобъекты $\overline{x} : \{ \overline{x} \mid \varphi \} \to \{ \overline{x} \mid \top \}$ и $\upgamma^\varphi$ эквивалентны.
Легко индукцией по структуре терма $t$ показать, что подобъекты $\upgamma^t_m$ и $\overline{x} : \{ \overline{x} \mid t\!\downarrow \} \to \{ \overline{x} \mid \top \}$ эквивалентны
и $\upgamma^t_d$ соответствует морфизму $t : \{ \overline{x} \mid t\!\downarrow \} \to \{ y \mid \top \}$.
Используя этот факт, легко показать необходимое свойство формул.

Таким образом, $\cat{C}_T$ действительно является моделью $\mathrm{Cart} \overline{\omega} T$.
Следовательно существует уникальный функтор $F : \cat{C}_T' \to \cat{C}_T$, являющийся морфизмом моделей.
Постороим функтор $G$ в обратную сторону.
Объект $\{ \overline{x} \mid \varphi \}$ отображается в $d(\gamma^\varphi)$.
Если $(t_1, \ldots t_k) : \{ \overline{x} \mid \varphi \} \to \{ \overline{y} \mid \psi \}$ -- морфизм, то у нас есть стрелка $d(\upgamma^\varphi) \to d(\psi[t_1/y_1, \ldots t_k/y_k] \land t_1\!\downarrow \land \ldots \land t_k\!\downarrow)$.
По \dlem{form-subst} у нас есть стрелка $d(\psi[t_1/y_1, \ldots t_k/y_k] \land t_1\!\downarrow \land \ldots \land t_k\!\downarrow) \to d(\psi)$.
Мы определяем $G(t_1, \ldots t_k)$ как композицию этих двух стрелок.
Легко видеть, что $G$ сохраняет тождественные морфизмы.
\Rlem{term-subst} влечет, что $G$ сохраняет композицию.

Мы не можем воспользоваться универсальным свойством $\cat{C}_T'$, чтобы доказать, что $G \circ F = \fs{Id}$, так как $G$ не является морфизмом моделей $\mathrm{Cart} \overline{\omega} T$.
Вместо этого мы построим естественный изоморфизм между этими функторами.
Для этого мы определим частичную функцию $\alpha$ на замкнутых термах $t$ теории $\mathrm{Cart} \overline{\omega} T$ таких, что $\sststile{T}{} t\!\downarrow$.
Эта функция будет удовлетворять следующим условиям:
\begin{itemize}
\item Если $t$ имеет сорт $\fs{ob}$ и определен, то $\alpha_t$ -- это морфизм $GF(t) \to t$.
\item Если $t$ имеет сорт $\fs{hom}$ и определен, то $\alpha_t$ -- это пара морфизмов $\alpha_t^d : GF(d(t)) \to d(t)$ и $\alpha_t^c : GF(c(t)) \to c(t)$.
\end{itemize}
Функция $\alpha_t$ определяется рекурсией по $t$:
\begin{itemize}
\item $\alpha_{d(f)} = \alpha^d_f$ и $\alpha_{c(f)} = \alpha^c_f$.
\item $\alpha^d_{\mathrm{id}(x)} = \alpha^c_{\mathrm{id}(x)} = \alpha_x$.
\item $\alpha^d_{\circ(g,f)}$ и $\alpha^c_{\circ(g,f)}$ определены если $\sststile{\mathrm{Cart} \overline{\omega} T}{} \alpha^c_f = \alpha^d_g$.
В этом случае $\alpha^d_{\circ(g,f)} = \alpha^d_f$ и $\alpha^c_{\circ(g,f)} = \alpha^c_g$.
\item $\alpha^d_{!(x)} = \alpha_x$ и $\alpha^c_{!(x)} = \alpha_1 =\ !(GF(1))$.
\item $\alpha_{\pi_i(f_1,f_2)}$ определено если $\sststile{\mathrm{Cart} \overline{\omega} T}{} \alpha^c_{f_1} = \alpha^c_{f_2}$.
В этом случае $\alpha^d_{\pi_i(f_1,f_2)} = \fs{pair}(\circ(\alpha^d_{f_1},GF(\pi_1(f_1,f_2))),\circ(\alpha^d_{f_2},GF(\pi_2(f_1,f_2))),f_1,f_2)$ и $\alpha^c_{\pi_i(f_1,f_2)} = \alpha^d_{f_i}$.
\item $\alpha_{\fs{pair}(a,b,f,g)}$ определено если $\sststile{\mathrm{Cart} \overline{\omega} T}{} \alpha^d_a = \alpha^d_b \land \alpha^c_a = \alpha^d_f \land \alpha^c_b = \alpha^d_g \land \alpha^c_f = \alpha^c_g$.
В этом случае $\alpha^d_{\fs{pair}(a,b,f,g)} = \alpha^d_a$ и $\alpha^c_{\fs{pair}(a,b,f,g)} = \alpha^d_{\pi_1(f_1,f_2)}$.
\item Так как $GF(\upgamma^s) = \upgamma^s$, то мы можем определить $\alpha_{\upgamma^s}$ как $\mathrm{id}(\upgamma^s)$.
\item $GF(d(\upgamma^R)) = d(\upgamma^{R(x_1, \ldots x_n)})$. Этот объект изоморфен $d(\upgamma^R)$, так что мы можем определить $\alpha^d_{\upgamma^R}$ как этот изоморфизм.
Кроме того, мы определяем $\alpha^c_{\upgamma^R}$ как $\alpha_{\upgamma^{s_1} \times \ldots \times \upgamma^{s_n}}$.
\item Мы определяем $\alpha^d_{\upgamma^\sigma_m}$ и $\alpha^d_{\upgamma^\sigma_d}$ как изоморфизм между $d(\upgamma^{\sigma(x_1, \ldots x_n) \downarrow})$ и $d(\upgamma^\sigma)$.
Мы определяем $\alpha^c_{\upgamma^\sigma_m}$ как $\alpha_{\upgamma^{s_1} \times \ldots \times \upgamma^{s_n}}$ и $\alpha^c_{\upgamma^\sigma_d}$ как $\alpha_{\upgamma^s}$.
\item $\alpha^d_{\upgamma^a} = \alpha^d_{\upgamma^\varphi}$ и $\alpha^c_{\upgamma^a} = \alpha^d_{\upgamma^\psi}$.
\end{itemize}
Индукцией по естественному выводу секвенции $\sststile{}{} t = s$ легко показать, что $\alpha_t$ и $\alpha_s$ определены, если $t$ и $s$ имеют сорт $\fs{ob}$, то секвенция $\sststile{}{} \alpha_t = \alpha_s$ выводима,
и если $t$ и $s$ имеют сорт $\fs{hom}$, то секвенция $\sststile{}{} \alpha^d_t = \alpha^d_s \land \alpha^c_t = \alpha^c_s \land \circ(t,\alpha^d_t) = \circ(\alpha^c_t,GF(t))$ выводима.
Это влечет, что $\alpha$ задает естественное преобразование между функторами $G \circ F$ и $\fs{Id}$.
Легко видеть, что это преобразование является изоморфизмом.

Нам осталось доказать, что $F \circ G$ изоморфен $\fs{Id}$.
Пусть $X = \{ \overline{x} : \overline{s} \mid \varphi \}$ -- объект $\cat{C}_T$.
Тогда $FG(X) = F(d(\upgamma_\varphi))$.
Мы уже видели в начале доказательства, что этот объект изоморфен исходному объекту $X$.
Легко видеть, что этот изоморфизм естественен.
\end{proof}

\subsection{Категория теорий}

Пусть $T$ и $T'$ -- две $\lambda$-достижимые теории.
Тогда мы можем определить категорию морфизмов между ними как категорию функторов между их классифицирующими категориями.

TODO

\subsection{Разделение аксиом}

Многие теории содержат аксиомы вида
\[ \sigma(x_1, \ldots x_k)\!\downarrow\ \sststile{}{x_1, \ldots x_k} \varphi \]
и зачастую эти аксиомы можно убрать из теории, не сильно ее меняя.
Конкретно, мы докажем, что если отсальные аксиомы удовлетворяют простому естественному условию, то теоремы определенного вида выводимы без этих аксиом.

Мы будем говорить, что аксиомы теории \emph{разделены}, если множество аксиом состоит из двух подмножеств $\mathcal{A}_f$ и $\mathcal{A}_e$, удовлетворяющих следующим условиям:
\begin{enumerate}
\item \label{it:sep-f} Секвенции в $\mathcal{A}_f$ имеют вид $\sigma(x_1, \ldots x_k)\!\downarrow\ \sststile{}{x_1, \ldots x_k} \chi$.
\item \label{it:sep-e} Если секвенция вида $\sststile{}{} \varphi$ выводима в $\mathcal{A}_f \cup \mathcal{A}_e$, то она выводима и в $\mathcal{A}_e$.
\end{enumerate}

Разумеется, мы всегда можем взять пустое множество в качестве $\mathcal{A}_f$, но, чем оно больше, тем лучше.
Зачастую в качестве $\mathcal{A}_f$ можно взять максимальное возможное множество.
Например, это верно для теорий категорий и конечно полных категорий.
Это следует из следующей леммы.
Пусть множество аксиом некоторой теории состоит из двух подмножеств $\mathcal{A}_f$ и $\mathcal{A}_e$.
Мы будем говорить, что подтермы некоторой формулы $\varphi$ определены в $\mathcal{A}_e$, если для любой аксиомы $\sigma(x_1, \ldots x_k)\!\downarrow\ \sststile{}{x_1, \ldots x_k} \chi$ в $\mathcal{A}_f$
и любого подтерма вида $\sigma(t_1, \ldots t_k)$ формулы $\varphi$ секвенция $\sststile{}{} \chi[t_1/x_1, \ldots t_k/x_k]$ выводима из $\mathcal{A}_e$.

\begin{lem}[der-separated-closed]
Пусть $\mathcal{A}_f$ -- подмножество множества аксиом некоторой теории, удовлетворяющее условию~\eqref{it:sep-f}.
Тогда аксиомы этой теории разделены тогда и только тогда, когда выполнено следующее условие.
Для каждой аксиомы $\varphi \sststile{}{V} \psi$ и каждой замкнутой подстановки $\rho$, если секвенция $\sststile{}{} \varphi[\rho] \land \rho\!\downarrow$ выводима из $\mathcal{A}_e$,
и подтермы $\varphi[\rho]$ определены, то подтермы $\psi[\rho]$ тоже определены.
\end{lem}
\begin{proof}
Допустим аксиомы разделены.
Тогда, если секвенция $\sststile{}{} \varphi[\rho] \land \rho\!\downarrow$ выводима из $\mathcal{A}_e$, то секвенция $\sststile{}{} \psi[\rho]$ выводима из $\mathcal{A}_e \cup \mathcal{A}_f$.
Отсюда следует, что подтермы $\psi$ определены в $\mathcal{A}_f \cup \mathcal{A}_e$ относительно $\rho$.
По предположению они определены и в $\mathcal{A}_e$.

Теперь предположим, что условие леммы выполнено.
Сначала мы докажем, что для любой замкнутой формулы $\varphi$ такой, что секвенция $\sststile{}{} \varphi$ выводима в $\mathcal{A}_e$, ее подтермы определены в $\mathcal{A}_e$,
Мы докажем это индукцией по выводу $\sststile{}{} \varphi$ в системе естественного вывода.
Для большинства правил это очевидно.
Для правила \axref{na} это верно по предположению.
Единственный нетривиальный случай -- это \axref{nl}:
\begin{center}
\AxiomC{$\sststile{}{} a = b$}
\AxiomC{$\sststile{}{} \psi[a/x]$}
\RightLabel{\axref{nl}}
\BinaryInfC{$\sststile{}{} \psi[b/x]$}
\DisplayProof
\end{center}
Если $\sigma(t_1, \ldots t_k)$ является подтермом $b$, то необходимое свойство следует из индукционной гипотезы для $\sststile{}{} a = b$.
Иначе $\sigma$ принадлежит $\psi$ и существуют термы $t_1'$, \ldots $t_k'$ и формула $\psi'$ такие, что $t_i = t_i'[b/x]$, $\psi = \psi'[\sigma(t_1', \ldots t_k')/y]$.
По индукционной гипотезе секвенция $\sststile{}{} \chi[t_1'[a/x]/x_1, \ldots t_k'[a/x]/x_k]$ выводима.
Так как секвенция $\sststile{}{} a = b$ выводима, это влечет, что секвенция $\sststile{}{} \chi[t_1/x_1, \ldots t_k/x_k]$ также выводима.

Теперь, если некоторая секвенция $\sststile{}{} \varphi$ выводима в $\mathcal{A}_f \cup \mathcal{A}_e$, то мы докажем индукцией по ее естественному выводу, что она выводится и в $\mathcal{A}_e$.
Единственный нетривиальный случай -- это правило \axref{na} для аксиом из $\mathcal{A}_f$, которое следует из только что доказанного факта.
\end{proof}

\begin{lem}[der-separated-closed]
Пусть $T$ -- теория с разделенными аксиомами.
Если секвенция $\sststile{}{} \psi$ выводима в $T$, то она выводима и из аксиом $\mathcal{A}_e$.
\end{lem}
\begin{proof}
Во-первых, докажем следующий факт.
Если секвенция $\sststile{}{} \psi$ выводима из $\mathcal{A}_e$, то для любой аксиомы $\sigma(x_1, \ldots x_k)\!\downarrow\ \sststile{}{x_1, \ldots x_k} \chi$ из $\mathcal{A}_f$ и любого подтерма $\sigma(t_1, \ldots t_k)$ формулы $\psi$
секвенция $\sststile{}{} \chi[t_1/x_1, \ldots t_k/x_k]$ выводима из $\mathcal{A}_e$.
Мы докажем это индукцией по выводу $\sststile{}{} \psi$ в системе естественного вывода.
Для большинства правил это очевидно.
Для правила \axref{na} это следует из \eqref{it:sep-e}.
Единственный нетривиальный случай -- это \axref{nl}:
\begin{center}
\AxiomC{$\sststile{}{} a = b$}
\AxiomC{$\sststile{}{} \psi[a/x]$}
\RightLabel{\axref{nl}}
\BinaryInfC{$\sststile{}{} \psi[b/x]$}
\DisplayProof
\end{center}
Если $\sigma(t_1, \ldots t_k)$ является подтермом $b$, то необходимое свойство следует из индукционной гипотезы для $\sststile{}{} a = b$.
Иначе $\sigma$ принадлежит $\psi$ и существуют термы $t_1'$, \ldots $t_k'$ и формула $\psi'$ такие, что $t_i = t_i'[b/x]$, $\psi = \psi'[\sigma(t_1', \ldots t_k')/y]$.
По индукционной гипотезе секвенция $\sststile{}{} \chi[t_1'[a/x]/x_1, \ldots t_k'[a/x]/x_k]$ выводима.
Так как секвенция $\sststile{}{} a = b$ выводима, это влечет, что секвенция $\sststile{}{} \chi[t_1/x_1, \ldots t_k/x_k]$ также выводима.

Теперь мы можем доказать утверждение леммы индукцией по выводу $\sststile{}{} \psi$.
Единственный нетривиальный случай -- это правило вывода для аксиом из $\mathcal{A}_f$:
\smallskip
\begin{center}
\AxiomC{$\sststile{}{} t_i\!\downarrow$, $1 \leq i \leq k$}
\AxiomC{$\sststile{}{} \sigma(t_1, \ldots t_k)\!\downarrow$}
\RightLabel{\axlabel{na}}
\BinaryInfC{$\sststile{}{} \chi[t_1/x_1, \ldots t_k/x_k]$}
\DisplayProof
\end{center}
По индукционной гипотезе секвенция $\sststile{}{} \sigma(t_1, \ldots t_k)\!\downarrow$ выводима из $\mathcal{A}_e$.
Только что доказанный факт влечет, что секвенция $\sststile{}{} \chi[t_1/x_1, \ldots t_k/x_k]$ также выводима из $\mathcal{A}_e$.
\end{proof}

Условие в определении разделения аксиом можно усилить.
Для этого нам понадобится ввести новое определение.
Пусть $\varphi_1 \land \ldots \land \varphi_n \sststile{}{V} \psi$ -- секвенция в некоторой теории $T$.
Мы будем говорить, что \emph{подтермы посылки этой секвенции определены} в $T$,
если секвенция $\varphi_1 \land \ldots \land \varphi_i \sststile{}{V} t\!\downarrow$ выводима в $T$ для любого подтерма $t$ формулы $\varphi_{i+1}$.

\begin{prop}[der-separated]
Пусть $T$ -- теория с разделенными аксиомами.
Если секвенция $\varphi \sststile{}{V} \psi$ выводима в $T$ и подтермы ее посылки определены в $T$, то она выводима из $\mathcal{A}_e$.
\end{prop}
\begin{proof}
Пусть $V = \{ x_1, \ldots x_m \}$ и $\varphi = \varphi_1 \land \ldots \land \varphi_n$.
Пусть $T_j = T \cup \{ \sststile{}{} c_i\!\downarrow\ \mid 1 \leq i \leq m \} \cup \{ \sststile{}{} \varphi_i[c_1/x_1, \ldots c_m/x_m]\ \mid 1 \leq i < j \}$.
Мы докажем индукцией по $j$, что аксиомы $T_j$ разделены.
Для $T_0$ это очевидно.
Докажем, что если аксиомы $T_j$ разделены, то это верно и для $T_{j+1}$.
Единственная новая аксиома $T_{j+1}$ -- это $\sststile{}{} \varphi_j[c_1/x_1, \ldots c_m/x_m]$.
Пусть $\sigma(y_1, \ldots y_k) \sststile{}{y_1, \ldots y_k} \chi$ -- аксиома из $\mathcal{A}_f$, и $\sigma(t_1, \ldots t_k)$ -- подтерм $\varphi_j[c_1/x_1, \ldots c_m/x_m]$.
Так как подтермы $\varphi$ определены, то по \dlem{mcf} секвенция $\sststile{}{} \sigma(t_1, \ldots t_k)\!\downarrow$ выводима в $T_j$.
Так как аксиомы $T_j$ разделены, то секвенция $\sststile{}{} \chi[t_1/y_1, \ldots t_k/y_k]$ выводима в $T_j \setminus \mathcal{A}_f$.

Теперь мы можем доказать, что $\varphi \sststile{}{V} \psi$ выводима из $\mathcal{A}_e$.
По \dlem{mcf} для этого достаточно доказать, что секвенция $\sststile{}{} \psi[c_1/x_1, \ldots c_m/x_m]$ выводима в $T_{n+1} \setminus \mathcal{A}_f$.
По \dlem{mcf} она выводима в $T_{n+1}$, и по \dlem{der-separated-closed} она выводима и в $T_{n+1} \setminus \mathcal{A}_f$.
\end{proof}

\subsection{Конфлюэнтные теории}

Зачастую можно выбрать направление для аксиом, постулирующих равенства, так, чтобы получившееся отношение обладало свойством конфлюэнтности.
Для таких теорий можно доказать несколько полезных утверждений.
В этом разделе мы определим понятие конфлюэнтных теорий, обладающих этим свойством, и докажем их свойства.
Для этого нам понадобится определить несколько понятий из теории абстрактных систем редукций \cite{Terese,klop-trs,ohlebusch-advanced}:

\begin{enumerate}
\item \emph{Абстрактная система редукций} -- это множество $A$ вместе с бинарным отношением $\Rightarrow$ на нем.
Мы будем обозначать $\Rightarrow^*$ рефлексивное и транзитивное замыкание отношения $\Rightarrow$.
Если $\Rightarrow_1$ и $\Rightarrow_2$ -- два отношения, то мы будем писать $\Rightarrow_1 \Rightarrow_2$ для обозначения следующего отношения:
$t \Rightarrow_1 \Rightarrow_2 t'$ тогда и только тогда, когда существует терм $s$ такой, что $t \Rightarrow_1 s$ и $s \Rightarrow_2 t'$.
\item Элемент $a$ \emph{редуцируется} к элементу $a'$ если $a \Rightarrow^* a'$.
\emph{Последовательность редукций} -- это конечная или бесконечная последовательность элементов $a_i$ таких, что $a_0 \Rightarrow a_1 \Rightarrow a_2 \Rightarrow \ldots$.
\item Два элемента $a$ и $b$ \emph{соединимы} если существует элемент $c$ такой, что $a \Rightarrow^* c$ и $b \Rightarrow^* c$.
Мы будем также говорить, что $a$ и $b$ соединимы отношением $\Rightarrow$ если оно не ясно из контекста.
Элемент $a$ \emph{конфлюэнтен} если $a \Rightarrow^* b$ и $a \Rightarrow^* c$ влечет, что термы $b$ и $c$ соединимы.
Система \emph{конфлюэнтна} если все ее элементы конфлюэнтны.
Эквивалентно, система конфлюэнтны если любая пара ее элементов соединима.
\item Два элемента $a$ и $b$ называются \emph{$\Rightarrow$-эквивалентными}, если существует последовательность элементов $a_1$, \ldots $a_n$ такая, что $a = a_1$, $b = a_n$,
и для всех $1 \leq i < n$ либо $a_i \Rightarrow a_{i+1}$, либо $a_{i+1} \Rightarrow a_i$.
\item Элемент $a$ называется \emph{нормальной формой} если не существует элемента $a'$ такого, что $a \Rightarrow a'$.
Мы будем писать $a \Rightarrow^\nf b$ когда $a \Rightarrow^* b$ и $b$ является нормальной формой.
Мы будем говорить, что элемент $a$ \emph{имеет нормальную форму} (или, что оно \emph{слабо нормализуем}), если $a \Rightarrow^\nf b$ для некоторого $b$.
Система является \emph{слабо нормализующей} (WN), если все ее элементы имеют нормальную форму.
\item Элемент $a$ \emph{сильно нормалищуем}, если не существует бесконечной последовательности редукций, начинающейся в $a$.
Система является \emph{сильно нормалиющей} (SN), если все ее элементы сильно нормализуемы.
\item Подмножество $A'$ множества $A$ \emph{замкнуто} относительно $\Rightarrow$, если $a' \in A'$ и $a' \Rightarrow a$ влечет, что $a \in A'$.
\end{enumerate}

Лемма Ньюмана говорит, что для того, чтобы проверить, что сильно нормализующая система конфлюэнтна, достаточно проверить более слабое условие, которое называется \emph{локальной конфлюэнтностью}:

\begin{lem}[newman][Лемма Ньюмана]
Пусть $A$ -- сильно нормализующая абстрактная система редукций.
Допустим, что для любых $a,b,c \in A$ таких, что $a \Rightarrow b$ и $a \Rightarrow c$, существует $d \in A$ такой, что $b \Rightarrow^* d$ и $c \Rightarrow^* d$.
Тогда $A$ конфлюэнтна.
\end{lem}
\begin{proof}
Доказательство приведено, например, в \cite[Lemma~2.2.5]{ohlebusch-advanced}.
\end{proof}

\emph{Система переписывания термов} -- это бинарное множество $R$ на множестве термов некоторой теории, удовлетворяющее следующим условиям:
\begin{enumerate}
\item Если $R(t,s)$, то $\FV(s) \subseteq \FV(t)$.
\item Если $R(t,s)$, то $t$ не является переменной.
\end{enumerate}
Система переписываения термов $R$ называется \emph{лево-линейной}, если для всех термов $t$ и $s$ таких, что $R(t,s)$, каждая переменная встречается в $t$ не более одного раза.

Для каждой системы переписывания термов $R$ мы можем определить отношение $\Rightarrow_R$ на множестве термов следующим образом: если $R(t,s)$, то
\[ c[x \repl t[x_1 \repl t_1, \ldots x_k \repl t_k]] \Rightarrow_R c[x \repl s[x_1 \repl t_1, \ldots x_k \repl t_k]] \]
для всех $c$, $x_1$, \ldots $x_k$ и $t_1$, \ldots $t_k$.
Мы будем обозначать $\Term_T$ множество термов теории $T$.
Таким образом, у любой системы переписывания термов есть подлежащая абстрактная система редукций $(\Term_T,\Rightarrow_R)$.

Пусть $V$ -- множество переменных, а $\varphi$ -- формула такая, что $\FV(\varphi) \subseteq V$.
Мы будем говорить, что терм $t$ некоторой теории $T$ \emph{определен} по отношению к паре $(V,\varphi)$, если секвенция $\varphi \sststile{}{V} t\!\downarrow$ выводима в $T$.
Мы будем писать $\Term_{T,V,\varphi}^d$ для обозначения множества термов, определенных по отношению к $(V,\varphi)$.
Мы будем писать $\Term_T^d$ для обозначения множества $\Term_{T,\varnothing,\top}^d$.

\begin{defn}[directed]
Пусть $T$ -- теория с разделенными аксиомами.
\emph{Система редукций} на $T$ -- это абстрактная система редукций $\Rightarrow_T$ на $\Term^d_T$ такая, что следующие условие выполнены:
\begin{enumerate}
\item \label{it:dir-first} Для каждой пары термов $t$ и $s$ таких, что $t \Rightarrow_T s$, секвенция $\sststile{}{} t = s$ выводима.
\item \label{it:dir-second} Для каждой подстановки $\rho$, каждого терма $c$ и каждой аксиомы вида $\psi \sststile{}{V} t = s$ в $\mathcal{A}_e$ таких,
что секвенция $\sststile{}{} \psi[\rho] \land \rho\!\downarrow \land c[t[\rho]/x]\!\downarrow$ выводима,
термы $c[t[\rho]/x]$ и $c[s[\rho]/x]$ эквивалентны в системе $(\Term_T^d,\Rightarrow_T)$.
\end{enumerate}
\end{defn}

\begin{remark}[trs-theory]
Зачастую система редукций на теории $T$ определена как сужение $\Rightarrow_R$ на множество $\Term^d_T$ для некоторой системы переписывания термов $R$.
В этом случае условия \eqref{it:dir-first} и \eqref{it:dir-second} эквивалентны следующим условиям:
\begin{enumerate}
\item \label{it:trs-dir-first} Для каждой подстановки $\rho$ и каждой пары термов $t$ и $s$ таких, что $(t,s) \in R$, если $\sststile{T}{} t[\rho]\!\downarrow \land s[\rho]\!\downarrow$, то $\sststile{T}{} t[\rho] = s[\rho]$.
\item \label{it:trs-dir-second} Для каждой подстановки $\rho$ и каждой аксиомы вида $\psi \sststile{}{V} t = s$ в $\mathcal{A}_e$ таких,
что секвенция $\sststile{}{} \psi[\rho] \land \rho\!\downarrow$ выводима, термы $t[\rho]$ и $s[\rho]$ эквивалентны в системе $(\Term_T^d,\Rightarrow_T)$.
\end{enumerate}
\end{remark}

\begin{example}[dir-ax]
Пусть $\mathcal{A}_f$ -- множество аксиом в некоторой теории $T$, удовлетворяющее условию~\eqref{it:sep-f}.
Пусть $\mathcal{A}_d$ -- множество аксиом в $T$ вида $\varphi \sststile{}{x_1, \ldots x_k} \sigma(x_1, \ldots x_k)\!\downarrow$ такое,
что для любой такой аксиомы в $\mathcal{A}_d$ и любой аксиомы вида $\sigma(x_1, \ldots x_k)\!\downarrow\ \sststile{}{x_1, \ldots x_k} \chi$ верно, что $\varphi = \chi$.

Пусть $R$ -- система переписывания термов на $T$, сохраняющая сорта термов.
Пусть $\mathcal{A}_c$ -- множество аксиом вида $t\!\downarrow\ \sststile{}{\FV(t)} t = s$ для каждой пары $(t,s) \in R$.
Если аксиомы $\mathcal{A}_f \cup (\mathcal{A}_d \cup \mathcal{A}_c)$ разделены, то $R$ задает систему редукций на $\mathcal{A}_f \cup \mathcal{A}_d \cup \mathcal{A}_c$ как описано в \premark{trs-theory}.
\end{example}

\begin{example}[cat-red]
Теорию категорий можно представить в виде, описаном в \pexample{dir-ax}.
Аксиомы $\mathcal{A}_f \cup \mathcal{A}_d$ эквивалентны следующему набору аксиом:
\begin{align*}
& \sststile{}{f} d(f)\!\downarrow \land c(f)\!\downarrow \\
& \sststile{}{x} \fs{id}(x)\!\downarrow \\
c(f) = d(g) & \ssststile{}{f,g} \circ(g,f)\!\downarrow
\end{align*}
Отношение $R$ состоит из следующих пар:
\begin{align*}
d(\fs{id}(x)) & \Rightarrow_R x \\
c(\fs{id}(x)) & \Rightarrow_R x \\
d(\circ(g,f)) & \Rightarrow_R d(f) \\
c(\circ(g,f)) & \Rightarrow_R c(g) \\
\circ(\fs{id}(x),f) & \Rightarrow_R f \\
\circ(f,\fs{id}(x)) & \Rightarrow_R f \\
\circ(\circ(h,g),f) & \Rightarrow_R \circ(h,\circ(g,f))
\end{align*}
\end{example}

Теперь мы докажем техническую лемму, которая говорит, что секвенция $\varphi \sststile{}{V} t = s$ доказуема в теории $T$ тогда и только тогда,
когда термы $t$ и $s$ эквивалентны в отношении, которое пораждается правой стороной аксиом и равенствами в $\varphi$.

\begin{lem}[der-eq]
Секвенция $\varphi \sststile{}{V} t = s$ выводима в теории $T$ тогда и только тогда, когда существуют термы $t_1, \ldots t_n$ такие, что $t = t_1$, $s = t_n$ и для всех $1 \leq i < n$ верно, что $t_i = c[a/x]$ и $t_{i+1} = c[b/x]$
для некоторых термов $a$, $b$ и $c$ таких, что переменная $x$ встречается в $c$ ровно один раз, и одно из следующих условий выполнено:
\begin{enumerate}
\item Существует применение правила \axref{na}, в котором посылка выводима и заключение имеет вид либо $\varphi \sststile{}{V} a = b$, либо $\varphi \sststile{}{V} b = a$.
Кроме того, естественный вывод этого заключения является подвыводом вывода секвенции $\varphi \sststile{}{V} t = s$.
\item $\varphi = \varphi_1 \land \ldots \land \varphi_k$, и существует $j$ такой, что $\varphi_j$ равно либо $a = b$, либо $b = a$.
\end{enumerate}
\end{lem}
Кроме того, секвенции $\varphi \sststile{}{V} t_i\!\downarrow$ выводимы для всех $1 \leq i \leq n$.
\begin{proof}
Если такая последовательность термов существует, то легко показать, что секвенция $\varphi \sststile{}{V} t = s$ выводима по правилам естественного вывода.
Обратное утверждение мы докажем индукцией по естественному выводу секвенции $\varphi \sststile{}{V} t = s$.
Правила \axref{nv}, \axref{np} и \axref{nf} очевидны.
Правила \axref{nh} и \axref{na} следует из предположения.
В этом случае мы берем $t_1 = a = t$, $t_2 = b = s$ и $c = x$.
Теперь рассмотрим правило \axref{ns}.
Если $t_1$, \ldots $t_n$ -- последовательность для $\varphi \sststile{}{V} s = t$, то мы можем взять последовательность $t_n$, \ldots $t_1$ для $\varphi \sststile{}{V} t = s$.

Нам осталось рассмотреть правило \axref{nl}:
\begin{center}
\AxiomC{$\varphi \sststile{}{V} p = q$}
\AxiomC{$\varphi \sststile{}{V} t'[p/y] = s'[q/y]$}
\RightLabel{\axref{nl}}
\BinaryInfC{$\varphi \sststile{}{V} t'[q/y] = s'[q/y]$}
\DisplayProof
\end{center}
Мы можем предположить, что переменная $y$ встречается ровно один раз в формуле $t' = s'$, так как общий случай следует из этого частного.
Пусть $t_1$, \ldots $t_n$ -- последовательность термов для $\varphi \sststile{}{V} p = q$, и $s_1$, \ldots $s_m$ -- последовательность для $\varphi \sststile{}{V} t'[p/y] = s'[p/y]$.
Тогда $t'[t_n/y]$, \ldots $t'[t_1/y] = s_1$, \ldots $s_m = s'[t_1/y]$, \ldots $s'[t_n/y]$ -- последовательность для $\varphi \sststile{}{V} t'[q/y] = s'[q/y]$.

Теперь мы докажем, что секвенция $\varphi \sststile{}{V} t_i\!\downarrow$ выводима индукцией по $i$.
Это верно для $i = 1$ по предположению.
Препдоложим, что это верно для некоторого $i$.
Тогда $t_i = c[a/x]$, $t_{i+1} = c[b/x]$ и секвенция $\varphi \sststile{}{V} a = b$ выводима.
По правилу \axref{nl} секвенция $\varphi \sststile{}{V} t_{i+1}\!\downarrow$ также выводима.
\end{proof}

В следующем утверждении мы докажем основное свойство теорий с системами редукцией.

\begin{prop}[conf-main]
Пусть $T$ -- теория с системой редукций.
Секвенция $\sststile{}{} t = s$ выводим тогда и только тогда, когда термы $t$ и $s$ эквивалентны в системе $(\Term_T^d,\Rightarrow_T)$.
\end{prop}
\begin{proof}
Если $t$ и $s$ эквивалентны в $(\Term_T^d,\Rightarrow_T)$, то существует зигзаг $\Rightarrow_T$-редукций между ними.
Так как отношение $\sststile{T}{} - = -$ является отношением эквивалентности на множестве $\Term_T^d$, то мы можем предположить, что $t \Rightarrow_T s$.
Тогда условие~\eqref{it:dir-first} определения~\ndefn{directed} влечет, что $\sststile{T}{} t = s$.

Если $t$ и $s$ такие термы, что $\sststile{T}{} t = s$, то эта секвенция выводима из аксиом $\mathcal{A}_e$.
Тогда \rlem{der-eq} влечет, что существует последовательность $t_1$, \ldots $t_n$ элементов $\Term^d_T$ такая, что $t = t_1$, $s = t_n$ и для всех $1 \leq i < n$
существуют аксиома $\psi \sststile{}{V} a = b$ в $\mathcal{A}_e$, подстановка $\rho$ и терм $c$ такие,
что секвенция $\sststile{}{} \psi[\rho] \land \rho\!\downarrow$ выводима, $t_i = c[a[\rho]/y]$ и $t_{i+1} = c[b[\rho]/y]$ (или наоборот).
Условие~\eqref{it:dir-second} определения~\ndefn{directed} влечет, что $t_i$ и $t_{i+1}$ эквивалентны в системе $(\Term_T^d,\Rightarrow_T)$.
\end{proof}

\begin{cor}[conf-main]
Пусть $T$ -- теория с системой редукций.
Тогда система $(\Term_T^d,\Rightarrow_T)$ конфлюэнтна тогда и только тогда, когда любая пара термов $t$ и $s$ таких, что $\sststile{T}{} t = s$, соединима в этой системе.
\end{cor}

\begin{example}
Теория категорий конфлюэнтна.
Легко видеть, что она SN.
По \dlem{newman} достаточно проверить локальную конфлюэнтность, что легко сделать, перебрав варианты.
\end{example}

\bibliographystyle{amsplain}
\bibliography{ref}

\end{document}
