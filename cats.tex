\documentclass[reqno]{amsart}

\usepackage{amssymb}
\usepackage{hyperref}
\usepackage{mathtools}
\usepackage[all]{xy}
\usepackage{verbatim}
\usepackage{ifthen}
\usepackage{xargs}
\usepackage{bussproofs}
\usepackage{turnstile}
\usepackage{etex}

\hypersetup{colorlinks=true,linkcolor=blue}

\newcommand{\axlabel}[1]{(#1) \phantomsection \label{ax:#1}}
\newcommand{\axtag}[1]{\label{ax:#1} \tag{#1}}
\newcommand{\axref}[1]{(\hyperref[ax:#1]{#1})}

\newcommand{\newref}[4][]{
\ifthenelse{\equal{#1}{}}{\newtheorem{h#2}[hthm]{#4}}{\newtheorem{h#2}{#4}[#1]}
\expandafter\newcommand\csname r#2\endcsname[1]{#3~\ref{#2:##1}}
\expandafter\newcommand\csname R#2\endcsname[1]{#4~\ref{#2:##1}}
\expandafter\newcommand\csname n#2\endcsname[1]{\ref{#2:##1}}
\newenvironmentx{#2}[2][1=,2=]{
\ifthenelse{\equal{##2}{}}{\begin{h#2}}{\begin{h#2}[##2]}
\ifthenelse{\equal{##1}{}}{}{\label{#2:##1}}
}{\end{h#2}}
}

\newref[section]{thm}{Theorem}{Theorem}
\newref{lem}{Lemma}{Lemma}
\newref{prop}{Proposition}{Proposition}
\newref{cor}{Corollary}{Corollary}
\newref{cond}{Condition}{Condition}

\theoremstyle{definition}
\newref{defn}{Definition}{Definition}
\newref{example}{Example}{Example}

\theoremstyle{remark}
\newref{remark}{Remark}{Remark}

\newcommand{\fs}[1]{\mathrm{#1}}
\newcommand{\cat}[1]{\mathbf{#1}}
\newcommand{\scat}[1]{\mathcal{#1}}
\newcommand{\Hom}{\fs{Hom}}
\renewcommand{\hom}{\fs{hom}}
\newcommand{\id}{\fs{id}}
\newcommand{\Id}{\fs{Id}}
\newcommand{\Set}{\cat{Set}}
\newcommand{\uSet}{\fs{Set}}
\newcommand{\uType}{\fs{Type}}
\newcommand{\ob}[1]{#1_0}
\newcommand{\fob}[1]{#1_0}
\newcommand{\Prop}{\fs{Prop}}
\newcommand{\El}{\fs{El}}

\numberwithin{figure}{section}

\newcommand{\pb}[1][dr]{\save*!/#1-1.2pc/#1:(-1,1)@^{|-}\restore}
\newcommand{\po}[1][dr]{\save*!/#1+1.2pc/#1:(1,-1)@^{|-}\restore}

\begin{document}

\title{Cats}

\author{Valery Isaev}

\begin{abstract}
% TODO
\end{abstract}

\maketitle

\section{Introduction}

% TODO

\section{Basic category theory}

In this section, we recall the definition of categories and basic constructions on them in homotopy type theory.
We also prove a few simple standard propositions.

\subsection{Definitions}

The purpose of this subsection is mostly to recall definitions from \cite{univalent-cats} and to fix the notation.

A \emph{precategory} $\scat{C}$ consists of a type of objects $\ob{\scat{C}}$ (which we often denote simply by $\scat{C}$), a set of maps $\hom_\scat{C}(x,y)$ for every pair of objects $x,y : \scat{C}$,
an associative composition function $\circ : \hom_\scat{C}(y,z) \to \hom_\scat{C}(x,y) \to \hom_\scat{C}(x,z)$, and an identity morphism $\id_x : \hom_\scat{C}(x,x)$ which is a two-sided identity for $\circ$.
We will omit the subscript in $\hom_\scat{C}(x,y)$ if the category is clear from the context.
A precategory is a \emph{category} if the obvious map from $\Id(x,y)$ to the type of isomorphisms between any pair of objects $x$ and $y$ is an equivalence.

A \emph{functor} between precategories $\scat{C}$ and $\scat{D}$ consists of a map $\fob{F} : \ob{\scat{C}} \to \ob{\scat{D}}$ (which we often denote simply by $F$)
together with a map $F : \hom_\scat{C}(x,y) \to \hom_\scat{D}(F(x),F(y))$ for every pair of objects $x$ and $y$ such that $F$ preserves identity morphisms and composition.
A natural transformation between functors $F,G : \scat{C} \to \scat{D}$ is a map $\alpha_x : \hom_\scat{D}(F(x),G(x))$ defined for every object $x : \scat{C}$ such that $G(f) \circ \alpha_x = \alpha_y \circ F(f)$ for every map $f : \hom_\scat{C}(x,y)$.
Functors and natural transformations form a precategory $\scat{D}^\scat{C}$ which is a category whenever $\scat{D}$ is.

A functor $F : \scat{C} \to \scat{D}$ is \emph{faithful} (resp., \emph{full}, \emph{fully faithfull}) if the map $F : \hom_\scat{C}(x,y) \to \hom_\scat{D}(F(x),F(y))$ is injective (resp., surjective, bijective) for all $x$ and $y$.
We will call fully faithfull functors \emph{embeddings}.
A functor $F : \scat{C} \to \scat{D}$ is essentially surjective on objects if, for every $y : \scat{D}$, there merely exists an object $x : \scat{C}$ such that $F(x)$ is isomorphic to $y$.
For every precategory $\scat{C}$, there exists a category $\widehat{\scat{C}}$ together with an essentially surjective embedding $i_\scat{C} : \scat{C} \to \widehat{\scat{C}}$.
The category $\widehat{\scat{C}}$ is called \emph{the Rezk completion of $\scat{C}$}.
It satisfies a universal property: for every category $\scat{D}$ and every functor $F : \scat{C} \to \scat{D}$, there is a unique functor $\widehat{F} : \widehat{\scat{C}} \to \scat{D}$ such that $\widehat{F} \circ i_\scat{C} = F$.

Let $\uType$ be a universe of types and let $\uSet$ be its subuniverse of sets.
Then we have the category $\Set$ whose types are elements of $\uSet$ and maps are functions.
A precategory $\scat{C}$ is \emph{locally small} (with respect to $\uSet$) if the set of maps $\hom_\scat{C}(x,y)$ is equivalent to a set in $\uSet$ for all $x$ and $y$.
A precategory $\scat{C}$ is \emph{small} (with respect to $\uType$) if it is locally small and the type of objects of $\scat{C}$ is equivalent to a type in $\uType$.

The \emph{opposite} of a (pre)category $\scat{C}$ is the (pre)category $\scat{C}^\fs{op}$ with the same type of objects in which the set of maps is defined as follows: $\hom_{\scat{C}^\fs{op}}(x,y) = \hom_\scat{C}(y,x)$.
For every locally small precategory $\scat{C}$, there is an embedding $\cat{y} : \scat{C} \to \Set^{\scat{C}^\fs{op}}$ called \emph{Yoneda embedding}.
It is defined by the formula $\cat{y}(x) = \hom_\scat{C}(-,x)$.

\subsection{Full subcategories}

A \emph{full subprecategory} of a precategory $\scat{D}$ is a type over $\ob{\scat{D}}$.
A \emph{full subcategory} of a category $\scat{D}$ is a proposition over $\ob{\scat{D}}$.
A \emph{(pre)category embedded in $\scat{D}$} is a (pre)category $\scat{C}$ together with an embedding $\scat{C} \to \scat{D}$.
The following lemma shows that these notions are equivalent:

\begin{lem}
Full subprecategories of a precategory $\scat{D}$ are equivalent to precategories embedded in $\scat{D}$.
Full subcategories of a category of $\scat{D}$ are equivalent to categories $\scat{C}$ embedded in $\scat{D}$.
\end{lem}
\begin{proof}
Let $P$ be a type over $\ob{\scat{D}}$.
We define a precategory $\scat{C}_P$ and an embedding $F_P : \scat{C}_P \to \scat{D}$.
The type of objects of $\scat{C}_P$ is defined as $\Sigma_{x : \ob{\scat{D}}} P(x)$.
The set of maps $\hom_{\scat{C}_P}((x,p),(x',p'))$ is defined as $\hom_\scat{D}(x,x')$.
The composition and identity maps are defined in the same way as in $\scat{D}$.
The functor $F_P : \scat{C}_P \to \scat{D}$ is defined in the obvious way: $F_P(x,p) = x$ for an object $(x,p)$ and $F_P(f) = f$ for a map $f$.
Clearly, $F_P$ is an embedding.
Now, let $\scat{C}$ be a precategory and let $F : \scat{C} \to \scat{D}$ be an embedding.
We define a type $P_F(x)$ for every $x : \ob{\scat{D}}$ as follows: $P_F(x) = \Sigma_{y : \ob{\scat{C}}} \Id(F(y),x)$.

Let us prove that constructions that we described are mutually inverse.
Let $P$ be a type over $\ob{\scat{D}}$.
Then $P_{F_P}(x)$ is equivalent to $\Sigma_{x' : \ob{\scat{D}}} P(x') \times \Id(x',x)$ which is equivalent to $P(x)$.
Let $F : \scat{C} \to \scat{D}$ be an embedding.
We need to prove that $\scat{C}$ is isomorphic to $\scat{C}_{P_F}$.
The type of objects of the latter precategory is defined as $\Sigma_{x : \ob{\scat{D}}} \Sigma_{y : \ob{\scat{C}}} \Id(F(y),x)$ which is equivalent to $\ob{\scat{C}}$.
Thus, we have an obvious isomorphism $G : \scat{C} \to \scat{C}_{P_F}$.
The equality $F_{P_F} \circ G = F$ holds by definition.
This completes the proof of the first assertion.

Now, let us prove the second assertion.
Let $\scat{D}$ be a category.
Let $P$ be a type over $\ob{\scat{D}}$ such that, for every $x : \ob{\scat{D}}$, the type $P(x)$ is a proposition.
We need to show that $\scat{C}_P$ is a category.
Indeed, since $P(x)$ is a proposition, the type $\Id((x,p),(x',p'))$ of paths between objects $(x,p)$ and $(x',p')$ of $\scat{C}_P$ is equivalent to $\Id(x,x')$,
which is equivalent to the type $x \simeq y$ of isomorphisms of $x$ and $x'$ since $\scat{D}$ is a category and $x \simeq y$ is equivalent to $(x,p) \simeq (x',p')$ by the definition of $\scat{C}_P$.
Now, let $\scat{C}$ be a category and let $F : \scat{C} \to \scat{D}$ be an embedding.
We need to show that $P_F(x)$ is a proposition for every $x : \ob{\scat{D}}$.
Since $\scat{C}$ and $\scat{D}$ are categories and $F$ is an embedding of categories, it follows that $\fob{F}$ is an embedding of types.
Since $P_F(x)$ is a fiber of $\fob{F}$ over $x$ and $\fob{F}$ is an embedding, it follows that $P_F(x)$ is a proposition.
This completes the proof.
\end{proof}

A full subprecategory $P$ (or a subcategory) is \emph{small} with respect to a universe $\uType$ if $P(x)$ is equivalent to a type in $\uType$ for every $x : \ob{\scat{D}}$.
Thus, a full subprecategory of $\scat{D}$ is simply a map $\ob{\scat{D}} \to \uType$.
If $\scat{D}$ is small, then the equivalences defined in the previous lemma restrict to equivalences between small full sub(pre)categories of $\scat{D}$ and small (pre)categories embedded in $\scat{D}$.

% In general, we do not have a type of full subprecategories of a given precategory, but we do have a type of full subcategories of a given category if we have a subtype classifier
% (that is, a univalent universe $\Prop$ consisting of propositions such that, for every proposition $P$, there exists an element $\chi_P : \Prop$ such that $\El(\chi_P)$ is equivalent to $P$).

\bibliographystyle{amsplain}
\bibliography{ref}

\end{document}
