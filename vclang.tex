\documentclass{amsart}

\usepackage{etex}
\usepackage[english,russian]{babel}
\usepackage[utf8]{inputenc}
\usepackage{amssymb}
\usepackage[all]{xy}
\usepackage{verbatim}
\usepackage{ifthen}
\usepackage{xargs}
\usepackage{bussproofs}
\usepackage{type1ec}
\usepackage{stmaryrd}
\usepackage{tikz}
\usepackage{tikz-qtree}

\providecommand\WarningsAreErrors{false}
\ifthenelse{\equal{\WarningsAreErrors}{true}}{\renewcommand{\GenericWarning}[2]{\GenericError{#1}{#2}{}{This warning has been turned into a fatal error.}}}{}

\newcommand{\newref}[4][]{
\ifthenelse{\equal{#1}{}}{\newtheorem{h#2}[hthm]{#4}}{\newtheorem{h#2}{#4}[#1]}
\expandafter\newcommand\csname r#2\endcsname[1]{\ref{#2:##1}}
\expandafter\newcommand\csname R#2\endcsname[1]{#4~\ref{#2:##1}}
\newenvironmentx{#2}[2][1=,2=]{
\ifthenelse{\equal{##2}{}}{\begin{h#2}}{\begin{h#2}[##2]}
\ifthenelse{\equal{##1}{}}{}{\label{#2:##1}}
}{\end{h#2}}
}

\newref[section]{thm}{теорема}{Теорема}
\newref{lem}{лемма}{Лемма}
\newref{prop}{утверждение}{Утверждение}
\newref{cor}{следствие}{Следствие}

\theoremstyle{definition}
\newref{defn}{definition}{Definition}
\newref{example}{example}{Example}

\theoremstyle{remark}
\newref{remark}{замечание}{Замечание}

\newcommand{\red}{\Rightarrow}
\newcommand{\deq}{\Leftrightarrow}
\renewcommand{\ll}{\llbracket}
\newcommand{\rr}{\rrbracket}
\newcommand{\cat}[1]{\mathbf{#1}}
\renewcommand{\C}{\cat{C}}
\newcommand{\Set}{\cat{Set}}
\newcommand{\ccat}{\cat{CCat}}

\numberwithin{figure}{section}

\newcommand{\pb}[1][dr]{\save*!/#1-1.2pc/#1:(-1,1)@^{|-}\restore}
\newcommand{\po}[1][dr]{\save*!/#1+1.2pc/#1:(1,-1)@^{|-}\restore}

\begin{document}

\title{Vclang}

\author{Валерий Исаев}

% \begin{abstract}
% Abstract
% \end{abstract}

\maketitle

\section{Vccore}

Язык vclang состоит из двух частей: ядра (vccore) и фронтенд (vclang).
Пользователь пишет код на vclang, и он транслируется в vccore.
Как это происходит мы обсудим позже, в этом разделе мы опишем ядро.

Программа на vccore состоит из множества определений (то есть \emph{неупорядоченного} списка).
Каждое определение имеет уникальное имя.
Кроме того, каждое определение имеет контекст, который мы будем называть \emph{контекстом определения}.
Этот контекст можно рассматривать просто как дополнительные параметры этого определения.
Рассмотрим несколько примеров:
\begin{itemize}
\item Если в vclang объявлена функция $f$, то в vccore она будет представляться как определение с именем $f$ и пустым контекстом.
\item Если в vclang в классе $A$ объявлена статическая функция $f$, то в vccore она будет представляться как определение с именем $A.f$ (имя может быть любое, главное обеспечить уникальность) и пустым контекстом.
\item Если в vclang в классе $A$ объявлена нестатическая функция $f$, то в vccore она будет представляться как определение с именем $A.f$ и контекстом $x : A$, так как мы должны указать значение типа $A$ при вызове $f$.
\item Если в vclang в классе $A$ объявлен нестатический класс $B$, а в классе $B$ объявлена статическая функция $f$, то в vccore $B$ будет представляться как определение с именем $A.B$ и контекстом $x : A$, а $f$ -- как определение с именем $A.B.f$ и контекстом $x : A$.
\item Если в vclang в классе $A$ объявлен нестатический класс $B$, а в классе $B$ объявлена нестатическая функция $f$, то в vccore $f$ будет представляться как определение с именем $A.B.f$ и контекстом $x : A, y : B\ x$.
\end{itemize}
Формально это преобразование из vclang в vccore будет описано позже.

Определения бывают четырех видов.
Каждое из них состоит из данных, описанных выше, плюс дополнительный набор данных.
\begin{itemize}
\item \emph{Абстрактное} определение содержит сигнатуру, которая является просто произвольным типом (то есть выражением на vccore; выражения будут описаны ниже). Абстрактное определение не может иметь пустой контекст определения.
\item \emph{Функция} содержит сигнатуру, которая является типом, и тело, которое также является выражением.
\item \emph{Алгебраический тип данных} содержит некоторый контекст, описывающий параметры типа данных, и список конструкторов, каждый из которых состоит из имени и контекста, описывающего его параметры.
\item \emph{Класс} содержит список имен определений (состоящий из нестатических определений, объявленных в этом классе). Таким образом, классы не содержат самих определений, а только ссылки на них, сами определения задаются отдельно.
\end{itemize}

Примеры определений приведены ниже. В квадранных скобках указан контекст определения.
\begin{align*}
& [\ ]\ \mathbf{function}\ b : Nat \to Nat => \lambda x \Rightarrow x \\
& [\ ]\ \mathbf{class}\ A\ \{ p, q \} \\
& [x : A]\ \mathbf{abstract}\ p : Nat \to Nat \\
& [y : A]\ \mathbf{function}\ q : Nat => p\ y\ (b\ 7) \\
& [\ ]\ \mathbf{data}\ List\ (A : \mathbf{Type}) \\
& \qquad |\ nil \\
& \qquad |\ cons\ (head : A)\ (tail : List\ A)
\end{align*}
Обратите внимание, что при вызове определения параметры из его контекста определения становятся просто дополнительными аргументами.

\bibliographystyle{amsplain}
\bibliography{ref}

\end{document}
