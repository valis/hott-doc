\documentclass[reqno]{amsart}

\usepackage{amssymb}
\usepackage{hyperref}
\usepackage{mathtools}
\usepackage[all]{xy}
\usepackage{ifthen}
\usepackage{xargs}
\usepackage{bussproofs}
\usepackage{turnstile}
\usepackage{verbatim}

\hypersetup{colorlinks=true,linkcolor=blue}

\renewcommand{\turnstile}[6][s]
    {\ifthenelse{\equal{#1}{d}}
        {\sbox{\first}{$\displaystyle{#4}$}
        \sbox{\second}{$\displaystyle{#5}$}}{}
    \ifthenelse{\equal{#1}{t}}
        {\sbox{\first}{$\textstyle{#4}$}
        \sbox{\second}{$\textstyle{#5}$}}{}
    \ifthenelse{\equal{#1}{s}}
        {\sbox{\first}{$\scriptstyle{#4}$}
        \sbox{\second}{$\scriptstyle{#5}$}}{}
    \ifthenelse{\equal{#1}{ss}}
        {\sbox{\first}{$\scriptscriptstyle{#4}$}
        \sbox{\second}{$\scriptscriptstyle{#5}$}}{}
    \setlength{\dashthickness}{0.111ex}
    \setlength{\ddashthickness}{0.35ex}
    \setlength{\leasturnstilewidth}{2em}
    \setlength{\extrawidth}{0.2em}
    \ifthenelse{%
      \equal{#3}{n}}{\setlength{\tinyverdistance}{0ex}}{}
    \ifthenelse{%
      \equal{#3}{s}}{\setlength{\tinyverdistance}{0.5\dashthickness}}{}
    \ifthenelse{%
      \equal{#3}{d}}{\setlength{\tinyverdistance}{0.5\ddashthickness}
        \addtolength{\tinyverdistance}{\dashthickness}}{}
    \ifthenelse{%
      \equal{#3}{t}}{\setlength{\tinyverdistance}{1.5\dashthickness}
        \addtolength{\tinyverdistance}{\ddashthickness}}{}
        \setlength{\verdistance}{0.4ex}
        \settoheight{\lengthvar}{\usebox{\first}}
        \setlength{\raisedown}{-\lengthvar}
        \addtolength{\raisedown}{-\tinyverdistance}
        \addtolength{\raisedown}{-\verdistance}
        \settodepth{\raiseup}{\usebox{\second}}
        \addtolength{\raiseup}{\tinyverdistance}
        \addtolength{\raiseup}{\verdistance}
        \setlength{\lift}{0.8ex}
        \settowidth{\firstwidth}{\usebox{\first}}
        \settowidth{\secondwidth}{\usebox{\second}}
        \ifthenelse{\lengthtest{\firstwidth = 0ex}
            \and
            \lengthtest{\secondwidth = 0ex}}
                {\setlength{\turnstilewidth}{\leasturnstilewidth}}
                {\setlength{\turnstilewidth}{2\extrawidth}
        \ifthenelse{\lengthtest{\firstwidth < \secondwidth}}
            {\addtolength{\turnstilewidth}{\secondwidth}}
            {\addtolength{\turnstilewidth}{\firstwidth}}}
        \ifthenelse{\lengthtest{\turnstilewidth < \leasturnstilewidth}}{\setlength{\turnstilewidth}{\leasturnstilewidth}}{}
    \setlength{\turnstileheight}{1.5ex}
    \sbox{\turnstilebox}
    {\raisebox{\lift}{\ensuremath{
        \makever{#2}{\dashthickness}{\turnstileheight}{\ddashthickness}
        \makehor{#3}{\dashthickness}{\turnstilewidth}{\ddashthickness}
        \hspace{-\turnstilewidth}
        \raisebox{\raisedown}
        {\makebox[\turnstilewidth]{\usebox{\first}}}
            \hspace{-\turnstilewidth}
            \raisebox{\raiseup}
            {\makebox[\turnstilewidth]{\usebox{\second}}}
        \makever{#6}{\dashthickness}{\turnstileheight}{\ddashthickness}}}}
        \mathrel{\usebox{\turnstilebox}}}

\newcommand{\newref}[4][]{
\ifthenelse{\equal{#1}{}}{\newtheorem{h#2}[hthm]{#4}}{\newtheorem{h#2}{#4}[#1]}
\expandafter\newcommand\csname r#2\endcsname[1]{#3~\ref{#2:##1}}
\expandafter\newcommand\csname R#2\endcsname[1]{#4~\ref{#2:##1}}
\expandafter\newcommand\csname n#2\endcsname[1]{\ref{#2:##1}}
\newenvironmentx{#2}[2][1=,2=]{
\ifthenelse{\equal{##2}{}}{\begin{h#2}}{\begin{h#2}[##2]}
\ifthenelse{\equal{##1}{}}{}{\label{#2:##1}}
}{\end{h#2}}
}

\newref[section]{thm}{theorem}{Theorem}
\newref{lem}{lemma}{Lemma}
\newref{prop}{proposition}{Proposition}
\newref{cor}{corollary}{Corollary}
\newref{cond}{condition}{Condition}
\newref{conj}{conjecture}{Conjecture}

\theoremstyle{definition}
\newref{defn}{definition}{Definition}
\newref{example}{example}{Example}

\theoremstyle{remark}
\newref{rem}{remark}{Remark}

\newcommand{\red}{\Rightarrow}
\newcommand{\deq}{\equiv}
\newcommand{\repl}{:=}
\newcommand{\idtype}{\rightsquigarrow}

\newcommand{\cat}[1]{\mathbf{#1}}
\newcommand{\C}{\cat{C}}
\newcommand{\Cat}{\cat{Cat}}
\newcommand{\Mod}[1]{#1\text{-}\cat{Mod}}
\newcommand{\Th}{\cat{Th}}
\newcommand{\emptyCtx}{\mathbf{1}}
\newcommand{\Set}{\cat{Set}}
\newcommand{\sSet}{\cat{sSet}}
\newcommand{\lcsSet}{\sSet^{\mathrm{lc}}}
\newcommand{\lcN}{N^{\mathrm{lc}}}
\newcommand{\lcC}{\mathfrak{C}^{\mathrm{lc}}}
\newcommand{\lcI}{\I^{\mathrm{lc}}}
\newcommand{\cSet}{\cat{cSet}}
\newcommand{\Hom}{\mathrm{Hom}}

\newcommand{\we}{\mathcal{W}}
\newcommand{\I}{\mathrm{I}}
\newcommand{\J}{\mathrm{J}}
\newcommand{\class}[2]{#1\text{-}\mathrm{#2}}
\newcommand{\Icell}[1][\I]{\class{#1}{cell}}
\newcommand{\Icof}[1][\I]{\class{#1}{cof}}
\newcommand{\Jcell}[1][]{\Icell[\J#1]}
\newcommand{\cyli}{i}

\numberwithin{figure}{section}

\newcommand{\pb}[1][dr]{\save*!/#1-1.2pc/#1:(-1,1)@^{|-}\restore}
\newcommand{\po}[1][dr]{\save*!/#1+1.2pc/#1:(1,-1)@^{|-}\restore}

\begin{document}

\title[Equivalence between quasicategories and models of type theory]{Equivalence between locally cartesian (closed) quasicategories and models of homotopy type theory with an interval type}

\author{Valery Isaev}

\begin{abstract}
Abstract.
\end{abstract}

\maketitle

\section{Introduction}

\begin{comment}
\section{Cubically enriched categories}

In this section we recall the definition of the category $\Cat_\square$ of cubically enriched categories and Quillen adjunction between
the cateogry simplicial sets $\sSet$ (with Joyal model structure) and $\Cat_\square$.
We also define Quillen adjunction between $\Cat_\square$ and categories of models of suitable algebraic dependent type theories.

Let $D$ be poset $\{0,1\}$ where $0 \leq 1$.
For every $0 \leq n$, $1 \leq i \leq n+1$, $\epsilon \in D$, let $\delta^{i,\epsilon}_n : D^n \to D^{n+1}$ and $\sigma^i_n : D^{n+1} \to D^n$ be the following morphisms of posets:
\begin{align*}
\delta^{i,\epsilon}_n(x_1, \ldots x_n) & = (x_1, \ldots x_{i-1}, \epsilon, x_{i+1}, \ldots x_n) \\
\sigma^i_n(x_1, \ldots x_{n+1}) & = (x_1, \ldots x_{i-1}, x_{i+1}, \ldots x_{n+1})
\end{align*}
Category of cubes $\square$ is the subcategory of the category of posets on objects of the form $D^n$ generated by morphisms $\delta^{i,\epsilon}_n$ and $\sigma^i_n$.
Let $\cSet$ be the category of cubical sets $\Set^{\square^{op}}$.
Let $\square^n \in \cSet$ be the representable presheaf on $D^n$.
Let $d^{i,\epsilon}_n : \square^n \to \square^{n+1}$ and $s^i_n : \square^{n+1} \to \square^n$ be the maps represented by $\delta^{i,\epsilon}_n$ and $\sigma^i_n$.

There is a (unique) closed monoidal structure $(\otimes,1)$ on $\cSet$ such that $\square^n \otimes \square^m \cong \square^{n+m}$ and $1 = \square^0$ (see, for example, \cite{jardine}).
Let $X$ be an object of a cocomplete monoidal category $\C$.
Then there exists a (unique) colimit-preserving strong monoidal functor $F : \cSet \to \C$ such that $F(\square^1) = X$.
This functor has a right adjoint $G : \C \to \cSet$ given by formula $G(X) = \Hom_\C(F(y(-)),X)$, where $y : \square \to \cSet$ is the Yoneda embedding.
An examples of this construction is the adjunction $|-| : \cSet \rightleftarrows \sSet : S$ such that $|\square^1| = \Delta^1$.

For every $n \geq 0$, let $\partial \square^n$ be the following coequalizer:
\[ \coprod_{1 \leq i < j \leq n, \epsilon_1,\epsilon_2 \in D} \square^{n-2} \rightrightarrows \coprod_{1 \leq i \leq n, \epsilon \in D} \square^{n-1} \to \partial \square^n, \]
where one of the morphisms maps $(i,j,\epsilon_1,\epsilon_2)$-th component to $(i,\epsilon_1)$-th component by map $d^{j-1,\epsilon_2}_{n-2}$,
the other one maps $(i,j,\epsilon_1,\epsilon_2)$-th component to $(j,\epsilon_2)$-th component by map $d^{i,\epsilon_1}_{n-2}$,
Consider morphism $f : \coprod_{1 \leq i \leq n, \epsilon \in D} \square^{n-1}$ that is defined on $(i,\epsilon)$-th component by $d^{i,\epsilon}_{n-1}$.
Then this morphism determines a morphism $i^n : \partial \square^n \to \square^n$.

For every $n \geq 0$, $1 \leq i \leq n$, $\epsilon \in D$, let $\sqcap^n_{i,\epsilon}$ be the following coequalizer:
\[ \coprod_{\substack{1 \leq j_1 < j_2 \leq n, \gamma_1,\gamma_2 \in D, \\ (j_1,\gamma_1) \neq (i,\epsilon), (j_2,\gamma_2) \neq (i,\epsilon)}} \square^{n-2} \rightrightarrows \coprod_{\substack{1 \leq j \leq n, \gamma \in D, \\ (j,\gamma) \neq (i,\epsilon)}} \square^{n-1} \to \sqcap^n_{i,\epsilon}, \]
where morphisms are defined as before.
The restriction of $f$ determines a morphism $u^n_{i,\epsilon} : \sqcap^n_{i,\epsilon} \to \square^n$.

There is a model structure on $\cSet$ with maps $i^n : \partial \square^n \to \square^n$ as generating cofibrations
and maps $u^n_{i,\epsilon} : \sqcap^n_{i,\epsilon} \to \square^n$ as generating trivial cofibrations (see, for example, \cite[Example~52, Theorem~88]{jardine}).
Functors $|-| : \cSet \rightleftarrows \sSet : S$ determine a Quillen equivalence.

Let $\Cat_\Delta$ and $\Cat_\square$ be the categories of simplicially and cubically enriched categories respectively (that is, categories enriched in $\sSet$ and $\cSet$).
Bergner constructed a model structure on $\Cat_\Delta$ in \cite{bergner}.
Lurie generalized this construction (\cite[Proposition~A.3.2.4]{lurie-topos}).
In particular, this proposition applies to the category of cubically enriched categories.
To describe the class of cofibrations of this model category, we need too define a functor $ar : \cSet \to \Cat_\square$.
For every cubical set $A$, let $ar(A)$ be the cubically enriched category with two objects $X$ and $Y$ and $\Hom(X,Y) = A$, $\Hom(X,X) = \Hom(Y,Y) = 1$, $\Hom(Y,X) = \varnothing$.
The class of cofibrations is generated by the set of maps of the form $ar(i^n)$ and map $\varnothing \to 1$.

\begin{rem}
Fibrations in $\Cat_\square$ do not have simple description, but fibrant objects have.
Since the category of simplicial sets is excellent, the category of cubical sets is also excellent by \cite[Remark~A.3.2.21]{lurie-topos}.
By \cite[Theorem~A.3.2.24]{lurie-topos}, a cubically enriched category $C$ is fibrant
if and only if for every objects $X$ and $Y$ of $C$, $\Hom_C(X,Y)$ is a fibrant cubical set.
Also, a map $F : C \to D$ between cubically enriched categories, where $D$ is fibrant, is a fibration if and only if the following conditions are satisfied:
\begin{enumerate}
\item For every object $X$ and $Y$ of $C$, $\Hom_C(X,Y) \to \Hom_D(F(X),F(Y))$ is a fibration of cubical sets.
\item For every object $X$ of $C$ and every equivalence $f : F(X) \to Y$ in $D$, there exists an equivalence $f' : X \to Y'$ in $C$ such that $F(f') = f$.
\end{enumerate}
\end{rem}

There is a pair of adjoint functors $\mathfrak{C}_\Delta : \sSet \rightleftarrows \Cat_\Delta : N_\Delta$ which is defined in \cite[Section~1.1.5]{lurie-topos}.
We will need analogous functors for cubically enriched categories.
We define a cubically enriched category $\mathfrak{C}_\square(\Delta^n)$ as follows:
\begin{itemize}
\item The objects of $\mathfrak{C}_\square(\Delta^n)$ are natural numbers between $0$ and $n$.
\item For every $0 \leq j < i \leq n$, $\Hom_{\mathfrak{C}_\square(\Delta^n)}(i,j) = \varnothing$.
\item For every $0 \leq i \leq n$, $\Hom_{\mathfrak{C}_\square(\Delta^n)}(i,i) = 1$.
\item For every $0 \leq i < j \leq n$, $\Hom_{\mathfrak{C}_\square(\Delta^n)}(i,j) = \square^{j-i-1}$.
\item For every object $i$ of $\mathfrak{C}_\square(\Delta^n)$, the identity morphism on $i$ is the unique element of $\Hom_{\mathfrak{C}_\square(\Delta^n)}(i,i)$.
\item For every $0 \leq i < j < k \leq n$, the composition
\[ \Hom_{\mathfrak{C}_\square(\Delta^n)}(i,j) \otimes \Hom_{\mathfrak{C}_\square(\Delta^n)}(j,k) \to \Hom_{\mathfrak{C}_\square(\Delta^n)}(i,k) \]
is the composite $\square^{j-i-1} \otimes \square^{k-j-1} \cong \square^{k-i-2} \to \square^{k-i-1}$.
\end{itemize}
\end{comment}

\section{Functors between $\sSet$ and $\Mod{T}$}

In this section we define a Quillen adjunction between $\sSet$ and $\Mod{T}$ for every appropriate algebraic dependent type theory $T$.
Recall that $sq_l$ is the theory with the following axioms:
\medskip
\begin{center}
\AxiomC{$\Gamma \vdash i : I$}
\AxiomC{$\Gamma \vdash j : I$}
\BinaryInfC{$\Gamma \vdash sq_l(i,j) : I$}
\DisplayProof
\end{center}
\begin{align*}
sq_l(left,j) & = left \\
sq_l(right,j) & = j \\
sq_l(i,left) & = left \\
sq_l(i,right) & = i
\end{align*}
To define functors between $\sSet$ and $\Mod{T}$, we also need to assume that $sq_l$ is associative.
Let $sq^a_l$ be the theory which has the same axioms as $sq_l$ together with the following axiom:
\[ sq_l(sq_l(i,j),k) = sq_l(i,sq_l(j,k)) \]

Let $T$ be any theory under $sq^a_l$.
For every finite nonempty linearly ordered set $J$, we define a model $\mathfrak{C}(\Delta^J)$ of $T$.
It is freely generated by generators and relations described below.
Generators of $\mathfrak{C}(\Delta^J)$ are $O_j : (ty,0)$ for every $j \in J$, and $M_{J'} : (tm,|J'|-1)$ for every subset $J'$ of $J$ such that $|J'| \geq 2$.
For every $j \in J$, $\mathfrak{C}(\Delta^J)$ has relation $\vdash O_j\ type$.
For every $J' = \{ j_1 < \ldots < j_n \} \subseteq J$ such that $n \geq 2$, it has the following relation:
\[ x : O_{j_1}, x_{j_2} : I, \ldots x_{j_{n-1}} : I \vdash M_{J'} : O_{j_n} \]
For every $j \notin J'$ such that $j_1 < j < j_n$, it has the following relations:
\begin{align}
x : O_{j_1}, x_{j_2} : I, \ldots x_{j_{n-1}} : I & \vdash M_{J' \cup \{j\}}[x_j \repl left] \deq M_{\{ j, \ldots j_n \}}[x \repl M_{\{ j_1, \ldots j \}}] \label{rel:left} \\
x : O_{j_1}, x_{j_2} : I, \ldots x_{j_{n-1}} : I & \vdash M_{J' \cup \{j\}}[x_j \repl right] \deq M_{J'} \label{rel:right}
\end{align}

The idea of this definition is similar to \cite[Definition~1.1.5.1]{lurie-topos}.
For every vertex $j$ of $\Delta^J$, we have a type $O_j$,
and for every $(n+1)$-dimensional face $J'$ of $\Delta^J$, we have an $n$-dimensional cube $M_{J'}$.
Relations describe faces of these cubes.

Let $sq_l(x_1, \ldots x_n) = sq_l(x_1, \ldots sq_l(x_{n-1},x_n) \ldots )$.
In particular, $sq_l(X) = x$ and $sq_l() = right$.
Now, we can extend $\mathfrak{C}$ to a functor $\Delta \to \Mod{T}$.
Let $f : J \to K$ be a monotone map between linearly ordered sets.
Then we define $\mathfrak{C}(f)$ as follows:
\begin{align*}
\mathfrak{C}(f)(O_j) & = O_{f(j)} \\
\mathfrak{C}(f)(M_{J'}) & =
\begin{cases}
    x                                                                  & \text{if } |f(J')| = 1 \\
    M_{f(J')}[\ \ldots\ x_{k_i} \repl sq_l(x_{f^{-1}(k_i)})\ \ldots\ ] & \text{if } |f(J')| > 1 \\
\end{cases}
\end{align*}
where $k_i$ runs from $k_2$ to $k_{m-1}$, $\{ k_1 < \ldots < k_m \} = f(J')$,
and $x_{f^{-1}(k_i)} = x_{j_1}, \ldots x_{j_n}$, $\{ j_1 < \ldots < j_n \} = f^{-1}(k_i)$.

Let us check that $\mathfrak{C}(f)$ preserves relations.
Let $J' = \{ j_1 < \ldots < j_i < j < j_{i+1} < \ldots < j_n \}$ be a subset of $J$, and let $f'$ be the restriction of $f$ to $J'$.
If $|f(J')| = 1$, then $\mathfrak{C}(f)(M_{J' \cup \{j\}}[x_j \repl left]) = \mathfrak{C}(f)(M_{\{ j, \ldots j_n \}}[x \repl M_{\{ j_1, \ldots j \}}]) = \mathfrak{C}(f)(M_{J' \cup \{j\}}[x_j \repl right]) = \mathfrak{C}(f)(M_{J'}) = x$.
Thus we may assume that $|f(J')| > 1$. 

Let $\rho(x_{f(j_i)}) = sq_l(x_{f'^{-1}(f(j_1))})$ for every $1 \leq i \leq n$ such that $f(j_i) \neq f(j)$.
Let us consider the first relation.
If $f(j) = f(j_1)$, then
\begin{align*}
\mathfrak{C}(f)(M_{J' \cup \{j\}}[x_j \repl left]) & = \\
M_{f(J')}[\rho] & = \\
M_{\{ f(j), \ldots f(j_n) \}}[\rho][x \repl x] & = \\
\mathfrak{C}(f)(M_{\{ j, \ldots j_n \}}[x \repl M_{\{ j_1, \ldots j \}}]) & .
\end{align*}
If $f(j) = f(j_n)$, then
\begin{align*}
\mathfrak{C}(f)(M_{J' \cup \{j\}}[x_j \repl left]) & = \\
M_{f(J')}[\rho] & = \\
x[x \repl M_{\{f(j_1), \ldots f(j)\}}[\rho]] & = \\
\mathfrak{C}(f)(M_{\{ j, \ldots j_n \}}[x \repl M_{\{ j_1, \ldots j \}}]) & .
\end{align*}
If $f(j_1) < f(j) < f(j_n)$, then
\begin{align*}
\mathfrak{C}(f)(M_{J' \cup \{j\}}[x_j \repl left]) & = \\
M_{f(J') \cup \{f(j)\}}[\rho, x_{f(j)} \repl sq_l(x_{f'^{-1}(f(j))}, x_j)][x_j \repl left] & = \\
M_{f(J') \cup \{f(j)\}}[\rho, x_{f(j)} \repl left] & = \\
M_{\{f(j), \ldots f(j_n)\}}[x \repl M_{\{f(j_1), \ldots f(j)\}}][\rho] & = \\
\mathfrak{C}(f)(M_{\{ j, \ldots j_n \}}[x \repl M_{\{ j_1, \ldots j \}}]) & .
\end{align*}

Let us consider the second relation.
If either $f(j) = f(j_1)$ or $f(j) = f(j_n)$, then $\mathfrak{C}(f)(M_{J' \cup \{j\}}[x_j \repl right]) = M_{f(J')}[\rho] = \mathfrak{C}(f)(M_{J'})$.
If $f(j_1) < f(j) < f(j_n)$, then
\begin{align*}
\mathfrak{C}(f)(M_{J' \cup \{j\}}[x_j \repl right]) & = \\
M_{f(J') \cup \{f(j)\}}[\rho, x_{f(j)} \repl sq_l(x_{f'^{-1}(f(j))}, x_j)][x_j \repl right] & = \\
M_{f(J') \cup \{f(j)\}}[\rho, x_{f(j)} \repl sq_l(x_{f'^{-1}(f(j))})] & = \\
\mathfrak{C}(f)(M_{J'}) & ,
\end{align*}
If $f(j_i) < f(j) < f(j_{i+1})$, then the last equality holds by \eqref{rel:right},
otherwise it holds immediately by definition of $\mathfrak{C}(f)(M_{J'})$.

Let $[n]$ be the set of natural numbers $\{ 0, \ldots n \}$ with the obvious linear order.
Then $\Delta$ is the full subcategory of linearly ordered sets on objects of the form $[n]$.
It is easy to see that $\mathfrak{C}$ preserves identity morphisms,
and the fact that it preserves compositions follows from associativity of $sq_l$.
Thus we have a functor $\mathfrak{C} : \Delta \to \Mod{T}$.
This functor (uniquely) extends to a colimit-preserving functor $\mathfrak{C} : \sSet \to \Mod{T}$,
which has a right adjoint $N : \Mod{T} \to \sSet$ defined by equation $N(X) = \Hom(\mathfrak{C}(-),X)$.

We can explicitly describe $\mathfrak{C} : \sSet \to \Mod{T}$.
Model $\mathfrak{C}(X)$ is generated by symbols $O_a : (ty,0)$ for every $a \in X_0$, and $M_a : (tm,n)$ for every $a \in X_n$, $n \geq 1$.
For every $a \in X_0$, $\mathfrak{C}(X)$ has relations $\vdash O_a\ type$ and $M_{s_0(a)} = (x : O_a \vdash x : O_a)$.
For every $a \in X_n$, $n \geq 1$, it has the following relation:
\[ x : O_{a|\Delta^{\{0\}}}, x_1 : I, \ldots x_{n-1} : I \vdash M_a : O_{a|\Delta^{\{n\}}} \]
For every $1 \leq j \leq n-1$, it has the following relations:
\begin{align*}
M_a[x_j \repl left] & = M_{a|\Delta^{\{ j, \ldots n \}}}[x \repl M_{a|\Delta^{\{ 0, \ldots j \}}}] \\
M_a[x_j \repl right] & = M_{a|\Delta^{\{0, \ldots j-1, j+1, \ldots n\}}}
\end{align*}
For every $0 \leq j \leq n$, it has the following relation:
\[ M_{s_j(a)} =
\begin{cases}
    M_a[x_1 \repl x_2, \ldots x_{n-1} \repl x_n]                                      & \text{if } j = 0 \\
    M_a[x_j \repl sq_l(x_j,x_{j+1}), x_{j+1} \repl x_{j+2}, \ldots x_{n-1} \repl x_n] & \text{if } 1 \leq j \leq n-1 \\
    M_a                                                                               & \text{if } j = n \\
\end{cases}
\]
For every morphism $f : X \to Y$ of simplicial sets, let $\mathfrak{C}(f)(O_a) = O_{f(a)}$ and $\mathfrak{C}(f)(M_a) = M_{f(a)}$.
It is easy to see that this definition preserves relations, identity morphisms and compositions.

\begin{prop}
Functors $\mathfrak{C} \dashv N$ determine a Quillen adjunction between $\sSet$ with the Joyal model structure and $\Mod{T}$ with the model structure defined in \cite{alg-models}.
\end{prop}
\begin{proof}
Note that $\mathfrak{C}(\partial \Delta^n)$ is isomorphic (over $\mathfrak{C}(\Delta^n)$) to the theory
freely generated by the same generators and relations as $\mathfrak{C}(\Delta^n)$ except for $M_{[n]}$.
Similarly, if $i \in [n]$, then $\mathfrak{C}(\Lambda^n_i)$ is freely generated by the same generators
and relations as $\mathfrak{C}(\partial \Delta^n)$ except for $M_{\{ 0, \ldots i-1, i+1, \ldots n \}}$.

Let us prove that $\mathfrak{C}$ preserves cofibration.
To do this, it is enough to show that $\mathfrak{C}(\partial \Delta^n) \to \mathfrak{C}(\Delta^n)$ is a cofibration for every $n$.
If $n = 0$, then this map equals to $i_{(ty,0)}$.
Suppose that $n > 0$.
First, let us define objects $\square^k$ and $\partial \square^k$.
Let $\square^k = F(\{ x : A, x_1 : I, \ldots x_k : I \vdash b : B \})$,
and let $\partial \square^k$ be freely generated by generators and relations described below.
Generators of $\partial \square^k$ are $A,B : (ty,0)$ and $b_{[i=c]} : (tm,k)$ for every $1 \leq i \leq k$, $c \in \{left,right\}$.
For every $1 \leq i \leq k$ and $c \in \{left,right\}$, it has the following relation:
\[ x : A, x_1 : I, \ldots x_{i-1} : I, x_{i+1} : I, \ldots x_k : I \vdash b_{[i=c]} : B \]
For every $1 \leq i_1 < i_2 \leq k$ and $c,c' \in \{left,right\}$, it has the following relation:
\[ b_{[i_2=c']}[x_{i_1} \repl c] = b_{[i_1=c]}[x_{i_2} \repl c'] \]

Map $i^k : \partial \square^k \to \square^k$ is defined in the obvious way: $i^k(b_{[i=c]}) = b[x_i \repl c]$.
Let $v : \square^{n-1} \to \mathfrak{C}(\Delta^n)$ be defined as follows: $v(A) = O_0$, $v(B) = O_n$, $v(b) = M_{[n]}$.
Map $v \circ i^k$ factors through $\mathfrak{C}(\partial \Delta^n) \to \mathfrak{C}(\Delta^n)$, and the following square is cocartesian:
\[ \xymatrix{ \partial \square^{n-1} \ar[r] \ar[d]_{i^k} & \mathfrak{C}(\partial \Delta^n) \ar[d] \\
              \square^{n-1} \ar[r]_v & \mathfrak{C}(\Delta^n)
            } \]
Thus we just need to show that $i^k$ is a cofibration.
But it is a pushout of $i_{(tm,1)} : F(\{ x : A \vdash B\ type \}) \to F(\{ x : A \vdash b : B \})$.
Indeed, let $v' : F(\{ x : A \vdash b : B \}) \to \square^k$ be defined as follows: $v'(b) = path(x_1.\,\ldots path(x_n.\,b) \ldots)$ and $v'(B)$ equals to
\begin{align*}
Path(x_1.\, \ldots Path(x_{n-2}.\, & Path(x_{n-1}.\,     Path(x_n.\,B, b[x_n \repl left], b[x_n \repl right]), \\
                                   & \qquad \qquad \quad path(x_n.\,b[x_{n-1} \repl left]), \\
                                   & \qquad \qquad \quad path(x_n.\,b[x_{n-1} \repl right])), \\
                                   & path(x_{n-1}.\,     path(x_n.\,b[x_{n-2} \repl left])), \\
                                   & path(x_{n-1}.\,     path(x_n.\,b[x_{n-2} \repl right]))) \ldots )
\end{align*}
Then $v' \circ i_{(tm,1)}$ factors through $i^k$, and the following square is cocartesian:
\[ \xymatrix{ F(\{ x : A \vdash B\ type \}) \ar[r] \ar[d]_{i_{(tm,1)}} & \partial \square^k \ar[d]^{i^k} \\
              F(\{ x : A \vdash b : B \}) \ar[r]_-{v'} & \square^k
            } \]

Now, let us prove that for every $X \in \Mod{T}$, $N(X)$ is fibrant.
Since fibrant objects in $\sSet$ are precisely quasicategories,
it is enough to show that $\mathfrak{C}(\Lambda^n_i) \to \mathfrak{C}(\Delta^n)$ is a trivial cofibration for every inner horn $\Lambda^n_i \to \Delta^n$.
First, for every $1 \leq i \leq k$, let $\sqcap^k_i$ be the object defined by the same generators and relations as $\partial \square^k$ except for $b_{[i=right]}$.
Then we have the following cocartesian square:
\[ \xymatrix{ \sqcap^{n-1}_i \ar[r] \ar[d] & \mathfrak{C}(\Lambda^n_i) \ar[d] \\
              \partial \square^{n-1} \ar[r] & \mathfrak{C}(\partial \Delta^n)
            } \]

Since $\sqcap^k_i \to \square^k$ is a cofibration, we just need to show that it is a weak equivalence.
Note that $\sqcap^k_i \to \square^k$ is isomorphic to $\sqcap^k_k \to \square^k$ for every $1 \leq i \leq k$.
Indeed, we can define isomorphism $g : \square^k \to \square^k$ as $g(b) = b[x_i \repl x_k, x_k \repl x_i]$
and isomorphism $f : \sqcap^k_i \to \sqcap^k_k$ as $f(b_{[i=left]}) = b_{[k=left]}[x_i \repl x_k]$, $f(b_{[k=c]}) = b_{[i=c]}[x_k \repl x_i]$,
and $f(b_{[j=c]}) = b_{[j=c]}[x_i \repl x_k, x_k \repl x_i]$ for every $j \neq i,k$.
Then $f$ and $g$ commute with $\sqcap^k_i \to \square^k$ and $\sqcap^k_k \to \square^k$.
Thus we just need to prove that $\sqcap^k_k \to \square^k$ is a weak equivalence.

Note that $\square^{k+1}$ is a (relative) cylinder object for $F(\{ A : (ty,0), B : (ty,0) \}) \to \square^k$.
Indeed, $\square^k \amalg_{F(\{ A : (ty,0), B : (ty,0) \})} \square^k \to \square^{k+1}$ is a cofibration,
and $\square^k \to \square^{k+1}$ is a weak equivalence since it is a pushout of a generating trivial cofibration $F(\{ \Gamma \vdash b : B \}) \to F(\{ \Gamma, y : I \vdash h : B \})$.

Every object has RLP with respect to $\sqcap^k_k \to \square^k$ (see \cite[Section~2.4]{alg-models}).
In particular, $\sqcap^k_k \to \square^k$ has a retract $\square^k \to \sqcap^k_k$.
The composite map $\square^k \to \sqcap^k_k \to \square^k$ is homotopic to $id$.
To show this, we just need to construct an appropriate map $\square^{k+1} \to \square^k$, which is easy to do using fillers.
Thus $\sqcap^k_k \to \square^k$ is a homotopy equivalence.

Finally, let us prove that if $f : X \to Y$ is a fibration in $\Mod{T}$, then $N(f)$ is a fibration in $\sSet$.
Since $N(Y)$ is fibrant, by \cite[Corollary~2.4.6.5]{lurie-topos}, we just need to prove the following conditions:
\begin{enumerate}
\item $N(f)$ is an inner fibration.
\item For every object $A$ of $N(X)$ and every equivalence $p : N(f)(A) \to B$ in $N(Y)$, there exists an equivalence $p' : A \to B'$ in $N(X)$ such that $N(f)(p') = p$.
\end{enumerate}
The first condition follows from the fact that maps $\mathfrak{C}(\Lambda^n_i) \to \mathfrak{C}(\Delta^n)$ are trivial cofibrations.
The second condition can be reformulated as follows: for every closed type $A$ of $X$ and every equivalence $p : f(A) \to B$ in $Y$,
there exists an equivalence $p' : A \to B'$ in $X$ such that $f(p') = p$.
But this follows from the fact that $f$ has RLP with respect to $\J_{\I_{ty}}$.
\end{proof}

\section{Simplicial sets with chosen cones}

Let $T^a_\Sigma = coe_1 + \sigma + Path + wUA + \Sigma + sq^a_l$.
Quillen adjunction between $\sSet$ and $\Mod{T^a_\Sigma}$ that we constructed in the previous section is not a Quillen equivalence.
Model category $\Mod{T^a_\Sigma}$ should represent the $(\infty,1)$-category of finitely complete $(\infty,1)$-categories (and exact functor),
but $\sSet$ represent the $(\infty,1)$-category of all $(\infty,1)$-categories.
To fix this problem, we define model category $\lcsSet$ which is closely related to $\sSet$ and which is Quillen equivalent to $\Mod{T^a_\Sigma}$.
We will prove later that the fibration category corresponding to $\lcsSet$ is equivalent to the fibration category of finitely complete quasicategories.

Objects of $\lcsSet$ are simplicial sets ``with chosen cones for finite diagrams''.
More precisely, an object of $\lcsSet$ is a pointed simplicial set $(X, * \in X_0)$ with a function that to every pair of edges $f : B \to D$, $g : C \to D$ in $X$ associates a square $\Delta^1 \times \Delta^1 \to X$ as follows:
\[ \xymatrix{ A \ar@{-->}[r] \ar@{-->}[d] & B \ar[d]^f \\
              C \ar[r]_g & D
            } \]
Morphisms of $X,Y \in \lcsSet$ are morphisms of pointed simplicial sets that preserve chosen squares.

We will call objects of $\lcsSet$ \emph{simplicial sets with chosen cones}.
The intention is that these chosen cones are the terminal object and pullback squares of $X$.
Of course, in general, this may not be the case, but we will construct a model structure on $\lcsSet$ in which an object is fibrant if and only if it is a quasicategory and chosen cones are limit diagrams.

Let $X$ be a model of some theory $T$.
Edges of $N(X)$ are classified by maps $\mathfrak{C}(\Delta^1) \to X$, or equivalently by terms $x : A \vdash b : B$.
Squares $\Delta^1 \times \Delta^1 \to N(X)$ are classified by the following data:
types $A,B,C,D \in X_{(ty,0)}$ and terms $f,g,f',g',r \in X_{(tm,1)}$, $h_1,h_2 \in X_{(tm,2)}$ such that $y : B \vdash f : D$, $z : C \vdash g : D$, $x : A \vdash f' : C$, $x : A \vdash g' : B$, $x : A \vdash r : D$,
$x : A, i : I \vdash h_1 : D$, $x : A, i : I \vdash h_2 : D$, $h_1[left] = f[y \repl g']$, $h_1[right] = h_2[right] = r$, and $h_2[left] = g[z \repl f']$.
These data can be depicted in the following diagram (where the top and the bottom triangles correspond to $h_1$ and $h_2$ respectively):
\[ \xymatrix{ A \ar[r]^{g'} \ar[d]_{f'} \ar[rd]^r & B \ar[d]^f \\
              C \ar[r]_g & D
            } \]

If $T$ is under $T^a_\Sigma$, then $N : \Mod{T} \to \sSet$ factors through the forgetful functor $U : \lcsSet \to \sSet$.
Let the chosen vertex of $N(X)$ be the interval type.
For every $y : B \vdash f : D$ and $z : C \vdash g : D$, the chosen square in $N(X)$ is defined as follows:
$A = \Sigma(y : B, \Sigma(z : C, f \idtype g))$, $f' = (x : A \vdash \pi_1(\pi_2(x)) : C)$, $g' = (x : A \vdash \pi_1(x) : B)$,
$r = (x : A \vdash g[z \repl f'] : D)$, $h_1 = (x : A, y : I \vdash at(\pi_2(\pi_2(x)),y) : D)$, and $h_2 = (x : A, y : I \vdash r : D)$.
We will denote this functor by $\lcN : \Mod{T} \to \lcsSet$.

Functor $\lcN$ has a left adjoint $\lcC : \lcsSet \to \Mod{T}$.
For every $X \in \lcsSet$, $\lcC(X)$ is defined by the same generators and relations as $\mathfrak{C}(X)$
together relations that make chosen cones in $X$ equal to corresponding data in the theory.
Let us give a more explicit description of models of the form $\mathfrak{C}(X)$.
By \cite[Remark~3.4]{alg-models}, elements of $\mathfrak{C}(X)_{(p,n)}$ are equivalence classes of closed terms constructed from function symbols of $T^a_\Sigma$.
The (partial) equivalence relation on the set of terms is generated by a relation $\red$ such that every term has a unique normal form
(that is, for every term $t$, there exists a unique term $t'$ such that $t \red^* t'$ and there is no term $t''$ such that $t' \red t''$).
This implies that there is a bijection between the set of equivalences classes of terms and the set of terms in a normal form.

Relation $\red$ is defined as the minimal relation that is closed under application of function symbols and that contains all axioms in which direction is chosen in the obvious way.

To construct a model structure on $\lcsSet$, we need to prove a few technical lemmas.

\begin{lem}
\end{lem}
\begin{proof}
\end{proof}

Forgetful functor $U : \lcsSet \to \sSet$ has a left adjoint $F : \sSet \to \lcsSet$.
We define the set of generating cofibrations $\lcI$ of $\lcsSet$ to be the set $F(\I)$, where $\I$ is the set of inclusions $\partial \Delta^n \to \Delta^n$.

\begin{prop}
There is a combinatorial model structure on $\lcsSet$ such that $\lcI$ is a set of generating cofibrations and a map $f$ is a weak equivalence if and only if $\lcC(f)$ is a weak equivalence in $\Mod{T^a_\Sigma}$.
Adjunction $\lcC \dashv \lcN$ is a Quillen equivalence.
\end{prop}
\begin{proof}
By a theorem of Jeff Smith (see, for example, \cite[Proposition~A.2.6.8]{lurie-topos}),
to construct such model structure, it is enough to prove that for every map $f$ in $\lcsSet$ which has RLP with respect to $\lcI$, $\lcC(f)$ is a weak equivalence.
Let $f : X \to Y$ be such a map.
\end{proof}

\bibliographystyle{amsplain}
\bibliography{ref}

\end{document}
