\documentclass{amsart}

\usepackage[english,russian]{babel}
\usepackage[utf8]{inputenc}
\usepackage{amssymb}
\usepackage[all]{xy}
\usepackage{verbatim}
\usepackage{ifthen}
\usepackage{xargs}
\usepackage{bussproofs}
\usepackage{type1ec}
\usepackage{stmaryrd}
% \usepackage[T2A]{fontenc}

\providecommand\WarningsAreErrors{false}
\ifthenelse{\equal{\WarningsAreErrors}{true}}{\renewcommand{\GenericWarning}[2]{\GenericError{#1}{#2}{}{This warning has been turned into a fatal error.}}}{}

\newcommand{\newref}[4][]{
\ifthenelse{\equal{#1}{}}{\newtheorem{h#2}[hthm]{#4}}{\newtheorem{h#2}{#4}[#1]}
\expandafter\newcommand\csname r#2\endcsname[1]{\ref{#2:##1}}
\expandafter\newcommand\csname R#2\endcsname[1]{#4~\ref{#2:##1}}
\newenvironmentx{#2}[2][1=,2=]{
\ifthenelse{\equal{##2}{}}{\begin{h#2}}{\begin{h#2}[##2]}
\ifthenelse{\equal{##1}{}}{}{\label{#2:##1}}
}{\end{h#2}}
}

\newref[section]{thm}{теорема}{Теорема}
\newref{lem}{лемма}{Лемма}
\newref{prop}{утверждение}{Утверждение}
\newref{cor}{следствие}{Следствие}

\theoremstyle{definition}
\newref{defn}{definition}{Definition}
\newref{example}{example}{Example}

\theoremstyle{remark}
\newref{remark}{замечание}{Замечание}

\newcommand{\red}{\Rightarrow}
\newcommand{\deq}{\Leftrightarrow}
\renewcommand{\ll}{\llbracket}
\newcommand{\rr}{\rrbracket}
\newcommand{\cat}[1]{\mathbf{#1}}
\renewcommand{\C}{\cat{C}}

\numberwithin{figure}{section}

\newcommand{\pb}[1][dr]{\save*!/#1-1.2pc/#1:(-1,1)@^{|-}\restore}
\newcommand{\po}[1][dr]{\save*!/#1+1.2pc/#1:(1,-1)@^{|-}\restore}

\begin{document}

\title{Bidirectional type theory}

\author{Валерий Исаев}

% \begin{abstract}
% Abstract
% \end{abstract}

\maketitle

\section{Введение}

\section{Теория типов Мартин-Лёфа}

Теория типов Мартин-Лёфа $T_I$ состоит из счетного множества переменных $Var$, множества термов $Term_I$, отношения эквивалентности на нем $\deq$ и трех отношений $- \vdash$, $- \vdash -$, $- \vdash - : - $.
Множество контекстов $Ctx_I$ определяется следующим образом:
\[ Ctx_I = \{ x_1 : A_1, \ldots x_n : A_n\ |\ x_i \in Var, A_i \in Term_I \} \]
Отношения $- \vdash$, $- \vdash -$ и $- \vdash - : -$ заданы на множествах $Ctx_I$, $Ctx_I \times Term_I$ и $Ctx_I \times Term_I \times Term_I$ соответственно.
Они определяются в таблице~\ref{table:inf-rules}.

\begin{table}

\medskip
\begin{center}
\AxiomC{}
\RightLabel{$(CTX_1)$}
\UnaryInfC{$\vdash$}
\DisplayProof
\quad
\AxiomC{$\Gamma \vdash A$}
\RightLabel{, $x \notin \Gamma$ $(CTX_2)$}
\UnaryInfC{$\Gamma, x : A \vdash$}
\DisplayProof
\quad
\AxiomC{$\Gamma \vdash$}
\RightLabel{, $x : A \in \Gamma$ $(VAR)$}
\UnaryInfC{$\Gamma \vdash x : A$}
\DisplayProof
\end{center}

\medskip
\begin{center}
\AxiomC{$\Gamma \vdash a : A$}
\AxiomC{$\Gamma \vdash B$}
\RightLabel{, $A \deq B$ $(CONV)$}
\BinaryInfC{$\Gamma \vdash a : B$}
\DisplayProof
\end{center}

\medskip
\begin{center}
\AxiomC{$\Gamma \vdash A$}
\AxiomC{$\Gamma, x : A \vdash B$}
\RightLabel{$(PI)$}
\BinaryInfC{$\Gamma \vdash \Pi (x : A) B$}
\DisplayProof
\end{center}

\medskip
\begin{center}
\AxiomC{$\Gamma, x : A \vdash b : B$}
\RightLabel{$(LAM_I)$}
\UnaryInfC{$\Gamma \vdash \lambda x. b : \Pi (x : A) B$}
\DisplayProof
\quad
\AxiomC{$\Gamma \vdash f : \Pi (x : A) B$}
\AxiomC{$\Gamma \vdash a : A$}
\RightLabel{$(APP)$}
\BinaryInfC{$\Gamma \vdash f\ a : B[x \mapsto a]$}
\DisplayProof
\end{center}

\medskip
\begin{center}
\AxiomC{$\Gamma \vdash A$}
\AxiomC{$\Gamma, x : A \vdash B$}
\RightLabel{$(SIGMA)$}
\BinaryInfC{$\Gamma \vdash \Sigma (x : A) B$}
\DisplayProof
\end{center}

\medskip
\begin{center}
\AxiomC{$\Gamma \vdash a : A$}
\AxiomC{$\Gamma, x : A \vdash B$}
\AxiomC{$\Gamma \vdash b : B[x \mapsto a]$}
\RightLabel{$(PAIR_I)$}
\TrinaryInfC{$\Gamma \vdash (a, b) : \Sigma (x : A) B$}
\DisplayProof
\end{center}

\medskip
\begin{center}
\AxiomC{$\Gamma \vdash p : \Sigma (x : A) B$}
\RightLabel{$(PROJ_1)$}
\UnaryInfC{$\Gamma \vdash proj_1\ p : A$}
\DisplayProof
\quad
\AxiomC{$\Gamma \vdash p : \Sigma (x : A) B$}
\RightLabel{$(PROJ_2)$}
\UnaryInfC{$\Gamma \vdash proj_2\ p : B[x \mapsto proj_1\ p]$}
\DisplayProof
\end{center}

\medskip
\begin{center}
\AxiomC{$\Gamma \vdash a_1 : A$}
\AxiomC{$\Gamma \vdash a_2 : A$}
\RightLabel{$(ID)$}
\BinaryInfC{$\Gamma \vdash Id\ A\ a_1\ a_2$}
\DisplayProof
\quad
\AxiomC{$\Gamma \vdash a : A$}
\RightLabel{$(REFL_I)$}
\UnaryInfC{$\Gamma \vdash refl : Id\ A\ a\ a$}
\DisplayProof
\end{center}

\medskip
\begin{center}
\def\extraVskip{1pt}
\Axiom$\fCenter \Gamma, x_1 : A, x_2 : A, t : Id\ A\ x_1\ x_2 \vdash B$
\noLine
\UnaryInf$\fCenter \Gamma, x : A \vdash b : B[x_1 \mapsto x, x_2 \mapsto x, t \mapsto refl]$
\noLine
\UnaryInf$\fCenter \Gamma \vdash p : Id\ A\ a_1\ a_2$
\RightLabel{$(J)$}
\UnaryInf$\fCenter \Gamma \vdash J\ (\lambda x_1 x_2 t. B)\ (\lambda x. b)\ a_1\ a_2\ p : B[x_1 \mapsto a_1, x_2 \mapsto a_2, t \mapsto p]$
\DisplayProof
\end{center}

\medskip
\begin{center}
\AxiomC{$\Gamma \vdash$}
\RightLabel{$(U)$}
\UnaryInfC{$\Gamma \vdash U$}
\DisplayProof
\quad
\AxiomC{$\Gamma \vdash A : U$}
\RightLabel{$(EL)$}
\UnaryInfC{$\Gamma \vdash A$}
\DisplayProof
\end{center}

\medskip
\begin{center}
\AxiomC{$\Gamma \vdash A : U$}
\AxiomC{$\Gamma, x : A \vdash B : U$}
\RightLabel{$(U \text{-} PI)$}
\BinaryInfC{$\Gamma \vdash \Pi (x : A) B : U$}
\DisplayProof
\end{center}

\medskip
\begin{center}
\AxiomC{$\Gamma \vdash A : U$}
\AxiomC{$\Gamma, x : A \vdash B : U$}
\RightLabel{$(U \text{-} SIGMA)$}
\BinaryInfC{$\Gamma \vdash \Sigma (x : A) B : U$}
\DisplayProof
\end{center}

\medskip
\begin{center}
\AxiomC{$\Gamma \vdash A : U$}
\AxiomC{$\Gamma \vdash a_1 : A$}
\AxiomC{$\Gamma \vdash a_2 : A$}
\RightLabel{$(U \text{-} ID)$}
\TrinaryInfC{$\Gamma \vdash Id\ A\ a_1\ a_2 : U$}
\DisplayProof
\end{center}

\bigskip
\caption{Правила вывода $T_I$.}
\label{table:inf-rules}
\end{table}

Мы будем работать с теорией типов с $\Pi$ типами, $\Sigma$ типами, $Id$ типами и одной вселенной $U$.
Разумеется, теория может быть расширена другими конструкциями.

Множество термов определяется индуктивно следующим набором правил:
\[ Term_I := x\ |\ U\ | \]
\[ \Pi (x : A) B\ |\ \lambda x. b\ |\ b\ a\ | \]
\[ \Sigma (x : A) B\ |\ (a, b)\ |\ proj_1\ p\ |\ proj_2\ p\ | \]
\[ Id\ A\ a_1\ a_2\ |\ refl\ |\ J\ (\lambda x_1 x_2 t. B)\ (\lambda x. b)\ a_1\ a_2\ p, \]
где $x$ - переменная, $A$, $B$, $a$, $a_1$, $a_2$, $b$ и $p$ - термы.

На множестве термов очевидным образом определяется операция подстановки.
Если $b$ - терм, $\rho$ - частичная функция из $Var$ в $Term_I$, то $b[\rho]$ - терм, в котором все свободные вхождения переменных $x$ из домена $\rho$ заменены на $\rho(x)$.

Отношение $\deq$ является наименьшим отношением конгруэнтности (то есть отношением эквивалентности, замкнутым относительно конструкций термов), удовлетворяющим следующим свойствам:
\begin{itemize}
\item $(\lambda x. b)\ a \deq b[x \mapsto a]$.
\item $proj_1\ (a, b) \deq a$.
\item $proj_2\ (a, b) \deq b$.
\item $J\ (\lambda x_1 x_2 t. B)\ (\lambda x. b)\ a\ a\ refl \deq b[x \mapsto a]$.
\item $\lambda x. b\ x \deq b$, если $x \notin FV(b)$.
\item $(proj_1\ p, proj_2\ p) \deq p$.
\end{itemize}

Контекстуальная категория теории $\C_{T_I}$ определяется следующим образом:
\[ Ob(\C_{T_I}) = \{ \Gamma \in Ctx_I\ |\ \Gamma \vdash \}. \]
Если $\Gamma = x_1 : A_1, \ldots x_n : A_n$ и $\Delta = y_1 : B_1, \ldots y_k : B_k$ - пара контекстов,
то множество $CT_I(\Gamma, \Delta)$ определяется как множество функций $\rho : \{ y_1, \ldots y_k\} \to Term_I$ таких, что
\[ \Gamma \vdash \rho(y_i) : B_i[\rho |_{y_1, \ldots y_{i-1}}] \text{ для всех } 1 \leq i \leq k. \]
Тогда множество морфизмов $\C_{T_I}(\Gamma, \Delta)$ определяется как множество $CT_I(\Gamma, \Delta)$ с точностью до отношения эквивалентности $\deq$:
\[ \rho \deq \rho' \text{ тогда и только тогда, когда } \rho(y_i) \deq \rho'(y_i) \text{ для всех } 1 \leq i \leq k. \]
Тождественный морфизм $id_\Delta$ определяется как функция $id_\Delta(y_i) = y_i$.
Если $\rho : \Gamma \to \Delta$, $\tau : \Delta \to E$, где $E = z_1 : C_1, \ldots z_m : C_m$, то композиция $\tau \circ \rho$ определяется как функция $(\tau \circ \rho)(z_i) = \tau(z_i)[\rho]$.
Отношения типизации и эквивалентности определены так, что эти определения действительно задают категорию.

\section{Вариации теории}

\subsection{Теория типов с явно типизированными конструкторами}

Мы опишем версию теории типов, в которой конструкторы типизированы явно.
Подобная теория описывается в \cite{luo94}.
Множество термов задается тем же набором правил за исключением того, что мы заменяем правила $\lambda x. b$, $(a, b)$ и $refl$:
\[ Term_E := \ldots\ |\ \lambda x : A. b\ |\ (a, b)_{\Sigma (x : A) B}\ |\ refl\ a \]

Правила вывода теории такие же как и раньше за исключением того, что мы заменяем $(LAM_I)$, $(PAIR_I)$ и $(REFL_I)$ следующими правилами:

\medskip
\begin{center}
\AxiomC{$\Gamma, x : A \vdash b : B$}
\RightLabel{$(LAM_E)$}
\UnaryInfC{$\Gamma \vdash \lambda x : A. b : \Pi (x : A) B$}
\DisplayProof
\end{center}
\medskip

\begin{center}
\AxiomC{$\Gamma \vdash a : A$}
\AxiomC{$\Gamma, x : A \vdash B$}
\AxiomC{$\Gamma \vdash b : B[x \mapsto a]$}
\RightLabel{$(PAIR_E)$}
\TrinaryInfC{$\Gamma \vdash (a, b)_{\Sigma (x : A) B} : \Sigma (x : A) B$}
\DisplayProof
\end{center}
\medskip

\begin{center}
\AxiomC{$\Gamma \vdash a : A$}
\RightLabel{$(REFL_E)$}
\UnaryInfC{$\Gamma \vdash refl\ a : Id\ A\ a\ a$}
\DisplayProof
\end{center}
\medskip

Отношение $\deq$ определяется так же как и раньше за исключением того, что правила конгруэнтности не распространяются на подтерм $A$ в $\lambda x : A. b$, подтерм $B$ в $(a, b)_{\Sigma (x : A) B}$ и подтерм $a$ в $refl\ a$.
Другими словами
\begin{align*}
& \lambda x : A. b \deq \lambda x : A'. b' \text{ только если } A = A' \text{ и } b \deq b'. \\
& (a, b)_{\Sigma (x : A) B} \deq (a', b')_{\Sigma (x : A') B'} \text{ только если } a \deq a' \text{, } b \deq b' \text{, } A = A' \text{ и } B = B'. \\
& refl\ a \deq refl\ a' \text{ только если } a = a'.
\end{align*}

Контекстуальная категория теории $\C_{T_E}$ определяется аналогичным $\C_{T_I}$ образом.
Существует очевидная функция $F : Term_E \to Term_I$, удовлетворяющая следуюшим свойствам:
\begin{align*}
& F(\lambda x : A. b) = \lambda x. F(b) \\
& F((a, b)_{\Sigma (x : A) B}) = (F(a), F(b)) \\
& F(refl\ a) = refl
\end{align*}
Эта функция задает функтор $F : \C_{T_E} \to \C_{T_I}$ очевидным образом.
Позже мы докажем, что этот функтор является эквивалентностью категорий.

\subsection{Двунаправленная теория типов}

Здесь мы определим теорию $T_B$.
Данная теория имеет то же множество термов, что и предыдущая $Term_E$.
Вместо отношения $- \vdash - : -$ мы вводим два отношения $- \vdash - \Uparrow -$ и $- \vdash - \Downarrow -$.
Правила вывода приведены в таблице~\ref{table:bi-rules}.

\begin{table}

\medskip
\begin{center}
\AxiomC{}
\RightLabel{$(CTX_1)$}
\UnaryInfC{$\vdash$}
\DisplayProof
\quad
\AxiomC{$\Gamma \vdash A$}
\RightLabel{, $x \notin \Gamma$ $(CTX_2)$}
\UnaryInfC{$\Gamma, x : A \vdash$}
\DisplayProof
\quad
\AxiomC{$\Gamma \vdash$}
\RightLabel{, $x : A \in \Gamma$ $(VAR)$}
\UnaryInfC{$\Gamma \vdash x \Uparrow A$}
\DisplayProof
\end{center}

\medskip
\begin{center}
\AxiomC{$\Gamma \vdash a \Uparrow A$}
\AxiomC{$\Gamma \vdash B$}
\RightLabel{, $A \deq B$ $(CONV)$}
\BinaryInfC{$\Gamma \vdash a \Downarrow B$}
\DisplayProof
\end{center}

\medskip
\begin{center}
\AxiomC{$\Gamma \vdash A$}
\AxiomC{$\Gamma, x : A \vdash B$}
\RightLabel{$(PI)$}
\BinaryInfC{$\Gamma \vdash \Pi (x : A) B$}
\DisplayProof
\quad
\AxiomC{$\Gamma, x : A \vdash b \Uparrow B$}
\RightLabel{$(LAM_E)$}
\UnaryInfC{$\Gamma \vdash \lambda x : A. b \Uparrow \Pi (x : A) B$}
\DisplayProof
\end{center}

\medskip
\begin{center}
\AxiomC{$\Gamma \vdash f \Uparrow C$}
\AxiomC{$\Gamma \vdash a \Downarrow A$}
\RightLabel{, $C \red^*_h \Pi (x : A) B$ $(APP)$}
\BinaryInfC{$\Gamma \vdash f\ a \Uparrow B[x \mapsto a]$}
\DisplayProof
\end{center}

\medskip
\begin{center}
\AxiomC{$\Gamma \vdash A$}
\AxiomC{$\Gamma, x : A \vdash B$}
\RightLabel{$(SIGMA)$}
\BinaryInfC{$\Gamma \vdash \Sigma (x : A) B$}
\DisplayProof
\end{center}

\medskip
\begin{center}
\AxiomC{$\Gamma \vdash a \Downarrow A$}
\AxiomC{$\Gamma, x : A \vdash B$}
\AxiomC{$\Gamma \vdash b \Downarrow B[x \mapsto a]$}
\RightLabel{$(PAIR_E)$}
\TrinaryInfC{$\Gamma \vdash (a, b)_{\Sigma (x : A) B} \Uparrow \Sigma (x : A) B$}
\DisplayProof
\end{center}

\medskip
\begin{center}
\AxiomC{$\Gamma \vdash p \Uparrow C$}
\RightLabel{, $C \red^*_h \Sigma (x : A) B$ $(PROJ_1)$}
\UnaryInfC{$\Gamma \vdash proj_1\ p \Uparrow A$}
\DisplayProof
\end{center}

\medskip
\begin{center}
\AxiomC{$\Gamma \vdash p \Uparrow C$}
\RightLabel{, $C \red^*_h \Sigma (x : A) B$ $(PROJ_2)$}
\UnaryInfC{$\Gamma \vdash proj_2\ p \Uparrow B[x \mapsto proj_1\ p]$}
\DisplayProof
\end{center}

\medskip
\begin{center}
\AxiomC{$\Gamma \vdash a_1 \Downarrow A$}
\AxiomC{$\Gamma \vdash a_2 \Downarrow A$}
\RightLabel{$(ID)$}
\BinaryInfC{$\Gamma \vdash Id\ A\ a_1\ a_2$}
\DisplayProof
\quad
\AxiomC{$\Gamma \vdash a \Uparrow A$}
\RightLabel{$(REFL_E)$}
\UnaryInfC{$\Gamma \vdash refl\ a \Uparrow Id\ A\ a\ a$}
\DisplayProof
\end{center}

\medskip
\begin{center}
\def\extraVskip{0.5pt}
\Axiom$\fCenter \Gamma, x_1 : A, x_2 : A, t : Id\ A\ x_1\ x_2 \vdash B$
\noLine
\UnaryInf$\fCenter \Gamma, x : A \vdash b \Downarrow B[x_1 \mapsto x, x_2 \mapsto x, t \mapsto refl\ x]$
\noLine
\UnaryInf$\fCenter \Gamma \vdash p \Downarrow Id\ A\ a_1\ a_2$
\RightLabel{$(J)$}
\def\extraVskip{2pt}
\UnaryInf$\fCenter \Gamma \vdash J\ (\lambda x_1 x_2 t. B)\ (\lambda x. b)\ a_1\ a_2\ p \Uparrow B[x_1 \mapsto a_1, x_2 \mapsto a_2, t \mapsto p]$
\DisplayProof
\end{center}

\medskip
\begin{center}
\AxiomC{$\Gamma \vdash$}
\RightLabel{$(U)$}
\UnaryInfC{$\Gamma \vdash U$}
\DisplayProof
\quad
\AxiomC{$\Gamma \vdash A \Downarrow U$}
\RightLabel{$(EL)$}
\UnaryInfC{$\Gamma \vdash A$}
\DisplayProof
\end{center}

\medskip
\begin{center}
\AxiomC{$\Gamma \vdash A \Downarrow U$}
\AxiomC{$\Gamma, x : A \vdash B \Downarrow U$}
\RightLabel{$(U \text{-} PI)$}
\BinaryInfC{$\Gamma \vdash \Pi (x : A) B \Uparrow U$}
\DisplayProof
\end{center}

\medskip
\begin{center}
\AxiomC{$\Gamma \vdash A \Downarrow U$}
\AxiomC{$\Gamma, x : A \vdash B \Downarrow U$}
\RightLabel{$(U \text{-} SIGMA)$}
\BinaryInfC{$\Gamma \vdash \Sigma (x : A) B \Uparrow U$}
\DisplayProof
\end{center}

\medskip
\begin{center}
\AxiomC{$\Gamma \vdash A \Downarrow U$}
\AxiomC{$\Gamma \vdash a_1 \Downarrow A$}
\AxiomC{$\Gamma \vdash a_2 \Downarrow A$}
\RightLabel{$(U \text{-} ID)$}
\TrinaryInfC{$\Gamma \vdash Id\ A\ a_1\ a_2 \Uparrow U$}
\DisplayProof
\end{center}

\bigskip
\caption{Правила вывода $T_B$.}
\label{table:bi-rules}
\end{table}

Отношение $\red^*_h$ определяется как рефллексивное транзитивное замыкание отношения $\red_h$:
\begin{itemize}
\item Если $b \red_h b'$, то $b\ a \red_h b'\ a$.
\item $(\lambda x : A. b)\ a \red_h b[x \mapsto a]$.
\item Если $p \red_h p'$, то $proj_1\ p \red_h proj_1\ p'$.
\item Если $p \red_h p'$, то $proj_2\ p \red_h proj_2\ p'$.
\item $proj_1\ (a, b)_{\Sigma (x : A) B} \red_h a$.
\item $proj_2\ (a, b)_{\Sigma (x : A) B} \red_h b$.
\item Если $p \red_h p'$, то $J\ (\lambda x_1 x_2 t. B)\ (\lambda x. b)\ a_1\ a_2\ p \red_h J\ (\lambda x_1 x_2 t. B)\ (\lambda x. b)\ a_1\ a_2\ p'$.
\item $J\ (\lambda x_1 x_2 t. B)\ (\lambda x. b)\ a\ a\ (refl\ a) \red_h b[x \mapsto a]$.
\end{itemize}

Контекстуальная категория $\C_{T_B}$ данной теории определяется аналогичным $\C_{T_I}$ образом, где в определении морфизмов вместо отношения $- \vdash - : -$ используется отношение $- \vdash - \Downarrow -$.

\begin{prop}
Верны следующие утверждения:
\begin{itemize}
\item Если в $T_B$ выводится $\Gamma \vdash$, то в $T_E$ также выводится $\Gamma \vdash$.
\item Если в $T_B$ выводится $\Gamma \vdash A$, то в $T_E$ также выводится $\Gamma \vdash A$.
\item Если в $T_B$ выводится $\Gamma \vdash a \Uparrow A$ или $\Gamma \vdash a \Downarrow A$, то в $T_E$ выводится $\Gamma \vdash a : A$.
\end{itemize}
\end{prop}
\begin{proof}
Элементарная индукция по выводу.
\end{proof}

\begin{prop}
Верны следующие утверждения:
\begin{itemize}
\item Если в $T_E$ выводится $\Gamma \vdash$, то в $T_B$ также выводится $\Gamma \vdash$.
\item Если в $T_E$ выводится $\Gamma \vdash A$, то в $T_B$ также выводится $\Gamma \vdash A$.
\item Если в $T_E$ выводится $\Gamma \vdash a : A$, то в $T_B$ выводится $\Gamma \vdash a \Downarrow A$.
\end{itemize}
\end{prop}
\begin{proof}
TODO
\end{proof}

Предыдущие два утверждения доказывают, что отнашения $- \vdash$, $- \vdash -$ и $- \vdash - : -$ в $T_E$ и $- \vdash$, $- \vdash -$ и $- \vdash - \Downarrow -$ в $T_B$ совпадают.
Отсюда следует, что категории $\C_{T_E}$ и $\C_{T_B}$ совпадают.
Отношение $- \vdash - \Uparrow -$ является вспомогательным.
Из правила $(CONV)$ следует, что оно является подмножеством $- \vdash - \Downarrow -$,
и следующее утверждение показывает, что для любых $\Gamma$ и $a$ существует максимум один терм $A$ такой, что $\Gamma \vdash a \Uparrow A$.
Таким образом, $\Gamma \vdash a \Uparrow A$ означает, что $A$ - это в некотором смысле канонический тип $a$.

\begin{prop}
Если $\Gamma \vdash a \Uparrow A_1$ и $\Gamma \vdash a \Uparrow A_2$, то $A_1 = A_2$.
\end{prop}
\begin{proof}
Заметим, что если $a \red_h b_1$ и $a \red_h b_2$, то $b_1 = b_2$.
Следовательно, если термы $b_1$ и $b_2$ находятся в $\red_h$-нормальной форме и $a \red^*_h b_1$ и $a \red^*_h b_2$, то $b_1 = b_2$.
При этом термы вида $\Sigma (x : A) B$ и $\Pi (x : A) B$ находятся в $\red_h$-нормальной форме.

Используя эти замечания, утверждение элементарно доказывается индукцией по выводу.
\end{proof}

\section{Эквивалентность теорий}

Мы уже определили функтор $\C_{T_I} \to \C_{T_E}$ (и, следовательно, $\C_{T_I} \to \C_{T_B}$).
Чтобы определить функтор в обратную сторону, нам понадобится вспомогательная конструкция.
Мы определим следующие отношения:
\begin{itemize}
\item $\ll - \vdash \rr = - \vdash$ на множестве $Ctx_I \times Ctx_E$.
\item $\ll - \vdash - \rr = - \vdash -$ на множестве $Ctx_I \times Term_I \times Ctx_E \times Term_E$.
\item $\ll - \vdash - : - \rr = - \vdash - \Uparrow -$ на множестве $Ctx_I \times Term_I \times Term_I \times Ctx_E \times Term_E \times Term_E$.
\item $\ll - \vdash - : - \rr = - \vdash - \Downarrow -$ на множестве $Ctx_I \times Term_I \times Term_I \times Ctx_E \times Term_E \times Term_E$.
\end{itemize}

Отношения определяются индуктивным образом:

\medskip
\begin{center}
\AxiomC{}
\UnaryInfC{$\ll \vdash \rr = \vdash$}
\DisplayProof
\quad
\AxiomC{$\ll \Gamma \vdash A \rr = \Gamma' \vdash A'$}
\RightLabel{, $x \notin \Gamma$}
\UnaryInfC{$\ll \Gamma, x : A \vdash \rr = \Gamma', x : A' \vdash$}
\DisplayProof
\end{center}

\medskip
\begin{center}
\AxiomC{$\ll \Gamma \vdash \rr = \Gamma' \vdash$}
\RightLabel{, $x : A \in \Gamma$, $x : A' \in \Gamma'$}
\UnaryInfC{$\ll \Gamma \vdash x : A \rr = \Gamma' \vdash x \Uparrow A'$}
\DisplayProof
\end{center}

\medskip
\begin{center}
\AxiomC{$\ll \Gamma \vdash a : A \rr = \Gamma' \vdash a' \Uparrow A'$}
\AxiomC{$\ll \Gamma \vdash B \rr = \Gamma' \vdash B'$}
\RightLabel{, $A \red^*_h B$}
\BinaryInfC{$\ll \Gamma \vdash a : B \rr = \Gamma' \vdash a' \Uparrow A'$}
\DisplayProof
\end{center}

\medskip
\begin{center}
\AxiomC{$\ll \Gamma \vdash a : A \rr = \Gamma' \vdash a' \Uparrow A'$}
\AxiomC{$\ll \Gamma \vdash B \rr = \Gamma' \vdash B'$}
\RightLabel{, $A \deq B$}
\BinaryInfC{$\ll \Gamma \vdash a : B \rr = \Gamma' \vdash a' \Downarrow B'$}
\DisplayProof
\end{center}

\medskip
\begin{center}
\AxiomC{$\ll \Gamma \vdash A \rr = \Gamma' \vdash A'$}
\AxiomC{$\ll \Gamma, x : A \vdash B \rr = \Gamma', x : A' \vdash B'$}
\BinaryInfC{$\ll \Gamma \vdash \Pi (x : A) B \rr = \Gamma' \vdash \Pi (x : A') B'$}
\DisplayProof
\end{center}

\medskip
\begin{center}
\AxiomC{$\ll \Gamma, x : A \vdash b : B \rr = \Gamma', x : A' \vdash b' \Uparrow B'$}
\UnaryInfC{$\ll \Gamma \vdash \lambda x. b : \Pi (x : A) B \rr = \Gamma' \vdash \lambda x : A'. b' \Uparrow \Pi (x : A') B'$}
\DisplayProof
\end{center}

\medskip
\begin{center}
\AxiomC{$\ll \Gamma \vdash f : \Pi (x : A) B \rr = \Gamma' \vdash f' \Uparrow C'$}
\AxiomC{$\ll \Gamma \vdash a : A \rr = \Gamma' \vdash a' \Downarrow A'$}
\RightLabel{, $C' \red^*_h \Pi (x : A') B'$}
\BinaryInfC{$\ll \Gamma \vdash f\ a : B[x \mapsto a] \rr = \Gamma' \vdash f'\ a' \Uparrow B'[x \mapsto a']$}
\DisplayProof
\end{center}

\medskip
\begin{center}
\AxiomC{$\ll \Gamma \vdash A \rr = \Gamma' \vdash A'$}
\AxiomC{$\ll \Gamma, x : A \vdash B \rr = \Gamma', x : A' \vdash B'$}
\BinaryInfC{$\ll \Gamma \vdash \Sigma (x : A) B \rr = \Gamma' \vdash \Sigma (x : A') B'$}
\DisplayProof
\end{center}

\medskip
\begin{center}
\def\extraVskip{0.5pt}
\Axiom$\fCenter \ll \Gamma \vdash a : A \rr = \Gamma' \vdash a' \Downarrow A'$
\noLine
\UnaryInf$\fCenter \ll \Gamma, x : A \vdash B \rr = \Gamma', x : A' \vdash B'$
\noLine
\UnaryInf$\fCenter \ll \Gamma \vdash b : B[x \mapsto a] \rr = \Gamma' \vdash b' \Downarrow B'[x \mapsto a']$
\def\extraVskip{2pt}
\UnaryInf$\fCenter \ll \Gamma \vdash (a, b) : \Sigma (x : A) B \rr = \Gamma' \vdash (a', b')_{\Sigma (x : A') B'} \Uparrow \Sigma (x : A') B'$
\DisplayProof
\end{center}

\medskip
\begin{center}
\AxiomC{$\ll \Gamma \vdash p : \Sigma (x : A) B \rr = \Gamma' \vdash p' \Uparrow C'$}
\RightLabel{, $C' \red^*_h \Sigma (x : A') B'$}
\UnaryInfC{$\ll \Gamma \vdash proj_1\ p : A \rr = \Gamma' \vdash proj_1\ p' \Uparrow A'$}
\DisplayProof
\end{center}

\medskip
\begin{center}
\AxiomC{$\ll \Gamma \vdash p : \Sigma (x : A) B \rr = \Gamma' \vdash p' \Uparrow C'$}
\RightLabel{, $C' \red^*_h \Sigma (x : A') B'$}
\UnaryInfC{$\ll \Gamma \vdash proj_2\ p : B[x \mapsto proj_1\ p] \rr = \Gamma' \vdash proj_2\ p' \Uparrow B'[x \mapsto proj_1\ p']$}
\DisplayProof
\end{center}

\medskip
\begin{center}
\AxiomC{$\ll \Gamma \vdash a_1 : A \rr = \Gamma' \vdash a_1' \Downarrow A'$}
\AxiomC{$\ll \Gamma \vdash a_2 : A \rr = \Gamma' \vdash a_2' \Downarrow A'$}
\BinaryInfC{$\ll \Gamma \vdash Id\ A\ a_1\ a_2 \rr = \Gamma' \vdash Id\ A'\ a_1'\ a_2'$}
\DisplayProof
\end{center}

\medskip
\begin{center}
\AxiomC{$\ll \Gamma \vdash a : A \rr = \Gamma' \vdash a' \Uparrow A'$}
\UnaryInfC{$\ll \Gamma \vdash refl : Id\ A\ a\ a \rr = \Gamma' \vdash refl\ a' \Uparrow Id\ A'\ a'\ a'$}
\DisplayProof
\end{center}

\medskip
\begin{center}
\def\extraVskip{0.5pt}
\Axiom$\ll \fCenter \Gamma, x_1 : A, x_2 : A, t : Id\ A\ x_1\ x_2 \vdash B \rr = \Gamma', x_1 : A', x_2 : A', t : Id\ A'\ x_1\ x_2 \vdash B'$
\noLine
\UnaryInf$\ll \fCenter \Gamma, x : A \vdash b : B[x_i \mapsto x, t \mapsto refl] \rr = \Gamma', x : A' \vdash b' \Downarrow B[x_i \mapsto x, t \mapsto refl\ x]$
\noLine
\UnaryInf$\ll \fCenter \Gamma \vdash p : Id\ A\ a_1\ a_2 \rr = \Gamma' \vdash p' \Downarrow Id\ A'\ a_1'\ a_2'$
\def\extraVskip{2pt}
\UnaryInf$\ll \fCenter \Gamma \vdash J\ (\lambda x_1 x_2 t. B)\ (\lambda x. b)\ a_1\ a_2\ p : B[x_1 \mapsto a_1, x_2 \mapsto a_2, t \mapsto p] \rr =$
\def\extraVskip{0.5pt}
\noLine
\UnaryInf$\fCenter \Gamma' \vdash J\ (\lambda x_1 x_2 t. B')\ (\lambda x. b')\ a_1'\ a_2'\ p' \Uparrow B'[x_1 \mapsto a_1', x_2 \mapsto a_2', t \mapsto p']$
\def\extraVskip{2pt}
\DisplayProof
\end{center}

\medskip
\begin{center}
\AxiomC{$\ll \Gamma \vdash \rr = \Gamma' \vdash$}
\UnaryInfC{$\ll \Gamma \vdash U \rr = \Gamma' \vdash U$}
\DisplayProof
\quad
\AxiomC{$\ll \Gamma \vdash A : U \rr = \Gamma' \vdash A' \Downarrow U$}
\UnaryInfC{$\ll \Gamma \vdash A \rr = \Gamma' \vdash A'$}
\DisplayProof
\end{center}

\medskip
\begin{center}
\AxiomC{$\ll \Gamma \vdash A : U \rr = \Gamma' \vdash A' \Downarrow U $}
\AxiomC{$\ll \Gamma, x : A \vdash B : U \rr = \Gamma', x : A' \vdash B' \Downarrow U$}
\BinaryInfC{$\ll \Gamma \vdash \Pi (x : A) B : U \rr = \Gamma' \vdash \Pi (x : A') B' \Uparrow U$}
\DisplayProof
\end{center}

\medskip
\begin{center}
\AxiomC{$\ll \Gamma \vdash A : U \rr = \Gamma' \vdash A' \Downarrow U $}
\AxiomC{$\ll \Gamma, x : A \vdash B : U \rr = \Gamma', x : A' \vdash B' \Downarrow U$}
\BinaryInfC{$\ll \Gamma \vdash \Sigma (x : A) B : U \rr = \Gamma' \vdash \Sigma (x : A') B' \Uparrow U$}
\DisplayProof
\end{center}

\medskip
\begin{center}
\AxiomC{$\ll \Gamma \vdash A : U \rr = \Gamma' \vdash A' \Downarrow U $}
\AxiomC{$\ll \Gamma \vdash a_1 : A \rr = \Gamma' \vdash a_1' \Downarrow A'$}
\AxiomC{$\ll \Gamma \vdash a_2 : A \rr = \Gamma' \vdash a_2' \Downarrow A'$}
\TrinaryInfC{$\ll \Gamma \vdash Id\ A\ a_1\ a_2 : U \rr = \Gamma' \vdash Id\ A'\ a_1'\ a_2' \Uparrow U$}
\DisplayProof
\end{center}
\bigskip

Одна из проблем теории $T_I$, которую призвана исправить теория $T_E$ заключается в том, что тип терма не определяется по нему однозначно.
Однако в некоторых случаях это может быть верно, как показывает следующее утверждение.

\begin{prop}
Если $\Gamma \vdash a : A_1$, $\Gamma \vdash a : A_2$ в $T_I$, $a$ находится в нормальной форме и не является конструктором (то есть не равен $\lambda x. b$, $(a, b)$ и $refl$), то $A_1 \deq A_2$.
\end{prop}
\begin{proof}
Доказательство индукцией по выводу $\Gamma \vdash a : A_i$.
Если один из выводов заканчивается правилом $(CONV)$, то утверждение верно по индукционной гипотезе.
В противном случае они должны заканчиваться одним и тем же правилом.
Случаи правил для контекстов, типов и вселенных, а так же правила $(VAR)$ и $(J)$ очевидны.
Случаи $(LAM_I)$, $(PAIR_I)$ и $(REFL_I)$ невозможны, так как мы предположили, что терм не является конструктором.

Осталось рассмотреть случаи $(APP)$, $(PROJ_1)$ и $(PROJ_2)$.
В случае аппликации $f\ a$ терм $f$ не может быть конструктором, следовательно к нему применима индукционная гипотеза, следовательно его тип $\Pi (x : A) B$ определяется однозначно с точностью до отношения $\deq$.
По конфлюэнтности следует, что $B$ и, следовательно, $B[x \mapsto a]$ также однозначно определены.
Случаи $(PROJ_1)$ и $(PROJ_2)$ разбираются аналогично.
\end{proof}

Теперь мы хотим доказать, что отношения интерпретации определяют функции.
\begin{prop}
Верны следующие утверждения:
\begin{itemize}
\item Если $\ll \Gamma \rr = \Gamma_1$ и $\ll \Gamma \rr = \Gamma_2$, то $\Gamma_1 = \Gamma_2$.
\item Если $\ll \Gamma \vdash A \rr = \Gamma_1 \vdash A_1$ и $\ll \Gamma \vdash A \rr = \Gamma_2 \vdash A_2$, то $\Gamma_1 = \Gamma_2$, и $A_1 = A_2$.
\item Если $\ll \Gamma \vdash a : A \rr = \Gamma_1 \vdash a_1 \Uparrow A_1$, $\ll \Gamma \vdash a : A \rr = \Gamma_2 \vdash a_2 \Uparrow A_2$ и $a$ находится в нормальной форме, то $\Gamma_1 = \Gamma_2$, $a_1 = a_2$ и $A_1 = A_2$.
\item Если $\ll \Gamma \vdash a : A_1 \rr = \Gamma_1 \vdash a_1 \Downarrow A'_1$, $\ll \Gamma \vdash a : A_2 \rr = \Gamma_2 \vdash a_2 \Downarrow A'_2$, $a$ находится в нормальной форме и $A_1 \deq A_2$, то $\Gamma_1 = \Gamma_2$, $a_1 = a_2$ и $A_1' \deq A_2'$.
\end{itemize}
\end{prop}

\bibliographystyle{amsplain}
\bibliography{ref}

\end{document}
