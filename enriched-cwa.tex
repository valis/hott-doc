\documentclass[reqno]{amsart}

\usepackage{amssymb}
\usepackage{hyperref}
\usepackage{mathtools}
\usepackage[all]{xy}
\usepackage{verbatim}
\usepackage{ifthen}
\usepackage{xargs}
\usepackage{bussproofs}
\usepackage{turnstile}
\usepackage{etex}

\hypersetup{colorlinks=true,linkcolor=blue}

\newcommand{\axlabel}[1]{(#1) \phantomsection \label{ax:#1}}
\newcommand{\axtag}[1]{\label{ax:#1} \tag{#1}}
\newcommand{\axref}[1]{(\hyperref[ax:#1]{#1})}

\newcommand{\newref}[4][]{
\ifthenelse{\equal{#1}{}}{\newtheorem{h#2}[hthm]{#4}}{\newtheorem{h#2}{#4}[#1]}
\expandafter\newcommand\csname r#2\endcsname[1]{#3~\ref{#2:##1}}
\expandafter\newcommand\csname R#2\endcsname[1]{#4~\ref{#2:##1}}
\expandafter\newcommand\csname n#2\endcsname[1]{\ref{#2:##1}}
\newenvironmentx{#2}[2][1=,2=]{
\ifthenelse{\equal{##2}{}}{\begin{h#2}}{\begin{h#2}[##2]}
\ifthenelse{\equal{##1}{}}{}{\label{#2:##1}}
}{\end{h#2}}
}

\newref[section]{thm}{Theorem}{Theorem}
\newref{lem}{Lemma}{Lemma}
\newref{prop}{Proposition}{Proposition}
\newref{cor}{Corollary}{Corollary}
\newref{cond}{Condition}{Condition}

\theoremstyle{definition}
\newref{defn}{Definition}{Definition}
\newref{example}{Example}{Example}

\theoremstyle{remark}
\newref{remark}{Remark}{Remark}

\newcommand{\fs}[1]{\mathrm{#1}}
\newcommand{\Hom}{\fs{Hom}}
\newcommand{\cat}[1]{\mathcal{#1}}
\newcommand{\C}{\cat{C}}
\newcommand{\D}{\cat{D}}
\newcommand{\K}{\cat{K}}
\newcommand{\op}{\fs{op}}
\newcommand{\id}{\fs{id}}
\newcommand{\Id}{\fs{Id}}
\newcommand{\Set}{\fs{Set}}
\newcommand{\sSet}{\fs{sSet}}
\newcommand{\sCat}{\fs{sCat}}
\newcommand{\sfCat}{\fs{sfCat}}
\newcommand{\colim}{\fs{colim}}

\newcommand{\I}{\fs{I}}
\newcommand{\J}{\fs{J}}
\newcommand{\we}{\mathcal{W}}
\newcommand{\class}[2]{#1\text{-}\mathrm{#2}}
\newcommand{\Icell}[1][\I]{\class{#1}{cell}}
\newcommand{\Icof}[1][\I]{\class{#1}{cof}}
\newcommand{\Iinj}[1][\I]{\class{#1}{inj}}
\newcommand{\Jcell}[1][]{\Icell[\J#1]}
\newcommand{\Jcof}[1][]{\Icof[\J#1]}
\newcommand{\Jinj}[1][]{\Iinj[\J#1]}

\numberwithin{figure}{section}

\newcommand{\ct}{%
  \mathchoice{\mathbin{\raisebox{0.25ex}{$\displaystyle\centerdot$}}}%
             {\mathbin{\raisebox{0.25ex}{$\centerdot$}}}%
             {\mathbin{\raisebox{0.25ex}{$\scriptstyle\,\centerdot\,$}}}%
             {\mathbin{\raisebox{0.25ex}{$\scriptscriptstyle\,\centerdot\,$}}}
}

\newcommand{\pb}[1][dr]{\save*!/#1-1.2pc/#1:(-1,1)@^{|-}\restore}
\newcommand{\po}[1][dr]{\save*!/#1+1.2pc/#1:(1,-1)@^{|-}\restore}

\begin{document}

\title{Title}

\author{Valery Isaev}

\begin{abstract}
Abstract
\end{abstract}

\maketitle

\section{Introduction}

\section{Simplicial categories with fibrations}

In this section, we will define a model structure on the category of simplicial categories with fibrations and prove that it is Quillen equivalent to the model category of simplicial categories.

Let us establish some notation and terminology.
We will denote by $\sSet$ and $\sCat$ the categories of simplicial sets and simplicial categories, respectively.
We will denote by $N : \sCat \to \sSet$ the simplicial nerve functor.
A simplicial category is \emph{fibrant} if its simplicial $\Hom$-sets are Kan complexes.
A map of a simplicial category is an \emph{equivalence} if it becomes an isomorphism in the homotopy category.

We will say that a map $g : Y \to Z$ in a simplicial category $\C$ is a \emph{basic fibration} if, for every object $X$, the map of simplicial sets $g \circ - : \Hom(X,Y) \to \Hom(X,Z)$ is a Kan fibration.
For every object $Z$ of $\C$, we define a simplicial category $\C/Z$ as follows.
Its objects are basic fibrations $f : X \to Z$ and the simplicial set $\Hom_{\C/Z}(X,Y)$ of morphisms between $f : X \to Z$ and $g : Y \to Z$ is the following pullback:
\[ \xymatrix{ \Hom_{\C/Z}(X,Y) \ar[r] \ar[d] & \Hom_\C(X,Y) \ar[d]^{g \circ -} \\
              1 \ar[r]_-f                    & \Hom_\C(X,Z)
            } \]
The composition and identity maps are defined in the obvious way.
Since $g \circ -$ is a Kan fibration, $\Hom_{\C/Z}(X,Y)$ is a Kan complex, so $\C/Z$ is a fibrant simplicial category.
If $\C$ has enough basic fibrations, then $N(\C/Z)$ is equivalent to $N(\C)/Z$:

\begin{prop}[over-cat]
Let $\C$ be a fibrant simplicial category.
Suppose that every map of $\C$ factors into an equivalence followed by a basic fibration.
Then $N(\C/Z)$ is equivalent to $N(\C)/Z$ for every object $Z$ of $\C$.
\end{prop}
\begin{proof}
The proof of \cite[Lemma~6.1.3.13]{lurie-topos} applies in this setting.
\end{proof}

A \emph{simplicial category with fibrations} is a simplicial category which is equipped with a distinguished class of maps, called fibrations.
A morphism of simplicial categories with fibraions is a morphism of simplicial categories preserving fibrations.
The category of small simplicial categories with fibrations is finitely locally presentable since it is the category of models of a finitary essentially algebraic theory.
We will denote this category by $\sfCat$.

We will construct a left semi-model structure on the category of small simplicial categories with fibrations.
We will use the notation from \cite{hovey}.
We will work only with combinatorial (left) semi-model categories.
Thus, the following proposition will be useful:

\begin{prop}[semi-model-cat]
Let $\C$ be a category and let $\we$ be a class of morphisms of $\C$.
Then there is a combinatorial semi-model structure on $\C$ with $\we$ as a class of weak equivalences if and only if there are sets $\I$ and $\J$ of morphisms of $\C$ such that the following conditions hold:
\begin{enumerate}
\item $\we$ satisfies 2-out-of-3 and is closed under retracts.
\item $\Iinj \subseteq \we$.
\item $\J \subseteq \Icof$ and every map in $\Jcell$ with a domain which is an $\Icell$ complex belongs to $\we$.
\item $\Jinj \cap \we \subseteq \Iinj$.
\end{enumerate}
The class of cofibrations in this model category is $\Icof$ and the class of fibrations is $\Jinj$.
\end{prop}

This proposition implies that semi-model structures can be transferred:
\begin{prop}[semimodel-transfer]
Let $F : \C \to \D$ be functor between locally presentable categories with a right adjoint $U : \D \to \C$.
Suppose that there is a semi-model structure on $\C$ with $\we$ as the class of weak equivalences, $\I$ as a set of generating cofibrations, and $\J$ as a set of generating trivial cofibrations.
Then there is a semi-model structure on $\D$ with $U^{-1}(\we)$ as the class of weak equivalences, $F(\I)$ as a set of generating cofibrations, and $F(\J)$ as a set of generating trivial cofibrations
if and only if, for every map $f$ in $\Icell[F(\J)]$ with a domain which is an $\Icell[F(\I)]$ complex, the map $U(f)$ is a weak equivalence.
\end{prop}

Recall that the category of small simplicial categories is a combinatorial model category \cite[Proposition~A.3.2.4]{lurie-topos}.
The forgetful functor $U : \sfCat \to \sCat$ has both left and right adjoints.
Moreover, $U \circ (-)^\flat$ is isomorphic to $\Id$, where $(-)^\flat$ is the left adjoint of $U$.
It follows that the model structure on $\sCat$ can be transferred to $\sfCat$.
Moreover, $(-)^\flat \dashv U$ is a Quillen equivalence between $\sCat$ and $\sfCat$.

Let $\C$ be a combinatorial semi-model category with $\I$ and $\J$ as sets of generating cofibrations and generating trivial cofibrations, respectively.
Let $S$ be a set of maps of $\C$ such that maps in $\Icell[(\J \cup S)]$ with domains in $\Icell[(\I \cup S)]$ are weak equivalences.
Then \rprop{semi-model-cat} implies that $\I \cup S$ and $\J \cup S$ are sets of generating cofibrations and generating trivial cofibrations, respectively, for a semi-model structure on $\C$ with the same class of weak equivalences.
Moreover, the identity functor is a Quillen equivalence between the original model structure and the new one.

Let $[0]$ be the terminal simplicial category and let $[1]$ be the walking arrow considered as simplicial category with discrete simplicial $\Hom$-sets.
We will denote the domain object of $[1]$ by $b$ and its codomain object by $c$.
Let $[1]_f$ be the simplicial category with fibrations which has two objects and one non-identity map which is a fibration.
We will denote by $\gamma : [0] \to [1]_f$ the morphism that maps the unique object of $[0]$ to the codomain of $[1]_f$.

Let $[1]_e$ be any simplicial category with two objects $a$ and $b$ such that the map $[a,b] : [0] \amalg [0] \to [1]_e$ is a cofibration and $[1]_e \to [0]$ is a weak equivalence.
Consider the simplicial category with fibrations $[1]_e \amalg_{[0]} [1]_f$, where both morphisms $[0] \to [1]_f$ and $[0] \to [1]_e$ map the unique object of $[0]$ to $b$.
Let $\varphi : [1] \to [1]_e \amalg_{[0]} [1]_f$ be any morphism such that $\varphi(b) = a$ and $\varphi(c) = c$.
If $Z$ is a simplicial category with fibrations such that the underlying simplicial category is fibrant, then $Z$ has the right lifting property with respect to $\varphi$ if and only if every map factors into an equivalence followed by a fibration.

If $V$ is a simplicial set and $U$ is a simplicial subset of $V$, then we will denote by $[2]_{U,V}$ the simplicial category with fibrations that has three objects $0$, $1$, and $2$ such that
\[ \Hom_{[2]_{U,V}}(x,y) =
  \begin{cases}
    \varnothing & \quad \text{if } x > y \\
    \Delta^0    & \quad \text{if } x = y \\
    U           & \quad \text{if } x = 0, y = 1 \\
    \Delta^0    & \quad \text{if } x = 1, y = 2 \\
    V           & \quad \text{if } x = 0, y = 2
  \end{cases}
\]
The composition is defined in the obvious way.
The only fibration in $[2]_{U,V}$ is the unique map between $1$ and $2$.
Then we have the obvious maps $\lambda^n_k : [2]_{\Lambda^n_k,\Delta^n} \to [2]_{\Delta^n,\Delta^n}$.
A simplicial category with fibrations $Z$ has the right lifting property with respect to the maps $\lambda^n_k$ if and only if every fibration in $Z$ is a basic fibration.

\begin{prop}
There is a semi-model structure on the category of small simplicial categories with fibrations satisfying the following properties.
Weak equivalences are precisely the maps that are weak equivalences of the underlying simplicial categories.
Generating cofibrations are the generating cofibrations of simplicial categories together with the maps $\lambda^n_k$ and the map $\gamma$.
A simplicial category with fibrations $Z$ is fibrant if and only if the following conditions hold:
\begin{enumerate}
\item \label{it:fib-a} The underlying simplicial category of $Z$ is fibrant.
\item \label{it:fib-b} Every map in $Z$ factors into an equivalence followed by a fibration.
\item \label{it:fib-c} Every fibration in $Z$ is a basic fibration.
\end{enumerate}
Every fibration $p : \Hom(x,y)$ in a cofibrant simplicial category with fibrations is an epimorphism (that is, the map $- \circ p : \Hom(y,b) \to \Hom(x,b)$ is a monomorphism of simplicial sets for every object $b$).
The forgetful functor $U : \sfCat \to \sCat$ is a right Quillen functor and is a part of a Quillen equivalence.
\end{prop}
\begin{proof}
Let $S = \{ \varphi \} \cup \{ \lambda^n_k \mid k \leq n, n \in \mathbb{N} \}$.
We will use $\I^\flat \cup S$ as a set of generating cofibrations.
Let us show that $\I^\flat \cup S$ generate the same class of cofibrations as $\I^\flat \cup S'$, where $S' = \{ \gamma \} \cup \{ \lambda^n_k \mid k \leq n, n \in \mathbb{N} \}$.
First, the map $\gamma$ is a cofibration since it is a retract of the cofibration $[0] \xrightarrow{\gamma} [1]_f \to [1]_e \amalg_{[0]} [1]_f$.
Let us show that the map $\varphi$ belongs to $\Icof[(\I^\flat \cup S')]$.
Consider the simplicial category with fibrations $C = ([1]_e \amalg_{[0]} [1]_f) \amalg_{[1]} D^2$,
where $[1] \to [1]_e \amalg_{[0]} [1]_f$ maps the non-identity map of $[1]$ to the equivalence and $D^2$ is the simplicial category with two objects $x$ and $y$,
two non-identity morphisms between $x$ and $y$ and one $1$-simplex between them.
Then the morphism $[1] \to C$ that maps $b$ to $a$ and $c$ to $c$ belongs to $\Icof[(\I^\flat \cup S')]$ and the morphism $[1] \to [1]_e \amalg_{[0]} [1]_f$ is a retract of $[1] \to C$.

A simplicial category with fibrations satisfies conditions \eqref{it:fib-a}-\eqref{it:fib-c} if and only if it has the right lifting property with respect to $\J^\flat \cup S$.
Thus, we just need to show that maps in $\Icell[(\J^\flat \cup S)]$ with domains in $\Icell[(\I^\flat \cup S)]$ are weak equivalences (of simplicial categories).
Since weak equivalences are closed under transfinite composition, it is enough to prove that pushouts of maps in $S$ with domains in $\Icell[(\I^\flat \cup S)]$ are weak equivalences.

Let $h : \Hom_Z(a,c)$ be a map in a simplicial category with fibrations.
Let $\varphi' : Z \to Z'$ be the pushout of $\varphi$ along the map $[1] \to Z$ corresponding to $h$.
We need to show that $\varphi'$ is a weak equivalence.
The only object of $Z'$ that does not belong to $Z$ is $b$ and it is equivalent to $a$.
Thus, we just need to show that, for every pair of objects $x$ and $y$ of $Z$, the map $\Hom_Z(x,y) \to \Hom_{Z'}(\varphi'(x),\varphi'(y))$ is a weak equivalence of simplicial sets.
Note that $\varphi$ can be factored into maps $[1] \xrightarrow{\varphi_1} [1] \amalg_{[0]} [1]_f \xrightarrow{\varphi_2} [1]_e \amalg_{[0]} [1]_f$.
Let $\varphi_1'$ and $\varphi_2'$ be pushouts of $\varphi_1$ and $\varphi_2$, respectively.
The map $\Hom_Z(x,y) \to \Hom(\varphi_1'(x),\varphi_1'(y))$ is an isomorphism.
The proof that the map $\Hom(\varphi_1'(x),\varphi_1'(y)) \to \Hom(\varphi_2'(x),\varphi_2'(y))$ is a weak equivalence is the same as in \cite[Proposition~A.3.2.4]{lurie-topos}.

Now, let us prove that every fibration $p : \Hom(x,y)$ in a cofibrant simplicial category with fibrations is an epimorphism.
It is enough to prove this for $\Icell[(\I^\flat \cup S)]$ complexes since every cofibrant simplicial category with fibrations is a retract of such a complex.
Let $Z$ be an $\Icell[(\I^\flat \cup S)]$ complex.
Then $Z = \colim_{\alpha < \lambda} Z_\alpha$, where $Z_\alpha \to Z_{\alpha + 1}$ is a pushout of a map in $\I^\flat \cup S$, $Z_0 = 0$, and $Z_\kappa = \colim_{\alpha < \kappa} Z_\alpha$.
We will prove by induction on $\lambda$ that, for every fibration $p : \Hom_Z(x,y)$ and every object $b$, the map $- \circ p : \Hom_Z(y,b) \to \Hom_Z(x,b)$ is a monomorphism.
This is obvious for $\lambda = 0$.
For limit ordinals, this follows from induction hypothesis.
If $\lambda = \alpha+1$, then $Z_\alpha \to Z_{\alpha+1}$ is a pushout of one of the generating cofibrations.
If it is a cofibration of the form $f^\flat$, then we can use the description of simplicial $\Hom$-sets of $Z_{\alpha+1}$ given in \cite[Proposition~A.3.2.4]{lurie-topos} to conclude that $- \circ p$ is a monomorphism.
If it is $\varphi$, then the description of simplicial $\Hom$-sets given above implies that $- \circ p$ is an isomorphism.
If it is $\lambda^n_k$, then simplicial $\Hom$-sets can be explicitly described in a similar way.
This description also implies $- \circ p$ is a monomorphism.

Finally, let $Z$ be a $\Icell[(\I^\flat \cup S)]$ complex and let $\lambda' : Z \to Z'$ be a pushout of a map $\lambda^n_k : [2]_{\Lambda^n_k,\Delta^n} \to [2]_{\Delta^n,\Delta^n}$.
Let $z$ be the image of $0$ and let $p : \Hom(x,y)$ be the image of the unique fibration in $[2]_{\Lambda^n_k,\Delta^n}$.
For every object $c$, we define a map of simplicial sets $f(c)$ as the pushout-product of $\Lambda^n_k \to \Delta^n$ and $- \circ p : \Hom_Z(y,c) \to \Hom(x,c)$.
Since $p$ is a fibration in a cofibrant simplicial category with fibrations, $- \circ p$ is a cofibration and $f(c)$ is a trivial cofibration.

Let $a$ and $b$ be a pair of objects of $Z$.
We define a sequence of trivial cofibrations of simplicial sets $f_i : N_i \to M_i$ as follows:
$f_0 = \id_{\Hom_Z(a,z)} \times f(b)$ and $f_{i+1}$ is the pushout-product of $f_i$ and $f(z)$.
We can also define maps $g_i : N_i \to \Hom_Z(a,b)$ such that $\varphi'_{a,b} : \Hom_Z(a,b) \to \Hom_{Z'}(\varphi'(a),\varphi'(b))$ is the pushout of $\coprod_{i \in \mathbb{N}} f_i$ along $[g_i]_{i \in \mathbb{N}}$.
Since $f_i$ are trivial cofibrations, $\varphi'_{a,b}$ is also a trivial cofibration.
Since $\varphi' : Z \to Z'$ is surjective on objects, this implies that it is a weak equivalence of simplicial categories.
\end{proof}

\section{Categories of marked objects}

Categories of marked objects were defined in \cite{marked-obj}.
Here, we recall their definition and basic properties.

\begin{defn}[marked-obj]
Let $\C$ be a category, let $\mathcal{K}$ be a small category, and let $\mathcal{F} : \mathcal{K} \to \C$ be a functor.
A \emph{$\K$-marked object} of $\C$ is a pair $(X,\mathcal{E})$, where $X$ is an object of $\C$ and $\mathcal{E} : \mathcal{K}^{op} \to \Set$ is a subfunctor of $\Hom(\mathcal{F}(-),X)$.
Morphisms $f : \mathcal{F}(K) \to X$ that belong to $\mathcal{E}$ will be called \emph{marked}.
A morphism of marked objects is a morphism of the underlying objects that preserves marked morphisms.
The category of marked objects will be denoted by $\C^m$.
\end{defn}

The forgetful functor $U : \C^m \to \C$ has a left adjoint $(-)^\flat : \C \to \C^m$ and a right adjoint $(-)^\sharp : \C \to \C^m$.
For every $X \in \C$, $X^\flat$ is the marked object in which no morphisms are marked (that is, $X^\flat = (X,\varnothing)$),
and $X^\sharp$ is the marked object in which all morphisms are marked (that is, $X^\sharp = (X,\coprod_{K \in \mathcal{K}} \Hom(K,X))$).
Objects of the form $X^\flat$ and of the form $X^\sharp$ will be called \emph{flat} and \emph{sharp} respectively.

\begin{thm}[mark-main]
Let $\C$ be a combinatorial model category and let $\mathcal{F} : \mathcal{K} \to \C$ be a functor such that, for every $K \in \mathcal{K}$, the object $\mathcal{F}(K)$ is cofibrant.
Then $\C^m$ is a combinatorial model category in which a map is a cofibration if and only if the underlying map is a cofibration in $\C$.
Both adjoint pairs $(-)^\flat \dashv U$ and $U \dashv (-)^\sharp$ are Quillen pairs.
If $\C$ is left proper, then so is $\C^m$.
A marked object $X$ is fibrant in $\C^m$ if and only if the following conditions hold:
\begin{itemize}
\item The underlying object $U(X)$ is fibrant in $\C$
\item Marked maps in $X$ are stable under homotopy (that is, if two maps $K \to U(X)$ are homotopic and one of them is marked, then so is the other).
\end{itemize}
\end{thm}

\begin{remark}
This theorem also holds for left semi-model categories with one additional assumption that the domains of generating cofibrations are cofibrant.
This assumption is required for Jeff Smith's theorem for left semi-model categories.
\end{remark}

The model category of marked objects has very explicit description of 

\section{Finite limits in categories with fibrations}

Let $\C$ be a simplicial category.
An object $1$ of $\C$ is \emph{terminal} if the simplicial set $\Hom(c,1)$ is contractible for every object of $\C$.
If $1$ is a terminal object of $\C$, then it is also a terminal objects of $N(\C)$.
A \emph{product} of objects $a$ and $b$ of $\C$ is an object $a \times b$ together with maps $\pi_1 : \Hom(a \times b, a)$ and $\pi_2 : \Hom(a \times b, b)$ such that the following map is an equivalence for every object $c$:
\[ \Hom(c, a \times b) \to \Hom(c,a) \times \Hom(c,b) \]
If $a \times b$ is a product of $a$ and $b$, then it is a product of these objects in $N(\C)$.
Now, \rprop{over-cat} implies that a pullback of two basic fibrations in $\C$ is a pullback in $N(\C)$.

\bibliographystyle{amsplain}
\bibliography{ref}

\end{document}
