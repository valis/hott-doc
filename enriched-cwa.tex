\documentclass[reqno]{amsart}

\usepackage{amssymb}
\usepackage{hyperref}
\usepackage{mathtools}
\usepackage[all]{xy}
\usepackage{verbatim}
\usepackage{ifthen}
\usepackage{xargs}
\usepackage{bussproofs}
\usepackage{turnstile}
\usepackage{etex}

\hypersetup{colorlinks=true,linkcolor=blue}

\newcommand{\axlabel}[1]{(#1) \phantomsection \label{ax:#1}}
\newcommand{\axtag}[1]{\label{ax:#1} \tag{#1}}
\newcommand{\axref}[1]{(\hyperref[ax:#1]{#1})}

\newcommand{\newref}[4][]{
\ifthenelse{\equal{#1}{}}{\newtheorem{h#2}[hthm]{#4}}{\newtheorem{h#2}{#4}[#1]}
\expandafter\newcommand\csname r#2\endcsname[1]{#3~\ref{#2:##1}}
\expandafter\newcommand\csname R#2\endcsname[1]{#4~\ref{#2:##1}}
\expandafter\newcommand\csname n#2\endcsname[1]{\ref{#2:##1}}
\newenvironmentx{#2}[2][1=,2=]{
\ifthenelse{\equal{##2}{}}{\begin{h#2}}{\begin{h#2}[##2]}
\ifthenelse{\equal{##1}{}}{}{\label{#2:##1}}
}{\end{h#2}}
}

\newref[section]{thm}{Theorem}{Theorem}
\newref{lem}{Lemma}{Lemma}
\newref{prop}{Proposition}{Proposition}
\newref{cor}{Corollary}{Corollary}
\newref{cond}{Condition}{Condition}

\theoremstyle{definition}
\newref{defn}{Definition}{Definition}
\newref{example}{Example}{Example}

\theoremstyle{remark}
\newref{remark}{Remark}{Remark}

\newcommand{\fs}[1]{\mathrm{#1}}
\newcommand{\Hom}{\fs{Hom}}
\newcommand{\cat}[1]{\mathcal{#1}}
\newcommand{\op}{\fs{op}}
\newcommand{\id}{\fs{id}}
\newcommand{\Set}{\fs{Set}}
\newcommand{\sSet}{\fs{sSet}}
\newcommand{\cSet}{\fs{cSet}}
\newcommand{\sCat}{\fs{sCat}}
\newcommand{\cCat}{\fs{cCat}}

\numberwithin{figure}{section}

\newcommand{\ct}{%
  \mathchoice{\mathbin{\raisebox{0.25ex}{$\displaystyle\centerdot$}}}%
             {\mathbin{\raisebox{0.25ex}{$\centerdot$}}}%
             {\mathbin{\raisebox{0.25ex}{$\scriptstyle\,\centerdot\,$}}}%
             {\mathbin{\raisebox{0.25ex}{$\scriptscriptstyle\,\centerdot\,$}}}
}

\newcommand{\pb}[1][dr]{\save*!/#1-1.2pc/#1:(-1,1)@^{|-}\restore}
\newcommand{\po}[1][dr]{\save*!/#1+1.2pc/#1:(1,-1)@^{|-}\restore}

\begin{document}

\title{Title}

\author{Valery Isaev}

\begin{abstract}
Abstract
\end{abstract}

\maketitle

\section{Introduction}

\section{Enriched categories with fibrations}

Let $(\cat{V},\otimes,e)$ be a monoidal model category.
Then we will denote by $[0]_\cat{V}$ the $\cat{V}$-enriched category that has a single object $X$ such that $\Hom_{[0]_\cat{V}}(X,X) = e$.
If $V$ is an object of $\cat{V}$, we will denote by $[1]_V$ the $\cat{V}$-enriched category with two objects $X$ and $Y$ and the following maps:
\begin{align*}
\Hom_{[1]_V}(X,X) & = e \\
\Hom_{[1]_V}(Y,Y) & = e \\
\Hom_{[1]_V}(X,Y) & = V \\
\Hom_{[1]_V}(Y,X) & = 0
\end{align*}

We will say $\cat{V}$ is \emph{good} if the category of $\cat{V}$-enriched categories is a model category with the following classes of cofibrations and weak equivalences:
\begin{itemize}
\item Cofibrations are generated by the class consisting of the map $0 \to [0]_\cat{V}$ and maps $[1]_i : [1]_U \to [1]_V$ for every cofibration $i : U \to V$.
\item A map $f : C \to D$ is a weak equivalence if and only if, for all objects $X$ and $Y$ of $C$, the map $f : \Hom_C(X,Y) \to \Hom_D(f(X),f(Y)$ is a weak equivalence
and, for every object $Z$ of $D$, there is an object $Z'$ of $C$ such that $f(Z')$ is isomorphic to $Z$ in the homotopy category of $D$.
\end{itemize}

Sufficient conditions for $\cat{V}$ to be good are given in \cite[Proposition~A.3.2.4]{lurie-topos}.
Categories of simplicial and cubical sets are both good.
A \emph{simplicial (resp., cubical) category} is a category enriched in simplicial (resp., cubical) sets.
The categories of simplicial and cubical sets will be denoted by $\sSet$ and $\cSet$, respectively.
The categories of simplicial and cubical categories will be denoted by $\sCat$ and $\cCat$, respectively.

An \emph{interval object} in $\cat{V}$ is an object $I$ together with a cofibration $e \amalg e \to I$ and a weak equivalence $I \to e$ such that their composite equals $[\id_e,\id_e]$.
Categories of simplicial and cubical sets both have an interval object ($\Delta^1$ and $\square^1$, respectively).
For every monoidal category $\cat{V}$ with an interval object $I$, there is a unique monoidal functor $F_I : \cSet \to \cat{V}$ sending the interval object $\square^1$ to $I$ and commuting with colimits.
If $\cat{V}$ is a monoidal model category, this functor is a left Quillen functor.
In particular, there is a monoidal left Quillen functor $F_{\Delta^1} : \cSet \to \sSet$ sending $\square^1$ to $\Delta^1$.
This functor is a Quillen equivalence. % TODO: Find a reference for this.
It follows that $F_{\Delta^1}$ is also a Quillen equivalence \cite[Remark~A.3.2.6]{lurie-topos}.

\bibliographystyle{amsplain}
\bibliography{ref}

\end{document}
