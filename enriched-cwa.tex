\documentclass[reqno]{amsart}

\usepackage{amssymb}
\usepackage{hyperref}
\usepackage{mathtools}
\usepackage[all]{xy}
\usepackage{verbatim}
\usepackage{ifthen}
\usepackage{xargs}
\usepackage{bussproofs}
\usepackage{turnstile}
\usepackage{etex}

\hypersetup{colorlinks=true,linkcolor=blue}

\newcommand{\axlabel}[1]{(#1) \phantomsection \label{ax:#1}}
\newcommand{\axtag}[1]{\label{ax:#1} \tag{#1}}
\newcommand{\axref}[1]{(\hyperref[ax:#1]{#1})}

\newcommand{\newref}[4][]{
\ifthenelse{\equal{#1}{}}{\newtheorem{h#2}[hthm]{#4}}{\newtheorem{h#2}{#4}[#1]}
\expandafter\newcommand\csname r#2\endcsname[1]{#3~\ref{#2:##1}}
\expandafter\newcommand\csname R#2\endcsname[1]{#4~\ref{#2:##1}}
\expandafter\newcommand\csname n#2\endcsname[1]{\ref{#2:##1}}
\newenvironmentx{#2}[2][1=,2=]{
\ifthenelse{\equal{##2}{}}{\begin{h#2}}{\begin{h#2}[##2]}
\ifthenelse{\equal{##1}{}}{}{\label{#2:##1}}
}{\end{h#2}}
}

\newref[section]{thm}{Theorem}{Theorem}
\newref{lem}{Lemma}{Lemma}
\newref{prop}{Proposition}{Proposition}
\newref{cor}{Corollary}{Corollary}
\newref{cond}{Condition}{Condition}

\theoremstyle{definition}
\newref{defn}{Definition}{Definition}
\newref{example}{Example}{Example}

\theoremstyle{remark}
\newref{remark}{Remark}{Remark}

\newcommand{\fs}[1]{\mathrm{#1}}
\newcommand{\Hom}{\fs{Hom}}
\newcommand{\cat}[1]{\mathcal{#1}}
\newcommand{\C}{\cat{C}}
\newcommand{\op}{\fs{op}}
\newcommand{\id}{\fs{id}}
\newcommand{\Id}{\fs{Id}}
\newcommand{\Set}{\fs{Set}}
\newcommand{\sSet}{\fs{sSet}}
\newcommand{\sCat}{\fs{sCat}}
\newcommand{\sfCat}{\fs{sfCat}}

\newcommand{\I}{\fs{I}}
\newcommand{\J}{\fs{J}}
\newcommand{\class}[2]{#1\text{-}\mathrm{#2}}
\newcommand{\Icell}[1][\I]{\class{#1}{cell}}

\numberwithin{figure}{section}

\newcommand{\ct}{%
  \mathchoice{\mathbin{\raisebox{0.25ex}{$\displaystyle\centerdot$}}}%
             {\mathbin{\raisebox{0.25ex}{$\centerdot$}}}%
             {\mathbin{\raisebox{0.25ex}{$\scriptstyle\,\centerdot\,$}}}%
             {\mathbin{\raisebox{0.25ex}{$\scriptscriptstyle\,\centerdot\,$}}}
}

\newcommand{\pb}[1][dr]{\save*!/#1-1.2pc/#1:(-1,1)@^{|-}\restore}
\newcommand{\po}[1][dr]{\save*!/#1+1.2pc/#1:(1,-1)@^{|-}\restore}

\begin{document}

\title{Title}

\author{Valery Isaev}

\begin{abstract}
Abstract
\end{abstract}

\maketitle

\section{Introduction}

\section{Simplicial categories with fibrations}

In this section, we will define a model structure on the category of simplicial categories with fibrations and prove that it is Quillen equivalent to the model category of simplicial categories.

Let us establish some notation and terminology.
We will denote by $\sSet$ and $\sCat$ the categories of simplicial sets and simplicial categories, respectively.
We will denote by $N : \sCat \to \sSet$ the simplicial nerve functor.
A simplicial category is \emph{fibrant} if its simplicial $\Hom$-sets are Kan complexes.
A map of a simplicial category is an \emph{equivalence} if it becomes an isomorphism in the homotopy category.

We will say that a map $g : Y \to Z$ in a simplicial category $\C$ is a \emph{basic fibration} if, for every object $X$, the map of simplicial sets $g \circ - : \Hom(X,Y) \to \Hom(X,Z)$ is a Kan fibration.
For every object $Z$ of $\C$, we define a simplicial category $\C/Z$ as follows.
Its objects are basic fibrations $f : X \to Z$ and the simplicial set $\Hom_{\C/Z}(X,Y)$ of morphisms between $f : X \to Z$ and $g : Y \to Z$ is the following pullback:
\[ \xymatrix{ \Hom_{\C/Z}(X,Y) \ar[r] \ar[d] & \Hom_\C(X,Y) \ar[d]^{g \circ -} \\
              1 \ar[r]_-f                    & \Hom_\C(X,Z)
            } \]
The composition and identity maps are defined in the obvious way.
Since $g \circ -$ is a Kan fibration, $\Hom_{\C/Z}(X,Y)$ is a Kan complex, so $\C/Z$ is a fibrant simplicial category.
If $\C$ has enough basic fibrations, then $N(\C/Z)$ is equivalent to $N(\C)/Z$:

\begin{prop}
Let $\C$ be a fibrant simplicial category.
Suppose that every map of $\C$ factors into an equivalence followed by a basic fibration.
Then $N(\C/Z)$ is equivalent to $N(\C)/Z$ for every object $Z$ of $\C$.
\end{prop}
\begin{proof}
The proof of \cite[Lemma~6.1.3.13]{lurie-topos} applies in this setting.
\end{proof}

A \emph{simplicial category with fibrations} is a simplicial category which is equipped with a distinguished class of maps, called fibrations.
A morphism of simplicial categories with fibraions is a morphism of simplicial categories preserving fibrations.
The category of small simplicial categories with fibrations is finitely locally presentable since it is the category of models of a finitary essentially algebraic theory.
We will denote this category by $\sfCat$.

Recall that the category of small simplicial categories is a combinatorial model category \cite[Proposition~A.3.2.4]{lurie-topos}.
The forgetful functor $U : \sfCat \to \sCat$ has both left and right adjoints.
Moreover, $U \circ F$ is isomorphic to $\Id$, where $F$ is the left adjoint of $U$.
It follows that the model structure on $\sCat$ can be transferred to $\sfCat$.
Moreover, $F \dashv U$ is a Quillen equivalence between $\sCat$ and $\sfCat$.

Let $\C$ be a combinatorial model category with $\I$ and $\J$ as sets of generating cofibrations and generating trivial cofibrations, respectively.
Let $S$ be a set of maps of $\C$ such that maps in $\Icell[(\J \cup S)]$ are weak equivalences.
Then \cite[Theorem~2.1.19]{hovey} implies that $\I \cup S$ and $\J \cup S$ are sets of generating cofibrations and generating trivial cofibrations, respectively, for a model structure on $\C$ with the same class of weak equivalences.
Moreover, the identity functor is a Quillen equivalence between the original model structure and the new one.

Let $[0]$ be the terminal simplicial category and let $[1]$ be the walking arrow considered as simplicial category with discrete simplicial $\Hom$-sets.
We will denote the domain object of $[1]$ by $B$ and its codomain object by $C$.
Let $[1]_e$ be any simplicial category with two objects $A$ and $B$ such that the map $[A,B] : [0] \amalg [0] \to [1]_e$ is a cofibration and $[1]_e \to [0]$ is a weak equivalence.
Consider the simplicial category $[1]_i \amalg_{[0]} [1]$, where both morphisms $[0] \to [1]$ and $[0] \to [1]_e$ map the unique object of $[0]$ to $B$.
We will make this simplicial category into a simplicial category with fibrations by declaring the image of $[1]$ to be the only fibration.
Let $f : [1] \to [1]_i \amalg_{[0]} [1]$ be the unique morphism such that $f(B) = A$ and $f(C) = C$.
If $X$ is a simplicial category with fibrations such that the underlying simplicial category is fibrant, then $X$ has the right lifting property with respect to $f$ if and only if every map factors into an equivalence followed by a fibration.

\bibliographystyle{amsplain}
\bibliography{ref}

\end{document}
