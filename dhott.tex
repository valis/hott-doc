\documentclass[reqno]{amsart}

\usepackage{amssymb}
\usepackage{hyperref}
\usepackage{mathtools}
\usepackage[all]{xy}
\usepackage{verbatim}
\usepackage{ifthen}
\usepackage{xargs}
\usepackage{bussproofs}
\usepackage{turnstile}
\usepackage{etex}

\hypersetup{colorlinks=true,linkcolor=blue}

\newcommand{\axlabel}[1]{(#1) \phantomsection \label{ax:#1}}
\newcommand{\axtag}[1]{\label{ax:#1} \tag{#1}}
\newcommand{\axref}[1]{(\hyperref[ax:#1]{#1})}

\newcommand{\newref}[4][]{
\ifthenelse{\equal{#1}{}}{\newtheorem{h#2}[hthm]{#4}}{\newtheorem{h#2}{#4}[#1]}
\expandafter\newcommand\csname r#2\endcsname[1]{#3~\ref{#2:##1}}
\expandafter\newcommand\csname R#2\endcsname[1]{#4~\ref{#2:##1}}
\expandafter\newcommand\csname n#2\endcsname[1]{\ref{#2:##1}}
\newenvironmentx{#2}[2][1=,2=]{
\ifthenelse{\equal{##2}{}}{\begin{h#2}}{\begin{h#2}[##2]}
\ifthenelse{\equal{##1}{}}{}{\label{#2:##1}}
}{\end{h#2}}
}

\newref[section]{thm}{Theorem}{Theorem}
\newref{lem}{Lemma}{Lemma}
\newref{prop}{Proposition}{Proposition}
\newref{cor}{Corollary}{Corollary}
\newref{cond}{Condition}{Condition}

\theoremstyle{definition}
\newref{defn}{Definition}{Definition}
\newref{example}{Example}{Example}

\theoremstyle{remark}
\newref{remark}{Remark}{Remark}

\newcommand{\type}{}
\newcommand{\ob}{}
\newcommand{\term}{1}
\newcommand{\unit}{()}

\newcommand{\fs}[1]{\mathrm{#1}}
\newcommand{\cat}[1]{\mathbf{#1}}
\newcommand{\scat}[1]{\mathcal{#1}}
\newcommand{\sSet}{\cat{sSet}}
\newcommand{\Hom}{\fs{Hom}}

\numberwithin{figure}{section}

\newcommand{\ct}{%
  \mathchoice{\mathbin{\raisebox{0.25ex}{$\displaystyle\centerdot$}}}%
             {\mathbin{\raisebox{0.25ex}{$\centerdot$}}}%
             {\mathbin{\raisebox{0.25ex}{$\scriptstyle\,\centerdot\,$}}}%
             {\mathbin{\raisebox{0.25ex}{$\scriptscriptstyle\,\centerdot\,$}}}
}

\newcommand{\pb}[1][dr]{\save*!/#1-1.2pc/#1:(-1,1)@^{|-}\restore}
\newcommand{\po}[1][dr]{\save*!/#1+1.2pc/#1:(1,-1)@^{|-}\restore}

\begin{document}

\title{An indexed type theory for synthetic $\infty$-categories}

\author{Valery Isaev}

\begin{abstract}
% TODO
\end{abstract}

\maketitle

\section{Introduction}

% TODO

\section{Models of indexed type theories}

In this section, we discuss several examples of models of indexed type theories.
Indexed type theories are defined in \cite{indexed-tt} as certain essentially algebraic theory.
Thus, we have a notion of a model of such a theory.
We will call models of indexed unary (resp., dependent) type theories \emph{contextually indexed categories} (resp., \emph{contextually indexed contextual categories}).
A contextually indexed category is a contextual category $B$ together with a category indexed over $B$, that is a functor $B^\fs{op} \to \cat{Cat}$.
A contextually indexed contextual category is a contextual category $B$ together with a contextual category indexed over $B$, that is a functor $B^\fs{op} \to \cat{ConCat}$.

One class of contextually indexed contextual categories was already defined in \cite{indexed-tt}.
An indexed type theory can be interpreted in an appropriate homotopy type theory.
This implies that there is a forgetful functor $U$ from the category of contextual categories to the category of contextually indexed contextual categories.
If $M$ is a contextual category, then $U(M)$ will be called \emph{the canonical indexing of $M$ over itself}.
The underlying contextual category of $U(M)$ is indeed $M$ itself.

Let $F$ be a functor between underlying categories of contextual categories $\scat{B}'$ and $\scat{B}$.
If $\scat{C}$ is a contextual category over $\scat{B}$, then \emph{change of base} of $\scat{C}$ along $F$ is defined as $\scat{C} \circ F^\fs{op}$ and denoted by $F^*(\scat{C})$.
If $F$ and $G$ are isomorphic functors between categories underlying contextual categories $\scat{B}'$ and $\scat{B}$, then $F^*(\scat{C})$ and $G^*(\scat{C})$ are also isomorphic for every contextuall indexed (contextual) category $\scat{C}$.

The intended models of theories that we will be discussing in this paper can be described as full subcategories of the category $\sSet^{\Delta^\fs{op}}$,
where $\sSet$ is the canonical indexing of the category of simplicial set (considered as a contextual category as described in \cite{kap-lum-voe}) over itself.
Our aim here is to define the model $\sSet^{\Delta^\fs{op}}$.
Since it requires little additional effort, we will define more general contextually indexed category $\scat{C}^\scat{J}$ for every appropriate category $\scat{C}$ and category $\scat{J}$.

\subsection{Change of base}

The local universes construction defined in \cite{local-universes}, for every category $\scat{C}$ with a chosen pullbacks, terminal object, and a class of maps called fibrations,
gives us a contextual category $\scat{C}_!$ whose types over $\Gamma$ are diagrams of the form
\[ \xymatrix{               & E_A \ar@{->>}[d] \\
              \Gamma \ar[r] & V_A
            } \]
where $E_A \to V_A$ is a fibration.
Terms of this type are sections $\Gamma \to E_A$.
For every type $\Gamma \vdash A$, the extended context $\Gamma, x : A$ is the pullback of this diagram.
If $\scat{C}$ satisfies additional conditions listed in \cite[Definition~4.2.1]{local-universes}, then $\scat{C}_!$ models unit types, identity types, $\Sigma$-types, and $\Pi$-types.

If $\scat{B}$ is a category with fibrations, then the underlying category of $\scat{B}_!$ is equivalent to the category $\scat{B}_f$ of fibration objects of $\scat{B}$.
Thus, change of base for (contextual) categories indexed over $\scat{B}_!$ is defined for every functor $\scat{B}'_f \to \scat{B}_f$.
If $F : \scat{B}' \to \scat{B}$ is a functor that preserves fibrations, terminal objects, and pullbacks of fibrations, then there is an obvious contextual functor between contextual categories $F_! : \scat{B}'_! \to \scat{B}_!$.
The underlying functor of $F_!$ is isomorphic to $F_f : \scat{B}'_f \to \scat{B}_f$, the restriction of $F$ to fibrant objects of $\scat{B}'$.

Let $\scat{B}$ be a contextual category, let $\scat{C}$ be a category with fibrations, and let $F : \scat{B} \to \scat{C}$ be a functor.
We showed above that if the image of $F$ consists of fibrant objects, then we can define change of base $F^*(\scat{C}_!)$.
Actually, we can define this contextual category for arbitrary $F$.
This is a straightforward generalization of the local universes model.
Closed indexed types in context $\Gamma$ are diagrams of the form
\[ \xymatrix{                   & E_A \ar@{->>}[d] \\
              F(\Gamma) \ar[r]  & V_A
            } \]
where $E_A \to V_A$ is a fibration.
Contexts, terms, and non-closed types are defined as before.

We are mainly interested in locally small contextually indexed categories since most of the constructions in \cite{indexed-tt} use this property.
In general, local smallness of $\scat{C}$ does not imply local smallness of $F^*(\scat{C})$.
We consider a special case when it does:

\begin{prop}
Let $\scat{B}$ be a contextual category, let $\scat{C}$ be a category with fibrations, and let $F : \scat{B} \to \scat{C}$ be a functor with a right adjoint $G : \scat{C} \to \scat{B}$.
Suppose that, for every fibration $f : A \to B$, the pullback functor $f^* : \scat{C}/B \to \scat{C}/A$ has a right adjoint $\Pi_f : \scat{C}/A \to \scat{C}/B$ and this functor preserves fibrations.
Then $F^*(\scat{C}_!)$ is locally Cartesian closed and has dependent $\Hom$-types.
In particular, it is locally small.
\end{prop}
\begin{proof}
% TODO
\end{proof}

\subsection{Contextually indexed categories of diagrams}

\bibliographystyle{amsplain}
\bibliography{ref}

\end{document}
