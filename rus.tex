\documentclass{amsart}

\usepackage[english,russian]{babel}
\usepackage[utf8]{inputenc}
\usepackage{amssymb}
\usepackage[all]{xy}
\usepackage{verbatim}
\usepackage{ifthen}
\usepackage{xargs}
\usepackage{bussproofs}
\usepackage{type1ec}
\usepackage{stmaryrd}
% \usepackage[T2A]{fontenc}

\providecommand\WarningsAreErrors{false}
\ifthenelse{\equal{\WarningsAreErrors}{true}}{\renewcommand{\GenericWarning}[2]{\GenericError{#1}{#2}{}{This warning has been turned into a fatal error.}}}{}

\newcommand{\newref}[4][]{
\ifthenelse{\equal{#1}{}}{\newtheorem{h#2}[hthm]{#4}}{\newtheorem{h#2}{#4}[#1]}
\expandafter\newcommand\csname r#2\endcsname[1]{\ref{#2:##1}}
\expandafter\newcommand\csname R#2\endcsname[1]{#4~\ref{#2:##1}}
\newenvironmentx{#2}[2][1=,2=]{
\ifthenelse{\equal{##2}{}}{\begin{h#2}}{\begin{h#2}[##2]}
\ifthenelse{\equal{##1}{}}{}{\label{#2:##1}}
}{\end{h#2}}
}

\newref[section]{thm}{теорема}{Теорема}
\newref{lem}{лемма}{Лемма}
\newref{prop}{утверждение}{Утверждение}
\newref{cor}{следствие}{Следствие}

\theoremstyle{definition}
\newref{defn}{definition}{Definition}
\newref{example}{example}{Example}

\theoremstyle{remark}
\newref{remark}{замечание}{Замечание}

\newcommand{\cat}[1]{\mathbf{#1}}
% \newcommand{\C}{\cat{C}}
\newcommand{\bs}{\beta\sigma}
\newcommand{\ebs}{=_{\bs}}
\newcommand{\rbs}{\to_{\bs}}
\newcommand{\bst}{\bs\tau}
\newcommand{\ebst}{=_{\bst}}
\newcommand{\rbst}{\to_{\bst}}
\newcommand{\sSet}{\cat{sSet}}
\renewcommand{\ll}{\llbracket}
\newcommand{\rr}{\rrbracket}

\numberwithin{figure}{section}

\begin{document}

\makeatletter
\def\@settitle{\begin{center}%
    \baselineskip14\p@\relax
    \bfseries
    \@title
  \end{center}%
}

\title{Гомотопическая теория типов с типом интервала}

\author{Валерий Исаев}

% \begin{abstract}
% Abstract
% \end{abstract}

\maketitle

\section{Введение}

\section{Синтаксис}

В данном разделе мы приведем правила вывода для базовой системы.
Позже мы расширим их индуктивными типами данных и записями с условиями.

\centerAlignProof

\begin{table}

\medskip
\begin{center}
\AxiomC{}
\UnaryInfC{$\varnothing \vdash$}
\DisplayProof
\quad
\AxiomC{$\Gamma \vdash A$}
\RightLabel{, $x \notin \Gamma$}
\UnaryInfC{$\Gamma, x : A \vdash$}
\DisplayProof
\quad
\AxiomC{$\Gamma \vdash$}
\RightLabel{, $x : A \in \Gamma$}
\UnaryInfC{$\Gamma \vdash x : A$}
\DisplayProof
\end{center}

\medskip
\begin{center}
\AxiomC{$\Gamma \vdash a : A$}
\AxiomC{$\Gamma \vdash B$}
\RightLabel{, $A =_{\beta \sigma \tau} B$}
\BinaryInfC{$\Gamma \vdash a : B$}
\DisplayProof
\end{center}

\medskip
\begin{center}
\AxiomC{$\Gamma \vdash A$}
\AxiomC{$\Gamma, x : A \vdash B$}
\BinaryInfC{$\Gamma \vdash \Pi (x : A) B$}
\DisplayProof
\end{center}

\medskip
\begin{center}
\AxiomC{$\Gamma, x : A \vdash b : B$}
\UnaryInfC{$\Gamma \vdash \lambda x. b : \Pi (x : A) B$}
\DisplayProof
\quad
\AxiomC{$\Gamma \vdash f : \Pi (x : A) B$}
\AxiomC{$\Gamma \vdash a : A$}
\BinaryInfC{$\Gamma \vdash f\ a : B[x := a]$}
\DisplayProof
\end{center}

\medskip
\begin{center}
\AxiomC{$\Gamma \vdash$}
\UnaryInfC{$\Gamma \vdash I$}
\DisplayProof
\quad
\AxiomC{$\Gamma \vdash$}
\UnaryInfC{$\Gamma \vdash left : I$}
\DisplayProof
\quad
\AxiomC{$\Gamma \vdash$}
\UnaryInfC{$\Gamma \vdash right : I$}
\DisplayProof
% \quad
% \AxiomC{$\Gamma \vdash i : I$}
% \AxiomC{$\Gamma \vdash j : I$}
% \BinaryInfC{$\Gamma \vdash squeeze\ i\ j : I$}
% \DisplayProof
\end{center}

\medskip
\begin{center}
\AxiomC{$\Gamma, x : I \vdash A$}
\AxiomC{$\Gamma \vdash i : I$}
\AxiomC{$\Gamma \vdash a : A[x := i]$}
\AxiomC{$\Gamma \vdash j : I$}
\QuaternaryInfC{$\Gamma \vdash coe_{\lambda x. A}\ i\ a\ j : A[x := j]$}
\DisplayProof
\end{center}

\begin{comment}
\medskip
\begin{center}
\AxiomC{$\Gamma, x : I \vdash A$}
\AxiomC{$\Gamma \vdash a : A[x := i]$}
\AxiomC{$\Gamma \vdash i : I$}
\AxiomC{$\Gamma \vdash j : I$}
\QuaternaryInfC{$\Gamma \vdash lift_{\lambda x. A}\ i\ a\ j : A[x := j]$}
\DisplayProof
\end{center}

\medskip
\begin{center}
\def\extraVskip{1pt}
\Axiom$\fCenter \Gamma \vdash i : I$
\noLine
\UnaryInf$\fCenter \Gamma \vdash j : I$
\Axiom$\fCenter \Gamma \vdash a : \Pi (y : I) A$
\noLine
\UnaryInf$\fCenter \Gamma \vdash a' : \Pi (y : I) A$
\noLine
\UnaryInf$\fCenter \Gamma \vdash f : I \to A[y := i]$
\def\extraVskip{2pt}
\RightLabel{, $a\ i =_\beta f\ left$, $f\ right =_\beta a'\ i$}
\BinaryInfC{$\Gamma \vdash fill_{\lambda y. A}\ a\ a'\ i\ f\ j : I \to A[y := j]$}
\DisplayProof
\end{center}
\end{comment}

\bigskip
\caption{Правила вывода.}
\label{table:inf-rules}
\end{table}

Главное нововведение данной системы - это тип интервала $I$.
У него есть два конструктора ($left$ и $right$) и одно правило элиминации ($coe$).
% Также мы определили функцию $squeeze$, которая нам понадобится позже для определения элиминатора для типа путей $J$.

Правила редукции:
\begin{itemize}
\item $(\lambda x.b)\ a \to_\beta b[x := a]$
% \item $squeeze\ left\ j \to_\beta left$
% \item $squeeze\ right\ j \to_\beta j$
% \item $squeeze\ i\ left \to_\beta left$
% \item $squeeze\ i\ right \to_\beta i$
% \item $coe_{\lambda k.A}\ left\ a\ left \to_\beta a$
% \item $coe_{\lambda k.A}\ right\ a\ right \to_\beta a$
\item $coe_{\lambda k.A}\ i\ a\ i \to_\beta a$
\item $coe_{\lambda k.A}\ i\ a\ j \to_\sigma a$, если $k \notin FV(A)$
\end{itemize}

% Первое правило - обычныая $\beta$-редукцию для лямбда-термов.
% Следующие четыре правила описывают поведение функцию $squeeze$, таким образом она определяет ретракцию квадрата на отрезок.

Поведение элиминатора $coe$ можно описать следующим образом:
по расслоению $\lambda x. A$ над $I$ и по точке $a$ в слое над некоторой точкой $i$ интервала $coe_{\lambda x. A}\ i\ a$ конструирует сечение этого расслоения.
% Первые два правила редукции для $coe$ говорят, что это сечение в точке $i$ возвращает $a$ (при $i$ равном $left$ и $right$).
Первое правило редукции для $coe$ говорит, что это сечение в точке $i$ возвращает $a$.
Последнее правило говорит, что есть расслоение тривиально, то сечение константно.
Оно необходимо для того, чтобы $J$ удовлетворяло обычному правилу редукции для него.
Без $\sigma$-правила для это будет верно только с точностью до эквивалетности, то есть мы всегда можем найти путь между $coe_{\lambda. A}\ i\ a\ j$ и $a$.
Это правило несколько отличается от остальных правил редукций и не является настолько же важным, поэтому мы обозначаем его другой буквой.

Одно из важных свойств систем типов, которое нам понадобится, - это свойство \emph{каноничности}.
Мы будем говорить, что система обладает этим свойством, если все замкнутые термы в нормальной форме имеют канонический вид,
    то есть являются конструктором, возможно примененным к аргументам.
Система, описаная выше, не обладает этим свойством.
Чтобы исправить эту проблему, мы добавим еще больше правил редукции.
Нам нужно добавить $\tau$-правило для каждого типа в системе.
Пока единственный тип у нас - это $\Pi$-тип, так что мы добавляем одно правило:
\[ coe_{\lambda k. \Pi (a : A) B}\ i\ (\lambda a. b)\ j \to_\tau \lambda a'. coe_{\lambda k. B[a := coe_{\lambda k. A}\,j\,a'\,k]}\ i\ (b[a := coe_{\lambda k. A}\ j\ a'\ i])\ j \]

Теперь мы сформулируем несколько стандартных мета-теоретических свойств системы.
Все доказательства стандартны, поэтому мы будем приводить в основном только их наброски.
Начнем со следующего простого утверждения, которое говорит, что типизация замкнута относительно редукций.
\begin{prop}
Пусть $A_1 \rbst A'_1$, \ldots $A_n \rbst A'_n$, $A \rbst A'$ и $a \rbst a'$.
Тогда верны следующие утверждения:
\begin{itemize}
\item Если $x_1 : A_1, \ldots x_n : A_n \vdash$, то $x_1 : A'_1, \ldots x_n : A'_n \vdash$.
\item Если $x_1 : A_1, \ldots x_n : A_n \vdash A$, то $x_1 : A'_1, \ldots x_n : A'_n \vdash A'$.
\item Если $x_1 : A_1, \ldots x_n : A_n \vdash a : A$, то $x_1 : A'_1, \ldots x_n : A'_n \vdash a' : A'$.
\end{itemize}
\end{prop}
\begin{proof}
Единственный интересный случай - это $coe$.
Для $\beta$ и $\sigma$ правил всё просто.
Для доказательтсва $\tau$ правила нужно использовать $\beta$ правило.
Именно поэтому $\beta$ правило для $coe$ определено для всех $i$, а не только для $left$ и $right$.
\end{proof}

Теперь мы докажем, что отношение $\rbst$ \emph{конфлюентно}.
Это свойство говорит, что если $t \rbst q$ и $t \rbst r$, то существует терм $s$ такой, что $q \rbst s$ и $r \rbst s$.
Доказательство в основном стандартно, единственный не очевидный момент - это правила для $coe$, но не сложно адаптировать доказательство и для них.
Во-первых, мы введем новое отношение редукции $\to_p$, которое определяется индуктивно:
\begin{itemize}
\item $t \to_p t$.
\item Если $b \to_p b'$ и $a \to_p a'$, то $(\lambda x. b)\ a \to_p b'[x := a']$.
% \item $squeeze\ left\ j \to_p left$.
% \item Если $j \to_p j'$, то $squeeze\ right\ j \to_p j'$.
% \item $squeeze\ i\ left \to_p left$.
% \item Если $i \to_p i'$, то $squeeze\ i\ right \to_p i'$.
\item Если $a \to_p a'$ и $i \ebst j$, то $coe_{\lambda k. A}\ i\ a\ j\ \to_p a'$.
\item Если $a \to_p a'$ и $k \notin FV(A)$, то $coe_{\lambda k. A}\ i\ a\ j\ \to_p a'$.
\item Если $A \to_p A'$, $B \to_p B'$, $i \to_p i'$, $b \to_p b'$ и $j \to_p j'$, то
    \[ coe_{\lambda k. \Pi (a : A) B}\ i\ (\lambda a. b)\ j \to_p \lambda a'. coe_{\lambda k. B'[a := coe_{\lambda k. A'}\,j'\,a'\,k]}\ i'\ (b'[a := coe_{\lambda k. A'}\ j'\ a'\ i'])\ j' \]
\item Если $f \to_p f'$ и $a \to_p a'$, то $f\ a \to_p f'\ a'$.
\item Если $a \to_p a'$, то $\lambda x. a \to_p \lambda x. a'$.
\item Если $A \to_p A'$ и $B \to_p B'$, то $\Pi (a : A) B \to_p \Pi (a : A') B'$.
% \item Если $i \to_p i'$ и $j \to_p j'$, то $squeeze\ i\ j \to_p squeeze\ i'\ j'$.
\item Если $A \to_p A'$, $i \to_p i'$, $a \to_p a'$ и $j \to_p j'$, то $coe_{\lambda x. A}\ i\ a\ j \to_p coe_{\lambda x. A'}\ i'\ a'\ j'$.
\end{itemize}

Теперь мы хотим показать, что $\to_p$ конфлюентно.
Чтобы доказать часть с $\beta$-редукцией для лямбда-термов, нам потребуются следующая лемма.

\begin{lem}
Если $b \to_p b'$ и $a \to_p a'$, то $b[x := a] \to_p b'[x := a']$.
\end{lem}
\begin{proof}
Простая индукция по определению $b \to_p b'$.
В случае $b \to_p b$ мы продолжаем индукцией по построению терма $b$.
\end{proof}

\begin{lem}
Отношение $\to_p$ конфлюентно.
\end{lem}
\begin{proof}
Пусть $b \to_p b'$ и $b \to_p b''$.
Нужно показать, что существует терм $c$ такой, что $b' \to_p c$ и $b'' \to_p c$.
Мы делаем это индукцией по определению отношений $b \to_p b'$ и $b \to_p b''$.
Большинство пунктов элементарно.
Для случая $(\lambda x. b)\ a \to_p b'[x := a']$ мы используем предыдущую лемму.
Для случая $coe_{\lambda k. A}\ i\ a\ j \to_p a'$ когда $i \ebst j$ мы используем тот факт, что $t \to_p s$ влечет $t \ebst s$.

TODO: Написать подробнее?
\end{proof}

\begin{prop}
Отношение $\rbst$ конфлюентно.
\end{prop}
\begin{proof}
Это следует из предыдущей леммы и того факта, что транзитивное замыкание $\to_p$ совпадает с $\rbst$.
\end{proof}

\begin{remark}
Отношение $\rbs$ также конфлюентно.
Доказательство этого факта аналогично доказательству предыдущего с тем отличием, что в определении $\to_p$ нужно опустить правило, касающееся $\tau$.
\end{remark}

Теперь докажем, что система обладает свойством каноничности.

\begin{prop}
Пусть $\Gamma$ - это контекст вида $x_1 : I, \ldots x_n : I$.
Тогда верно следующее:
\begin{itemize}
\item Если $A$ - тип в контексте $\Gamma$, находящийся в нормальной форме, то $A$ является либо типом интервала, либо $\Pi$-типом.
\item Если $a$ - терм типа $A$ в контексте $\Gamma$, находящийси в нормальной форме, и $A$ $\bst$-эквивалентно $\Pi$-типу, то $a$ имеет вид $\lambda x. a'$.
\end{itemize}
\end{prop}
\begin{proof}
Доказательство индукцией по выводу $\Gamma \vdash A$ и $\Gamma \vdash a : A$.
Случай $\Gamma \vdash x : A$, где $x$ - переменная, следует из конфлюентности.
Единственный интересный случай - это правило для $coe$.
По предположению индукции $A$ является либо типом интервала, либо $\Pi$-типом.
Но тогда терм не находится в нормальной форме, так как в первом случае он редуцируется по $\sigma$-правилу, а во втором по $\tau$-правилу.
\end{proof}

\begin{cor}
Описанная система с $\bst$-правилами редукций обладает свойством каноничности.
\end{cor}

Теперь мы докажем \emph{сильную нормализуемость}.
Мы говорим, что терм сильно нормализуем, если не существует бесконечной последовательности редукций, начинающейся с этого терма.
Мы говорим, что теория сильно нормализуема, если $\Gamma \vdash a : A$ влечет, что $a$ сильно нормалиуем, и $\Gamma \vdash A$ влечет, что $A$ сильно нормализуем.

Для доказательства сильной нормализуемости нам понадобится понятие \emph{насыщенного} множества.
Множество сильно нормализуемых термов мы будем обозначать $SN$.
Если терм $t$ $\bst$-редуцируется к $s$ за один шаг, мы будем писать $t \to_1 s$.
Множество термов $s$, к которым $t$ редуцируется за один шаг, мы будем обозначать $red_1(t)$ (то есть $red_1(t) = \{ s\ |\ t \to_1 s \}$).
Термы, которые не являются ни абстракцией, ни $\Pi$-типом, мы будем называть \emph{простыми}, и множество простых термов обозначать $S$.
Мы будем говорить, что множество термов $X$ насыщенно, если выполнены следующие условия:
\begin{description}
\item[(SAT1)] $X \subseteq SN$.
\item[(SAT2)] Если $t \in X$ и $t \to_1 s$, то $s \in X$.
\item[(SAT3)] Если $t \in S$, и $red_1(t)$ является подмножеством $X$, то $t \in X$.
\end{description}

Условие \textbf{(SAT3)}, в частности, означает, что любой простой терм в нормальной форме должен принадлежать $X$.
В частности, все переменные принадлежат $X$, следовательно $X$ не пусто.
Если $X \subseteq SN$, то существует минимальное насыщенное множество, содержащее $X$.
Мы называем это множество насыщением $X$ и обозначаем $sat(X)$.

Чтобы доказать сильную нормализуемость, мы введем частичную функцию $\ll - \rr : Term \to SAT$,
    где $Term$ - множество термов, а $SAT$ - множество насыщенных подмножеств $Term$.
Для этого мы сначала определим ее график $G \subseteq Term \times SAT$ индуктивным образом.

\begin{center}
\AxiomC{$(t,A) \in G$}
\RightLabel{, если $t \to_1 s$}
\UnaryInfC{$(s,A) \in G$}
\DisplayProof
\end{center}

\begin{center}
\AxiomC{$\forall s \in red_1(t)\ (s,A) \in G$}
\RightLabel{, если $t \in S$, и $red_1(t)$ не пусто}
\UnaryInfC{$(t,A) \in G$}
\DisplayProof
\end{center}

\begin{center}
\AxiomC{}
\RightLabel{, если $t \in S$, и $red_1(t)$ пусто}
\UnaryInfC{$(t,SN) \in G$}
\DisplayProof
\end{center}

\begin{center}
\AxiomC{}
\UnaryInfC{$(I, sat(\{left, right\})) \in G$}
\DisplayProof
\end{center}

\begin{center}
\AxiomC{$(t,A) \in G$}
\AxiomC{$\forall a \in A\ (s[x := a], B_a) \in G$}
\BinaryInfC{$(\Pi (x : t) s, \{ f\ |\ \forall a \in A\ (f\ a \in B_a)\}) \in G$}
\DisplayProof
\end{center}

Индукцией по построению $G$ не сложно показать, что если $t \ebst s$, $(t,A) \in G$ и $(s,A') \in G$, то $A = A'$.
Следовательно $G$ действительно ялвяется графиком частичной функции $\ll - \rr : Term \to SAT$.
Пусть $Type \subseteq Term$ будет множество тех термов, на которых $\ll - \rr$ определена.
Индукцией по построению $G$ не сложно показать, что $Type \subseteq SN$, откуда следует, что $Type$ насыщено.

Для заключительной части доказательства нам понадобится еще одно понятие.
\emph{Означивание} - это частичная функцию из множества переменных в множество термов.
Означивание, которое нигде не определено, мы обозначаем $\varnothing$.
Если $\rho$ - означивание, и $t$ - терм, то $\rho[x := t]$ - это означивание, которое на $x$ возвращает $t$ и на остальных переменных определено также как и $\rho$.
Если $\rho$ - означивание, и $t$ - терм, то $t[\rho]$ - это терм, который определяется как
    $t[x_1 := \rho(x_1), \ldots x_n := \rho(x_n)]$, где $\{x_1, \ldots x_n$\} - это домен $\rho$.
Если $\rho$ - означивание, и $X$ - множество термов, то $X[\rho] = \{ t[\rho]\ |\ t \in X \}$.

Теперь мы докажем простую техническую лемму.

\begin{lem}[nat-of-int]
Пусть $A \in Type$.
Тогда для любой инволюции $\varphi$ (то есть такого означивания, что для любого терма $t$ верно $t[\varphi][\varphi] = t$)
верно, что $A[\varphi] \in Type$, и $\ll A[\varphi] \rr = \ll A \rr [\varphi]$.
\end{lem}
\begin{proof}
Индукцией по построению $(A, \ll A \rr) \in G$.
Первые четыре случая легко следуют из того факта, что $A' \to_1 A$ влечет $A'[\varphi] \to_1 A[\varphi]$ для любого означивания $\varphi$.
Откуда следует, что для любой инволюции $\varphi$ верно, что $red_1(A[\varphi]) = red_1(A)[\varphi]$, $SN[\varphi] = SN$, и $sat(X[\varphi]) = sat(X)[\varphi]$.

Последний случай:
\begin{center}
\AxiomC{$(A, \ll A \rr) \in G$}
\AxiomC{$\forall a \in \ll A \rr \ (B[x := a], \ll B[x := a] \rr) \in G$}
\BinaryInfC{$(\Pi (x : A) B, \{ f\ |\ \forall a \in \ll A \rr\ (f\ a \in \ll B[x := a] \rr)\}) \in G$}
\DisplayProof
\end{center}
По индукционной гипотезе мы знаем, что $(A[\varphi], \ll A \rr [\varphi]) \in G$, и $\forall a \in \ll A \rr\ (B[x := a][\varphi], \ll B[x := a] \rr [\varphi]) \in G$.
Так как $B[x := a][\varphi] = B[\varphi][x := a[\varphi]]$, то последнее выражение можно переписать как
    $\forall a \in \ll A \rr [\varphi]\ (B[\varphi][x := a], \ll B[x := a[\varphi]] \rr [\varphi]) \in G$.
Откуда получаем, что \[ ((\Pi (x : A) B)[\varphi], \{ f\ |\ \forall a \in \ll A \rr [\varphi]\ (f\ a \in \ll B[x := a[\varphi]] \rr [\varphi]) \}) \in G \].
Таким образом $(\Pi (x : A) B)[\varphi] \in Type$.
Осталось убедиться, что множество, описанное выше, совпадает с множеством $\{ f[\varphi]\ |\ \forall a \in \ll A \rr\ (f\ a \in \ll B[x := a] \rr) \}$.
Это легко следует из того факта, что $\varphi$ - инволюция.
\end{proof}

Так как множество переменных бесконечно, то его можно разбить на два бесконечных непересекающихся равномощных подмножества $Var$ и $Var'$.
Мы будем предполагать, что в правилах вывода учавствуют только переменные из $Var$.
Теперь мы определим частичную функцию $\ll - \rr$ из множества контекстов в множество подмножеств множества означиваний:
\[ \ll \varnothing \rr = \{ \varnothing \} \]
\[ \ll \Gamma, x : A \rr = \{ \rho[x := a]\ |\ \rho \in \ll \Gamma \rr, a \in \ll A[\rho] \rr, FV(a) \subseteq dom(\rho) \cup \{ x \} \cup Var' \} \]
Причем, мы считаем, что $\ll \Gamma, x : A \rr$ определено тогда и только тогда,
    когда $\ll \Gamma \rr$ определено, и для любого $\rho \in \ll \Gamma \rr$ верно, что $A[\rho] \in Type$.

\begin{prop}[sn]
Верны следующие утверждения:
\begin{itemize}
\item Если $\Gamma \vdash$, то $\ll \Gamma \rr$ определено.
\item Если $\Gamma \vdash A$, то $\ll \Gamma \rr$ определено,
    и для любого $\rho \in \ll \Gamma \rr$ верно, что $A[\rho] \in Type$.
\item Если $\Gamma \vdash a : A$, то $\ll \Gamma \rr$ определено,
    и для любого $\rho \in \ll \Gamma \rr$ верно, что $A[\rho] \in Type$, и $a[\rho] \in \ll A[\rho] \rr$.
\end{itemize}
\end{prop}
\begin{proof}
Индукцией по выводу.
\begin{itemize}
\item Случаи
\begin{center}
\AxiomC{}
\UnaryInfC{$\varnothing \vdash$}
\DisplayProof
\quad
\AxiomC{$\Gamma \vdash A$}
\RightLabel{, $x \notin \Gamma$}
\UnaryInfC{$\Gamma, x : A \vdash$}
\DisplayProof
\end{center}
следуют из определения $\ll - \rr$ для контекстов.

\item Случай
\begin{center}
\AxiomC{$\Gamma \vdash$}
\RightLabel{, $x : A \in \Gamma$}
\UnaryInfC{$\Gamma \vdash x : A$}
\DisplayProof
\end{center}
Мы знаем, что $\Gamma = \Gamma', x : A, \Gamma''$.
Если $\rho \in \ll \Gamma \rr$, то $A[\rho]$ = $A[\rho|_{\Gamma'}]$, т.к. в $A$ не встречаются свободные переменные из $x : A, \Gamma''$.
Так как $\ll \Gamma \rr$ опеделено, то $\ll \Gamma', x : A \rr$ также определено,
    следовательно $A[\rho|_{\Gamma'}] \in Type$ и $\rho(x) \in \ll A[\rho|_{\Gamma'}] \rr$, что и требовалось.

\item Случай
\begin{center}
\AxiomC{$\Gamma \vdash a : A$}
\AxiomC{$\Gamma \vdash B$}
\RightLabel{, $A =_{\beta \sigma \tau} B$}
\BinaryInfC{$\Gamma \vdash a : B$}
\DisplayProof
\end{center}
Так как $\ll A[\rho] \rr$ и $\ll B[\rho] \rr$ определены, и $A[\rho] \ebst B[\rho]$, то $\ll A[\rho] \rr = \ll B[\rho] \rr$.
Следовательно $a[\rho] \in \ll A[\rho] \rr = \ll B[\rho] \rr$.

\item Случай
\begin{center}
\AxiomC{$\Gamma \vdash A$}
\AxiomC{$\Gamma, x : A \vdash B$}
\BinaryInfC{$\Gamma \vdash \Pi (x : A) B$}
\DisplayProof
\end{center}
Пусть $\rho \in \ll \Gamma \rr$. Нам нужно показать, что $(\Pi (x : A) B)[\rho] \in Type$.
Для этого достаточно показать, что для любого $a \in \ll A[\rho] \rr$ верно, что $B[\rho][x := a] = B[\rho[x := a]] \in Type$.
По индукционной гипотезе мы знаем, что если $FV(a) \subseteq dom(\rho) \cup \{ x \} \cup Var'$, то это верно.
Пусть теперь $a$ - произвольный, и пусть $V = FV(a) \setminus dom(\rho) \cup \{ x \} \cup Var'$.
Так как $V$ - конечное множество, то мы можем выбрать $V' \subseteq Var'$ равномощное $V$.
Теперь мы можем определить означивание $\phi$, которое каждой переменной из $V$ сопоставляет соответствующую переменную из $V'$ и наоборот.
Таким образом, $\phi$ - инволюция.

По лемме~\rlem{nat-of-int} мы знаем, что $a[\varphi] \in \ll A[\rho][\varphi] \rr$, но $A[\rho][\varphi] = A[\rho]$, так как $\varphi$ не меняет свободные переменные $A[\rho]$.
По построению $FV(a[\varphi]) \subseteq dom(\rho) \cup \{ x \} \cup Var'$, следовательно $B[\rho][x := a[\varphi]] \in Type$.
Но $B[\rho][x := a[\varphi]] = B[\rho][x := a][\varphi]$, следовательно $B[\rho][x := a] \in Type$ по лемме~\rlem{nat-of-int}, что и требовалось показать.

\item Случай
\begin{center}
\AxiomC{$\Gamma, x : A \vdash b : B$}
\UnaryInfC{$\Gamma \vdash \lambda x. b : \Pi (x : A) B$}
\DisplayProof
\end{center}
Мы должны показать, что для любых $\rho \in \ll \Gamma \rr$ и $a \in \ll A[\rho] \rr$ верно, что $(\lambda x. b[\rho]) a \in \ll B[\rho][x := a] \rr$.
По индукционной гипотезе мы знаем, что если $FV(a) \subseteq dom(\rho) \cup \{ x \} \cup Var'$, то $b[\rho][x := a] \in \ll B[\rho][x := a] \rr$.
Так как $b[\rho]$ и $a$ сильно нормализуемы, то в любой достаточно длинной последовательности редукций, начинающейся с $(\lambda x. b[\rho]) a$ внешний редекс будет сокращен.
При этом мы получим терм к которому редуцируется $b[\rho][x := a]$, следовательно по \textbf{(SAT2)} он лежит в $\ll B[\rho][x := a] \rr$.
Так как любая последовательность редукций заканчивается термами в $\ll B[\rho][x := a] \rr$, то \textbf{(SAT3)} влечет, что $(\lambda x. b[\rho]) a$ сам принадлежит этому множеству.

Теперь, если $a$ произвольный, то мы выбираем инволюцию $\varphi$ также как в предыдущем пункте.
Тогда $a[\varphi] \in \ll A[\rho] \rr$, и, как мы только что видели, $(\lambda x. b[\rho]) (a[\varphi]) \in \ll B[\rho][x := a[\varphi]] \rr$.
Так как $(\lambda x. b[\rho]) (a[\varphi]) = ((\lambda x. b[\rho]) a)[\varphi]$, и $B[\rho][x := a[\varphi]] = B[\rho][x := a][\varphi]$, то
    по лемме~\rlem{nat-of-int} $(\lambda x. b[\rho]) a \in \ll B[\rho][x := a] \rr$, что и требовалось показать.

\item Случай
\begin{center}
\AxiomC{$\Gamma \vdash f : \Pi (x : A) B$}
\AxiomC{$\Gamma \vdash a : A$}
\BinaryInfC{$\Gamma \vdash f\ a : B[x := a]$}
\DisplayProof
\end{center}
Пусть $\rho \in \ll \Gamma\rr$.
По индукционной гипотезе мы знаем, что $f[\rho]\ a[\rho] \in \ll B[\rho][x := a[\rho]] \rr$.
Но $B[\rho][x := a[\rho]] = B[x := a][\rho]$, следовательно $(f\ a)[\rho] \in \ll B[x := a][\rho]\rr$, что и требовалось показать.

\item Случаи
\begin{center}
\AxiomC{$\Gamma \vdash$}
\UnaryInfC{$\Gamma \vdash I$}
\DisplayProof
\quad
\AxiomC{$\Gamma \vdash$}
\UnaryInfC{$\Gamma \vdash left : I$}
\DisplayProof
\quad
\AxiomC{$\Gamma \vdash$}
\UnaryInfC{$\Gamma \vdash right : I$}
\DisplayProof
\end{center}
элементарны.

\item Случай
\begin{center}
\AxiomC{$\Gamma, x : I \vdash A$}
\AxiomC{$\Gamma \vdash i : I$}
\AxiomC{$\Gamma \vdash a : A[x := i]$}
\AxiomC{$\Gamma \vdash j : I$}
\QuaternaryInfC{$\Gamma \vdash coe_{\lambda x. A}\ i\ a\ j : A[x := j]$}
\DisplayProof
\end{center}
Пусть $\rho \in \ll \Gamma \rr$.
Сначала мы покажем, что $A[x := j][\rho] \in Type$.
Действительно, $A[x := j][\rho] = A[\rho][x := j[\rho]] = A[\rho[x := j[\rho]]]$, и это множество принадлежит $Type$ по индукционной гипотезе.

Теперь мы должны доказать, что $(coe_{\lambda x. A}\ i\ a\ j)[\rho] \in \ll A[x := j][\rho] \rr$.
Мы сделаем это индукцией по выводу $\Gamma, x : I \vdash A$.
Конкретно, мы докажем следующее утверждение.
Пусть $\Gamma, x : I, z_1 : Z_1, \ldots z_n : Z_n \vdash A$, и дерево вывода для него является поддеревом вывода
    $\Gamma, x : I \vdash A$ (чтобы мы смогли использовать условие утверждения~\rprop{sn} для него).
Пусть $i_1, j_1 \in \ll I \rr$ такие, что $FV(i_1) \cup FV(j_1) \subseteq dom(\rho) \cup Vars'$.
Пусть $t_1, \ldots t_n$ - такая последовательность термов, что $FV(t_i) \subseteq dom(\rho) \cup \{ x, z_1, \ldots z_n \} \cup Vars'$, и
$\rho[x := k][z_1 := t_1[x := k]] \ldots [z_n := t_n[x := k]] \in \ll \Gamma, x : I, z_1 : Z_1, \ldots z_n : Z_n \rr$, где $k \in \{ i_1, j_1 \}$.
Мы будем обозначать это означивание $\rho_i$ при $k = i_1$ и $\rho_j$ при $k = j_1$, а также мы будем писать $\rho_x$ при $k = x$.
Тогда если $a_1 \in \ll A[\rho_i] \rr$, то $coe_{\lambda x. A[\rho_x]}\ i_1\ a_1\ j_1 \in \ll A[\rho_j] \rr$.

По аргументу, аналогичному тому, который мы приводили в случае аппликации, нам достаточно показать,
    что $\beta$, $\sigma$ и $\tau$ редукции для $coe$ переводят $coe_{\lambda x. A[\rho_x]}\ i_1\ a_1\ j_1$ в терм, лежащий в множестве $\ll A[\rho_j] \rr$.
Случаи $\beta$ и $\sigma$ редукций не представляют проблем, так как они сразу же следуют из того факта, что $a_1 \in \ll A[\rho_i] \rr$, и в обоих случаях $A[\rho_i] = A[\rho_j]$.
Пусть теперь $coe_{\lambda x. \Pi (a : A[\rho_x]) B[\rho_x]}\ i_1\ (\lambda a. b_1)\ j_1$ редуцируется по $\tau$-правилу к
    $\lambda a'. coe_{\lambda x. B[\rho_x][a := coe_{\lambda x. A[\rho_x]}\,j_1\,a'\,x]}\ i_1\ (b_1[a := coe_{\lambda x. A[\rho_x]}\ j_1\ a'\ i_1])\ j_1$.
Обозначим этот терм за $f$.

Мы должны показать, что $f$ принадлежит $\ll (\Pi (a : A) B)[\rho_j] \rr = \ll \Pi (a : A[\rho_j]) B[\rho_j]\rr$.
Для этого достаточно показать, что для любого $a_2 \in \ll A[\rho_j] \rr$ верно, что $f\,a_2 \in \ll B[\rho_j][a := a_2] \rr$,
    при этом мы можем предположить, что $FV(a_2) \subseteq dom(\rho) \cup \{ z_1, \ldots z_n \} \cup Vars'$, используя аргумент аналогичный тому, который мы применяли в случае абстракции.
По индукционной гипотезе $coe_{\lambda x. A[\rho_x]}\ j_1\ a_2\ i_1 \in \ll A[\rho_i] \rr$.
Также мы знаем, что $\lambda a. b_1 \in \Pi (a : A[\rho_i]) B[\rho_i]$,
    следовательно $b_1[a := coe_{\lambda x. A[\rho_x]}\ j_1\ a_2\ i_1] \in B[\rho_i][a := coe_{\lambda x. A[\rho_x]}\ j_1\ a_2\ i_1]$.
Обозначим этот терм за $b_2$.

Теперь нам нужно применить индукционную гипотезу для терма $coe_{\lambda x. B[\rho_x][a := coe_{\lambda x. A[\rho_x]}\,j_1\,a_2\,x]}\ i_1\ b_2\ j_1$.
Для этого возьмем $z_{n+1} = a$, $Z_{n+1} = A$ и $t_{n+1} = coe_{\lambda x. A[\rho_x]}\ j_1\ a_2\ x$.
Тогда мы получим, что этот терм, а следовательно и $f\,a_2$ принадлежит $\ll B[\rho_j][a := coe_{\lambda x. A[\rho_x]}\,j_1\,a_2\,j_1] \rr$.
Так как термы $B[\rho_j][a := coe_{\lambda x. A[\rho_x]}\,j_1\,a_2\,j_1]$ и $B[\rho_j][a := a_2]$ $\tau$-эквивалентны,
    то $f\,a_2 \in \ll B[\rho_j][a := a_2] \rr$, что и требовалось доказать.

\end{itemize}
\end{proof}

\begin{cor}
Система сильно нормализуема.
\end{cor}
\begin{proof}
Действительно, достаточно в качестве $\rho$ в условии утверждения взять тривиальное окружение, которое каждую $x$ из $\Gamma$ отображает в $x$.
\end{proof}

\section{Бла}

Now, we show how to add a univalent universe to the system.

\medskip
\begin{center}
\AxiomC{$\Gamma \vdash$}
\UnaryInfC{$\Gamma \vdash Type$}
\DisplayProof
\quad
\AxiomC{$\Gamma \vdash$}
\UnaryInfC{$\Gamma \vdash I : Type$}
\DisplayProof
\quad
\AxiomC{$\Gamma \vdash A : Type$}
\AxiomC{$\Gamma, x : A \vdash B : Type$}
\BinaryInfC{$\Gamma \vdash \Pi (x : A) B : Type$}
\DisplayProof
\end{center}

\medskip
\begin{center}
\Axiom$\fCenter \Gamma \vdash A : Type$
\noLine
\UnaryInf$\fCenter \Gamma \vdash B : Type$
\def\extraVskip{1pt}
\Axiom$\fCenter \Gamma \vdash f : A \to B$
\noLine
\UnaryInf$\fCenter \Gamma \vdash g : B \to A$
\Axiom$\fCenter \Gamma \vdash p : \Pi (a : A) (Path\ A\ (g\ (f\ a))\ a)$
\noLine
\UnaryInf$\fCenter \Gamma \vdash q : \Pi (b : B) (Path\ B\ (f\ (g\ b))\ b)$
\def\extraVskip{2pt}
\AxiomC{$\Gamma \vdash i : I$}
\QuaternaryInfC{$\Gamma \vdash iso\ A\ B\ f\ g\ p\ q\ i : Type$}
\DisplayProof
\end{center}
\medskip

We also add the following reduction rules:
\begin{itemize}
\item $iso\ A\ B\ f\ g\ p\ q\ left \to_\beta A$
\item $iso\ A\ B\ f\ g\ p\ q\ right \to_\beta B$
\item $coe_{\lambda k. iso\,A\,B\,f\,g\,p\,q\,k}\ left\ a\ right \to_\beta f\ a$
\item $coe_{\lambda k. iso\,A\,B\,f\,g\,p\,q\,k}\ right\ b\ left \to_\beta g\ b$
\end{itemize}

Later we will show that these rules imply the univalence axiom in its usual form.

\subsection{Data types with conditions}

An ordinary data type $D$ with parameters $a_1 : A_1, \ldots a_n : A_n$ is described by a list of its constructors:
\[ c_1 : (x_1 : X_1) \ldots (x_{k_1} : X_{k_1}) \to D\ a_1 \ldots a_n \]
\[ \vdots \]
\[ c_m : (z_1 : Z_1) \ldots (z_{k_m} : Z_{k_m}) \to D\ a_1 \ldots a_n \]

A data type with conditions also allows us to put a condition on a constructor.
A condition on a constructor is simply a partial definition of a function with the type of this constructor.
Syntactically, this means that constructors now can be evaluated if its arguments match one of the clauses of this partially defined function.
Also, this means that when we define a function over such a data type we need to check that this definition respects conditions.

Semantically, ordinary data types allow to define coproducts of types, and data types with conditions allow us to define pushouts.
One of the maps in the pushout diagram must be given by a collection of constructors of a data type, which means it is (often can be interpreted as) a cofibration.
This implies that this pushout is allways a correct homotopy pushout.

Let us give an example. We can define integers as the data type with two constructors:
\[ negative : \mathbb{N} \to \mathbb{Z} \]
\[ positive : \mathbb{N} \to \mathbb{Z}. \]

The problem is that we get two different zeros: $negative\ 0$ and $positive\ 0$.
Of course, we can define $-1$ to be $negative\ 0$, $-2$ to be $negative\ 1$, and so one, but this can easily lead to a confusion.
It is better to identify the positive and the negative zeros, that is to define $\mathbb{Z}$ as the pushout of $1 \overset{0}\to \mathbb{N}$ with itself.
So, we add the following condition to the definition of $\mathbb{Z}$:
\[ negative\ 0 = positive\ 0. \]
Now, when we define a function $f$ over $\mathbb{Z}$, we need to check that $f\ (negative\ 0) \ebs f\ (positive\ 0)$.

Using data types with conditions, we can define higher inductive types.
For example, the circle $S^1$ can be defined as follows: it has two constructors
\[ base : S^1 \]
\[ loop : I \to S^1 \]
and two conditions
\[ loop\ left = base \]
\[ loop\ right = base. \]

After we add path types, it will be posible to describe elimination rules for such data types,
but it is more convenient to simply define functions over them.

\subsection{Records with conditions}

Records with conditions are dual to data types with conditions.
Ordinary records allow us to define product of types, and records with conditions allow us to define pullbacks.
A record is given by a list of fields, and a record with conditions can additionally put conditions on some fields.
A condition on a field is a partially defined function with the type of this field.
Semantically, such a record is given by a pullback of a function $(B \to X) \to (A \to X)$ for some cofibration $A \to B$, which means that it is still fibrant.

Let us show how to define path types as a record with conditions.
It will be a record $Path$ with three parameters $A : I \to Type$, $a : A\ left$, and $a' : A\ right$.
We will define a ``heterogeneous path type'' because it seems impossible to define useful elimination rules for higher inductive types using only homogeneous path types.
We abbreviate $Path\ (\lambda _. A)\ a\ a'$ to $a = a'$.
The record $Path\ (A : I \to Type)\ (a : A\ left)\ (a' : A\ right)$ has one constructor
\[ path : ((i : I) \to A\ i) \to Path\ (A : I \to Type)\ (a : A\ left)\ (a' : A\ right), \]
one field
\[ at : (i : I) \to A\ i, \]
and two conditions
\[ at\ left = a \]
\[ at\ right = a'. \]

When we define an element of a record with conditions, we need to check that it satisfies them.
And when we access fields with conitions, they might evaluate, if arguments to it match some condition on it.
For example, if $p : a = a'$, then $p.at\ left$ evaluates to $a$, $p.at\ right$ evaluates to $a'$,
and if we want to consruct a path of type $a = a'$, then we need to specify a function $f : I \to A$ such that $f\ left \ebs a$ and $f\ right \ebs a'$.

We can define the $J$ rule as follows:
\begin{itemize}
\item[] $idp : (A : Type) (a : A) \to Path\ (\lambda _. A)\ a\ a$
\item[] $idp\ A\ a = path (\lambda i. a)$
\item[]
\item[] $J : (A : Type) (C : (x\ y : A) \to x = y \to Type) \to$
\item[] \qquad $((a : A) \to C\ a\ a\ (idp\ A\ a)) \to (a\ a' : A) (p : a = a') \to C\ a\ a'\ p$
\item[] $J\ A\ C\ d\ a\ a'\ p =\ coe_{\lambda i.\,C\,a\,(p.at\,i)\,(path\,(\lambda j.\,p.at\,(squeeze\,i\,j)))}\ left\ (d\ a)\ right$
\end{itemize}

Now, we can prove the univalence axiom in its usual form, which states that a certain function $F$ is equivalence.
Function $F$ maps path types $A = B$ to the type of functions $A \to B$ which are equivalences.
We can construct an inverse $G$ to function $F$: given an equivalence $A \to B$, $G$ construct a path $A = B$ using $iso$.
If $f : A \to B$ is an equivalence, then $F\ (G\ f)$ definitionally equals to $f$ because of the compuation rule for $iso$.
If $p : A = B$, then we can show that there is a path between $G\ (F\ p)$ and $p$ using $J$.
Actually, we can't use $J$ directly because of the size issues, but we can expand its definition, and then it works fine.

\section{A model of data types and records with conditions}

In this section, we will show how to interpret data types and records with conditions in simplicial sets.

\subsection{Data types with conditions}

Each constructor of a data type gives us an accessible functor $C_i : \sSet \to \sSet$.

% If a data type without conditions has $n$ constructors, then we can interpret it as an algebra for endofunctor $\coprod_{1 \leq i \leq n} C_i$.

\subsection{Records with conditions}

% \bibliographystyle{amsplain}
% \bibliography{ref}

\end{document}
