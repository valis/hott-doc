\documentclass[reqno]{amsart}

\usepackage{amssymb}
\usepackage{hyperref}
\usepackage{mathtools}
\usepackage[all]{xy}
\usepackage{verbatim}
\usepackage{ifthen}
\usepackage{xargs}
\usepackage{bussproofs}
\usepackage{turnstile}
\usepackage{etex}
\usepackage{todonotes}

\hypersetup{colorlinks=true,linkcolor=blue}

\renewcommand{\turnstile}[6][s]
    {\ifthenelse{\equal{#1}{d}}
        {\sbox{\first}{$\displaystyle{#4}$}
        \sbox{\second}{$\displaystyle{#5}$}}{}
    \ifthenelse{\equal{#1}{t}}
        {\sbox{\first}{$\textstyle{#4}$}
        \sbox{\second}{$\textstyle{#5}$}}{}
    \ifthenelse{\equal{#1}{s}}
        {\sbox{\first}{$\scriptstyle{#4}$}
        \sbox{\second}{$\scriptstyle{#5}$}}{}
    \ifthenelse{\equal{#1}{ss}}
        {\sbox{\first}{$\scriptscriptstyle{#4}$}
        \sbox{\second}{$\scriptscriptstyle{#5}$}}{}
    \setlength{\dashthickness}{0.111ex}
    \setlength{\ddashthickness}{0.35ex}
    \setlength{\leasturnstilewidth}{2em}
    \setlength{\extrawidth}{0.2em}
    \ifthenelse{%
      \equal{#3}{n}}{\setlength{\tinyverdistance}{0ex}}{}
    \ifthenelse{%
      \equal{#3}{s}}{\setlength{\tinyverdistance}{0.5\dashthickness}}{}
    \ifthenelse{%
      \equal{#3}{d}}{\setlength{\tinyverdistance}{0.5\ddashthickness}
        \addtolength{\tinyverdistance}{\dashthickness}}{}
    \ifthenelse{%
      \equal{#3}{t}}{\setlength{\tinyverdistance}{1.5\dashthickness}
        \addtolength{\tinyverdistance}{\ddashthickness}}{}
        \setlength{\verdistance}{0.4ex}
        \settoheight{\lengthvar}{\usebox{\first}}
        \setlength{\raisedown}{-\lengthvar}
        \addtolength{\raisedown}{-\tinyverdistance}
        \addtolength{\raisedown}{-\verdistance}
        \settodepth{\raiseup}{\usebox{\second}}
        \addtolength{\raiseup}{\tinyverdistance}
        \addtolength{\raiseup}{\verdistance}
        \setlength{\lift}{0.8ex}
        \settowidth{\firstwidth}{\usebox{\first}}
        \settowidth{\secondwidth}{\usebox{\second}}
        \ifthenelse{\lengthtest{\firstwidth = 0ex}
            \and
            \lengthtest{\secondwidth = 0ex}}
                {\setlength{\turnstilewidth}{\leasturnstilewidth}}
                {\setlength{\turnstilewidth}{2\extrawidth}
        \ifthenelse{\lengthtest{\firstwidth < \secondwidth}}
            {\addtolength{\turnstilewidth}{\secondwidth}}
            {\addtolength{\turnstilewidth}{\firstwidth}}}
        \ifthenelse{\lengthtest{\turnstilewidth < \leasturnstilewidth}}{\setlength{\turnstilewidth}{\leasturnstilewidth}}{}
    \setlength{\turnstileheight}{1.5ex}
    \sbox{\turnstilebox}
    {\raisebox{\lift}{\ensuremath{
        \makever{#2}{\dashthickness}{\turnstileheight}{\ddashthickness}
        \makehor{#3}{\dashthickness}{\turnstilewidth}{\ddashthickness}
        \hspace{-\turnstilewidth}
        \raisebox{\raisedown}
        {\makebox[\turnstilewidth]{\usebox{\first}}}
            \hspace{-\turnstilewidth}
            \raisebox{\raiseup}
            {\makebox[\turnstilewidth]{\usebox{\second}}}
        \makever{#6}{\dashthickness}{\turnstileheight}{\ddashthickness}}}}
        \mathrel{\usebox{\turnstilebox}}}

\newcommand{\newref}[4][]{
\ifthenelse{\equal{#1}{}}{\newtheorem{h#2}[hthm]{#4}}{\newtheorem{h#2}{#4}[#1]}
\expandafter\newcommand\csname r#2\endcsname[1]{#3~\ref{#2:##1}}
\expandafter\newcommand\csname R#2\endcsname[1]{#4~\ref{#2:##1}}
\expandafter\newcommand\csname n#2\endcsname[1]{\ref{#2:##1}}
\newenvironmentx{#2}[2][1=,2=]{
\ifthenelse{\equal{##2}{}}{\begin{h#2}}{\begin{h#2}[##2]}
\ifthenelse{\equal{##1}{}}{}{\label{#2:##1}}
}{\end{h#2}}
}

\newref[section]{thm}{theorem}{Theorem}
\newref{lem}{lemma}{Lemma}
\newref{prop}{proposition}{Proposition}
\newref{cor}{corollary}{Corollary}
\newref{cond}{condition}{Condition}

\theoremstyle{definition}
\newref{defn}{definition}{Definition}
\newref{example}{example}{Example}

\theoremstyle{remark}
\newref{remark}{remark}{Remark}

\newcommand{\deq}{\equiv}
\newcommand{\cat}[1]{\mathbf{#1}}
\newcommand{\C}{\cat{C}}
\newcommand{\PAlg}[1]{#1\text{-}\cat{PAlg}}
\newcommand{\Mod}[1]{#1\text{-}\cat{Mod}}
\newcommand{\Th}{\cat{Th}}
\newcommand{\algtt}{\cat{AlgTT}}
\newcommand{\ThC}{\Th_{\mathcal{C}}}
\newcommand{\emptyCtx}{\mathbf{1}}
\newcommand{\tta}{\mathbb{T}_0}
\newcommand{\tts}{\mathbb{T}_1}

\newcommand{\we}{\mathcal{W}}
\newcommand{\fib}{\mathcal{Fib}}
\newcommand{\cof}{\mathcal{Cof}}
\newcommand{\I}{\mathrm{I}}
\newcommand{\J}{\mathrm{J}}
\newcommand{\class}[2]{#1\text{-}\mathrm{#2}}
\newcommand{\Iinj}[1][\I]{\class{#1}{inj}}
\newcommand{\Icell}[1][\I]{\class{#1}{cell}}
\newcommand{\Icof}[1][\I]{\class{#1}{cof}}
\newcommand{\Jinj}[1][]{\Iinj[\J#1]}
\newcommand{\Jcell}[1][]{\Icell[\J#1]}
\newcommand{\Jcof}[1][]{\Icof[\J#1]}

\numberwithin{figure}{section}

\newcommand{\pb}[1][dr]{\save*!/#1-1.2pc/#1:(-1,1)@^{|-}\restore}
\newcommand{\po}[1][dr]{\save*!/#1+1.2pc/#1:(1,-1)@^{|-}\restore}

\begin{document}

\title{Models of Algebraic Dependent Type Theories}

\author{Valery Isaev}

\begin{abstract}
\end{abstract}

\maketitle

 \makeatletter
    \providecommand\@dotsep{5}
  \makeatother
  \listoftodos\relax

\section{Introduction}

\section{Functors between categories of models and theories}

In this section we will prove that the category of models of $\mathbb{T}$ is a coreflective full subcategory of theories under $\mathbb{T}$.

Let $\mathbb{T} = ((\mathcal{S},\mathcal{F},\mathcal{P}),\mathcal{A})$ be a standard partial Horn theory and let $M = (A,\alpha,\beta)$ be its model.
If $\sigma \in \mathcal{F}$, $\sigma : s_1 \times \ldots \times s_k \to s$, then we will write $M(\sigma) : A_{s_1} \times \ldots A_{s_k} \to A_s$
    for the partial function defined as $M(\sigma)(a_1, \ldots a_k) = \alpha(x_i \mapsto a_i)(\sigma(x_1, \ldots x_k))$.
If $R \in \mathcal{P}$, $R : s_1 \times \ldots \times s_k$ then we will write $M(R)$ for the subset of $A_{s_1} \times \ldots A_{s_k}$,
    which consists of tuples $(a_1, \ldots a_k)$ such that $\beta(x_i \mapsto a_i)(R(x_1, \ldots x_k)) = \top$.

For every model $M$ of $\mathbb{T}$, we define a theory $Lang(M)$.
It has function and predicate symbols of $\mathbb{T}$ together with function symbol $O_a : s$ for every $a \in A_s$.
Axioms of $Lang(M)$ are axioms of $\mathbb{T}$ together with the following sequents:
\begin{align*}
& \sststile{}{} O_a \downarrow \\
& \sststile{}{} \sigma(O_{a_1}, \ldots O_{a_k}) = O_{M(\sigma)(a_1, \ldots a_k)} \\
& \sststile{}{} R(O_{a_1}, \ldots O_{a_k})
\end{align*}
for every $a \in A_s$, every $a_i \in A_{s_i}$,
every $\sigma \in \mathcal{F}$ such that $M(\sigma)(a_1, \ldots a_k)$ is defined,
and every $R \in \mathcal{P}$ such that $(a_1, \ldots a_k) \in M(R)$.

Models of $Lang(M)$ are just models of $\mathbb{T}$ together with a morphism from $M$.
That is, categories $M/\Mod{\mathbb{T}}$ and $\Mod{Lang(M)}$ are isomorphic.
In particular, $A$ has a natural structure of a model of $Lang(M)$ defined as follows:
\begin{align*}
\alpha'(f)(O_a) & = a \\
\alpha'(f)(\sigma(x_1, \ldots x_k)) & = \alpha(f)(\sigma(x_1, \ldots x_k)) \\
\beta'(f)(R(x_1, \ldots x_k)) & = \beta(f)(R(x_1, \ldots x_k))
\end{align*}

\begin{lem}[cl-term]
If $t \in Term_\mathcal{F}(\varnothing)_s$ is such that $\sststile{}{} t \downarrow$ is a theorem of $Lang(M)$,
    then there is a unique $a \in A_s$ such that $\sststile{}{} t = O_a$ is a theorem of $Lang(M)$.
\end{lem}
\begin{proof}
Since $(A,\alpha',\beta')$ is a model of $Lang(M)$, for every theorem $\varphi \sststile{}{V} \psi$ of $Lang(M)$
    and every total function $f : V \to A$, if $\beta'(f)(\varphi) = \top$, then $\beta'(f)(\psi) = \top$.
In particular, if $\sststile{}{} O_a = O_{a'}$, then $a = a'$.
Hence if $\sststile{}{} t = O_a$ and $\sststile{}{} t = O_{a'}$, then $a = a'$, so such $a$ is unique.

Let us prove its existence.
We do this by induction on $t$.
If $t = O_a$, then we are done.
If $t = \sigma(t_1, \ldots t_k)$, then by induction hypothesis, $\sststile{}{} t = \sigma(O_{a_1}, \ldots O_{a_k})$ for some $a_1$, \ldots $a_k$.
Note that if $\sststile{}{} \sigma(O_{a_1}, \ldots O_{a_k})\!\!\downarrow$ is derivable, then $M(\sigma)(a_1, \ldots a_k)$ is defined.
Thus $\sststile{}{} \sigma(O_{a_1}, \ldots O_{a_k}) = O_{M(\sigma)(a_1, \ldots a_k)}$ is also derivable.
\end{proof}

For every morphism $h : M \to N$ of models of $\mathbb{T}$, we can define a morphism $Lang(h) : Lang(M) \to Lang(N)$ of theories under $\mathbb{T}$ as $Lang(h)(O_a) = O_{h(a)}$.
Thus $Lang$ is a functor $\Mod{\mathbb{T}} \to \mathbb{T}/\Th_\mathcal{S}$.

\begin{prop}
$Lang$ is fully faithful.
\end{prop}
\begin{proof}
Let $h_1$, $h_2$ be morphisms of models such that $Lang(h_1) = Lang(h_2)$.
Then $O_{h_1(a)} = Lang(h_1)(O_a) = Lang(h_2)(O_a) = O_{h_2(a)}$, and by \rlem{cl-term}, $h_1(a) = h_2(a)$.
Thus $Lang$ is faithful.

Let $M_1 = (A_1,\alpha_1,\beta_1)$ and $M_2 = (A_2,\alpha_2,\beta_2)$ be models of $\mathbb{T}$,
    and let $h : Lang(M_1) \to Lang(M_2)$ be a morphism of theories under $\mathbb{T}$.
Then by \rlem{cl-term}, for every $a \in A_1$, there is a unique $h'(a) \in A_2$ such that $\sststile{}{} h(O_a) = O_{h'(a)}$ is a theorem of $Lang(M_2)$.
Let us show that $h' : A_1 \to A_2$ is a morphism of models $M_1$ and $M_2$.
Indeed, if $M_1(\sigma)(a_1, \ldots a_k)$ is defined, then $\sststile{}{} \sigma(O_{a_1}, \ldots O_{a_k}) = O_{M_1(\sigma)(a_1, \ldots a_k)}$ is a theorem of $Lang(M_1)$.
Hence \[ \sststile{}{} \sigma(O_{h'(a_1)}, \ldots O_{h'(a_k)}) = O_{h'(M_1(\sigma)(a_1, \ldots a_k))} \] is a theorem of $Lang(M_2)$.
But \[ \sststile{}{} \sigma(O_{h'(a_1)}, \ldots O_{h'(a_k)}) = O_{M_2(\sigma)(h'(a_1), \ldots h'(a_k))} \] is also a theorem of $Lang(M_2)$.
Hence by \rlem{cl-term}, $h'(M_1(\sigma)(a_1, \ldots a_k)) = M_2(\sigma)(h'(a_1), \ldots h'(a_k))$.

If $(a_1, \ldots a_k) \in M_1(R)$, then $\sststile{}{} R(O_{a_1}, \ldots O_{a_k})$ is a theorem of $Lang(M_1)$.
Hence $\sststile{}{} R(O_{h'(a_1)}, \ldots O_{h'(a_k)})$ is a theorem of $Lang(M_2)$.
Since $M_2$ is a model of $Lang(M_2)$, it follows that $(h'(a_1), \ldots h'(a_k)) \in M_2(R)$.

Thus $h'$ is a morphism of models.
Note that by definition of $h'$, $Lang(h') = h$.
Hence $Lang$ is full.
\end{proof}

Now, let us describe a functor $Syn : \mathbb{T}/\Th_\mathcal{S} \to \Mod{\mathbb{T}}$.
For every $i : \mathbb{T} \to \mathbb{T}'$, let $Syn(i) = i^*(0_{\mathbb{T}'})$, where $0_{\mathbb{T}'}$ is the initial object of $\Mod{\mathbb{T}'}$,
    and $i^* : \Mod{\mathbb{T}'} \to \Mod{\mathbb{T}}$ is the functor that was defined in \cite{alg-tt}.
If $f : \mathbb{T}_1 \to \mathbb{T}_2$ is a morphism of theories under $\mathbb{T}$, then let $Syn(f) = i_1^*(!_{f^*(0_{\mathbb{T}_2})})$,
    where $!_{f^*(0_{\mathbb{T}_2})}$ is the unique morphism $0_{\mathbb{T}_1} \to f^*(0_{\mathbb{T}_2})$.

The construction of initial models of partial Horn theories was given in \cite{PHL}.
Let us repeat it here.
Let $\mathbb{T} = ((\mathcal{S},\mathcal{F},\mathcal{P}),\mathcal{A})$ be a standard partial Horn theory.
First, we define a partial equivalence relations on sets $Term_\mathcal{F}(\varnothing)$ as $t_1 \sim t_2$ if and only if $\sststile{}{} t_1 = t_2$ is a theorem of $\mathbb{T}$.
The interpretation of $R \in \mathcal{P}$ consists of tuples $(t_1, \ldots t_k)$ such that $\sststile{}{} R(t_1, \ldots t_k)$ is derivable in $\mathbb{T}$.
Then $\mathcal{S}$-set $Term_\mathcal{F}(\varnothing)/\!\sim$ has a natural structure of a model of $((\mathcal{S},\mathcal{F},\mathcal{P}),\mathcal{A})$, and this model is initial.

\begin{prop}
$Syn$ is right adjoint to $Lang$.
\end{prop}
\begin{proof}
Let $\epsilon_{\mathbb{T}'} : Lang(Syn(\mathbb{T}')) \to \mathbb{T}'$ be defined as $\epsilon_{\mathbb{T}'}(O_t) = t$.
It is easy to see that $\epsilon_{\mathbb{T}'}$ preserves axioms of $Lang(Syn(\mathbb{T}'))$.
Moreover, $\epsilon$ is natural in $\mathbb{T}'$.
Let us prove that $\epsilon$ is the counit of the adjunction.
Let $f : Lang(M) \to \mathbb{T}'$ be a morphism.
Then we need to show that there is a unique morphism $g : Lang(M) \to Lang(Syn(\mathbb{T}'))$ such that $\epsilon_{\mathbb{T}'} \circ g = f$.
By \rlem{cl-term}, there is a unique $t$ such that $g(O_a) = O_t$.
Since $t = \epsilon_{\mathbb{T}'}(g(O_a)) = f(O_a)$, $g$ must satisfy equation $g(O_a) = O_{f(O_a)}$.
Thus $g$ is unique.
It is easy to see that this $g$ preserves axioms of $Lang(M)$; hence it defines a morphism $g : Lang(M) \to Lang(Syn(\mathbb{T}'))$.
\end{proof}

Let $\mathbb{T}$ be an algebraic dependent type theory.
Then categories $\mathbb{T}/\algtt^0$ and $\mathbb{T}/\ThC$ are isomorphic.
Thus we still have adjoint functors $Lang^0 \dashv Syn^0 : \mathbb{T}/\algtt^0 \to \Mod{\mathbb{T}}$.
If $\mathbb{T}$ is a contextual theory, then category $\mathbb{T}/\algtt^0_{con}$ is a full subcategory of $\mathbb{T}/\algtt^0$.
We have functor $Syn^0_{con} : \mathbb{T}/\algtt^0_{con} \to \Mod{\mathbb{T}}$ which is defined as the composition of the inclusion $\mathbb{T}/\algtt^0_{con} \to \mathbb{T}/\algtt^0$ and $Syn$.
Let us construct its left adjoint $Lang^0_{con} : \Mod{\mathbb{T}} \to \mathbb{T}/\algtt^0_{con}$.

Theory $Lang^0_{con}(M)$ has has function and predicate symbols of $\mathbb{T}$ together with function symbol $O_a : (ctx,n) \to (p,n)$ for every $a \in A_{(p,n)}$, $p \in \{ ty, tm \}$.
For every $a \in A_{(p,n)}$, we define term $C_a$ of sort $(p,n)$ by induction on $n$.
If $n = 0$, then let $C_a = O_a(\emptyCtx)$.
If $n > 0$, then let $C_a = O_a(C_{M(ctx_{p,n})(a)})$.

Axioms of $Lang^0_{con}(M)$ are axioms of $\mathbb{T}$ together with the following sequents:
\begin{align*}
O_a(\Gamma) \downarrow & \sststile{}{\Gamma} \Gamma = C_{M(ctx_{p,n})(a)}\\
& \sststile{}{} \sigma(C_{a_1}, \ldots C_{a_k}) = C_{M(\sigma)(a_1, \ldots a_k)} \\
& \sststile{}{} R(C_{a_1}, \ldots C_{a_k})
\end{align*}
for every $a \in A_s$, every $a_i \in A_{s_i}$,
every $\sigma \in \mathcal{F}$ such that $M(\sigma)(a_1, \ldots a_k)$ is defined,
and every $R \in \mathcal{P}$ such that $(a_1, \ldots a_k) \in M(R)$.

For every morphism $h : M \to N$ of models of $\mathbb{T}$, we can define a morphism $Lang^0_{con}(h) : Lang^0_{con}(M) \to Lang^0_{con}(N)$ of theories under $\mathbb{T}$ as $Lang(h)(O_a(\Gamma)) = O_{h(a)}(\Gamma)$.
Thus $Lang^0_{con}$ is a functor $\Mod{\mathbb{T}} \to \mathbb{T}/\algtt^0_{con}$.

\begin{prop}
$Lang^0_{con}$ is left adjoint to $Syn^0_{con}$.
\end{prop}
\begin{proof}
To prove this, it is enough to show that the composition of $Lang^0_{con}$ and the inclusion $\mathbb{T}/\algtt^0_{con} \to \mathbb{T}/\algtt^0$ is isomorphic to $Lang^0$.
Let $\alpha_M : Lang^0(M) \to Lang^0_{con}(M)$ be defined as $\alpha_M(O_a) = C_a$.
Let $\beta_M : Lang^0_{con}(M) \to Lang^0(M)$ be defined as $\beta_M(O_a(\Gamma)) = O_a|_{\Gamma = C_{M(ctx_{p,n})(a)}}$.
To prove that $\alpha$ and $\beta$ are inverses of each other, we need to show that $\sststile{}{\Gamma} O_a(\Gamma) \cong C_a|_{\Gamma = C_{M(ctx_{p,n})(a)}}$ is a theorem of $Lang^0_{con}(M)$.
But this follows from the first axiom of $Lang^0_{con}(M)$.
\end{proof}

For every algebraic dependent type theory with substitutions $\mathbb{T}$, there are adjoint functors $Lang^1 \dashv Syn^1 : \mathbb{T}/\algtt^1 \to \Mod{\mathbb{T}}$.
For every contextual algebraic dependent type theory with substitutions $\mathbb{T}$, there are adjoint functors $Lang^1_{con} \dashv Syn^1_{con} : \mathbb{T}/\algtt^1_{con} \to \Mod{\mathbb{T}}$.
These functors are defined in the same way as $Lang^0 \dashv Syn^0$ and $Lang^0_{con} \dashv Syn^0_{con}$.

For every morphism of theories $f : \mathbb{T} \to \mathbb{T}'$ there is a functor $f^* : \Mod{\mathbb{T}'} \to \Mod{\mathbb{T}}$ which was consttucted in \cite{alg-tt}.
We also can define functor $f_* : \Mod{\mathbb{T}} \to \Mod{\mathbb{T}'}$ as $f_*(M) = Syn(Lang(M) \amalg_{\mathbb{T}} \mathbb{T}')$.
It was shown in \cite{PHL} that $f_*$ is left adjoint to $f^*$.
This theorem was proved there only for a weaker notion of morphisms of theories, but the proof also works for general morphisms as defined in \cite{alg-tt}.

Functor $f_*$ can be used to present a model of a theory by generators and relations.
Let $\mathbb{T}$ be a fixed $\mathcal{S}$-theory.
Note that models of the empty theory are just $\mathcal{S}$-sets.
If $f : 0 \to \mathbb{T}$ is the unique morphism from the empty theory, then $f^*(M)$ is just the underlying $\mathcal{S}$-set of $M$,
    and $f_*(X)$ is the free model of $\mathbb{T}$ on $\mathcal{S}$-set $X$.
We will denote this free model by $F(X)$.
If $R$ is a set of axioms in the language of theory $Lang(X) \amalg \mathbb{T}$,
    then let $F(X,R)$ be a model of $\mathbb{T}$ defined as $Syn(Lang(X) \amalg \mathbb{T} \cup R)$.
By definition of $Syn$, to construct a morphism $F(X,R) \to M$ it is necessary and sufficient
    to construct a morphism from $X$ to the underlying $\mathcal{S}$-set of $M$ such that relations from $R$ are true in $M$.

Sometimes we will omit the set of generators if it can be inferred from the set of relations.
For examples, we will write $F(\{\,\vdash p : Id(A,a,a')\,\})$ for the model $F(\{\,a : (tm,0), a' : (tm,0), A : (ty,0), p : (tm,0)\,\}, \{\,ty(p) = Id(A,a,a')\,\})$.
Another examples is $F(\{\,A_1, \ldots A_n \vdash a : A\,\})$ which equals to $F(\{\,A_i : (ty,i), A : (ty,n), a : (tm,n)\,\}, \{\,ty(a) = A, ft^{i+1}(A) = A_{n-i}\,\})$.
Thus this model is isomorphic to the free model $F(\{\,a : (tm,n)\,\})$.

\section{Theories with an interval type}

In this section we describe the theory of an interval type.
We describe several constructions in this theory which we will need later.
In partical, we will show that theories with an interval type and path types also have $Id$ types.

Let $I_0$ be a theory with function symbols $I : (ty,0)$, $left : (tm,0)$, $right : (tm,0)$, and $coe : (ty,1) \times (tm,0) \times (tm,0) \to (tm,0)$, and the following axioms:
\begin{center}
\AxiomC{}
\UnaryInfC{$\vdash I\ type$}
\DisplayProof
\quad
\AxiomC{}
\UnaryInfC{$\vdash left : I$}
\DisplayProof
\quad
\AxiomC{}
\UnaryInfC{$\vdash right : I$}
\DisplayProof
\end{center}

\medskip
\begin{center}
\AxiomC{$I \vdash D\ type$}
\AxiomC{$\vdash d : D[left]$}
\AxiomC{$\vdash i : I$}
\TrinaryInfC{$\vdash coe(D, d, i) : D[i]$}
\DisplayProof
\qquad
\AxiomC{$I \vdash D\ type$}
\AxiomC{$\vdash d : D[left]$}
\BinaryInfC{$\vdash coe(D, d, left) \deq d$}
\DisplayProof
\end{center}
Let $I$ be the regularization of $I_0$.
We will work with theories under $I$, which we call theories with \emph{an interval type}.

\subsection{Path types}

Let $HPath$ be a regular theory with function symbols $Path : (ty,n) \times (tm,n) \times (tm,n) \to (ty,n)$,
    $path : (tm,n+1) \to (tm,n)$, and $at : (tm,n) \times (tm,n) \times (tm,n) \times (tm,n) \to (tm,n)$, and the following axioms:
\begin{center}
\AxiomC{$\Gamma \vdash a : A$}
\AxiomC{$\Gamma \vdash a' : A$}
\BinaryInfC{$\Gamma \vdash Path(A, a, a')\ type$}
\DisplayProof
\quad
\AxiomC{$\Gamma \vdash A\ type$}
\AxiomC{$\Gamma, I \vdash a : A\!\uparrow$}
\BinaryInfC{$\Gamma \vdash path(a) : Path(A, a[left], a[right])$}
\DisplayProof
\end{center}

\smallskip
\begin{center}
\AxiomC{$\Gamma \vdash p : Path(A, a, a')$}
\AxiomC{$\Gamma \vdash i : I$}
\BinaryInfC{$\Gamma \vdash at(a, a', p, i) : A$}
\DisplayProof
\end{center}

\smallskip
\begin{center}
\AxiomC{$\Gamma \vdash A\ type$}
\AxiomC{$\Gamma, I \vdash a : A\!\uparrow$}
\AxiomC{$\Gamma \vdash i : I$}
\TrinaryInfC{$\Gamma \vdash at(a[left], a[right], path(a), i) \deq a[i]$}
\DisplayProof
\end{center}

\smallskip
\begin{center}
\AxiomC{$\Gamma \vdash p : Path(A, a, a')$}
\UnaryInfC{$\Gamma \vdash at(a, a', p, left) \deq a$}
\DisplayProof
\quad
\AxiomC{$\Gamma \vdash p : Path(A, a, a')$}
\UnaryInfC{$\Gamma \vdash at(a, a', p, right) \deq a'$}
\DisplayProof
\end{center}
We will work with theories under $HPath$, which we call theories with \emph{homogeneous path types}.

\todo{Add some constructions}

\subsection{Univalence}

If $\Gamma \vdash A\ type$, $\Gamma \vdash B\ type$, $\Gamma \vdash C\ type$, $\Gamma, A \vdash f : B\!\uparrow$ and $\Gamma, B \vdash g : C\!\uparrow$,
    then we will write $g \circ f$ for term $\Gamma, A \vdash subst(g, v_n, \ldots v_1, f) : C\!\uparrow$.

We will consider regular theory $UA$ under $HPath$, which has additional symbols
\[ iso : (ty,n)^2 \times (tm,n+1)^4 \times (tm,n) \to (ty,n) \]
and the following axioms:
\medskip
\begin{center}
\def\extraVskip{0.5pt}
\Axiom$\fCenter \Gamma \vdash A\ type$
\noLine
\UnaryInf$\fCenter \Gamma \vdash B\ type$
\Axiom$\fCenter \Gamma, A \vdash f : B\!\uparrow$
\noLine
\UnaryInf$\fCenter \Gamma, B \vdash g : A\!\uparrow$
\Axiom$\fCenter \Gamma, A \vdash p : Path(A\!\uparrow, g \circ f, v_0)$
\noLine
\UnaryInf$\fCenter \Gamma, B \vdash q : Path(B\!\uparrow, f \circ g, v_0)$
\Axiom$\fCenter \Gamma \vdash i : I$
\def\extraVskip{2pt}
\QuaternaryInfC{$\Gamma \vdash iso(A, B, f, g, p, q, i)\ type$}
\DisplayProof
\end{center}

\medskip
\begin{center}
\def\extraVskip{0.5pt}
\Axiom$\fCenter \Gamma \vdash A\ type$
\noLine
\UnaryInf$\fCenter \Gamma \vdash B\ type$
\Axiom$\fCenter \Gamma, A \vdash f : B\!\uparrow$
\noLine
\UnaryInf$\fCenter \Gamma, B \vdash g : A\!\uparrow$
\Axiom$\fCenter \Gamma, A \vdash p : Path(A\!\uparrow, g \circ f, v_0)$
\noLine
\UnaryInf$\fCenter \Gamma, B \vdash q : Path(B\!\uparrow, f \circ g, v_0)$
\def\extraVskip{2pt}
\TrinaryInfC{$\Gamma \vdash iso(A, B, f, g, p, q, left) \deq A$}
\DisplayProof
\end{center}

\medskip
\begin{center}
\def\extraVskip{0.5pt}
\Axiom$\fCenter \Gamma \vdash A\ type$
\noLine
\UnaryInf$\fCenter \Gamma \vdash B\ type$
\Axiom$\fCenter \Gamma, A \vdash f : B\!\uparrow$
\noLine
\UnaryInf$\fCenter \Gamma, B \vdash g : A\!\uparrow$
\Axiom$\fCenter \Gamma, A \vdash p : Path(A\!\uparrow, g \circ f, v_0)$
\noLine
\UnaryInf$\fCenter \Gamma, B \vdash q : Path(B\!\uparrow, f \circ g, v_0)$
\def\extraVskip{2pt}
\TrinaryInfC{$\Gamma \vdash iso(A, B, f, g, p, q, right) \deq B$}
\DisplayProof
\end{center}

\medskip
\begin{center}
\def\extraVskip{0.5pt}
\Axiom$\fCenter \Gamma \vdash A\ type$
\noLine
\UnaryInf$\fCenter \Gamma \vdash B\ type$
\Axiom$\fCenter \Gamma, A \vdash f : B\!\uparrow$
\noLine
\UnaryInf$\fCenter \Gamma, B \vdash g : A\!\uparrow$
\Axiom$\fCenter \Gamma, A \vdash p : Path(A\!\uparrow, g \circ f, v_0)$
\noLine
\UnaryInf$\fCenter \Gamma, B \vdash q : Path(B\!\uparrow, f \circ g, v_0)$
\def\extraVskip{2pt}
\TrinaryInfC{$\Gamma \vdash coe(iso(A\!\uparrow, B\!\uparrow, f\!\uparrow_1, g\!\uparrow_1, p\!\uparrow_1, q\!\uparrow_1, v_0), a, right) \deq f[a]$}
\DisplayProof
\end{center}
\medskip

This theory is similar to the univalence axiom, but it is defined for all types.
The univalence axiom for a universe follows from the assumption that this universe is closed under $iso$.

\todo{Add some constructions}

\section{A model structure on models of theories with an interval type}

In this section we construct a model structure on the category of models of an algebraic type theory with enough additional structure.

Let $\mathbb{T}$ be a theory under $UA$.
Then we define a model structure on the category of models of $\mathbb{T}$.
For an introduction to the theory of model categories see \cite{hovey}.
We will need the following proposition to construct our model structure:

\begin{prop}[model-cat]
Suppose that $\C$ is a complete and cocomplete category, $\we$ is a class of morphisms of $\C$, and $\I$, $\J$ are sets of morphisms of $\C$.
Then $\C$ is a cofibrantly generated model category with $\I$ as the set of generating cofibrations,
$\J$ as the set of generating trivial cofibrations, and $\we$ as the class of weak equivalences if and only if the following conditions are satisfied:
\begin{description}
\item[(A1)] The domains of maps in $\I$ and $\J$ are small relative to $\Icell$ and $\Jcell$ respectively.
\item[(A2)] $\we$ has 2-out-of-3 property and is closed under retracts.
\item[(A3)] $\Iinj \subseteq \we$.
\item[(A4)] $\Jcell \subseteq \we \cap \Icof$.
\item[(A5)] Either $\Jinj \cap \we \subseteq \Iinj$ or $\Icof \cap \we \subseteq \Jcof$.
\end{description}
\end{prop}

If $H = (I, A_1, \ldots A_n \vdash) \in X_{(ctx,n+1)}$, then for every $c \in \{ left, right \}$, let $c^*(H) = (A_1[c], \ldots A_n[c] \vdash) \in X_{(ctx,n)}$.
A \emph{homotopy between} contexts $\Gamma,\Delta \in X_{(ctx,n)}$ is a context $H \in X_{(ctx,n+1)}$ such that $\vdash ft^n(H) \equiv I$, $left^*(H) = \Gamma$ and $right^*(H) = \Delta$.
Contexts $\Gamma$ and $\Delta$ are \emph{homotopic} if there is a homotopy between them.
In this case we will write $\Gamma \sim \Delta$.

A \emph{relative homotopy between} types $(\Gamma \vdash A), (\Gamma \vdash B) \in X_{(ty,n)}$ is a type $(\Gamma, I \vdash H) \in X_{(ty,n+1)}$ such that $H[left] = A$ and $H[right] = B$.
Types $A$ and $B$ are \emph{homotopic relative to $(ctx,n)$} if there is a relative homotopy between them.
In this case we will write $A \sim_r B$.

A \emph{relative homotopy between} terms $(\Gamma \vdash a : A), (\Gamma \vdash b : A) \in X_{(tm,n)}$ is a term $(\Gamma, I \vdash h : A\!\uparrow) \in X_{(tm,n+1)}$ such that $h[left] = a$ and $h[right] = b$.
Terms $a$ and $b$ are \emph{homotopic relative to $(ty,n)$} if there is a relative homotopy between them.
In this case we will write $a \sim_r b$.

\begin{remark}[types-hom-ua]
To define a relative homotopy between types $(\Gamma \vdash A), (\Gamma \vdash B) \in X_{(ty,n)}$,
    it is enought to specify terms $(\Gamma, A \vdash f : B\!\uparrow), (\Gamma, B \vdash g : A\!\uparrow) \in X_{(tm,n+1)}$
    such that $g \circ f \sim_r v_0(A)$ and $f \circ g \sim_r v_0(B)$.
Then we can define a relative homotopy $\Gamma, I \vdash iso(A\!\uparrow, B\!\uparrow, f\!\uparrow_1, g\!\uparrow_1, path(h_1\!\!\uparrow_2), path(h_2\!\!\uparrow_2), v_0)$ between $A$ and $B$,
    where $h_1$ is a relative homotopy between $g \circ f$ and $v_0(A)$, and $h_2$ is a relative homotopy between $f \circ g$ and $v_0(B)$.
\end{remark}

There is a bijection between morphisms $F(\{\,A : (p,n)\,\}) \to X$ and elements $X_{(p,n)}$.
We can directly describe homotopies between morphisms $f,g : F(\{\,A : (p,n)\,\}) \to X$.
For this, we define (relative) cylinder objects $C$ together with maps $i^0,i^1 : F(\{\,A : (p,n)\,\}) \to C$.
Then a homotopy between $f$ and $g$ can be described as a map $h : C \to X$ such that $h \circ i^0 = f$ and $h \circ i^1 = g$.

A \emph{cylinder object} of sort $(ctx,n)$ is $F(\{\,H : (ctx,n+1)\,\}, \{\,ft^n(H) = I(\emptyCtx)\,\})$.
We will denote this object by $Cyl_{(ctx,n)}$.
Inclusions $i^0_{(ctx,n)}, i^1_{(ctx,n)} : F(\{\,\Gamma : (ctx,n)\,\}) \to Cyl_{(ctx,n)}$ are defined as $i^k_{(ctx,n)}(\Gamma) = c_k^*(H)$, where $c_0 = left$ and $c_1 = right$.

A \emph{relative cylinder object} of sort $(ty,n)$ is $F(\{\,H : (ty,n+1)\,\}, \{\,ft(H) = I(ft^2(H))\,\})$.
We will denote this object by $RCyl_{(ty,n)}$.
Inclusions $i^0_{(ty,n)}, i^1_{(ty,n)} : F(\{\,A : (ty,n)\,\}) \to RCyl_{(ty,n)}$ are defined as $i^k_{(ty,n)}(A) = H[c_k]$.

A \emph{relative cylinder object} of sort $(tm,n)$ is $F(\{\,h : (tm,n+1), A : (ty,n)\,\}, \{\,ty(h) \allowbreak = (I(ft(A)) \vdash A\!\!\uparrow)\,\})$.
We will denote this object by $RCyl_{(tm,n)}$.
Inclusions $i^0_{(tm,n)}, i^1_{(tm,n)} : F(\{\,a : (tm,n)\,\}) \to RCyl_{(tm,n)}$ are defined as $i^k_{(tm,n)}(a) = h[c_k]$.

Let $e : \{ ty, tm \} \to \{ ctx, ty \}$ be the function defined as $e(ty) = ctx$, $e(tm) = ty$.
Also we define $d_p : (p,n) \to (e(p),n)$ as $d_{ty} = ft$, $d_{tm} = ty$.
If $p \in \{ ty, tm \}$, then let $\I_p$ be the set of maps of the form $\delta_{(p,n)} : F(\{\,A : (e(p),n)\,\}) \to F(\{\,a : (p,n)\,\})$, $\delta_{(p,n)}(A) = d_p(a)$.
We define the set $\I$ of generating cofibrations to be the union $\I_{ty} \cup \I_{tm}$.
Let $\J_p$ be the set of arrows of the form $i^0_{(p,n)}$.
We define the set $\J$ of generating trivial cofibrations to be the union $\J_{ty} \cup \J_{tm}$.

\begin{remark}
A relative homotopy between $f,g : F(\{\,A : (p,n)\,\}) \to X$ exists only if $f \circ \delta_{(p,n)} = g \circ \delta_{(p,n)}$, and in this case a homotopy $h$ is a map such that the following triangle commutes:
\[ \xymatrix{ F(\{\,a : (p,n)\,\}) \amalg_{F(\{\,A : (e(p),n)\,\})} F(\{\,a : (p,n)\,\}) \ar[d]_{[i^0,i^1]} \ar[r]^-{[f,g]} & X \\
              RCyl_{(p,n)} \ar@{-->}[ur]_h
            } \]
\end{remark}

Given two morphisms $f \in \I$ and $g : X \to Y$, we say that $f$ \emph{has the left lifting property (LLP) up to $\sim_r$} with respect to $g$,
and $g$ \emph{has the right lifting property (RLP) up to $\sim_r$} with respect to $f$ if for every commutative square of the form
\[ \xymatrix{ F(\{\,A : (e(p),n)\,\}) \ar[r]^-u \ar@{}[dr]|(.7){\sim_r} \ar[d]_f & X \ar[d]^g \\
              F(\{\,a : (p,n)\,\}) \ar[r]_-v \ar@{-->}[ur]^h                     & Y,
            } \]
there is a dotted arrow $h : F(\{\,a : (p,n)\,\}) \to X$ such that $h \circ f = u$ and $(g \circ h) \sim_r v$.
We define the set $\we$ of weak equivalences as the set of maps that have RLP up to $\sim_r$ with respect to every map in $\I$.
Thus a map $g : X \to Y$ is a weak equivalence if and only if the following conditions are satisfied:
\begin{enumerate}
\item For every context $\Gamma \in X_{(ctx,n)}$ and every type $A \in Y_{(ty,n)}$ in context $g(\Gamma)$, there exists a type $A' \in X_{(ty,n)}$ in context $\Gamma$ such that $g(A')$ and $A$ are homotopic relative to $(ctx,n)$.
    This is true if and only if $g$ has RLP up to $\sim_r$ with respect to every map in $\I_{ty}$.
    In this case we say that $g$ is \emph{essentially surjective on types}.
\item For every type $A \in X_{(ty,n)}$ and every term $a \in Y_{(tm,n)}$ of type $g(A)$, there exists a term $a' \in X_{(tm,n)}$ of type $A$ such that $g(a')$ and $a$ are homotopic relative to $(ty,n)$.
    This is true if and only if $g$ has RLP up to $\sim_r$ with respect to every map in $\I_{tm}$.
    In this case we say that $g$ is \emph{essentially surjective on terms}.
\end{enumerate}

\begin{lem}
Relation $\sim_r$ is reflexive, symmetric, transitive and closed under composition.
\end{lem}
\begin{proof}
If $h$ is a homotopy between $f$ and $g$, then $k \circ h$ is a homotopy between $k \circ f$ and $k \circ g$.

Reflexivity follows from the fact that there is a map $r : RCyl_{(p,n)}(h) \to F(\{ a : (p,n) \})$ such that $r \circ i^0_{(p,n)} = r \circ i^1_{(p,n)} = id$.
This map is defined as $r(h) = (I(ctx(a)) \vdash a\!\!\uparrow)$.
Then $f \circ r$ is a homotopy between $f$ and $f$.

Let $h_1, h_2 : RCyl_{(tm,n)}(h) \to X$ be relative homotopies such that $h_1 \circ i^1 = h_2 \circ i^0$.
Then we define a relative homotopy $h_3$ between $h_1 \circ i^0$ and $h_2 \circ i^1$.
Let $h_3(h) = (ctx(h_1(h)) \vdash at(h_1[left]\!\uparrow, h_2[right]\!\uparrow, coe(Path(ty(h_1), h_1[left]\!\uparrow, h_2), path(h_1), right)\!\uparrow, v_0))$.

Let $H_1, H_2 : RCyl_{(ty,n)}(H) \to X$ be relative homotopies such that $H_1 \circ i^1 = H_2 \circ i^0$.
Then we define a relative homotopy $H_3$ between $H_1 \circ i^0$ and $H_2 \circ i^1$.
Let $H_3(H) = $.
\todo{Finish the proof}
\end{proof}

\begin{lem}
Let $k : X \to Y$ be a weak equivalence.
If $f,g : F(\{\,a : (p,n)\,\}) \to X$ are maps such that $k \circ f \sim_r k \circ g$, then $f \sim_r g$.
\end{lem}
\begin{proof}
\end{proof}

\bibliographystyle{amsplain}
\bibliography{ref}

\end{document}
