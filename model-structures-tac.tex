\documentclass{tac}

% TAC style, 12 pp, Xy-pic ver 3.10, TeX Live 2015/dev/Debian

\usepackage{etex}
\usepackage{hyperref}
\usepackage{amssymb}
\usepackage{amsmath}
\usepackage[all]{xy}
\usepackage{ifthen}
\usepackage{xargs}
\usepackage{upgreek}

\hypersetup{colorlinks=true,linkcolor=blue}

\author{Valery Isaev}

\title{On fibrant objects in model categories}

\copyrightyear{2016}

\keywords{Quillen model structures, fibrant objects}
\amsclass{55U35}

\newcommand{\newref}[4][]{
\ifthenelse{\equal{#1}{}}{\newtheorem{h#2}[hthm]{#4}}{\newtheorem{h#2}{#4}[#1]}
\expandafter\newcommand\csname r#2\endcsname[1]{#3~\ref{#2:##1}}
\expandafter\newcommand\csname R#2\endcsname[1]{#4~\ref{#2:##1}}
\expandafter\newcommand\csname n#2\endcsname[1]{\ref{#2:##1}}
\newenvironmentx{#2}[2][1=,2=]{
\ifthenelse{\equal{##2}{}}{\begin{h#2}}{\begin{h#2}[##2]}
\ifthenelse{\equal{##1}{}}{}{\label{#2:##1}}
}{\end{h#2}}
}

\newref[section]{thm}{Theorem}{Theorem}
\newref{lem}{Lemma}{Lemma}
\newref{prop}{Proposition}{Proposition}
\newref{cor}{Corollary}{Corollary}
\newref{rem}{Remark}{Remark}
\newref{exmp}{Example}{Example}

\theoremstyle{definition}
\newref{defn}{Definition}{Definition}

\newcommand{\we}{\mathcal{W}}
\newcommand{\fib}{\mathcal{F}}
\newcommand{\cof}{\mathcal{C}}
\newcommand{\cat}[1]{\mathbf{#1}}
\newcommand{\C}{\cat{C}}
\newcommand{\D}{\cat{D}}

\newcommand{\I}{\mathrm{I}}
\newcommand{\J}{\mathrm{J}}
\newcommand{\class}[2]{#1\text{-}\mathrm{#2}}
\newcommand{\Iinj}[1][\I]{\class{#1}{inj}}
\newcommand{\Icell}[1][\I]{\class{#1}{cell}}
\newcommand{\Icof}[1][\I]{\class{#1}{cof}}
\newcommand{\Jinj}[1][]{\Iinj[\J#1]}
\newcommand{\Jcell}[1][]{\Icell[\J#1]}
\newcommand{\Jcof}[1][]{\Icof[\J#1]}
\newcommand{\cyli}{\upgamma}

\newcommand{\po}[1][dr]{\save*!/#1+1.2pc/#1:(1,-1)@^{|-}\restore}

\begin{document}

\maketitle

\begin{abstract}
In this paper, we study properties of maps between fibrant objects in model categories.
We give a characterization of weak equivalences between fibrant objects.
If every object of a model category is fibrant, then we give a simple description of a set of generating cofibrations.
We show that to construct such a model structure it is enough to check some relatively simple conditions.
\end{abstract}

\section{Introduction}

The framework of model categories, introduced in \cite{quillen}, is an important tool in homotopy theory.
In this paper, we study properties of maps between fibrant objects.
We will show that weak equivalence between fibrant objects have a simple characterization in terms of generating cofibrations (\rprop{min-we}),
which is similar to the characterization described in \cite{vogt-we}.
In particular, if every object of a model category is fibrant, then we get a complete characterization of weak equivalences.
We will prove several properties of model categories using this characterization.

For example, we will show that a right Quillen functor $U : \D \to \C$ reflects weak equivalence between fibrant objects if and only if
$F(\cof)$ generates the class of cofibrations of $\D$ in a certain sense, where $F$ is a left adjoint to $U$ and $\cof$ is the class of cofibrations of $\C$ (\rprop{refl-fibr-char}).
This proposition is useful since this condition on right Quillen functors appears in the characterization of Quillen equivalences. 

If we fix the class of cofibrations, then the class of weak equivalences for which there exists model structure in which all objects are fibrant is unique if it exists,
so this raises a natural question: when does such a model structure exist?
We prove several necessary and sufficient conditions for this to be true.
Examples of applications of these constructions are model structures on strict $\omega$-categories (which was constructed in \cite{folk}) and topological spaces.
Examples of new model structures constructed by means of results of this paper will be given in \cite{alg-models}.

If a model structure in which all objects are fibrant exists, then the class of weak equivalences of this model structure
is the smallest class of weak equivalences among model structures with this class of cofibrations.
Thus a model category in which all objects are fibrant is left-determinant (as defined in \cite{left-det}).
But the converse is not true.
For example, the category of simplicial sets (as well as every Grothendieck topos \cite{cisinski})
carries a left-determinant model structure with monomorphisms as cofibrations,
but a model structure on this category in which all objects are fibrant and with this class of cofibrations does not exist.
To see this, consider a cylinder $C$ for the terminal object.
Then $C \amalg_{\Delta[0]} C$ does not have a map from $\Delta[1]$ which
maps faces of this simplex to points of $C$ which were not amalgamated.
Thus $C \amalg_{\Delta[0]} C$ cannot be fibrant.

Thus the class of model structures in which all objects are fibrant is much narrower than the class of left-determinant model structures.
On the other hand model categories in which all objects are fibrant have properties not shared by left-determinant ones.
Often a set of generating trivial cofibrations is defined using cardinality bounds on domains and codomains of maps.
For example, left Bousfield localizations are often constructed in this way.
In general, there is no useful explicit description of generating trivial cofibrations, but if all objects of a model category are fibrant,
then we can give a simple explicit construction of a set of generating trivial cofibrations (\rprop{min-cof-gen}).
Also, as we already noted, model categories in which all objects are fibrant have a simple characterization of weak equivalences in terms of generating cofibrations.

Let us discuss the main obstacles that we face when we try to construct such model structure in which all objects are fibrant.
Since we have an explicit characterization of weak equivalences in terms of cofibrations,
we just need to check that the classes of weak equivalences, cofibrations and fibrations satisfy the required conditions.
The first problem is to prove that this class of weak equivalences satisfies the 2-out-of-3 property.
The most difficult part is to show that if $f : X \to Y$ and $g : Y \to Z$ are maps such that $f$ and $g \circ f$ are weak equivalences, then $g$ is also a weak equivalence.
For example, it is one of the main obstacles in \cite{folk} and \cite{alg-models}.

\Rthm{main} implies that if we can prove the 2-out-of-3 property for weak equivalences,
then to construct the model structure, we just need to prove that the maps in $\Jcell[_\I]$ are weak equivalences,
where $\J_\I$ is the set of generating cofibrations defined in section~\ref{sec:exist}.
A similar problem occurs when we try to construct a transferred model structure.
If $F : \C \to \D$ is a left adjoint functor and $\C$ is a cofibrantly generated model category, then to construct a model structure on $\D$,
we just need to prove that maps in $\Icell[F(\J)]$ are weak equivalences, where $\J$ is a set of generating trivial cofibrations of $\C$.
This condition holds if we can construct functorial path objects for fibrant objects.
Similar result holds in our case too (\rcor{main-path}).
Also, analogous result holds if we there are cylinder objects instead of path objects (\rcor{main-cyl}).

Our main result gives a necessary and sufficient condition for the existence of a model structure that satisfies some conditions.
Thus we may compare our theorem to other analogous criterion that exist in the literature.
We already mentioned the transferred model structures.
Another such result is proved in \cite{olschok}, which generalizes a theorem of Cisinski proved in \cite{cisinski}.
First of all, the results of Olschok applies to model categories in which all objects are cofibrant and our results applies to model categories in which all objects are fibrant.
Since there are not many model categories in which both of these conditions hold, the two results rarely can be applied to construct the same model structure.
The results of Olschok use Jeff Smith's theorem, and hence applies only to locally presentable categories.
Our results applies to all bicomplete categories and give a model structure with explicitly described set of generating trivial cofibrations.
Finally, the class of weak equivalences is completely determined by the class of cofibrations in our setting
while the theorems of Olschok give us some freedom in choosing the class of weak equivalences.

The paper is organized as follows.
In section 2, we recall basic properties of model categories and establish the notation that we use in this paper.
In section 3, we prove several results that are related to fibrant objects, fibrations and trivial cofibrations between them.
First of all, we give a characterization of weak equivalences between fibrant objects.
Then we prove that if the fibrant objects are determined by a set $S$ of trivial cofibrations,
then trivial cofibrations and fibrations with fibrant codomains are determined by a set of generating cofibrations which is explicitly defined in terms of $S$.
Finally, we prove a necessary and sufficient condition for a right Quillen functor to reflect weak equivalences between fibrant objects.
In section 4, we demonstrate a method of constructing model structures in which all objects are fibrant.
We prove several necessary and sufficient conditions for such a model structure to exist.

\section{Preliminaries}
\label{sec:preliminaries}

In this section, we recall several standard definition and properties related to model categories.
Most of the propositions are either standard or a variation of such propositions and will be given either without a proof or with a short proof.
The definition that we use are also mostly standard.
We want to apply these definitions in categories without model structure, so we need a slightly more general definitions of cylinder and path objects.

Let $\C$ be a category and let $V$ be an object of $\C$.
Then a \emph{cylinder object} for $V$ is an object $C(V)$ together with maps $\cyli_0,\cyli_1 : V \to C(V)$.
If $i : U \to V$ is a morphism of $\C$, then a \emph{relative cylinder object} for $\C$
is a cylinder object $(C_U(V),\cyli_0,\cyli_1)$ for $V$ such that $\cyli_0 \circ i = \cyli_1 \circ i$.
If $\C$ has the initial object $0$, then a relative cylinder object for $0 \to V$ is just a cylinder object for $V$.
A morphism of cylinder objects $C(V_1)$ and $C(V_2)$ is a pair of maps $f : V_1 \to V_2$ and $C(f) : C(V_1) \to C(V_2)$ which commute with $\cyli_0$ and $\cyli_1$.
Note that we do not require that there exists a map $s : C(V) \to V$ such that $s \circ \cyli_0 = s \circ \cyli_1 = id_V$.
We also do not require maps $\cyli_0,\cyli_1$ to be weak equivalences in any sense.

A \emph{(left) homotopy} (with respect to $C(V)$) between maps $f,g : V \to X$ is a map $h : C(V) \to X$ such that $h \circ \cyli_0 = f$ and $h \circ \cyli_1 = g$.
Maps are \emph{homotopic} if there is a homotopy between them.
Maps are \emph{homotopic relative to $i : U \to V$} (with respect to $C_U(V)$) if there is a homotopy with respect to $C_U(V)$ between them.
Note that maps are homotopic relative to $i$ only if $f \circ i = g \circ i$.
If maps $f$ and $g$ are homotopic, then we write $f \sim g$, and if they are homotopic relative to $i$, then we write $f \sim_i g$.

Note that $\sim$ is reflexive if and only if there exists a map $s : C_U(V) \to V$ such that $s \circ \cyli_0 = s \circ \cyli_1 = id_V$.
In this case we will say that $C_U(V)$ is \emph{reflexive}.
Relation $\sim^r$ is reflexive if and only if there exists a map $t : X \to P(X)$ such that $p_0 \circ t = p_1 \circ t = id_X$.
In this case we will say that $P(X)$ is \emph{reflexive}.
If $P(X)$ is reflexive, then we say that a right homotopy $h : V \to P(X)$ between $f,g : V \to X$
is \emph{constant on $i : U \to V$} if $h \circ i = t \circ f \circ i$.
In this case, we write $f \sim^r_i g$ and say that $f$ and $g$ are \emph{homotopic relative to $i$}.

Let $\C$ be a category and let $X$ be an object of $\C$.
A \emph{path object} for $X$ is an object $P(X)$ together with maps $p_0,p_1 : P(X) \to X$.
A \emph{(right) homotopy} between morphisms $f,g : V \to X$ is a map
$h : V \to P(X)$ such that $p_0 \circ h = f$ and $p_1 \circ h = g$.
We say that $f$ and $g$ are right homotopic and write $f \sim^r g$
if there exists a right homotopy $h : V \to P(X)$ between them.
A morphism of path objects $P(X)$ and $P(Y)$ is a pair of maps $f : X \to Y$ and $P(f) : P(X) \to P(Y)$ which commute with $p_0$ and $p_1$.

Let $V,Y$ be objects of a category $\C$ and $R$ some relation on the set $\C(V,Y)$.
Given two morphisms $f : U \to V$ and $g : X \to Y$, we say that $f$ \emph{has the left lifting property (LLP) up to $R$} with respect to $g$,
and $g$ \emph{has the right lifting property (RLP) up to $R$} with respect to $f$ if for every commutative square of the form
\[ \xymatrix{ U \ar[r]^u \ar@{}[dr]|(.7){R} \ar[d]_f & X \ar[d]^g \\
              V \ar[r]_v \ar@{-->}[ur]^h             & Y,
            } \]
there is a dotted arrow $h : V \to X$ such that $h \circ f = u$ and $(g \circ h)\,R\,v$.
We say a map $f$ has the right lifting property up to $R$ with respect to
an object $V$ if it has this property with respect to the map $0 \to V$.
Given a morphism $f$ and a set of morphisms $\I$, the map $f$ has the left (right) lifting property up to $R$
with respect to $\I$ if it has this property with respect to all morphisms in $\I$.
Note that a map has the right (left) lifting property if it has this property up to the equality relation.
Often, we need to talk about maps which have RLP up to $\sim_i$ with respect to some map $i$.
In this case, we will say that a map has RLP up to relative homotopy with respect to $i$.

Let $R$ be the maximal relation on the set $\C(V,Y)$, that is for every $f_1,f_2 : V \to Y$, we have $f_1\,R\,f_2$.
We will say that $g : X \to Y$ is \emph{pure} with respect to $f : U \to V$ if $g$ has RLP up to $R$ with respect to $f$.
Thus $g$ is pure with respect to $f$ if and only if every square as above has a lift in which the top triangle commutes but the lower need not.
The notion of pure morphism is (formally) similar to the concept of
$\lambda$-pure morphism, used in the theory of accessible categories.

We list a few elementary properties of pure morphisms in the following proposition:

\begin{prop}
The following holds in every category $\C$:
\begin{enumerate}
\item If $g$ has RLP up to some relation with respect to $f$, then $g$ is pure with respect to $f$.
\item Pure maps are closed under composition.
\item If $f : X \to Y$ and $g : Y \to Z$ are maps such that $g \circ f$ is pure, then $f$ is also pure.
\item Every split monomorphism is pure with respect to all maps.
\item If a map is pure with respect to itself, then it is a split monomorphism.
\end{enumerate}
\end{prop}

Model categories were introduced in \cite{quillen}.
For an introduction to the theory of model categories see \cite{hirschhorn,hovey}.
We will denote the sets of cofibrations, fibrations and weak equivalences of a model category by $\cof$, $\fib$, and $\we$, respectively.

Let $\C$ be a model category.
Then there is a reflexive relative cylinder object $C_U(V)$ for every map $i : U \to V$ of $\C$.
There is also a reflexive path object $P(X)$ for every object $X$ of $\C$.
The following proposition is standard:
\begin{prop}[path-cyl]
If $i : U \to V$ is a cofibration and $X$ is a fibrant object, then maps $f,g : V \to X$ are left homotopic relative to $i$ if and only if they are right homotopic relative to $i$.
This homotopy relation is an equivalence relation.
\end{prop}

The definitions of a deformation retract and a strong deformation retract are standard.
A map $f : X \to Y$ is \emph{an inclusion of a deformation retract} if there is
a map $g : Y \to X$ such that $g \circ f = id_X$ and $f \circ g \sim id_Y$.
A map $f : X \to Y$ is \emph{an inclusion of a strong deformation retract} if the homotopy is relative to $f$.

The following lemmas are slight generalizations of standard properties of model categories.

\begin{lem}[hom-ext][Homotopy extension property]
Let $\C$ be a category, and let $i : U \to V$ be a morphism of $\C$.
Suppose that for an object $X$ of $\C$, there exists a path object $p_0,p_1 : P(X) \to X$ such that $p_0$ has RLP with respect to $i$.

Let $i : U \to V \in \cof$, $u : U \to X$, and $v : V \to X$ be maps, and let $h : U \to P(X)$ be a homotopy between $v \circ i$ and $u$.
Then there exists a map $v' : V \to X$ and a homotopy $h' : V \to P(X)$ between $v$ and $v'$ such that $h = h' \circ i$.
\end{lem}
\proof
Let $h : U \to P(X)$ be a homotopy between $v \circ i$ and $u$.
Consider the following diagram:
\[ \xymatrix{ U \ar[r]^-h \ar[d]_i & P(X) \ar[d]^{p_0} \\
              V \ar[r]_v & X
            } \]
By assumption, we have a lift $h' : V \to P(X)$.
Then we can define $v'$ as $p_1 \circ h'$.
\endproof

This lemma can be illustrated as follows:
\[ \xymatrix{ U \ar[r]^u \ar[d]_i \ar@{}[dr]|(.3){\sim^r} & X \\
              V \ar[ur]_v &
            }
\qquad \qquad
   \xymatrix{ U \ar[r]^u \ar[d]_i \ar@{}[dr]|(.62){\sim^r} & X \\
              V \ar@{-->}[ur]^{v'} \ar@/_1pc/[ur]_v &
            } \]
If we have a diagram on the left, then we can find a map $v'$ such that the diagram on the right commutes.
Moreover, if we restrict the homotopy between $v$ and $v'$ on $U$, then we get the original homotopy between $v \circ i$  and $u$.

Let $\sim^{r*}$ be the reflexive transitive closure of $\sim^r$.
Then the previous lemma also holds for $\sim^{r*}$ in place of $\sim^r$:

\begin{lem}[hom-ext-rtc]
Let $\C$ be a category that satisfies conditions of the previous lemma.
Let $i : U \to V \in \cof$, $u : U \to X$, and $v : V \to X$ be maps, and let $h_1, \ldots, h_n : U \to P(X)$ be a sequence of homotopies
such that $p_1 \circ h_j = p_0 \circ h_{j+1}$ for every $1 \leq j < n$, $p_0 \circ h_1 = v \circ i$ and $p_1 \circ h_n = u$.
Then there exists a map $v' : V \to X$ and a sequence of homotopies $h'_1, \ldots, h'_n : V \to P(X)$
such that $p_1 \circ h'_j = p_0 \circ h'_{j+1}$ for every $1 \leq j < n$, $p_0 \circ h'_1 = v$ and $p_1 \circ h'_n = v'$ and $h_j = h'_j \circ i$ for every $1 \leq j \leq n$.
\end{lem}
\proof
Apply the previous lemma $n$ times.
\endproof

Let $\C$ be a category and let $\I$ be a class of morphisms of $\C$.
Then we define $\Iinj$ to be the class of morphisms of $\C$ that has RLP with respect to $\I$,
$\Icof$ to be the class of morphisms of $\C$ that has LLP with respect to $\Iinj$, and
$\Icell$ to be the class of transfinite compositions of pushouts of elements of $\I$.
Elements of $\Icell$ are called \emph{relative $\I$-cell complexes}.
Every relative $\I$-cell complex belongs to $\Icof$.

We say that a set $\I$ of maps of a cocomplete category $\C$ \emph{permits the small object argument}
if the domains of maps in $\I$ are small relative to $\Icell$.
The following proposition is standard and appears in this form in \cite{hovey}.

\begin{prop}[model-cat]
Suppose that $\C$ is a complete and cocomplete category, $\we$ is a class of morphisms of $\C$, and $\I$, $\J$ are sets of morphisms of $\C$.
Then $\C$ is a cofibrantly generated model category with $\I$ as the set of generating cofibrations,
$\J$ as the set of generating trivial cofibrations, and $\we$ as the class of weak equivalences if and only if the following conditions are satisfied:
\begin{description}
\item[(A1)] $\I$ and $\J$ permit the small object argument.
\item[(A2)] $\we$ has the 2-out-of-3 property and is closed under retracts.
\item[(A3)] $\Iinj \subseteq \we$.
\item[(A4)] $\Jcell \subseteq \we \cap \Icof$.
\item[(A5)] Either $\Jinj \cap \we \subseteq \Iinj$ or $\Icof \cap \we \subseteq \Jcof$.
\end{description}
\end{prop}

\section{Properties of model categories}

In this section, we will prove various properties of model categories that are related to fibrant objects.
In particular, we will give a characterization of weak equivalences between fibrant objects.

\subsection{Weak equivalences between fibrant objects}

The following propositions give useful characterizations of trivial
cofibrations and weak equivalences between fibrant objects.
The characterization for weak equivalences is similar to the one described in \cite{vogt-we}.

\begin{prop}[min-triv-cof]
Let $\C$ be a model category.
A cofibration between fibrant objects is a weak equivalence if and only if it is an inclusion of a strong deformation retract.
\end{prop}

This characterization of trivial cofibrations is probably well-known.
Similar propositions are proved in \cite{hirschhorn}, but we could not find a reference for this property, so we include a proof for the sake of convenience.

\proof
Every inclusion of a deformation retract becomes an isomorphism in the homotopy category and every such map is a weak equivalence.
To prove the converse, let us show that every weak equivalence with a fibrant domain is pure with respect to cofibrations.
This follows from the fact that every weak equivalence can be factored into a trivial cofibrant and a trivial fibration.
Every trivial fibration is pure with respect to cofibrations, and every trivial cofibration $f : X \to Y$ with a fibrant domain is a split monomorphism since we have a lift in the diagram below,
hence it is pure with respect to any map.
\[ \xymatrix{ X \ar[r]^{id_X} \ar[d]_f & X \\
              Y \ar@{-->}[ur]
            } \]

Now, let $f : X \to Y$ be a trivial cofibration between fibrant objects.
Since $X$ is fibrant, $f$ has a retraction $g : Y \to X$.
By the 2-out-of-3 property, $g$ is a weak equivalence.
Since $Y$ is fibrant, $g$ is pure with respect to cofibrations.
Consider the following diagram:
\[ \xymatrix{ Y \amalg_X Y \ar[rr]^-{[f \circ g, id_Y]} \ar[d]_{[\cyli_0,\cyli_1]} & & Y \ar[d]^g \\
              C_X(Y) \ar[rr]_-{g \circ s} & & X
            } \]
Since $g$ is pure with respect to cofibrations, we have a lift, which gives us a homotopy between $f \circ g$ and $id_Y$.
\endproof

Now, we need to prove a technical lemma, which we formulate in a general form since we will need it later.

\begin{lem}[we-bot]
Let $\C$ be a finitely cocomplete category, and let $i : U \to V$ and $g : Y \to Z$ be maps of $\C$.
Let $C_U(V)$ be a relative cylinder object for $i$.
Suppose that for every $A \in \{ Y, Z \}$, there exists a path object $p_0,p_1 : P(A) \to A$, which satisfy the following conditions:
\begin{enumerate}
\item For every $A \in \{ Y, Z \}$, $p_0 : P(A) \to A$ has RLP with respect to $i$ and it is pure with respect to $[\cyli_0,\cyli_1] : V \amalg_U V \to C_U(V)$.
\item There exists a morphism of path objects $(g,P(g)) : P(Y) \to P(Z)$.
\item Either $p_1 : P(Z) \to Z$ has RLP with respect to $i$ or there exists a map $s : P(Z) \to P(Z)$ such that $p_0 \circ s = p_1$ and $p_1 \circ s = p_0$.
\end{enumerate}

Let $f : X \to Y$ be a map of $\C$ such that $f$ has RLP up to $\sim^{r*}$ with respect to $U$, and $g \circ f$ has RLP with respect to $i$ up to $\sim_i$.
Then $g$ also has RLP with respect to $i$ up to $\sim_i$.
\end{lem}
\proof
Suppose that we have a commutative square as below.
Then there exists a map $u_x : U \to X$ such that $f \circ u_x \sim^{r*} u$ which means that there is a sequence of homotopies $h^1, \ldots, h^n : U \to P(Y)$ for some $n$
such that $p_0 \circ h^1 = f \circ u_x$, $p_1 \circ h^n = u$, and $p_1 \circ h^j = p_0 \circ h^{j+1}$ for every $1 \leq j < n$.
\[ \xymatrix{   \ar@{}[dr]|(.7){\sim^{r*}}            & X \ar[d]^f \\
              U \ar@{-->}[ur]^{u_x} \ar[d]_i \ar[r]_u & Y \ar[d]^g \\
              V \ar[r]_v                              & Z
            } \]

We can define a sequence of homotopies $h^1_u, \ldots, h^n_u$ between $g \circ f \circ u_x$ and $v \circ i$.
If $p_1$ has RLP with respect to $i$, then let $h^j_u = P(g) \circ h^j$; otherwise let $h^j_u = s \circ s \circ P(g) \circ h^j$.
By \rlem{hom-ext-rtc}, there exists a map $v_z : V \to Z$ and a sequence of homotopies $h^1_3, \ldots, h^n_3 : V \to P(Z)$ between $v_z$ and $v$ such that $h^j_3 \circ i = h^j_u$.
Indeed, if $p_1$ has RLP with respect to $i$, then we can apply \rlem{hom-ext-rtc} to $p_1,p_0 : P(Z) \to Z$.
If we defined $h^j_u$ as $s \circ s \circ P(g) \circ h^j$, then we can apply \rlem{hom-ext-rtc} to $s \circ P(g) \circ h^j$ to get a sequence of homotopies $h^1_4, \ldots, h^n_4$ between $v$ and $v_z$.
Then we can define $h^j_3$ as $s \circ h^j_4$.

By assumption on $g \circ f$, there exists a map $v_x : V \to X$ such that $g \circ f \circ v_x$ and $v_z$ are relatively homotopic.
Note that $h^1, \ldots, h^n$ is a sequence of homotopies between $f \circ u_x = f \circ v_x \circ i$ and $u$.
Thus, by \rlem{hom-ext-rtc}, we have a map $v_y : V \to Y$ and a sequence of homotopies $h^1_y, \ldots, h^n_y : V \to P(Y)$ between $f \circ v_x$ and $v_y$ such that $h^j_y \circ i = h^j$.
In particular, $v_y \circ i = u$.
Thus we only need to prove that $g \circ v_y$ and $v$ are homotopic relative to $i$.
If $h^j_u = P(g) \circ h^j$, then let $h^j_1 = P(g) \circ h^j_y$.
If $h^j_u = s \circ s \circ P(g) \circ h^j$, then let $h^j_1 = s \circ s \circ P(g) \circ h^j_y$.
Then $h^j_1 \circ i = h^j_u$.
Thus we have a sequence of maps $[h^j_1,h^j_3] : V \amalg_U V \to P(Z)$.

Now, let us show that $p_1 \circ h^j_1 \sim_i p_1 \circ h^j_3$ whenever $p_0 \circ h^j_1 \sim_i p_0 \circ h^j_3$.
Indeed, if $h_0$ is a relative homotopy between $p_0 \circ h^j_1$ and $p_0 \circ h^j_3$, then consider the following diagram:
\[ \xymatrix{ V \amalg_U V \ar[rr]^{[h^j_1, h^j_3]} \ar[d]_{[\cyli_0,\cyli_1]} & & P(Z) \ar[d]^{p_0} \\
              C_U(V) \ar[rr]_{h_0} & & Z
            } \]
Since $p_0$ is pure with respect to $[\cyli_0,\cyli_1]$, we have a lift $q : C_U(V) \to P(Z)$.
Then $p_1 \circ q$ is a relative homotopy between $p_1 \circ h^j_1$ and $p_1 \circ h^j_3$.

Finally, to complete the proof we need to show that $g \circ v_y$ and $v$ are relatively homotopic.
Since $g \circ v_y = p_1 \circ h^n_1$ and $v = p_1 \circ h^n_3$, the previous observation implies that it is enough to prove that $p_0 \circ h^1_1$ and $p_0 \circ h^1_3$ are relatively homotopic.
But $p_0 \circ h^1_1 = g \circ f \circ v_x$ and $p_0 \circ h^1_3 = v_z$ and these maps are relatively homotopic by definition of $v_x$.
\endproof

If $\I$ is a class of maps of a category $\C$, then let us denote by $\J_\I$ the class of maps $\cyli_0 : V \to C_U(V)$ for each $U \to V \in \I$.
This notation is slightly abusive since the class $\J_\I$ depends on the choice of cylinder objects $C_U(V)$.
If the cylinder objects are not specified explicitly in some statement, then it holds for any choice of these objects.

Now, we will show that if $\I$ generates the class of cofibrations of a model category in which all objects are fibrant,
then a map is a weak equivalence if and only if it has RLP up to relative homotopy with respect to $\I$.
Thus the following proposition implies that if a map of such a model category is a weak equivalence and belongs to $\Jinj[_\I]$, then it is a trivial fibration.
This implies that $\J_\I$ can be used as a set of generating trivial cofibrations if it permits the small object argument (see \rprop{min-cof-gen}).

\begin{prop}[triv-fib-iinj]
Let $\C$ be a category, and let $\I$ be a class of maps of $\C$.
If $f : X \to Y \in \Jinj[_\I]$ has RLP up to relative homotopy with respect to $\I$, then $f \in \Iinj$.
\end{prop}
\proof
Suppose we have a commutative square as below.
We need to find a lift $V \to X$ such that both triangles commute.
\[ \xymatrix{ U \ar[d]_{i \in \I} \ar[r]^u \ar@{}[dr]|(.7){\sim_i} & X \ar[d]^f \\
              V \ar[r]_v \ar@{-->}[ur]^g                           & Y
            } \]
By assumption, we have a lift $g : V \to X$ together with a relative homotopy $h : C_U(V) \to Y$ between $f \circ g$ and $v$.
Since $f$ has RLP with respect to $\cyli_0$, we have a lift $h' : C_U(V) \to X$ in the following diagram:
\[ \xymatrix{ V \ar[r]^g \ar[d]_{\cyli_0}         & X \ar[d]^f \\
              C_U(V) \ar[r]^-h \ar@{-->}[ur]^{h'} & Y.
            } \]
Then $h' \circ \cyli_1$ is a required lift in the original square.
Indeed, the top triangle commutes because $h' \circ \cyli_1 \circ i = h' \circ \cyli_0 \circ i = g \circ i = u$ and the bottom triangle commutes because $f \circ h' \circ \cyli_1 = h \circ \cyli_1 = v$.
\endproof

Now we assume that $\C$ is a model category and $C_U(V)$ is a correct relative cylinder object, that is it is a factorization of the map $[id_V,id_V] : V \amalg_U V \to V$
into a cofibration $[\cyli_0,\cyli_1] : V \amalg_U V \to C_U(V)$ followed by a weak equivalence $C_U(V) \to V$.
The following propositions gives a characterization of weak equivalences between fibrant objects.

\begin{prop}[min-we]
Let $\C$ be a model category, and let $\I$ be a class of maps of $\C$ which generates the class of cofibrations (that is, the class of cofibrations equals to $\Icof$).
Let $f : X \to Y$ be a morphism of $\C$ such that $X$ and $Y$ are fibrant.
Then the following conditions are equivalent:
\begin{enumerate}
\item $f$ is a weak equivalence.
\item $f$ has RLP up to relative homotopy with respect to cofibrations.
\item $f$ has RLP up to relative homotopy with respect to $\I$.
\end{enumerate}
\end{prop}
\proof
$(1 \Rightarrow 2)$
Consider the following commutative square in which $c$ is a cofibration:
\[ \xymatrix{ U \ar[r] \ar[d]_c & X \ar[d]^f \\
              V \ar[r]          & Y
            } \]

Factor $f$ into a trivial cofibration $i : X \to Z$ followed by a trivial fibration $p : Z \to Y$.
Since $c$ is a cofibration and $p$ is a trivial fibration, we have a lift $q : V \to Z$.
Since $i$ is a trivial cofibration between fibrant objects, by \rprop{min-triv-cof}, it has a retraction $r : Z \to X$ such that $i \circ r \sim_i id_Z$.
\[ \xymatrix{ U \ar[r] \ar[dd]_c        & X \ar[d]_i \ar@{=}[r] \ar@{}[dr]|(.7){\sim_i} & X \ar[d]^i \\
                                        & Z \ar[d]_p \ar@{=}[r] \ar@{-->}[ur]^r         & Z          \\
              V \ar[r] \ar@{-->}[ur]^q  & Y                                             &
            } \]
Then $r \circ q$ is a required lift.

$(2 \Rightarrow 3)$ Obvious.

$(3 \Rightarrow 1)$
Factor $f$ into a trivial cofibration $f' : X \to Z$ followed by a fibration $g : Z \to Y$.
Let us prove that $f'$ has RLP up to $\sim^r$ with respect to every object.
By \rprop{min-triv-cof}, there exists a map $g' : Z \to X$ and a homotopy $f' \circ g' \sim_{f'} id_Z$.
By \rprop{path-cyl}, we have a right homotopy $h : Z \to P(Z)$ between $f' \circ g'$ and $id_Z$.
For every morphism $u : U \to Z$ we can define a lift $u' : U \to Y$ as $g' \circ u$.
Then $h \circ u$ is a homotopy between $f' \circ u'$ and $u$.

Since $Z$ and $Y$ are fibrant, conditions of \rlem{we-bot} are satisfied.
Hence, $g$ has RLP up to relative homotopy with respect to $\I$.
Since $\J_\I$ consists of trivial cofibrations, by \rprop{triv-fib-iinj}, $g$ is a trivial fibration.
Thus $f$ is a weak equivalence by the 2-out-of-3 property.
\endproof

Often the class of cofibrations is generated by a much smaller class $\I$.
Thus the last condition in the previous proposition gives us a useful characterization of weak equivalences between fibrant objects which is easy to verify in practice.
In particular, if every object in a model category is fibrant, then this proposition gives us a complete characterization of weak equivalences, which we will use in the next section.
The equivalence $1 \Leftrightarrow 3$ is well-known for topological spaces.
A variation of this proposition is proved in \cite{vogt-we},
where a map between fibrant objects is proved to be a weak equivalence if and only if it satisfies a slightly stronger version of the second condition of \rprop{min-we}.

Let $\C$ be a model category in which every object is fibrant.
If $\C$ has a set of generating cofibrations $\I$ which satisfies a mild additional hypothesis, then using the characterization of weak equivalence that we gave in \rprop{min-we},
we can construct a set of generating trivial cofibrations for this model structure.
This shows that the model category is cofibrantly generated in this case,
and the set of generating trivial cofibrations has a simple explicit description in terms of generating cofibrations.

\begin{prop}[min-cof-gen]
Let $\C$ be a model category in which every object is fibrant.
Suppose that the class of cofibrations is generated by a set $\I$ such that
the domains and the codomains of maps in $\I$ are small relative to $\Icell$.
Then the model structure is cofibrantly generated.
\end{prop}
\proof
By the small object argument, there exists a reflexive relative cylinder object $C_U(V)$ such that $[\cyli_0,\cyli_1] : V \amalg_U V \to C_U(V)$ is a cofibration and the map $C_U(V) \to V$ is a trivial fibration.
We prove that $\J_\I$ is a set of generating trivial cofibrations.
Let us check the conditions of \rprop{model-cat}:
\begin{description}
\item[(A1)] The set $\I$ permits the small object argument by assumption.
By \cite[Proposition~2.1.16]{hovey}, the codomains of maps in $\I$ are small relative to $\Icof$.
Since maps in $\J_\I$ are cofibrations (see $\mathbf{(A4)}$ below) and the domains of maps in $\J_\I$ are the codomains of maps in $\I$,
the set $\J_\I$ also permits the small object argument.
\item[(A2)] The class of weak equivalences is closed under retracts and satisfies 2-out-of-3 since $\C$ is a model category.
\item[(A3)] Since $\I$ generates cofibrations, the class $\Iinj$ consists of trivial fibrations.
\item[(A4)] A map $\cyli_0 : V \to C_U(V)$ in $\J_\I$ is the composition of maps $V \to V \amalg_U V \to C_U(V)$,
where the first map is a pushout of a cofibration $U \to V$ and the second map is a cofibration by assumption.
Moreover, the map $C_U(V) \to V$ is a weak equivalence; hence, $\cyli_0$ is also a weak equivalence by 2-out-of-3.
\item[(A5)] By \rprop{triv-fib-iinj} and \rprop{min-we}, if $f \in \Jinj[_\I] \cap \we$, then $f \in \Iinj$.
\end{description}
\endproof

\subsection{Trivial cofibrations and fibrations with fibrant codomains}

If some of the objects of a model category are not fibrant, then sometimes we can characterize fibrant objects as those which have
the RLP with respect to some set of trivial cofibrations $S$ which is considerably smaller than a set of generating cofibrations.
For example, the Joyal model structure on simplicial sets has weak Kan complexes as fibrant objects,
which are simplicial sets that have RLP with respect to inner horns.
But inner horns do not generate the whole class of trivial cofibrations.
Another example is the category of marked simplicial sets constructed in \cite[Proposition~3.1.3.7]{lurie-topos}.
Finally, \cite[Lemma~3.3.11]{hirschhorn} implies that a left Bousfield localization of a model category
also has an explicit description of a set of maps characterizing fibrant objects provided the original model category has such a description.

Let $S$ be a set of trivial cofibrations such that an object is fibrant if and only if it has the RLP with respect to $S$.
We should not expect that there is always a simple explicit description of a set of generating trivial cofibrations in terms of $S$.
But we can show that the class $\Icof[(\J_\I \cup S)]$ contains all trivial cofibrations with fibrant codomains.
Moreover, a map with a fibrant codomain is a fibration if and only if it has the RLP with respect to this set.
For simplicial sets with the Joyal model structure, (a slightly stronger version of) this result was proved in \cite[Corollary~2.4.6.5]{lurie-topos} (where it is attributed to Joyal).

\begin{prop}
Let $\C$ be a model category.
Let $\I$ be a set of generating cofibrations, and let $S$ be a set of trivial cofibrations
such that every object that has the RLP with respect to $S$ is fibrant.
Suppose that the domains and the codomains of maps in $\I$ and the domains of maps in $S$ are small relative to $\Icell$.

Then a map with a fibrant codomain is a trivial cofibration if and only if it belongs to $\Icof[(\J_\I \cup S)]$.
A map with a fibrant codomain is a fibration if and only if it belongs to $\Iinj[(\J_\I \cup S)]$.
\end{prop}
\proof
Maps in $S$ are trivial cofibrations by assumption and maps in $\J_\I$ are trivial cofibrations by $\mathbf{(A4)}$ in the proof of \rprop{min-cof-gen}.
Thus maps in $\Icof[(\J_\I \cup S)]$ are trivial cofibrations and fibrations belong to $\Iinj[(\J_\I \cup S)]$.

Let $f : X \to Z$ be a trivial cofibration such that $Z$ is fibrant.
Factor $f$ into a map $g : X \to Y \in \Icell[(\J_\I \cup S)]$ followed by a map $h : Y \to Z \in \Iinj[(\J_\I \cup S)]$.
\[ \xymatrix{ X \ar[r]^g \ar[d]_f & Y \ar[d]^h \\
              Z \ar@{=}[r] \ar@{-->}[ur] & Z
            } \]
By the 2-out-of-3 property, $h$ is a weak equivalence.
Since it is a weak equivalence between fibrant objects, by \rprop{min-we}, it has RLP up to relative homotopy with respect to $\I$.
By \rprop{triv-fib-iinj}, $f \in \Iinj$.
Hence, we have a lift $Z \to Y$ as shown in the diagram above.
Thus $f$ is a retract of $g$ and it belongs to $\Icof[(\J_\I \cup S)]$.

Let $f : X \to Y$ be a map in $\Iinj[(\J_\I \cup S)]$ such that $Y$ is fibrant.
To prove that it is a fibration, we need to show that it has RLP with respect to every trivial cofibration.
Let $i : U \to V$ be a trivial cofibration, and let $u : U \to X$ and $v : V \to Y$ be maps such that the obvious square commutes.
Factor $v$ into a map $V \to Z \in \Icell[S]$ followed by a map $Z \to Y \in \Iinj[S]$.
\[ \xymatrix{ U \ar[rr]^u \ar[d]_i & & X \ar[d]^f \\
              V \ar[r] & Z \ar[r] \ar@{-->}[ur] & Y
            } \]
Since $Z$ is fibrant and the map $U \to Z$ is a trivial cofibration, it belongs to $\Icof[(\J_\I \cup S)]$.
Since $f$ belongs to $\Iinj[(\J_\I \cup S)]$, we have a lift $Z \to X$ in the diagram above.
Thus $f$ has RLP with respect to $i$.
\endproof

\subsection{Quillen equivalences}

Let $U : \D \to \C$ be a right Quillen functor with left adjoint $F : \C \to \D$.
Then $F \dashv U$ is a Quillen equivalence if and only if $U$ reflects weak equivalences between fibrant objects and
for every cofibrant $X$, map $X \xrightarrow{\eta_X} UF(X) \xrightarrow{U(r_{F(X)})} URF(X)$ is a weak equivalence,
where $\eta$ is the unit of the adjunction and $r_{F(X)}$ is a fibrant replacement for $F(X)$.
In this subsection, we give an equivalent condition for the first part of this characterization, which is easier to verify sometimes.
We prove that $U$ reflects weak equivalences between fibrant objects if and only if $F(\cof_\C)$ generates $\cof_\D$ in a certain sense,
where $\cof_\C$ and $\cof_\D$ are classes of cofibrations of $\C$ and $\D$ respectively.

To describe this characterization, we need to introduce a bit of notation.
This notation is similar to the usual notation related to lifting properties, but we work with lifting properties up to a homotopy relation.
Let $\I$ be a class of cofibrations of some model category $\C$.
Then we define $\Iinj_h$ as the set of maps of $\C$ between fibrant objects which have RLP up to relative homotopy with respect to $\I$.
We define $\Icof_h$ as the set of cofibrations of $\C$ which have LLP up to relative homotopy with respect to $\Iinj_h$.
We will call elements of $\Icof_h$ \emph{weak $\I$-cofibrations}.

As usual, we have the following properties: $\I \subseteq \Icof_h$, $\Iinj[\Icof_h]_h = \Iinj_h$, $\Icof[\Icof_h]_h = \Icof_h$, $\Iinj \subseteq \Iinj_h$,
and if $\I \subseteq \J$, then $\Jinj_h \subseteq \Iinj_h$ and $\Icof_h \subseteq \Jcof_h$.
\Rprop{min-we} implies that $\Iinj[\cof]_h = \we$ and if $\Icof = \cof$, then $\Iinj_h = \we$ and $\Icof_h = \cof$.
The following lemma is analogous to \cite[Lemma~2.1.8]{hovey}:

\begin{lem}
Let $U : \D \to \C$ be a right Quillen functor with left adjoint $F : \C \to \D$.
Then the following are true:
\begin{enumerate}
\item \label{it:adj} If $f$ is a cofibration in $\C$ and $g$ is a map between fibrant objects in $\D$,
then $F(f)$ has LLP up to relative homotopy with respect to $g$ if and only if $f$ has this property with respect to $U(g)$.
\item \label{it:adj-prop} $U(\Iinj[F\I]_h) \subseteq \Iinj_h$ for every class of cofibrations $\I$.
\item $F(\Icof_h) \subseteq \Icof[F\I]_h$ for every class of cofibrations $\I$.
\end{enumerate}
\end{lem}
\proof
The proof is identical to the proof of the analogous lemma for ordinary lifting properties.
\begin{enumerate}
\item Since $F$ is a left Quillen functor, it preserves relative cylinder objects, and hence also relations $\sim_f$ for every cofibration $f$.
Now, the statement holds by adjunction as usual.
\item Let $g \in \Iinj[F\I]_h$ and $f \in \I$.
Then $g$ has RLP up to $\sim_{F(f)}$ with respect to $F(f)$, and by \eqref{it:adj}, $U(g)$ has RLP up to $\sim_f$ with respect to $f$.
Thus $U(g) \in \Iinj_h$, as required.
\item Let $f \in \Icof_h$ and $g \in \Iinj[F\I]_h$.
Then by \eqref{it:adj-prop}, $U(g)$ has RLP up to $\sim_f$ with respect to $f$.
By \eqref{it:adj}, $g$ has RLP up to $\sim_{F(f)}$ with respect to $F(f)$.
Thus $F(f) \in \Icof[F\I]_h$.
\end{enumerate}
\endproof

Now, we can prove the characterization of Quillen equivalences.
Sometimes this condition is easier to verify.
For example, if $\Icof[F(\cof_\C)]$ contains cofibrations of $\D$, then this condition holds immediately.

\begin{prop}[refl-fibr-char]
Let $U : \D \to \C$ be a right Quillen functor with a left adjoint $F : \C \to \D$.
Then $U$ reflects weak equivalences between fibrant objects if and only if $\cof_\D \subseteq \Icof[F(\cof_\C)]_h$.
\end{prop}
\proof
Let $f : X \to Y$ be a map of $\D$ such that $X$ and $Y$ are fibrant.
Then $U(f)$ is a weak equivalence if and only if $U(f) \in \Iinj[\cof_\C]_h$ by \rprop{min-we}.
But $U(f) \in \Iinj[\cof_\C]_h$ if and only if $f \in \Iinj[F(\cof_\C)]_h$ by the previous lemma.
Thus we only need to prove that $\Iinj[F(\cof_\C)]_h \subseteq \we_\D$ if and only if $\cof_\D \subseteq \Icof[F(\cof_\C)]_h$.
But this follows from the facts that $\we_\D = \Iinj[\cof_\D]_h$ and $\Icof[\cof_\D]_h = \cof_\D$ which hold by \rprop{min-we}.
\endproof

\begin{cor}[Quillen-equiv]
A Quillen adjunction $F \dashv U$ is a Quillen equivalence if and only if the map $X \to URF(X)$ is a weak equivalence for every cofibrant $X$ and $\cof_\D \subseteq \Icof[F(\cof_\C)]_h$.
\end{cor}

\subsection{Properties of weak $\I$-cofibrations}

Just as ordinary cofibrations, the class of weak $\I$-cofibrations is closed under retracts as we will see in the proposition below.
But it is also closed under weaker notion of retracts that we now introduce.
Let $f : X \to Y$ and $g : Z \to W$ be maps of some model category.
Then we say that $f$ is a weak retract of $g$ if there exists a diagram of the following form:
\[ \xymatrix{ X \ar[r] \ar[d]_f & Z \ar[r] \ar[d]_g & X \ar[d] \\
              Y \ar[r]          & W \ar[r]          & Y'
            } \]
where the composition of the top row is the identity morphism and the composition of the bottom row is a trivial cofibration.

In order to use \rcor{Quillen-equiv}, we need to be able to construct weak $\I$-cofibrations.
The following proposition gives us several useful closure properties.

\begin{prop}
For every model category $\C$ and every class of cofibrations $\I$ in it,
the class of weak $\I$-cofibrations is closed under pushouts, transfinite compositions, and weak retracts.
\end{prop}
\proof
Let $i : U \to V$ be a weak $\I$-cofibration.
Let $i' : U' \to V'$ be a pushout of $i$, and let $f : X \to Y$ be a map in $\Iinj_h$.
Then consider the following diagram:
\[ \xymatrix{ U \ar[r] \ar[d]^i &     U' \ar[r] \ar[d]^{i'} & X \ar[d]^f \\
              V \ar[r]          & \po V' \ar[r]             & Y
            } \]
Since $i \in \Icof_h$, there exist a map $v : V \to X$ and a homotopy $h : V \to P(Y)$ between $f \circ v$ and $V \to V' \to Y$ constant on $i$.
Then by the universal property of pushouts, there exist maps $v' : V' \to X$ and $h' : V' \to P(Y)$ such that $V \to V' \xrightarrow{v'} X$ is $v$,
$U' \xrightarrow{i'} V' \xrightarrow{v'} X$ is $U' \to X$, $V \to V' \xrightarrow{h'} P(Y)$ is $h$, and $h' \circ i'$ is the constant homotopy:
\[ \xymatrix{ U \ar[r] \ar[d]_i &     U' \ar[r] \ar[d]_{i'} \ar@{}[dr]|(.7){\sim^r_{i'}} & X \ar[d]^f \\
              V \ar[r]          & \po V' \ar[r] \ar@{-->}[ur]^{v'}                       & Y
            } \]
Thus $i'$ has LLP up to $\sim_{i'}$ with respect to $f$.

Now, let us show that weak $\I$-cofibrations are closed under transfinite compositions.
Let $X : \lambda \to \C$ be a $\lambda$-sequence such that the maps $x_\alpha : X_\alpha \to X_{\alpha+1}$ are weak $\I$-cofibrations.
Consider the following commutative square:
\[ \xymatrix{ X_0 \ar[r]^u \ar[d] & Y \ar[d]^f \\
              X_\lambda \ar[r]_v & Z
            } \]
where $Y$ and $Z$ are fibrant and $f \in \Iinj_h$.
We construct partial functions $g$ and $h$ from $\lambda+1$ to the class of morphisms of $\C$
such that $g_\alpha : X_\alpha \to Y$ and $h_\alpha : X_\alpha \to P(Z)$ by transfinite recursion on $\alpha$.
Let $g_0 = u$, and let $h_0$ be the constant homotopy on $f \circ u$.

The maps $g_{\alpha+1}$ and $h_{\alpha+1}$ are defined whenever $g_\alpha$ and $h_\alpha$ are defined and
$h_\alpha$ is a homotopy between $f \circ g_\alpha$ and $v \circ x_{\alpha,\lambda}$ (where $x_{\alpha,\lambda}$ is the map $X_\alpha \to X_\lambda$).
By the homotopy extension property for right homotopies in model categories, there exist a map $v' : X_{\alpha+1} \to Z$ such that $v' \circ x_\alpha = f \circ g_\alpha$
and a homotopy $h' : X_{\alpha+1} \to P(Z)$ between $v'$ and $v \circ x_{\alpha+1,\lambda}$ such that $h' \circ x_\alpha = h_\alpha$:
\[ \xymatrix{ X_\alpha \ar[rr]^{g_\alpha} \ar[d]_{x_\alpha}                                                & & Y \ar[d]^f \\
              X_{\alpha+1} \ar@/^/[rr]^-{v'} \ar@/_/[rr]_{v\,\circ\,x_{\alpha+1,\lambda}} \ar@{}[rr]|-{\sim} & & Z
            } \]
Since $x_\alpha$ is a weak $\I$-cofibration, there exist a map $g_{\alpha+1} : X_{\alpha+1} \to Y$ such that $g_{\alpha+1} \circ x_\alpha = g_\alpha$
and a homotopy $h'' : C_{X_\alpha}(X_{\alpha+1}) \to Z$ between $f \circ g_{\alpha+1}$ and $v'$.
Consider the following diagram:
\[ \xymatrix{ X_{\alpha+1} \ar[rrr]^{h'} \ar[d]_{\cyli_1}                                                                                 & & & P(Z) \ar[d] \\
              C_{X_\alpha}(X_{\alpha+1}) \ar[rrr]_-{\langle h'', v\,\circ\,x_{\alpha+1,\lambda}\,\circ\,s \rangle} \ar@{-->}[rrru]^{h'''} & & & Z \times Z
            } \]
Since $\cyli_1$ is a trivial cofibration and $P(Z) \to Z \times Z$ is a fibration, there exists a lift $h'''$ in this diagram.
We define $h_{\alpha+1}$ as $h''' \circ \cyli_0$.
Note that $h_{\alpha+1}$ is a homotopy between $f \circ g_{\alpha+1}$ and $v \circ x_{\alpha+1,\lambda}$.
Moreover, $h_{\alpha+1} \circ x_\alpha = h''' \circ \cyli_0 \circ x_\alpha = h''' \circ \cyli_1 \circ x_\alpha = h' \circ x_\alpha = h_\alpha$.

If $\alpha$ is a limit ordinal, the maps $g_\alpha$ and $h_\alpha$ are defined if and only if $g_{\beta+1} \circ x_\beta = g_\beta$ and $h_{\beta+1} \circ x_\beta = h_\beta$ for every $\beta < \alpha$.
If this holds, then the maps $\{ g_\beta \}_{\beta < \alpha}$ and $\{ h_\beta \}_{\beta < \alpha}$ determine cones for $\alpha \to \lambda \xrightarrow{X} \C$,
so we can define $g_\alpha$ and $h_\alpha$ by the universal property of colimits.
It is easy to see by induction on $\alpha$ that maps $g_\alpha$ and $h_\alpha$ are always defined and commute with the maps $x_\alpha$.
In particular, we have a map $g_\lambda : X_\lambda \to Y$ such that $g_\lambda \circ x_{0,\lambda}$ equals to $u$.
Moreover, the map $h_\lambda : X_\lambda \to P(Z)$ is a relative homotopy between $f \circ g_\lambda$ and $v$.

Finally, let us prove that $\I$-cofibrations are closed under weak retracts.
Suppose that $i : U \to V$ is a weak retract of a weak $\I$-cofibration $f : X \to Y$.
Then there exist maps $V \to Y$ and $Y \to V'$ such that $V \to Y \to V'$ is a trivial cofibration.
Let $g : Z \to W \in \Iinj_h$, $u : U \to Z$, and $v : V \to W$ be maps such that $v \circ i = g \circ u$.
Since $W$ is fibrant, $v$ factors through $V \to Y \to V'$.
Thus we have the following diagram:
\[ \xymatrix{ U \ar[r] \ar[d]_i & X \ar[r] \ar[d]_f & U \ar[r]^u \ar[d] & Z \ar[d]^g \\
              V \ar[r]          & Y \ar[r]          & V' \ar[r]         & W
            } \]
Since $f$ is a weak $\I$-cofibration, there exists a lift $Y \to Z$ such that $Y \to Z \xrightarrow{g} W$ is relatively homotopic to $Y \to V' \to W$.
Then $V \to Y \to Z$ is the required lift in the original square.
\endproof

We can also prove the following standard proposition:

\begin{prop}
Let $\I$ be a set of maps in a model category admitting the small object argument.
Then every weak $\I$-cofibration is a weak retract of a relative $\I$-cell complex.
\end{prop}
\proof
Let $f : X \to Y$ be a weak $\I$-cofibration.
Let $r_Y : Y \to R(Y)$ be a fibrant replacement for $Y$.
Factor $r_Y \circ f$ into a relative $\I$-cell complex $X \to Z$ followed by a map $Z \to R(Y) \in \Iinj$.
Since $Z \to R(Y)$ is a weak equivalence between fibrant objects, we have a lift in the following diagram:
\[ \xymatrix{ X \ar[r] \ar[d]_f             & Z \ar[d] \\
              Y \ar[r]_{r_Y} \ar@{-->}[ur]  & R(Y)
            } \]
Thus $f$ is a weak retract of $X \to Z$.
\endproof

\section{Existence of model structures}
\label{sec:exist}

In this section, we will give necessary and sufficient conditions for the existence of a model structure in which all objects are fibrant.
Throughout this section let $\C$ be a fixed complete and cocomplete category and $\I$ a set of maps of $\C$
such that the domains and the codomains of maps in $\I$ are small relative to $\Icell$.
Suppose that for every map $i : U \to V$ in $\I$, there exists a reflexive relative cylinder object $C_U(V)$ such that $[\cyli_0,\cyli_1] : V \amalg_U V \to C_U(V) \in \Icof$.
Note that such a cylinder object always exists by the small object argument, but sometimes we can choose another object which is more convenient to work with.

Let $\J_\I = \{\ \cyli_0 : V \to C_U(V)\ |\ i : U \to V \in \I \ \}$, and
let $\we_\I$ be the set of maps which have RLP up to relative homotopy with respect to $\I$.
We will consider the following conditions:
\begin{align}
& \text{For every composable $f \in \Jcell[_\I]$ and $g$, if $g \circ f \in \we_\I$, then $g \in \we_\I$} \label{cond:main} \tag{*} \\
& \text{For every composable $f \in \Jcell[_\I] \cup \we_\I$ and $g$, if $g \circ f \in \we_\I$, then $g \in \we_\I$} \label{cond:strong-main} \tag{*'}
\end{align}

First, let us prove a general lemma about split monomorphisms.
It seems that it should be well-known, but we could not find a proof of this result in the literature, so we include it here for the sake of convenience.

\begin{lem}[split-mono]
Split monomorphisms are closed under pushouts, retracts, and transfinite compositions.
\end{lem}
\proof
The only nontrivial statement is the last part.
To prove that split monomorphisms are closed under transfinite compositions, consider a $\lambda$-sequence $X : \lambda \to \C$ such that the maps $x_\alpha : X_\alpha \to X_{\alpha+1}$ are split monomorphisms.
We construct a partial function $r$ from $\lambda+1$ to the class of morphisms of $\C$ such that $r_\alpha : X_\alpha \to X_0$ by transfinite recursion on $\alpha$.
Let $r_0 = id_{X_0}$.
Let $r_{\alpha+1} = r_\alpha \circ x'_\alpha$ whenever $r_\alpha$ is defined, where $x'_\alpha : X_{\alpha+1} \to X_\alpha$ is a splitting of $x_\alpha$.
Finally, if $\alpha$ is a limit ordinal and $r_{\beta+1} \circ x_\beta = r_\beta$ for every $\beta < \alpha$,
then the maps $r_\beta$ determine a cone for $\alpha \to \lambda \xrightarrow{X} \C$.
Since $X_\alpha$ is a colimit of this diagram, we can define $r_\alpha$ by the universal property if these conditions hold; otherwise, $r_\alpha$ is undefined.
It is easy to see by induction on $\alpha$ that maps $r_\alpha$ are always defined and commute with the maps $x_\alpha$.
In particular, we have a retraction $r_\lambda : X_\lambda \to X_0$ of the map $X_0 \to X_\lambda$.
\endproof

Now, we can prove the main technical lemma of this section:

\begin{lem}[main]
If condition~\eqref{cond:main} holds, then the following are true:
\begin{enumerate}
\item \label{it:we-fact} Every weak equivalence factors into a map in $\Jcell[_\I]$ followed by a map in $\Iinj$.
\item \label{it:we-lift} Every weak equivalence has RLP up to relative homotopy with respect to $\Icof$.
\item \label{it:we-top} For every $f : X \to Y$ and $g : Y \to Z$, if $g \in \we_\I$ and $g \circ f \in \we_\I$, then $f \in \we_\I$.
\item For every $f : X \to Y$ and $g : Y \to Z$, if $f \in \we_\I$ and $g \in \we_\I$, then $g \circ f \in \we_\I$.
\item $\Jcell[_\I] \subseteq \we_\I$.
\end{enumerate}
\end{lem}
\proof
Let $f : X \to Z$ be a weak equivalence.
Factor $f$ into maps $f' : X \to Y \in \Jcell[_\I]$ and $g : Y \to Z \in \Jinj[_\I]$.
By assumption, $g \in \we_\I$.
By \rprop{triv-fib-iinj}, $g \in \Iinj$.
This proves \eqref{it:we-fact}.

This implies that weak equivalences are pure with respect to cofibrations.
Indeed, maps in $\Iinj$ are pure with respect to cofibrations and maps in $\Jcell[_\I]$ are pure with respect to all maps since they are split monomorphisms.
The last statement follows from \rlem{split-mono} since the maps $\cyli_0 : V \to C_U(V)$ are split monomorphisms by assumption.

Now, let us prove \eqref{it:we-lift}.
Since every weak equivalence factors into a map in $\Jcell[_\I]$ followed by a map in $\Iinj$,
to prove that every weak equivalence has RLP up to $\sim_i$ with respect to every cofibration $i : U \to V$,
it is enough to show that every map in $\Jcell[_\I]$ has this property.
Let $f : X \to Y$ be a map in $\Jcell[_\I]$.
It has a retraction $g : Y \to X$ which is a weak equivalence by condition~\eqref{cond:main}.
Hence, $g$ is pure with respect to cofibrations.
Let $u : U \to X$ and $v : V \to Y$ be maps such that the obvious square commutes.
Consider the following diagram:
\[ \xymatrix{ V \amalg_U V \ar[rr]^-{[f \circ g \circ v, v]} \ar[d]_{[\cyli_0,\cyli_1]} & & Y \ar[d]^g \\
              C_U(V) \ar[rr]_-{g \circ v \circ s} & & X
            } \]
Since $g$ is pure with respect to cofibrations, we have a relative homotopy between $f \circ g \circ v$ and $v$.
Thus $g \circ v$ is a required lift in the original square:
\[ \xymatrix{ U \ar[r]^u \ar[d]_i \ar@{}[dr]|(.7){\sim_i}   & X \ar[d]^f \\
              V \ar[r]_v \ar@{-->}[ur]^{g \circ v}          & Y
            } \]

Now, let us prove \eqref{it:we-top}.
Let $f : X \to Y$ and $g : Y \to Z$ be maps such that $g \in \we_\I$ and $g \circ f \in \we_\I$.
Consider the following diagram:
\[ \xymatrix{ U \ar[r]^u \ar[d]_i & X \ar[d]^f \\
              V \ar[r]^v \ar[rd]  & Y \ar[d]^g \\
                                  & Z
            } \]
Since $g \circ f \in \we_\I$, we have a lift $q : V \to X$ and a homotopy $h : C_U(V) \to Z$ between $g \circ f \circ q$ and $g \circ v$.
Consider the following diagram:
\[ \xymatrix{ V \amalg_U V \ar[r]^-{[f \circ q, v]} \ar[d]_{[\cyli_0,\cyli_1]} & Y \ar[d]^g \\
              C_U(V) \ar[r]_-h & Z
            } \]
Since $g$ is pure with respect to cofibrations, we have a lift $h' : C_U(V) \to Y$ which gives us a homotopy between $f \circ q$ and $v$.

Now, let us show that every $f : X \to Y \in \Jcell[_\I]$ is a weak equivalence.
Let $g : Y \to X$ be a retraction of $f$.
Since $g \circ f = id_X$, the retraction $g$ is a weak equivalence by \eqref{cond:main}.
By \eqref{it:we-top}, $f$ is a weak equivalence too.

Finally, let us prove that weak equivalences are closed under compositions.
To do this, it is enough to show that relation $\sim_i$ is transitive.
Let $h_0 : C_U(V) \to X$ be a homotopy between $f : V \to X$ and $f' : V \to X$, and
let $h_1 : C_U(V) \to X$ be a homotopy between $f' : V \to X$ and $f'' : V \to X$.
Then consider the following diagram:
\[ \xymatrix{ V \ar[r]^{\cyli_1} \ar[d]_{\cyli_0} & C_U(V) \ar[d]^{p_0} \\
              C_U(V) \ar[r]_{p_1} \ar[d]_s & \po Z \ar[d]^q \\
              V \ar[r]_{\cyli_1} & \po C_U(V)
            } \]
Since $p_0 \in \Jcell[_\I]$, $q$ is a weak equivalence.
Consider the following diagram:
\[ \xymatrix{ V \amalg_U V \ar[rr]^-{[p_0 \circ \cyli_0, p_1 \circ \cyli_1]} \ar[d]_{[\cyli_0,\cyli_1]} & & Z \ar[d]^q \\
              C_U(V) \ar@{=}[rr] & & C_U(V)
            } \]
Since $q$ is pure with respect to cofibrations, we have a lift $h : C_U(V) \to Z$.
Then $[h_0,h_1] \circ h$ is a homotopy between $f$ and $f''$.
\endproof

\begin{thm}[main]
Let $\C$ be a bicomplete category, and let $\I$ be a set of maps of $\C$ such that the domains and the codomains of maps in $\I$ are small relative to $\Icell$.
Suppose that for every map $i : U \to V$ in $\I$, there exists a reflexive relative cylinder object $C_U(V)$ such that $[\cyli_0,\cyli_1] : V \amalg_U V \to C_U(V) \in \Icof$.

Then there exists a model structure on $\C$ with $\Icof$ as the class of cofibrations and $\Jcof[_\I]$ as the class of trivial cofibrations
if and only if condition~\eqref{cond:strong-main} holds.
\end{thm}
\proof
First, suppose that such a model structure on $\C$ exists.
Then every object is fibrant (since $\Jcof[_\I]$ are split monomorphisms by \rlem{split-mono}) and $C_U(V)$ is a correct cylinder object.
By \rprop{min-we}, $\we_\I$ is the class of weak equivalences.
Hence, condition~\eqref{cond:strong-main} holds.

Now, suppose that condition~\eqref{cond:strong-main} holds.
Let us verify the conditions of \rprop{model-cat}:
\begin{description}
\item[(A1)] Since the domains of maps in $\J_\I$ are the codomains of maps in $\I$, the classes $\I$ and $\J_\I$ permit the small object argument by assumption.
\item[(A2)] The closure of $\we_\I$ under retracts is obvious.
One part of the 2-out-of-3 property holds by assumption and the other two parts follow from \rlem{main}.
\item[(A3)] Since $C_U(V)$ is reflexive, if a map has RLP with respect to every map in $\I$, it also has RLP up to relative homotopy with respect to these maps.
\item[(A4)] Maps in $\J_\I$ are cofibrations by definition. Maps in $\Jcell[_\I]$ are weak equivalences by \rlem{main}.
\item[(A5)] \Rprop{triv-fib-iinj} implies that if a map is a weak equivalence and has RLP with respect to $\Jinj[_\I]$, then it has RLP with respect to $\I$.
\end{description}
\endproof

If maps in $\I$ satisfy some mild additional assumptions, then we can simplify the condition in \rthm{main}:
\begin{prop}[main]
Suppose that the domains of maps in $\I$ are cofibrant.
Then conditions \eqref{cond:main} and \eqref{cond:strong-main} are equivalent.
\end{prop}
\proof
Condition~\eqref{cond:strong-main} obviously implies condition~\eqref{cond:main}.
Let us prove the converse.
Let $f : X \to Y$ and $g : Y \to Z$ be maps such that $f \in \we_\I$ and $g \circ f \in \we_\I$.
By \rlem{main}, we can factor $f$ into maps $f' : X \to X' \in \Jcell[_\I]$ and $g' : X' \to Y \in \Iinj$.
Since $f' \in \Jcell[_\I]$ and $g \circ g' \circ f' = g \circ f \in \we_\I$, condition~\eqref{cond:main} implies that $g \circ g' \in \we_\I$.
Consider the following diagram:
\[ \xymatrix{ & X' \ar[d]^{g'} \\
              U \ar[r]_u \ar[d]_i \ar@{-->}[ur]^{u'} & Y \ar[d]^g \\
              V \ar[r]_v & Z
            } \]
Since $U$ is cofibrant, we have a lift $u' : U \to X'$.
Since $g \circ g' \in \we_\I$, we have a lift $v' : V \to X'$ such that $g \circ g' \circ v' \sim_i v$.
Then $g' \circ v'$ is a required lift in the original square.
\endproof

Thus the main problem is to verify condition~\eqref{cond:main}.
There are a few ways to do this, but the idea is the same:
we need to assume that there exists some notion of homotopy on sets of maps which satisfies some conditions.
There are two standard ways to do this: using path and cylinder objects.
Now, we present this constructions.
Note that we use very weak notions of path and cylinder objects (see section~\ref{sec:preliminaries}).

\begin{prop}[main-path]
Condition~\eqref{cond:main} holds if and only if for every object $X$,
there exists a path object $P(X)$ such that the following conditions hold:
\begin{enumerate}
\item For every $f : X \to Y$, there exists a morphism of path objects $(f,P(f)) : P(X) \to P(Y)$,
\item Either $p_1$ has RLP with respect to $\I$ or there exists a map $s : P(X) \to P(X)$ such that $p_0 \circ s = p_1$ and $p_1 \circ s = p_0$.
\item $p_0$ has RLP with respect to $\I$.
\item \label{it:either} Either path objects are reflexive and maps $\langle p_0, p_1 \rangle : P(X) \to X \times X$ have RLP with respect to $\J_\I$
or maps in $\Jcell[_\I]$ have RLP up to $\sim^{r*}$ with respect to the domains of maps in $\I$.
\end{enumerate}
\end{prop}
\proof
If condition~\eqref{cond:main} holds, then we can define path objects as usual using a factorization
of the diagonal $X \to X \times X$ into maps $X \to P(X) \in \Jcell[_\I]$ and $P(X) \to X \times X \in \Jinj[_\I]$.
The second and the third conditions follow from \rlem{main} and \rprop{triv-fib-iinj}.
The first condition is obvious since $P(f)$ can be constructed as a lift in the following diagram:
\[ \xymatrix{ X \ar[r] \ar[d]            & Y \ar[r]          & P(Y) \ar[d] \\
              P(X) \ar[r] \ar@{-->}[urr] & X \times X \ar[r] & Y \times Y
            } \]
The last condition is also obvious since the path object is reflexive and $\langle p_0, p_1 \rangle \in \Jinj[_\I]$ by construction.

Let us prove the converse.
If the second option of \eqref{it:either} holds, then condition~\eqref{cond:main} holds by \rlem{we-bot}.
Thus we only need to prove that the first option of \eqref{it:either} implies the second.
Indeed, let $f : X \to Y$ be a map in $\Jcell[_\I]$, and let $g : Y \to X$ be its retraction.
Consider the following diagram:
\[ \xymatrix{ X \ar[rr]^-{t \circ f} \ar[d]_f & & P(Y) \ar[d]^{\langle p_0, p_1 \rangle} \\
              Y \ar[rr]_-{\langle f \circ g, id_Y \rangle} \ar@{-->}[urr] & & Y \times Y
            } \]
We have a lift $h : Y \to P(Y)$ which gives us a right homotopy between $f \circ g$ and $id_Y$.
Now, for every $u : U \to Y$, we can define a lift $u' = g \circ u : U \to X$ and a homotopy $h \circ u$ between $f \circ u'$ and $u$.
Thus $f$ has RLP up to $\sim^r$ with respect to any object.
\endproof

\begin{cor}[main-path]
Let $\C$ be a category and let $\I$ be a set of maps of $\C$ such that the domains of maps in $\I$ are cofibrant and the conditions of \rthm{main} are satisfied.
Then there exists a model structure on $\C$ with $\Icof$ as the class of cofibrations and $\Jcof[_\I]$ as the class of trivial cofibrations
if and only if there exist path objects that satisfy the conditions of \rprop{main-path}.
\end{cor}

\begin{exmp}
An example of a model category defined in this way is a folk model structure on the category of $\omega$-categories which was constructed in \cite{folk}.
The conditions of \rprop{main-path} follow from the results of \cite{folk}, but some of them are not needed for this proposition.
Thus the construction of this model structure can be somewhat simplified using the general results of this section.
\end{exmp}

Instead of path objects, we could try to use cylinder objects to verify condition~\eqref{cond:main}.
The advantage of this approach is that we do not need to define a cylinder object for every object of the category,
only for objects that are domains and codomains of generating cofibrations.
The disadvantage is that we still need to verify that maps in $\Jcell$ has RLP up to $\sim$ with respect to the domains of generating cofibrations.

Recall that a cylinder object for an object $X$ is any object $C(X)$ together with maps $\cyli_0,\cyli_1 : X \to C(X)$.
We do not require $C(X)$ to be a proper cylinder object.
That is, maps $X \to C(X)$ may not be weak equivalences and a map $C(X) \to X$ may not exist at all.

\begin{prop}[main-cyl]
Suppose that, for every object $X$ which is either the domain or the codomain of a map in $\I$,
there exists a cylinder object $C(X)$ such that the following conditions hold:
\begin{enumerate}
\item \label{it:cyl-func} For every $i : U \to V \in \I$, there exists a morphism of cylinder objects $(i,C(i)) : C(U) \to C(V)$.
\item There exists a map $s : C(X) \to C(X)$ such that $s \circ i_0 = i_1$, $s \circ i_1 = i_0$, and $C(i) \circ s = s \circ C(i)$.
\item These cylinder objects satisfy the homotopy extension property. That is,
if $i : U \to V \in \cof$, $u : U \to X$ and $v : V \to X$ are maps, and $h : C(U) \to X$ is a homotopy between $v \circ i$ and $u$,
then there exists a map $v' : V \to X$ and a homotopy $h' : C(V) \to X$ between $v$ and $v'$ such that $h = h' \circ C(i)$.
\item \label{it:cyl-hom} Maps in $\Jcell[_\I]$ have RLP up to $\sim^*$ with respect to the domains of maps in $\I$.
\item \label{it:cyl-cyl} For every $i : U \to V \in \I$, we have a lift $p$ in the following diagram:
\[ \xymatrix{ V \amalg_U V \ar[r]^-f \ar[d]_{[\cyli_0,\cyli_1]} & T \\
              C_U(V) \ar@{-->}[ur]_p
            } \]
where $T = C_U(V) \amalg_{(V \amalg_U V)} (C(V) \amalg_{C(U)} C(V))$ is the pushout of maps $[\cyli_0,\cyli_1] : V \amalg_U V \to C_U(V)$
and $\cyli_0 \amalg_{\cyli_0} \cyli_0 : V \amalg_U V \to C(V) \amalg_{C(U)} C(V)$,
and $f : V \amalg_U V \to T$ is the composite $V \amalg_U V \xrightarrow{\cyli_1 \amalg_{\cyli_1} \cyli_1} C(V) \amalg_{C(U)} C(V) \to T$.
\end{enumerate}
Then condition~\eqref{cond:main} holds.
If the domains of maps in $\I$ are cofibrant, then the converse is true.
\end{prop}
\proof
Assume that conditions \eqref{it:cyl-func}-\eqref{it:cyl-cyl} hold.
Then the proof of \eqref{cond:main} is similar to the proof of \rlem{we-bot}.
Suppose that we have a commutative square as below, where $f \in \Jcell[_\I]$ and $g \circ f \in \we_\I$.
By condition~\eqref{it:cyl-hom}, there exists a map $u_x : U \to X$ and a sequence of homotopies $h^1, \ldots, h^n : C(U) \to Y$ between $f \circ u_x$ and $u$.
\[ \xymatrix{   \ar@{}[dr]|(.7){\sim^*}               & X \ar[d]^f \\
              U \ar@{-->}[ur]^{u_x} \ar[d]_i \ar[r]_u & Y \ar[d]^g \\
              V \ar[r]_v                              & Z
            } \]
Then we have a sequence of homotopies $g \circ h^1 \circ s, \ldots, g \circ h^n \circ s$ between $v \circ i$ and $g \circ f \circ u_x$.
By the homotopy extension property, there exists a map $v_z : V \to Z$ and a sequence of homotopies $h^1_3, \ldots, h^n_3 : C(V) \to Z$ between $v$ and $v_z$ such that $h^j_3 \circ C(i) = g \circ h^j \circ s$.

Since $g \circ f \in \we_\I$, there exists a map $v_x : V \to X$ and a homotopy $h_2 : C_U(V) \to Z$ between $g \circ f \circ v_x$ and $v_z$.
Note that $h^1 \circ s \circ s, \ldots, h^n \circ s \circ s$ is a sequence of homotopies between $f \circ u_x = f \circ v_x \circ i$ and $u$.
Thus, by the homotopy extension property, we have a map $v_y : V \to Y$ and a sequence of homotopies $h^1_y, \ldots, h^n_y : C(V) \to Y$ between $f \circ v_x$ and $v_y$ such that $h^j_y \circ C(i) = h^j \circ s \circ s$.
In particular, $v_y \circ i = u$.
Thus we only need to prove that $g \circ v_y$ and $v$ are homotopic relative to $i$.
If we let $h^j_1 = g \circ h^j_y$, then $h^j_1 \circ C(i) = g \circ h^j \circ s \circ s = h^j_3 \circ s \circ C(i)$.
Thus we have a sequence of maps $[h^j_1, h^j_3 \circ s] : C(V) \amalg_{C(U)} C(V) \to Z$.

If $[h^j_1, h^j_3 \circ s] \circ (\cyli_0 \amalg_{\cyli_0} \cyli_0) : V \amalg_U V \to Z$ extends to $C_U(V)$, then $[h^j_1, h^j_3 \circ s] \circ (\cyli_1 \amalg_{\cyli_1} \cyli_1)$ also extends to $C_U(V)$.
Indeed, if $[h^j_1, h^j_3 \circ s] \circ (\cyli_0 \amalg_{\cyli_0} \cyli_0) = h_0 \circ [\cyli_0,\cyli_1]$ for some $h_0 : C_U(V) \to Z$,
then there is a map $q : T \to Z$ constructed by the universal property of the pushout $T$.
Then $q \circ p : C_U(V) \to Z$ is an extension of $[h^j_1, h^j_3 \circ s] \circ (\cyli_1 \amalg_{\cyli_1} \cyli_1)$, where $p : C_U(V) \to T$ is the map from condition~\eqref{it:cyl-cyl}.

Finally, note that $h_2$ is an extension of $[g \circ f \circ v_x, v_z] = [h^1_1, h^1_3 \circ s] \circ (\cyli_0 \amalg_{\cyli_0} \cyli_0)$.
It follows that we have an extension of $[h^n_1, h^n_3 \circ s] \circ (\cyli_1 \amalg_{\cyli_1} \cyli_1) = [g \circ v_y, v]$,
which defines a relative homotopy between $g \circ v_y$ and $v$.

Now, let us assume that the domains of maps in $\I$ are cofibrant and condition~\eqref{cond:main} holds.
Then \rthm{main} and \rprop{main} imply that we have a structure of a model category.
Thus the conditions \eqref{it:cyl-func}-\eqref{it:cyl-hom} are obvious.
Let us prove condition~\eqref{it:cyl-cyl}.
Consider the following diagram:
\[ \xymatrix{                                                                    & & V \amalg_U V \ar[r] \ar[d]_{\cyli_0 \amalg_{\cyli_0} \cyli_0} & C_U(V) \ar[d] \\
                V \amalg_U V \ar[rr]^-{\cyli_1 \amalg_{\cyli_1} \cyli_1} \ar[dd] & & C(V) \amalg_{C(U)} C(V) \ar[d] \ar[r]                         & \po T \ar[d] \\
                                                                                 & & V \amalg_U V \ar[r]                                           & \po C_U(V) \ar@{=}[d] \\
                C_U(V) \ar@{=}[rrr]                                              & &                                                               & C_U(V)
            } \]
Since the map $\cyli_0 \amalg_{\cyli_0} \cyli_0 : V \amalg_U V \to C(V) \amalg_{C(U)} C(V)$ is a trivial cofibration, the map $C_U(V) \to T$ is also a trivial cofibration.
It follows that the map $T \to C_U(V)$ is a weak equivalence, which implies that it has the RLP with respect to cofibrations up to a relative homotopy.
Since the horizontal map $V \amalg_U V \to T$ in the diagram is $f$, we have a map $p : C_U(V) \to T$ such that $V \amalg_U V \to C_U(V) \overset{p}\to T$ equals to $f$.
\endproof

\begin{cor}[main-cyl]
Let $\C$ be a category and let $\I$ be a set of maps of $\C$ such that the domains of maps in $\I$ are cofibrant and the conditions of \rthm{main} are satisfied.
Then there exists a model structure on $\C$ with $\Icof$ as the class of cofibrations and $\Jcof[_\I]$ as the class of trivial cofibrations
if and only if there exist cylinder objects that satisfy the conditions of \rprop{main-cyl}.
\end{cor}

\begin{exmp}
An example of a model category defined in this way is the usual model structure on topological spaces.
If we define $C(X)$ as the usual cylinder $[0,1] \times X$, then the conditions of \rprop{main-cyl} are easy to verify directly.
\end{exmp}

\begin{references*}

\bibitem{cisinski}
Denis-Charles Cisinski, \emph{Théories homotopiques dans les topos}, Journal
  of Pure and Applied Algebra \textbf{174} (2002), no.~1, 43 -- 82.

\bibitem{hirschhorn}
P.S. Hirschhorn, \emph{Model categories and their localizations}, Mathematical
  Surveys and Monographs, Amer Mathematical Society, 2009.

\bibitem{hovey}
M.~Hovey, \emph{Model categories}, Mathematical Surveys and Monographs,
  American Mathematical Society, 1999.

\bibitem{alg-models}
V.~{Isaev}, \emph{Model structures on categories of models of type theories},
  (2016), \href {http://arxiv.org/abs/1607.07407} {\path{arXiv:1607.07407}}.

\bibitem{folk}
Yves Lafont, Fran{\c c}ois M{\'e}tayer, and Krzysztof Worytkiewicz, \emph{{A
  folk model structure on omega-cat}}, Advances in Mathematics \textbf{224}
  (2010), no.~3, 1183--1231.

\bibitem{lurie-topos}
Jacob Lurie, \emph{Higher topos theory}, Annals of mathematics studies,
  Princeton University Press, Princeton, N.J., Oxford, 2009.

\bibitem{quillen}
Daniel~G. Quillen, \emph{Homotopical algebra}, Lecture Notes in Mathematics,
  vol.~43, Springer-Verlag, 1967.

\bibitem{left-det}
J.~Rosick{\'y} and W.~Tholen, \emph{Left-determined model categories and
  universal homotopy theories}, Transactions of the American Mathematical
  Society \textbf{355} (2003), 3611--3623.

\bibitem{vogt-we}
R.~M. Vogt, \emph{{The HELP-Lemma And Its Converse In Quillen Model
  Categories}},  (2010), \href {http://arxiv.org/abs/1004.5249}
  {\path{arXiv:1004.5249}}.

\bibitem{olschok}
Marc Olschok, \emph{Left determined model structures for locally presentable
  categories}, Applied Categorical Structures \textbf{19} (2011), no.~6,
  901--938.

\end{references*}

\end{document}
