\documentclass[reqno]{amsart}

\usepackage{amssymb}
\usepackage{hyperref}
\usepackage{mathtools}
\usepackage[all]{xy}
\usepackage{verbatim}
\usepackage{ifthen}
\usepackage{xargs}
\usepackage{bussproofs}
\usepackage{turnstile}
\usepackage{etex}

\hypersetup{colorlinks=true,linkcolor=blue}

\renewcommand{\turnstile}[6][s]
    {\ifthenelse{\equal{#1}{d}}
        {\sbox{\first}{$\displaystyle{#4}$}
        \sbox{\second}{$\displaystyle{#5}$}}{}
    \ifthenelse{\equal{#1}{t}}
        {\sbox{\first}{$\textstyle{#4}$}
        \sbox{\second}{$\textstyle{#5}$}}{}
    \ifthenelse{\equal{#1}{s}}
        {\sbox{\first}{$\scriptstyle{#4}$}
        \sbox{\second}{$\scriptstyle{#5}$}}{}
    \ifthenelse{\equal{#1}{ss}}
        {\sbox{\first}{$\scriptscriptstyle{#4}$}
        \sbox{\second}{$\scriptscriptstyle{#5}$}}{}
    \setlength{\dashthickness}{0.111ex}
    \setlength{\ddashthickness}{0.35ex}
    \setlength{\leasturnstilewidth}{2em}
    \setlength{\extrawidth}{0.2em}
    \ifthenelse{%
      \equal{#3}{n}}{\setlength{\tinyverdistance}{0ex}}{}
    \ifthenelse{%
      \equal{#3}{s}}{\setlength{\tinyverdistance}{0.5\dashthickness}}{}
    \ifthenelse{%
      \equal{#3}{d}}{\setlength{\tinyverdistance}{0.5\ddashthickness}
        \addtolength{\tinyverdistance}{\dashthickness}}{}
    \ifthenelse{%
      \equal{#3}{t}}{\setlength{\tinyverdistance}{1.5\dashthickness}
        \addtolength{\tinyverdistance}{\ddashthickness}}{}
        \setlength{\verdistance}{0.4ex}
        \settoheight{\lengthvar}{\usebox{\first}}
        \setlength{\raisedown}{-\lengthvar}
        \addtolength{\raisedown}{-\tinyverdistance}
        \addtolength{\raisedown}{-\verdistance}
        \settodepth{\raiseup}{\usebox{\second}}
        \addtolength{\raiseup}{\tinyverdistance}
        \addtolength{\raiseup}{\verdistance}
        \setlength{\lift}{0.8ex}
        \settowidth{\firstwidth}{\usebox{\first}}
        \settowidth{\secondwidth}{\usebox{\second}}
        \ifthenelse{\lengthtest{\firstwidth = 0ex}
            \and
            \lengthtest{\secondwidth = 0ex}}
                {\setlength{\turnstilewidth}{\leasturnstilewidth}}
                {\setlength{\turnstilewidth}{2\extrawidth}
        \ifthenelse{\lengthtest{\firstwidth < \secondwidth}}
            {\addtolength{\turnstilewidth}{\secondwidth}}
            {\addtolength{\turnstilewidth}{\firstwidth}}}
        \ifthenelse{\lengthtest{\turnstilewidth < \leasturnstilewidth}}{\setlength{\turnstilewidth}{\leasturnstilewidth}}{}
    \setlength{\turnstileheight}{1.5ex}
    \sbox{\turnstilebox}
    {\raisebox{\lift}{\ensuremath{
        \makever{#2}{\dashthickness}{\turnstileheight}{\ddashthickness}
        \makehor{#3}{\dashthickness}{\turnstilewidth}{\ddashthickness}
        \hspace{-\turnstilewidth}
        \raisebox{\raisedown}
        {\makebox[\turnstilewidth]{\usebox{\first}}}
            \hspace{-\turnstilewidth}
            \raisebox{\raiseup}
            {\makebox[\turnstilewidth]{\usebox{\second}}}
        \makever{#6}{\dashthickness}{\turnstileheight}{\ddashthickness}}}}
        \mathrel{\usebox{\turnstilebox}}}

% \providecommand\WarningsAreErrors{false}
% \ifthenelse{\equal{\WarningsAreErrors}{true}}{\renewcommand{\GenericWarning}[2]{\GenericError{#1}{#2}{}{This warning has been turned into a fatal error.}}}{}

\newcommand{\axlabel}[1]{(#1) \phantomsection \label{ax:#1}}
\newcommand{\axtag}[1]{\label{ax:#1} \tag{#1}}
\newcommand{\axref}[1]{(\hyperref[ax:#1]{#1})}

\newcommand{\newref}[4][]{
\ifthenelse{\equal{#1}{}}{\newtheorem{h#2}[hthm]{#4}}{\newtheorem{h#2}{#4}[#1]}
\expandafter\newcommand\csname r#2\endcsname[1]{#3~\ref{#2:##1}}
\expandafter\newcommand\csname R#2\endcsname[1]{#4~\ref{#2:##1}}
\expandafter\newcommand\csname n#2\endcsname[1]{\ref{#2:##1}}
\newenvironmentx{#2}[2][1=,2=]{
\ifthenelse{\equal{##2}{}}{\begin{h#2}}{\begin{h#2}[##2]}
\ifthenelse{\equal{##1}{}}{}{\label{#2:##1}}
}{\end{h#2}}
}

\newref[section]{thm}{Theorem}{Theorem}
\newref{lem}{Lemma}{Lemma}
\newref{prop}{Proposition}{Proposition}
\newref{cor}{Corollary}{Corollary}

\theoremstyle{definition}
\newref{defn}{Definition}{Definition}
\newref{example}{Example}{Example}

\theoremstyle{remark}
\newref{remark}{Remark}{Remark}

\newcommand{\red}{\Rightarrow}
\newcommand{\deq}{\equiv}
\renewcommand{\ll}{\llbracket}
\newcommand{\rr}{\rrbracket}
\newcommand{\cat}[1]{\mathbf{#1}}
\newcommand{\C}{\cat{C}}
\newcommand{\D}{\cat{D}}
\newcommand{\Set}{\cat{Set}}
\newcommand{\PSet}{\cat{PSet}}
\newcommand{\PMnd}{\cat{PMnd}}
\newcommand{\ccat}{\cat{CCat}}
\newcommand{\algtt}{\cat{TT}}
\newcommand{\substTh}{\mathbb{S}}
\newcommand{\Mod}[1]{#1\text{-}\cat{Mod}}
\newcommand{\PAlg}[1]{#1\text{-}\cat{PAlg}}
\newcommand{\Th}{\cat{Th}}
\newcommand{\St}{\cat{St}}
\newcommand{\PSt}{\cat{PSt}}
\newcommand{\cSt}[1][c]{#1\text{-}\St}
\newcommand{\emptyCtx}{*}
\newcommand{\ThC}{\Th_{\mathcal{C}}}

\numberwithin{figure}{section}

\newcommand{\pb}[1][dr]{\save*!/#1-1.2pc/#1:(-1,1)@^{|-}\restore}
\newcommand{\po}[1][dr]{\save*!/#1+1.2pc/#1:(1,-1)@^{|-}\restore}

\begin{document}

\title{Algebraic Presentations of Dependent Type Theories}

\author{Valery Isaev}

\begin{abstract}
In this paper, we propose an abstract definition of dependent type theories as essentially algebraic theories.
One of the main advantages of this definition is its composability: simple theories can be combined into more complex ones,
and different properties of the resulting theory may be deduced from properties of the basic ones.
We define a category of algebraic dependent type theories which allows us not only to combine theories but also to consider equivalences between them.
We also study models of such theories and show that one can think of them as contextual categories with additional structure.
\end{abstract}

\maketitle

 \makeatletter
    \providecommand\@dotsep{5}
  \makeatother

\section{Introduction}

Type theories with dependent types originally were defined by Per Martin-L\"{o}f, who introduced several versions of the system \cite{MLTT72,MLTT73,MLTT79}.
There were also several theories and extensions of Martin-L\"{o}f's theory proposed by different authors (\cite{CoC,luo94} to name a few).
These theories may have different inference rules, different computation rules, and different constructions.
Many of these theories have common parts and similar properties,
but the problem is that there is no general definition of a type theory such that all of these theories would be a special case of this definition,
so that their properties could be studied in general and applied to specific theory when necessary.
In this paper we propose such a definition based on the notion of essentially algebraic theories.

Another problem of the usual way of defining type theories is that they are not composable.
Some constructions in type theories are independent of each other (such as $\Pi$, $\Sigma$, and $Id$ types),
and others may dependent on other constructions (such as universes),
so we could hope that we can study these constructions independently (at least if they are of the first kind)
and deduce properties of combined theory from the properties of these basic constructions.
But this is not the way it is usually done.
For example, constructing models of dependent type theories is a difficult task because of the so called coherence problem.
There are several proposed solutions to this problems, but the question we are interested in is how to combine them.
Often only the categorical side of the question is considered,
but some authors do consider specific theories \cite{streicher,pitts},
and the problem in this case is that their work cannot be applied to other similar theories (at least formally).

When defining a type theory there are certain questions to be addressed regarding syntactic traits of the theory.
One such question is how many arguments to different construction can be omitted and how to restore them when constructing a model of the theory.
For example, we want to define application as a function of two arguments $app(f,a)$, but sometimes it is convenient to have additional arguments which allows to infer a type of $f$.
It is possible to prove that additional information in the application term may be omitted (for example, see \cite{streicher}), but it is a nontrivial task.
Another question of this sort is whether we should use a typed or an untyped equality.
Typed equality is easier to handle when defining a model of the theory, but untyped is closer to actual implementation of the language.
Algebraic approach allows us to separate these syntactic details from essential aspects of the theory.

Yet another problem is that some constructions may be defined in several different ways.
For example, $\Sigma$ types can be defined using projections (\rexample{sigma-eta}) and using an eliminator (\rexample{sigma-no-eta}).
The question then is whether these definitions are equivalent in some sense.
The difficulty of this question stems from the fact that some equivalences may hold in one definition judgmentally, but in the other only propositionally;
so it may be difficult (or impossible) to construct a map from the first version of the definition to the second one.

In this paper, using the formalism of essentially algebraic theories, we introduce the notion of
\emph{algebraic dependent type theories} which provide a possible solution the problems described above.
We define a category of algebraic dependent type theories.
Coproducts and more generally colimits in this category allow us to combine simple theories into more complex ones.
For example, the theory with $\Sigma$, $\Pi$ and $Id$ types may be described as coproduct $T_\Sigma \amalg T_\Pi \amalg T_{Id}$
where $T_\Sigma$, $T_\Pi$ and $T_{Id}$ are theories of $\Sigma$, $\Pi$ and $Id$ types respectively.

There is a natural notion of a model of an essentially algebraic theory.
Thus the algebraic approach to defining type theories automatically equips every type theory with a (locally presentable) category of its models.
We will show that models of the initial theory are precisely contextual categories,
and that models of an arbitrary theory are contextual categories with an additional structure (which depends on the theory).
An example of a general construction that works for all theories with enough structure is the construction of a model structure on the category of models described in \cite{alg-models}.

Since we have a category of type theories, there is a natural notion of equivalence between them, namely the isomorphism.
In most cases this equivalence is too strong, so it is necessary to consider weaker notions of equivalence, but in some cases it might be useful.
For example, if two theories differ only by the amount of arguments to some of the constructions,
then they are isomorphic (assuming omitted arguments can be inferred from the rest).
A weaker notion of equivalence of theories is Morita equivalence.
Two theories are Morita equivalent if there is a Quillen equivalence between the categories of models of these theories.
We will not consider this notion in this paper.

Usually, we can use all constructions of a type theory in every context.
We consider an additional structure on theories which allows us to do this.
We call theories with this additional structure \emph{prestable}.
Then, an algebraic dependent type theory is a prestable theory with substitutions which commute with every operation in the theory.
We also consider \emph{stable} theories in which all axioms are stable under context extensions.
If we think of models of a prestable theory as some sort of category with some additional structure,
then the prestable structure allows us to pass to slices of this category.
Then a prestable theory is stable if not only the category itself but also every slice category has this additional structure.

The paper is organized as follows.
In section 2, we define the category of partial Horn theories and discuss its properties.
In section 3, we define an example of partial Horn theory and prove that the category of its models is equivalent to the category of contextual categories.
In section 4, we define algebraic type theories, describe a simplified version of the syntax that can be used with these theories, and give a few standard examples of such theories.

\section{Partial Horn theories}
\label{sec:PHT}

There are several equivalent ways of defining essentially algebraic theories (\cite{LPC}, \cite{GAT}, \cite{PHL}, \cite[D 1.3.4]{elephant}).
We use approach introduced in \cite{PHL} under the name of partial Horn theories since it is the most convenient one.
We define morphisms of partial Horn theories in terms of morphisms of monads and left modules over them.
In this section we review necessary for our development parts of the theory of monads, left modules over them and partial Horn theories.
We also define algebraic dependent type theories as certain partial Horn theories.

\subsection{Monads and left modules over them}

We recall definitions of monads and left modules over a monad.
For our purposes the following definitions (see \cite{manes-algebraic-theories}) will be more convenient than the ordinary ones.
\begin{defn}
A \emph{monad} $(T,\eta,(-)^*)$ on a category $\C$ consists of a function $T : Ob(\C) \to Ob(\C)$,
a function $\eta$ that to each $A \in Ob(\C)$ assign a morphism $\eta_A : A \to T(A)$,
and a function that to each $A,B \in Ob(\C)$ assigns a function $(-)^* : Hom_\C(A,T(B)) \to Hom_\C(T(A),T(B))$, satisfying the following conditions:
\begin{itemize}
\item $\eta_A^* = id_{T(A)}$.
\item For every $\rho : A \to T(B)$, $\rho^* \circ \eta_A = \rho$.
\item For every $\rho : A \to T(B)$, $\sigma : B \to T(C)$, $\sigma^* \circ \rho^* = (\sigma^* \circ \rho)^*$.
\end{itemize}

A \emph{left module} $(M,(-)^\circ)$ over a monad $(T,\eta,(-)^*)$ with values in a category $\D$ consists of a function $M : Ob(\C) \to Ob(\D)$
and a function that to each $A,B \in Ob(\C)$ assigns a function $(-)^\circ : Hom_\C(A,T(B)) \to Hom_\D(M(A),M(B))$, satisfying the following conditions:
\begin{itemize}
\item $\eta_A^\circ = id_{M(A)}$.
\item For every $\rho : A \to T(B)$, $\sigma : B \to T(C)$, $\sigma^\circ \circ \rho^\circ = (\sigma^* \circ \rho)^\circ$.
\end{itemize}
\end{defn}
These data and axioms imply that $T$ and $M$ are functorial: if $f : A \to B$, then we can define $T(f)$ as $(\eta_B \circ f)^*$ and $M(f)$ as $(\eta_B \circ f)^\circ$.
Moreover, $\eta$, $(-)^*$ and $(-)^\circ$ are natural.

\begin{defn}
A morphism of monads $(T,\eta,(-)^*)$ and $(T',\eta',(-)^{*'})$ on $\C$ is a function $\alpha$ that to each $A \in Ob(\C)$ assigns a morphism $\alpha_A : T(A) \to T'(A)$,
satisfying the following conditions:
\begin{itemize}
\item $\alpha_A \circ \eta_A = \eta'_A$.
\item For every $\rho : A \to T(B)$, $\alpha_B \circ \rho^* = (\alpha_B \circ \rho)^{*'} \circ \alpha_A$.
\end{itemize}

Let $(M,(-)^\circ)$ and $(M',(-)^{\circ'})$ be left modules with values in $\D$ over monads $(T,\eta,(-)^*)$ and $(T',\eta',(-)^{*'})$ respectively.
A morphism between them is a pair of functions $(\alpha,\beta)$, where $\alpha$ is a morphism of monads $T$ and $T'$,
and $\beta$ assigns to each $A \in Ob(\C)$ a morphism $\beta_A : M(A) \to M'(A)$,
such that, for every $\rho : A \to T(B)$, $\beta_B \circ \rho^\circ = (\alpha_B \circ \rho)^{\circ'} \circ \beta_A$.
\end{defn}
These data and axioms imply that $\alpha$ and $\beta$ are natural.

Let $\mathcal{S}$ be a set of sorts, and let $(T,\eta,(-)^*)$ be a monad on the category of $\mathcal{S}$-sets.
We think of elements of $T(V)_s$ as terms of sort $s$ with free variables in $V$.
Given $t \in T(V)_s$ and $\rho : V \to T(V')$, we will write $t[\rho] \in T(V')_s$ for $\rho^*(t)$.
Let $(F,(-)^\circ)$ be a left module over $T$ with values in $\Set$.
We think of elements of $F(V)$ as formulas with free variables in $V$.
Given $\varphi \in F(V)$ and $\rho : V \to T(V')$, we will write $\varphi[\rho] \in F(V')$ for $\rho^\circ(\varphi)$.

Let $T : \Set^\mathcal{S} \to \Set^\mathcal{S}$ be a monad.
Then \emph{a free variables structure} on $T$ is a function $FV$ that to each $t \in T(V)_s$ assigns a subset of $V$, that is $FV(t) \subseteq V$, called the set of free variables of $t$.
This function must satisfy the following conditions:
\begin{align*}
FV(\eta(x)) & = x \\
FV(t[\rho]) & = \bigcup_{x \in FV(t)} FV(\rho(x))
\end{align*}

Let $F : \Set^\mathcal{S} \to \Set$ be a left module over $T$.
Then \emph{a free variables structure} on $F$ is a function $FV$ that to each $\varphi \in F(V)$ assigns a subset of $V$, that is $FV(\varphi)$, called the set of free variables of $\varphi$.
This function must satisfy the following condition:
\[ FV(\varphi[\rho]) = \bigcup_{x \in FV(\varphi)} FV(\rho(x)) \]

\emph{A module of formulas} over $T$ is a left module $F$ over $T$ together with a function
    $\land : F(V) \times F(V) \to F(V)$ and a constant $\top \in F(V)$ for every $V \in \Set^\mathcal{S}$, satisfying the following conditions:
\begin{itemize}
\item For every $\rho : V \to T(V')$, $\top[\rho] = \top$.
\item For every $\rho : V \to T(V')$, $(\varphi \land \psi)[\rho] = \varphi[\rho] \land \psi[\rho]$.
\end{itemize}

For every monad $T$ on $\Set^\mathcal{S}$ we define a left module $E$ with values in $\Set$.
For every $V \in \Set^\mathcal{S}$, let $E(V)$ be the set of triples $(s,t,t')$, where $s \in \mathcal{S}$, and $t,t' \in T(V)_s$.
For every $\rho : V \to T(V')$ and $(s,t,t') \in E(V)$, we let $(s,t,t')[\rho] = (s,t[\rho],t'[\rho])$.
We think of $(s,t,t')$ as a formula asserting the equality of terms $t$ and $t'$.
We write $t =_s t'$ (or simply $t = t'$) for $(s,t,t')$.
\emph{A module of formulas with equality} over $T$ is a module $F$ of formulas over $T$ together with a morphism $e : E \to F$.

\begin{defn}[mon-pres]
A \emph{monadic presentation of a partial Horn theory} is a triple $(T,F,\mu)$, where
    $T : \Set^\mathcal{S} \to \Set^\mathcal{S}$ is a finitary monad with a free variables structure,
    $F : \Set^\mathcal{S} \to \Set$ is a finitary module of formulas with equality and a free variables structure, and
    $\mu_V : T(V) \times F(V) \to T(V)$ is a function such that the following conditions hold:
\begin{itemize}
\item For every $\rho : V \to T(V')$, $\mu_V(t,\varphi)[\rho] = \mu_{V'}(t[\rho],\varphi[\rho])$.
\item $\mu_V(t, \top) = t$.
\item $\mu_V(t, \varphi \land \psi) = \mu_V(\mu_V(t, \varphi), \psi)$.
\end{itemize}
A morphism of triples $(T,F,\mu)$ and $(T',F',\mu')$ is a morphism $f$ of left modules $(T,F)$ and $(T',F')$ such that $f$ preserves free variables, equality, $\top$, $\land$ and $\mu$.
The category of monadic presentations of partial Horn theories with $\mathcal{S}$ as the set of sorts is denoted by $\PMnd_\mathcal{S}$.
\end{defn}

\subsection{The category of partial Horn theories}
\label{sec:PHT-rules}

Let $\mathcal{S}$ be a set of sorts, $T : \Set^\mathcal{S} \to \Set^\mathcal{S}$ a monad with a free variables structure,
    and $\mathcal{P}$ a set of predicate symbols together with a function that to each $R \in \mathcal{P}$
    assigns its signature $R : s_1 \times \ldots \times s_n$, where $s_1, \ldots s_n \in \mathcal{S}$.

Let $\mathcal{F}$ be a set of function symbols together with a function that to each $\sigma \in \mathcal{F}$ assigns its signature $\sigma : s_1 \times \ldots \times s_n \to s$, where $s_1, \ldots s_n, s \in \mathcal{S}$.
Then we can define an example of a monad over $\Set^\mathcal{S}$.
For each $V \in \Set^\mathcal{S}$ we can define a set $Term_\mathcal{F}(V)_s$ of terms of sort $s$ inductively:
\begin{itemize}
\item If $x \in V_s$, then $x \in Term_\mathcal{F}(V)_s$.
\item If $\sigma : s_1 \times \ldots \times s_n \to s$ and $t_i \in Term_\mathcal{F}(V)_{s_i}$, then $\sigma(t_1, \ldots t_n) \in Term_\mathcal{F}(V)_s$.
\end{itemize}
If $\rho : V \to Term_\mathcal{F}(V')$, then substitution is defined as follows:
\begin{align*}
x[\rho] & = \rho(x) \\
\sigma(a_1, \ldots a_k)[\rho] & = \sigma(a_1[\rho], \ldots a_k[\rho])
\end{align*}
Thus $Term_\mathcal{F} : \Set^\mathcal{S} \to \Set^\mathcal{S}$ is a monad, which we call the standard monad (over $\mathcal{F}$).

An \emph{atomic formula} with free variables in $V$ is an expression either of the form $t_1 =_s t_2$ (we will usually omit $s$ in the notation),
    where $s \in \mathcal{S}$ and $t_1, t_2 \in T(V)_s$, or of the form $R(t_1, \ldots t_n)$, where $R \in \mathcal{P}$, $R : s_1 \times \ldots \times s_n$ and $t_i \in T(V)_{s_i}$.
A \emph{Horn formula} (over $\mathcal{P}$) with free variables in $V$ is an expression of the form $\varphi_1 \land \ldots \land \varphi_n$ where $\varphi_i$ are atomic formulas.
If $n = 0$, then we write such a formula as $\top$.
The set of Horn formulas with free variables in $V$ is denoted by $Form_\mathcal{P}(V)$.
If $\varphi \in Form_\mathcal{P}(V)$ and $\rho : V \to T(V')$, then we will write $\varphi[\rho]$ for a formula defined as follows:
\begin{align*}
(t = t')[\rho] & = (t[\rho] = t'[\rho]) \\
R(t_1, \ldots t_k)[\rho] & = R(t_1[\rho], \ldots t_k[\rho]) \\
(\varphi_1 \land \ldots \land \varphi_n)[\rho] & = \varphi_1[\rho] \land \ldots \land \varphi_n[\rho]
\end{align*}
Thus $Form_\mathcal{P}$ is a left module over $T$.
Moreover, a free variables structure on $Form_\mathcal{P}$ is defined as follows:
\begin{align*}
FV(t = t') & = FV(t) \cup FV(t') \\
FV(R(t_1, \ldots t_k)) & = FV(t_1) \cup \ldots \cup FV(t_k) \\
FV(\varphi_1 \land \ldots \land \varphi_n) & = FV(\varphi_1) \cup \ldots \cup FV(\varphi_n)
\end{align*}

A \emph{Horn sequent} is an expression of the form $\varphi \sststile{}{V} \psi$, where $\varphi$ and $\psi$ are Horn formulas with free variables in $V$.
We will often write $\varphi_1, \ldots \varphi_n \sststile{}{V} \psi_1, \ldots \psi_k$ instead of $\varphi_1 \land \ldots \land \varphi_n \sststile{}{V} \psi_1 \land \ldots \land \psi_k$.
A \emph{partial Horn theory} is a set of Horn sequents.
The rules of \emph{partial Horn logic} are listed below.
If $\mathcal{A}$ is a partial Horn theory, then a \emph{theorem} of $\mathcal{A}$ is a sequent derivable from $\mathcal{A}$ in this logic.
\begin{center}
$\varphi \sststile{}{V} \varphi$ \axlabel{b1}
\qquad
\AxiomC{$\varphi \sststile{}{V} \psi$}
\AxiomC{$\psi \sststile{}{V} \chi$}
\RightLabel{\axlabel{b2}}
\BinaryInfC{$\varphi \sststile{}{V} \chi$}
\DisplayProof
\qquad
$\varphi \sststile{}{V} \top$ \axlabel{b3}
\end{center}

\medskip
\begin{center}
$\varphi \land \psi \sststile{}{V} \varphi$ \axlabel{b4}
\qquad
$\varphi \land \psi \sststile{}{V} \psi$ \axlabel{b5}
\qquad
\AxiomC{$\varphi \sststile{}{V} \psi$}
\AxiomC{$\varphi \sststile{}{V} \chi$}
\RightLabel{\axlabel{b6}}
\BinaryInfC{$\varphi \sststile{}{V} \psi \land \chi$}
\DisplayProof
\end{center}

\medskip
\begin{center}
$\sststile{}{x} x\!\downarrow$ \axlabel{a1}
\qquad
$x = y \land \varphi \sststile{}{V,x,y} \varphi[y/x]$ \axlabel{a2}
\end{center}

\medskip
\begin{center}
\AxiomC{$\varphi \sststile{}{V} \psi$}
\RightLabel{, $x \in FV(\varphi)$, $t \in T(V')$ \axlabel{a3}}
\UnaryInfC{$\varphi[t/x] \sststile{}{V,V'} \psi[t/x]$}
\DisplayProof
\end{center}
\medskip
Here, $t/x$ denotes a function $\rho : V \to T(V \cup V')$ such that $\rho(x) = t$ and $\rho(y) = y$ if $y \neq x$.

Note that this set of rules is a generalization of the one described in \cite{PHL}.
If $T$ is the standard monad $Term_\mathcal{F}$, then these rules are equivalent to the rules from \cite{PHL}.
In particular, the following sequents are derivable if $x \in FV(t)$:
\begin{align*}
R(t_1, \ldots t_k) & \sststile{}{V} t_i = t_i \axtag{a4} \\
t_1 = t_2 & \sststile{}{V} t_i = t_i \axtag{a4'} \\
t[t'/x]\!\downarrow & \sststile{}{V} t' = t' \axtag{a5}
\end{align*}

We will need the following lemmas from \cite{PHL}:
\begin{lem}[cong-a]
For every $u_i,v_i \in T(V)_{s_i}$ and $t \in T(\{ x_1 : s_1, \ldots x_n : s_n\})_s$,
sequents $u_1 = v_1 \land \ldots \land u_n = v_n \sststile{}{V} t[x_i \mapsto u_i] \cong t[x_i \mapsto v_i]$ are theorems of any theory.
\end{lem}

\begin{lem}
Sequent $y = x \land \varphi[y/x] \sststile{}{V} \varphi$ is a theorem of any theory.
\end{lem}

Using the previous lemma we prove the following fact:

\begin{lem}[cong-b]
For every $u_i,v_i \in T(V)_{s_i}$ and $\varphi \in Form_\mathcal{P}(\{ x_1 : s_1, \ldots x_n : s_n\})$,
sequent $u_1 = v_1 \land \ldots \land u_n = v_n \land \varphi[x_i \mapsto u_i] \sststile{}{V} \varphi[x_i \mapsto v_i]$ is a theorem of any theory.
\end{lem}
\begin{proof}
By the previous lemma we have $y_n = x_n \land \varphi[y_n/x_n] \sststile{}{x_1 : s_1, \ldots x_n : s_n, y_n : s_n} \varphi$ is provable.
If we take $\varphi$ to be equal to $y_n = x_n \land \varphi[y_n/x_n]$, then we get sequent
$y_{n-1} = x_{n-1} \land y_n = x_n \land \varphi[y_n/x_n,y_{n-1}/x_{n-1}] \sststile{}{x_1 : s_1, \ldots x_n : s_n, y_{n-1} : s_{n-1}, y_n : s_n} y_n = x_n \land \varphi[y_n/x_n]$.
By \axref{b2} we get sequent
\[ y_{n-1} = x_{n-1} \land y_n = x_n \land \varphi[y_n/x_n,y_{n-1}/x_{n-1}] \sststile{}{x_1 : s_1, \ldots x_n : s_n, y_{n-1} : s_{n-1}, y_n : s_n} \varphi. \]
Repeating this argument we can conclude that
\[ y_1 = x_1 \land \ldots \land y_n = x_n \land \varphi[y_1/x_1, \ldots y_n/x_n] \sststile{}{x_1 : s_1, \ldots x_n : s_n, y_1 : s_1, y_n : s_n} \varphi. \]
By \axref{a3} we conclude that the required sequent is derivable.
\end{proof}

Now we define a functor $PT : \Set^\mathcal{S} \to \Set^\mathcal{S}$ of partial terms.
We let $PT(V)_s$ to be the set of expressions $t|_\varphi$ where $t \in T(V)_s$ and $\varphi \in Form_\mathcal{P}(V)$.
If $\varphi = \top$, then we will write $t|_\varphi$ simply as $t$.
If $p \in PT(V)_s$, $p = t|_\varphi$ and $\psi \in Form_\mathcal{P}(V)$, then we will write $p|_\psi$ for $t|_{\varphi \land \psi}$.

We will use the following abbreviations:
\begin{align*}
t\!\downarrow & \text{ means } t = t \\
\varphi \sststile{}{V} t \leftrightharpoons s & \text{ means } \varphi \land t\!\downarrow \land s\!\downarrow\,\sststile{}{V} t = s \\
\varphi \sststile{}{V} t \cong s & \text{ means } \varphi \land t\!\downarrow\,\sststile{}{V} t = s \text{ and } \varphi \land s\!\downarrow\,\sststile{}{V} t = s \\
\varphi \ssststile{}{V} \psi & \text{ means } \varphi \sststile{}{V} \psi \text{ and } \psi \sststile{}{V} \varphi \\
R(t_1|_{\varphi_1}, \ldots t_k|_{\varphi_k}) & \text{ means } R(t_1, \ldots t_k) \land \varphi_1 \land \ldots \land \varphi_k \\
t|_\varphi = s|_\psi & \text{ means } t = s \land \varphi \land \psi \\
t|_\varphi\!\downarrow & \text{ means } t\!\downarrow\!\land \varphi \\
\chi \sststile{}{V} t|_\varphi \leftrightharpoons s|_\psi & \text{ means } \chi \land t|_\varphi\!\downarrow, s|_\psi\!\downarrow\,\sststile{}{V} t = s \\
\chi \sststile{}{V} t|_\varphi \cong s|_\psi & \text{ means } \chi \land t|_\varphi\!\downarrow\,\sststile{}{V} t = s \land \psi \text{ and } \chi \land s|_\psi\!\downarrow\,\sststile{}{V} t = s \land \varphi
\end{align*}

Now we define substitution functions for partial terms.
For every $\rho : V \to PT(V')$, $t \in T(V)_s$ and $\varphi \in Form_\mathcal{P}(V)$,
we define $t[\rho] \in PT(V')_s$, $\varphi[\rho] \in Form_\mathcal{P}(V')$ and $t_\varphi[\rho] \in PT(V')_s$ as follows:
\begin{align*}
t[\rho] & = t[\rho_1]|_{\bigcup_{x \in FV(t)} \rho_2(x)} \\
R(t_1, \ldots t_k)[\rho] & = R(t_1[\rho], \ldots t_k[\rho]) \\
(\varphi_1 \land \ldots \land \varphi_n)[\rho] & = \varphi_1[\rho] \land \ldots \land \varphi_n[\rho] \\
t|_\varphi[\rho] & = t[\rho]|_{\varphi[\rho]}
\end{align*}
where if $\rho(x) = t|_\varphi$, then $\rho_1(x) = t$ and $\rho_2(x) = \varphi$.
Free variables of $t|_\varphi$ is defined as follows: $FV(t|_\varphi) = FV(t) \cup FV(\varphi)$.

Note that $PT$ is not a monad in general since this substitution does not satisfy axioms.
To fix this we introduce an equivalence relation on sets $PT(V)_s$ and $Form_\mathcal{P}(V)$.
Let $\mathbb{T}$ be a partial Horn theory.
For every $t, t' \in PT(V)_s$, $t \sim t'$ if and only if $FV(t) = FV(t')$ and $\sststile{}{V} t \cong t'$ is a theorem of $\mathbb{T}$.
For every $\varphi, \psi \in Form_\mathcal{P}(V)$, $\varphi \sim \psi$ if and only if $FV(\varphi) = FV(\psi)$ and $\varphi \ssststile{}{V} \psi$ is a theorem of $\mathbb{T}$.
Then let $P(V)_s = PT(V)_s/\!\!\sim$ and $F(V) = Form_\mathcal{P}(V)/\!\!\sim$.
For every $x \in V_s$, $\eta_V(x)$ is the equivalence class of $x|_\top$.
Substitution functions respect equivalence relations, and it is easy to see that they define a structure of a monad and of a left module over it on $T$ and $F$.
For every $t,t' \in T(V)_s$, $e(s,t,t')$ is the equivalence class of $t = t'$.
For every $t \in T(V)_s$ and $\varphi \in F(V)$, let $\mu_V(t,\varphi) = t|_\varphi$.
It is easy to see that $(P,F,\mu)$ satisfies axioms of monadic presentations.
We will call it the monadic presentation of partial Horn theory $\mathbb{T}$ and denote by $P(\mathbb{T})$.

The category of partial Horn theories over $\mathcal{S}$ has tuples $(T,\mathcal{P},\mathcal{A})$ as objects,
    where $T$ is a finitary monad with a free variables structure, $\mathcal{P}$ is a set of predicate symbols and $\mathcal{A}$ is a set of axioms.
Morphisms of partial Horn theories $\mathbb{T}$ and $\mathbb{T}'$ are morphisms of their monadic presentations.
The category of partial Horn theories over $\mathcal{S}$ is denoted by $\Th^T_\mathcal{S}$.

\begin{prop}[mor-def]
Let $\mathbb{T} = (T,\mathcal{P},\mathcal{A})$ and $\mathbb{T}' = (T',\mathcal{P}',\mathcal{A}')$ be partial Horn theories,
    and let $P(\mathbb{T}) = (P,F,\mu)$ and $P(\mathbb{T}') = (P',F',\mu')$ be their monadic presentations.
To construct a morphism of these theories, it is enough to specify the following data:
\begin{itemize}
\item A morphism of monads $\alpha : T \to P'$ that preserves free variables.
\item For every $R \in \mathcal{P}$, $R : s_1 \times \ldots \times s_k$,
    a formula $\beta(R) \in F'(\{ x_1 : s_1, \ldots x_k : s_k \})$ such that $FV(\beta(R)) = \{ x_1, \ldots x_k \}$.
\end{itemize}
Then there is a morphism of left modules $f : (T,Form_\mathcal{P}) \to (T',F')$
    such that $f(\sigma(x_1, \ldots x_k)) = \alpha(\sigma)$ and $f(R(x_1, \ldots x_k)) = \beta(R)$.
If $f$ preserves axioms of $\mathbb{T}$, then it extends to a morphism of theories.
Moreover, there is at most one morphism with these properties.
\end{prop}
\begin{proof}
Morphism $f$ is already defined on terms, and we can define it on formulas as follows:
\begin{align*}
f(a = b) & = f(a) = f(b) \\
f(R(a_1, \ldots a_k)) & = \beta(R)[x_i \mapsto f(a_i)] \\
f(\varphi_1 \land \ldots \land \varphi_n) & = f(\varphi_1) \land \ldots \land f(\varphi_n)
\end{align*}
We also can define $f$ on partial terms:
\[ f(t|_\varphi) = f(t)|_{f(\varphi)} \]
It is easy to see that $f$ preserves substitution.
Thus to prove that $f$ extends to a morphism of theories, we only need to show that it preserves theorems of $\mathbb{T}$.
By assumption, it preserves axioms, thus we only need to check that application of $f$ preserves inference rules.
This is obvious for \axref{b1}-\axref{b6} and \axref{a1}.
For \axref{a2} and \axref{a3} it follows from the facts that $f(\varphi[t/x]) = f(\varphi)[f(t)/x]$ and $FV(f(\varphi)) = FV(\varphi)$.

Now, let us prove that $f$ is unique.
Let $f$ and $f'$ be morphisms of theories such that $f(t) = f'(t)$ for every $t \in T(V)_s$, and
    $f(R(x_1, \ldots x_k)) = f'(R(x_1, \ldots x_k))$ for every $R \in \mathcal{P}$.
Then we prove that $f = f'$.

Let us prove that $f(\varphi) = f'(\varphi)$ for every $\varphi \in Form_\mathcal{P}(V)$.
It is enough to prove this for atomic formulas $\varphi$.
If $\varphi$ equals to $t = t'$, then $f(\varphi)$ equals to $f(t) = f(t')$ and $f'(\varphi)$ equals to $f'(t) = f'(t')$.
We know that $\sststile{}{V} f(t) \cong f'(t)$ and $\sststile{}{V} f(t') \cong f'(t')$.
Thus by transitivity and symmetry we can conclude that $f(t) = f(t)' \sststile{}{V} f'(t) = f'(t')$.

If $\varphi = R(t_1, \ldots t_k)$, then $f(\varphi) = f(R(x_1, \ldots x_k))[x_i \mapsto f(t_i)]$
    and $f'(\varphi) = f'(R(x_1, \ldots x_k))[x_i \mapsto f'(t_i)]$.
We know that $f(R(x_1, \ldots x_k)) \sststile{}{x_1, \ldots x_k}$ \linebreak $f'(R(x_1, \ldots x_k))$.
Since $FV(f(R(x_1, \ldots x_k))) = \{ x_1, \ldots x_k \}$, by \axref{a3} we can conclude that $f(\varphi) \sststile{}{V} f'(R(x_1, \ldots x_k))[x_i \mapsto f(t_i)]$.
Since $f'(R(x_1, \ldots x_k))[x_i \mapsto f(t_i)] \sststile{}{V} f(t_i)\!\downarrow$, \rlem{cong-b} implies that
    $f'(R(x_1, \ldots x_k))[x_i \mapsto f(t_i)] \sststile{}{V} f'(\varphi)$.
By \axref{b2} we conclude that $f(\varphi) \sststile{}{V} f'(\varphi)$.
The same argument shows that $f'(\varphi) \sststile{}{V} f(\varphi)$.

Finally, it is easy to see that $f(t) = f'(t)$ for every $t \in PT(V)_s$.
Thus $f = f'$.
\end{proof}

Note that if $T$ is the standard monad $Term_\mathcal{F}$, then to define a morphism of monads $T \to T'$,
    it is enough to specify for every $\sigma \in \mathcal{F}$, $\sigma : s_1 \times \ldots \times s_k \to s$,
    a partial term $\alpha(\sigma) \in T'(\{ x_1 : s_1, \ldots x_k : s_k \})$ such that $FV(\alpha(\sigma)) = \{ x_1, \ldots x_k \}$.
Then there is a unique morphism of monads $f : T \to T'$ such that $f(\sigma(x_1, \ldots x_k)) = \alpha(\sigma)$.

Now, let us define a category $\Th_\mathcal{S}$ of standard partial Horn theories.
Its objects are tuples $((\mathcal{S},\mathcal{F},\mathcal{P}),\mathcal{A})$, where $\mathcal{F}$ is a set of function symbols,
    $\mathcal{P}$ is a set of relation symbols, and $\mathcal{A}$ is a set of axioms over $(Term_\mathcal{F},Form_\mathcal{P})$.
Morphisms of standard partial Horn theories are morphisms of corresponding partial Horn theories.
Thus $\Th_\mathcal{S}$ is (equivalent to) a full subcategory of $\Th^T_\mathcal{S}$.

\subsection{Models of partial Horn theories}

Given a monad $T : \Set^\mathcal{S} \to \Set^\mathcal{S}$, we define a category of its partial algebras.
A \emph{partial algebra} over $T$ is a pair $(A,\alpha)$, where $A$ is an $\mathcal{S}$-set and $\alpha_V : Hom_{\PSet^\mathcal{S}}(V,A) \to Hom_{\PSet^\mathcal{S}}(T(V),A)$,
    where $\PSet$ is the category of sets and partial functions between them.
This pair must satisfy the following conditions:
\begin{itemize}
\item For every partial function $f : V \to A$, $\alpha_V(f) \circ \eta_V = f$.
\item For every total function $\rho : V \to T(V')$ and every partial function $f : V' \to A$, $\alpha_V(\alpha_{V'}(f) \circ \rho) = \alpha_{V'}(f) \circ \rho^*$.
\end{itemize}
A morphism of partial algebras $(A,\alpha)$ and $(A',\alpha')$ is a total morphism $h : A \to A'$ of $\mathcal{S}$-sets
    such that, for every partial function $f : V \to A$ and every $t \in T(V)_s$, if $\alpha_V(f)(t)$ is defined,
    then $\alpha'_V(h \circ f)(t)$ is also defined and $h(\alpha_V(f)(t)) = \alpha'_V(h \circ f)(t)$.

\begin{lem}[par-alg-str]
If $Term_\mathcal{F}$ is the standard monad, then categories of partial algebras over $Term_\mathcal{F}$
    and partial structures for signature $(\mathcal{S},\mathcal{F},\varnothing)$ as defined in \cite{PHL} are isomorphic.
\end{lem}
\begin{proof}
A partial structure for signature $(\mathcal{S},\mathcal{F},\varnothing)$ is an $\mathcal{S}$-set $A$ together with a collection of partial functions
    $A(\sigma) : A_{s_1} \times \ldots \times A_{s_n} \to A_s$ for every $\sigma \in \mathcal{F}$, $\sigma : s_1 \times \ldots \times s_n \to s$.
Given such partial structure, we define a partial algebra $F(A)$ over $Term_\mathcal{F}$ as $(A,\alpha)$, where $\alpha$ is defined as follows:
\begin{align*}
\alpha_V(f)(x) & = f(x) \\
\alpha_V(f)(\sigma(t_1, \ldots t_n)) & = A(\sigma)(\alpha_V(f)(t_1), \ldots \alpha_V(f)(t_n))
\end{align*}
For every morphism $h : A \to A'$ of partial structures, let $F(h) = h$.

For every partial algebra $(A,\alpha)$, we define a partial structure $G(A,\alpha)$.
Let $G(A,\alpha) = A$ and $G(A,\alpha)(\sigma)(a_1, \ldots a_n) = \alpha_{x_1, \ldots x_n}(x_i \mapsto a_i)(\sigma(x_1, \ldots x_n))$.
For every morphism $h : (A,\alpha) \to (A',\alpha')$ of partial algebras, let $G(h) = h$.
It is easy to see that functors $F$ and $G$ determine isomorphisms of categories.
\end{proof}

If $F : \Set^\mathcal{S} \to \Set$ is a left module of formulas over $T$, then we define a category of its partial algebras.
A \emph{partial algebra} over $(T,F)$ is a partial algebra $(A,\alpha)$ over $T$ together with a function $\beta_V : Hom_{\PSet^\mathcal{S}}(V,A) \to Hom_\Set(F(V),\Omega)$,
    where $\Omega = \{ \top, \bot \}$ is the set of truth-values.
This function must satisfy the following conditions:
\begin{itemize}
\item For every total function $\rho : V \to T(V')$ and every partial function $f : V' \to A$, $\beta_V(\alpha_{V'}(f) \circ \rho) = \beta_{V'}(f) \circ \rho^\circ$.
\item For every partial function $f : V \to A$, $\beta_V(f)(\top) = \top$.
\item For every partial function $f : V \to A$, $\beta_V(f)(\varphi \land \psi) = \beta_V(f)(\varphi) \land \beta_V(f)(\psi)$,
    where $P \land Q = \top$ if and only if $P = \top$ and $Q = \top$.
\end{itemize}
A morphism of partial algebras $(A,\alpha,\beta)$ and $(A',\alpha',\beta')$ is a morphism $h$ of partial algebras $(A,\alpha)$ and $(A',\alpha')$
    such that, for every partial function $f : V \to A$ and every $\varphi \in F(V)$, if $\beta_V(f)(\varphi) = \top$, then $\beta'_V(h \circ f)(\varphi) = \top$.

We define a function $\epsilon_V : Hom_{\PSet^\mathcal{S}}(V,A) \to Hom_\Set(E(V),\Omega)$ for the left module $E$ of equality.
Let $\epsilon_V(e(s,t,t')) = \top$ if and only if $\alpha_V(f)(t)$ and $\alpha_V(f)(t')$ are defined and equal.
If $F$ is a left module of formulas with equality over $T$, then we say that a partial algebra $(A,\alpha,\beta)$ is standard
    if, for every partial function $f : V \to A$, $e_V \circ \beta_V(f) = \epsilon_V(f)$, where $e_V : E(V) \to F(V)$.

\begin{lem}[par-alg-pred]
If $Term_\mathcal{F}$ is the standard monad and $Form_\mathcal{P}$ is the left module of Horn formulas,
    then categories of partial algebras over $(Term_\mathcal{F},Form_\mathcal{P})$ and partial structures for signature $(\mathcal{S},\mathcal{F},\mathcal{P})$ are isomorphic.
\end{lem}
\begin{proof}
A partial structure for signature $(\mathcal{S},\mathcal{F},\mathcal{P})$ is a partial structure $A$ for signature $(\mathcal{S},\mathcal{F},\varnothing)$
    together with a relation $A(R) \subseteq A_{s_1} \times \ldots \times A_{s_n}$ for every $R \in \mathcal{P}$, $R : s_1 \times \ldots \times s_n$.
Given such partial structure, we define a partial algebra $F(A)$ over $(Term_\mathcal{F},Form_\mathcal{P})$ as $(A,\alpha,\beta)$,
    where $(A,\alpha)$ is the partial algebra defined in \rlem{par-alg-str}, and $\beta$ defined as follows:
\begin{align*}
\beta_V(f)(t =_s t') & = \epsilon_V(e(s,t,t')) \\
\beta_V(f)(R(t_1, \ldots t_n)) & = \top \text{ if and only if } (\alpha_V(f)(t_1), \ldots \alpha_V(f)(t_n)) \in A(R) \\
\beta_V(f)(\varphi_1 \land \ldots \land \varphi_n) & = \beta_V(f)(\varphi_1) \land \ldots \land \beta_V(f)(\varphi_n)
\end{align*}
For every morphism $h : A \to A'$ of partial structures, let $F(h) = h$.

For every partial algebra $(A,\alpha,\beta)$, we define a partial structure $G(A,\alpha,\beta)$.
We already defined interpretation of function symbols in \rlem{par-alg-str}.
For every $R \in \mathcal{P}$, let $G(A,\alpha,\beta)(R) = \{ (a_1, \ldots a_n)\ |\ \beta_{x_1, \ldots x_n}(x_i \mapsto a_i)(R(x_1, \ldots x_n)) = \top \}$.
For every morphism $h : (A,\alpha,\beta) \to (A',\alpha',\beta')$ of partial algebras, let $G(h) = h$.
It is easy to see that functors $F$ and $G$ determine isomorphisms of categories.
\end{proof}

If $(T,F,\mu)$ is a monadic presentation, then we define a category of its partial algebras as a full subcategory of partial algebras over $(T,F)$.
A partial algebra $(A,\alpha,\beta)$ over $(T,F)$ is a partial algebra over $(T,F,\mu)$ if, for every partial function $f : V \to A$,
    every $t \in T(V)_s$ and every $\varphi \in F(V)$, $\alpha_V(f)(\mu_V(t,\varphi))$ is defined if and only if $\alpha_V(f)(t)$ is defined and $\beta_V(f)(\varphi) = \top$,
    and $\alpha_V(f)(\mu_V(t,\varphi))$ equals to $\alpha_V(f)(t)$ when it is defined.
The category of partial algebras over $(T,F,\mu)$ will be denoted by $\PAlg{(T,F,\mu)}$.

\begin{lem}
If $Term_\mathcal{F}$ is the standard monad and $\mathbb{T} = (Term_\mathcal{F},\mathcal{P},\mathcal{A})$ is a partial Horn theory,
    then categories of partial algebras over $P(\mathbb{T})$ and models of $\mathbb{T}$ as defined in \cite{PHL} are isomorphic.
\end{lem}
\begin{proof}
Using \rlem{par-alg-pred}, models of $\mathbb{T}$ can be described as partial algebras $(A,\alpha',\beta')$ over $(Term_\mathcal{F},Form_\mathcal{P})$
    such that, for every derivable sequent $\varphi \ssststile{}{V} \psi$ of $\mathbb{T}$ and every partial function $f : V \to A$, $\beta'_V(f)(\varphi) = \beta'_V(f)(\psi)$.

Let $(A,\alpha,\beta)$ be a partial algebra over $P(\mathbb{T})$.
Then we define a partial algebra $F(A,\alpha,\beta)$ over $(Term_\mathcal{F},Form_\mathcal{P})$.
Let $F(A,\alpha,\beta) = (A,\alpha',\beta')$, where $\alpha'_V(f)(t) = \alpha_V(f)([t|_\top]_\sim)$ and $\beta'_V(f)(\varphi) = \alpha_V(f)([\varphi]_\sim)$,
    where $[t|_\top]_\sim$ and $[\varphi]_\sim$ are equivalence classes of $t_\top$ and $\varphi$ in $P(V)$ and $F(V)$ respectively.
Then $F(A,\alpha,\beta)$ is a model of $\mathbb{T}$.
Indeed, if $\varphi \ssststile{}{V} \psi$ is a theorem of $\mathbb{T}$, then $\varphi' \ssststile{}{V} \psi'$
    is also a theorem of $\mathbb{T}$, where $\varphi' = \varphi \land x_1 \land \ldots \land x_n$, $\psi' = \psi \land y_1 \land \ldots \land y_k$,
    $x_1, \ldots x_n$ is the set of free variables of $\psi$, and $y_1, \ldots y_k$ is the set of free variables of $\varphi$.
It follows that $[\varphi']_\sim = [\psi']_\sim$; hence $\beta'_V(f)(\varphi') = \beta'_V(f)(\varphi')$.
But $\beta'_V(f)(\varphi) = \beta'_V(f)(\varphi')$ and $\beta'_V(f)(\psi) = \beta'_V(f)(\psi')$; hence $F(A,\alpha,\beta)$ is a model of $\mathbb{T}$.
If $h$ is a morphism of partial algebras over $P(\mathbb{T})$, then let $F(h) = h$.

Let $(A,\alpha',\beta')$ be a model of $\mathbb{T}$.
Then we define a partial algebra $G(A,\alpha',\beta')$ over $P(\mathbb{T})$.
Let $G(A,\alpha',\beta') = (A,\alpha,\beta)$, where $\beta_V(f)([\varphi]_\sim) = \beta'_V(f)(\varphi)$, and $\alpha_V(f)([t|_\varphi]_\sim)$ is defined
    if and only if $\alpha'_V(f)(t)$ is defined and $\beta'_V(f)(\varphi) = \top$, and in this case $\alpha_V(f)([t|_\varphi]_\sim) = \alpha'_V(f)(t)$.
These definitions do not depend on the choice of a representative of the equivalence classes.
Indeed, if $\varphi \sim \psi$, then $\varphi \ssststile{}{V} \psi$ is a theorem of $\mathbb{T}$,
    and in this case $\beta'_V(f)(\varphi) = \beta'_V(f)(\psi)$ since $A$ is a model of $\mathbb{T}$.
The same argument shows that the definition of $\alpha$ does not depend on the choice of a representative of $[t|_\varphi]_\sim$.
If $h$ is a morphism of models, then let $G(h) = h$.
It is easy to see that functors $F$ and $G$ determine isomorphisms of categories.
\end{proof}

Finally, we prove a proposition which shows that if $\mathbb{T}'$ is a partial Horn theory under $\mathbb{T}$,
    then we can think of models of $\mathbb{T}'$ as models of $\mathbb{T}$ with additional structure.

\begin{prop}[func-mod]
For every morphism of monadic presentations $f : (P,F,\mu) \to (P',F',\mu')$, there is a faithful functor $f^* : \PAlg{(P',F',\mu')} \to \PAlg{(P,F,\mu)}$
    such that $id_{(P,F,\mu)}^*$ is the identity functor and $(g \circ f)^* = f^* \circ g^*$.
\end{prop}
\begin{proof}
If $(A,\alpha,\beta)$ is a partial algebra over $(P',F',\mu')$, then let $f^*(A,\alpha,\beta) = (A, e \mapsto \alpha_V(e) \circ f_V, e \mapsto \beta_V(e) \circ f_V)$.
If $h : (A,\alpha,\beta) \to (A',\alpha',\beta')$ is a morphism of partial algebras, then let $f^*(h) = h$.
It is easy to see that these definitions satisfy all required conditions.
\end{proof}

\subsection{Properties of the category of theories}
\label{sec:prop}

Now we prove a few properties of the category of theories.
We begin with a proof of the existence of colimits.

\begin{prop}[th-cocomplete]
Category $\Th_\mathcal{S}$ is cocomplete.
\end{prop}
\begin{proof}
First, let $\{ \mathbb{T}_i \}_{i \in S} = \{ ((\mathcal{S},\mathcal{F}_i,\mathcal{P}_i),\mathcal{A}_i) \}_{i \in S}$ be a set of theories.
Then we can define its coproduct $\coprod\limits_{i \in S} \mathbb{T}_i$ as the theory with $\coprod\limits_{i \in S} \mathcal{F}_i$ as the set of function symbols and $\coprod\limits_{i \in S} \mathcal{A}_i$ as the set of axioms.
Morphisms $f_i : \mathbb{T}_i \to \coprod\limits_{i \in S} \mathbb{T}_i$ are defined in the obvious way.
If $g_i : \mathbb{T}_i \to X$ is a collection of morphisms, then \rprop{mor-def} implies that there is a unique morphism $g : \coprod\limits_{i \in S} \mathbb{T}_i \to X$
    satisfying $g(\sigma(x_1, \ldots x_n)) = g_i(\sigma(x_1, \ldots x_n))$ and $f(R(x_1, \ldots x_n)) = f_i(R(x_1, \ldots x_n))$
    for every $\sigma \in \mathcal{F}_i$ and $R \in \mathcal{P}_i$.

Now, let $f,g : \mathbb{T}_1 \to \mathbb{T}_2$ be a pair of morphisms of theories.
Then we can define their coequalizer $\mathbb{T}$ as the theory with the same set of function and predicate symbols as $\mathbb{T}_2$ and the set of axioms which consists of the axioms of $\mathbb{T}_2$
together with $\sststile{}{x_1, \ldots x_n} f(\sigma(x_1, \ldots x_n)) \cong g(\sigma(x_1, \ldots x_n))$ for each function symbols $\sigma$ of $\mathbb{T}_1$
and $f(R(x_1, \ldots x_n)) \ssststile{}{x_1, \ldots x_n} g(R(x_1, \ldots x_n))$ for each predicate symbols $R$ of $\mathbb{T}_1$.
Then we can define $e : \mathbb{T}_2 \to \mathbb{T}$ as identity function on terms and formulas.
By \rprop{mor-def}, $e \circ f = e \circ g$.
If $h : \mathbb{T}_2 \to X$ is such that $h \circ f = h \circ g$, then it extends to a morphism $\mathbb{T} \to X$ since additional axioms are preserved by the assumption on $h$.
This extension is unique since $e$ is an epimorphism.
\end{proof}

Now we give a characterization of monomorphisms.

\begin{prop}[mono]
A morphism of theories $f : \mathbb{T}_1 \to \mathbb{T}_2$ is a monomorphism if and only if, for every sequent $\varphi \sststile{}{V} \psi$ of $\mathbb{T}_1$,
if $f(\varphi) \sststile{}{V} f(\psi)$ is a theorem of $\mathbb{T}_2$, then $\varphi \sststile{}{V} \psi$ is a theorem of $\mathbb{T}_1$.
\end{prop}
\begin{proof}
First, let us prove the ``if'' part.
Let $g,h : \mathbb{T} \to \mathbb{T}_1$ be a pair of morphisms such that $f \circ g = f \circ h$.
If $t \in PTerm_\Sigma(V)_s$, then $\sststile{}{V} f(g(t)) \cong f(h(t))$; hence $\sststile{}{V} g(t) \cong h(t)$.
If $\varphi \in Form_\mathcal{P}(V)$, then $f(g(\varphi)) \ssststile{}{V} f(h(\varphi))$; hence $g(\varphi) \ssststile{}{V} h(\varphi)$.
Thus $g = h$.

Now, let us prove the ``only if'' part.
Suppose that $f$ is a monomorphism.
Let $\varphi \sststile{}{V} \psi$ be a sequent of $\mathbb{T}_1$ such that $f(\varphi) \sststile{}{V} f(\psi)$ is a theorem of $\mathbb{T}_2$.
Let $\mathbb{T}$ be a theory which consists of a single predicate symbol $R : s_1 \times \ldots \times s_n \times s'_1 \times \ldots \times s'_k$
where $s_1, \ldots s_n$ are sorts of variables in $FV(\varphi)$ and $s'_1, \ldots s'_k$ are sorts of variables in $FV(\psi)$.
Let $g : \mathbb{T} \to \mathbb{T}_1$ be a morphism defined by $g(R(x_1, \ldots x_n, y_1, \ldots y_k)) = \varphi \land y_1\!\downarrow \land \ldots \land y_k\!\downarrow$ and
let $h : \mathbb{T} \to \mathbb{T}_1$ be a morphism defined by $h(R(x_1, \ldots x_n, y_1, \ldots y_k)) = \varphi \land \psi$.
By \rprop{mor-def}, $f \circ g = f \circ h$, hence $g = h$ which implies that $\varphi \sststile{}{V} \psi$.
\end{proof}

Let $\mathbb{T} = ((\mathcal{S},\mathcal{F},\mathcal{P}),\mathcal{A})$ and $\mathbb{T}' = ((\mathcal{S}',\mathcal{F}',\mathcal{P}'),\mathcal{A}')$ be a pair of theories.
Then we say that $\mathbb{T}'$ is a \emph{subtheory} of $\mathbb{T}$ if $\mathcal{S}' \subseteq \mathcal{S}$, $\mathcal{F}' \subseteq \mathcal{F}$, $\mathcal{P}' \subseteq \mathcal{P}$ and $\mathcal{A}' \subseteq \mathcal{A}$.
If $\mathbb{T}'$ is a subtheory of a theory $\mathbb{T}$, then we often need to know when a theorem of $\mathbb{T}$ is a theorem of $\mathbb{T}'$.
The lemma below gives us a simple criterion for this.
First, we need to introduce a bit of notation.
Let $t$ is a term over the signature of $\mathbb{T}$ such that there is no subterm of a sort that does not belong to $\mathcal{S}'$.
Then we define a term $Ret(t)$ over the signature of $\mathbb{T}'$ as follows:
\begin{align*}
Ret(x) & = x \\
Ret(\sigma(t_1, \ldots t_n)) & = \sigma(Ret(t_1), \ldots Ret(t_n)) \text{, if $\sigma \in \mathcal{F}'$} \\
Ret(\sigma(t_1, \ldots t_n)) & = x_s \text{, if $\sigma \notin \mathcal{F}'$ and $\sigma : s_1 \times \ldots \times s_n \to s$}
\end{align*}
where $x_s$ is a variable of sort $s$ that is not a free variable of $t$.

If $\varphi$ is an atomic formula over the signature of $\mathbb{T}$, then we define a formula $Ret(\varphi)$ over the signature of $\mathbb{T}'$ as follows:
\begin{align*}
Ret(t = t') & = (Ret(t) = Ret(t')) \text{, if $Ret(t)$ and $Ret(t')$ are defined} \\
Ret(R(t_1, \ldots t_n)) & = R(Ret(t_1), \ldots Ret(t_n)) \text{, if $Ret(t_i)$ is defined for every $i$} \\
Ret(\varphi) & = \top \text{, otherwise}
\end{align*}
For an arbitrary Horn formula $\varphi$ we define $Ret(\varphi)$ as follows:
\[ Ret(\varphi_1 \land \ldots \land \varphi_n) = Ret(\varphi_1) \land \ldots \land Ret(\varphi_n) \]
For every partial term $t|_\varphi$, let $Ret(t|_\varphi) = Ret(t)|_{Ret(\varphi)}$.
If $S$ is sequent $\varphi \sststile{}{V} \psi$ in the signature of $\mathbb{T}$,
    then we define sequent $Ret(S)$ in the signature of $\mathbb{T}'$ as $Ret(\varphi) \sststile{}{V \cup FV(Ret(\varphi)) \cup FV(Ret(\psi))} Ret(\psi)$.

\begin{lem}[subtheory]
Let $\mathbb{T}'$ be a subtheory of $\mathbb{T}$.
Suppose that, for every axiom $S$ of $\mathbb{T}$, $Ret(S)$ is a theorem of $\mathbb{T}'$.
Then if a sequent in the signature of $\mathbb{T}'$ is provable in $\mathbb{T}$, then it is also provable in $\mathbb{T}'$.
\end{lem}
\begin{proof}
If $S$ is a sequent in the signature of $\mathbb{T}'$, then $Ret(S) = S$.
Thus we only need to prove that if $S$ is a theorem of $\mathbb{T}$, then $Ret(S)$ is a theorem of $\mathbb{T}'$.
For axioms this is true by assumption.
We need to check that $Ret(-)$ preserves inference rules.
This is clearly true for rules \axref{b1}-\axref{b6} and \axref{a1}.

Let us consider rule \axref{a2}.
Let $S$ equals $x = y \land \varphi \sststile{}{x:s,y:s,V} \varphi[y/x]$.
Note that $Ret(\varphi[y/x])$ is defined if and only if $Ret(\varphi)$ is defined, and in this case $Ret(\varphi[y/x]) = Ret(\varphi)[y/x]$.
Thus $Ret(S)$ is either of the form $x = y \land Ret(\varphi) \sststile{}{x:s,y:s,V,FV(Ret(\varphi))} Ret(\varphi)[y/x]$,
or of the form $x = y \sststile{}{x:s,y:s,V} \top$, or of the form $\top \sststile{}{x:s,y:s,V} \top$.
In all of these cases $Ret(S)$ is a theorem of $\mathbb{T}'$.

Finally, let us consider rule \axref{a3}.
To prove that it preserves the required property, it is enough to show that $\varphi$ is a formula of $(\mathcal{S}',\mathcal{F}',\mathcal{P}')$ if and only if $\varphi[t/x]$ is.
If $x \notin FV(\varphi)$, then $\varphi = \varphi[t/x]$.
Suppose that $x \in FV(\varphi)$ and $\varphi$ is a formula of $(\mathcal{S}',\mathcal{F}',\mathcal{P}')$.
If $x$ has sort $s$, then $s \in \mathcal{S}'$.
We need to show that a term of sort $s$ is a term of $(\mathcal{S}',\mathcal{F}',\mathcal{P}')$.
But this follows from the assumption on the set of function symbols.
\end{proof}

Sometimes it is convenient to have a sort which consists of a single element.
Let $\mathcal{S}$ be a set of sorts and let $s_0$ be a sort in $\mathcal{S}$.
Then we define a theory $\mathbb{T}_{s_0}$ which consists of a single function symbol $\emptyCtx : s_0$
    and two axioms: $\sststile{}{} \emptyCtx\!\downarrow$ and $\sststile{}{x} x = \emptyCtx$.
Then, for every theory $\mathbb{T} \in \Th_\mathcal{S}$, there is at most one morphism from $\mathbb{T}_{s_0}$ to $\mathbb{T}$.
If such morphism exists, we will say that $s_0$ is \emph{trivial} in $\mathbb{T}$.
Thus $\mathbb{T}_{s_0}/\Th_\mathcal{S}$ is (equivalent to) a full subcategory of $\Th_\mathcal{S}$.

As an application of the previous results we will prove that adding a trivial sort does not change the category of theories.
Every theory $\mathbb{T} \in \Th_\mathcal{S}$ is naturally a theory in $\Th_{\mathcal{S} \amalg \{ s_0 \}}$.
Thus we have a functor $i : \Th_\mathcal{S} \to \mathbb{T}_{s_0}/\Th_{\mathcal{S} \amalg \{ s_0 \}}$ such that $i(\mathbb{T}) = \mathbb{T} \amalg \mathbb{T}_{s_0}$.
\begin{prop}[triv-sort]
Functor $i : \Th_\mathcal{S} \to \mathbb{T}_{s_0}/\Th_{\mathcal{S} \amalg \{ s_0 \}}$ is an equivalence of categories.
\end{prop}
\begin{proof}
Let $\mathbb{T}_1,\mathbb{T}_2 \in \Th_\mathcal{S}$ be theories with $P(\mathbb{T}_i) = (T_i,F_i,\mu_i)$, $i = 1,2$.
Let $\alpha,\beta : \mathbb{T}_1 \to \mathbb{T}_2$ be morphisms such that $i(\alpha) = i(\beta)$.
Then, for every $t \in T_1$, sequent $\sststile{}{V} i(\alpha)(t) \cong i(\beta)(t)$ is a theorem of $i(\mathbb{T}_2)$.
Since $\mathbb{T}_2$ is (isomorphic to) a subtheory of $i(\mathbb{T}_2)$, by \rlem{subtheory}, sequent $\sststile{}{V} \alpha(t) \cong \beta(t)$ is a theorem of $\mathbb{T}_2$.
Analogously, we can show that $\alpha(\varphi) \ssststile{}{V} \beta(\varphi)$ is a theorem of $\mathbb{T}_2$ for every $\varphi \in F_1$.
Thus $i$ is faithful.

Let $\alpha : i(\mathbb{T}_1) \to i(\mathbb{T}_2)$ be a morphism.
For every $t \in T_1(V)_s$, let $\beta(t) = Ret(\alpha(t))$ and, for every $\varphi \in F_1(V)$, let $\beta(\varphi) = Ret(\alpha(\varphi))$.
Since $Ret$ preserves substitution, $\land$ and $\top$, this defines a morphism $\beta : \mathbb{T}_1 \to \mathbb{T}_2$.
Since $s_0$ is trivial in $i(\mathbb{T}_2)$, $Ret(t) = t$ and $Ret(\varphi) = \varphi$ for every partial term $t$ and every formula $\varphi$.
Thus $i(\beta) = \alpha$; hence $i$ is full.

Let $\mathbb{T} \in \Th_{\mathcal{S} \amalg \{ s_0 \}}$ be a theory with trivial $s_0$.
Then we define a theory $\mathbb{T}' \in \Th_\mathcal{S}$.
It has a predicate symbol $R : s'_1 \times \ldots \times s'_n$ for every predicate symbol $R : s_1 \times \ldots \times s_n$ of $\mathbb{T}$,
    where $s'_1, \ldots s'_n$ is the subsequence of $s_1, \ldots s_n$ consisting of sorts from $\mathcal{S}$.
It has a function symbol $\sigma : s'_1 \times \ldots \times s'_n \to s$ for every function symbol
    $\sigma : s_1 \times \ldots \times s_n \to s$ of $\mathbb{T}$ such that $s \in \mathcal{S}$.
Also, for every function symbol $\sigma : s_1 \times \ldots \times s_n \to s_0$ of $\mathbb{T}$,
    there is a predicate symbol $R_\sigma : s_1' \times \ldots \times s'_n$ in $\mathbb{T}'$.

For every term $t$ of $\mathbb{T}$ of a sort from $\mathcal{S}$, we can define a term $r(t)$ of $\mathbb{T}'$.
Term $r(t)$ is obtained from $t$ by omitting subterms of sort $s_0$.
For every formula $\varphi$ of $\mathbb{T}$, we can define a formula $r(\varphi)$ of $\mathbb{T}'$:
\begin{align*}
r(t =_{s_0} t') & = \top \\
r(t =_s t') & = (r(t) =_s r(t')) \\
r(R(t_1, \ldots t_n)) & = R(r(t'_1), \ldots r(t'_n)) \\
r(\varphi_1 \land \ldots \land \varphi_n) & = r(\varphi_1) \land \ldots \land r(\varphi_n)
\end{align*}
where $t'_1, \ldots t'_n$ is the subsequence of $t_1, \ldots t_n$ consisting of the terms of sorts from $\mathcal{S}$.
Axioms of $\mathbb{T}'$ are sequents of the form $r(\varphi) \sststile{}{FV(r(\varphi)) \cup FV(r(\psi))} r(\psi)$ for every axiom $\varphi \sststile{}{V} \psi$ of $\mathbb{T}$.
It is easy to see that $i(\mathbb{T}')$ is isomorphic to $\mathbb{T}$.
Thus $i$ is essentially surjective on objects.
\end{proof}

\section{Theory of substitutions}
\label{sec:T1}

In this section we define an example of partial Horn theories $\substTh$, which we call the theory of substitutions.
We also prove that the category of models of this theory is equivalent to the category of contextual categories
We will use this theory later to define algebraic dependent type theories.

\subsection{Definition of $\substTh$}
\label{sec:T1-def}

Let $\mathcal{C} = \{ ctx, tm \} \times \mathbb{N}$ be the set of sorts.
We will write $(ty,n)$ for $(ctx,n+1)$.
Sort $(tm,n)$ represents terms in contexts of length $n$, sort $(ctx,n)$ represents contexts of length $n$, and sort $(ty,n)$ represents types in contexts of length $n$.

There are two ways to define substitution: either to substitute the whole context (full substitution) or only a part of it (partial substitution).
Using ordinary type theoretic syntax the full substitution can be described by the following inference rule:
\begin{center}
\AxiomC{$A_1, \ldots A_n \vdash A\ type$}
\AxiomC{$\Gamma \vdash a_1 : A_1[]$ \quad \ldots \quad $\Gamma \vdash a_n : A_n[a_1, \ldots a_{n-1}]$}
\BinaryInfC{$\Gamma \vdash A[a_1, \ldots a_n]\ type$}
\DisplayProof
\end{center}
\medskip
The partial substitution is described by the following inference rule:
\begin{center}
\AxiomC{$\Gamma, A_1, \ldots A_n \vdash A\ type$}
\AxiomC{$\Gamma \vdash a_1 : A_1$ \quad \ldots \quad $\Gamma \vdash a_n : A_n[a_1, \ldots a_{n-1}]$}
\BinaryInfC{$\Gamma \vdash A[a_1, \ldots a_n]\ type$}
\DisplayProof
\end{center}
\medskip
The partial substitution was used in \cite{b-systems}, but we will use the full version since it is stronger.
To make these operations equivalent, we need to add another operation to the partial substitution, and even more axioms.
Thus our approach seems to be somewhat more convenient.

The set of function symbols of $\substTh$ consists of the following symbols:
\begin{align*}
ft_n          & : (ty,n) \to (ctx,n) \\
ty_n          & : (tm,n) \to (ty,n) \\
v_{n,i}       & : (ctx,n) \to (tm,n) \text{, } 0 \leq i < n \\
subst_{p,n,k} & : (ctx,n) \times (p,k) \times (tm,n)^k \to (p,n) \text{, } p \in \{ tm, ty \}
\end{align*}

Let $ft^i_n : (ctx,n+i) \to (ctx,n)$ and $ctx_{p,n} : (p,n) \to (ctx,n)$ be the following derived operations:
\begin{align*}
ft^0_n(A) & = A \\
ft^{i+1}_n(A) & = ft^i_n(ft_{n+i}(A)) \\
ctx_{ty,n}(t) & = ft_n(t) \\
ctx_{tm,n}(t) & = ft_n(ty_n(t))
\end{align*}

Auxiliary predicates $Hom_{n,k} : (ctx,n) \times (ctx,k) \times (tm,n)^k$ are defined as follows: $Hom_{n,k}(B, A, a_1, \ldots a_k)$ holds if and only if
\[ ty_n(a_i) = subst_{ty,n,i-1}(B, ft^{k-i}_i(A), a_1, \ldots a_{i-1}) \text{ for each } 1 \leq i \leq k \]
The idea is that a tuple of terms should represent a morphism in a contextual category.
So $Hom_{n,k}(B, A, a_1, \ldots a_k)$ holds if and only if $(a_1, \ldots a_k)$ is a morphism with domain $A$ and codomain $B$.
Note that if $Hom_{n,k}(B, A, a_1, \ldots a_k)$, then $ft_n(ty_n(a_i)) = B$.

The set of axioms of $\substTh$ consists of the axioms asserting that $(ctx,0)$ is trivial and the axioms we list below.
The following axioms describe when functions are defined:
\begin{align}
\label{ax:def-var}
                                             & \sststile{}{A}           v_{n,i}(A) \downarrow \\
\label{ax:def-subst}
Hom_{n,k}(B, ctx_{p,k}(a), a_1, \ldots a_k)  & \ssststile{}{B, a, a_i}  subst_{p,n,k}(B, a, a_1, \ldots a_k) \downarrow
\end{align}

The following axioms describe the ``typing'' of the constructions we have:
\begin{align}
\label{ax:type-var}
& \sststile{}{A}         ty_n(v_{n,i}(A)) = subst_{ty,n,n-i-1}(A, ft^i_{n-i}(A), v_{n,n-1}(A), \ldots v_{n,i+1}(A)) \\
\label{ax:type-subst-ty}
& \sststile{}{B, A, a_i} ft_n(subst_{ty,n,k}(B, A, a_1, \ldots a_k)) \leftrightharpoons B \\
\label{ax:type-subst-tm}
& \sststile{}{B, a, a_i} ty_n(subst_{tm,n,k}(B, a, a_1, \ldots a_k)) \leftrightharpoons subst_{ty,n,k}(B, ty_k(a), a_1, \ldots a_k)
\end{align}

The following axioms prescribe how $subst_{p,n,k}$ must be defined on indices ($v_{n,i}$):
\begin{align}
\label{ax:subst-var}
& \sststile{}{a}         subst_{p,n,n}(ctx_{p,n}(a), a, v_{n,n-1}(ctx_{p,n}(a)), \ldots v_{n,0}(ctx_{p,n}(a))) = a \\
\label{ax:var-subst}
& Hom_{n,k}(B, A, a_1, \ldots a_k) \sststile{}{B, a_i, A} subst_{tm,n,k}(B, v_{k,i}(A), a_1, \ldots a_k) = a_{k-i}
\end{align}

The last axiom say that substitution must be ``associative'':
\begin{align}
\label{ax:subst-subst}
& Hom_{n,k}(C, B, b_1, \ldots b_k) \land Hom_{k,m}(B, ctx_{p,m}(a), a_1, \ldots a_m) \sststile{}{C, b_i, B, a_i, a} \\ \notag
& subst_{p,n,k}(C, subst_{p,k,m}(B, a, a_1, \ldots a_m), b_1, \ldots b_k) = \\ \notag
& subst_{p,n,m}(C, a, subst_{tm,n,k}(C, a_1, b_1, \ldots b_k), \ldots subst_{tm,n,k}(C, a_m, b_1, \ldots b_k))
\end{align}

\subsection{Models of $\substTh$}

Here we show that the category of models of $\substTh$ is equivalent to the category of contextual categories.
First, we construct a functor $F : \Mod{\substTh} \to \ccat$.
Let $M$ be a model of $\substTh$.
Then the set of objects of level $n$ of $F(M)$ is $M(ctx,n)$.
For each $A \in M(ctx,n)$, $B \in M(ctx,k)$ morphisms from $A$ to $B$ are tuples $(a_1, \ldots a_k)$ such that $a_i \in M(tm,n)$ and $Hom_{n,k}(A, B, a_1, \ldots a_k)$.

For each $0 \leq i \leq n$ axiom~\eqref{ax:type-var} implies
\[ \sststile{}{A} Hom_{n,n-i}(A, ft^i_{n-i}(A), v_{n,n-1}(A), \ldots v_{n,i}(A)). \]
For each $A \in M(ctx,n)$ we define $id_A : A \to A$ as tuple
\[ (v_{n,n-1}(A), \ldots v_{n,0}(A)) \]
and $p_A : A \to ft(A)$ as tuple
\[ (v_{n,n-1}(A), \ldots v_{n,1}(A)). \]

Now, we introduce some notation.
If $B \in M(ctx,n)$, $a \in M(p,k)$, and $f = (a_1, \ldots a_k) : B \to ctx_{p,k}(a)$ is a morphism, then we define $a[f] \in M(p,n)$ as $subst_{p,n,k}(B, a, a_1, \ldots a_k)$.
By axiom \eqref{ax:def-subst} this construction is total.

If $A \in M(ctx,n)$, $B \in M(ctx,k)$, $C \in M(ctx,m)$, $f : A \to B$, and $(c_1, \ldots c_m) : B \to C$,
    then we define composition $(c_1, \ldots c_m) \circ f$ as $(c_1[f], \ldots c_m[f])$.
The following sequence of equations shows that $(c_1, \ldots c_m) \circ f : A \to C$.
\begin{align*}
ty_n(c_i[f]) & = \text{(by axiom~\eqref{ax:type-subst-tm})} \\
ty_k(c_i)[f] & = \text{(since $Hom_{k,m}(c_1, \ldots c_m)$)} \\
ft^{m-i}_i(C)[c_1, \ldots c_{i-1}][f] & = \text{(by axiom~\eqref{ax:subst-subst})} \\
ft^{m-i}_i(C)[c_1[f], \ldots c_{i-1}[f]] &
\end{align*}

With these notations we can rewrite axioms \eqref{ax:type-subst-tm}, \eqref{ax:subst-var} and \eqref{ax:subst-subst} as follows:
\begin{align*}
ty_n(a[f]) & = A[f] \\
\text{ for each } f : B \to ft_k(A) & \text{, where } A = ty_k(a) \\
a[id_{ctx_{p,n}(a)}] & = a \\
a[g][f] & = a[g \circ f] \\
\text{ for each } f : C \to B \text{ and } & g : B \to ctx_{p,m}(a)
\end{align*}

Associativity of the composition follows from axiom~\eqref{ax:subst-subst}, and the fact that $id$ is identity for it follows from axioms \eqref{ax:subst-var} and \eqref{ax:var-subst}.

For every $A \in M(ty,k)$ there is a bijection $\varphi$ between the set of $a \in M(tm,k)$ such that $ty_k(a) = A$
    and the set of morphisms $f : ft_k(A) \to A$ such that $p_A \circ f = id_{ft_k(A)}$.
For every such $a \in M(tm,k)$ we define $\varphi(a)$ as
\[ (v_{k,k-1}(ft_k(A)), \ldots v_{k,0}(ft_k(A)), a). \]
Note that if $(a_1, \ldots a_{k+1}) : B \to A$ is a morphism, then axiom~\eqref{ax:var-subst} implies that $p_A \circ (a_1, \ldots a_{k+1})$ equals to $(a_1, \ldots a_k)$.
Thus $\varphi(a)$ is a section of $p_A$.
Clearly, $\varphi$ is injective.
Let $f : ft_k(A) \to A$ be a section of $p_A$; then first $k$ components of $f$ must be identity on $ft_k(A)$.
So if $a$ is the last component of $f$, then $\varphi(a)$ equals to $f$.
Hence $\varphi$ is bijective.

If $A \in M(ty,k)$, $B \in M(ctx,n)$, and $f = (a_1, \ldots a_k) : B \to ft_k(A)$, then we define $f^*(A)$ as $A[f] = subst_{ty,n,k}(B, A, a_1, \ldots a_k)$.
Map $q(f,B)$ defined as the tuple with $i$-th component equals to
\[ \left\{
  \begin{array}{lr}
    a_i[v_{n+1,n}(A[f]), \ldots v_{n+1,1}(A[f])] & \text{ if } 1 \leq i \leq k \\
    v_{n+1,0}(A[f])                              & \text{ if } i = k+1
  \end{array}
\right. \]
Now we have the following commutative square:
\[ \xymatrix{ A[f] \ar[r]^-{q(f,A)} \ar[d]_{p_{A[f]}} & A \ar[d]^{p_A} \\
              B \ar[r]_-f                             & ft_k(A)
            } \]
We need to prove that this square is Cartesian.
By proposition~2.3 of \cite{c-systems} it is enough to construct a section $s_{f'} : B \to A[f]$ of $p_{A[f]}$
    for each $f' = (a_1, \ldots a_k, a_{k+1}) : B \to A$ and prove a few properties of $s_{f'}$.
We define $s_{f'}$ to be equal to $\varphi(a_{k+1})$.
Axioms \eqref{ax:var-subst} and \eqref{ax:subst-subst} implies that $q(f,B) \circ s_{f'} = f$.
To complete the proof that the square above is Cartesian we need, for every $g : ft_k(A) \to ft_m(C)$ and $A = C[g]$, prove that $s_{f'} = s_{q(g,C) \circ f'}$.
The last component of $q(g,C) \circ f'$ equals to $v_{n+1,0}(C[g])[f'] = a_{k+1}$.
Thus the last components of $q(g,C) \circ f'$ and $f'$ coincide, hence $s_{f'} = s_{q(g,C) \circ f'}$.

We are left to prove that operations $A[f]$ and $q(f,A)$ are functorial.
Equations $A[id_{ft_k(A)}] = A$ and $A[f \circ g] = A[f][g]$ are precisely axioms \eqref{ax:subst-var} and \eqref{ax:subst-subst}.
The fact that $q(id_{ft_k(A)}, A) = id_A$ follows from axiom~\ref{ax:var-subst}.
Now let $g : C \to B$ and $f : B \to ft_k(A)$ be morphisms; we need to show that $q(f \circ g, A) = q(f,A) \circ q(g,A[f])$.
The last component of $q(f,A) \circ q(g,A[f])$ equals to $v_{n+1,0}(A[f])[q(g,A[f])] = v_{m+1,0}(A[f][g])$,
    which equals to the last component of $q(f \circ g, A)$, namely $v_{m+1,0}(A[f \circ g])$.
If $1 \leq i \leq k$, then $i$-th component of $q(f,A) \circ q(g,A[f])$ equals to
\[ a_i[v_{n+1,n}(A[f]), \ldots v_{n+1,1}(A[f])][q(g,A[f])] = a_i[b_1', \ldots b_n'] \]
where $a_i$ is $i$-th component of $f$, $b_i$ is $i$-th component of $g$, and $b_i'$ equals to $b_i[v_{m+1,m}(A[f][g]), \ldots v_{m+1,1}(A[f][g])]$.
$i$-th component of $q(f \circ g, A)$ equals to
\[ a_i[g][v_{m+1,m}(A[f \circ g]), \ldots v_{m+1,1}(A[f \circ g])] = a_i[b_1'', \ldots b_n''], \]
where $b_i'' = b_i[v_{m+1,m}(A[f \circ g]), \ldots v_{m+1,1}(A[f \circ g])]$.
Thus $q(f \circ g, A) = q(f,A) \circ q(g,A[f])$.
This completes the construction of contextual category $F(M)$.

\begin{prop}[T1-CCat]
$F$ is functorial, and functor $F : \Mod{\substTh} \to \ccat$ is an equivalence of categories.
\end{prop}
\begin{proof}
Given a map of $\substTh$ models $\alpha : M \to N$, we define a map of contextual categories $F(\alpha) : F(M) \to F(N)$.
$F(\alpha)$ is already defined on objects.
Let $f = (a_1, \ldots a_k) \in Hom_{n,k}(B,A)$.
We define $F(\alpha)(f)$ as $(\alpha(a_1), \ldots \alpha(a_k)) \in Hom_{n,k}(\alpha(B), \alpha(A))$.
$F(\alpha)$ preserves identity morphisms, compositions, $f^*(A)$, and $q(f,A)$ since all of these operations are defined in terms of $\substTh$ operations.
Clearly, $F$ preserves identity maps and compositions of maps of $\substTh$ models.
Thus $F$ is a functor.

First, note that if $a \in M(tm,k)$ and $\alpha : M \to N$, then $F(\alpha)(\varphi(a)) = \varphi(\alpha(a))$.
Indeed, consider the following sequence of equations:
\begin{align*}
F(\alpha)(\varphi(a)) & = \\
F(\alpha)(v_{k,k-1}(ctx_{tm,k}(a)), \ldots v_{k,0}(ctx_{tm,k}(a)), a) & = \\
(v_{k,k-1}(ctx_{tm,k}(\alpha(a))), \ldots v_{k,0}(ctx_{tm,k}(\alpha(a))), \alpha(a)) & = \\
\varphi(\alpha(a)) & .
\end{align*}

Now, we prove that $F$ is faithful.
Let $\alpha,\beta : M \to N$ be a pair of maps of $\substTh$ models such that $F(\alpha) = F(\beta)$.
Then $\alpha$ and $\beta$ coincide on contexts.
Given $a \in M(tm,n)$ we have the following equation: $\alpha(a) = \varphi^{-1}(F(\alpha)(\varphi(a))) = \varphi^{-1}(F(\beta)(\varphi(a))) = \beta(a)$.

Now, we prove that $F$ is full.
Let $\alpha : F(M) \to F(N)$ be a map of contextual categories.
Then we need to define $\beta : M \to N$ such that $F(\beta) = \alpha$.
If $A \in M(ctx,n)$, then we let $\beta(A) = \alpha(A)$.
Note that if $f : ft_n(A) \to A$ is a section of $p_A$, then $\alpha(f)$ is a section of $\alpha(A)$.
If $a \in M(tm,n)$, then we let $\beta(a) = \varphi^{-1}(\alpha(\varphi(a)))$.

Maps $F(\beta)$ and $\alpha$ agree on contexts.
We prove by induction on $k$ that they coincide on morphisms $f = (a_1, \ldots a_k) \in M(Hom_{n,k})(B,A)$.
If $k = 0$, then $F(A)$ is terminal objects, hence $F(\beta) = \alpha$.
Suppose $k > 0$ and consider the following equation: $f = q((a_1, \ldots a_{k-1}), A) \circ \varphi(a_k)$.
By induction hypothesis we know that $F(\beta)(q((a_1, \ldots a_{k-1}), A)) = \alpha(q((a_1, \ldots a_{k-1}), A))$.
Thus we only need to prove that $F(\beta)(\varphi(a_k)) = \alpha(\varphi(a_k))$.
But $F(\beta)(\varphi(a_k)) = \varphi(\beta(a_k)) = \varphi(\varphi^{-1}(\alpha(\varphi(a_k)))) = \alpha(\varphi(a_k))$.

Finally, we prove that $F$ is essentially surjective on objects.
Given contextual category $C$ we define $\substTh$ model $M$.
Let $M(ctx,n)$ be equal to $Ob_n(C)$ and $M(tm,n)$ be the set of pairs of objects $A \in Ob_{n+1}(C)$ and sections of $p_A : A \to ft_n(A)$.
Let $ty_n$ be the obvious projection.
We will usually identify $a \in M(tm,n)$ with the section $ctx_{tm,n}(a) \to ty_n(a)$.

For each $n,k \in \mathbb{N}$ we define partial function
\[ subst_{ty,n,k} : M(ctx,n) \times M(ty,k) \times M(tm,n)^k \to M(ty,n) \]
such that $ft_n(subst_{ty,n,k}(B, A, a_1, \ldots a_k)) = B$.
We also define morphism
\[ q_{n,k} \in Hom_{n+1,k}(subst_{ty,n,k}(B, A, a_1, \ldots a_k), A) \]
whenever $subst_{ty,n,k}(B, A, a_1, \ldots a_k)$ is defined.
We define $subst_{ty,n,k}$ and $q_{n,k}$ by induction on $k$.
Let $subst_{ty,n,0}(B,A) = !_B^*(A)$ and $q_{n,0} = q(!_B,A)$ where $!_B : B \to Ob_0(C)$ is the unique morphism.
\[ \xymatrix{ subst_{ty,n,0}(B,A) \ar[r]^-{q_{n,0}} \ar[d] \pb & A \ar[d]^{p_A} \\
              B \ar[r]_{!_B} & 1
            } \]

Let $subst_{ty,n,k+1}(B, A, a_1, \ldots a_{k+1})$ be defined whenever $subst_{ty,n,k}(B, ft_k(A), \allowbreak a_1, \ldots a_k)$ is defined
    and $ty_n(a_{k+1}) = subst_{ty,n,k}(B, ft_k(A), a_1, \ldots a_k)$.
In this case we let $subst_{ty,n,k+1}(B, A, a_1, \ldots a_{k+1}) = f^*(A)$ and $q_{n,k+1} = q(f,A)$ where $f$ is the composition of $a_{k+1}$ and $q_{n,k}$.
\[ \xymatrix{ subst_{ty,n,k+1}(B, A, a_1, \ldots a_{k+1}) \ar[rr]^-{q_{n,k+1}} \ar[d] \pb & & A \ar[d]^{p_A} \\
              B \ar[r]_-{a_{k+1}} & ty_n(a_{k+1}) \ar[r]_-{q_{n,k}} & ft_k(A)
            } \]
It is easy to see by induction on $k$ that axiom~\eqref{ax:def-subst} holds.
Axiom~\eqref{ax:type-subst-ty} holds by definition of $subst_{ty,n,k}$.

The definition of predicates $Hom_{n,k}$ makes sense in $M$ now.
Thus we can define as before the set $Hom^M_{n,k}(B,A)$ of morphisms in $M$ as the set of tuples $(a_1, \ldots a_k)$ such that $Hom_{n,k}(B, A, a_1, \ldots a_k)$.
There is a bijection $\alpha : Hom^M_{n,k}(B,A) \to Hom_{n,k}(B,A)$ such that
    $subst_{ty,n,k}(B, A, a_1, \ldots a_k) = \alpha(a_1, \ldots a_k)^*(A)$ and $q_{n,k} = q(\alpha(a_1, \ldots a_k), A)$.
We define $\alpha$ by induction on $k$.
Both $Hom^M_{n,0}(B,A)$ and $Hom_{n,0}(B,A)$ are singletons, so there is a unique bijection between them.
If $(a_1, \ldots a_k) \in Hom^M_{n,k}(B,ft_k(A))$, then there is a bijection between morphisms $f \in Hom_{n,k+1}(B,A)$
    satisfying $p_A \circ f = \alpha(a_1, \ldots a_k)$ and sections of $p_{\alpha(a_1, \ldots a_k)^*(A)}$.
By induction hypothesis these sections are just sections of $p_{subst_{ty,n,k}(B, A, a_1, \ldots a_k)}$.
This gives us a bijection between $Hom^M_{n,k+1}(B,A)$ and $Hom_{n,k+1}(B,A)$, namely $\alpha(a_1, \ldots a_{k+1}) = q(\alpha(a_1, \ldots a_k), A) \circ a_{k+1}$.
Then the required equations hold by definition.

Now we define total functions $v_{n,i} : M(ctx,n) \to M(tm,n)$.
Let $v_{n,i}(A) = (p^{i+1}(A)^*(ft^i_{n-i}(A)), s_{p^i_A})$.
\[ \xymatrix{ p^{i+1}(A)^*(ft^i_{n-i}(A)) \ar[r] \ar[d] \pb & ft^i_{n-i}(A) \ar[d]^{p_{ft^i_{n-i}(A)}} \\
              A \ar[r]_{p^{i+1}(A)} \ar@/^1pc/[u]^{s_{p^i_A}} \ar[ur]_{p^i_A} & ft^{i+1}_{n-i-1}(A)
            } \]
Axiom~\eqref{ax:def-var} holds by definition.
By induction on $n-i$ it is easy to see that $\alpha(v_{n,n-1}(A), \ldots v_{n,i}(A))$ equals to $p_A^i : A \to ft^i_{n-i}(A)$.
Axiom~\eqref{ax:type-var} follows from the following sequence of equations:
\begin{align*}
subst_{ty,n,n-i-1}(A, ft^i_{n-i}(A), v_{n,n-1}(A), \ldots v_{n,i+1}(A)) & = \\
\alpha(v_{n,n-1}(A), \ldots v_{n,i+1}(A))^*(ft^i_{n-i}(A)) & = \\
p^{i+1}(A)^*(ft^i_{n-i}(A)) & = \\
ty_n(v_{n,i}(A)) & .
\end{align*}
Axiom~\eqref{ax:subst-var} follows from the facts that $\alpha(v_{n,n-1}(ft_n(A)), \ldots v_{n,0}(ft_n(A))) = id_{ft_n(A)}$ and $id_{ft_n(A)}^*(A) = A$.

Now we define partial functions $subst_{tm,n,k} : M(ctx,n) \times M(tm,k) \times M(tm,n)^k \to M(tm,n)$.
Function $subst_{tm,n,k}(B, a, a_1, \ldots a_k)$ is defined whenever \[ Hom_{n,k}(B, ctx_{tm,k}(a), a_1, \ldots a_k) \] holds.
In this case we let $subst_{tm,n,k}(B, a, a_1, \ldots a_k) = a[\alpha(a_1, \ldots a_k)]$ where $a[f] = s_{a \circ f}$.
Axioms \eqref{ax:def-subst} and \eqref{ax:type-subst-tm} hold by definition.
Axiom~\eqref{ax:subst-var} follows from the fact that $id_{ctx_{tm,n}(a)}^*(a) = a$.

To prove axiom~\eqref{ax:var-subst} note that $p_A \circ \alpha(a_1, \ldots a_{k+1}) = \alpha(a_1, \ldots a_k)$ by definition of $\alpha$.
Hence $p^i(A) \circ \alpha(a_1, \ldots a_k) = \alpha(a_1, \ldots a_{k-i})$.
Also note that $s_{\alpha(a_1, \ldots a_k)} = a_k$.
Now the axiom follows from the following equations:
\begin{align*}
subst_{tm,n,k}(B, v_{k,i}(A), a_1, \ldots a_k) & = \\
s_{v_{k,i}(A) \circ \alpha(a_1, \ldots a_k)} & = \\
s_{q(p^{i+1}(A), ft^i_{n-i}(A)) \circ v_{k,i}(A) \circ \alpha(a_1, \ldots a_k)} & = \\
s_{p^i(A) \circ \alpha(a_1, \ldots a_k)} & = \\
s_{\alpha(a_1, \ldots a_{k-i})} & = \\
a_{k-i} & .
\end{align*}

Now we prove that $\alpha$ preserves compositions.
To do this we need to show that $\alpha(a_1, \ldots a_k) \circ f = \alpha(a_1[f], \ldots a_k[f])$.
We do this by induction on $k$.
For $k = 0$ it is trivial and for $k > 0$ we have the following sequence of equations:
\begin{align*}
\alpha(a_1, \ldots a_k) \circ f & = \\
q(\alpha(a_1, \ldots a_{k-1}), A) \circ a_k \circ f & = \\
q(\alpha(a_1, \ldots a_{k-1}), A) \circ q(f, B[\alpha(a_1, \ldots a_k)]) \circ a_k[f] & = \\
q(\alpha(a_1, \ldots a_{k-1}) \circ f, A) \circ a_k[f] & = \\
q(\alpha(a_1[f], \ldots a_{k-1}[f]), A) \circ a_k[f] & = \\
\alpha(a_1[f], \ldots a_k[f]) & .
\end{align*}

Now axiom \eqref{ax:subst-subst} follows from the facts that $\alpha$ preserves compositions and $(f \circ g)^*(A) = f^*(g^*(A))$.
This completes the construction of $\substTh$ model $M$ from a contextual category $C$.
To finish the proof we need to show that $F(M)$ is isomorphic to $C$.
The isomorphism is given by bijection $\alpha$.
We already saw that $\alpha$ preserves the structure of contextual categories.
Thus $\alpha$ is a morphism of contextual categories, and it is easy to see that $\alpha^{-1}$ also preserves the structure.
Hence $\alpha$ is isomorphism and $F$ is an equivalence.
\end{proof}

Let $u : \substTh \to \mathbb{T}$ be an algebraic dependent type theory with substitution.
Then it follows from \rprop{func-mod} and \rprop{T1-CCat} that models of $\mathbb{T}$ are contextual categories with additional structure,
    where $u^* : \Mod{\mathbb{T}} \to \Mod{\substTh}$ is the forgetful functor.

\section{Algebraic dependent type theories}

In this section we consider partial Horn theories with additional structure which we call \emph{stable}.
We also define the category $\algtt$ of algebraic dependent type theories and give a few examples of such theories.

\subsection{Stable theories}

First, let us define prestable theories.
Let $\mathcal{S}_0$ be a set of (non-dependent) sorts.
Then we define the corresponding set $\mathcal{S}$ of dependent sorts as $\mathcal{S}_0 \times \mathbb{N}$.
Suppose that $\mathcal{S}_0$ contains a distinguished sort $ctx$.
Let $\mathbb{T}_{\mathcal{S}_0}$ be a theory with the following function symbols:
\begin{align*}
\emptyCtx  &: (ctx,0) \\
ft_n & : (ctx,n+1) \to (ctx,n) \\
ctx_{p,n} & : (p,n) \to (ctx,n) \text{ for every } p \in \mathcal{S}_0
\end{align*}
and axioms of $\mathbb{T}_{(ctx,0)}$ as defined in section~\ref{sec:prop} together with the following axiom:
\[ \sststile{}{x} ctx_{ctx,n}(x) = x \]

To define prestable theories, we need to introduce a few auxiliary constructions.
First, we define a function $L : \mathcal{C} \to \mathcal{C}$ as follows:
\begin{align*}
L(ctx,n) & = L(ctx,n+1) \\
L(tm,n) & = L(tm,n+1)
\end{align*}

For every set $\mathcal{F}$ of function symbols, we define another set $L(\mathcal{F})$ which consists of symbols $L(\sigma)$ for every $\sigma \in \mathcal{F}$.
If $\sigma : s_1 \times \ldots \times s_k \to s$, then $L(\sigma) : (ctx,1) \times L(s_1) \times \ldots \times L(s_k) \to L(s)$.
For every set of variables $V$ we define a set $L(V)$ which contains a variable $x$ of sort $L(s)$ for every variable $x$ of sort $s$ in $V$.
For every terms $\Gamma \in Term_{L(\mathcal{F})}(L(V))_{(ctx,1)}$ and $t \in Term_{\mathcal{F}}(V)_{(p,n)}$,
we define a partial term $L(\Gamma,t) \in PTerm_{L(\mathcal{F})}(L(V))_{(p,n+1)}$ as follows:
\begin{align*}
L(\Gamma, x) & = x|_{L(ctx_{p,n})(\Gamma, x) \downarrow} \\
L(\Gamma, \sigma(t_1, \ldots t_k)) & = L(\sigma)(\Gamma, L(\Gamma, t_1), \ldots L(\Gamma, t_k))
\end{align*}

For every set $\mathcal{P}$ of relation symbols, we define set $L(\mathcal{P})$ which consists of symbols
    $L(R) : (ctx,1) \times L(s_1) \times \ldots \times L(s_k)$ for every $R \in \mathcal{P}$, $R : s_1 \times \ldots \times s_k$.
For every formula $\varphi \in Form_\mathcal{P}(V)$ and term $\Gamma \in Term_{L(\mathcal{F})}(L(V))_{(ctx,1)}$,
we define a formula $L(\Gamma, \varphi) \in Form_{L(\mathcal{P})}(L(V))$ as follows:
\begin{align*}
L(\Gamma, t_1 = t_2) & = (L(\Gamma, t_1) = L(\Gamma, t_2)) \\
L(\Gamma, R(t_1, \ldots t_k)) & = L(R)(\Gamma, L(\Gamma, t_1), \ldots L(\Gamma, t_k))
\end{align*}

Now, let us define a functor $L : \mathbb{T}_{\mathcal{S}_0}/\Th_\mathcal{S} \to \mathbb{T}_{\mathcal{S}_0}/\Th_\mathcal{S}$.
Let $L((\mathcal{S}, \mathcal{F}, \mathcal{P}), \mathcal{A}) = ((\mathcal{S}, L(\mathcal{F}) \cup \mathcal{F}_{\mathcal{S}_0}, L(\mathcal{P})), \mathcal{A}' \cup \mathcal{A}_{\mathcal{S}_0})$,
where $\mathcal{F}_{\mathcal{S}_0}$ and $\mathcal{A}_{\mathcal{S}_0}$ are the sets of function symbols and axioms of $\mathbb{T}_{\mathcal{S}_0}$, and $\mathcal{A}'$ consists of the following axioms:
\[ ft^n(ctx_{p,n+1}(x)) = \Gamma \sststile{}{\Gamma,x} ctx_{p,n+1}(x) = L(ctx_{p,n})(\Gamma,x) \]
for every $p \in \mathcal{S}_0$,
\begin{align*}
L(\sigma)(\Gamma, x_1, \ldots x_k)\!\downarrow & \sststile{}{\Gamma, x_1, \ldots x_k} ft^n(ctx_{p,n}(L(\sigma)(\Gamma, x_1, \ldots x_k))) = \Gamma \\
L(\sigma)(\Gamma, x_1, \ldots x_k)\!\downarrow & \sststile{}{\Gamma, x_1, \ldots x_k} ft^{n_i}(ctx_{p_i,n_i}(x_i)) = \Gamma
\end{align*}
for every $\sigma \in \mathcal{F}$, $\sigma : (p_1,n_1) \times \ldots \times (p_k,n_k) \to (p,n)$ and every $1 \leq i \leq k$,
\[ L(R)(\Gamma, x_1, \ldots x_k) \sststile{}{\Gamma, x_1, \ldots x_k} ft^{n_i}(ctx_{p_i,n_i}(x_1)) = \Gamma \]
for every $R \in \mathcal{P}$, $R : (p_1,n_1) \times \ldots \times (p_k,n_k)$ and every $1 \leq i \leq k$.

If $f : \mathbb{T} \to \mathbb{T}'$, then let $L(f) : L(\mathbb{T}) \to L(\mathbb{T}')$ be defined as follows:
\begin{align*}
L(f)(L(\sigma)(\Gamma, x_1, \ldots x_k)) & = L(\Gamma, f(\sigma(x_1, \ldots x_k))) \\
L(f)(L(R)(\Gamma, x_1, \ldots x_k)) & = L(\Gamma, f(R(x_1, \ldots x_k)))
\end{align*}
It is easy to see that this defines a morphism of theories and that $L$ preserves identity morphisms and compositions.

\begin{defn}
A \emph{prestable (essentially) algebraic theory} is an algebra for functor $L$,
that is a pair $(\mathbb{T},\alpha)$, where $\mathbb{T}$ is a theory under $\mathbb{T}_{\mathcal{S}_0}$ and $\alpha : L(\mathbb{T}) \to \mathbb{T}$.
The category $\PSt_{\mathcal{S}_0}$ of prestable theories is the category of algebras for $L$.
\end{defn}

The construction of colimits in \rprop{th-cocomplete} implies that $L$ preserves colimits.
It follows that $\PSt_{\mathcal{S}_0}$ is cocomplete.
Since $L$ preserves colimits, the forgetful functor $\PSt_{\mathcal{S}_0} \to \mathbb{T}_{\mathcal{S}_0}/\Th_\mathcal{S}$
has a left adjoint $pst : \mathbb{T}_{\mathcal{S}_0}/\Th_\mathcal{S} \to \PSt_{\mathcal{S}_0}$, which we call the prestabilization functor.
More generally, for every $(\mathbb{T}_a,\alpha) \in \PSt_{\mathcal{S}_0}$,
we define a left adjoint $pst_{(\mathbb{T}_a,\alpha)} : \mathbb{T}_a/\Th_{\mathcal{S}} \to (\mathbb{T}_a,\alpha)/\PSt_{\mathcal{S}_0}$
to the forgetful functor $U_{(\mathbb{T}_a,\alpha)} : (\mathbb{T}_a,\alpha)/\PSt_{\mathcal{S}_0} \to \mathbb{T}_a/\Th_{\mathcal{S}}$.
Let $a : \mathbb{T}_a \to \mathbb{T}$ be an object of $\mathbb{T}_a/\Th_{\mathcal{S}}$.
Let $e : L^\infty(\mathbb{T}) \to E$ be the coequalizer of the following maps:
\[ \xymatrix{ \coprod\limits_{n \in \mathbb{N}} L^{n+1}(T_a) \ar@<+1ex>[rr]^{\coprod\limits_{n \in \mathbb{N}} L^n(f)} \ar@<-1ex>[rr]_{\coprod\limits_{n \in \mathbb{N}} L^n(g)}
    & & \coprod\limits_{n \in \mathbb{N}} L^n(L^\infty(\mathbb{T})) \ar@{^{(}->}[r]^-{i^n} & L^\infty(\mathbb{T}) } \]
where $L^\infty(X)$ is the following colimit:
\[ X \to X \amalg L(X) \to X \amalg L(X \amalg L(X)) \to \ldots \]
and $f,g : L(\mathbb{T}_a) \to L^\infty(\mathbb{T})$ are defined as follows:
$f$ is the composite $L(\mathbb{T}_a) \xrightarrow{\alpha} \mathbb{T}_a \xrightarrow{a} \mathbb{T} \hookrightarrow L^\infty(\mathbb{T})$,
and $g$ is the composite $L(\mathbb{T}_a) \xrightarrow{L(a)} L(\mathbb{T}) \hookrightarrow L^\infty(\mathbb{T})$.
Since $L$ preserves colimits, $L(E)$ is a coequalizer of $i^{n+1} \circ \coprod_{n \in \mathbb{N}} L^{n+1}(f)$ and $i^{n+1} \circ \coprod_{n \in \mathbb{N}} L^{n+1}(g)$.
By the universal property of coequalizers we have a map $\beta : L(E) \to E$.
We define $pst_{(\mathbb{T}_a,\alpha)}(a)$ as $(E,\beta)$, and morphism $(\mathbb{T}_a,\alpha) \to (E,\beta)$
as the composite $\mathbb{T}_a \xrightarrow{a} \mathbb{T} \hookrightarrow L^\infty(\mathbb{T}) \xrightarrow{e} E$.
This map is a morphism of algebras for $L$ since $e$ coequalizes $f$ and $g$.
Moreover, if $(D,\delta)$ is an object of $(\mathbb{T}_a,\alpha)/\PSt_{\mathcal{S}_0} $,
then a map $L^\infty(\mathbb{T}) \to D$ is a morphism of algebras if and only if it factors through $E$.
It follows that $pst_{(\mathbb{T}_a,\alpha)}$ is left adjoint to $U_{(\mathbb{T}_a,\alpha)}$.

\begin{defn}
A prestable theory is called \emph{stable} if the following theorem holds for every axiom $\varphi \sststile{}{x_1 : (p_1,n_1), \ldots x_k : (p_k,n_k)} \psi$ in $\mathcal{A}$:
\[ \alpha L(\Gamma,\varphi) \land \bigwedge_{1 \leq i \leq k} ft^{n_i}(ctx_{p_i,n_i}(x_i)) = \Gamma \sststile{}{\Gamma, x_1, \ldots x_k} \alpha L(\Gamma,\psi). \]
The category of stable theories is denoted by $\St_{\mathcal{S}_0}$.

Let $c$ be the prestable theory generated by a single constant $c : (ctx,1)$.
Then a prestable theory under $c$ is called \emph{$c$-stable} if the following theorems are derivable:
\begin{align*}
L(\sigma)(\Gamma, x_1, \ldots x_k)\!\downarrow & \sststile{}{\Gamma, x_1, \ldots x_k} \Gamma = c \\
L(R)(\Gamma, x_1, \ldots x_k) & \sststile{}{\Gamma, x_1, \ldots x_k} \Gamma = c \\
\alpha L(c,\varphi) \land \bigwedge_{1 \leq i \leq k} ft^{n_i}(ctx_{p_i,n_i}(x_i)) = c & \sststile{}{x_1, \ldots x_k} \alpha L(c,\psi)
\end{align*}
for every function symbol $\sigma$, every predicate symbol $R$, and every axiom $\varphi \sststile{}{x_1, \ldots x_k} \psi$.
The category of $c$-stable theories is denoted by $\cSt_{\mathcal{S}_0}$.
\end{defn}

The theory of substitutions is stable.
Indeed, we can define maps $\alpha : L(\substTh) \to \substTh$ as follows:
\begin{align*}
\alpha(L(ty_n)(\Gamma,a)) & = ty_{n+1}(a)|_{ft^n(ctx_{tm,n+1}(a)) = \Gamma} \\
\alpha(L(v_{n,i})(\Gamma,\Delta)) & = v_{n+1,i}(\Delta)|_{ft^n(\Delta) = \Gamma}
\end{align*}
and $\alpha(L(subst_{p,n,k})(\Gamma, \Delta, B, a_1, \ldots a_k))$ is defined as
\[ subst_{p,n+1,k+1}(\Delta, B, v_{n+1,n}(\Delta), a_1, \ldots a_k)|_{ft^n(\Delta) = \Gamma} \]

The categories of stable and $c$-stable theories are cocomplete.
The inclusion functors $\St_{\mathcal{S}_0} \to \PSt_{\mathcal{S}_0}$ and $\cSt_{\mathcal{S}_0} \to \PSt_{\mathcal{S}_0}$ have left adjoints,
which are defined as the functors that add the required stability axioms.
We call these left adjoints \emph{the stabilization functors}.

\subsection{Contextual theories}

The definition of prestable theories has a disadvantage that terms contain a lot of redundant information.
For example, when we describe a term we need to repeat the context in which it is defined several times.
The following notion allows us to omit this redundant information as we discuss below.

\begin{defn}
Let $\mathbb{T}_b$ be a prestable theory.
A \emph{contextual theory under $\mathbb{T}_b$} is a prestable theory $\mathbb{T}$ such that the following conditions hold:
\begin{enumerate}
\item There exists a set of function symbols $\mathcal{F}_0$ such that the set of function symbols of $\mathbb{T}$ consists of
function symbols of $\mathbb{T}_b$ together with symbols $\sigma_m : (ctx,m) \times (p_1,n_1+m) \times \ldots \times (p_k,n_k+m) \to (p,n+m)$
for every $\sigma : (p_1,n_1) \times \ldots \times (p_k,n_k) \to (p,n) \in \mathcal{F}_0$ and $m \in \mathbb{N}$.
Moreover, if $\sigma : s_1 \times \ldots \times s_k \to s \in \mathcal{F}_0$, then $s \neq (ctx,0)$.
\item There exists a set of predicate symbols $\mathcal{P}_0$ such that the set of predicate symbols of $\mathbb{T}$ consists of
predicate symbols of $\mathbb{T}_b$ together with symbols $R_m : (ctx,m) \times (p_1,n_1+m) \times \ldots \times (p_k,n_k+m)$
for every $R : (p_1,n_1) \times \ldots \times (p_k,n_k) \in \mathcal{P}_0$ and $m \in \mathbb{N}$.
\item Axioms of $\mathbb{T}_b$ hold in $\mathbb{T}$.
\item $\alpha_\mathbb{T} : L(\mathbb{T}) \to \mathbb{T}$ is defined as follows:
\begin{align*}
\alpha_\mathbb{T}(L(\sigma_m)(\Gamma, \Delta, x_1, \ldots x_k)) & = \sigma_{m+1}(\Delta, x_1, \ldots x_k)|_{ctx^{n+m}(\Delta) = \Gamma} \\
\alpha_\mathbb{T}(L(R_m)(\Gamma, \Delta, x_1, \ldots x_k)) & = R_{m+1}(\Delta, x_1, \ldots x_k) \land ctx^{n+m}(\Delta) = \Gamma
\end{align*}
and for every symbol of $\mathbb{T}_b$, it is defined in the same way as in $\mathbb{T}_b$.
\end{enumerate}
\end{defn}

Since we can always infer the index $m$ for every function symbol $\sigma_m$ if we know its sort, we usually omit this index.
To specify the omitted argument, we use the following syntax: $\Gamma \vdash t$,
which stands for $\sigma(\Gamma, t_1, \ldots t_k)$ if $t = \sigma(t_1, \ldots t_k)$ and for $x|_{ctx(x) = \Gamma}$ if $t = x$.
Of course, if some arguments are omitted in $\Gamma$, then we need to know its context too in order to infer them.
Thus, we may write $A_1, \ldots A_n \vdash t$ which stands for $(\ldots ((\emptyCtx \vdash A_1) \vdash A_2) \ldots \vdash A_n) \vdash t$.
We also use this notation in formulas: $\Gamma \vdash t \equiv t'$ stands for $(\Gamma \vdash t) = (\Gamma \vdash t')$
and $\Gamma \vdash R(t_1, \ldots t_k)$ stands for $R(\Gamma, (\Gamma \vdash t_1), \ldots (\Gamma \vdash t_k))$.

Also, we use the standard notation: $\Gamma \vdash A\ type$ stands for $\Gamma \vdash A\!\downarrow$ if $A : (ty,n)$ and $\Gamma \vdash a : A$ stands for $ty(\Gamma \vdash a) = (\Gamma \vdash A)$.
Sequents $\varphi_1 \land \ldots \land \varphi_n \sststile{}{V} \psi$ and $\varphi_1 \land \ldots \land \varphi_n \ssststile{}{V} \psi$ are written as
\medskip
\begin{center}
\AxiomC{$\varphi_1$}
\AxiomC{$\ldots$}
\AxiomC{$\varphi_n$}
\TrinaryInfC{$\psi$}
\DisplayProof
\qquad
and
\qquad
\AxiomC{$\varphi_1$}
\AxiomC{$\ldots$}
\AxiomC{$\varphi_n$}
\doubleLine
\RightLabel{,}
\TrinaryInfC{$\psi$}
\DisplayProof
\end{center}
respectively.

Finally, we use the following syntax:
\[ \sigma(A^1_1, \ldots A^1_{n_1}.\ b_1, \ldots A^k_1, \ldots A^k_{n_k}.\ b_k) \]
for a term of sort $(p,m+n)$ in a contextual theory, where $\sigma : (p_1,n_1) \times \ldots \times (p_k,n_k) \to (p,n)$,
$b_i$ is a term of sort $(p_i,m+n_i)$, and $A^i_j$ is a term of sort $(ty,m+j-1)$.
Term $\Gamma \vdash \sigma(A^1_1, \ldots A^1_{n_1}.\ b_1, \ldots A^k, \ldots A^k_{n_k}.\ b_k)$ stands for
\[ \sigma_m(\Gamma, (\Gamma, A^1_1, \ldots A^1_{n_1} \vdash b_1), \ldots (\Gamma, A^k_1, \ldots A^k_{n_k} \vdash b_k)). \]
Of course, if some $b_i$ is a variable, then we can omit $A^i_1, \ldots A^i_{n_i}$.
We also can omit this context if there is a theorem of the following form: 
\[ E \vdash \sigma_m(x_1, \ldots x_k)\!\downarrow\ \sststile{}{E, x_1, \ldots x_k} E \vdash ctx(x_i) \equiv \Delta \]
for some $\Delta$ such that $x_i \notin FV(\Delta)$.
Then $A^i_1, \ldots A^i_{n_i}$ must be equal to $((\Gamma \vdash ft^{n_i-1}(\Delta)), \ldots (\Gamma \vdash \Delta))[\rho]$,
where $\rho(E) = \Gamma$ and $\rho(x_j) = (\Gamma, A^j_1, \ldots A^j_{n_j} \vdash b_j)$.

The following lemma shows that we can always replace a prestable theory with a contextual one.

\begin{lem}[stable-con]
Let $\mathbb{T}_b$ be a prestable theory.
Every prestable theory under $\mathbb{T}_b$ is isomorphic to a contextual theory under $\mathbb{T}_b$.
\end{lem}
\begin{proof}
Let $\mathbb{T}$ be a prestable theory together with a map $f : \mathbb{T}_b \to \mathbb{T}$ with $\mathcal{F}_0$ and $\mathcal{P}_0$ as the sets of function and predicate symbols.
First, note that we may assume that for every $\sigma : s_1 \times \ldots \times s_k \to s$ in $\mathcal{F}_0$, $s \neq (ctx,0)$.
Indeed, we can always replace such a function symbol with a predicate symbol $R_\sigma : s_1 \times \ldots \times s_k$.

Second, note that for every term $t \in Term_{\mathcal{F}_0}(V)_{(p,n)}$ and every $m \in \mathbb{N}$, we can construct the following partial term:
\[ \alpha L(ft^{m-1}(\Gamma), \alpha L(ft^{m-2}(\Gamma), \ldots \alpha L(\Gamma, t))) \]
in $PTerm_{\mathbb{T}}(L^m(V) \amalg \{ \Gamma : (ctx,m) \})_{(p,n+m)}$, which we denote by $\Gamma \times t$.
Analogously, we can define for every formula $\varphi \in Form_{\mathbb{T}}(V)$ and every $m \in \mathbb{N}$,
a formula $\Gamma \times \varphi \in Form_{\mathbb{T}}(L^m(V) \amalg \{ \Gamma : (ctx,m) \})$.

Let $\mathbb{T}'$ be a contextual theory under $\mathbb{T}_b$ defined from the sets $\mathcal{F}_0$ and $\mathcal{P}_0$.
Note that every term (formula, sequent) of $\mathbb{T}$ is naturally a term (formula, sequent) of $\mathbb{T}'$.
Axioms of $\mathbb{T}'$ is the axioms of $\mathbb{T}$ together with the following axioms:
\begin{align*}
& \sststile{}{x_1, \ldots x_k} \tau_0(\emptyCtx, x_1, \ldots x_k) \cong f(\tau(x_1, \ldots x_k)) \\
P_0(\emptyCtx, x_1, \ldots x_k) & \ssststile{}{x_1, \ldots x_k} f(P(x_1, \ldots x_k)) \\
& \sststile{}{\Gamma, x_1, \ldots x_k} \Gamma \times \sigma_0(\emptyCtx, x_1, \ldots x_k) \cong \sigma_m(\Gamma, x_1, \ldots x_k) \\
\Gamma \times R_0(\emptyCtx, x_1, \ldots x_k) & \ssststile{}{\Gamma, x_1, \ldots x_k} R_m(\Gamma, x_1, \ldots x_k)
\end{align*}
for every function symbol $\tau$ and predicate symbol $P$ of $\mathbb{T}_b$ and every $\sigma \in \mathcal{F}_0$ and $R \in \mathcal{P}_0$.

There is an obvious map $\mathbb{T} \to \mathbb{T}'$ and we can define a map $\mathbb{T}' \to \mathbb{T}$
which sends $\sigma_m(\Gamma, x_1, \ldots x_k)$ to $\Gamma \times \sigma_0(\emptyCtx, x_1, \ldots x_k)$,
$R_m(\Gamma, x_1, \ldots x_k)$ to $\Gamma \times R_0(\emptyCtx, x_1, \ldots x_k)$,
$\tau_0(\Gamma, x_1, \ldots x_k)$ to $f(\tau(x_1, \ldots x_k))|_{\Gamma\downarrow}$, and $P_0(\Gamma, x_1, \ldots x_k)$ to $f(P(x_1, \ldots x_k))|_{\Gamma\downarrow}$.
Axioms guarantee that these maps are inverses of each other.
\end{proof}

Contextual theories constructed in the previous lemma are not convenient in practice, but usually theories are defined in a contextual form.
It is easy to define such theory: we just need to specify sets $\mathcal{F}_0$ and $\mathcal{P}_0$ and the set of axioms.
It is also easy to define a morphism of contextual theories since we only need to define it on symbols from $\mathcal{F}_0$ and $\mathcal{P}_0$.
Then it uniquely extends to a morphism of prestable theories.

\subsection{Algebraic dependent type theories}

Algebraic dependent type theories are prestable theories under $\substTh$ in which substitution commutes with all function symbols.
To define such theories, we need to define weakening first.
For every $p \in \{ty,tm\}$, the operations of weakening $wk^{m,l}_{p,n} : (ctx,n+m) \times (p,n+l) \to (p,n+m+l)$ are defined as follows:
\begin{align*}
wk^{m,0}_{p,n}(\Gamma,a) & = subst_{p,n+m,n}(\Gamma, a, v_{n+m-1}, \ldots v_m) \\
wk^{m,l+1}_{p,n}(\Gamma,a) & = subst_{p,n+m+l+1,n+l+1}(\Gamma', a, v_{n+m+l}, \ldots v_{m+l+1}, v_l, \ldots v_0),
\end{align*}
where $\Gamma' = wk^{m,l}_{ty,n}(\Gamma,ctx(a))$.
We also define $wk^{m,l}_{ctx,n} : (ctx,n+m) \times (ctx,n+l) \to (ctx,n+m+l)$ as follows:
\begin{align*}
wk^{m,0}_{ctx,n}(\Gamma,a) & = \Gamma \\
wk^{m,l+1}_{ctx,n}(\Gamma,a) & = wk^{m,l}_{ty,n}(\Gamma,a).
\end{align*}

Now, we need to introduce a new derived operation.
For every $m,n,k \in \mathbb{N}$ and $p \in \{ ctx, ty, tm \}$, we define the following function:
\[ subst^m_{p,n,k} : (ctx,n) \times (p,k+m) \times (tm,n)^k \to (p,n+m). \]
First, let $subst^0_{ctx,n,k}(B, A, a_1, \ldots a_k) = B$ and $subst^{m+1}_{ctx,n,k} = subst^m_{ty,n,k}$.
If $p \in \{ ty, tm \}$, then let $subst^m_{p,n,k}(B, a, a_1, \ldots a_k)$ be equal to
\[ subst_{p,n+m,k+m}(B', a, wk^{m,0}_{tm,n}(a_1), \ldots wk^{m,0}_{tm,n}(a_k), v_{m-1}, \ldots v_0), \]
where $B' = subst^m_{ctx,n,k}(B, ctx_{k+m}(a), a_1, \ldots a_k)$.

\begin{defn}
A prestable theory under $\substTh$ is an \emph{algebraic dependent type theory} if,
for every $\sigma \in \mathcal{F}$, $\sigma : (p_1,n_1) \times \ldots \times (p_k,n_k) \to (p,n)$
and every $R \in \mathcal{P}$, $R : (p_1,n_1) \times \ldots \times (p_k,n_k)$, the following sequents are derivable in it:
\medskip
\begin{center}
\AxiomC{$\Delta \times \sigma(b_1, \ldots b_k) \downarrow$}
\AxiomC{$\bigwedge_{1 \leq i \leq m} ty(a_i) = subst_{ty,l,i-1}(\Gamma, ft^{m-i}(\Delta), a_1, \ldots a_{i-1})$}
\BinaryInfC{$subst^n_{p,l,m}(\Gamma, \Delta \times \sigma(b_1, \ldots b_k), a_1, \ldots a_m) = \Gamma \times \sigma(b_1', \ldots b_k')$}
\DisplayProof
\end{center}
\medskip

\begin{center}
\AxiomC{$\Delta \times R(b_1, \ldots b_k)$}
\AxiomC{$\bigwedge_{1 \leq i \leq m} ty(a_i) = subst_{ty,l,i-1}(\Gamma, ft^{m-i}(\Delta), a_1, \ldots a_{i-1})$}
\BinaryInfC{$\Gamma \times R(b_1', \ldots b_k')$}
\DisplayProof
\end{center}
\medskip
where $b_i' = subst^{n_i}_{p_i,l,m}(\Gamma, b_i, a_1, \ldots a_m)$.

The category of algebraic dependent type theories will be denoted by $\algtt$.
\end{defn}

The construction of colimits in \rprop{th-cocomplete} implies that $\algtt$ is closed under colimits in $\substTh/\PSt_\mathcal{C}$.
The inclusion functor $\algtt \to \substTh/\PSt_\mathcal{C}$ has a left adjoint $\substTh/\PSt_\mathcal{C} \to \algtt$, which simply adds the required axioms.

We can prove a stronger version of \rlem{stable-con} for algebraic dependent type theories:
\begin{lem}[adtt-con]
Every algebraic dependent type theory is isomorphic to a contextual theory in which every function symbol in $\mathcal{F}_0$ has a signature of the form
\[ \sigma : s_1 \times \ldots \times s_k \to (p,0), \]
where $p \in \{ ty,tm \}$.
\end{lem}
\begin{proof}
Let $\mathbb{T}$ be an algebraic dependent type theory.
By \rlem{stable-con}, we may assume that $\mathbb{T}$ is contextual.
Then we define theory $\mathbb{T}'$ which has the same predicate symbols as $\mathbb{T}$.
For every $\sigma : (p_1,n_1) \times \ldots \times (p_k,n_k) \to (p,n)$ in $\mathcal{F}_0$, we add the following function symbol to $\mathbb{T}'$:
\[ \sigma' : (p_1,n_1) \times \ldots \times (p_k,n_k) \times (tm,0)^n \to (p,0). \]
Then we define $f(\sigma_0(\Gamma, x_1, \ldots x_k))$ as
\[ \sigma'_n(\Gamma, wk^{n,n_1}_{p_1,0}(\Gamma, x_1), \ldots wk^{n,n_k}_{p_k,0}(\Gamma, x_k), v_{n-1}, \ldots v_0). \]
For every predicate symbol $R$, we define $f(R(x_1, \ldots x_k))$ as $R(x_1, \ldots x_k)$.
For every axiom $\varphi \sststile{}{V} \psi$ of $\mathbb{T}$, we add axiom $f(\varphi) \sststile{}{V} f(\psi)$ to $\mathbb{T}'$.
Then $f$ is a morphism of theories $f : \mathbb{T} \to \mathbb{T}'$.

Moreover, there is a morphism $g : \mathbb{T}' \to \mathbb{T}$, which is defined as follows:
\begin{align*}
g(\sigma'_0(\Gamma, x_1, \ldots x_k, y_1, \ldots y_n)) & = subst^n_{p,0,0}(\Gamma, \sigma_0(\Gamma, x_1, \ldots x_k), y_1, \ldots y_n) \\
g(R(x_1, \ldots x_k)) & = R(x_1, \ldots x_k)
\end{align*}
The axioms of algebraic dependent type theories imply that $f$ and $g$ are inverses of each other.
\end{proof}

When we say that an algebraic dependent type theory is contextual (or presented in a contextual form), then we assume that it has a form as described in the previous lemma.

If an algebraic dependent type theory is presented in a contextual form, then every term is equivalent to a term in which substitution operations are applied only to variables.
We can as usual omit the first argument to $subst_{p,n,k}$.
Also, if $X : (p,n+k)$ and $a_1, \ldots a_k : (tm,n)$, then we write $X[a_1, \ldots a_k]$ for
\[ subst_{p,n,k}(X, v_{n-1}, \ldots v_0, a_1, \ldots a_k). \]

One last problem is that we often need to apply weakening operations to variables.
It is not convenient to do this explicitly, so we introduce named variables in our terms.
Let $Var$ be some fixed countable set of variables.
To distinguish these variable from the ones that we used before we will call the latter \emph{metavariables}.
First, we assume that every metavariable $X$ of sort $(p,n)$ is equipped with a sequence of variables of length $n$, which we call the context of this metavariable.
Usually, we do not specify the context of a metavariable explicitly since it can be inferred from formulas and terms in which this metavariable appears.

Second, every binding should be annotated with a variable.
In particular, instead of $A_1, \ldots A_n \vdash b$ we should write $x_1 : A_1, \ldots x_n : A_n \vdash b$
and instead of $\sigma(A^1_1, \ldots A^1_{n_1}.\ b_1, \ldots A^k, \ldots A^k_{n_k}.\ b_k)$ we should write
\[ \sigma((x_1 : A^1_1), \ldots (x_{n_1} : A^1_{n_1}).\ b_1, \ldots ((x_1 : A^k_1), \ldots (x_{n_k} : A^k_{n_k}).\ b_k) \]

Now, we may use variables instead of de Bruijn indices.
If a variable $x_i$ appears in a context $x_1, \ldots x_n$, then it is decoded into expression $v_{n-i}$.
Every metavariable should appear in a context where all variables from its context are available.
Then a metavariable $X$ with context $x_1, \ldots x_n$ should be replaced with expression $subst(X, x_1, \ldots x_n)$.
We may also write $X[x_{i_1} \mapsto a_{i_1}, \ldots x_{i_k} \mapsto a_{i_k}]$,
which is replaced with expression $subst(X, a_1, \ldots a_n)$, where $a_j = x_j$ if $j \notin \{ i_1, \ldots i_k \}$.
Finally, we may write $ft^i(X)$, which works like a metavariable with context $x_1, \ldots x_{n-i}$.

\subsection{Examples}

Now, let us describe a few examples of algebraic dependent type theories with substitution.
If we take their stabilization, then we get theories corresponding to usual constructions of the type theory.
Every theory is presented in the contextual form.
Also, to simplify the notation, we use the following agreement.
For every sequent of the form $\varphi \sststile{}{} \Gamma \vdash A\ type$, there is also sequent $\Gamma \vdash A\ type \sststile{}{} \varphi$
and, for every sequent of the form $\varphi \sststile{}{} \Gamma \vdash a : A$, there is also sequent $\Gamma \vdash a\!\downarrow\ \sststile{}{} \varphi$.

\begin{example}
The theory of unit types with eta rules has function symbols $\top : (ty,0)$ and $unit : (tm,0)$ and the following axioms:
\medskip
\begin{center}
\AxiomC{}
\UnaryInfC{$\vdash \top\ type$}
\DisplayProof
\quad
\AxiomC{}
\UnaryInfC{$\vdash unit : \top$}
\DisplayProof
\quad
\AxiomC{$\vdash t : \top$}
\UnaryInfC{$\vdash t \deq unit$}
\DisplayProof
\end{center}
\end{example}

\begin{example}
The theory of unit types without eta rules has function symbols $\top : (ty,0)$, $unit : (tm,0)$ and $\top\text{-}elim : (ty,1) \times (tm,0) \times (tm,0) \to (tm,0)$.
The axioms for $\top$ and $unit$ are the same, and the axioms for $\top\text{-}elim$ are
\medskip
\begin{center}
\AxiomC{$x : \top \vdash D\ type$}
\AxiomC{$\vdash d : D[x \mapsto unit]$}
\AxiomC{$\vdash t : \top$}
\TrinaryInfC{$\vdash \top\text{-}elim(x.\,D, d, t) : D[x \mapsto t]$}
\DisplayProof
\end{center}

\medskip
\begin{center}
\AxiomC{$x : \top \vdash D\ type$}
\AxiomC{$\vdash d : D[x \mapsto unit]$}
\BinaryInfC{$\vdash \top\text{-}elim(x.\,D, d, unit) \deq d$}
\DisplayProof
\end{center}
\end{example}

\begin{example}[sigma-eta]
The theory of $\Sigma$ types with eta rules has function symbols
\begin{align*}
\Sigma & : (ty,1) \to (ty,0) \\
pair & : (ty,1) \times (tm,0) \times (tm,0) \to (tm,0) \\
proj_1 & : (ty,1) \times (tm,0) \to (tm,0) \\
proj_2 & : (ty,1) \times (tm,0) \to (tm,0)
\end{align*}
and the following axioms:
\medskip
\begin{center}
\AxiomC{}
\UnaryInfC{$\vdash \Sigma(x.\,B)\ type$}
\DisplayProof
\quad
\AxiomC{$\vdash b : B[x \mapsto a]$}
\UnaryInfC{$\vdash pair(x.\,B, a, b) : \Sigma(x.\,B)$}
\DisplayProof
\end{center}

\medskip
\begin{center}
\AxiomC{$\vdash p : \Sigma(x.\,B)$}
\UnaryInfC{$\vdash proj_1(x.\,B, p) : ft(B)$}
\DisplayProof
\quad
\AxiomC{$\vdash p : \Sigma(x.\,B)$}
\UnaryInfC{$\vdash proj_2(x.\,B, p) : B[x \mapsto proj_1(x.\,B, p)]$}
\DisplayProof
\end{center}

\medskip
\begin{center}
\AxiomC{$\vdash b : B[x \mapsto a]$}
\UnaryInfC{$\vdash proj_1(x.\,B, pair(x.\,B, a, b)) \deq a$}
\DisplayProof
\qquad
\AxiomC{$\vdash b : B[x \mapsto a]$}
\UnaryInfC{$\vdash proj_2(x.\,B, pair(x.\,B, a, b)) \deq b$}
\DisplayProof
\end{center}

\medskip
\begin{center}
\AxiomC{$\vdash p : \Sigma(x.\,B)$}
\UnaryInfC{$\vdash pair(x.\,B, proj_1(x.\,B, p), proj_2(x.\,B, p)) \deq p$}
\DisplayProof
\end{center}
\end{example}

\begin{example}[sigma-no-eta]
The theory of $\Sigma$ types without eta rules has the following function symbols:
\begin{align*}
\Sigma & : (ty,1) \to (ty,0) \\
pair & : (ty,1) \times (tm,0) \times (tm,0) \to (tm,0) \\
\Sigma\text{-}elim & : (ty,1) \times (tm,2) \times (tm,0) \to (tm,0)
\end{align*}
The axioms for $\Sigma$ and $pair$ are the same, and the axioms for $\Sigma\text{-}elim$ are
\medskip
\begin{center}
\AxiomC{$z : \Sigma(x.\,B) \vdash D\ type$}
\AxiomC{$x : ft(B), y : B \vdash d : D'$}
\AxiomC{$\vdash p : \Sigma(x.\,B)$}
\TrinaryInfC{$\vdash \Sigma\text{-}elim(z.\,D, x y.\,d, p) : D[z \mapsto p]$}
\DisplayProof
\end{center}

\medskip
\begin{center}
\def\extraVskip{1pt}
\AxiomC{$z : \Sigma(x.\,B) \vdash D\ type$}
\AxiomC{$x : ft(B), y : B \vdash d : D'$}
\AxiomC{$\vdash b : B[x \mapsto a]$}
\TrinaryInfC{$\vdash \Sigma\text{-}elim(z.\,D, x y.\,d, pair(x.\,B, a, b)) \deq d[x \mapsto a, y \mapsto b]$}
\DisplayProof
\end{center}
\end{example}

where $D' = D[z \mapsto pair(x.\,B, x, y)]$.

\begin{example}[pi-eta]
The theory of $\Pi$ types with eta rules has function symbols
\begin{align*}
\Pi & : (ty,1) \to (ty,0) \\
\lambda & : (tm,1) \to (tm,0) \\
app & : (ty,1) \times (tm,0) \times (tm,0) \to (tm,0)
\end{align*}
and the following axioms:
\medskip
\begin{center}
\AxiomC{}
\UnaryInfC{$\vdash \Pi(x.\,B)\ type$}
\DisplayProof
\quad
\AxiomC{}
\UnaryInfC{$\vdash \lambda(x.\,b) : \Pi(x.\,ty(b))$}
\DisplayProof
\end{center}

\medskip
\begin{center}
\AxiomC{$\vdash f : \Pi(x.\,B)$}
\AxiomC{$\vdash a : ft(B)$}
\BinaryInfC{$\vdash app(x.\,B, f, a) : B[x \mapsto a]$}
\DisplayProof
\end{center}

\medskip
\begin{center}
\AxiomC{$\vdash a : ft(B)$}
\UnaryInfC{$\vdash app(x.\,B, \lambda(x.\,b), a) \deq b[x \mapsto a]$}
\DisplayProof
\quad
\AxiomC{$\vdash f : \Pi(x.\,B)$}
\UnaryInfC{$\vdash \lambda(y.\,app(x.\,B, f, y)) \deq b$}
\DisplayProof
\end{center}
\end{example}

\begin{example}[Id]
The theory of identity types has function symbols
\begin{align*}
Id & : (tm,0) \times (tm,0) \to (ty,0) \\
refl & : (tm,0) \to (tm,0) \\
J & : (ty,3) \times (tm,1) \times (tm,0) \times (tm,0) \times (tm,0) \to (tm,0)
\end{align*}
and the following inference rules:
\medskip
\begin{center}
\AxiomC{$\vdash ty(a) \deq ty(a')$}
\UnaryInfC{$\vdash Id(a, a')\ type$}
\DisplayProof
\quad
\AxiomC{}
\UnaryInfC{$\vdash refl(a) : Id(a, a)$}
\DisplayProof
\end{center}

\medskip
\begin{center}
\AxiomC{$x : A, y : A, z : Id(x, y) \vdash D\ type$}
\AxiomC{$x : A \vdash d : D'$}
\AxiomC{$\vdash p : Id(a, a')$}
\TrinaryInfC{$\vdash J(x y z.\,D, x.\,d, a, a', p) : D[x \mapsto a, y \mapsto a', z \mapsto p]$}
\DisplayProof
\end{center}

\medskip
\begin{center}
\AxiomC{$x : A, y : A, z : Id(x, y) \vdash D\ type$}
\AxiomC{$x : A \vdash d : D'$}
\BinaryInfC{$\vdash J(x y z.\,D, x.\,d, a, a, refl(a)) \deq d[x \mapsto a]$}
\DisplayProof
\end{center}
\medskip

where $A = ty(a)$ and $D' = D[y \mapsto x, z \mapsto refl(x)]$.
\end{example}

\begin{example}
We define an endofunctor $U$ on the category of algebraic dependent type theories.
For every such theory $\mathbb{T}$, theory $U(\mathbb{T})$ has the same symbols as $\mathbb{T}$,
but it also has a universe which is closed under all function symbols of $\mathbb{T}$.

Let $\mathbb{T}$ be an algebraic dependent type theory in a contextual form.
Then $U(\mathbb{T})$ has the same predicate symbols as $\mathbb{T}$ and the following function symbols:
\begin{align*}
U & : (ty,0) \\
El & : (tm,0) \to (ty,0) \\
\sigma & : s_1 \times \ldots \times s_k \to (p,0) \\
\sigma^U & : U(s_1) \times s_1 \times \ldots \times U(s_k) \times s_k \to (tm,0)
\end{align*}
for every function symbol $\sigma : s_1 \times \ldots \times s_k \to (p,0)$ of $\mathbb{T}$, where $U(p,n_i) = (tm,0) \times \ldots \times (tm,n_i)$.

Theory $U(\mathbb{T})$ has the following axioms:
\medskip
\begin{center}
\AxiomC{}
\UnaryInfC{$\vdash U\ type$}
\DisplayProof
\qquad
\AxiomC{$\vdash a : U$}
\doubleLine
\UnaryInfC{$\vdash El(a)\ type$}
\DisplayProof
\end{center}
\medskip

For every function symbol $\sigma : (p_1,n_1) \times \ldots \times (p_k,n_k) \to p$ of $\mathbb{T}$ and every $1 \leq i \leq k$, we add the following axioms to $U(\mathbb{T})$:
\[ \vdash \sigma^U(t_1, \ldots, t_m)\!\downarrow\ \sststile{}{t_1, \ldots t_m}\ \vdash \sigma^U(t_1, \ldots t_m) : U \]
\[ \sststile{}{V}\ \vdash El(\sigma^U(\ldots, a_1, \ldots a_{n_i+1}, b, \ldots)) \cong e_p(\sigma(\ldots, b|_{\varphi_i}, \ldots)), \]
where $a_1, \ldots a_{n_i+1}, b$ are variables that correspond to $i$-th variable in $\sigma$, $e_{ty}(x) = x$, and $e_{tm}(x) = ty(x)$, and $\varphi_i$ equals to
\[ \bigwedge_{1 \leq j \leq n_i+1} El(a_j) = ft^{n_i+1-j}(e_{p_i}(b)). \]

To define the rest of the axioms of $U(\mathbb{T})$, we need to introduce a few auxiliary functions.
For every set of variables $V$, we define a set $U(V)$ as follows:
\[ V \amalg \{ x^j : (tm,n-j)\ |\ x : (p,n) \in V, 0 \leq j \leq n \}. \]
Now, we define a function $U : Term_\mathbb{T}(V)_{(ty,n)} \to Term_{U(\mathbb{T})}(U(V))_{(tm,n)}$ as follows:
\begin{align*}
U(ft^j(e_p(x))) & = x^j \\
U(ft^{j+1}(e_p(\sigma_n(\Gamma, t_1, \ldots t_k)))) & = U(ft^j(\Gamma)) \\
U(e_p(\sigma_n(\Gamma, t_1, \ldots t_k))) & = \sigma^U_n(\Gamma, t_1', \ldots t_k'),
\end{align*}
where $t_i' = U(ft^{n_i}(e_{p_i}(t_i))), \ldots U(e_{p_i}(t_i)), t_i$.

We add all axioms of $\mathbb{T}$ to $U(\mathbb{T})$ and, for every axiom $\varphi \sststile{}{V} \psi$ of $\mathbb{T}$, we add the following axiom:
\[ U(\varphi) \land \bigwedge_{x \in V} \xi_x \sststile{}{U(V)} U(\psi), \]
where $U(R(t_1, \ldots t_k))$ equals to $R(t_1, \ldots t_k)$,
$U(t_1 = t_2)$ equals to $U(e_p(t_1)) = U(e_p(t_2)) \land t_1 = t_2$,
and $\xi_x$ equals to $(e_p(x) = El(x^0)) \land \bigwedge_{1 \leq j \leq n} ft(El(x^{j-1})) = El(x^j)$.

Finally, let us show that $U$ is a functor.
Let $f : \mathbb{T} \to \mathbb{T}'$ be a morphism of algebraic dependent type theories.
Then $U(f)$ is defined in the obvious way on $U$, $El$ and symbols from $\mathbb{T}$.
Define $U(f)(\sigma^U(\ldots, x^{n_i}_i, \ldots x^0_i, x_i \ldots))$ as
\[ U(e_p(f(\sigma(x_1, \ldots x_k))))|_{\bigwedge_{1 \leq i \leq k} \xi_{x_k}}. \]
It is easy to see that $U(f)$ is a morphism of contextual theories and that $U$ preserves identity morphisms and compositions.
Thus, $U$ is a functor.
\end{example}

\begin{example}
There is a natural map $\mathbb{T} \to U(\mathbb{T})$.
We define $U^\omega(\mathbb{T})$ as the colimit of the following sequence:
\[ \mathbb{T} \to U(\mathbb{T}) \to U^2(\mathbb{T}) \to \ldots \]
Then $U^\omega(\mathbb{T})$ is the theory with a hierarchy of universes closed under constructions of $\mathbb{T}$.
\end{example}

\bibliographystyle{amsplain}
\bibliography{ref}

\end{document}
