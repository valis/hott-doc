\documentclass{amsart}

\usepackage[english,russian]{babel}
\usepackage[utf8]{inputenc}
\usepackage{amssymb}
\usepackage[all]{xy}
\usepackage{verbatim}
\usepackage{ifthen}
\usepackage{xargs}
\usepackage{bussproofs}
\usepackage{type1ec}
\usepackage{stmaryrd}
% \usepackage[T2A]{fontenc}

\providecommand\WarningsAreErrors{false}
\ifthenelse{\equal{\WarningsAreErrors}{true}}{\renewcommand{\GenericWarning}[2]{\GenericError{#1}{#2}{}{This warning has been turned into a fatal error.}}}{}

\newcommand{\newref}[4][]{
\ifthenelse{\equal{#1}{}}{\newtheorem{h#2}[hthm]{#4}}{\newtheorem{h#2}{#4}[#1]}
\expandafter\newcommand\csname r#2\endcsname[1]{\ref{#2:##1}}
\expandafter\newcommand\csname R#2\endcsname[1]{#4~\ref{#2:##1}}
\newenvironmentx{#2}[2][1=,2=]{
\ifthenelse{\equal{##2}{}}{\begin{h#2}}{\begin{h#2}[##2]}
\ifthenelse{\equal{##1}{}}{}{\label{#2:##1}}
}{\end{h#2}}
}

\newref[section]{thm}{теорема}{Теорема}
\newref{lem}{лемма}{Лемма}
\newref{prop}{утверждение}{Утверждение}
\newref{cor}{следствие}{Следствие}

\theoremstyle{definition}
\newref{defn}{definition}{Definition}
\newref{example}{example}{Example}

\theoremstyle{remark}
\newref{remark}{замечание}{Замечание}

\newcommand{\red}{\Rightarrow}
\newcommand{\deq}{\Leftrightarrow}
\renewcommand{\ll}{\llbracket}
\newcommand{\rr}{\rrbracket}
\newcommand{\cat}[1]{\mathbf{#1}}
\renewcommand{\C}{\cat{C}}

\numberwithin{figure}{section}

\newcommand{\pb}[1][dr]{\save*!/#1-1.2pc/#1:(-1,1)@^{|-}\restore}
\newcommand{\po}[1][dr]{\save*!/#1+1.2pc/#1:(1,-1)@^{|-}\restore}

\begin{document}

\title{Абстрактные теории типов}

\author{Валерий Исаев}

% \begin{abstract}
% Abstract
% \end{abstract}

\maketitle

\section{Абстрактные теории типов}

\begin{defn}
\emph{Абстрактная теория типов} - это финитарная монада над категорией контекстуальных категорий.
\end{defn}

Во-первых, мы опишем общую схему построения ряда примеров абстрактных теорий типов.
Пусть $S$ - некоторое множество симовлов, каждому их которых сопоставлена конечная последовательность конечных множеств (его арность).
Эта функция будет обозначаться $ar : S \to FinSet*$.

Пусть $\C$ - контекстуальная категория.
Тогда для любого множества $V$ определим класс термов $Term(V)$ индуктивным образом:
\begin{itemize}
\item Если $x \in V$, то $v_x \in Term(V)$.
\item Если $A \in Ob(\C)_{n + 1}$ и $a_1, \ldots a_n \in Term(V)$ то $O(A, a_1, \ldots a_n) \in Term(V)$.
\item Если $A \in Ob(\C)_n$, $f : A \to B$ и $a_1, \ldots a_n \in Term(V)$, то $f(a_1, \ldots a_n) \in Term(V)$.
\item Если $s \in S$, $ar(s) = (X_1, \ldots X_n)$, $a_1 \in Term(V \amalg X_1), \ldots a_n \in Term(V \amalg X_n)$, то $s(a_1, \ldots a_n) \in Term(V)$.
\end{itemize}

Означивание - это фукнция вида $\rho : U \to Term(V)$.
Если $\rho$ - означивание и $X$ - множество, то означивание $\rho \amalg X : U \amalg X \to Term(V \amalg X)$ определяется как $(\rho \amalg X)(u) = Term(i)(\rho(u))$, если $u \in U$, где $i$ - вложение $V \to V \amalg X$, и $(\rho \amalg X)(x) = v_x$, если $x \in X$.

На множествах термов задается операция подстановки.
Если $a \in Term(U)$ и $\rho : U \to Term(V)$, то мы определяем терм $a[\rho] \in Term(V)$ рекурсией по $a$:
\begin{align*}
& v_x[\rho] = \rho(x) \\
& O(A, a_1, \ldots a_n)[\rho] = O(A, a_1[\rho], \ldots a_n[\rho]) \\
& f(a_1, \ldots a_n)[\rho] = f(a_1[\rho], \ldots a_n[\rho]) \\
& s(a_1, \ldots a_n)[\rho] = s(a_1[\rho \amalg X_1], \ldots a_n[\rho \amalg X_n]) \text{, если } ar(s) = (X_1, \ldots X_n)
\end{align*}

Класс $Ctx_n$ контекстов длины $n$ определяется индуктивно:
\begin{itemize}
\item $\diamond \in Ctx_0$.
\item Если $\Gamma \in Ctx_n$, $A \in Term(\{ 0, \ldots n - 1 \})$, то $\Gamma, A \in Ctx_{n + 1}$.
\end{itemize}

Теперь определим отношения типизации $- \vdash$, $- \vdash -$, $- \vdash - : -$, $- \vdash - \deq -$ и $- \vdash - \deq - : -$ на классах $Ctx$, $Ctx \times Type$, $Ctx \times Term \times Type$, $Ctx \times Type \times Type$ и $Ctx \times Term \times Term \times Type$ соответственно.

\medskip
\begin{center}
\AxiomC{}
\UnaryInfC{$\diamond \vdash$}
\DisplayProof
\quad
\AxiomC{$\Gamma \vdash A$}
\RightLabel{, $x \notin \Gamma$}
\UnaryInfC{$\Gamma, x : A \vdash$}
\DisplayProof
\end{center}

\medskip
\begin{center}
\AxiomC{$\Gamma \vdash A$}
\RightLabel{, $x \notin \Gamma$}
\UnaryInfC{$\Gamma, x : A \vdash x : A$}
\DisplayProof
\quad
\AxiomC{$\Gamma \vdash x : A$}
\AxiomC{$\Gamma \vdash B$}
\RightLabel{, $y \notin \Gamma$}
\BinaryInfC{$\Gamma, y : B \vdash x : A$}
\DisplayProof
\end{center}

\medskip
\begin{center}
\AxiomC{$\Gamma \vdash$}
\AxiomC{$\Gamma \vdash a_1 : O(ft^n(A))\ \ldots\ \Gamma \vdash a_n : O(ft(A), a_1, \ldots a_{n-1})$}
\RightLabel{, $A \in Ob(\C)_{n+1}$}
\BinaryInfC{$\Gamma \vdash O(A, a_1, \ldots a_n)$}
\DisplayProof
\end{center}

\medskip
\begin{center}
\AxiomC{$\Gamma \vdash$}
\AxiomC{$\Gamma \vdash a_1 : O(ft^{n-1}(A))\ \ldots\ \Gamma \vdash a_n : O(A, a_1, \ldots a_{n-1})$}
\RightLabel{, $A \in Ob(\C)_n$, $B \in Ob(\C)_{m+1}$, $f : A \to B$}
\BinaryInfC{$\Gamma \vdash f(a_1, \ldots a_n) : O(B, \pi^m_B f(a_1, \ldots a_n), \ldots \pi_B f(a_1, \ldots a_n))$}
\DisplayProof
\end{center}

\medskip
\begin{center}
\AxiomC{$\Gamma \vdash$}
\AxiomC{$\Gamma \vdash a_1 : O(ft^{n-1}(A))\ \ldots\ \Gamma \vdash a_n : O(A, a_1, \ldots a_{n-1})$}
\RightLabel{, $A \in Ob(\C)_n$, $B \in Ob(\C)_{m+1}$, $C \in Ob(\C)_{k+1}$, $f : A \to B$, $g : B \to C$}
\BinaryInfC{$\Gamma \vdash g f(a_1, \ldots a_n) \deq g(\pi^m_B f(a_1, \ldots a_n), \ldots f(a_1, \ldots a_n)) : O(C, \pi^k_C g f(a_1, \ldots a_n), \ldots \pi_C g f(a_1, \ldots a_n))$}
\DisplayProof
\end{center}
\medskip

К этим правилам также добавляются правила для символов из $S$.

Теперь мы можем определить контекстуальную категорию $T_S(\C)$.
Объекты этой категории - это контексты.
Для каждой пары контекстов $\Gamma$ и $\Delta = A_1, \ldots A_n$ мы определяем множество морфизмов $T_S(\C)(\Gamma, \Delta)$ как множество классов эквивалентностей множества последовательностей $(a_1, \ldots a_n)$ термов таких, что $\Gamma \vdash a_i : A_i[j \mapsto a_j]$.
Две такие последовательности $(a_1, \ldots a_n)$ и $(a_1', \ldots a_n')$ эквивалентны, если $\Gamma \vdash a_i \deq a_i' : A_i[j \mapsto a_j]$.

\section{Обобщенные алгебраические теории}

Большой класс примером абстрактных теорий типов строится при помощи обобщенных алгебраических теорий, которые были введены в \cite{GAT}.
Мы будем использовать определенный частный случай таких теорий и слегка модифицируем определение.

Обобщенная алгебраическая теория состоит из счетного множества переменных $Var$, множества символов $S$ и множества правил вывода, синтаксис описания которых приведен ниже.
По множествам $Var$ и $S$ мы строим множество выражений $Expr_S$ индуктивным образом:
\begin{itemize}
\item Если $v \in Var$, то $v \in Expr_S$.
\item Если $s \in S \cup \{ Ctx, Ty, Tm \}$, $e_1, \ldots e_n \in Expr_S$, то $s(e_1, \ldots e_n) \in Expr_S$.
\end{itemize}

Правило вывода состоит из последовательности посылок $P_1. \ldots P_n$ и заключения $C$.
Такое правило записывается как $P_1, \ldots P_n \implies C$ или как
\medskip
\begin{center}
\AxiomC{$P_1$}
\AxiomC{$\ldots$}
\AxiomC{$P_n$}
\TrinaryInfC{$C$}
\DisplayProof
\end{center}

Каждая посылка является парой из переменной $v$ и выражение $e$ и записывается как $v \in e$.
Заключения имеют один из следующих видов:
\begin{itemize}
\item Заключения, вводящие типы. Такие заключения состоят из пары выраженией $e$ и $\Gamma$ и записываются как $e \in Ty(\Gamma)$.
\item Заключения, вводящие термы. Такие заключения состоят из троек выраженией $e$, $\Gamma$ и $A$ и записываются как $e \in Tm(\Gamma, A)$.
\item Заключения, вводящие равенства типов. Такие заключения состоят из троек выраженией $e_1$, $e_2$ и $\Gamma$ и записываются как $e_1 = e_2 \in Ty(\Gamma)$.
\item Заключения, вводящие равенства термов. Такие заключения состоят из четверок выраженией $e_1$, $e_2$, $\Gamma$ и $A$ и записываются как $e_1 = e_2 \in Tm(\Gamma, A)$.
\end{itemize}

\bibliographystyle{amsplain}
\bibliography{ref}

\end{document}
