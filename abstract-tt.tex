\documentclass{amsart}

\usepackage[english,russian]{babel}
\usepackage[utf8]{inputenc}
\usepackage{amssymb}
\usepackage[all]{xy}
\usepackage{verbatim}
\usepackage{ifthen}
\usepackage{xargs}
\usepackage{bussproofs}
\usepackage{type1ec}
\usepackage{stmaryrd}
% \usepackage[T2A]{fontenc}

\providecommand\WarningsAreErrors{false}
\ifthenelse{\equal{\WarningsAreErrors}{true}}{\renewcommand{\GenericWarning}[2]{\GenericError{#1}{#2}{}{This warning has been turned into a fatal error.}}}{}

\newcommand{\newref}[4][]{
\ifthenelse{\equal{#1}{}}{\newtheorem{h#2}[hthm]{#4}}{\newtheorem{h#2}{#4}[#1]}
\expandafter\newcommand\csname r#2\endcsname[1]{\ref{#2:##1}}
\expandafter\newcommand\csname R#2\endcsname[1]{#4~\ref{#2:##1}}
\newenvironmentx{#2}[2][1=,2=]{
\ifthenelse{\equal{##2}{}}{\begin{h#2}}{\begin{h#2}[##2]}
\ifthenelse{\equal{##1}{}}{}{\label{#2:##1}}
}{\end{h#2}}
}

\newref[section]{thm}{теорема}{Теорема}
\newref{lem}{лемма}{Лемма}
\newref{prop}{утверждение}{Утверждение}
\newref{cor}{следствие}{Следствие}

\theoremstyle{definition}
\newref{defn}{definition}{Definition}
\newref{example}{example}{Example}

\theoremstyle{remark}
\newref{remark}{замечание}{Замечание}

\newcommand{\red}{\Rightarrow}
\newcommand{\deq}{\Leftrightarrow}
\renewcommand{\ll}{\llbracket}
\newcommand{\rr}{\rrbracket}
\newcommand{\cat}[1]{\mathbf{#1}}
\renewcommand{\C}{\cat{C}}

\numberwithin{figure}{section}

\newcommand{\pb}[1][dr]{\save*!/#1-1.2pc/#1:(-1,1)@^{|-}\restore}
\newcommand{\po}[1][dr]{\save*!/#1+1.2pc/#1:(1,-1)@^{|-}\restore}

\begin{document}

\title{Абстрактные теории типов}

\author{Валерий Исаев}

% \begin{abstract}
% Abstract
% \end{abstract}

\maketitle

\section{Абстрактные теории типов}

Асбтрактная теория типов $T = (\mathcal{S}_T, a_{\mathcal{S}_T}, \mathcal{R}_T, a_{\mathcal{R}_T}, \mathcal{I}_T)$ будет состоять из множества функциональных символов $\mathcal{S}_T$, множества предикатных символов $\mathcal{R}_T$, функций $a_{\mathcal{S}_T}$ и $a_{\mathcal{R}_T}$ и множества правил вывода $\mathcal{I}_T$.
Функция $a_{\mathcal{R}_T}$ сопоставляет каждому $R \in \mathcal{R}_T$ его арность $a_{\mathcal{R}_T}(R) \in \mathbb{N}$.
Функция $a_{\mathcal{S}_T}$ сопоставляет каждому $f \in \mathcal{S}_T$ его арность $a_{\mathcal{S}_T}(f) \in \mathbb{N}^*$, то есть $a_{\mathcal{S}_T}(f)$ - конечная последовательность натуральных чисел.
Определение правил вывода будет приведено ниже.

Для определения правил вывода мы будем использовать формализм логики первого порядка.
Во-первых, определим сигнатуру логики первого порядка $\Sigma_{\mathcal{S}, \mathcal{R}}$.
Множество функциональных символов этой сигнатуры состоит из $\mathcal{S} \amalg \{ subst \}$.
Арность $f \in \mathcal{S}$ равна длине $a_\mathcal{S}(f)$ и арность $subst$ равна $2$.
Терм $subst(a, b)$ мы будем записывать как $a[b]$.
Пусть $Term_{\mathcal{S}}$ - множество термов этой сигнатуры.
Множество предикатных символов сигнатуры состоит из $\mathcal{R} \amalg \{ Ty_n\ |\ n \in \mathbb{N} \} \amalg \{ Tm_n\ |\ n \in \mathbb{N} \}$.
Арность $R \in \mathcal{R}$ равна $a_\mathcal{R}(R)$, арность $Ty_n$ равна $n + 1$ и арность $Tm_n$ равна $n + 2$.

Теперь мы можем дать определение правила вывода.
Правило вывода теории $T$ - это замкнутая формула в сигнатуре $\Sigma_{\mathcal{S}_T, \mathcal{R}_T}$ вида $\forall x_1 \ldots \forall x_l (P_1 \land \ldots \land P_n \to C)$, где $P_1$, \ldots $P_n$ и $C$ - атомарные формулы.
Формулы $P_i$ называются посылками правила, а формула $C$ - его заключением.

Пусть $Th$ - произвольная теория сигнатуры $\Sigma_{\mathcal{S}_T, \mathcal{R}_T}$.
Мы будем говорить, что посылка или заключение $P_i$ правила вывода $\forall x_1 \ldots \forall x_l (P_1 \land \ldots \land P_n \to P_{n+1})$ корректно в $Th$, если выполнено одно из следующих условий:
\begin{itemize}
\item $P_i = R(a_1, \ldots a_k)$ для некоторого $R \in \mathcal{R}_T$ и некоторых термов $a_1$, \ldots $a_k$.
\item $P_i = Tm_k(A_1, \ldots A_k, a, A_{k + 1})$ для некоторого $k \in \mathbb{N}$ и некоторых термов $A_1$, \ldots $A_{m + 1}$ и $a$ таких, что для любого $1 \leq j \leq m + 1$ в $Th$ выводима формула $P_1 \land \ldots \land P_{i - 1} \to Ty_{j - 1}(A_1, \ldots A_j)$.
    Кроме того, если $i \leq n$, то $a$ должен быть переменной, и если $i = n + 1$, то либо $a$ является переменной, либо $a = f(b_1, \ldots b_m)$ для некоторого $f \in \mathcal{S}_T$ такого, что $a_{\mathcal{S}_T}(f) = (s_1, \ldots s_m)$ и для всех $1 \leq j \leq m$ в $Th$ выводима либо формула $P_1 \land \ldots \land P_n \to Ty_{s_j}(d_1, \ldots d_{s_j}, b_j)$, либо формула $P_1 \land \ldots \land P_n \to Tm_{s_j}(d_1, \ldots d_{s_j}, b_j, d_{s_j + 1})$ для некоторых $d_1$, \ldots $d_{s_j + 1}$.
\item $P_i = Ty_k(A_1, \ldots A_k, a)$ для некоторого $k \in \mathbb{N}$ и некоторых термов $A_1$, \ldots $A_k$ и $a$ для которых выполнены условия из предыдущего пункта.
\end{itemize}

Мы будем говорить, что правило вывода корректно в $Th$, если все посылки и заключение корректно в $Th$.
Мы будем говорить, что множество правил вывода $\mathcal{I}$ корректно, если каждая правило вывода из $\mathcal{I}$ корректно в $Th_0 \cup \mathcal{I}$, где теория $Th_0$ состоит из следующих двух формул:
\begin{align*}
& \forall a \forall A \forall B (Ty_0(A) \land Tm_0(a, A) \land Ty_1(A, B) \to Ty_0(B[a])) \\
& \forall a \forall A \forall b \forall B (Ty_0(A) \land Tm_0(a, A) \land Tm_1(A, b, B) \to Tm_0(b[a], B[a]))
\end{align*}

\begin{defn}
Асбтрактная теория типов $T = (\mathcal{S}_T, a_{\mathcal{S}_T}, \mathcal{R}_T, a_{\mathcal{R}_T}, \mathcal{I}_T)$ состоит из множества функциональных символов $\mathcal{S}_T$, множества предикатных символов $\mathcal{R}_T$, их арностей $a_{\mathcal{S}_T}$ и $a_{\mathcal{R}_T}$ и корректного множества правил вывода $\mathcal{I}_T$.
\end{defn}

\section{Абстрактные теории типов как монады}

\begin{defn}
\emph{Абстрактная теория типов} - это финитарная монада над категорией контекстуальных категорий.
\end{defn}

Покажем как по обобщенной алгебраической теории построить абстрактную теорию типов.
Пусть $\C$ - контекстуальная категория и $S$ - множество симовлов теории.
Тогда определим класс термов $Term$ индуктивным образом:
\begin{itemize}
\item Если $A \in Ob(\C)_{n + 1}$ и $a_1, \ldots a_n \in Term$ то $O(A, a_1, \ldots a_n) \in Term$.
\item Если $A \in Ob(\C)_n$, $f : A \to B$ и $a_1, \ldots a_n \in Term$, то $f(a_1, \ldots a_n) \in Term$.
\item Если $s \in S$, $a_1, \ldots a_n \in Term$, то $s(a_1, \ldots a_n) \in Term$.
\end{itemize}

Класс контекстов определяется следующим образом:
\[ Ctx = \{ x_1 : A_1, \ldots x_n : A_n\ |\ x_1, \ldots x_n \in Var, A_1, \ldots A_n \in Term \}. \]
Пустой контекст обозначается $\diamond$.

Теперь определим отношения типизации $- \vdash$, $- \vdash -$ и $- \vdash - : -$ на классах $Ctx$, $Ctx \times Term$ и $Ctx \times Term \times Term$ соответственно.

\medskip
\begin{center}
\AxiomC{}
\UnaryInfC{$\diamond \vdash$}
\DisplayProof
\quad
\AxiomC{$\Gamma \vdash A$}
\RightLabel{, $x \notin \Gamma$}
\UnaryInfC{$\Gamma, x : A \vdash$}
\DisplayProof
\end{center}

\medskip
\begin{center}
\AxiomC{$\Gamma \vdash A$}
\RightLabel{, $x \notin \Gamma$}
\UnaryInfC{$\Gamma, x : A \vdash x : A$}
\DisplayProof
\quad
\AxiomC{$\Gamma \vdash x : A$}
\AxiomC{$\Gamma \vdash B$}
\RightLabel{, $y \notin \Gamma$}
\BinaryInfC{$\Gamma, y : B \vdash x : A$}
\DisplayProof
\end{center}

\medskip
\begin{center}
\AxiomC{$\Gamma \vdash$}
\AxiomC{$\Gamma \vdash a_1 : O(ft^n(A))\ \ldots\ \Gamma \vdash a_n : O(ft(A), a_1, \ldots a_{n-1})$}
\RightLabel{, $A \in Ob(\C)_{n+1}$}
\BinaryInfC{$\Gamma \vdash O(A, a_1, \ldots a_n)$}
\DisplayProof
\end{center}

\medskip
\begin{center}
\AxiomC{$\Gamma \vdash$}
\AxiomC{$\Gamma \vdash a_1 : O(ft^n(A))\ \ldots\ \Gamma \vdash a_n : O(ft(A), a_1, \ldots a_{n-1})$}
\RightLabel{, $f : A \to B$, $B \in Ob(\C)_{m+1}$}
\BinaryInfC{$\Gamma \vdash f(a_1, \ldots a_n) : O(B, \pi^m_B f(a_1, \ldots a_n), \ldots \pi_B f(a_1, \ldots a_n))$}
\DisplayProof
\end{center}

\medskip
\begin{center}
\AxiomC{$\Gamma \vdash$}
\UnaryInfC{$\Gamma \vdash \top$}
\DisplayProof
\quad
\AxiomC{$\Gamma \vdash$}
\UnaryInfC{$\Gamma \vdash tt : \top$}
\DisplayProof
\quad
\AxiomC{$\Gamma \vdash a : \top$}
\UnaryInfC{$\Gamma \vdash a \deq tt : \top$}
\DisplayProof
\end{center}

\section{Обобщенные алгебраические теории}

Большой класс примером абстрактных теорий типов строится при помощи обобщенных алгебраических теорий, которые были введены в \cite{GAT}.
Мы будем использовать определенный частный случай таких теорий и слегка модифицируем определение.

Обобщенная алгебраическая теория состоит из счетного множества переменных $Var$, множества символов $S$ и множества правил вывода, синтаксис описания которых приведен ниже.
По множествам $Var$ и $S$ мы строим множество выражений $Expr_S$ индуктивным образом:
\begin{itemize}
\item Если $v \in Var$, то $v \in Expr_S$.
\item Если $s \in S \cup \{ Ctx, Ty, Tm \}$, $e_1, \ldots e_n \in Expr_S$, то $s(e_1, \ldots e_n) \in Expr_S$.
\end{itemize}

Правило вывода состоит из последовательности посылок $P_1. \ldots P_n$ и заключения $C$.
Такое правило записывается как $P_1, \ldots P_n \implies C$ или как
\medskip
\begin{center}
\AxiomC{$P_1$}
\AxiomC{$\ldots$}
\AxiomC{$P_n$}
\TrinaryInfC{$C$}
\DisplayProof
\end{center}

Каждая посылка является парой из переменной $v$ и выражение $e$ и записывается как $v \in e$.
Заключения имеют один из следующих видов:
\begin{itemize}
\item Заключения, вводящие типы. Такие заключения состоят из пары выраженией $e$ и $\Gamma$ и записываются как $e \in Ty(\Gamma)$.
\item Заключения, вводящие термы. Такие заключения состоят из троек выраженией $e$, $\Gamma$ и $A$ и записываются как $e \in Tm(\Gamma, A)$.
\item Заключения, вводящие равенства типов. Такие заключения состоят из троек выраженией $e_1$, $e_2$ и $\Gamma$ и записываются как $e_1 = e_2 \in Ty(\Gamma)$.
\item Заключения, вводящие равенства термов. Такие заключения состоят из четверок выраженией $e_1$, $e_2$, $\Gamma$ и $A$ и записываются как $e_1 = e_2 \in Tm(\Gamma, A)$.
\end{itemize}

\bibliographystyle{amsplain}
\bibliography{ref}

\end{document}
