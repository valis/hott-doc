\documentclass{amsart}

\usepackage[english,russian]{babel}
\usepackage[utf8]{inputenc}
\usepackage{amssymb}
\usepackage[all]{xy}
\usepackage{verbatim}
\usepackage{ifthen}
\usepackage{xargs}
\usepackage{bussproofs}
\usepackage{type1ec}
\usepackage{stmaryrd}
% \usepackage[T2A]{fontenc}

\providecommand\WarningsAreErrors{false}
\ifthenelse{\equal{\WarningsAreErrors}{true}}{\renewcommand{\GenericWarning}[2]{\GenericError{#1}{#2}{}{This warning has been turned into a fatal error.}}}{}

\newcommand{\newref}[4][]{
\ifthenelse{\equal{#1}{}}{\newtheorem{h#2}[hthm]{#4}}{\newtheorem{h#2}{#4}[#1]}
\expandafter\newcommand\csname r#2\endcsname[1]{\ref{#2:##1}}
\expandafter\newcommand\csname R#2\endcsname[1]{#4~\ref{#2:##1}}
\newenvironmentx{#2}[2][1=,2=]{
\ifthenelse{\equal{##2}{}}{\begin{h#2}}{\begin{h#2}[##2]}
\ifthenelse{\equal{##1}{}}{}{\label{#2:##1}}
}{\end{h#2}}
}

\newref[section]{thm}{теорема}{Теорема}
\newref{lem}{лемма}{Лемма}
\newref{prop}{утверждение}{Утверждение}
\newref{cor}{следствие}{Следствие}

\theoremstyle{definition}
\newref{defn}{definition}{Definition}
\newref{example}{example}{Example}

\theoremstyle{remark}
\newref{remark}{замечание}{Замечание}

\newcommand{\red}{\Rightarrow}
\newcommand{\deq}{\Leftrightarrow}
\renewcommand{\ll}{\llbracket}
\newcommand{\rr}{\rrbracket}
\newcommand{\cat}[1]{\mathbf{#1}}
\renewcommand{\C}{\cat{C}}

\numberwithin{figure}{section}

\newcommand{\pb}[1][dr]{\save*!/#1-1.2pc/#1:(-1,1)@^{|-}\restore}
\newcommand{\po}[1][dr]{\save*!/#1+1.2pc/#1:(1,-1)@^{|-}\restore}

\begin{document}

\title{Абстрактные теории типов}

\author{Валерий Исаев}

% \begin{abstract}
% Abstract
% \end{abstract}

\maketitle

\section{Абстрактные теории типов}

Асбтрактная теория типов $T = (\mathcal{S}_T, a_{\mathcal{S}_T}, \mathcal{R}_T, a_{\mathcal{R}_T}, \mathcal{I}_T)$ будет состоять из множества функциональных символов $\mathcal{S}_T$, множества предикатных символов $\mathcal{R}_T$, функций $a_{\mathcal{S}_T}$ и $a_{\mathcal{R}_T}$ и множества правил вывода $\mathcal{I}_T$.
Функция $a_{\mathcal{R}_T}$ сопоставляет каждому $R \in \mathcal{R}_T$ его арность $a_{\mathcal{R}_T}(R) \in \mathbb{N}$.
Функция $a_{\mathcal{S}_T}$ сопоставляет каждому $f \in \mathcal{S}_T$ его арность $a_{\mathcal{S}_T}(f) \in \mathbb{N}^*$, то есть $a_{\mathcal{S}_T}(f)$ - конечная последовательность натуральных чисел.
Определение правил вывода будет приведено ниже.

Для определения правил вывода мы будем использовать формализм логики первого порядка.
Во-первых, определим сигнатуру логики первого порядка $\Sigma_{\mathcal{S}, \mathcal{R}}$.
Множество функциональных символов этой сигнатуры состоит из $\mathcal{S} \amalg \{ subst_n\ |\ n \in \mathbb{N}, n \geq 1 \} \amalg \{ v_n\ |\ n \in \mathbb{N} \} \amalg \{ lift_{n,k}\ |\ n, k \in \mathbb{N} \} $.
Арность $f \in \mathcal{S}$ равна длине $a_\mathcal{S}(f)$, арность $subst_n$ равна $n + 1$, арность $v_n$ равна $0$ и арность $lift_{n,k}$ равна $1$.
Терм $subst_n(b, a_1, \ldots a_n)$ мы будем записывать как $b[a_1, \ldots a_n]$.
Терм $lift_{n,k}(b)$ мы будем записывать как $\Uparrow^n_k b$, и вместо $\Uparrow^n_0 b$ мы будем писать просто $\Uparrow^n b$.
Пусть $Term_{\mathcal{S}}$ - множество термов этой сигнатуры.
Множество предикатных символов сигнатуры состоит из $\{ R_n\ |\ R \in \mathcal{R} \amalg \{ Ty, Tm \}, n \in \mathbb{N} \}$.
Арность $R_n$ равна $\left\{ \begin{array}{c l}
                                n + a_\mathcal{R}(R), & \text{ если } R \in \mathcal{R} \\
                                n + 1, & \text{ если } R = Ty \\
                                n + 2, & \text{ если } R = Tm
                             \end{array}\right.$.

Теперь мы можем дать определение правила вывода.
Правило вывода теории $T$ - это замкнутая формула в сигнатуре $\Sigma_{\mathcal{S}_T, \mathcal{R}_T}$ вида $\forall x_1 \ldots \forall x_l (P_1 \land \ldots \land P_n \to C)$, где $P_1$, \ldots $P_n$ и $C$ - атомарные формулы.
Формулы $P_i$ называются посылками правила, а формула $C$ - его заключением.

\begin{comment}
Пусть $Th$ - произвольная теория сигнатуры $\Sigma_{\mathcal{S}_T, \mathcal{R}_T}$.
Мы будем говорить, что посылка или заключение $P_i$ правила вывода $\forall x_1 \ldots \forall x_l (P_1 \land \ldots \land P_n \to P_{n+1})$ корректно в $Th$, если выполнено одно из следующих условий:
\begin{itemize}
\item $P_i = R(a_1, \ldots a_k)$ для некоторого $R \in \mathcal{R}_T$ и некоторых термов $a_1$, \ldots $a_k$.
\item $P_i = Tm_k(A_1, \ldots A_k, a, A_{k + 1})$ для некоторого $k \in \mathbb{N}$ и некоторых термов $A_1$, \ldots $A_{m + 1}$ и $a$ таких, что для любого $1 \leq j \leq m + 1$ в $Th$ выводима формула $P_1 \land \ldots \land P_{i - 1} \to Ty_{j - 1}(A_1, \ldots A_j)$.
    Кроме того, если $i \leq n$, то $a$ должен быть переменной, и если $i = n + 1$, то либо $a$ является переменной, либо $a = f(b_1, \ldots b_m)$ для некоторого $f \in \mathcal{S}_T$ такого, что $a_{\mathcal{S}_T}(f) = (s_1, \ldots s_m)$ и для всех $1 \leq j \leq m$ в $Th$ выводима либо формула $P_1 \land \ldots \land P_n \to Ty_{s_j}(d_1, \ldots d_{s_j}, b_j)$, либо формула $P_1 \land \ldots \land P_n \to Tm_{s_j}(d_1, \ldots d_{s_j}, b_j, d_{s_j + 1})$ для некоторых $d_1$, \ldots $d_{s_j + 1}$.
\item $P_i = Ty_k(A_1, \ldots A_k, a)$ для некоторого $k \in \mathbb{N}$ и некоторых термов $A_1$, \ldots $A_k$ и $a$ для которых выполнены условия из предыдущего пункта.
\end{itemize}

Мы будем говорить, что правило вывода корректно в $Th$, если все посылки и заключение корректно в $Th$.
Мы будем говорить, что множество правил вывода $\mathcal{I}$ корректно, если каждая правило вывода из $\mathcal{I}$ корректно в $Th_0 \cup \mathcal{I}$, где теория $Th_0$ состоит из следующих формул:
\end{comment}

\begin{defn}
    \emph{Асбтрактная теория типов} (АТТ) $T = (\mathcal{S}_T, a_{\mathcal{S}_T}, \mathcal{R}_T, a_{\mathcal{R}_T}, \mathcal{I}_T)$ состоит из множества функциональных символов $\mathcal{S}_T$, множества предикатных символов $\mathcal{R}_T$, их арностей $a_{\mathcal{S}_T}$ и $a_{\mathcal{R}_T}$ и корректного множества правил вывода $\mathcal{I}_T$.
\end{defn}

Пусть теория $Th^0$ состоит из следующего набора формул:
\begin{itemize}
\item $\Uparrow^n_k v_i = \left\{\begin{array}{c l}
                                    v_i, & \text{ если } i < k \\
                                    v_{i + n}, & \text{ если } i \geq k
                                 \end{array}\right.$
\item Для всех $f \in \mathcal{S}_T$ таких, что $a_{\mathcal{S}_T}(f) = (s_1, \ldots s_m)$, формула \\
    $\forall a_1 \ldots \forall a_m (\Uparrow^n_k f(a_1, \ldots a_m) = f(\Uparrow^n_{k + s_1} a_1, \ldots \Uparrow^n_{k + s_m} a_m))$
\item $v_i[a_1, \ldots a_n] = \left\{\begin{array}{c l}
                                    a_{n - i}, & \text{ если } i < n \\
                                    v_{i - n}, & \text{ если } i \geq n
                                 \end{array}\right.$
\item Для всех $f \in \mathcal{S}_T$ таких, что $a_{\mathcal{S}_T}(f) = (s_1, \ldots s_m)$, формула $\forall a_1 \ldots \forall a_n \forall b_1 \ldots \forall b_m$ $(f(b_1, \ldots b_m)[a_1, \ldots a_n] = f(b_1[\Uparrow^{s_1} a_1, \ldots \Uparrow^{s_1} a_n], \ldots b_m[\Uparrow^{s_m} a_1, \ldots \Uparrow^{s_m} a_n]))$
\item Для всех $0 \leq i < m$ формула $\forall C_1 \ldots \forall C_m (Ty_{m - 1}(C_1, \ldots C_m) \to Tm_m(C_1, \ldots C_m, v_i, \Uparrow^{i + 1} C_{m - i}))$
\item $\forall C_1 \ldots \forall C_m \forall a_0 \ldots \forall a_n \forall A_0 \ldots \forall A_n \forall B (Tm_m(C_1, \ldots C_m, a_0, A_0) \land \ldots \land Tm_m(C_1, \ldots C_m, a_n, A_n[a_0, \ldots a_{n-1}]) \land Ty_n(A_1, \ldots A_n, B) \to Ty_m(C_1, \ldots C_m, B[a_0, \ldots a_n]))$
\item $\forall C_1 \ldots \forall C_m \forall a_0 \ldots \forall a_n \forall A_0 \ldots \forall A_n \forall b \forall B (Tm_m(C_1, \ldots C_m, a_0, A_0) \land \ldots \land Tm_m(C_1, \ldots C_m, a_n, A_n[a_0, \ldots a_{n-1}]) \land Tm_n(A_1, \ldots A_n, b, B) \to Tm_m(C_1, \ldots C_m, b[a_0, \ldots a_n], B[a_0, \ldots a_n]))$
\end{itemize}
Тогда определим теорию $Th^0_T$ как объединение $Th^0$ и $\mathcal{I}_T$.

\begin{defn}
\emph{Модель абстрактной теории типов} $T$ - это модель теории $Th^0_T$.
\end{defn}

Модели $T$ образуют категорию, это просто категория моделей $Th^0_T$.
Мы будем обозначать эту категорию $Mod(T)$.

\subsection{Теории с равенством}

Абстрактная теория типов с равенством - это абстрактная теория типов с выбранными предикатными символами $eq$ и $Eq$ арности $3$ и $2$ соответственно.
Определим теорию $Th$ как объединение $Th^0$ и следующего набора формул:
\begin{itemize}
\item $\forall C_1 \ldots \forall C_m \forall a \forall a' \forall A (eq_m(C_1, \ldots C_m, a, a', A) \to a = a')$
\item $\forall C_1 \ldots \forall C_m \forall A \forall A' (Eq_m(C_1, \ldots C_m, A, A') \to A = A')$
\end{itemize}
Если $T$ - АТТ с равенством, то определим теорию $Th_T$ как объединение $Th$ и $\mathcal{I}_T$.

Модель АТТ с равенством - это модель теории $Th_T$.
Категория моделей $T$ будет обозначаться $Mod(T)$.
Любой АТТ $T$ можно сопоставить АТТ с равенством $T'$, добавив два новых предикатных символа $eq$ и $Eq$.
Заметим, что категории $Mod(T)$ и $Mod(T')$ могут быть различны.

\subsection{Модели теорий}

Пусть $D$ - модель теории $Th^0_T$.
Тогда построим контекстуальную категорию $U(D)$ следующим образом.
Множество объектов $Ob(U(D))_n$ определяется как следующее подмножество $D^n$:
\[ \{ (C_1, \ldots C_n) \in D^n\ |\ D \models Ty_i(C_1, \ldots C_{i + 1}) \text{ для всех } 0 \leq i < n \} \]
Множество морфизмов $Hom_{U(D)}((C_1, \ldots C_m), (B_1, \ldots B_n))$ определяется как следующее подмножество $D^n$:
\[ \{ (b_1, \ldots b_n) \in D^n\ |\ D \models Tm_m(C_1, \ldots C_m, b_{i + 1}, \Uparrow^m B_{i + 1}[b_1, \ldots b_i]) \text{ для всех } 0 \leq i < n \} \]
Тождественный морфизм $id_{(C_1, \ldots C_m)}$ определяется как $(v_{m - 1}, \ldots v_0)$.

\section{Абстрактные теории типов как монады}

\begin{defn}
\emph{Абстрактная теория типов} - это финитарная монада над категорией контекстуальных категорий.
\end{defn}

Во-первых, мы опишем общую схему построения ряда примеров абстрактных теорий типов.
Пусть $S$ - некоторое множество симовлов, каждому их которых сопоставлена конечная последовательность конечных множеств (его арность).
Эта функция будет обозначаться $ar : S \to FinSet*$.

Пусть $\C$ - контекстуальная категория.
Тогда для любого множества $V$ определим класс термов $Term(V)$ индуктивным образом:
\begin{itemize}
\item Если $x \in V$, то $v_x \in Term(V)$.
\item Если $A \in Ob(\C)_{n + 1}$ и $a_1, \ldots a_n \in Term(V)$ то $O(A, a_1, \ldots a_n) \in Term(V)$.
\item Если $A \in Ob(\C)_n$, $f : A \to B$ и $a_1, \ldots a_n \in Term(V)$, то $f(a_1, \ldots a_n) \in Term(V)$.
\item Если $s \in S$, $ar(s) = (X_1, \ldots X_n)$, $a_1 \in Term(V \amalg X_1), \ldots a_n \in Term(V \amalg X_n)$, то $s(a_1, \ldots a_n) \in Term(V)$.
\end{itemize}

Означивание - это фукнция вида $\rho : U \to Term(V)$.
Если $\rho$ - означивание и $X$ - множество, то означивание $\rho \amalg X : U \amalg X \to Term(V \amalg X)$ определяется как $(\rho \amalg X)(u) = Term(i)(\rho(u))$, если $u \in U$, где $i$ - вложение $V \to V \amalg X$, и $(\rho \amalg X)(x) = v_x$, если $x \in X$.

На множествах термов задается операция подстановки.
Если $a \in Term(U)$ и $\rho : U \to Term(V)$, то мы определяем терм $a[\rho] \in Term(V)$ рекурсией по $a$:
\begin{align*}
& v_x[\rho] = \rho(x) \\
& O(A, a_1, \ldots a_n)[\rho] = O(A, a_1[\rho], \ldots a_n[\rho]) \\
& f(a_1, \ldots a_n)[\rho] = f(a_1[\rho], \ldots a_n[\rho]) \\
& s(a_1, \ldots a_n)[\rho] = s(a_1[\rho \amalg X_1], \ldots a_n[\rho \amalg X_n]) \text{, если } ar(s) = (X_1, \ldots X_n)
\end{align*}

Класс $Ctx_n$ контекстов длины $n$ определяется индуктивно:
\begin{itemize}
\item $\diamond \in Ctx_0$.
\item Если $\Gamma \in Ctx_n$, $A \in Term(\{ 0, \ldots n - 1 \})$, то $\Gamma, A \in Ctx_{n + 1}$.
\end{itemize}

Теперь определим отношения типизации $- \vdash$, $- \vdash -$, $- \vdash - : -$, $- \vdash - \deq -$ и $- \vdash - \deq - : -$ на классах $Ctx$, $Ctx \times Type$, $Ctx \times Term \times Type$, $Ctx \times Type \times Type$ и $Ctx \times Term \times Term \times Type$ соответственно.

\medskip
\begin{center}
\AxiomC{}
\UnaryInfC{$\diamond \vdash$}
\DisplayProof
\quad
\AxiomC{$\Gamma \vdash A$}
\RightLabel{, $x \notin \Gamma$}
\UnaryInfC{$\Gamma, x : A \vdash$}
\DisplayProof
\end{center}

\medskip
\begin{center}
\AxiomC{$\Gamma \vdash A$}
\RightLabel{, $x \notin \Gamma$}
\UnaryInfC{$\Gamma, x : A \vdash x : A$}
\DisplayProof
\quad
\AxiomC{$\Gamma \vdash x : A$}
\AxiomC{$\Gamma \vdash B$}
\RightLabel{, $y \notin \Gamma$}
\BinaryInfC{$\Gamma, y : B \vdash x : A$}
\DisplayProof
\end{center}

\medskip
\begin{center}
\AxiomC{$\Gamma \vdash$}
\AxiomC{$\Gamma \vdash a_1 : O(ft^n(A))\ \ldots\ \Gamma \vdash a_n : O(ft(A), a_1, \ldots a_{n-1})$}
\RightLabel{, $A \in Ob(\C)_{n+1}$}
\BinaryInfC{$\Gamma \vdash O(A, a_1, \ldots a_n)$}
\DisplayProof
\end{center}

\medskip
\begin{center}
\AxiomC{$\Gamma \vdash$}
\AxiomC{$\Gamma \vdash a_1 : O(ft^{n-1}(A))\ \ldots\ \Gamma \vdash a_n : O(A, a_1, \ldots a_{n-1})$}
\RightLabel{, $A \in Ob(\C)_n$, $B \in Ob(\C)_{m+1}$, $f : A \to B$}
\BinaryInfC{$\Gamma \vdash f(a_1, \ldots a_n) : O(B, \pi^m_B f(a_1, \ldots a_n), \ldots \pi_B f(a_1, \ldots a_n))$}
\DisplayProof
\end{center}

\medskip
\begin{center}
\AxiomC{$\Gamma \vdash$}
\AxiomC{$\Gamma \vdash a_1 : O(ft^{n-1}(A))\ \ldots\ \Gamma \vdash a_n : O(A, a_1, \ldots a_{n-1})$}
\RightLabel{, $A \in Ob(\C)_n$, $B \in Ob(\C)_{m+1}$, $C \in Ob(\C)_{k+1}$, $f : A \to B$, $g : B \to C$}
\BinaryInfC{$\Gamma \vdash g f(a_1, \ldots a_n) \deq g(\pi^m_B f(a_1, \ldots a_n), \ldots f(a_1, \ldots a_n)) : O(C, \pi^k_C g f(a_1, \ldots a_n), \ldots \pi_C g f(a_1, \ldots a_n))$}
\DisplayProof
\end{center}
\medskip

К этим правилам также добавляются правила для символов из $S$.

Теперь мы можем определить контекстуальную категорию $T_S(\C)$.
Объекты этой категории - это контексты.
Для каждой пары контекстов $\Gamma$ и $\Delta = A_1, \ldots A_n$ мы определяем множество морфизмов $T_S(\C)(\Gamma, \Delta)$ как множество классов эквивалентностей множества последовательностей $(a_1, \ldots a_n)$ термов таких, что $\Gamma \vdash a_i : A_i[j \mapsto a_j]$.
Две такие последовательности $(a_1, \ldots a_n)$ и $(a_1', \ldots a_n')$ эквивалентны, если $\Gamma \vdash a_i \deq a_i' : A_i[j \mapsto a_j]$.

\section{Обобщенные алгебраические теории}

Большой класс примером абстрактных теорий типов строится при помощи обобщенных алгебраических теорий, которые были введены в \cite{GAT}.
Мы будем использовать определенный частный случай таких теорий и слегка модифицируем определение.

Обобщенная алгебраическая теория состоит из счетного множества переменных $Var$, множества символов $S$ и множества правил вывода, синтаксис описания которых приведен ниже.
По множествам $Var$ и $S$ мы строим множество выражений $Expr_S$ индуктивным образом:
\begin{itemize}
\item Если $v \in Var$, то $v \in Expr_S$.
\item Если $s \in S \cup \{ Ctx, Ty, Tm \}$, $e_1, \ldots e_n \in Expr_S$, то $s(e_1, \ldots e_n) \in Expr_S$.
\end{itemize}

Правило вывода состоит из последовательности посылок $P_1. \ldots P_n$ и заключения $C$.
Такое правило записывается как $P_1, \ldots P_n \implies C$ или как
\medskip
\begin{center}
\AxiomC{$P_1$}
\AxiomC{$\ldots$}
\AxiomC{$P_n$}
\TrinaryInfC{$C$}
\DisplayProof
\end{center}

Каждая посылка является парой из переменной $v$ и выражение $e$ и записывается как $v \in e$.
Заключения имеют один из следующих видов:
\begin{itemize}
\item Заключения, вводящие типы. Такие заключения состоят из пары выраженией $e$ и $\Gamma$ и записываются как $e \in Ty(\Gamma)$.
\item Заключения, вводящие термы. Такие заключения состоят из троек выраженией $e$, $\Gamma$ и $A$ и записываются как $e \in Tm(\Gamma, A)$.
\item Заключения, вводящие равенства типов. Такие заключения состоят из троек выраженией $e_1$, $e_2$ и $\Gamma$ и записываются как $e_1 = e_2 \in Ty(\Gamma)$.
\item Заключения, вводящие равенства термов. Такие заключения состоят из четверок выраженией $e_1$, $e_2$, $\Gamma$ и $A$ и записываются как $e_1 = e_2 \in Tm(\Gamma, A)$.
\end{itemize}

\bibliographystyle{amsplain}
\bibliography{ref}

\end{document}
