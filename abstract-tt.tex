\documentclass{amsart}

\usepackage[english,russian]{babel}
\usepackage[utf8]{inputenc}
\usepackage{amssymb}
\usepackage[all]{xy}
\usepackage{verbatim}
\usepackage{ifthen}
\usepackage{xargs}
\usepackage{bussproofs}
\usepackage{type1ec}
\usepackage{stmaryrd}
% \usepackage[T2A]{fontenc}

\providecommand\WarningsAreErrors{false}
\ifthenelse{\equal{\WarningsAreErrors}{true}}{\renewcommand{\GenericWarning}[2]{\GenericError{#1}{#2}{}{This warning has been turned into a fatal error.}}}{}

\newcommand{\newref}[4][]{
\ifthenelse{\equal{#1}{}}{\newtheorem{h#2}[hthm]{#4}}{\newtheorem{h#2}{#4}[#1]}
\expandafter\newcommand\csname r#2\endcsname[1]{\ref{#2:##1}}
\expandafter\newcommand\csname R#2\endcsname[1]{#4~\ref{#2:##1}}
\newenvironmentx{#2}[2][1=,2=]{
\ifthenelse{\equal{##2}{}}{\begin{h#2}}{\begin{h#2}[##2]}
\ifthenelse{\equal{##1}{}}{}{\label{#2:##1}}
}{\end{h#2}}
}

\newref[section]{thm}{теорема}{Теорема}
\newref{lem}{лемма}{Лемма}
\newref{prop}{утверждение}{Утверждение}
\newref{cor}{следствие}{Следствие}

\theoremstyle{definition}
\newref{defn}{definition}{Definition}
\newref{example}{example}{Example}

\theoremstyle{remark}
\newref{remark}{замечание}{Замечание}

\newcommand{\red}{\Rightarrow}
\newcommand{\deq}{\Leftrightarrow}
\renewcommand{\ll}{\llbracket}
\newcommand{\rr}{\rrbracket}
\newcommand{\cat}[1]{\mathbf{#1}}
\renewcommand{\C}{\cat{C}}

\numberwithin{figure}{section}

\newcommand{\pb}[1][dr]{\save*!/#1-1.2pc/#1:(-1,1)@^{|-}\restore}
\newcommand{\po}[1][dr]{\save*!/#1+1.2pc/#1:(1,-1)@^{|-}\restore}

\begin{document}

\title{Абстрактные теории типов}

\author{Валерий Исаев}

% \begin{abstract}
% Abstract
% \end{abstract}

\maketitle

\section{Абстрактные теории типов}

\begin{defn}
\emph{Абстрактная теория типов} - это финитарная монада над категорией контекстуальных категорий.
\end{defn}

Покажем как по обобщенной алгебраической теории построить абстрактную теорию типов.
Пусть $\C$ - контекстуальная категория и $S$ - множество симовлов теории.
Тогда определим класс термов $Term$ индуктивным образом:
\begin{itemize}
\item Если $A \in Ob(\C)$, то $O(A) \in Term$.
\item Если $f : A \to B$ и $t \in Term$, то $f(t) \in Term$.
\item Если $s \in S$, $t_1, \ldots t_n \in Term$, то $s(t_1, \ldots t_n) \in Term$.
\end{itemize}

Класс контекстов определяется следующим образом:
\[ Ctx = \{ x_1 : A_1, \ldots x_n : A_n\ |\ x_1, \ldots x_n \in Var, A_1, \ldots A_n \in Term \}. \]
Пустой контекст обозначается $\diamond$.
Каждому контексту $\Gamma$ мы сопоставляем объект категории $\C$ следующим образом:
\begin{itemize}
\item $| \diamond | = 1 \in Ob(\C)_0$.
\item $| \Gamma, x : O(A) | = A$.
\item $| \Gamma, x : A' | = | \Gamma |$, если $A'$ не имеет вид $O(A)$.
\end{itemize}

Теперь определим отношения типизации $- \vdash$, $- \vdash -$ и $- \vdash - : -$ на классах $Ctx$, $Ctx \times Term$ и $Ctx \times Term \times Term$ соответственно.

\medskip
\begin{center}
\AxiomC{}
\UnaryInfC{$\diamond \vdash$}
\DisplayProof
\quad
\AxiomC{$\Gamma \vdash A$}
\RightLabel{, $x \notin \Gamma$}
\UnaryInfC{$\Gamma, x : A \vdash$}
\DisplayProof
\end{center}

\medskip
\begin{center}
\AxiomC{$\Gamma \vdash A$}
\RightLabel{, $x \notin \Gamma$}
\UnaryInfC{$\Gamma, x : A \vdash x : A$}
\DisplayProof
\quad
\AxiomC{$\Gamma \vdash x : A$}
\AxiomC{$\Gamma \vdash B$}
\RightLabel{, $y \notin \Gamma$}
\BinaryInfC{$\Gamma, y : B \vdash x : A$}
\DisplayProof
\end{center}

\medskip
\begin{center}
\AxiomC{$\Gamma \vdash$}
\RightLabel{, $A \in Ob(\C)$}
\UnaryInfC{$\Gamma \vdash O(A)$}
\DisplayProof
\quad
\AxiomC{$\Gamma \vdash a : O(A)$}
\RightLabel{, $f : A \to B$}
\UnaryInfC{$\Gamma \vdash f(a) : O(B)$}
\DisplayProof
\end{center}

\medskip
\begin{center}
\AxiomC{$\Gamma \vdash$}
\UnaryInfC{$\Gamma \vdash \top$}
\DisplayProof
\quad
\AxiomC{$\Gamma \vdash$}
\UnaryInfC{$\Gamma \vdash tt : \top$}
\DisplayProof
\quad
\AxiomC{$\Gamma \vdash a : \top$}
\UnaryInfC{$\Gamma \vdash a \deq tt : \top$}
\DisplayProof
\end{center}

\section{Обобщенные алгебраические теории}

Большой класс примером абстрактных теорий типов строится при помощи обобщенных алгебраических теорий, которые были введены в \cite{GAT}.
Мы будем использовать определенный частный случай таких теорий и слегка модифицируем определение.

Обобщенная алгебраическая теория состоит из счетного множества переменных $Var$, множества символов $S$ и множества правил вывода, синтаксис описания которых приведен ниже.
По множествам $Var$ и $S$ мы строим множество выражений $Expr_S$ индуктивным образом:
\begin{itemize}
\item Если $v \in Var$, то $v \in Expr_S$.
\item Если $s \in S \cup \{ Ctx, Ty, Tm \}$, $e_1, \ldots e_n \in Expr_S$, то $s(e_1, \ldots e_n) \in Expr_S$.
\end{itemize}

Правило вывода состоит из последовательности посылок $P_1. \ldots P_n$ и заключения $C$.
Такое правило записывается как $P_1, \ldots P_n \implies C$ или как
\medskip
\begin{center}
\AxiomC{$P_1$}
\AxiomC{$\ldots$}
\AxiomC{$P_n$}
\TrinaryInfC{$C$}
\DisplayProof
\end{center}

Каждая посылка является парой из переменной $v$ и выражение $e$ и записывается как $v \in e$.
Заключения имеют один из следующих видов:
\begin{itemize}
\item Заключения, вводящие типы. Такие заключения состоят из пары выраженией $e$ и $\Gamma$ и записываются как $e \in Ty(\Gamma)$.
\item Заключения, вводящие термы. Такие заключения состоят из троек выраженией $e$, $\Gamma$ и $A$ и записываются как $e \in Tm(\Gamma, A)$.
\item Заключения, вводящие равенства типов. Такие заключения состоят из троек выраженией $e_1$, $e_2$ и $\Gamma$ и записываются как $e_1 = e_2 \in Ty(\Gamma)$.
\item Заключения, вводящие равенства термов. Такие заключения состоят из четверок выраженией $e_1$, $e_2$, $\Gamma$ и $A$ и записываются как $e_1 = e_2 \in Tm(\Gamma, A)$.
\end{itemize}

\bibliographystyle{amsplain}
\bibliography{ref}

\end{document}
