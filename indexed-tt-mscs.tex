\documentclass[reqno]{mscs}

\usepackage{amssymb}
\usepackage{hyperref}
\usepackage{mathtools}
\usepackage[all]{xy}
\usepackage{verbatim}
\usepackage{ifthen}
\usepackage{xargs}
\usepackage{bussproofs}
\usepackage{turnstile}
\usepackage{etex}

\hypersetup{colorlinks=true,linkcolor=blue}

\newcommand{\axlabel}[1]{(#1) \phantomsection \label{ax:#1}}
\newcommand{\axtag}[1]{\label{ax:#1} \tag{#1}}
\newcommand{\axref}[1]{(\hyperref[ax:#1]{#1})}

\newcommand{\newref}[4][]{
\ifthenelse{\equal{#1}{}}{\newtheorem{h#2}[hthm]{#4}}{\newtheorem{h#2}{#4}[#1]}
\expandafter\newcommand\csname r#2\endcsname[1]{#3~\ref{#2:##1}}
\expandafter\newcommand\csname R#2\endcsname[1]{#4~\ref{#2:##1}}
\expandafter\newcommand\csname n#2\endcsname[1]{\ref{#2:##1}}
\newenvironmentx{#2}[2][1=,2=]{
\ifthenelse{\equal{##2}{}}{\begin{h#2}}{\begin{h#2}[##2]}
\ifthenelse{\equal{##1}{}}{}{\label{#2:##1}}
}{\end{h#2}}
}

\newref[section]{thm}{Theorem}{Theorem}
\newref{lem}{Lemma}{Lemma}
\newref{prop}{Proposition}{Proposition}
\newref{cor}{Corollary}{Corollary}
\newref{cond}{Condition}{Condition}
\newref{conj}{Conjecture}{Conjecture}
\newref{defn}{Definition}{Definition}
\newref{example}{Example}{Example}
\newref{remark}{Remark}{Remark}

\newcommand{\type}{}
\newcommand{\ob}{}
\newcommand{\term}{1}
\newcommand{\unit}{()}

\newcommand{\fs}[1]{\mathrm{#1}}
\newcommand{\cat}[1]{\mathcal{#1}}
\newcommand{\subst}{\fs{subst}}
\newcommand{\Hom}{\fs{Hom}}
\newcommand{\chom}{\fs{hom}}
\newcommand{\Ob}{\fs{Ob}}
\newcommand{\Id}{\fs{Id}}
\newcommand{\refl}{\fs{refl}}
\newcommand{\sym}[1]{#1^{-1}}
\newcommand{\id}{\fs{id}}
\newcommand{\pmap}{\fs{ap}}
\newcommand{\Fib}{\fs{Fib}}
\newcommand{\fib}{\ \fs{fib}}
\newcommand{\El}{\fs{El}}
\newcommand{\qCat}{\fs{qCat}}
\newcommand{\sSpace}{\fs{sSpace}}
\newcommand{\Idext}{\fs{Homext}}
\newcommand{\Cat}{\fs{qCat}}

\numberwithin{figure}{section}

\newcommand{\ct}{%
  \mathchoice{\mathbin{\raisebox{0.25ex}{$\displaystyle\centerdot$}}}%
             {\mathbin{\raisebox{0.25ex}{$\centerdot$}}}%
             {\mathbin{\raisebox{0.25ex}{$\scriptstyle\,\centerdot\,$}}}%
             {\mathbin{\raisebox{0.25ex}{$\scriptscriptstyle\,\centerdot\,$}}}
}

\newcommand{\pb}[1][dr]{\save*!/#1-1.2pc/#1:(-1,1)@^{|-}\restore}
\newcommand{\po}[1][dr]{\save*!/#1+1.2pc/#1:(1,-1)@^{|-}\restore}

\title{Indexed type theories}

\author{Valery Isaev}

\begin{document}

\maketitle

\begin{abstract}
In this paper, we define indexed type theories which are related to indexed ($\infty$-)categories in the same way as (homotopy) type theories are related to ($\infty$-)categories.
We define several standard constructions for such theories including finite (co)limits, arbitrary (co)products, exponents, object classifiers, and orthogonal factorization systems.
We also prove that these constructions are equivalent to their type theoretic counterparts such as $\Sigma$-types, unit types, identity types, finite higher inductive types, $\Pi$-types, univalent universes, and higher modalities.
\end{abstract}

\section{Introduction}

Indexed categories were defined in \cite{indexed-cats} (see also \cite[B1]{elephant}).
We define an analogue of this notion using the language of type theory.
Ordinary homotopy type theory is (conjecturally) an internal language of $\infty$-categories with some additional structure depending on constructions that we assume in the theory.
Often we need to assume that the $\infty$-category is at least locally Cartesian closed.
Indexed type theories allows us to discuss properties of a larger class of (indexed) $\infty$-categories.

Indexed type theories can be useful even when applied to $\infty$-categories which have all the required structure such as $\infty$-toposes.
One problem of ordinary homotopy type theory is that every construction must be characterized by an internal property.
For example, we will define orthogonal factorization systems in section~\ref{sec:refl-fib}.
In ordinary homotopy type theory only internal factorization systems (which are a special case of the external ones) can been defined \cite{modality-hott}.

A similar problem occurs when we try to describe certain universes.
For example, we could try to add a universe $\mathcal{U}_\mathrm{cov}$ of discrete Segal types and covariant maps between them to the theory described in \cite{riehl-dhott}.
A naive definition of such a universe postulates that we have an equivalence between the type of functions $X \to \mathcal{U}_\mathrm{cov}$ and the type of covariant fibrations over $X$.
This is not a correct definition since this condition is too strong.
The correct definition requires only an equivalence between the space of maps from $X$ to $\mathcal{U}_\mathrm{cov}$ and the space of covariant fibrations over $X$.
It is not clear whether it is possible to formulate this condition in ordinary homotopy type theory, but it is easy to do this in an indexed type theory as we will see in section~\ref{sec:class}.

A similar problem with universes is considered in \cite{int-univ-hott}, where modal operators are used to solve it.
Our approach uses a similar two-level system, but the details are different.
We do not assume the existence of any modal operators.
Moreover, in the context of spatial HoTT \cite{cohesive-hott}, the second level is assumed to have a rich type-theoretic structure; we do not assume it in general.
A system similar to ours is considered in \cite{fib-fib-cats}, but that paper mostly discusses the semantics and contains only a basic version of the language.

There are several kinds of indexed type theories.
The most basic one is indexed \emph{unary} type theories.
Such a theory has two levels: the base theory and the indexed theory.
The base theory has the usual type-theoretic syntax and we can assume that it has all the usual constructions of homotopy type theory if necessary.
On the other hand, the number of constructions available at the indexed level is very restricted (in general), but we can use the base theory to reason about objects living at the indexed level.
Even though unary theories is the most basic version of indexed type theories, it is possible to develop the whole theory just within this limited context.
To demonstrate this, we will consider various categorical constructions in unary theories such as limits, colimits, objects classifiers, and orthogonal factorization systems.
Indexed unary type theories can be compared to the approach to higher category theory based on categories enriched in spaces, with the difference that instead of spaces we have types of the base theory.
Actually, we can think of models of indexed unary type theories as a particularly strict way of modeling indexed $\infty$-categories.

The second kind of indexed type theories that we will consider is indexed \emph{dependent} type theories.
Such theories have models only in finitely complete indexed $\infty$-categories, but they are more convenient since we can use dependent types on both levels.
In general, we do not assume that the second level is closed under all type-theoretic constructions such as $\Pi$-types and coproducts, but some of them might exists in particular models.
We will consider each of these constructions separately and prove that their existence imply the existence of the corresponding categorical construction.
Moreover, a categorical construction exists if and only if the \emph{weak} version of the corresponding type-theoretic construction does.
We will prove this for the following constructions:
\begin{enumerate}
\item Finite limits in unary theories and $\Sigma$-types, unit types, and identity types in dependent theories.
\item Finite colimits in unary theories and finite higher inductive types in dependent theories.
\item Exponents in unary theories and $\Pi$-types in dependent theories.
\item Object classifiers in unary theories and univalent universes in dependent theories.
\item Orthogonal factorization systems in unary theories and higher modalities in dependent theories.
\end{enumerate}
Note that it is impossible to give these characterizations in ordinary type theory since the categorical description can be given only in the internal language and such a description is stronger than the external one.
The interpretation of our categorical descriptions in a model is almost immediately equivalent to the usual description of these constructions.
Thus, the equivalences that we are going to prove fill the gap between categorical and type-theoretic descriptions of these constructions.

The paper is organized as follows.
In section~\ref{sec:unary}, we define indexed unary type theories.
In section~\ref{sec:equivalence}, we define the notion of an equivalence in unary theories.
In section~\ref{sec:colimits}, we define limits and colimits in unary theories.
In section~\ref{sec:dependent}, we define indexed dependent type theories.
In section~\ref{sec:lccc}, we define exponent and $\Pi$-types in indexed theories.
In section~\ref{sec:colimits-dep}, we define limits and colimits in dependent theories.
In section~\ref{sec:initial}, we prove the initial type theorem, which is the first step in the proof of the general adjoint functor theorem.
In section~\ref{sec:class}, we defined object classifiers.
In section~\ref{sec:refl-fib}, we define orthogonal factorization systems.

\section{Indexed unary type theories}
\label{sec:unary}

We can think about an indexed type theory as a syntactic representation of indexed $\infty$-categories, that is a contravariant functor $F$ from a (small) $\infty$-category $\cat{B}$ to the $\infty$-category of (small) $\infty$-categories.
An indexed type theory consists of two levels.
The first level is just an ordinary type theory and it represents $\cat{B}$.
Since we are mostly interested in the case when $\cat{B}$ is the $\infty$-category of spaces,
we can assume that the first level has all usual constructions such as identity types, $\Sigma$-types, $\Pi$-types, (univalent) universes, and (higher) inductive types.
Nevertheless, in general, we will assume that the base theory has only identity types and $\Sigma$-types; all additional assumptions will be explicitly specified.
We will often talk about functions, but this is only for notational convenience and does not assume that function types exist.
Terms of type $A \to B$ correspond to terms of type $B$ in context $x : A$, so we can talk about functions $A \to B$ as long as this type does not appear inside other types.

The second level of the theory represents $\infty$-categories $F(\Gamma)$ for various objects $\Gamma$ of $\cat{B}$.
In this section, we will discuss \emph{indexed unary type theories}, that is indexed type theories in which the second level consists of unary type theories.
A unary type theory is a non-dependent type theory in which contexts consist of exactly one type.

\subsection{The definition}

Indexed unary type theories have four kinds of judgments:
\[ \Gamma \vdash A \type \qquad \Gamma \vdash a : A \qquad \Gamma \mid \cdot \vdash B \ob \qquad \Gamma \mid x : A \vdash b : B \]

In each of these judgments, $\Gamma$ is a context, that is a sequence of the form $x_1 : A_1, \ldots x_n : A_n$, where $A_1$, \ldots $A_n$ are types and $x_1$, \ldots $x_n$ are pairwise distinct variables.
Judgments $\Gamma \vdash A \type$ and $\Gamma \vdash a : A$ represent types and terms of the first level of the theory.
We will call such types and terms \emph{base types} and \emph{base terms}, respectively.
The collection of rules that involve only judgments for base types and base terms will be called the base (sub)theory.
When we say that the base theory has some construction such as $\Pi$-types or universes, this means that there are usual rules for these constructions formulated in terms of these judgments.

Judgments $\Gamma \mid \cdot \vdash A \ob$ represent types of the second level of the theory.
We will call these types \emph{indexed types} to distinguish them from base types.
In a judgment $\Gamma \mid x : A \vdash b : B$, $x$ is a variable which is distinct from the variables in $\Gamma$, $A$ and $B$ are indexed types, and $b$ is a term of the second level of the theory.
We will call such terms \emph{indexed terms}.
Indexed types represent objects indexed by $\Gamma$ and indexed terms $\Gamma \mid x : A \vdash b : B$ represent morphisms between $A$ and $B$.

We have the usual rules for variables and substitutions for the base theory:
\begin{center}
\AxiomC{}
\UnaryInfC{$x_1 : A_1, \ldots x_n : A_n \vdash x_i : A_i$}
\DisplayProof
\end{center}

\begin{center}
\def\extraVskip{1pt}
\Axiom$\fCenter \Gamma \vdash b_1 : B_1$
\noLine
\UnaryInf$\fCenter \ldots$
\noLine
\UnaryInf$\fCenter \Gamma \vdash b_k : B_k[b_1/y_1, \ldots b_{k-1}/y_{k-1}]$
\Axiom$\fCenter \Gamma, y_1 : B_1, \ldots y_k : B_k \vdash C \type$
\def\extraVskip{2pt}
\BinaryInfC{$\Gamma \vdash C[b_1/y_1, \ldots b_k/y_k] \type$}
\DisplayProof
\end{center}

\begin{center}
\def\extraVskip{1pt}
\Axiom$\fCenter \Gamma \vdash b_1 : B_1$
\noLine
\UnaryInf$\fCenter \ldots$
\noLine
\UnaryInf$\fCenter \Gamma \vdash b_k : B_k[b_1/y_1, \ldots b_{k-1}/y_{k-1}]$
\Axiom$\fCenter \Gamma, y_1 : B_1, \ldots y_k : B_k \vdash c : C$
\def\extraVskip{2pt}
\BinaryInfC{$\Gamma \vdash c[b_1/y_1, \ldots b_k/y_k] : C[b_1/y_1, \ldots b_k/y_k]$}
\DisplayProof
\end{center}

We also have the usual equations for substitution:
\begin{align*}
y_i[b_1/y_1, \ldots b_k/y_k] & = b_i \\
c[y_1/y_1, \ldots y_k/y_k] & = c \\
d[c_1/z_1, \ldots c_n/z_n][b_1/y_1, \ldots b_k/y_k] & = d[c_1'/z_1, \ldots c_n'/z_n],
\end{align*}
where $c_i' = c_i[b_1/y_1, \ldots b_k/y_k]$.

For every construction $\sigma(\overline{z_1}.\,c_1, \ldots \overline{z_n}.\,c_n)$ in the base theory, we have the following equation whenever variables $\overline{z_1}$, \ldots $\overline{z_n}$ are not free in $b_1$, \ldots $b_k$:
\[ \sigma(\ldots, \overline{z_i}.\,c_i, \ldots)[b_1/y_1, \ldots b_k/y_k] = \sigma(\ldots, \overline{z_i}.\,c_i[b_1/y_1, \ldots b_k/y_k], \ldots) \]
We also have the weakening operation which is left implicit as usual.
This concludes the description of basic rules of the base theory.
They are the usual rules of a dependent type theory which we include here so that they can be compared to the rules of the indexed theory.

Variables of the indexed theory represent identity morphisms and substitution represents composition:
\begin{center}
\AxiomC{}
\UnaryInfC{$\Gamma \mid x : A \vdash x : A$}
\DisplayProof
\qquad
\AxiomC{$\Gamma \mid \Delta \vdash b : B$}
\AxiomC{$\Gamma \mid y : B \vdash c : C$}
\BinaryInfC{$\Gamma \mid \Delta \vdash c[b/y] : C$}
\DisplayProof
\end{center}

These operations satisfy the obvious equations:
\begin{align*}
y[b/y] & = b \\
b[x/x] & = b \\
d[c/z][b/y] & = d[c[b/y]/z]
\end{align*}

We can also substitute base terms into indexed types and terms:
\begin{center}
\def\extraVskip{1pt}
\Axiom$\fCenter \Gamma \vdash b_1 : B_1$
\noLine
\UnaryInf$\fCenter \ldots$
\noLine
\UnaryInf$\fCenter \Gamma \vdash b_k : B_k[b_1/y_1, \ldots b_{k-1}/y_{k-1}]$
\Axiom$\fCenter \Gamma, y_1 : B_1, \ldots y_k : B_k \mid \cdot \vdash C \ob$
\def\extraVskip{2pt}
\BinaryInfC{$\Gamma \mid \cdot \vdash C[b_1/y_1, \ldots b_k/y_k] \ob$}
\DisplayProof
\end{center}

\begin{center}
\def\extraVskip{1pt}
\Axiom$\fCenter \Gamma \vdash b_1 : B_1$
\noLine
\UnaryInf$\fCenter \ldots$
\noLine
\UnaryInf$\fCenter \Gamma \vdash b_k : B_k[b_1/y_1, \ldots b_{k-1}/y_{k-1}]$
\Axiom$\fCenter \Gamma, y_1 : B_1, \ldots y_k : B_k \mid z : C \vdash d : D$
\def\extraVskip{2pt}
\BinaryInfC{$\Gamma \mid z : C[b_1/y_1, \ldots b_k/y_k] \vdash d[b_1/y_1, \ldots b_k/y_k] : D[b_1/y_1, \ldots b_k/y_k]$}
\DisplayProof
\end{center}

These operations represent reindexing along a morphism in the base category.
They satisfy the following equations:
\begin{align*}
x[b_1/y_1, \ldots b_k/y_k] & = x \\
d[c/z][b_1/y_1, \ldots b_k/y_k] & = d[b_1/y_1, \ldots b_k/y_k][c[b_1/y_1, \ldots b_k/y_k]/z] \\
c[y_1/y_1, \ldots y_k/y_k] & = c \\
d[c_1/z_1, \ldots c_n/z_n][b_1/y_1, \ldots b_k/y_k] & = d[c_1'/z_1, \ldots c_n'/z_n],
\end{align*}
where $c_i' = c_i[b_1/y_1, \ldots b_k/y_k]$.
The first two equations correspond to the fact that reindexing preserves identity morphisms and composition in the indexed theory.
The last two equations correspond to the fact that reindexing is functorial, that is it preserves identity morphisms and composition in the base theory.

For every construction $\sigma(\overline{z_1}.\,c_1, \ldots \overline{z_n}.\,c_k)$ in the indexed theory, we have the following equation whenever variables $\overline{z_1}$, \ldots $\overline{z_n}$ are not free in $b_1$, \ldots $b_k$:
\[ \sigma(\ldots, \overline{z_i}.\,c_i, \ldots)[b_1/y_1, \ldots b_k/y_k] = \sigma(\ldots, \overline{z_i}.\,c_i[b_1/y_1, \ldots b_k/y_k], \ldots) \]
We also have the weakening operation which is left implicit as usual.
This equation corresponds to the fact that all constructions in the indexed category must be stable under reindexing.

As we noted before, we assume that the base theory has identity types:
\begin{center}
\AxiomC{$\Gamma \vdash a : A$}
\AxiomC{$\Gamma \vdash a' : A$}
\BinaryInfC{$\Gamma \vdash \Id_A(a,a') \type$}
\DisplayProof
\qquad
\AxiomC{$\Gamma \vdash a : A$}
\UnaryInfC{$\Gamma \vdash \refl(a) : \Id_A(a,a)$}
\DisplayProof
\end{center}
\medskip

\begin{center}
\def\extraVskip{1pt}
\Axiom$\fCenter \Gamma \vdash a : A$
\noLine
\UnaryInf$\fCenter \Gamma \vdash a' : A$
\noLine
\UnaryInf$\fCenter \Gamma \vdash t : \Id_A(a,a')$
\Axiom$\fCenter \Gamma, x : A, p : \Id_A(a,x), \Delta \vdash D \type$
\noLine
\UnaryInf$\fCenter \Gamma, \Delta[a/x,\refl(a)/p] \vdash d : D[a/x,\refl(a)/p]$
\def\extraVskip{2pt}
\BinaryInfC{$\Gamma, \Delta[a'/x,t/p] \vdash J(a, x p \Delta.\,D, \Delta.\,d, a', t) : D[a'/x,t/p]$}
\DisplayProof
\end{center}

\[ J(a, x p \Delta.\,D, \Delta.\,d, a, \refl(a)) = d \]

We will sometimes omit the type in the notation $\Id_A(a,a')$.
The fact that the type $\Id(a,a')$ is inhabited will be denoted by $a \sim a'$.
If $\Gamma \vdash p : \Id_A(a,a')$, $\Gamma, x : A \vdash B \type$, and $\Gamma \vdash b : B[a/x]$, then we will write $\Gamma \vdash p_*(b) : B[a'/x]$ for the usual transport operation defined in terms of $J$.
Operation $\pmap$ is defined in terms of $J$ and has the following type:
if $\Gamma \vdash B \type$, $\Gamma, x : A \vdash b : B$, and $\Gamma \vdash p : \Id_A(a,a')$, then $\Gamma \vdash \pmap(x.b, p) : \Id_B(b[a/x],b[a'/x])$.

An indexed type theory is \emph{locally small} if there is a base type of its morphisms.
That is, it must contain the following rules and equations:
\begin{center}
\AxiomC{$\Gamma \mid \cdot \vdash A \ob$}
\AxiomC{$\Gamma \mid \cdot \vdash B \ob$}
\BinaryInfC{$\Gamma \vdash \Hom(A,B) \type$}
\DisplayProof
\qquad
\AxiomC{$\Gamma \mid x : A \vdash b : B$}
\UnaryInfC{$\Gamma \vdash \lambda x.\,b : \Hom(A,B)$}
\DisplayProof
\end{center}
\medskip

\begin{center}
\AxiomC{$\Gamma \vdash f : \Hom(A,B)$}
\AxiomC{$\Gamma \mid \Delta \vdash a : A$}
\BinaryInfC{$\Gamma \mid \Delta \vdash f\,a : B$}
\DisplayProof
\end{center}

\begin{align*}
(\lambda x.\,b)\,a & = b[a/x] \\
\lambda x.\,f\,x & = f
\end{align*}

We might also use notation $A \to B$ for $\Hom(A,B)$, but we prefer the latter notation since the former may be confusing.
Indeed, $A \to B$ might also denote the indexed type of functions if the indexed theory is Cartesian closed or the base type of function if $A$ and $B$ are base types.

If the indexed theory is locally small, then it should be true that indexed types carry the structure of an $\infty$-category.
Currently, it is not known whether it is possible to construct this structure internally, but we can at least construct its lower levels.
Morphisms between indexed types $A$ and $B$ are terms of type $\Hom(A,B)$.
The identity morphism $\id_A$ on an indexed type $A$ is $\lambda x.\,x : \Hom(A,A)$.
Composition of morphisms $f : \Hom(A,B)$ and $g : \Hom(B,C)$ is defined as $\lambda x.\,g\,(f\,x) : \Hom(A,C)$ and denoted by $g \circ f$.
Composition is strictly associative and identity morphisms are strictly unital.

If $f,g : \Hom(A,B)$ are morphisms, then a 2-morphism between them is a term $p : \Id_{\Hom(A,B)}(f,g)$.
Vertical composition $p \ct q$ of 2-morphisms $p$ and $q$ is defined as the usual operation of path concatenation.
The identity 2-morphism on $f : \Hom(A,B)$ is $\refl(f)$.
Vertical composition is associative, identity 2-morphisms are unital, and every 2-morphism is invertible.
These facts are true in a weak sense, that is up to a 3-morphism.
Let $f,g : \Hom(A,B)$ and $h,i : \Hom(B,C)$ be morphisms and let $p : \Id_{\Hom(A,B)}(f,g)$ and $q : \Id_{\Hom(B,C)}(h,i)$ be 2-morphisms.
The horizontal composition of $p$ and $q$ is a term $p * q$ of type $\Id_{\Hom(A,C)}(\lambda x.\,h\,(f\,x), \lambda x.\,i\,(g\,x))$.
To define $p * q$, we just need to eliminate $p$ and $q$ and then define $\refl(f) * \refl(h)$ as $\refl(\lambda x.\,h\,(f\,x))$.
It is easy to prove that usual properties of this operation hold.
Expressions $\refl(f) * q$, $p * \refl(g)$, and $\refl(f) * \refl(g)$ will be denoted by $f * q$, $p * g$, and $f * g$, respectively.

\begin{defn}[equiv]
An equivalence between indexed types $A$ and $B$ is a morphism $f : \Hom(A,B)$ such that there is a morphism $g : \Hom(B,A)$ such that $g \circ f \sim \id_A$ and $f \circ g \sim \id_B$.
\end{defn}

If the indexed theory is locally small, then not only base morphisms act on indexed types and terms, but also homotopies between them.
Let $\Gamma \vdash a : A$ and $\Gamma \vdash a' : A$ be two base terms.
If $\Gamma, x : A \mid \cdot \vdash B \ob$ be an indexed type, then we have indexed types $\Gamma \mid \cdot \vdash B[a/x] \ob$ and $\Gamma \mid \cdot \vdash B[a'/x] \ob$.
Let $\Gamma \vdash h : \Id_A(a,a')$ be homotopy between $a$ and $a'$.
Then we can construct an equivalence between $B[a/x]$ and $B[a'/x]$.
A map $f : \Hom(B[a/x],B[a'/x])$ is defined as $J(a, x p.\,\Hom(B[a/x],B), \id_{B[a/x]}, a', h)$.
A map $g : \Hom(B[a'/x],B[a/x])$ is constructed similarly: $J(a, x p.\,\Hom(B,B[a/x]), \id_{B[a/x]}, a', h)$.
To prove that $g \circ f$ and $f \circ g$ are homotopic to identity morphisms, it is enough to eliminate $h$ using $J$ and then both $g \circ f$ and $f \circ g$ become identity morphisms.

\subsection{Remarks on terminology}

An arbitrary extension of the theory presented in this section will be called \emph{an indexed unary type theory}.
Such an extension may consist of several constructions (such as $\Pi$-types), inference rules, and computational rules.
The same terminology will be applied to indexed dependent type theories defined in section~\ref{sec:dependent}.

We will say that a theory has some construction (such as products or $\Pi$-types) if necessary operations on terms and types can be defined in the theory, necessary inference rules are derivable, and necessary computational rules hold.
We do not assume that operations or inference rules are primitive.
For example, if the theory has $\Sigma$-types, then it has binary products since they can be defined in terms of $\Sigma$-types (even though they are not primitive).
A more complicated example is the theory of $\Pi$-types.
If a theory has $\Id$-types, $\Sigma$-types and non-dependent function types, then it (almost) has $\Pi$-types.
Indeed, $\Pi_{x : A} B$ can be defined as $\Sigma_{(f : A \to \Sigma_{x : A} B)} \Id(\pi_1 \circ f, \fs{id}_A)$ and operations of abstraction and application can be easily defined so that necessary inference rules are derivable.
The only problem is that $\eta$-equivalence holds only propositionally for this definition of $\Pi$-types.
We will call $\Pi$-types in which computational rules hold only propositionally \emph{weak $\Pi$-types}.
We will see many examples of weak versions of various constructions throughout this paper.

When we say that something is true for every $x : A$ this should be interpreted in the internal sense (as customary in HoTT).
The sentence ``for every $x : A$ it is true that $B(x)$'' is usually translated to the type $\Pi_{x : A} B(x)$ and it is said that the sentence is true when this type is inhabited.
Since we do not assume the existence of $\Pi$-types, we will translate this sentence to the judgement $\Gamma, x : A \vdash B(x)$
and we will say that the sentence is true if there is a term $b$ such that the judgement $\Gamma, x : A \vdash b : B (x)$ is derivable (from given assumptions, if any).

A quantification over types usually means a quantification over some universe, but we do not assume universes exist and moreover we do want to quantify over all types.
For this reason, the sentence ``for every type $A$ it is true that $P(A)$'' would mean that there is a term $p$ such that the following rule is derivable:
\begin{center}
\AxiomC{$\Gamma \vdash A$}
\AxiomC{$J_1$}
\AxiomC{\ldots}
\AxiomC{$J_n$}
\QuaternaryInfC{$\Gamma \vdash p : P(A)$}
\DisplayProof
\end{center}
where $J_1$, \ldots $J_n$ are other given assumptions, if any.

It is also possible to interpret a sentence ``for every $a : A$ it is true that $B(a)$'' in the same way we interpret such a sentence for types.
That is, we can interpret it as the statement that there is a term $b'(a)$, depending on $a$, such that the following rule is derivable:
\begin{center}
\AxiomC{$\Gamma \vdash a : A$}
\AxiomC{$J_1$}
\AxiomC{\ldots}
\AxiomC{$J_n$}
\QuaternaryInfC{$\Gamma \vdash b' : B(a)$}
\DisplayProof
\end{center}
where $J_1$, \ldots $J_n$ are other given assumptions, if any.
This interpretation is equivalent to the previous one since we can define $b'$ as $b[a/x]$ and we can define $b$ as $b'(x)$.

Constructions (such as products) in type theory are usually defined for all types.
We will also describe rules in this form, but we will often assume that some construction is defined only for some specific types.
For example, we could write down the usual rules for $\Pi$-types, but we do not actually assume that they exist for all types.
Instead, we might assume that $\Pi_{x : A} B$ is defined only for some concrete types $A$ and $B$.
Formally, this means that we substitute these specific types in the rules for $\Pi$-types and assume that resulting rules are derivable in our theory.
For example, we could take $A$ to be the type of natural numbers and require the rules for $\Pi$-types to hold for all $B$ and for $A = \mathbb{N}$.
If this is true in some specific indexed type theory, we may say that $\mathbb{N}$ is exponentiable in this theory.

The reason why we do not assume that constructions are defined for all types is that it allows us to prove theorems in a more general form.
For example, to prove that idempotents split (\rprop{split-idemp}), we only need to know that a very special kind of products exist.
Another reason is that we might want to add these constructions with some restrictions.
For example, the $\infty$-category of small $\infty$-categories does not have all $\Pi$-types, but a large class of $\Pi$-types still exists in this category.
If we can express the property of being exponentiable in type theory, we can add the rules for $\Pi$-types with the restriction that the domain is exponentiable.

\subsection{Models}
\label{sec:unary-models}

A \emph{model} of an indexed unary type theory is a model $\cat{B}$ of the underlying base theory (that is, a small contextual category with additional structure, depending on the constructions in the base theory)
together with a contravariant functor from the underlying category of $\cat{B}$ to the category of small categories (with additional structure, depending on the constructions in the indexed theory).
Models of indexed type theories are closely related to indexed $\infty$-categories.
We will show that every model of an indexed dependent type theory determines an indexed $\infty$-categories in section~\ref{sec:models}.
We believe that every model of a locally small indexed unary type theory also gives rise to an indexed $\infty$-category, but this is harder to prove.

A general method of constructing models of indexed type theories is described in \cite{indexed-models}.
It is shown there how to apply this method to construct several models.
For example, there is a model in which base contexts are interpreted as simplicial sets, dependent types of the base theory are interpreted as Kan fibrations as usual,
and indexed types are interpreted as categorical fibrations (in the quasicategorical sense) over the interpretation of the base context.
We will denote this model by $\qCat$.
We will see that many type-theoretic constructions can be interpreted in this models.
This allows us to use a type-theoretic language to reason about $\infty$-categories.

The model $\qCat$ is locally small.
If $C$ and $D$ are quasicategories, then $\Hom(C,D)$ is interpreted as the maximal Kan complex contained in the quasicategory $D^C$ of functors between $C$ and $D$.
That is, it is simply the underlying $\infty$-groupoid of this category presented as a Kan complex.
In particular, if $C = 1$ is the terminal quasicategory, then $\Hom(1,D)$ is the underlying $\infty$-groupoid of $D$.

Another example of a model of an indexed unary type theory is simplicial spaces.
Base contexts are interpreted as simplicial sets, dependent types of the base theory are interpreted as Kan fibrations,
and indexed types are interpreted as injective fibrations over the constant bisimplicial set corresponding to the interpretation of the base context.
We will denote this model by $\sSpace$.
Since bisimplicial sets form a sufficiently nice model category, this model supports all standard type-theoretic constructions at both base and indexed levels.
This model also supports various non-standard constructions which make the theory similar to the one described in \cite{riehl-dhott}.

The model $\sSpace$ is also locally small.
If $C$ and $D$ are bisimplicial sets, then $\Hom(C,D)$ is interpreted as $(D^C)_{[0]}$.
In particular, if $C = 1$ is the terminal bisimplicial set and $D$ is a complete Segal space, then $\Hom(1,D)$ is the underlying $\infty$-groupoid of $D$.
Note that there is no analogous operation in \cite{riehl-dhott}.
It seems that to describe it, either the theory should be two-level like the one presented in this paper or it should have modal operators like the theory presented in \cite{cohesive-hott}.

\section{Equivalences}
\label{sec:equivalence}

In this section, we define types that express the property of a map $f : \Hom(A,B)$ of being an equivalence and prove that they are equivalent.
We also prove a few simple properties of equivalences.
These questions were studied in \cite[Section~4]{hottbook} for ordinary homotopy type theory.
Most of the theorems in this section also hold in the framework of indexed unary type theories, but the proofs must be modified.

\subsection{Bi-invertible maps}

Let $f : \Hom(A,B)$ be a morphism.
We will say that $f$ is \emph{bi-invertible} if the following type is inhabited:
\[ \fs{biinv}(f) = \fs{linv}(f) \times \fs{rinv}(f), \]
where $\fs{linv}(f)$ and $\fs{rinv}(f)$ are types of left and right inverses of $f$, respectively:
\begin{align*}
\fs{linv}(f) & = \sum_{g : \Hom(B,A)} \Id(g \circ f, \id_A) \\
\fs{rinv}(f) & = \sum_{g : \Hom(B,A)} \Id(f \circ g, \id_B)
\end{align*}

\begin{prop}[biinv-equiv]
A map is bi-invertible if and only if it is an equivalence.
\end{prop}
\begin{proof}
Obviously, if a map is an equivalence, then it is bi-invertible.
Let us prove the converse.
Let $g : \Hom(B,A)$, $p : \Id(g \circ f, \id_A)$ be a left inverse of $f$ and let $g' : \Hom(B,A)$, $p' : \Id(f \circ g', \id_B)$ be a right inverse of $f$.
Then $g' * p : \Id(g \circ f \circ g', g')$.
Since $f \circ g' \sim \id_B$, there is a term of type $\Id(g,g')$.
It follows that $g$ is an inverse of $f$.
\end{proof}

\begin{lem}[lrinv-contr]
If $f$ is an equivalence, then types $\fs{linv}(f)$ and $\fs{rinv}(f)$ are contractible.
\end{lem}
\begin{proof}
If $f$ is an equivalence, then precomposition with $f$ is an equivalence between types $\Hom(B,C)$ and $\Hom(A,C)$.
Similarly, postcomposition with $f$ is an equivalence between types $\Hom(C,A)$ and $\Hom(C,B)$.
Since $\fs{linv}(f)$ and $\fs{rinv}(f)$ are fibres of these maps over the identity morphisms, \cite[Theorem~4.2.3]{hottbook} and \cite[Theorem~4.2.6]{hottbook} imply that these types are contractible.
Note that the proofs of these theorems work even if we do not have $\Pi$-types.
\end{proof}

\begin{prop}
The type $\fs{biinv}(f)$ is a proposition.
\end{prop}
\begin{proof}
This follows from \rprop{biinv-equiv} and \rlem{lrinv-contr}.
\end{proof}

\subsection{Half adjoint equivalences}

Let $f : \Hom(A,B)$ be a morphism.
We will say that $f$ is a \emph{half adjoint equivalence} if the following type is inhabited:
\[ \fs{ishae}(f) = \sum_{g : \Hom(B,A)} \sum_{\eta : \Id(g \circ f, \id_A)} \sum_{\epsilon : \Id(f \circ g, \id_B)} \Id(\eta * f, f * \epsilon). \]

\begin{prop}
A map is a half adjoint equivalence if and only if it is an equivalence.
\end{prop}
\begin{proof}
Obviously, if a map is a half adjoint equivalence, then it is an equivalence.
Let us prove the converse.
Let $g : \Hom(B,A)$, $\eta : \Id(g \circ f, \id_A)$, $\epsilon : \Id(f \circ g, \id_B)$ be an inverse of $f$.
Then we define $\epsilon' : \Id(f \circ g, \id_B)$ as concatenation of paths
$g * f * \sym{\epsilon} : \Id(f \circ g, f \circ g \circ f \circ g)$, $g * \eta * f : \Id(f \circ g \circ f \circ g, f \circ g)$, and $\epsilon : \Id(f \circ g, \id_B)$.
We need to prove that $f * \epsilon' \sim \eta * f$.

First, note that $\eta * f * g \sim f * g * \eta$.
Indeed, $(\eta * f * g) \ct \eta \sim \eta * \eta \sim (f * g * \eta) \ct \eta$.
Thus, if we cancel $\eta$, this gives us a homotopy between the original paths.
Now, we can finish the proof:
\begin{align*}
f * \epsilon' & \sim \\
(f * g * f * \sym{\epsilon}) \ct (f * g * \eta * f) \ct (f * \epsilon) & \sim \\
(f * g * f * \sym{\epsilon}) \ct (\eta * f * g * f) \ct (f * \epsilon) & \sim \\
(\eta * f * \sym{\epsilon}) \ct (f * \epsilon) & \sim \\
(\eta * f) \ct (f * \sym{\epsilon}) \ct (f * \epsilon) & \sim \\
\eta * f & .
\end{align*}
\end{proof}

\begin{prop}
The type $\fs{ishae}(f)$ is a proposition.
\end{prop}
\begin{proof}
We can assume that $f$ is an equivalence and prove that $\fs{ishae}(f)$ is contractible.
By \rlem{lrinv-contr}, the type $\Sigma_{g : \Hom(B,A)} \Id(g \circ f, \id_A)$ is contractible.
Thus, we just need to prove that, for every $g : \Hom(B,A)$ and $\eta : \Id(g \circ f, \id_A)$, the type $\Sigma_{\epsilon : \Id(f \circ g, \id_B)} \Id(\eta * f, f * \epsilon)$ is also contractible.

Since $f$ is an equivalence, the function $f * -$ is also an equivalence.
It follows that the type $\Id(\eta * f, f * \epsilon)$ is equivalent to the type $\Id(h(\eta * f), \epsilon)$, where $h$ is the inverse of $f * -$.
Thus, the type $\Sigma_{\epsilon : \Id(f \circ g, \id_B)} \Id(\eta * f, f * \epsilon)$ is equivalent to the type $\Sigma_{\epsilon : \Id(f \circ g, \id_B)} \Id(h(\eta * f), \epsilon)$, which is contractible by \cite[Lemma~3.11.8]{hottbook}.
\end{proof}

\subsection{Properties of equivalences}

\begin{prop}
Equivalences satisfy the 2-out-of-6 property.
That is, if $f : \Hom(A,B)$, $g : \Hom(B,C)$, and $h : \Hom(C,D)$ are maps such that $g \circ f$ and $h \circ g$ are equivalences, then so are the maps $f$, $g$, $h$, and $h \circ g \circ f$.
\end{prop}
\begin{proof}
Let $i : \Hom(C,A)$ be an inverse of $g \circ f$ and let $k : \Hom(D,B)$ be an inverse of $h \circ g$.
Since $g \circ f \circ i \sim \id_C$ and $h \circ g \circ k \sim \id_D$, \rprop{biinv-equiv} implies that $g$ is an equivalence.
The map $i \circ g$ is an inverse of $f$.
Indeed, $i \circ g \circ f \sim \id_A$ since $i$ is an inverse of $g \circ f$.
Since $g \circ f \circ i \sim \id_C$, it follows that $g \circ f \circ i \circ g \sim g$.
Since $g$ is an equivalence, this implies that $f \circ i \circ g \sim \id_B$.
Similarly, $g \circ k$ is an inverse of $h$.
The map $h \circ g \circ f$ is an equivalence since equivalences are closed under composition.
\end{proof}

A map $f : \Hom(A,B)$ is a \emph{quasi-retract} of a map $g : \Hom(C,D)$ if there is a commutative diagram of the form
\[ \xymatrix{ A \ar[r]^i \ar[d]_f & C \ar[r]^j \ar[d]_g & A \ar[d]_f \\
              B \ar[r]_k          & D \ar[r]_m          & B
            } \]
such that $j \circ i \sim \id_A$ and $m \circ k \sim \id_B$.

\begin{prop}
Equivalences are closed under quasi-retracts.
\end{prop}
\begin{proof}
Let $f : \Hom(A,B)$ be a retract of $g : \Hom(C,D)$ and let $i,j,k,m$ be maps as in the diagram above.
Let $h : \Hom(D,C)$ be an inverse of $g$.
Then $j \circ h \circ k$ is an inverse of $f$.
Indeed, $j \circ h \circ k \circ f \sim j \circ h \circ g \circ i \sim j \circ \id_C \circ i = j \circ i \sim \id_B$ and
$f \circ j \circ h \circ k \sim m \circ g \circ h \circ k \sim m \circ \id_D \circ k = m \circ k \sim \id_B$.
\end{proof}

\section{Limits and colimits}
\label{sec:colimits}

In this section, we will work in a locally small indexed unary type theory.
We will define specific finite (co)limits and arbitrary (co)products.

\subsection{Terminal and initial types}
\label{sec:term-init}

An indexed type $T$ is \emph{terminal} if, for every indexed type $X$, the type $\Hom(X,T)$ is contractible.
Formally, this means that their exist terms $c(X)$ and $p(X)$ such that the following rules are derivable:
\begin{center}
\AxiomC{$\Gamma \mid \cdot \vdash X \ob$}
\UnaryInfC{$\Gamma \vdash c(X) : \Hom(X,T)$}
\DisplayProof
\qquad
\AxiomC{$\Gamma \mid \cdot \vdash X \ob$}
\UnaryInfC{$\Gamma, f : \Hom(X,T) \vdash p(X) : \Id(f,c(X))$}
\DisplayProof
\end{center}
The second rule can be equivalently written as follows:
\begin{center}
\AxiomC{$\Gamma \vdash f : \Hom(X,T)$}
\UnaryInfC{$\Gamma \vdash p'(f) : \Id(f,c(X))$}
\DisplayProof
\end{center}

We will say that an indexed unary type theory \emph{has terminal types} if the following rule and the rules given above, in which $T$ is replaced with $1$, are derivable:
\begin{center}
\AxiomC{}
\UnaryInfC{$\Gamma \mid \cdot \vdash 1 \ob$}
\DisplayProof
\end{center}

Dually, an indexed type $T$ is \emph{initial} if, for every indexed type $X$, the type $\Hom(T,X)$ is contractible.
Terminal and initial types are unique up to unique equivalence, that is the type of equivalences between a pair of terminal or initial types is contractible.

A type which is both terminal and initial is called a \emph{zero type}.
This is an interesting example since a non-trivial $\infty$-category cannot be Cartesian closed and have a zero object at the same type.
This means that we cannot use ordinary homotopy type theory as an internal language of such $\infty$-categories.

For every nice enough model of the base theory, there is an obvious model of the theory of zero types in which indexed types over a base context $\Gamma$ are interpreted as pointed base types in this context.
To make the theory match this model more closely, we can add an indexed type $S^0$.
Then every indexed type $A$ has the underlying base type $\Hom(S^0,A)$.
This base type is pointed with the base pointed given by the zero map, that is the map $S^0 \to 0 \to A$, where $0$ is the zero type.
Moreover, for every map $f : \Hom(A,B)$, the map $f \circ - : \Hom(S^0,A) \to \Hom(S^0,B)$ preserves the base point.

Finally, we can make the theory match the model exactly.
For this, we assume that the base theory has $\Pi$-types.
For every pair of pointed base types $X$ and $Y$, let $X \to_* Y$ be the type of pointed maps between them.
We already defined a map $\Hom(A,B) \to (\Hom(S^0,A) \to_* \Hom(S^0,B))$.
Now, we can require this map to be an equivalence.
Also, for every pointed base type $(X,*)$, we can add an indexed type $R(X,*)$ together with a pointed equivalence between $\Hom(S^0,R(X,*))$ and $(X,*)$.
The theory of zero types with these additional constructions will be called \emph{the theory of pointed types}.

\subsection{Pullbacks and pushouts}

A \emph{pullback} of a pair of maps $f : \Hom(A,C)$ and $g : \Hom(B,C)$ is a triple $\pi_1 : \Hom(P, A)$, $\pi_2 : \Hom(P, B)$, $\pi_3 : \Id(f \circ \pi_1, g \circ \pi_2)$
such that the following function is an equivalence for every indexed type $P'$:
\[ \lambda h.\,(\pi_1 \circ h, \pi_2 \circ h, h * \pi_3) : \Hom(P',P) \to \Hom(P',A) \times_{\Hom(P',C)} \Hom(P',B), \]
where the pullback of types $\Hom(P',A) \times_{\Hom(P',C)} \Hom(P',B)$ is defined as usual:
\[ \sum_{\pi_1' : \Hom(P',A)} \sum_{\pi_2' : \Hom(P',B)} \Id(f \circ \pi_1', g \circ \pi_2'). \]
An indexed unary type theory \emph{has pullbacks} if a pullback exists for every pair of maps with a common codomain in every context.
Formally, this means that the following rule is derivable and $A \times_C B$ is a pullback of $f$ and $g$ in the sense defined above.
\begin{center}
\AxiomC{$\Gamma \vdash f : \Hom(A,C)$}
\AxiomC{$\Gamma \vdash g : \Hom(B,C)$}
\BinaryInfC{$\Gamma \mid \cdot \vdash A \times_C B \ob$}
\DisplayProof
\end{center}
\emph{Pushouts} $A \amalg_C B$ are defined dually.

\begin{remark}
Pullbacks and pushouts are unique up to unique equivalence.
\end{remark}

\begin{example}
If an indexed unary type theory has a terminal type $\term$ with a point $\unit : \term$, then the \emph{loop space type} of a pointed type $Y$, $y_0 : \Hom(\term,Y)$ is the pullback of $y_0$ and $y_0$.
Equivalently, the loop space type is the equalizer of $y_0$ and $y_0$.
Thus, the loop space type is a type $\Omega(Y,y_0)$ together with a homotopy $\Id_{\Hom(\Omega(Y,y_0), Y)}(\lambda s.\,y_0\,\unit, \lambda s.\,y_0\,\unit)$ satisfying the universal property.
Since $\Hom(X,-)$ preserves terminal types and pullbacks, we have the following equivalence:
\[ \Hom(X, \Omega(Y,y_0)) \simeq \Omega(\Hom(X,Y), \lambda x.\,y_0\,\unit), \]
where the second $\Omega$ is the usual loop space of a base type: $\Omega(S,s_0) = \Id_S(s_0,s_0)$.
\end{example}

\begin{example}
The \emph{suspension} $\Sigma X$ of a type $X$ is the pushout of the maps $\lambda x.\,\unit, \lambda x.\,\unit : \Hom(X,\term)$.
The \emph{$0$-sphere} $S^0$ is the coproduct $\term \amalg \term$.
The \emph{$(n+1)$-sphere} $S^{n+1}$ is the suspension $\Sigma S^n$.
\end{example}

We will say that a theory is \emph{stable} if it has a zero type, pullbacks, and pushouts and, for every indexed type $A$, the canonical maps $A \to \Omega \Sigma A$ and $\Sigma \Omega A \to A$ are equivalences.
This is just a reformulation of the definition of stable $\infty$-categories in the context of indexed type theories.
An obvious example of a model of such a theory is the one in which indexed types over a base context $\Gamma$ are interpreted as spectra of base types in this context.
We did not verify that this model exists, but we still can discuss stable theories from the syntactic perspective.

Let $S$ be a fixed indexed type.
Then, for every indexed type $A$, we can define a spectrum of base types $U_S(A)$.
Let $U_S(A)_n = \Hom(S,\Sigma^n(A))$.
We already saw in section~\ref{sec:term-init} that these types are pointed.
Now, we can define an equivalence $\Omega(U_S(A)_{n+1}) = \Omega(\Hom(S,\Sigma^{n+1}(A))) \simeq \Hom(S,\Omega\Sigma^{n+1}(A)) \simeq \Hom(S,\Sigma^n(A)) = U_S(A)_n$.
Thus, $U_S(A)$ is indeed a spectrum.
Moreover, for every map $f : \Hom(A,B)$, the maps $\Sigma^n(f) \circ - : U_S(A)_n \to U_S(B)_n$ determine a morphism of spectra.
We will denote these maps by $U_S(f)_n$.

The spectrum $U_S(A)$ is not indexed by the base type of natural numbers (we do not even assume that such a type exists).
That is, $U_S(A)$ is just a sequence of base types together with a sequence of structure maps.
To define $U_S(A)$ as an internal spectrum, we need to assume that the base theory has a type of natural numbers.
Moreover, we need the following rule
\begin{center}
\AxiomC{$\Gamma \mid \cdot \vdash A \ob$}
\AxiomC{$\Gamma \vdash n : \mathbb{N}$}
\BinaryInfC{$\Gamma \mid \cdot \vdash \Sigma^n(A) \ob$}
\DisplayProof
\end{center}
satisfying the following computation rules
\begin{align*}
\Sigma^0(A) & = A \\
\Sigma^{\fs{suc}(n)}(A) & = \Sigma(\Sigma^n(A))
\end{align*}
Now, the definition of the spectrum $U_S(A)$ given above makes sense internally.

We can also define the theory of spectra.
For this, we assume that the base theory has $\Pi$-types, natural numbers, and the rules given above.
For every pair of spectra $X$ and $Y$, let $\fs{Sp}(X,Y)$ be the type of morphisms between them.
We already defined a map $\Hom(A,B) \to \fs{Sp}(U_S(A),U_S(B))$.
Now, we can require this map to be an equivalence.
Also, for every spectrum $X$, we can add an indexed type $R(X)$ together with an equivalence between $U_S(R(X))$ and $X$.
The stable theory with these additional constructions will be called \emph{the theory of spectra}.

The theory of spectra can be used to reason about spectra in a more convenient way.
For example, composition of morphisms of spectra (presented as indexed types) is strictly associative.
Moreover, we can apply a type-theoretic language to various constructions on spectra such as their coproduct.
Finally, we can even consider an extension of indexed unary type theories in which the indexed level is a linear type theory.
The theory of spectra extened in such a way would allow us to work with the smash product of spectra in a convenient way.

\subsection{Finite (co)limits}
\label{sec:finite-limits}

A \emph{binary product} of indexed types $A$ and $B$ is an indexed type $A \times B$ together with a pair of maps $\pi_1 : \Hom(A \times B, A)$ and $\pi_2 : \Hom(A \times B, B)$
such that the following function is an equivalence for every indexed type $C$:
\[ \lambda h.\,(\pi_1 \circ h, \pi_2 \circ h) : \Hom(C, A \times B) \to \Hom(C,A) \times \Hom(C,B). \]
The inverse of this function will be denoted by $\langle -, - \rangle$.
An indexed unary type theory \emph{has binary products} if a binary product exists for every pair of types in every context.
\emph{Binary coproducts} $A \amalg B$ are defined dually.

An \emph{equalizer} of a pair of maps $f,g : \Hom(A,B)$ is a map $e : \Hom(E,A)$ together with a homotopy $p : \Id(f \circ e, g \circ e)$
such that the following function is an equivalence for every indexed type $E'$:
\[ \lambda h.\,(e \circ h, h * p) : \Hom(E', E) \to \sum_{e' : \Hom(E',A)} \Id(f \circ e', g \circ e'). \]
An indexed unary type theory \emph{has equalizers} if an equalizer exists for every parallel pair of maps in every context.
\emph{Coequalizers} are defined dually.

\begin{remark}
Binary (co)products and (co)equalizers are unique up to unique equivalence.
\end{remark}

We have the following standard proposition:

\begin{prop}[fin-lim]
An indexed unary type theory with a terminal type has pullbacks if and only if it has equalizers and binary products.
\end{prop}
\begin{proof}
First, suppose that the theory has a terminal type $1$ and pullbacks.
Then we can define a product of types $A$ and $B$ as the pullback of unique maps $!_A : \Hom(A,1)$ and $!_B : \Hom(B,1)$.
Since $\Hom(P,1)$ is contractible, the obvious projection $\Hom(P,A) \times_{\Hom(P,1)} \Hom(P,B) \to \Hom(P,A) \times \Hom(P,B)$ is an equivalence.
This implies that $A \times_1 B$ is a product of $A$ and $B$.

An equalizer of maps $f,g : \Hom(A,B)$ can be defined as the pullback of $\langle \id_B, \id_B \rangle : \Hom(B, B \times B)$ along $\langle f, g \rangle : \Hom(A, B \times B)$:
\[ \xymatrix{ E \ar[r]^s \ar[d]_e & B \ar[d]^{\langle \id_B, \id_B \rangle} \\
              A \ar[r]_-{\langle f, g \rangle} & B \times B
            } \]
By the definition of products, the type of homotopies $\Id_{\Hom(P, B \times B)}(r,r')$ is equivalent to the type $\Id_{\Hom(P,B)}(\pi_1 \circ r, \pi_1 \circ r') \times \Id_{\Hom(P,B)}(\pi_2 \circ r, \pi_2 \circ r')$.
Thus, by the definition of pushouts, we have the following equivalence:
\begin{align*}
& \Hom(P,E) \to \sum_{a : \Hom(P,A)} \sum_{b : \Hom(P,B)} \Id(f \circ a, b) \times \Id(g \circ a, b) \\
& \lambda q.\,(e \circ q, s \circ q, q * h_1, q * h_2),
\end{align*}
where $h_1 : \Id(f \circ e, s)$ and $h_2 : \Id(g \circ e, s)$ are certain homotopies.
The codomain of this function is equivalent to $\Sigma_{a : \Hom(P,A)} \Id(f \circ a, g \circ a)$.
This implies that we have the following equivalence:
\[ \lambda q.\,(e \circ q, q * (h_1 \ct \sym{h_2})) : \Hom(P,E) \to \sum_{a : \Hom(P,A)} \Id(f \circ a, g \circ a). \]
Thus, we can define an equalizer of maps $f$ and $g$ as the triple $E$, $e$, $h_1 \ct \sym{h_2}$.

Now, suppose that the theory has binary products and equalizers.
Let $f : \Hom(A,C)$ and $g : \Hom(B,C)$ be a pair of maps.
Let $e : P \to A \times B$, $h : \Id(f \circ \pi_1 \circ e, g \circ \pi_2 \circ e)$ be the equalizer of the maps $f \circ \pi_1, g \circ \pi_2 : A \times B \to C$.
Then we can define a pullback of $f$ and $g$ as the triple $\pi_1 \circ e$, $\pi_2 \circ e$, $h$.
The universal property of equalizers implies the universal property of pullbacks.
\end{proof}

\begin{defn}[fin-lim]
An indexed unary type theory \emph{has finite limits} if the equivalent conditions of \rprop{fin-lim} hold.
Dually, it has finite colimits if it has an initial type and pushouts.
\end{defn}

The theory of spectra (and any stable theory) has all finite limits and finite colimits by definition.
The theory of pointed types also has all finite limits.
It has a terminal type by definition.
Since the ``category'' of indexed types is equivalent to the ``category'' of pointed base types, to prove that it has pullbacks, it is enough to prove this for pointed base types.
Pullbacks of pointed types computed as pullbacks of the underlying types.
If the base theory has pushouts, then the theory of pointed types also has pushouts (and hence all finite colimits) since pushouts of pointed types are also computed as pushouts of the underlying types.

\subsection{(Co)products}
\label{sec:products}

A \emph{product} of an indexed type $\Gamma, i : I \mid \cdot \vdash B \ob$ is an indexed type $\Gamma \mid \cdot \vdash P \ob$ together with a term $\Gamma, i : I \vdash \pi : \Hom(P,B)$
such that the function $\pi \circ -$ has an inverse in the sense that there is a rule of the form
\begin{center}
\AxiomC{$\Gamma \mid \cdot \vdash P' \ob$}
\AxiomC{$\Gamma, i : I \vdash f : \Hom(P',B)$}
\BinaryInfC{$\Gamma \vdash \langle f \rangle_{i : I} : \Hom(P',P)$}
\DisplayProof
\end{center}
and the following types are inhabited:
\begin{align*}
& \Id(\pi \circ \langle f \rangle_{i : I}, f), \\
& \Id(\langle \pi \circ g \rangle_{i : I}, g).
\end{align*}

The theory of coproducts is defined dually.
A \emph{coproduct} of an indexed type $\Gamma, i : I \mid \cdot \vdash B \ob$ is an indexed type $\Gamma \mid \cdot \vdash C \ob$ together with a term $\Gamma, i : I \vdash \fs{in} : \Hom(B,C)$
such that the function $- \circ \fs{in}$ has an inverse in the sense that there is a rule of the form
\begin{center}
\AxiomC{$\Gamma \mid \cdot \vdash C' \ob$}
\AxiomC{$\Gamma, i : I \vdash f : \Hom(B,C')$}
\BinaryInfC{$\Gamma \vdash [ f ]_{i : I} : \Hom(C,C')$}
\DisplayProof
\end{center}
and the following types are inhabited:
\begin{align*}
& \Id([ f ]_{i : I} \circ \fs{in}, f) \\
& \Id([ g \circ \fs{in} ]_{i : I}, g).
\end{align*}

If the $\Pi$-type $\Pi_{i : I} \Hom(P',B)$ exists for all indexed types $\Gamma \mid \cdot \vdash P' \ob$, then a pair $P$, $\pi$ is a product of a family $B$ if and only if the following function is an equivalence for every indexed type $P'$:
\[ \lambda h.\,\lambda i.\,\pi \circ h : \Hom(P',P) \to \prod_{i : I} \Hom(P',B). \]
Dually, if the $\Pi$-type $\Pi_{i : I} \Hom(B,C')$ exists for all indexed types $\Gamma \mid \cdot \vdash C' \ob$, then a pair $C$, $\fs{in}$ is a product of a family $B$ if and only if the following function is an equivalence for every indexed type $C'$:
\[ \lambda h.\,\lambda i.\,h \circ \fs{in}(i) : \Hom(C,C') \to \prod_{i : I} \Hom(B,C'). \]

Products and coproducts are unique up to unique equivalence.
We will denote the product and the coproduct of a family $\Gamma, i : I \mid \cdot \vdash B \type$ by $\prod_{i : I} B$ and $\coprod_{i : I} B$, respectively.
We will say that the product $\prod_{i : I} B$ is \emph{extensional} if the type $\Hom(P, \prod_{i : I} B)$ satisfies functional extensionality as a weak $\Pi$-type $\Pi_{i : I} \Hom(P,B)$ for all indexed types $P$.
Similarly, we will say that the coproduct $\coprod_{i : I} B$ is \emph{extensional} if the type $\Hom(\coprod_{i : I} B, C)$ satisfies functional extensionality as a weak $\Pi$-type $\Pi_{i : I} \Hom(B,C)$ for all indexed types $C$.

\begin{example}
If $\Gamma \vdash I \type$ is a base type and $\Gamma \mid \cdot \vdash X \type$ is an indexed type, then the \emph{power} (or \emph{cotensor}) of $X$ by $I$ is the product $\prod_{i : I} X$.
The \emph{copower} (or \emph{tensor}) of $X$ by $I$ is the coproduct $\coprod_{i : I} X$.
The power will be denoted by $X^I$ and the tensor by $I \cdot X$.
\end{example}

\begin{example}
Let $S^1$ be the suspension of $1 \amalg 1$ in the base theory, that is a higher inductive type with two point constructors $N,S : S^1$ and two path constructors $L,R : \Id_{S^1}(N,S)$.
Then the product of a family $\Gamma, x : S^1 \mid \cdot \vdash B$ is the equalizer of the maps $L_*(-),R_*(-) : \Hom(B[N/x], B[S/x])$.
\end{example}

We can also define the theory of \emph{strict products}:
\begin{center}
\AxiomC{$\Gamma, i : I \mid \cdot \vdash B \ob$}
\UnaryInfC{$\Gamma \mid \cdot \vdash \prod_{i : I} B \ob$}
\DisplayProof
\qquad
\AxiomC{$\Gamma, i : I \mid \Delta \vdash b : B$}
\RightLabel{, $i \notin \mathrm{FV}(\Delta)$}
\UnaryInfC{$\Gamma \mid \Delta \vdash \lambda i.\,b : \prod_{i : I} B$}
\DisplayProof
\end{center}
\medskip

\begin{center}
\AxiomC{$\Gamma \mid \Delta \vdash f : \prod_{i : I} B$}
\AxiomC{$\Gamma \vdash j : I$}
\BinaryInfC{$\Gamma \mid \Delta \vdash f\,j : B[j/i]$}
\DisplayProof
\end{center}

\begin{align*}
(\lambda i.\,b)\,j & = b[j/i] \\
\lambda i.\,f\,i & = f
\end{align*}

The difference between weak and strict products is that the former requires types $\Hom(P',\prod_{i : I} B)$ and $\Pi_{i : I} \Hom(P',B)$ to be equivalent while the latter requires them to be \emph{isomorphic}.
The theory of \emph{weak products} is defined in the same way as the theory of strict products with the difference that the last two equations hold only propositionally.
This theory is simply a reformulation of the theory of products:

\begin{prop}
A type $\Gamma \mid \cdot \vdash \prod_{i : I} B$ is a product of a family $\Gamma, i : I \mid \cdot \vdash B \ob$ if and only if it is a weak product of this family.
\end{prop}
\begin{proof}
First, suppose that $\prod_{i : I} B$ is a product.
If $\Gamma, i : I \mid x : A \vdash b : B$, then we define $\lambda i.\,b$ as $\langle \lambda x.\,b \rangle_{i : I}\,x$.
If $\Gamma \mid x : A \vdash f : \prod_{i : I} B$ and $\Gamma \vdash j : I$, then we define $f\,j$ as $\pi[j/i]\,f$.
The $\beta$ and $\eta$ equations hold for these definitions:
\begin{align*}
& \lambda x.\,(\lambda i.b)\,j = \lambda x.\,\pi[j/i]\,(\langle \lambda x.b \rangle_{i : I}\,x) = (\pi \circ \langle \lambda x.b \rangle_{i : I})[j/i] \sim (\lambda x.b)[j/i] = \lambda x.\,b[j/i] \\
& \lambda x.\,\lambda i.\,f\,i = \lambda x.\,\langle \lambda x.\,\pi\,f \rangle_{i : I}\,x = \langle \lambda x.\,\pi\,f \rangle_{i : I} = \langle \pi \circ (\lambda x.f) \rangle_{i : I} \sim \lambda x.f
\end{align*}

Now, suppose that $\prod_{i : I} B$ is a weak product.
We define
\[ \Gamma, i : I \vdash \pi : \Hom(\prod_{i : I} B, B) \]
as $\lambda f.\,f\,i$.
If $\Gamma, i : I \vdash g : \Hom(P,B)$, then we define $\Gamma \vdash \langle g \rangle_{i : I} : \Hom(P, \prod_{i : I} B)$ as $\lambda x.\,\lambda i.\,g\,x$.
The required homotopies can be constructed as follows:
\begin{align*}
& \pi \circ \langle g \rangle_{i : I} = \lambda x.\,(\lambda i.\,g\,x)\,i \sim \lambda x.\,g\,x = g \\
& \langle \pi \circ g \rangle_{i : I} = \lambda x.\,\lambda i.\,g\,x\,i \sim \lambda x.\,g\,x = g
\end{align*}
\end{proof}

\begin{example}
If functional extensionality holds in the base theory, then the theory of pointed types has extensional products.
The product of a family $\Gamma, i : I \mid \cdot \vdash B_i$ is defined as $R(\prod_{i : I} \Hom(S^0,B_i), \lambda i.\,*_{B_i})$, where $*_{B_i}$ is the base point of $\Hom(S^0,B_i)$.
\end{example}

\begin{example}
If the base theory has pushouts and functional extensionality holds in it, then the theory of pointed types has extensional coproducts.
The coproduct of a family $\Gamma, i : I \mid \cdot \vdash B_i$ is defined as $R(\bigvee_{i : I} B_i, *)$, where $\bigvee_{i : I} B_i$ is the following pushouts:
\[ \xymatrix{ I                           \ar[r] \ar[d]_{\lambda i.\,(i,*_{B_i})} & 1 \ar[d]^{*} \\
              \sum_{i : I} \Hom(S^0, B_i) \ar[r]                                  & \po \bigvee_{i : I} B_i
            } \]
\end{example}

\begin{example}
Tensoring $I \cdot S^0$ is the free pointed type $I_* = R(I \amalg \{ * \}, *)$.
More generally, $I \cdot B$ is the smash product $I_* \wedge B$.
\end{example}

\begin{example}
If functional extensionality holds in the base theory, then the theory of spectra has extensional products.
The product of a family of spectra is defined pointwise.
\end{example}

\begin{example}
If the base theory has pushouts and functional extensionality holds in it, then the theory of spectra has extensional coproducts.
Let $\Gamma, i : I \mid \cdot \vdash B_i$ be a family of indexed types.
Then we can define a prespectrum $\Gamma, n : \mathbb{N} \vdash \bigvee_{i : I} U_S(B_i)_n$ with the structure maps induced by the structure maps of spectra $B_i$.
A tedious but straightforward computation shows that (the indexed type corresponding to) the spectrification of this prespectrum is the coproduct of $B_i$.
\end{example}

\begin{example}
Tensoring $I \cdot S$ is the suspension spectrum $\Sigma^\infty I_*$.
More generally, $I \cdot B$ is the smash product $\Sigma^\infty I_* \wedge B$.
\end{example}

Let us prove a few properties of products and coproducts.
To simplify the notation, we will assume that the base theory of an indexed theory with a product $\prod_{i : I} B$ has $\Pi$-types $\Pi_{i : I} \Hom(P,B)$ for all $P$.
The following proposition shows that the (co)product of a contractible family of types is any type of this family:

\begin{prop}
Let $I$ be a contractible type and let $i_0$ be a point of $I$.
Then types $\prod_{i : I} B$, $\coprod_{i : I} B$, and $B[i_0/i]$ are equivalent.
\end{prop}
\begin{proof}
Let $p(i)$ be a path between $i_0$ and $i : I$.
Then the pair $B[i_0/i], \pi_i = \lambda x.\,p(i)_*(x)$ is a product of $B$.
We can define $\langle f \rangle_{i : I}$ as $\lambda x.\,\sym{p(i)}_*(f\,x)$.
Clearly, this is an inverse to $\pi \circ -$.
Since $B[i_0/i],\pi$ is a product and products are unique up to equivalence, it follows that $B[i_0/i]$ is equivalent to $\prod_{i : I} B$.
Similar argument shows that it is also equivalent to $\coprod_{i : I} B$.
\end{proof}

The following proposition shows how to compute products and coproducts indexed by $\Sigma$-types:

\begin{prop}
Let $\Gamma \vdash I \type$ and $\Gamma, i : I \vdash J \type$ be base types and let $\Gamma, i : I, j : J \mid \cdot \vdash B \ob$ be an indexed type.
Then types $\prod_{(p : \Sigma_{i : I} J)} B[\pi_1(p)/i, \pi_2(p)/j]$ and $\prod_{i : I} \prod_{j : J} B$ are equivalent.
Dually, types $\coprod_{(p : \Sigma_{i : I} J)} B[\pi_1(p)/i, \pi_2(p)/j]$ and $\coprod_{i : I} \coprod_{j : J} B$ are equivalent.
\end{prop}
\begin{proof}
We will prove this statement for products; the case of coproducts is dual.
To do this, it is enough to show that $\prod_{i : I} \prod_{j : J} B$ is a product of $\Gamma , p : \Sigma_{i : I} J \mid \cdot \vdash B[\pi_1(p)/i, \pi_2(p)/j] \ob$.
We define projections as follows:
\[ \lambda f.\,f\,(\pi_1(p))\,(\pi_2(p)) : \Hom(\prod_{i : I} \prod_{j : J} B, B[\pi_1(p)/i,\pi_2(p)/j]). \]
We need to show that the following map is an equivalence:
\begin{align*}
& \Hom(X, \prod_{i : I} \prod_{j : J} B) \to \prod_{p : \sum_{i : I} J} \Hom(X, B[\pi_1(p)/i,\pi_2(p)/j]) \\
& \lambda g.\,\lambda p.\,\lambda x.\,g\,x\,(\pi_1(p))\,(\pi_2(p)).
\end{align*}
Note that this map factors through the following maps:
\begin{align*}
\lambda g.\,\lambda i j.\,\lambda x.\,g\,x\,i\,j & : \Hom(X, \prod_{i : I} \prod_{j : J} B) \to \prod_{i : I} \prod_{j : J} \Hom(X,B) \\
\lambda h p.\,h\,(\pi_1(p))\,(\pi_2(p)) & : (\prod_{i : I} \prod_{j : J} \Hom(X,B)) \to \prod_{p : \sum_{i : I} J} \Hom(X, B[\pi_1(p)/i,\pi_2(p)/j]).
\end{align*}
The first map is an equivalence since $\prod_{i : I} \prod_{j : J} B$ is a product and the fact that the second map is an equivalence is an easy exercise in the ordinary type theory.
\end{proof}

The following proposition shows that the product of an empty family of types is the terminal object and the coproduct of such a family is initial:

\begin{prop}
Suppose that the base theory has the empty type $\bot$.
Let $\Gamma, i : \bot \mid \cdot \vdash B \ob$ be an indexed type.
Then $\prod_{i : \bot} B$ is terminal and $\coprod_{i : \bot} B$ is initial.
\end{prop}
\begin{proof}
Since $\Hom(P, \prod_{i : \bot} B)$ is equivalent to $\Pi_{i : \bot} \Hom(P,B)$ and $\Hom(\coprod_{i : \bot} B, P)$ is equivalent $\Pi_{i : \bot} \Hom(B,P)$,
the statement follows from the fact that $\Pi_{i : \bot} X$ is contractible for every base type $X$.
\end{proof}

The following proposition shows how to compute products and coproducts indexed by pushouts:

\begin{prop}
Suppose that the base theory has the following pushout:
\[ \xymatrix{ K \ar[r]^-g \ar[d]_-f & J \ar[d]^-{f'} \\
              I \ar[r]_-{g'}        & \po I \amalg_K J.
            } \]
Let $\Gamma, s : I \amalg_K J \mid \cdot \vdash B \ob$ be an indexed type.
Then we have the following canonical equivalences:
\begin{align*}
\prod_{s : I \amalg_K J} B & \simeq (\prod_{i : I} B[g' i/s]) \times_{(\prod_{k : K} B[g' (f k) / s])} (\prod_{j : J} B[f' j/s]) \\
\coprod_{s : I \amalg_K J} B & \simeq (\coprod_{i : I} B[g' i/s]) \amalg_{(\coprod_{k : K} B[f' (g k) / s])} (\coprod_{j : J} B[f' j/s]).
\end{align*}
\end{prop}
\begin{proof}
We will construct the first equivalence; the second is its dual.
First, let us define maps that appears in the pullback in the statement of this proposition.
The map $\Hom(\prod_{i : I} B[g' i/s], \prod_{k : K} B[g' (f k) / s])$ is defined as $\lambda p.\,\lambda k.\,p\,(f\,k)$.
One of the constructors of the pushout $I \amalg_K J$ gives us a map $h : \Pi_{k : K} \Id(f'\,(g\,k),g'\,(f\,k))$.
The map $\Hom(\prod_{j : J} B[f' j/s], \prod_{k : K} B[g' (f k) / s])$ is defined as $\lambda p.\,\lambda k.\,(h\,k)_*(p\,(g\,k))$.

Now, we need to prove that, for every type $P$, the type $\Hom(P, \prod_{s : I \amalg_K J} B)$ is the weak $\Pi$-type $\Pi_{s : I \amalg_K J} \Hom(P,B)$.
By the universal property of pullbacks, we have an equivalence between $\Hom(P, \prod_{s : I \amalg_K J} B)$ and the following type:
\[ \Hom(P, \prod_{i : I} B[g' i/s]) \times_{\Hom(P, \prod_{k : K} B[g' (f k) / s])} \Hom(P, \prod_{j : J} B[f' j/s]). \]
By the universal property of products, this type is equivalent to the following one:
\[ (\prod_{i : I} \Hom(P, B[g' i/s])) \times_{(\prod_{k : K} \Hom(P, B[g' (f k) / s]))} (\prod_{j : J} \Hom(P, B[f' j/s])). \]
The standard argument about pushouts shows that this type is equivalent to the weak $\Pi$-type $\Pi_{s : I \amalg_K J} \Hom(P,B)$.
\end{proof}

The following corollary shows how to compute tensoring by different type-theoretic constructions:

\begin{cor}
We have the following canonical equivalences:
\begin{align*}
\bot \cdot X & \simeq 0 \\
(I \amalg_K J) \cdot X & \simeq I \cdot X \amalg_{K \cdot X} J \cdot X \\
\top \cdot X & \simeq X \\
(\sum_{i : I} J) \cdot X & \simeq \coprod_{i : I} J \cdot X \\
(\Sigma I) \cdot 1 & \simeq \Sigma (I \cdot 1) \\
S^n \cdot 1 & \simeq S^n
\end{align*}
\end{cor}
\begin{proof}
The first four equivalences follow from previous propositions.
The equivalence for suspension follows from previous equations since suspension is defined in terms of pushouts and terminal types.
The last equivalence follows from previous since spheres are defined in terms of suspensions, coproducts, and terminal types.
\end{proof}

\section{Indexed dependent type theories}
\label{sec:dependent}

In this section, we define the dependent version of indexed type theories.

\subsection{Basic rules}
\label{sec:dep-rules}

\emph{Indexed dependent type theories} have four kinds of judgments:
\[ \Gamma \vdash A \type \qquad \Gamma \vdash a : A \qquad \Gamma \mid \Delta \vdash B \ob \qquad \Gamma \mid \Delta \vdash b : B \]

In each of these judgments, $\Delta$ is an indexed context, that is a sequence of the form $y_1 : B_1, \ldots y_k : B_k$, where $B_1$, \ldots $B_k$ are indexed types and $y_1$, \ldots $y_k$ are pairwise distinct variables.
The base theory has the same rules as the base theory in indexed unary type theories.
The indexed theory has the following rules:
\begin{center}
\AxiomC{}
\UnaryInfC{$\Gamma \mid x_1 : A_1, \ldots x_n : A_n \vdash x_i : A_i$}
\DisplayProof
\end{center}

\begin{center}
\def\extraVskip{1pt}
\Axiom$\fCenter \Gamma \mid \Delta \vdash b_1 : B_1$
\noLine
\UnaryInf$\fCenter \ldots$
\noLine
\UnaryInf$\fCenter \Gamma \mid \Delta \vdash b_k : B_k[b_1/y_1, \ldots b_{k-1}/y_{k-1}]$
\Axiom$\fCenter \Gamma \mid \Delta, y_1 : B_1, \ldots y_k : B_k \vdash C \type$
\def\extraVskip{2pt}
\BinaryInfC{$\Gamma \mid \Delta \vdash C[b_1/y_1, \ldots b_k/y_k] \type$}
\DisplayProof
\end{center}

\begin{center}
\def\extraVskip{1pt}
\Axiom$\fCenter \Gamma \mid \Delta \vdash b_1 : B_1$
\noLine
\UnaryInf$\fCenter \ldots$
\noLine
\UnaryInf$\fCenter \Gamma \mid \Delta \vdash b_k : B_k[b_1/y_1, \ldots b_{k-1}/y_{k-1}]$
\Axiom$\fCenter \Gamma \mid \Delta, y_1 : B_1, \ldots y_k : B_k \vdash c : C$
\def\extraVskip{2pt}
\BinaryInfC{$\Gamma \mid \Delta \vdash c[b_1/y_1, \ldots b_k/y_k] : C[b_1/y_1, \ldots b_k/y_k]$}
\DisplayProof
\end{center}

We also have the usual equations for substitution:
\begin{align*}
y_i[b_1/y_1, \ldots b_k/y_k] & = b_i \\
c[y_1/y_1, \ldots y_k/y_k] & = c \\
d[c_1/z_1, \ldots c_n/z_n][b_1/y_1, \ldots b_k/y_k] & = d[c_1'/z_1, \ldots c_n'/z_n],
\end{align*}
where $c_i' = c_i[b_1/y_1, \ldots b_k/y_k]$.

We can also substitute base terms into indexed types and terms:
\begin{center}
\def\extraVskip{1pt}
\Axiom$\fCenter \Gamma \vdash b_1 : B_1$
\noLine
\UnaryInf$\fCenter \ldots$
\noLine
\UnaryInf$\fCenter \Gamma \vdash b_k : B_k[b_1/y_1, \ldots b_{k-1}/y_{k-1}]$
\Axiom$\fCenter \Gamma, y_1 : B_1, \ldots y_k : B_k \mid \Delta \vdash C \ob$
\def\extraVskip{2pt}
\BinaryInfC{$\Gamma \mid \Delta[b_1/y_1, \ldots b_k/y_k] \vdash C[b_1/y_1, \ldots b_k/y_k] \ob$}
\DisplayProof
\end{center}

\begin{center}
\def\extraVskip{1pt}
\Axiom$\fCenter \Gamma \vdash b_1 : B_1$
\noLine
\UnaryInf$\fCenter \ldots$
\noLine
\UnaryInf$\fCenter \Gamma \vdash b_k : B_k[b_1/y_1, \ldots b_{k-1}/y_{k-1}]$
\Axiom$\fCenter \Gamma, y_1 : B_1, \ldots y_k : B_k \mid \Delta \vdash c : C$
\def\extraVskip{2pt}
\BinaryInfC{$\Gamma \mid \Delta[b_1/y_1, \ldots b_k/y_k] \vdash c[b_1/y_1, \ldots b_k/y_k] : C[b_1/y_1, \ldots b_k/y_k]$}
\DisplayProof
\end{center}

These operations satisfy the following equations for all base terms $b_1$, \ldots $b_k$ and indexed terms $c_1$, \ldots $c_n$:
\begin{align*}
x[b_1/y_1, \ldots b_k/y_k] & = x \\
d[c_1/z_1, \ldots c_n/z_n][b_1/y_1, \ldots b_k/y_k] & = d[c_1'/z_1, \ldots c_n'/z_n],
\end{align*}
where $c_i' = c_i[b_1/y_1, \ldots b_k/y_k]$.

These operations satisfy the following equations for all base terms $b_1$, \ldots $b_k$, $c$, $c_1$, \ldots $c_n$:
\begin{align*}
c[y_1/y_1, \ldots y_k/y_k] & = c \\
d[c_1/z_1, \ldots c_n/z_n][b_1/y_1, \ldots b_k/y_k] & = d[c_1'/z_1, \ldots c_n'/z_n],
\end{align*}
where $c_i' = c_i[b_1/y_1, \ldots b_k/y_k]$.

We always assume that all constructions are stable under substitution of base terms.
Substitution of base terms corresponds to reindexing in indexed categories, so this means that all constructions are stable under reindexing.
We will assume that all constructions (which are defined in every context) are stable under substitution of indexed terms.
This operation corresponds to pulling back along the morphism corresponding to these terms.

We will consider examples of models of indexed theories in which some of the constructions are not stable under pullbacks.
For example, the initial type is not stable under pullbacks in models corresponding to stable $\infty$-categories (unless it is the trivial $\infty$-category).
We still can use initial types, but the elimination principle should be restricted (see section~\ref{sec:dep-initial}).

If we want to work with more complicated constructions which are not stable under pullbacks such as pushouts, then the type former itself should be restricted to the empty context.
The substitution cannot be applied to such constructions, but the weakening is still available for them.
This means that we can use such constructions in every context as usual, but the type former can be applied only to closed types and terms (that is, no indexed variables can appear in it, but base variables are still available).
See section~\ref{sec:dep-pushouts} for an example of such a construction.

We will always assume that the following rule is derivable:
\begin{center}
\def\extraVskip{1pt}
\Axiom$\fCenter \Gamma \vdash a : A$
\noLine
\UnaryInf$\fCenter \Gamma \vdash a' : A$
\noLine
\UnaryInf$\fCenter \Gamma \vdash t : \Id_A(a,a')$
\Axiom$\fCenter \Gamma, x : A, p : \Id_A(a,x), \Delta \mid E \vdash D \ob$
\noLine
\UnaryInf$\fCenter \Gamma, \Delta[a/x,\refl(a)/p] \mid E[a/x,\refl(a)/p] \vdash d : D[a/x,\refl(a)/p]$
\def\extraVskip{2pt}
\BinaryInfC{$\Gamma, \Delta[a'/x,t/p] \mid E[a'/x,t/p] \vdash J(a, x p \Delta E.\,D, \Delta E.\,d, a', t) : D[a'/x,t/p]$}
\DisplayProof
\end{center}

\[ J(a, x p \Delta E.\,D, \Delta E.\,d, a, \refl(a)) = d \]

It allows us to define $\pmap$ and transport in the same way as usual.
Moreover, we can define the following operation:
\begin{center}
\AxiomC{$\Gamma \vdash p : \Id_{\Hom(A,B)}(f,g)$}
\AxiomC{$\Gamma \mid \Delta \vdash a : A$}
\BinaryInfC{$\Gamma \mid \Delta \vdash \fs{hap}(p,a) : \Id_B(f\,a,g\,a)$}
\DisplayProof
\end{center}
It is defined as follows:
\[ \fs{hap}(p,a) = \pmap(f.\,f\,a,p). \]

The rules of indexed unary type theories are true in indexed dependent type theories.
This means that unary theories can be interpreted in dependent theories.
This implies that every model of an indexed dependent type theory is a model of the corresponding unary theory
(that is, there is a forgetful functor from the category of models of an indexed dependent theory to the category of models of an indexed unary theory).
Every model of an ordinary dependent type theory (that is, a small contextual category) is a model of an indexed dependent type theory.
This follows from the fact that indexed type theories can be interpreted in ordinary dependent type theories.

Judgments $\Gamma \mid \Delta \vdash A \ob$ and $\Gamma \mid \Delta \vdash a : A$ are interpreted as $\Gamma, \Delta \vdash A \type$ and $\Gamma, \Delta \vdash a : A$, respectively.
All the rules of indexed dependent type theories correspond to some rules of ordinary dependent type theories.
This interpretation will be called \emph{the canonical indexing of a dependent type theory over itself}.
It is analogous to the canonical indexing of a Cartesian category over itself.

\subsection{Models}
\label{sec:models}

A \emph{model} of an indexed dependent type theory is a model $\cat{B}$ of the underlying base theory together with a contravariant functor $F$ from the underlying category of $\cat{B}$ to the category of small contextual categories (with additional structure, depending on the constructions in the indexed theory).
We will show that every model of an indexed dependent type theory with unit types, $\Sigma$-types, and identity types gives rise to an indexed $\infty$-category.

\begin{remark}
We believe that every model of a locally small indexed unary type theory also gives rise to an indexed $\infty$-category, but this is harder to prove.
\end{remark}

First, recall that a \emph{homotopical category} is a category equipped with a class of maps, called weak equivalences, containing identity morphisms and satisfying the 2-out-of-6 property.
A standard way to construct an $\infty$-category out of a homotopical category is to apply the simplicial localization functor, the fibrant replacement functor, and the simplicial nerve functor.
We will denote the composite of these three functors by $L$.
A \emph{homotopical functor} is a functor between homotopical categories preserving weak equivalences.
A homotopical functor $F : \cat{C} \to \cat{D}$ induces a functor $L(F) : L(\cat{C}) \to L(\cat{D})$.

A fibration category is a homotopical category equipped with a class of maps, called \emph{fibrations}, satisfying certain axioms.
It was shown in \cite{tt-fibr-cat} that the category of models of a dependent type theory with $\Sigma$-types and identity types carries the structure of a fibration category.
An \emph{exact functor} is a functor between fibration categories preserving fibrations, terminal objects, and pullbacks along fibrations.
An exact functor induces an equivalence of homotopy categories if and only if it induces an equivalence of simplicial localizations \cite[Th\'eor\`eme~3.25]{cis10b}.

Now, let $F : \cat{B}^\fs{op} \to \fs{ConCat}$ be a model of an indexed dependent type theory with unit types, $\Sigma$-types, and identity types.
The underlying category of $\cat{B}$ is a homotopical category.
A map $f$ of $\cat{B}$ is a weak equivalence if it is an equivalence of contexts in the usual sense.
For every object $\Gamma$ of $\cat{B}$, let $U(F(\Gamma))$ be the underlying homotopical category of $F(\Gamma)$.
Then we have a functor $LUF : \cat{B}^\fs{op} \to \qCat$, where $\qCat$ is the category of small quasicategories.
This category can be extended to a fibrant simplicial category by defining $\Hom_\qCat(X,Y)$ to be the largest Kan complex contained in $Y^X$.
The quasicategory $\fs{Cat}_\infty$ of small quasicategories is defined as the simplicial nerve of $\qCat$.
Then we have a functor $N(LUF) : N(\cat{B})^\fs{op} \to \fs{Cat}_\infty$.
To prove that it extends to a functor $L(\cat{B})^\fs{op} \to \fs{Cat}_\infty$, it is enough show that every weak equivalence $f : X \to Y$ of $\cat{B}^\fs{op}$ is mapped to an equivalence of quasicategories $LUF(f) : LUF(X) \to LUF(Y)$.
Since $F(f)$ is an exact functor between fibration categories, to prove that it induces an equivalence of localizations, it is enough to prove that it induces an equivalence of homotopy categories.
This is obvious since $f$ has a homotopy inverse $f^{-1}$ and functor $F(f^{-1})$ is the inverse of $F(f)$ at the level of homotopy categories.

A general method of constructing models of indexed type theories is described in \cite{indexed-models}.
We already discussed two examples of models in section~\ref{sec:unary-models}.
These models are actually models of an indexed dependent type theory.
Indexed dependent types are interpreted as categorical fibrations (in the quasicategorical sense) in $\qCat$.
Indexed dependent types are interpreted as injective fibrations in $\sSpace$.

\subsection{Dependent $\Hom$-types}

Since every indexed dependent type theory has an underlying indexed unary type theory, the extensions that we discussed in the context of unary theories also applies to dependent versions.
Note that these constructions apply only to closed indexed types.
Sometimes we can extend the notion, so that it applies to indexed types in a non-empty context.

For example, there is a notion of locally small indexed dependent type theory.
We can also define the following dependent version of $\Hom$-types:
\begin{center}
\AxiomC{$\Gamma \mid \Delta \vdash B \ob$}
\UnaryInfC{$\Gamma \vdash \Hom(\Delta.B) \type$}
\DisplayProof
\qquad
\AxiomC{$\Gamma \mid \Delta \vdash b : B$}
\UnaryInfC{$\Gamma \vdash \lambda \Delta.\,b : \Hom(\Delta.B)$}
\DisplayProof
\end{center}
\medskip

\begin{center}
\AxiomC{$\Gamma \vdash f : \Hom(\Delta.B)$}
\AxiomC{$\Gamma \mid E \vdash a_1 : A_1\ \ldots\ \Gamma \mid E \vdash a_k : A_k[a_1/x_1, \ldots a_{k-1}/x_{k-1}]$}
\BinaryInfC{$\Gamma \mid E \vdash f\,a_1\,\ldots\,a_k : B[a_1/x_1, \ldots a_k/x_k]$}
\DisplayProof
\end{center}
where $\Delta = x_1 : A_1, \ldots x_k : A_k$.

\begin{align*}
(\lambda x_1 \ldots x_k.\,b)\,a_1\,\ldots\,a_k & = b[a_1/x_1, \ldots a_k/x_k] \\
\lambda x_1 \ldots x_k.\,f\,x_1\,\ldots\,x_k & = f
\end{align*}
If $\Delta$ is empty, then we will write $\Hom(\Delta.B)$ as $\Hom(B)$, abstraction as $\lambda(b)$, and the application operation as $f\,()$.

If a theory has dependent $\Hom$-types, then $J$ rule that we defined in section~\ref{sec:dep-rules} is derivable.
Indeed, if $E = y_1 : B_1, \ldots y_k : B_k$, then we can define $J(a, x p \Delta E.\,D, \Delta E.\,d, a', t)$ in terms of $J$ for base types:
\[ J(a, x p \Delta.\,\Hom(E.D), \Delta.\,\lambda E.\,d, a', t)\,y_1\,\ldots\,y_k, \]

If we have dependent $\Hom$-types, then we can define the following function:
\[ \lambda p.\,\lambda x.\,\fs{hap}(p,x) : \Id_{\Hom(A,B)}(f,g) \to \Hom(A, x.\,\Id_B(f\,x,g\,x)). \]
We will say that $\Hom$-types are \emph{extensional} if this function is an equivalence.
If we do not have dependent $\Hom$-types, we can replace this condition with the following constructions:
\begin{center}
\AxiomC{$\Gamma \mid x : A \vdash p : \Id_{B}(f\,x,g\,x)$}
\UnaryInfC{$\Gamma \vdash \Idext(x.\,p) : \Id_{\Hom(A,B)}(f,g)$}
\DisplayProof
\end{center}
\medskip

\begin{center}
\AxiomC{$\Gamma \mid x : A \vdash p : \Id_{B}(f\,x,g\,x)$}
\AxiomC{$\Gamma \mid \Delta \vdash a : A$}
\BinaryInfC{$\Gamma \mid \Delta \vdash \fs{hap_h}(x.\,p,a) : \Id_B(\fs{hap}(\Idext(x.\,p),a),p[a/x])$}
\DisplayProof
\end{center}
\medskip
The standard argument implies that we also have the following homotopy:
\begin{center}
\AxiomC{$\Gamma \vdash p : \Id_{\Hom(A,B)}(f,g)$}
\UnaryInfC{$\Gamma \mid \Delta \vdash \fs{hap_h'}(p) : \Id(\Idext(x.\,\fs{hap}(p,x)),p)$}
\DisplayProof
\end{center}
Thus, in the absence of dependent $\Hom$-types, we will say that $\Hom$-types are \emph{extensional} if these rules are derivable.
Clearly, the two conditions are equivalent when we do have dependent $\Hom$-types.

\begin{example}
The canonical indexing of a dependent type theory over itself is locally small if and only if it has non-dependent function types.
It has (extensional) dependent $\Hom$-types if and only if it has (extensional) $\Pi$-types.
\end{example}

\begin{example}
The model $\qCat$ has extensional dependent $\Hom$-types, identity types, $\Sigma$-types, and unit types.
\end{example}

\subsection{Weak dependent $\Hom$-types}

If the indexed theory is locally small and has identity types, $\Sigma$-types, unit types, and extensional $\Hom$-types, then we can define a weak version of dependent $\Hom$-types:
\[ \Hom(\Delta.B) = \sum_{f : \Hom(\Sigma(\Delta),\Sigma_{p : \Sigma(\Delta)} B[\pi_1(p)/x_1, \ldots \pi_k(p)/x_k])} \Id(\pi_1 \circ f, \id_{\Sigma(\Delta)}), \]
where $\Delta = x_1 : A_1, \ldots x_k : A_k$ and $\Sigma(\Delta)$ is defined inductively:
\begin{align*}
\Sigma(\cdot) & = \top \\
\Sigma(x : A, \Delta) & = \sum_{x : A} \Sigma(\Delta).
\end{align*}
The abstraction is defined as follows:
\[ \lambda x_1 \ldots x_k.\,b = (\lambda p.\,(p, b[\pi_1(p)/x_1, \ldots \pi_k(p)/x_k]), \refl(\id_{\Sigma(\Delta)})). \]
The application is defined as follows:
\[ f\,a_1\,\ldots\,a_k = \fs{hap}(\pi_2(f),(a_1, \ldots a_k))_*(\pi_2(\pi_1(f)\,(a_1, \ldots a_k))). \]
The beta rule holds judgmentally, but the eta rule holds only propositionally.
Indeed, $\lambda x_1 \ldots x_k.\,f\,x_1\,\ldots\,x_k$ equals to
\[ (\lambda p.\,(p,\fs{hap}(\pi_2(f),p')_*(\pi_2(\pi_1(f)\,p'))), \refl(\id_{\Sigma(\Delta)})), \]
where $p' = (\pi_1(p), \ldots \pi_k(p))$.
To prove that it is homotopic to $f$, we need to construct a homotopy of the following type:
\[ h : \Id(\pi_1(f), \lambda p.\,(p,\fs{hap}(\pi_2(f),p')_*(\pi_2(\pi_1(f)\,p')))) \]
such that $h * \pi_1$ is homotopic to $\pi_2(f)$.
To construct such a homotopy, we can use $\Idext$.
Then we need to define two homotopies for every $p : \Sigma(\Delta)$:
\begin{align*}
h_1 & : \Id(\pi_1(\pi_1(f)\,p),p) \\
h_2 & : \Id((h_1)_*(\pi_2(\pi_1(f)\,p)),\fs{hap}(\pi_2(f),p')_*(\pi_2(\pi_1(f)\,p')))
\end{align*}
The condition that $h * \pi_1$ is homotopic to $\pi_2(f)$ is satisfied if we put $h_1 = \fs{hap}(\pi_2(f),p)$.
Finally, to construct $h_2$, it is enough to note that $p'$ is homotopic to $p$.

Thus, the theory of dependent $\Hom$-types is a slightly stricter version of the theory of $\Hom$-types.
This is similar to the theory of $\Pi$-types being a strict version of the theory of non-dependent function types.

We can define a dependent version of $\fs{hap}$:
\begin{center}
\AxiomC{$\Gamma \vdash p : \Id_{\Hom(\Delta,B)}(f,g)$}
\def\extraVskip{1pt}
\Axiom$\fCenter \Gamma \mid E \vdash a_1 : A_1$
\noLine
\UnaryInf$\fCenter \ldots$
\noLine
\UnaryInf$\fCenter \Gamma \mid E \vdash a_k : A_k[a_1/x_1, \ldots a_{k-1}/x_{k-1}]$
\def\extraVskip{2pt}
\BinaryInfC{$\Gamma \mid E \vdash \fs{hap}(p, a_1, \ldots a_k) : \Id_{B[a_1/x_1, \ldots a_k/x_k]}(f\,a_1\,\ldots\,a_k,g\,a_1\,\ldots\,a_k)$}
\DisplayProof
\end{center}
It is defined as follows:
\[ \fs{hap}(p, a_1, \ldots a_k) = J(f, h q.\,\Id(f\,a_1\,\ldots\,a_k,h\,a_1\,\ldots\,a_k), \refl(f\,a_1\,\ldots\,a_k), g, p). \]
It is straightforward to check that if $\Hom$-types are extensional, then the following function is an equivalence:
\begin{align*}
& \Id_{\Hom(x_1 \ldots x_k.B)}(f,g) \to \Hom(x_1 \ldots x_k.\,\Id_B(f\,x_1\,\ldots\,x_k,g\,x_1\,\ldots\,x_k)) \\
& \lambda p.\,\lambda x_1 \ldots x_k.\,\fs{hap}(p, x_1, \ldots x_k).
\end{align*}

\section{Locally Cartesian closed indexed theories}
\label{sec:lccc}

In this section, we will define locally Cartesian closed unary type theories and discuss the relationship between them and indexed dependent type theories with $\Pi$-types.

\subsection{Types over contexts}

Let $p_A : \Hom(A,D)$ and $p_B : \Hom(B,D)$ be a pair of maps with the same codomain.
We will write $\Hom_D(A,B)$ for the type $\Sigma_{f : \Hom(A,B)} \Id(p_B \circ f, p_A)$.
If we think of maps $\Hom(X,D)$ as types over $D$, then $\Hom_D(A,B)$ is the type of morphisms between such types.
We will identify elements of $\Hom_D(A,B)$ with underlying morphisms $\Hom(A,B)$ and we will often omit the homotopy in $\Hom_D(A,B)$ when constructing an element of this type.

We will say that two maps are equivalent if there is an equivalence between their domains and an equivalence between their codomains such that the obvious square commutes.
We will say that two dependent types $\Delta \vdash A$ and $\Delta' \vdash A'$ are equivalent if there is an equivalence of contexts $\Delta, x : A$ and $\Delta', x : A'$.
If the theory has $\Sigma$-types and unit types, then, for every dependent type $\Delta \vdash B \ob$, we can define the following map:
\[ \pi_1 : \Hom(\sum_{p : \Sigma(\Delta)} B[\pi_1(p)/x_1, \ldots \pi_k(p)/x_k]), \Sigma(\Delta)), \]
where $\Delta = x_1 : A_1, \ldots x_k : A_k$.
Conversely, if the theory also has identity types, then, for every map $f : \Hom(A,B)$, we can define the following dependent type:
\[ y : B \vdash \sum_{x : A} \Id_B(f\,x,y). \]
These constructions are mutually inverse up to equivalences described above.
This implies that every dependent type in an arbitrary context is equivalent to a type in a context of size 1.

If $A$ and $B$ are dependent types in a context $\Delta$ in an indexed dependent type theory, then we will write $\Hom_\Delta(A,B)$ for $\Hom((\Delta, x : A). B)$.
This type corresponds to the previously defined type of maps over $\Delta$ through the equivalence between dependent types and types over $\Sigma(\Delta)$.

\subsection{Exponentiable maps}

Let $p_A : \Hom(A,D)$ be a morphism in an indexed unary type theory such that its pullbacks along any map exist.
The \emph{exponent over $D$} of $p_A$ and a map $p_B : \Hom(B,D)$ is a map $p_{B^A} : \Hom(B^A,D)$ together with a map $\fs{ev} : \Hom_D(B^A \times_D A, B)$ such that the following function is an equivalence for every indexed type $X$:
\[ \lambda f.\, \fs{ev} \circ (f \times_D A) : \Hom_D(X, B^A) \to \Hom_D(X \times_D A, B). \]
We will say that $p_A$ is \emph{exponentiable} if the exponent $B^A$ exists for all maps $p_B$.
Formally, this means that the following rules (together with the rules asserting that the function given above is an equivalence) are derivable:
\begin{center}
\AxiomC{$\Gamma \mid \cdot \vdash B \ob$}
\AxiomC{$\Gamma \vdash p_B : \Hom(B,D)$}
\BinaryInfC{$\Gamma \mid \cdot \vdash B^A \ob$}
\DisplayProof
\qquad
\AxiomC{$\Gamma \mid \cdot \vdash B \ob$}
\AxiomC{$\Gamma \vdash p_B : \Hom(B,D)$}
\BinaryInfC{$\Gamma \vdash p_{B^A} : \Hom(B^A, D)$}
\DisplayProof
\end{center}

\medskip
\begin{center}
\AxiomC{$\Gamma \mid \cdot \vdash B \ob$}
\AxiomC{$\Gamma \vdash p_B : \Hom(B,D)$}
\BinaryInfC{$\Gamma \vdash \fs{ev}(B,p_B) : \Hom_D(B^A \times_D A, B)$}
\DisplayProof
\end{center}
Since $B^A$, $p_{B^A}$, and $\fs{ev}$ depends on $B$ and $p_B$, we should write each of these constructions with arguments as we did for $\fs{ev}$ in the rule above, but to simplify the notation, we will omit them.

We will say that an indexed unary type theory is \emph{locally Cartesian closed} if it has finite limits and all maps are exponentiable.
Formally, this means that the following modifications of the rules given above are derivable:
\begin{center}
\AxiomC{$\Gamma \mid \cdot \vdash A \ob$}
\AxiomC{$\Gamma \mid \cdot \vdash B \ob$}
\AxiomC{$\Gamma \vdash p_A : \Hom(A,D)$}
\AxiomC{$\Gamma \vdash p_B : \Hom(B,D)$}
\QuaternaryInfC{$\Gamma \mid \cdot \vdash B^A \ob$}
\DisplayProof
\end{center}

\medskip
\begin{center}
\AxiomC{$\Gamma \mid \cdot \vdash A \ob$}
\AxiomC{$\Gamma \mid \cdot \vdash B \ob$}
\AxiomC{$\Gamma \vdash p_A : \Hom(A,D)$}
\AxiomC{$\Gamma \vdash p_B : \Hom(B,D)$}
\QuaternaryInfC{$\Gamma \vdash p_{B^A} : \Hom(B^A, D)$}
\DisplayProof
\end{center}

\medskip
\begin{center}
\AxiomC{$\Gamma \mid \cdot \vdash A \ob$}
\AxiomC{$\Gamma \mid \cdot \vdash B \ob$}
\AxiomC{$\Gamma \vdash p_A : \Hom(A,D)$}
\AxiomC{$\Gamma \vdash p_B : \Hom(B,D)$}
\QuaternaryInfC{$\Gamma \vdash \fs{ev}(A,p_A,B,p_B) : \Hom_D(B^A \times_D A, B)$}
\DisplayProof
\end{center}

We will say that an indexed type $A$ is \emph{exponentiable} if the unique map $!_A : \Hom(A,1)$ is exponentiable.
We will say that an indexed unary type theory is \emph{Cartesian closed} if it has a terminal object and binary products and all objects are exponentiable.

Dependent analogs of locally Cartesian closed theories are defined as usual.
First, let us describe non-dependent function types.
If $\Gamma \mid \Delta \vdash A \ob$ and $\Gamma \mid \Delta \vdash B \ob$ is a pair of indexed types, then the \emph{function type} $\Gamma \mid \Delta \vdash A \to B \ob$ is defined as follows:
\begin{center}
\AxiomC{$\Gamma \mid \Delta \vdash A \to B \ob$}
\AxiomC{$\Gamma \mid \Delta, x : A \vdash b : B$}
\BinaryInfC{$\Gamma \mid \Delta \vdash \lambda x.\,b : A \to B$}
\DisplayProof
\qquad
\AxiomC{$\Gamma \mid \Delta \vdash f : A \to B$}
\AxiomC{$\Gamma \mid \Delta \vdash a : A$}
\BinaryInfC{$\Gamma \mid \Delta \vdash f\,a : B$}
\DisplayProof
\end{center}

\begin{align*}
(\lambda x.\,b)\,a & = b[a/x] \\
\lambda x.\,f\,x & = f
\end{align*}
The \emph{$\Pi$-type} $\Pi_{x : A} B$ of a family $\Gamma \vdash \Delta, x : A \vdash B \ob$ is defined similarly:
\begin{center}
\AxiomC{$\Gamma \mid \Delta \vdash \Pi_{x : A} B \ob$}
\AxiomC{$\Gamma \mid \Delta, x : A \vdash b : B$}
\BinaryInfC{$\Gamma \mid \Delta \vdash \lambda x.\,b : \Pi_{x : A} B$}
\DisplayProof
\qquad
\AxiomC{$\Gamma \mid \Delta \vdash f : \Pi_{x : A} B$}
\AxiomC{$\Gamma \mid \Delta \vdash a : A$}
\BinaryInfC{$\Gamma \mid \Delta \vdash f\,a : B[a/x]$}
\DisplayProof
\end{center}

\begin{align*}
(\lambda x.\,b)\,a & = b[a/x] \\
\lambda x.\,f\,x & = f
\end{align*}

The last two equations are called $\beta$ and $\eta$ rules.
\emph{Weak function types} and \emph{weak $\Pi$-types} are defined in the same way except for the $\beta$ and $\eta$ rules which hold only propositionally.
We will denote them by $\beta(x.b,a)$ and $\eta(f)$:
\begin{align*}
\beta(x.b, a) & : \Id((\lambda x.\,b)\,a, b[a/x]) \\
\eta(f) & : \Id(\lambda x.\,f\,x, f)
\end{align*}

If an indexed dependent type theory has weak function types, then the underlying indexed unary type theory is locally Cartesian closed.
The converse does not hold.
We can construct weak $\Pi$-types from exponents, but they will not be stable under substitution.
We will show in \rprop{ccc} that weak $\Pi$-types for closed indexed types can be constructed if the underlying unary type theory is Cartesian closed.

\begin{example}
The model $\sSpace$ has $\Pi$-types, which implies that it is locally Cartesian closed.
\end{example}

\begin{example}[qcat-lim]
The model $\qCat$ has $\Pi$-types of closed indexed types (that is, indexed types in which no free indexed variables occur).
It cannot have all $\Pi$-types since the $\infty$-category of small $\infty$-categories is not locally Cartesian closed.
Nevertheless, we can use $\Pi$-types in the base theory to reason about indexed types.
Recall that we can think of closed indexed types as $\infty$-categories.
The base type of objects of an indexed type $C$ is defined as $\Hom(C)$ and we will denote it by $\Ob(C)$.
Let us demonstrate how we can use $\Pi$-types of the base theory to define the notion of a limit of a diagram.

First, we need to add an indexed type $\Delta^1$ together with terms $l,r : \Delta^1$ which is interpreted in the obvious way.
Now, for every closed indexed type $C$ and every pair $x,y : C$, we can define the type of morphisms of $C$ between $x$ and $y$ as $\Ob(\Sigma_{f : \Delta^1 \to C} \Id(f\,l, x) \times \Id(f\,r, y))$.
We will denote this type by $\chom_C(x,y)$.
Now, we can define the following predicate:
\begin{align*}
& \fs{isTerm} : \Ob(C) \to \fs{Prop} \\
& \fs{isTerm}(x) = \Pi_{y : \Ob(C)} \fs{isContr}(\chom_C(y,x))
\end{align*}
That is, we say that an object $x$ of $C$ is terminal if, for every object $y$, the type of morphisms between $y$ and $x$ is contractible.

Finally, if $F : J \to C$ is a diagram in $C$, then we can define the category of cones of $F$ as $\Sigma_{a : C} \Pi_{z : C} \chom_C(a,z)$ (we can use $\Pi$-types since $C$ is closed, see \rremark{pi-gen}).
We can define a limit of $F$ as a terminal object of the category of cones of $F$.
To prove various properties of limits, we need to put a Segal condition on indexed types.
This will allow us to develop the theory of $\infty$-categories in this synthetic context, but this is beyond the scope of this paper.
\end{example}

\emph{Functional extensionality} for (weak) $\Pi$-types is defined as usual:
\begin{center}
\AxiomC{$\Gamma \mid \Delta, x : A \vdash p : \Id_B(f\,x,g\,x)$}
\UnaryInfC{$\Gamma \mid \Delta \vdash \fs{funext}(x.p) : \Id_{\Pi_{x : A} B}(f,g)$}
\DisplayProof
\end{center}
\medskip

\begin{center}
\AxiomC{$\Gamma \mid \Delta \vdash p : \Id_{\Pi_{x : A} B}(f,g)$}
\UnaryInfC{$\Gamma \mid \Delta \vdash \fs{funext_h}(p) : \Id(\fs{funext}(x.\pmap(f.\,f\,x,p)), p)$}
\DisplayProof
\end{center}
\medskip

\begin{center}
\AxiomC{$\Gamma \mid \Delta, x : A \vdash p : \Id_B(f\,x,g\,x)$}
\AxiomC{$\Gamma \mid \Delta \vdash a : A$}
\BinaryInfC{$\Gamma \mid \Delta \vdash \fs{funext_h'}(x.p,a) : \Id(\pmap(f.\,f\,a,\fs{funext}(x.p)), p[a/x])$}
\DisplayProof
\end{center}

\begin{lem}[dep-exp]
Let $\Delta = y_1 : A_1, \ldots y_k : A_k$ be a context and let $\Gamma \mid \Delta \vdash \Pi_{x : A} B \ob$ be a $\Pi$-type in this context.
Then the following function is an equivalence:
\[ \lambda f.\,\lambda \overline{y} x.\,f\,\overline{y}\,x : \Hom(\Delta. \Pi_{x : A} B) \to \Hom((\Delta, x : A). B). \]
\end{lem}
\begin{proof}
The inverse of this function is given by $\lambda g.\,\lambda \overline{y}.\lambda x.\,g\,\overline{y}\,x$:
\begin{align*}
(\lambda f.\,\lambda \overline{y} x.\,f\,\overline{y}\,x) \circ (\lambda g.\,\lambda \overline{y}.\lambda x.\,g\,\overline{y}\,x) & = \\
\lambda g.\,\lambda \overline{y} x.\,(\lambda \overline{y}.\lambda x.\,g\,\overline{y}\,x)\,\overline{y}\,x & = \\
\lambda g.\,\lambda \overline{y} x.\,(\lambda x.\,g\,\overline{y}\,x)\,x & \sim \\
\lambda g.\,\lambda \overline{y} x.\,g\,\overline{y}\,x & \sim \\
\lambda g.g &
\end{align*}
\begin{align*}
(\lambda g.\,\lambda \overline{y}.\lambda x.\,g\,\overline{y}\,x) \circ (\lambda f.\,\lambda \overline{y} x.\,f\,\overline{y}\,x) & = \\
\lambda f.\,\lambda \overline{y}.\lambda x.\,(\lambda \overline{y} x.\,f\,\overline{y}\,x)\,\overline{y}\,x & = \\
\lambda f.\,\lambda \overline{y}.\lambda x.\,f\,\overline{y}\,x & \sim \\
\lambda f.\,\lambda \overline{y}.\,f\,\overline{y} & \sim \\
\lambda f.f &
\end{align*}
\end{proof}

\begin{prop}
Functional extensionality holds for all $\Pi$-types that exist in an indexed dependent type theory with extensional $\Hom$-types.
\end{prop}
\begin{proof}
Let $\Delta = y_1 : A_1, \ldots y_k : A_k$ be a context and let $\Gamma \mid \Delta \vdash f' : \Pi_{x : A} B$ and $\Gamma \mid \Delta \vdash g' : \Pi_{x : A} B$ be a pair of functions in this context.
Consider the following square:
\[ \xymatrix{ \Id_{\Hom(\Delta. \Pi_{x : A} B)}(f',g') \ar[r] \ar[d] & \Id_{\Hom((\Delta, x : A). B)}(\lambda \overline{y} x.\,f'\,\overline{y}\,x, \lambda \overline{y} x.\,g'\,\overline{y}\,x) \ar[d] \\
              \Hom(\Delta.\,\Id_{\Pi_{x : A} B}(f'\,\overline{y}, g'\,\overline{y})) \ar[r] & \Hom((\Delta, x : A).\,\Id_B(f'\,\overline{y}\,x, g'\,\overline{y}\,x))
            } \]
The bottom map is defined as $\lambda s.\,\lambda \overline{y} x.\,\pmap(f.\,f\,x,s\,\overline{y})$.
The left and right maps are defined as $\fs{hap}$.
These functions are equivalences since $\Hom$-types are extensional.
The top map is defined as $\lambda p.\,\pmap(f.\,\lambda \overline{y} x.\,f\,\overline{y}\,x,p)$.
\Rlem{dep-exp} implies that this function is also an equivalence.
Since the functions appearing in this square preserve $\refl$, path induction implies that it commutes.
Thus, the bottom map is an equivalence.
Let $r(f',g')$ be its inverse.

If $\Gamma \mid \Delta, x : A \vdash p : \Id(f\,x,g\,x)$, then we define $\fs{funext}(x.p)$ as follows:
\[ \Gamma \mid \Delta \vdash r(\lambda \overline{y}.f,\lambda \overline{y}.g)\,(\lambda \overline{y} x.p)\,\overline{y} : \Id_{\Pi_{x : A} B}(f,g). \]
It is easy to define $\fs{funext_h}$ and $\fs{funext_h'}$ using the fact that $r$ is an inverse of the function $\lambda s.\,\lambda \overline{y} x.\,\pmap(f.\,f\,x,s\,\overline{y})$.
\end{proof}

\begin{prop}[ccc]
The following conditions are equivalent in any indexed dependent type theory with unit types, $\Sigma$-types, identity types, and extensional $\Hom$-types:
\begin{enumerate}
\item Weak $\Pi$-types exist for all closed types.
That is, we have the following rule (with the usual rules for $\lambda$ and application and propositional $\beta$ and $\eta$ rules):
\begin{center}
\AxiomC{$\Gamma \mid \cdot \vdash A \ob$}
\AxiomC{$\Gamma \mid x : A \vdash B \ob$}
\BinaryInfC{$\Gamma \mid \Delta \vdash \Pi_{x : A} B \ob$}
\DisplayProof
\end{center}
\item Weak function types exist for all closed types.
\end{enumerate}
If these conditions hold, then the underlying indexed unary type theory is Cartesian closed.
\end{prop}
\begin{proof}
Clearly, the existence of $\Pi$-types implies the existence of function types.
Let us prove the converse.
We define $\Pi_{x : A} B$ as follows:
\[ \sum_{f : A \to \Sigma_{x : A} B} \Id_{A \to A}(\lambda x.\,\pi_1\,(f\,x), \lambda x.\,x). \]
If $\Gamma \mid \Delta, x : A \vdash b : B$, then we define $\lambda x.b$ as follows:
\[ (\lambda x.(x,b), \fs{funext}(x. h(b,x))), \]
where $h(b,x) = \beta(x.\,\pi_1\,((\lambda x.(x,b))\,x),x) \ct \pmap(p.\,\pi_1(p), \beta(x.(x,b),x)) \ct \sym{\beta(x.x,x)}$.
We will denote this pair by $\lambda^d x.b$ to distinguish it from the non-dependent $\lambda$.
If $\Gamma \mid \Delta \vdash p : \Pi_{x : A} B$ and $\Gamma \mid \Delta \vdash a : A$, then we define $p\,a$ as follows:
\[ h'(p,a)_*(\pi_2(\pi_1(p)\,a)), \]
where $h'(p,a) = \sym{\beta(x.\,\pi_1\,(\pi_1(p)\,x),a)} \ct \pmap(f.\,f\,a,\pi_2(p)) \ct \beta(x.x,a)$.

Let us prove that $(\lambda^d x.b)\,a = h'(\lambda^d x.b,a)_*((\lambda x.(x,b))\,a)$ is homotopic to $b[a/x]$.
First, note that we have the following homotopy:
\[ \beta(x.(x,b),a) : \Id_{\Sigma_{x : A} B}((\lambda x.(x,b))\,a,(a,b[a/x])). \]
A standard argument implies that $\pmap(p.\,\pi_1(p),\beta(x.(x,b),a))_*((\lambda x.(x,b))\,a)$ is homotopic to $b[a/x]$.
Thus, we just need to prove that $\pmap(p.\,\pi_1(p),\beta(x.(x,b),a))$ is homotopic to $h'(\lambda^d x.b,a)$:
\begin{align*}
h'(\lambda^d x.b,a) & = \\
\sym{\beta(x.\,\pi_1\,((\lambda x.(x,b))\,x),a)} \ct \pmap(f.\,f\,a,\fs{funext}(x. h(b,x))) \ct \beta(x.x,a) & \sim \\
\sym{\beta(x.\,\pi_1\,((\lambda x.(x,b))\,x),a)} \ct h(b,a) \ct \beta(x.x,a) & \sim \\
\pmap(p.\,\pi_1(p), \beta(x.(x,b),a)) & .
\end{align*}
The first homotopy follows from $\fs{funext_h'}$ and the second one from the definition of $h(b,a)$.

Now, let $(f,q)$ be a term of type $\Pi_{x : A} B$.
Let us prove that $\lambda^d x.\,(f,q)\,x$ is homotopic to $(f,q)$.
Let $b = h'((f,q),x)_*(\pi_2(f\,x))$.
Then
\[ \lambda^d x.\,(f,q)\,x = (\lambda x.(x,b),\fs{funext}(x.h(b,x))). \]
By $\Sigma$-extensionality, it is enough to construct a homotopy $t$ between $f$ and $\lambda x.(x,b)$ such that $t_*(q)$ is homotopic to $\fs{funext}(x.h(b,x))$.
We can define $t$ as follows:
\[ \fs{funext}(x. \Sigma \fs{ext}(h'((f,q),x),\refl) \ct \sym{\beta(x.(x,b),x)}). \]
Note that $t_*(q)$ is homotopic $\sym{\fs{funext}(x.s)} \ct q$, where
\[ s = \beta(x.\,\pi_1(f\,x),x) \ct \pmap(p.\,\pi_1(p\,x),t) \ct \sym{\beta(x.\,\pi_1((\lambda x.(x,b))\,x),x)}. \]
This is easy to prove by the path induction on $t$.
Thus, we just need to prove that $q$ is homotopic to $\fs{funext}(x.\,s \ct h(b,x))$.
By $\fs{funext_h}$, it is enough to prove that $\pmap(f.\,f\,x,q)$ is homotopic to $s \ct h(b,x)$.

We have the following sequence of homotopies:
\begin{align*}
\pmap(p.\,\pi_1(p\,x),t) & \sim \\
\pmap(p.\,\pi_1(p),\pmap(f.\,f\,x,t)) & = \\
\pmap(p.\,\pi_1(p),\pmap(f.\,f\,x, \fs{funext}(x. \Sigma \fs{ext}(h'((f,q),x),\refl) \ct \sym{\beta(x.(x,b),x)}))) & \sim \\
\pmap(p.\,\pi_1(p), \Sigma \fs{ext}(h'((f,q),x),\refl) \ct \sym{\beta(x.(x,b),x)}) & \sim \\
h'((f,q),x) \ct \sym{\pmap(p.\,\pi_1(p), \beta(x.(x,b),x))} & .
\end{align*}
This implies the existence of the required homotopy:
\begin{align*}
s \ct h(b,x) & \sim \\
\beta(x.\,\pi_1(f\,x),x) \ct \pmap(p.\,\pi_1(p\,x),t) \ct \pmap(p.\,\pi_1(p), \beta(x.(x,b),x)) \ct \sym{\beta(x.x,x)} & \sim \\
\beta(x.\,\pi_1(f\,x),x) \ct h'((f,q),x) \ct \sym{\beta(x.x,x)} & \sim \\
\pmap(f.\,f\,x,q) & .
\end{align*}
The first homotopy easily follows from the definitions of $s$ and $h(b,x)$.
The second homotopy follows from the sequence above.
The last homotopy follows from the definition of $h'((f,q),x)$.

Finally, let us prove that the existence of function types implies that the theory is Cartesian closed.
We define $B^A$ as $A \to B$.
The map $\fs{ev} : \Hom(B^A \times A, B)$ is defined as $\lambda p.\,\pi_1(p)\,(\pi_2(p))$.
The inverse of $\lambda f.\,\fs{ev} \circ (f \times A)$ is defined as follows:
\[ \lambda g. \lambda x. \lambda a.\, g\,(x,a) : \Hom(X \times A, B) \to \Hom(X, B^A). \]
\end{proof}

\begin{remark}[ccc-unstable]
The converse of \rprop{ccc} almost holds.
If the theory is Cartesian closed, then we can define $A \to B$ as $B^A$ and application $f\,a$ as $\fs{ev}(f,a)$.
We can use the universal property of $B^A$ to define $\lambda$.
The only problem is that $\lambda$ will not be stable under substitution.
\end{remark}

\begin{remark}[pi-gen]
It can be proved that exponential maps are closed under pullbacks.
In particular, if $A$ and $D$ are closed indexed types and $A$ is exponential, then $\pi_1 : D \times A \to D$ is also exponential.
We can express this in type-theoretic language by saying that there is the following (weak) $\Pi$-type:
\begin{center}
\AxiomC{$\Gamma \mid \cdot \vdash A \ob$}
\AxiomC{$\Gamma \mid \Delta, x : A \vdash B \ob$}
\BinaryInfC{$\Gamma \mid \Delta \vdash \Pi_{x : A} B \ob$}
\DisplayProof
\end{center}
We cannot prove that the existence of $\Pi$-types for closed types implies the existence of such $\Pi$-types because of the problem discussed in \rremark{ccc-unstable}.
In particular, we do not know whether such $\Pi$-types exist in $\qCat$.
Nevertheless, we can use them as in \rexample{qcat-lim}, but we need to be careful since $\lambda$ is not stable under substitution for such $\Pi$-types.
\end{remark}

\section{Limits and colimits in dependent theories}
\label{sec:colimits-dep}

In this section, we discuss the concepts of limits and colimits in indexed dependent type theories.

\subsection{Finite limits}

Clearly, an indexed dependent type theory has a terminal type if and only if it has a closed contractible type.
For example, this is true when it has the unit type.
If an indexed dependent type theory has identity types, $\Sigma$-types, and extensional $\Hom$-types, then it has pullbacks.
Indeed, we can define a pullback of maps $f : \Hom(A,C)$ and $g : \Hom(B,C)$ as $A \times_C B = \Sigma_{x : A} \Sigma_{y : B} \Id_C(f\,x,g\,y)$
with the obvious projections $\pi_1 : \Hom(A \times_C B, A)$, $\pi_2 : \Hom(A \times_C B, B)$ and the obvious homotopy between $f \circ \pi_1$ and $g \circ \pi_2$: namely, $\Idext(p.\,\pi_3(p))$.
We need to show that the following map is an equivalence:
\begin{align*}
& \Hom(P, A \times_C B) \to \sum_{F : \Hom(P,A)} \sum_{G : \Hom(P,B)} \Id(f \circ F, g \circ G) \\
& \lambda s.\,(\pi_1 \circ s, \pi_2 \circ s, s * \Idext(p.\,\pi_3(p))).
\end{align*}
The inverse of this map is defined as follows:
\[ \lambda t.\,\lambda p.\,(\pi_1(t)\,p, \pi_2(t)\,p, \fs{hap}(\pi_3(t),p)). \]
It is easy to see that these functions are inverse of each other using the fact that $s * \Idext(p.\,\pi_3(p)) = \Idext(p.\,\pi_3(s\,p))$.

\begin{remark}
The existence of $\Sigma$-types also implies the existence of products.
Thus, equalizers also can be constructed from $\Sigma$-types.
\end{remark}

The following proposition shows that $\Hom(x_1 \ldots x_n.\,-)$ commutes with $\Sigma$-types and identity types:
\begin{prop}[hom-sigma-id]
If an indexed dependent type theory has identity types, $\Sigma$-types, and extensional $\Hom$-types, then it has the following canonical equivalences:
\begin{align*}
\Hom(x_1 \ldots x_n.\,\Id_A(a,a')) & \simeq \Id_{\Hom(x_1 \ldots x_n. A)}(\lambda x_1 \ldots x_n.\,a, \lambda x_1 \ldots x_n.\,a') \\
\Hom(x_1 \ldots x_n.\,\sum_{y : A} B) & \simeq \sum_{f : \Hom(x_1 \ldots x_n. A)} \Hom(x_1 \ldots x_n.\,B[f\,x_1\,\ldots\,x_n / y]).
\end{align*}
\end{prop}
\begin{proof}
The first equivalence is simply the extensionality for $\Hom$-types.
The second equivalence is defined as follows:
\begin{align*}
& \lambda g.\,(\lambda x_1 \ldots x_n.\,\pi_1(g\,x_1\,\ldots\,x_n), \lambda x_1 \ldots x_n.\,\pi_2(g\,x_1\,\ldots\,x_n)) \\
& \lambda p.\,\lambda x_1 \ldots x_n.\,(\pi_1(p)\,x_1\,\ldots\,x_n, \pi_2(p)\,x_1\,\ldots\,x_n)
\end{align*}
It is easy to see that these functions are mutually inverse.
\end{proof}

\subsection{Initial types}
\label{sec:dep-initial}

Finite colimits can be defined in an indexed dependent type theory as higher inductive types.
We will say that the initial type $0$ in an indexed unary type theory is \emph{stable under pullbacks} if, for every type $B$ such that the product $B \times 0$ exists, this product is initial.

\begin{prop}[initial]
Let $0$ be a type in an indexed dependent type theory with identity types, $\Sigma$-types, and extensional $\Hom$-types.
Then the following conditions are equivalent:
\begin{enumerate}
\item The following rule is derivable:
\begin{center}
\AxiomC{$\Gamma \mid \Delta, x : 0, E \vdash D \type$}
\AxiomC{$\Gamma \mid \Delta \vdash a : 0$}
\BinaryInfC{$\Gamma \mid \Delta, E[a/x] \vdash 0\text{-}\fs{elim''}(x E.D,a) : D[a/x]$}
\DisplayProof
\end{center}
\item The following rule is derivable:
\begin{center}
\AxiomC{$\Gamma \mid \Delta \vdash D \type$}
\AxiomC{$\Gamma \mid \Delta \vdash a : 0$}
\BinaryInfC{$\Gamma \mid \Delta \vdash 0\text{-}\fs{elim'}(D,a) : D$}
\DisplayProof
\end{center}
\end{enumerate}
\end{prop}
If these equivalent conditions hold, then the type $0$ is a stable under pullbacks initial type.
\begin{proof}
The rule $0\text{-}\fs{elim'}$ is a special case of $0\text{-}\fs{elim''}$.
Conversely, if we have $0\text{-}\fs{elim'}$, then we can define $0\text{-}\fs{elim''}$ as follows:
\begin{center}
\AxiomC{$\Gamma \mid \Delta, x : 0, E \vdash D \type$}
\AxiomC{$\Gamma \mid \Delta \vdash a : 0$}
\BinaryInfC{$\Gamma \mid \Delta, E[a/x] \vdash 0\text{-}\fs{elim'}(D[a/x],a) : D[a/x]$}
\DisplayProof
\end{center}

Let us prove that these rules imply that $0$ is a stable under pullbacks initial type.
For every type $C$, we can define a map $\lambda x.\,0\text{-}\fs{elim'}(C,f\,x) : \Hom(B,C)$.
Let $g_1,g_2 : \Hom(B,C)$ be a pair of maps.
Then we can define a homotopy between them as follows:
\[ \Idext(x.\,0\text{-}\fs{elim'}(\Id_C(g_1\,x,g_2\,x),f\,x)) : \Id(g_1,g_2). \]
\end{proof}

\begin{remark}
The converse of \rprop{initial} almost holds.
If $0$ is a stable under pullbacks initial type, then we can define $0\text{-}\fs{elim'}$ in every context $\Delta$,
but these terms will not be stable under substitution of indexed terms.
\end{remark}

\begin{example}
Both models $\qCat$ and $\sSpace$ satisfy equivalent conditions of \rprop{initial}.
\end{example}

The theory of zero types does not satisfy conditions of \rprop{initial} unless it is trivial (that is, unless all indexed types are equivalent).
The reason is that the initial type in such theories is not stable under pullbacks.
To define an indexed dependent type theory in which the initial type is not necessarily stable under pullbacks, we need to modify the rules slightly.

\begin{prop}
Let $0$ be a type in an indexed dependent type theory with identity types, $\Sigma$-types, and extensional $\Hom$-types.
Then $0$ is initial if and only if the following rule is derivable:
\begin{center}
\AxiomC{$\Gamma \mid x : 0 \vdash D \type$}
\AxiomC{$\Gamma \mid \Delta \vdash a : 0$}
\BinaryInfC{$\Gamma \mid \Delta \vdash 0\text{-}\fs{elim}(x.D,a) : D[a/x]$}
\DisplayProof
\end{center}
\end{prop}
\begin{proof}
If $0$ is initial, then $\Gamma \mid x : 0 \vdash \fs{hap}(h,x)_*(\pi_2(f\,x)) : D$, where $f$ is the unique map $\Hom(0,\Sigma_{x : 0} D)$ and $h$ is the unique homotopy between $\pi_1 \circ f$ and $\id_0$.
Now, we can define $0\text{-}\fs{elim}(x.D,a)$ as $\fs{hap}(h,x)_*(\pi_2(f\,x))[a/x]$.

Conversely, suppose that the theory has the rule $0\text{-}\fs{elim}$.
Then, for every type $D$, we can define a map $\lambda x.\,0\text{-}\fs{elim}(y.D,x) : \Hom(0,D)$.
For all maps $f,g : \Hom(0,D)$, we can define a homotopy between them as follows:
\[ \Idext(x.\,0\text{-}\fs{elim}(y.\,\Id_D(f\,y,g\,y),x)) : \Id(f,g). \]
\end{proof}

\subsection{Dependent products}

We defined strict products in subsection~\ref{sec:products}.
We can define even stricter version of products, which we call \emph{strict dependent products}, in indexed dependent type theories:
\begin{center}
\AxiomC{$\Gamma, i : I \mid \Delta \vdash B \ob$}
\RightLabel{, $i \notin \mathrm{FV}(\Delta)$}
\UnaryInfC{$\Gamma \mid \Delta \vdash \prod_{i : I} B \ob$}
\DisplayProof
\qquad
\AxiomC{$\Gamma, i : I \mid \Delta \vdash b : B$}
\RightLabel{, $i \notin \mathrm{FV}(\Delta)$}
\UnaryInfC{$\Gamma \mid \Delta \vdash \lambda i.\,b : \prod_{i : I} B$}
\DisplayProof
\end{center}
\medskip

\begin{center}
\AxiomC{$\Gamma \mid \Delta \vdash f : \prod_{i : I} B$}
\AxiomC{$\Gamma \vdash j : I$}
\BinaryInfC{$\Gamma \mid \Delta \vdash f\,j : B[j/i]$}
\DisplayProof
\end{center}

\begin{align*}
(\lambda i.\,b)\,j & = b[j/i] \\
\lambda i.\,f\,i & = f
\end{align*}

\begin{example}
The canonical indexing of a dependent type theory over itself has strict dependent products if and only if it has $\Pi$-types.
\end{example}

\begin{example}
Both models $\qCat$ and $\sSpace$ have strict dependent products.
\end{example}

\subsection{Dependent coproducts}

Coproducts can be defined in a more type-theoretic way.
The theory of \emph{dependent coproducts} consists of the following rules:
\begin{center}
\AxiomC{$\Gamma, i : I \mid \cdot \vdash B_i \ob$}
\RightLabel{, $i \notin \mathrm{FV}(\Delta)$}
\UnaryInfC{$\Gamma \mid \Delta \vdash \coprod_{i : I} B_i \ob$}
\DisplayProof
\qquad
\AxiomC{$\Gamma \vdash j : I$}
\AxiomC{$\Gamma \mid \Delta \vdash b : B_j$}
\BinaryInfC{$\Gamma \mid \Delta \vdash (j,b) : \coprod_{i : I} B_i$}
\DisplayProof
\end{center}
\medskip

\begin{center}
\AxiomC{$\Gamma \mid z : \coprod_{i : I} B_i \vdash D \ob$}
\AxiomC{$\Gamma, i : I \mid x : B_i \vdash d : D[(i,x)/z]$}
\AxiomC{$\Gamma \mid \Delta \vdash e : \coprod_{i : I} B_i$}
\TrinaryInfC{$\Gamma \mid \Delta \vdash \coprod\text{-}\fs{elim}(z.D, i x.d, e) : D[e/z]$}
\DisplayProof
\end{center}
\medskip

\[ \coprod\text{-}\fs{elim}(z.D, i x.d, (j,b)) = d[j/i,b/x] \]

The theory of \emph{weak dependent coproducts} has the same rules except for the last equality which holds only propositionally.
Note that the coproduct is defined only for closed indexed types and that free indexed variables of the eliminator can occur only in its last argument.
The reason is that the unrestricted version forces coproducts to be stable under pullbacks as we will see in \rprop{coproduct-stable}.

The theory of (weak) dependent coproducts is more convenient than the theory of coproducts defined in section~\ref{sec:products} since the former allows us to use the type-theoretic language to reason about coproducts.
The following proposition shows that the theories of weak dependent coproducts and coproducts are actually equivalent.
This implies that we can use weak dependent coproducts to reason about pointed types, spectra, and other theories with coproducts.

\begin{prop}[dep-coprod-coprod]
Suppose that an indexed dependent type theory has $\Sigma$-types, identity types, and extensional $\Hom$-types.
Then it has weak dependent coproducts if and only if the underlying indexed unary type theory has coproducts.
\end{prop}
\begin{proof}
First, assume that the theory has weak dependent coproducts.
We will show that $\coprod_{i : I} B_i$ is the coproduct of $B_i$.
We define $\fs{in}_i$ as $\lambda b.(i,b)$.
Let $f$ be a map of the following form:
\[ \Gamma, i : I \vdash f : \Hom(B_i,C). \]
Then we define $\Gamma \vdash [f]_{i : I} : \Hom(\coprod_{i : I} B_i, C)$ as follows:
\[ [f]_{i : I} = \lambda e.\,\coprod\text{-}\fs{elim}(z.\,C, i x.\,f\,x, e). \]
Let us prove that $[f]_{i : I} \circ \fs{in}_i \sim f$:
\begin{align*}
[f]_{i : I} \circ \fs{in}_i & = \\
\lambda b.\,\coprod\text{-}\fs{elim}(z.\,C, i x.\,f\,x, (i,b)) & \sim \\
\lambda b.\,f\,b & = \\
f & .
\end{align*}
Finally, for every $g : \Hom(\coprod_{i : I} B_i, C)$, we need to prove that $[g \circ \fs{in}_i]_{i : I} \sim g$.
It is enough to prove that, for every $e : \coprod_{i : I} B_i$, there is a homotopy between $[g \circ \fs{in}_i]_{i : I}\,e$ and $g\,e$.
To do this, we can apply $\coprod\text{-}\fs{elim}$ to $e$.
Then we just need to construct a homotopy between $[g \circ \fs{in}_i]_{i : I}\,(i,x)$ and $g\,(i,x)$:
\begin{align*}
[g \circ \fs{in}_i]_{i : I}\,(i,x) & = \\
\coprod\text{-}\fs{elim}(z.C, i x.\,g\,(\fs{in}_i\,x), (i,x)) & \sim \\
g\,(\fs{in}_i\,x) & = \\
g\,(i,x) & .
\end{align*}

Now, let us prove the converse.
Assume that the theory has coproducts.
Then we define $(j,b)$ as $\fs{in}_j\,b$.
The eliminator $\coprod\text{-}\fs{elim}(z. D, i x. d, e)$ is defined as $\fs{hap}(h,e)_*(\pi_2\,(f\,e))$, where
\begin{align*}
f & : \Hom(\coprod_{i : I} B_i, \Sigma_{(z : \coprod_{i : I} B_i)} D) \\
f & = [\lambda x.\,((i,x),d)]_{i : I} \\
h & : \Id(\pi_1 \circ f, \fs{id})
\end{align*}
The existence of $h$ follows from the universal property of coproducts and the fact that $\pi_1 \circ f \circ \fs{in}_i \sim \fs{id} \circ \fs{in}_i$.
Finally, we need to show that
\[ \fs{hap}(h,(j,b)))_*(\pi_2\,(f\,(j,b))) \sim d[j/i,b/x]. \]
It is easy to see that $\fs{hap}(h,(j,b)) \sim \fs{ap}(\pi_1, \fs{hap}(\beta_j, b))$, where
\[ \beta_j : \Id(f \circ \fs{in}_j, \lambda b.\,((j,b), d[j/i,b/x])) \]
is one of the homotopies that appear in the definition of coproducts.
Now, the existence of the required homotopy follows from the fact that, for every $p : \Id_{\Sigma_{a : A} B}(x,y)$, there is a homotopy $\fs{ap}(\pi_1,p)_*(\pi_2(x)) \sim \pi_2(y)$.
\end{proof}

The theory of \emph{stable dependent coproducts} consists of the following rules:
\begin{center}
\AxiomC{$\Gamma, i : I \mid \Delta \vdash B_i \ob$}
\RightLabel{, $i \notin \mathrm{FV}(\Delta)$}
\UnaryInfC{$\Gamma \mid \Delta \vdash \coprod_{i : I} B_i \ob$}
\DisplayProof
\qquad
\AxiomC{$\Gamma \vdash j : I$}
\AxiomC{$\Gamma \mid \Delta \vdash b : B_j$}
\BinaryInfC{$\Gamma \mid \Delta \vdash (j,b) : \coprod_{i : I} B_i$}
\DisplayProof
\end{center}
\medskip

\begin{center}
\AxiomC{$\Gamma \mid \Delta, z : \coprod_{i : I} B_i \vdash D \ob$}
\AxiomC{$\Gamma, i : I \mid \Delta, x : B_i \vdash d : D[(i,x)/z]$}
\AxiomC{$\Gamma \mid \Delta \vdash e : \coprod_{i : I} B_i$}
\TrinaryInfC{$\Gamma \mid \Delta \vdash \coprod\text{-}\fs{elim}(z.D, i x.d, e) : D[e/z]$}
\DisplayProof
\end{center}
\medskip

\[ \coprod\text{-}\fs{elim}(z.D, i x.d, (j,b)) = d[j/i,b/x] \]

\begin{example}
The canonical indexing of a dependent type theory over itself always has dependent coproducts since we always assume that the base theory has $\Sigma$-types.
\end{example}

\begin{example}
The model $\qCat$ has stable dependent coproducts.
If we think of base types as $\infty$-groupoids and indexed types as $\infty$-categories, then tensoring $- \cdot 1$ is the functor that embeds $\infty$-groupoids into $\infty$-categories.
More generally, $X \cdot A$ is the product $(X \cdot 1) \times A$ (this is true in any Cartesian closed theory).
\end{example}

\begin{example}
The model $\sSpace$ also has stable dependent coproducts.
Tensoring $- \cdot 1$ embeds spaces into simplicial spaces as discrete objects.
\end{example}

\begin{prop}[coprod-sigma]
Suppose that an indexed dependent type theory has $\Sigma$-types, identity types, extensional $\Hom$-types, and (weak) stable dependent coproducts.
Then dependent coproducts commute with $\Sigma$-types.
More precisely, the following map is an equivalence for every dependent type $\Gamma, i : I \mid \Delta, d : D \vdash B_i \ob$:
\[ \lambda e. \coprod\text{-}\fs{elim}(z. \sum_{d : D} \coprod_{i : I} B_i, i p. (\pi_1(p),(i,\pi_2(p))), e) : \Hom_\Delta(\coprod_{i : I} \sum_{d : D} B_i, \sum_{d : D} \coprod_{i : I} B_i). \]
\end{prop}
\begin{proof}
The inverse of this map is defined as follows:
\[ \lambda p. \coprod\text{-}\fs{elim}(z. \coprod_{i : I} \sum_{d : D} B_i, i b.\,(i,(\pi_1(p),b)), \pi_2(p)). \]
It is easy to show that these maps are mutually inverse using the eliminator for coproducts.
\end{proof}

We will say that coproducts in an indexed unary type theory are \emph{stable under pullbacks} if, for every pair of maps $p : \Hom(\coprod_{i : I} B_i, D)$ and $r : \Hom(E,D)$,
the canonical map from $\coprod_{i : I} r^*(B_i)$ to $r^*(\coprod_{i : I} B_i)$ is an equivalence, where $r^*(X)$ is the pullback of $X$ along $r$.

\begin{prop}[coproduct-stable]
Suppose that an indexed dependent type theory has $\Sigma$-types, identity types, extensional $\Hom$-types, and (weak) stable dependent coproducts.
Then coproducts are stable under pullbacks.
\end{prop}
\begin{proof}
Let $\Gamma, i : I \mid \cdot \vdash B_i \ob$ be a dependent type and let $p : \Hom(\coprod_{i : I} B_i, D)$ and $r : \Hom(E,D)$ be maps.
We define $\Gamma, i : I \mid d : D \vdash B_i' \ob$ as the fiber of $B_i$ over $d$.
Then we have the following pullback squares:
\[ \xymatrix{ \coprod_{i : I} r^*(B_i) \ar[r] \ar[d]_\simeq \pb & \coprod_{i : I} B_i \ar[d]^\simeq \\
              \coprod_{i : I} \sum_{e : E} B_i'[r\,e/d] \ar[r] \ar[d]_\simeq \pb & \coprod_{i : I} \sum_{d : D} B_i' \ar[d]^\simeq \\
              \sum_{e : E} \coprod_{i : I} B_i'[r\,e/d] \ar[r] \ar[d] \pb & \sum_{d : D} \coprod_{i : I} B_i' \ar[d] \\
              E \ar[r] & D
            } \]
The bottom square is a pullback since substitutions correspond to pullbacks.
Vertical maps in the second row are equivalences by \rprop{coprod-sigma}.
Vertical maps in the first row are equivalences since $B_i \simeq \Sigma_{d : D} B_i'$ and $r^*(B_i) \simeq \Sigma_{e : E} B_i'[r\,e/d]$.
\end{proof}

\subsection{Pushouts}
\label{sec:dep-pushouts}

Pushouts can be defined in a more type-theoretic way.
The theory of \emph{weak dependent pushouts} consists of the following rules:
\medskip
\begin{center}
\AxiomC{$\Gamma \mid \cdot \vdash f : \Hom(A,B)$}
\AxiomC{$\Gamma \mid \cdot \vdash g : \Hom(A,C)$}
\BinaryInfC{$\Gamma \mid \Delta \vdash B \amalg_A C \ob$}
\DisplayProof
\end{center}
\medskip

\begin{center}
\AxiomC{$\Gamma \mid \Delta \vdash B \amalg_A C$}
\AxiomC{$\Gamma \mid \Delta \vdash b : B$}
\BinaryInfC{$\Gamma \mid \Delta \vdash \fs{inl}(b) : B \amalg_A C$}
\DisplayProof
\qquad
\AxiomC{$\Gamma \mid \Delta \vdash B \amalg_A C$}
\AxiomC{$\Gamma \mid \Delta \vdash c : C$}
\BinaryInfC{$\Gamma \mid \Delta \vdash \fs{inr}(c) : B \amalg_A C$}
\DisplayProof
\end{center}
\medskip

\begin{center}
\AxiomC{$\Gamma \mid \Delta \vdash B \amalg_A C$}
\AxiomC{$\Gamma \mid \Delta \vdash a : A$}
\BinaryInfC{$\Gamma \mid \Delta \vdash \fs{glue}(a) : \Id(\fs{inl}(f\,a),\fs{inr}(g\,a))$}
\DisplayProof
\end{center}
\medskip

\begin{center}
\def\extraVskip{1pt}
\Axiom$\fCenter \Gamma \mid w : B \amalg_A C \vdash D \ob$
\noLine
\UnaryInf$\fCenter \Gamma \mid \Delta \vdash e : B \amalg_A C$
\Axiom$\fCenter \Gamma \mid y : B \vdash d_1 : D[\fs{inl}(y)/w]$
\noLine
\UnaryInf$\fCenter \Gamma \mid z : C \vdash d_2 : D[\fs{inr}(z)/w]$
\noLine
\UnaryInf$\fCenter \Gamma \mid x : A \vdash d_3 : \Id(\fs{glue}(x)_*(d_1[f\,x/y]), d_2[g\,x/z])$
\def\extraVskip{2pt}
\BinaryInf$\fCenter \Gamma \mid \Delta \vdash \amalg\text{-}\fs{elim}(w.D, y.d_1, z.d_2, x.d_3, e) : D[e/w]$
\DisplayProof
\end{center}
\medskip

\begin{align*}
h_1(b) & : \Id(\amalg\text{-}\fs{elim}(w.D,y.d_1,z.d_2,x.d_3)[\fs{inl}(b)/w], d_1[b/y]) \\
h_2(c) & : \Id(\amalg\text{-}\fs{elim}(w.D,y.d_1,z.d_2,x.d_3)[\fs{inr}(c)/w], d_2[c/z])
\end{align*}
\medskip

\[ \xymatrix{ \fs{glue}(a)_*(\amalg\text{-}\fs{elim}(w.D,y.d_1,z.d_2,x.d_3)[\fs{inl}(f\,a)/w]) \ar@{=}[r] \ar@{=}[d] & \fs{glue}(a)_*(d_1[f\,a/y]) \ar@{=}[d]^{d_3[a/x]} \\
              \amalg\text{-}\fs{elim}(w.D,y.d_1,z.d_2,x.d_3)[\fs{inr}(g\,a)/w] \ar@{=}[r]_-{h_2(g\,a)} & d_2[g\,a/z]
            } \]
The last square must commute up to a homotopy $h_3(a)$.
The top arrow in this square is $\pmap(x.\,\fs{glue}(a)_*(x), h_1(f\,a))$ and the left arrow is defined by path induction on $\fs{glue}(a)$.

\begin{remark}
Of course, we can define a stricter version of this theory by replacing homotopies $h_1$ and $h_2$ with computational rules.
We will call such a theory simply the theory of \emph{dependent pushouts}.
\end{remark}

\begin{example}
The model $\qCat$ has dependent pushouts.
We can use pushouts to construct new examples of indexed types.
For example, if we have $\Delta^1$, then we can define $\Delta^n$ by induction as $\Delta^{n-1} \amalg_1 \Delta^1$.
\end{example}

\begin{prop}
Suppose that an indexed dependent type theory has $\Sigma$-types, identity types, and extensional $\Hom$-types.
Then it has weak dependent pushouts if and only if the underlying indexed unary type theory has pushouts.
\end{prop}
\begin{proof}
First, assume that the theory has weak dependent pushouts.
We will show that $B \amalg_A C$ is the pushout of $f : \Hom(A,B)$ and $g : \Hom(A,C)$.
We need to show that the following map is an equivalence:
\begin{align*}
F & : \Hom(B \amalg_A C, D) \to \sum_{b : \Hom(B,D)} \sum_{c : \Hom(C,D)} \Id(\lambda x.\,b\,(f\,x), \lambda x.\,c\,(g\,x)) \\
F & = \lambda h.\,(\lambda y.\,h\,(\fs{inl}(y)), \lambda z.\,h\,(\fs{inr}(z)), \Idext(x.\,\pmap(h,\fs{glue}(x))))
\end{align*}
The inverse of this function is defined as follows:
\[ G = \lambda (b,c,p).\,\lambda w.\,\amalg\text{-}\fs{elim}(w.D, y.\,b\,y, z.\,c\,z, x.\,\fs{hap}(p,x), w). \]

For every triple $(b,c,p)$, we need to show that $F\,(G\,(b,c,p)) \sim (b,c,p)$.
It is easy to construct this homotopy using $h_1$, $h_2$, and $h_3$.
For every map $h : \Hom(B \amalg_A C, D)$, we need to show that $G\,(F\,h) \sim h$.
We have the following homotopy:
\[ G\,(F\,h) \sim \lambda w.\,\amalg\text{-}\fs{elim}(w.D, y.\,h\,(\fs{inl}(y)), z.\,h\,(\fs{inr}(z)), x.\,\pmap(h, \fs{glue}(x)), w). \]
Let us denote the latter function by $h'$.

We can define a homotopy between $h'$ and $h$ as follows:
\[ \Idext(w.\,\amalg\text{-}\fs{elim}(w.\,\Id(h'\,w,h\,w), y.\,h_1(y), z.\,h_2(z), x.\,h_3'(x))), \]
where $h_3'(x)$ is a homotopy between $\fs{glue}(x)_*(h_1(f\,x))$ and $h_2(g\,x)$.
Since the former term is homotopic to $\sym{\pmap(h'\,\overline{x},\fs{glue}(x))} \ct h_1(f\,x) \ct \pmap(h\,\overline{x},\fs{glue}(x))$, we can construct $h_3'(x)$ using $h_3(x)$.

Now, let us prove the converse.
Assume that the theory has pushouts.
Thus, we have terms $\fs{inl}' : \Hom(B, B \amalg_A C)$, $\fs{inr}' : \Hom(C, B \amalg_A C)$, and $\fs{glue}' : \Id(\fs{inl}' \circ f, \fs{inr}' \circ g)$.
We define $\fs{inl}(b)$ as $\fs{inl}'\,b$, $\fs{inr}(c)$ as $\fs{inr}'\,c$, and $\fs{glue}(a)$ as $\fs{hap}(\fs{glue}',a)$.

Let us define the eliminator $\amalg\text{-}\fs{elim}(w.D, y.d_1, z.d_2, x.d_3, e)$.
We have the following commutative square:
\[ \xymatrix{ A \ar[rr]^-f \ar[d]_g                    & & B \ar[d]^{\lambda y.(\fs{inl}(y),d_1)} \\
              C \ar[rr]_-{\lambda z.(\fs{inr}(z),d_2)} & & \Sigma_{w : B \amalg_A C} D
            } \]
The commutativity of the square is witnessed by the following term:
\[ \fs{Idext}(x.\,\Sigma\fs{ext}(\fs{glue}(x),d_3)). \]
By the universal property of pushouts, we have the following terms:
\begin{align*}
\Gamma \mid w : B \amalg_A C & \vdash s(w) : \Sigma_{w : B \amalg_A C} D \\
\Gamma \mid y : B & \vdash h_1'(y) : \Id(s(\fs{inl}(y)),(\fs{inl}(y),d_1)) \\
\Gamma \mid z : C & \vdash h_2'(z) : \Id(s(\fs{inr}(z)),(\fs{inr}(z),d_2))
\end{align*}
and term $h_3'(x)$ which witnesses the commutativity of the following square:
\[ \xymatrix{ s(\fs{inl}(f\,x)) \ar@{=}[r]^-{h_1'(f\,x)} \ar@{=}[d]_{\pmap(w.s(w),\fs{glue}(x))} & (\fs{inl}(f\,x),d_1[f\,x/y]) \ar@{=}[d]^{\Sigma\fs{ext}(\fs{glue}(x),d_3)} \\
              s(\fs{inr}(g\,x)) \ar@{=}[r]_-{h_2'(g\,x)} & (\fs{inr}(g\,x),d_2[g\,x/z])
            } \]
By the universal property of pushouts, we have the following terms:
\begin{align*}
\Gamma \mid w : B \amalg_A C & \vdash q(w) : \Id(\pi_1(s(w)),w) \\
\Gamma \mid y : B & \vdash q_1(y) : \Id(h(\fs{inl}(y)),\pmap(\pi_1,h_1'(y))) \\
\Gamma \mid z : C & \vdash q_2(z) : \Id(h(\fs{inr}(z)),\pmap(\pi_1,h_2'(z)))
\end{align*}
and $q_3(x)$ which is a homotopy between two terms witnessing the commutativity of the following square:
\[ \xymatrix{ \pi_1(s(\fs{inl}(f\,x))) \ar@{=}[rr]^-{q(\fs{inl}(f\,x))} \ar@{=}[d]_{\pmap(w.\pi_1(s(w)),\fs{glue}(x))} & & \fs{inl}(f\,x) \ar@{=}[d]^{\fs{glue}(x)} \\
              \pi_1(s(\fs{inr}(g\,x))) \ar@{=}[rr]_-{q(\fs{inr}(g\,x))}                                                & & \fs{inr}(g\,x)
            } \]
One of these terms is defined by path induction on $\fs{glue}(x)$ and the other one is obtained from $h_3'(x)$, $q_1(f\,x)$, and $q_2(g\,x)$.
We define $\amalg\text{-}\fs{elim}(w.D, y.d_1, z.d_2, x.d_3, e)$ as $q_*(\pi_2(s(e)))$.
Maps $h_1$, $h_2$, and $h_3$ can be defined using $h_1'$ and $q_1$, $h_2'$ and $q_2$, and $h_3'$ and $q_3$, respectively.
\end{proof}

\emph{Stable dependent pushouts} are defined in the same way as dependent pushouts with the difference that all premises in all rules have an extension of $\Delta$ as their context.

\begin{example}
The model $\sSpace$ has stable dependent pushouts.
\end{example}

\begin{example}
Pushouts in $\qCat$ cannot be stable since pushouts of $\infty$-categories are not stable under pullbacks.
The same is true for pointed types and spectra.
\end{example}

We will say that pushouts in an indexed unary type theory are \emph{stable under pullbacks} if, for all maps $f : \Hom(A,B)$, $g : \Hom(A,C)$, $p : \Hom(B \amalg_A C, D)$, and $r : \Hom(E,D)$,
the canonical map from $r^*(B) \amalg_{r^*(A)} r^*(C)$ to $r^*(B \amalg_A C)$ is an equivalence, where $r^*(X)$ is the pullback of $X$ along $r$.

\begin{prop}[pushout-sigma]
Suppose that an indexed dependent type theory has $\Sigma$-types, identity types, extensional $\Hom$-types, and (weak) stable dependent pushouts.
Dependent pushouts commute with $\Sigma$-types.
More precisely, the following map is an equivalence for all maps $f : \Hom_\Delta(A,B)$ and $g : \Hom_\Delta(A,C)$:
\begin{align*}
& \Hom_\Delta((\sum_{d : D} B) \amalg_{(\sum_{d : D} A)} (\sum_{d : D} C), \sum_{d : D} B \amalg_A C) \\
& \lambda e. \amalg\text{-}\fs{elim}(w. \sum_{d : D} B \amalg_A C, (d,y).(d,\fs{inl}(y)), (d,z).(d,\fs{inr}(z)), (d,x).d_3, e),
\end{align*}
where $d_3 = (d,\pmap(w.(d,w),\fs{glue}(x)))$.
\end{prop}
\begin{proof}
The inverse of this map is defined as follows:
\[ \lambda (d,e). \amalg\text{-}\fs{elim}(w. (\sum_{d : D} B) \amalg_{(\sum_{d : D} A)} (\sum_{d : D} C), y.(d,y), z.(d.z), x.\pmap(w.(d,w),\fs{glue}(x)), e). \]
It is easy to show that these maps are mutually inverse using the eliminator for pushouts.
\end{proof}

\begin{prop}
Suppose that an indexed dependent type theory has $\Sigma$-types, identity types, extensional $\Hom$-types, and (weak) stable dependent pushouts.
Then pushouts are stable under pullbacks.
\end{prop}
\begin{proof}
Suppose that we have the following maps: $f : \Hom(A,B)$, $g : \Hom(A,C)$, $p : \Hom(B \amalg_A C, D)$, and $r : \Hom(E,D)$.
We define $d : D \vdash A' \ob$ as the fiber of $A$ over $d$.
Types $B'$ and $C'$ and maps $f' : \Hom_{d : D}(A',B')$ and $g' : \Hom_{d : D}(A',C')$ are defined similarly.
Then we have the following pullback squares:
\[ \xymatrix{ r^*(B) \amalg_{r^*(A)} r^*(C) \ar[r] \ar[d]_\simeq \pb & B \amalg_A C \ar[d]^\simeq \\
              (\Sigma_{e : E} B'[r\,e/d]) \amalg_{(\Sigma_{e : E} A'[r\,e/d])} (\Sigma_{e : E} C'[r\,e/d]) \ar[r] \ar[d]_\simeq \pb & (\Sigma_{d : D} B') \amalg_{(\Sigma_{d : D} A')} (\Sigma_{d : D} C') \ar[d]^\simeq \\
              \Sigma_{e : E} (B' \amalg_{A'} C')[r\,e/d] \ar[r] \ar[d] \pb & \Sigma_{d : D} B' \amalg_{A'} C' \ar[d] \\
              E \ar[r] & D
            } \]
The bottom square is a pullback since substitutions correspond to pullbacks.
Vertical maps in the second row are equivalences by \rprop{pushout-sigma}.
\end{proof}

\section{The initial type theorem}
\label{sec:initial}

The general adjoint functor theorem holds in the context of indexed categories \cite[IV.1]{indexed-cats} and in the context of $\infty$-categories \cite{infty-gaft}.
Thus, it is natural to assume that it also should hold in the context of indexed type theories.
To properly state this theorem, we need to define the notion of adjoint functors between models of such theories.
This paper focuses on internal properties of a single model of an indexed type theory.
So, we only consider the first step in the proof of the adjoint functor theorems, which is known as \emph{the initial object theorem} or \emph{the initial type theorem} in our case.
This theorem is proved in \cite[IV.1.1]{indexed-cats} for indexed categories and in \cite[Proposition~2.3.2]{infty-gaft} for $\infty$-categories.

\subsection{$h$-initial types}

In this subsection, we will prove an analogue of \cite[Proposition~2.2.2]{infty-gaft}.
This proposition states that $h$-initial objects are initial in finitely complete $\infty$-categories.
An object $Z$ is $h$-initial if the space $\Hom(Z,X)$ is connected (and inhabited) for all $X$.
This proposition has two problems in the context of indexed type theories.
The first one is that it seems that it is not enough to assume the existence of finite limits since this only implies that homotopy groups of $\Hom(Z,X)$ vanish which might be not enough to conclude that this type is contractible.
For this reason, we replace this condition with the condition that all powers exist.
The second problem is that the definition of $h$-initial objects involves the propositional truncation which might not exist.
We solve this problem by replacing the condition of connectedness by a weaker condition which can be formulated without the propositional truncation.

First, for every type $X$, we define a weakening of the condition that $X$ is inhabited.
Similar condition was defined in \cite[Definition~5]{gen-hedberg}: a type $X$ is \emph{populated}, written $\mathrm{isPop}(X)$, if every constant endofunction on $X$ has a fixed point.
We will say that $X$ is \emph{weakly populated}, written $\mathrm{isWPop}(X)$, if $\mathrm{isProp}(X) \to X$, that is if $X$ is inhabited whenever it is a proposition.
Clearly, $\mathrm{isWPop}(X)$ is a proposition.

\begin{prop}
We have the following sequence of implications:
\[ X \to \| X \| \to \mathrm{isPop}(X) \to \mathrm{isWPop}(X) \to \neg \neg X. \]
\end{prop}
\begin{proof}
The first implication is obvious and the second follows from the fact that $\mathrm{isPop}(X)$ is a proposition.
Suppose that $X$ is populated.
To prove that it is weakly populated, we may assume that it is a proposition.
Then the identity endofunction on $X$ is constant.
Thus, there exists a point in $X$ (namely, the fixed point of $\id_X$).
Finally, suppose that $X$ is weakly populated and let us prove that $\neg \neg X$.
Assume that $\neg X$.
This implies that $X$ is a proposition.
Since $X$ is weakly populated, it is inhabited, which is a contradiction.
\end{proof}

We will say that a type $X$ is \emph{weakly connected} if, for all $x, x' : X$, the type $\Id(x,x')$ is weakly populated.
We will say that an indexed type $Z$ is \emph{$h$-initial} if, for every indexed type $X$, the type $\Hom(Z,X)$ is weakly connected and inhabited.

\begin{prop}
Let $0$ be an indexed type in a locally small indexed unary type theory such that, for every indexed type $X$, the type $\Hom(0,X)$ is weakly connected.
If the theory has extensional powers, then $\Hom(0,X)$ is a proposition for every $X$.
\end{prop}
\begin{proof}
Since the theory has powers, there is a type $X^{\Hom(0,X)}$ such that the type $\Hom(0,X^{\Hom(0,X)})$ is equivalent to $\Hom(0,X) \to \Hom(0,X)$.
This implies that the type $\Hom(0,X) \to \Hom(0,X)$ is weakly connected.
Let $f,g : \Hom(0,X)$ be a pair of maps.
We need to construct a homotopy between them.

First, let us prove that the type $\Id_{\Hom(0,X) \to \Hom(0,X)}(\id, \lambda x.g)$ is a proposition.
Since powers are extensional, the functional extensionality holds for the type $\Hom(0,X) \to \Hom(0,X)$.
Thus, we just need to prove that, for all $x : \Hom(0,X) \vdash p : \Id(x,g)$ and $x : \Hom(0,X) \vdash q : \Id(x,g)$, there is a homotopy $x : \Hom(0,X) \vdash h : \Id(p,q)$.
It is enough to prove that, for all $x$ and $q$, there is a homotopy between $q$ and $p \ct \sym{p[g/x]}$, which follows by path induction.

Now, since $\Hom(0,X) \to \Hom(0,X)$ is weakly connected and $\Id(\id_{\Hom(0,X)}, \lambda x.g)$ is a proposition, the latter type is inhabited.
If $h$ is a homotopy between $\id$ and $\lambda x.g$, then $\fs{hap}(h,f)$ is a homotopy between $f$ and $g$.
\end{proof}

\begin{cor}
Any $h$-initial type in a locally small indexed unary type theory with extensional powers is initial.
\end{cor}

\subsection{Split idempotents}

An idempotent in a 1-category is a map $h : B \to B$ such that $h \circ h = h$.
In the setting of $\infty$-categories an idempotent consists of a map and an infinite amount of coherence data.
Lurie proved in \cite[Lemma~7.3.5.14]{lurie-algebra} (this lemma is available in the version dated September 14, 2014, but was removed in later versions)
that if $h$ is a map such that there exists a homotopy between $h \circ h$ and $h$ satisfying one additional coherence condition, then $h$ can be extended to an idempotent.

An idempotent $h$ in a 1-category is split if there are maps $f : A \to B$ and $g : B \to A$ such that $g \circ f = \id_A$ and $f \circ g = h$.
If the category is finitely complete, then every idempotent is split since $f$ can be defined as the equalizer of $h$ and $\id_B$ and $g$ exists by the universal property.
It is no longer true that the splitting of an idempotent in an $\infty$-category can be constructed as a limit of a finite diagram, but it is a limit of a countable diagram.
This question in the context of ordinary homotopy type theory was discussed by Shulman in \cite{split-idemp}.
We can repeat this argument in the setting of indexed type theories.

\begin{defn}
A map $h : \Hom(B,B)$ in an indexed unary type theory is \emph{idempotent} if there are the following terms:
\begin{align*}
I & : \Id_{\Hom(B,B)}(h, h \circ h) \\
J & : \Id_{\Id(h \circ h, h \circ h \circ h)}(\pmap(- \circ h, I), \pmap(h \circ -, I))
\end{align*}
\end{defn}

\begin{defn}
An idempotent map $h : \Hom(B,B)$ is \emph{split} if there exist maps $f : \Hom(A,B)$ and $g : \Hom(B,A)$ such that $g \circ f \sim \id_A$ and $f \circ g \sim h$.
\end{defn}

\begin{prop}[split-idemp]
If the base theory of a locally small indexed unary type theory has natural numbers $\mathbb{N}$ and the indexed theory has equalizers and extensional powers $B^\mathbb{N}$ for some type $B$, then every idempotent on $B$ is split.
\end{prop}
\begin{proof}
We define the splitting of an idempotent map $h : \Hom(B,B)$ as the limit of the following sequence:
\[ \ldots \xrightarrow{h} B \xrightarrow{h} B \xrightarrow{h} B. \]
More precisely, let $e : \Hom(A,B^\mathbb{N})$ be the equalizer of $\id, h \circ - \circ \mathrm{suc} : \Hom(B^\mathbb{N},B^\mathbb{N})$.
Let $d : \Id(e, h \circ e(-) \circ \mathrm{suc})$ be the witness of the fact that $e$ equalizes these maps.
Let $f : \Hom(A,B)$ be the following composite:
\[ A \xrightarrow{e} B^\mathbb{N} \xrightarrow{\lambda f.\,f\,0} B. \]
By the universal property of equalizers, to define a map $g : \Hom(B,A)$, it is enough to define a map $g' : \Hom(B,B^\mathbb{N})$ and a homotopy $p : \Id(g', h \circ g'(-) \circ \mathrm{suc})$.
Let $g' = \lambda b. \lambda n.\,h\,b$.
Since powers are extensional, to define $p$, it is enough to define a homotopy between $h$ and $h \circ h$, which we define as $I$.
It is easy to see that $f \circ g \sim h$.
Thus, we just need to prove that $g \circ f \sim \id_A$.

By the universal properties of equalizers, to construct this homotopy, it is enough to prove that for every $n : \mathbb{N}$, terms $(e,d)$ and $(e',d')$ of type $\Sigma_{r : \Hom(A,B^\mathbb{N})} \Id(r, h \circ r(-) \circ \mathrm{suc})$ are homotopic,
where $e' = \lambda a. \lambda n.\,h\,(e\,a\,0)$ and $d'$ is the homotopy between $e'$ and $\lambda a. \lambda n.\,h\,(h\,(e\,a\,0)))$ obtained from $I$.
First, we will construct a homotopy between $e$ and $e'$.
By the universal property of extensional powers, it is enough to prove that for every $n : \mathbb{N}$, maps $\lambda a.\,e\,a\,n$ and $\lambda a.\,h\,(e\,a\,0)$ are homotopic.
Let $s(n)$ be the homotopy between $\lambda a.e\,a\,n$ and $\lambda a.\,h\,(e\,a\,(n+1))$ obtained from $d$.
Then we define a homotopy between $\lambda a.\,e\,a\,n$ and $\lambda a.\,h\,(e\,a\,0)$ as $s(n) \ct T(n)$, where $T(n) : \Id(\lambda a.\,h\,(e\,a\,(n+1)), \lambda a.\,h\,(e\,a\,0))$ is defined by induction.
Let us write $I(a.t) : \Id(h \circ \lambda a.t, h \circ h \circ \lambda a.t)$ for $\pmap(\lambda f.\,\lambda a.\,f \circ t, I)$.
Then we define $T$ as follows:
\begin{align*}
T(0) & = I(a.\,e\,a\,1) \ct \pmap(h \circ -, \sym{s(0)}) \\
T(n+1) & = I(a.\,e\,a\,(n+2)) \ct \pmap(h \circ -, \sym{s(n+1)}) \ct T(n)
\end{align*}

Now, we need to construct a homotopy between the second components of pairs $(e,d)$ and $(e',d')$.
By the universal property of extensional powers, it is enough to prove that for every $n : \mathbb{N}$, the following square commutes:
\[ \xymatrix{ \lambda a.\,h\,(e\,a\,(n+1)) \ar@{=}[rr]^-{\pmap(h \circ -, s(n+1))} \ar@{=}[d]_{T(n)}    & & \lambda a.\,h\,(h\,(e\,a\,(n+2))) \ar@{=}[d]^{\pmap(h \circ -, T(n+1))} \\
              \lambda a.\,h\,(e\,a\,0) \ar@{=}[rr]_-{I(a.\,e\,a\,0)}                                    & & \lambda a.\,h\,(h\,(e\,a\,0))
            } \]
By the definition of $T(n+1)$, the top path is homotopic to $\pmap(h \circ -, s(n+1) \ct I(a.\,e\,a\,(n+2)) \ct \pmap(h \circ -, \sym{s(n+1)}) \ct T(n))$.
By $J$, we have a homotopy $\pmap(h \circ -, I(a.\,e\,a\,(n+2))) \sim I(a.\,h\,(e\,a\,(n+2)))$.
By path induction, we have a homotopy $\pmap(h \circ -, s(n+1)) \ct I(a.\,h\,(e\,a\,(n+2))) \ct \pmap(h \circ h \circ -, \sym{s(n+1)}) \sim I(a.\,e\,a\,(n+1))$.
Thus, we just need to prove that the following square is commutative:
\[ \xymatrix{ \lambda a.\,h\,(e\,a\,(n+1)) \ar@{=}[rr]^-{I(a.\,e\,a\,(n+1))} \ar@{=}[d]_{T(n)}  & & \lambda a.\,h\,(h\,(e\,a\,(n+1))) \ar@{=}[d]^{\pmap(h \circ -, T(n))} \\
              \lambda a.\,h\,(e\,a\,0) \ar@{=}[rr]_-{I(a.\,e\,a\,0)}                            & & \lambda a.\,h\,(h\,(e\,a\,0))
            } \]
We do this by induction on $n$.
If $n = 0$, then $I(a.\,e\,a\,1) \ct \pmap(h \circ -, T(0)) \sim I(a.\,e\,a\,1) \ct I(a.\,h\,(e\,a\,1)) \ct \pmap(h \circ -, \sym{s(0)})$ and $T(0) \ct I(a.\,e\,a\,0) = I(a.\,e\,a\,1) \ct \pmap(h \circ -, \sym{s(0)}) \ct I(a.\,e\,a\,0)$.
By path induction on $s(0)$, these terms are homotopic.
If $n = n' + 1$, then the top and the bottom paths in the diagram are homotopic to the following terms:
\begin{align*}
& I(a.\,e\,a\,(n+1)) \ct I(a.\,h\,(e\,a\,(n+1))) \ct \pmap(h \circ h \circ -, \sym{s(n)}) \ct \pmap(h \circ -, T(n')) \\
& I(a.\,e\,a\,(n+1)) \ct \pmap(h \circ -, \sym{s(n)}) \ct T(n') \ct I(a.\,e\,a\,0)
\end{align*}
By the induction hypothesis, we have a homotopy $T(n') \ct I(a.\,e\,a\,0) \sim I(a.\,e\,a\,n) \ct \pmap(h \circ -, T(n'))$.
Thus, we just need to prove that terms $I(a.\,h\,(e\,a\,(n+1))) \ct \pmap(h \circ h \circ -, \sym{s(n)})$ and $\pmap(h \circ -, \sym{s(n)}) \ct I(a.\,e\,a\,n)$ are homotopic, which follows by path induction on $s(n)$.
\end{proof}

\subsection{The initial type theorem}

A \emph{weakly initial family of indexed types} is a family $\Gamma, i : I \mid \cdot \vdash W_i \ob$ such that, for every indexed type $\Gamma \mid \cdot \vdash B \ob$, there exists an index $\Gamma \vdash i : I$ and a map $\Gamma \vdash b_i : \Hom(W_i,B)$.
We will prove the initial type theorem for indexed dependent type theories.
This is merely a technical convenience; the theorem should also be true in unary theories.

\begin{thm}
Suppose that a locally small indexed dependent type theory has $\Sigma$-types, identity types, extensional $\Hom$-types, extensional dependent products, and split idempotents.
If it has a weakly initial family of indexed types, then it also has the initial type.
\end{thm}
\begin{proof}
Let $W$ be the product of a weakly initial family of indexed types.
Then $W$ is weakly initial in the sense that, for every indexed type $B$, there exists a map from $W$ to $B$.
Let $Z$ be the following type:
\[ \sum_{(x : W)} \sum_{(h : \prod_{f : \Hom(W,W)} \Id(x,f\,x))} \prod_{(f : \Hom(W,W))} \prod_{(p : \Id(f, f \circ f))} \Id(\fs{hap}(p,x),\pmap(f,h\,f)). \]
Let $e : \Hom(Z,W)$ be the first projection and let $r : \Hom(W,Z)$ be any map.
We will prove that $h = r \circ e$ is idempotent.
We define $I : \Id(h, h \circ h)$ as follows:
\[ I = \Idext(z.\,\pmap(r, \pi_2\,z\,(e \circ r))). \]
To construct a homotopy $J$, it is enough to define the following homotopy:
\[ z : Z \vdash J' : \Id(\pmap(r, \pi_2\,(h\,z)\,(e \circ r), \pmap(h \circ r, \pi_2\,z\,(e \circ r))). \]
We define $J'$ as $\pmap(r,-)$ applied to the following sequence:
\[ \pi_2\,(h\,z)\,(e \circ r) \sim \fs{hap}(\Idext(w.\,\pi_2\,(r\,w)\,(e \circ r)),e\,z) \sim \pmap(e \circ r, \pi_2\,z\,(e \circ r)). \]
The second homotopy here is $\pi_3\,z\,(e \circ r)\,(\Idext(w.\,\pi_2\,(r\,w)\,(e \circ r)),e\,z)$.

Since idempotents are split, there exist a type $0$ and maps $q : \Hom(0,Z)$ and $p : \Hom(Z,0)$ such that $p \circ q \sim \id_0$ and $q \circ p \sim r \circ e$.
Let us prove that $0$ is initial.
For every type $B$, there exists a map $0 \xrightarrow{q} Z \xrightarrow{e} W \to B$.
Thus, it is enough to construct a homotopy between any two maps $f,g : \Hom(0,B)$.
Since there exists a map from $W$ to the equalizer of $f \circ p$ and $g \circ p$, there is a map $r' : \Hom(W,Z)$ such that $f \circ p \circ r' \sim g \circ p \circ r'$.
Then $\Idext(z.\,\pi_2\,z\,(e \circ r'))$ is a homotopy between $e$ and $e \circ r' \circ e$.
Then we have the following sequence of homotopies:
\[ p \circ r' \circ e \sim p \circ q \circ p \circ r' \circ e \sim p \circ r \circ e \circ r' \circ e \sim p \circ r \circ e \sim p \circ q \circ p \sim p. \]
It follows that $f$ and $g$ are homotopic:
\[ f \sim f \circ p \circ q \sim f \circ p \circ r' \circ e \circ q \sim g \circ p \circ r' \circ e \circ q \sim g \circ p \circ q \sim g. \]
\end{proof}

\section{Classifying morphisms}
\label{sec:class}

In this section, we will define the notion of classifying morphisms, discuss its relationship to the notions of universes and factorization systems.

\subsection{Truncated maps}

Let $f : \Hom(A,B)$ be a map in an indexed unary type theory and let $n$ be an integer $\geq -2$.
We will say that $f$ is \emph{$n$-truncated} if the map $f \circ - : \Hom(X,A) \to \Hom(X,B)$ is $n$-truncated for all indexed types $X$.
This definition makes sense in models of indexed unary type theories, but the problem is that it quantifies over indexed types.
This means we cannot define a predicate on $\Hom(A,B)$ which corresponds to the notion of $n$-truncated maps in models.
We can fix this problem if the indexed theory has pullbacks.

First, let us prove a few technical lemmas:

\begin{lem}[trunc-pb]
In an indexed unary type theory, $n$-truncated maps are closed under pullbacks.
\end{lem}
\begin{proof}
Suppose that we have a pullback square in which the right arrow is $n$-truncated:
\[ \xymatrix{ A \ar[r] \ar[d] \pb   & C \ar[d] \\
              B \ar[r]              & D.
            } \]
Since $\Hom(X,-)$ preserves pullbacks, the following square is also pullback and the right arrow is $n$-truncated by the definition of $n$-truncated maps in indexed type theories:
\[ \xymatrix{ \Hom(X,A) \ar[r] \ar[d] \pb   & \Hom(X,C) \ar[d] \\
              \Hom(X,B) \ar[r]              & \Hom(X,D).
            } \]

Thus, we just need to prove that $n$-truncated maps are closed under pullbacks in the base theory.
Suppose that we have the following pullback square:
\[ \xymatrix{ \sum_{b : B} P(f(b)) \ar[r] \ar[d] \pb    & \sum_{d : D} P(d) \ar[d] \\
              B \ar[r]_f                                & D.
            } \]
If the right arrow is $n$-truncated, then its fibers $P(d)$ are $n$-types for all $d : D$, but this implies that fibers of the left arrow are also $n$-types; hence, it is also $n$-truncated.
\end{proof}

\begin{lem}[trunc-total]
Let $B$ and $C$ be base types over $x : A$.
Then a function $f : B \to C$ is $n$-truncated if and only if the induced function $f' : \Sigma_{x : A} B \to \Sigma_{x : A} C$ is $n$-truncated.
\end{lem}
\begin{proof}
By \cite[Theorem~4.7.6]{hottbook}, the fiber of $f'$ over a point $(x,c)$ is equivalent to the fiber of $f$ over $c$.
Thus, fibers of $f'$ are $n$-truncated if and only if fibers of $f$ are $n$-truncated.
\end{proof}

The following lemma is similar to \cite[Lemma~5.5.6.15]{lurie-topos}, which is proved in the context of $\infty$-categories.

\begin{lem}[trunc-id]
A map $f : \Hom(A,B)$ is $n$-truncated if and only if the map $\langle \id_A, \id_A \rangle : \Hom(A, A \times_B A)$ is $(n-1)$-truncated.
\end{lem}
\begin{proof}
Since $\Hom(X,-)$ preserves and reflects $n$-truncatedness and pullbacks, it is enough to prove this fact for base types.
By \cite[Lemma~7.6.2]{hottbook}, a function $f : A \to B$ is $n$-truncated if and only if, for all $a,a' : A$, the function $\pmap(f,-) : \Id(a,a') \to \Id(f\,a,f\,a')$ is $(n-1)$-truncated.
By \rlem{trunc-total}, the latter function is $n$-truncated if and only if the induced function $\Sigma_{a : A} \Sigma_{a' : A} \Id(a,a') \to \Sigma_{a : A} \Sigma_{a' : A} \Id(f\,a,f\,a')$ is $(n-1)$-truncated.
The latter function is equivalent to $\lambda a.\,(a,a,\refl) : A \to \Sigma_{a : A} \Sigma_{a' : A} \Id(f\,a,f\,a')$, which is equivalent to $\langle \id_A, \id_A \rangle : \Hom(A, A \times_B A)$.
\end{proof}

\Rlem{trunc-id} implies that if the theory has pullbacks, then we can define a predicate on maps that corresponds to the notion of $n$-truncated maps by induction on $n$.

\begin{example}
The map $\Hom(S^0,-)$ preserves pullbacks and preserves and reflects equivalences.
Thus, \rlem{trunc-id} implies that a map of pointed types is $n$-truncated if and only if the underlying map of types is $n$-truncated.
\end{example}

\begin{example}
A map of spectra $f : A \to B$ is an equivalence if and only if $U_S(f)_n : U_S(A)_n \to U_S(B)_n$ is an equivalence for all $n : \mathbb{N}$.
Since $\Sigma$ is an equivalence, it preserves pullbacks.
It follows that $U_S(-)_n = \Hom(S,\Sigma^n(-))$ preserves pullbacks for all $n : \mathbb{N}$.
Now, \rlem{trunc-id} implies that a map of spectra $f$ is $k$-truncated if and only if $U_S(f)_n$ is $k$-truncated for all $n : \mathbb{N}$.

It follows that a map of spectra is $k$-truncated if and only if it is an equivalence.
Indeed, since $U_S(f)_n = \Omega^{k+2}(U_S(f)_{n+k+2})$ and $U_S(f)_{n+k+2}$ is $k$-truncated, $U_S(f)_n$ is $(-2)$-truncated.
Thus, $U_S(f)_n$ is an equivalence for all $n : \mathbb{N}$.
\end{example}

\subsection{Fibrations}

Suppose that we have a class of families of propositions over all indexed morphisms:
\begin{center}
\AxiomC{$\Gamma \mid \cdot \vdash A \ob$}
\AxiomC{$\Gamma \mid \cdot \vdash B \ob$}
\AxiomC{$\Gamma \vdash f : \Hom(A,B)$}
\TrinaryInfC{$\Gamma \vdash \Fib(f) \type$}
\DisplayProof
\end{center}
We will call maps $f$ together with an element of $\Fib(f)$ \emph{fibrations} and denote them by $\twoheadrightarrow$.
We will assume that $\Fib$ is closed under equivalences, that is if $f : \Hom(A,B)$ is a fibration and $e_1 : \Hom(A',A)$ and $e_2 : \Hom(B,B')$ are equivalences, then $e_2 \circ f \circ e_1$ is a fibration.

Sometimes $\Fib(f)$ is not a type, but a finite number of judgments of the form $\Gamma, \Delta_i \vdash A_i \type$.
For example, we might want to define $\Fib(f)$ as $\fs{isEquiv}(C(f))$ for some morphism $C(f)$.
In general, this is not a type, but a collection of four judgments.
If the base theory has $\Pi$-types, then we can always replace such a collection of judgments with a single type.
Even if the base theory does not have $\Pi$-types, we still can work with such definitions of $\Fib(f)$;
we just need to replace judgments of the form $\Gamma \vdash b : \Fib(f)$ with a finite collection of judgments of the form $\Gamma, \Delta_i \vdash a_i : A_i$.
For notational convenience, we will always assume that $\Fib(f)$ is a single type.

We can also define the dependent version of classes of fibrations:
\begin{center}
\AxiomC{$\Gamma \mid \Delta \vdash B \ob$}
\UnaryInfC{$\Gamma \vdash \Fib(\Delta.B) \type$}
\DisplayProof
\end{center}
We will also call dependent types $B$ together with an element of $\Fib(\Delta.B)$ fibrations.
It is often more convenient to work with the dependent version of this definition.
If the indexed theory has $\Sigma$-types and unit types, then, for every class of fibrations $\Fib$, we can define its dependent version as follows:
\[ \Fib(\Delta.B) = \Fib(\pi_1 : \Hom(\sum_{p : \Sigma(\Delta)} B[\pi_1(p)/x_1, \ldots \pi_n(p)/x_n],\Sigma(\Delta))). \]
Conversely, if the indexed theory also has identity types, then, for every dependent class of fibrations $\Fib$, we can define its non-dependent version:
\[ \Fib(f : \Hom(A,B)) = \Fib((y : B).\,\sum_{x : A} \Id(f\,x,y)). \]

\begin{example}
If the indexed theory has finite limits, then we can define a class of fibrations consisting of $n$-truncated maps (or $n$-truncated indexed types) as was explained in the previous subsection.
\end{example}

\begin{example}[qcat-fib]
The model $\qCat$ has many interesting examples of fibrations: Cartesian, coCartesian, covariant, and contravariant.
More examples of fibrations in this model can be found in \cite{fib-inf-cat}.
\end{example}

For every dependent class of fibrations $\Fib$, we can add a new sort of dependent types $\Gamma \mid \Delta \vdash A \fib$ consisting of types satisfying the predicate $\Fib$:
\begin{center}
\AxiomC{$\Gamma \mid \Delta \vdash A \fib$}
\UnaryInfC{$\Gamma \mid \Delta \vdash \El(A) \ob$}
\DisplayProof
\qquad
\AxiomC{$\Gamma \mid \Delta \vdash A \fib$}
\UnaryInfC{$\Gamma \vdash \fs{fp}(\Delta.A) : \Fib(\Delta.\,\El(A))$}
\DisplayProof
\end{center}
\medskip

\begin{center}
\AxiomC{$\Gamma \mid \Delta \vdash A \ob$}
\AxiomC{$\Gamma \vdash p : \Fib(\Delta.A)$}
\AxiomC{$\Gamma \mid E \vdash b_i : B_i[b_1/x_1, \ldots b_{i-1}/x_{i-1}]$}
\TrinaryInfC{$\Gamma \mid E \vdash \fs{rf}(\Delta.A, p, b_1, \ldots b_k) \fib$}
\DisplayProof
\end{center}
where $\Delta = x_1 : B_1, \ldots x_k : B_k$.
\[ \El(\fs{rf}(\Delta.A, p, b_1, \ldots b_k)) = A[b_1/x_1, \ldots b_k/x_k]. \]
We define equivalences between fibrations $\Gamma \mid \Delta \vdash A \fib$ and $\Gamma \mid \Delta \vdash B \fib$ as equivalences between underlying types $\El(A)$ and $\El(B)$.
We will often omit the function symbol $\El$.

We can assume various closure conditions on the class of fibrations in the usual way.
For example, we can assume that $\Fib$ is closed under contractible types.
This is true if and only if it contains all identity morphisms.
Similarly, $\Fib$ is closed under $n$-types (as a dependent class) if and only if it contains all $n$-truncated maps (as a non-dependent class).

\begin{prop}[fib-sigma]
A class of fibrations is closed under $\Sigma$-types if and only if the corresponding non-dependent class is closed under composition.
\end{prop}
\begin{proof}
First, suppose that $\Fib$ is closed under composition.
If we have fibrations $\Gamma \mid \Delta \vdash A$ and $\Gamma \mid \Delta, x : A \vdash B$, then the type $\Sigma_{x : A} B$ corresponds to the following map:
\[ \Sigma(\Delta, A, B) \xrightarrow{\simeq} \sum_{p : \Sigma(\Delta, A)} B[\overline{\pi_i(p)/x_i}] \xrightarrow{\pi_1} \Sigma(\Delta, A) \xrightarrow{\simeq} \sum_{p : \Sigma(\Delta)} A[\overline{\pi_i(p)/x_i}] \xrightarrow{\pi_1} \Sigma(\Delta). \]
The first and the third maps are equivalences and the second and the fourth maps are fibrations by assumption.
Thus, the type $\Sigma_{x : A} B$ is also a fibration.

Now, suppose that $\Fib$ is closed under $\Sigma$-types.
Let $f : \Hom(A,B)$ and $g : \Hom(B,C)$ be fibrations.
These maps correspond to the types $y : B \vdash \Sigma_{x : A} \Id(f\,x,y)$ and $z : C \vdash \Sigma_{y : B} \Id(g\,y,z)$.
The first type is equivalent to the type $z : C, p : \Sigma_{y : B} \Id(g\,y,z) \vdash \Sigma_{x : A} \Id(f\,x,\pi_1(p))$.
Since fibrations are closed under $\Sigma$-types, the type
\[ z : C \vdash \Sigma_{(p : \Sigma_{y : B} \Id(g y, z))} \sum_{x : A} \Id(f\,x,\pi_1(p)) \]
is a fibration.
This type corresponds to the following map:
\[ \pi_1 : \Hom(\sum_{z : C} \sum_{(p : \sum_{y : B} \Id(g y, z))} \sum_{x : A} \Id(f\,x,\pi_1(p)), C). \]
Since this map is equivalent to $g \circ f$, the composite is a fibration.
\end{proof}

\begin{prop}[fib-id]
Let $\Fib$ be a class of fibrations closed under pullbacks.
Then it is closed under identity types if and only if, for every fibration $p : \Hom(A,B)$, the map $\langle \id_A, \id_A \rangle : \Hom(A, A \times_B A)$ is also a fibration.
\end{prop}
\begin{proof}
First, suppose that $\Fib$ is closed under identity types.
Let $a_1 : A, a_2 : A, h : \Id(p\,a_1,p\,a_2)$ be an element of $A \times_B A$.
The fiber over this element is the following type:
\begin{align*}
\sum_{a : A} \sum_{h_1 : \Id(a_1,a)} \sum_{h_2 : \Id(a,a_2)} \Id(\pmap(p, h_1 \ct h_2), h) & \simeq \\
\sum_{h' : \Id(a_1,a_2)} \Id(\pmap(p,h'),h) & \simeq \\
\Id_{\sum_{a : A} \Id(p a_1, p a)}((a_1,\refl), (a_2,h)) & .
\end{align*}
Since $\Fib$ is closed under identity types, it is enough to show that $\sum_{a : A} \Id(p\,a_1,p\,a)$ is a fibration over $a_1,a_2,h$.
This follows from the fact that this type is a pullback of the type $\sum_{a : A} \Id(b,p\,a)$ over $b : B$ which is a fibration since it is the fiber of $p$ over $b$.

Now, let us prove the converse.
Let $\Gamma \mid \Delta \vdash c_1 : C$ and $\Gamma \mid \Delta \vdash c_2 : C$ be a pair of terms.
Then the following square is a pullback:
\[ \xymatrix{ \Sigma_{p : \Sigma(\Delta)} \Id(c_1',c_2') \ar[r]^-{c_1' \circ \pi_1} \ar@{->>}[d]_{\pi_1} \pb    & C' \ar@{->>}[d]^{\langle \id_{C'}, \id_{C'} \rangle} \\
              \Sigma(\Delta) \ar[r]_-{\langle c_1', c_2' \rangle}                                               & C' \times_{\Sigma(\Delta)} C'
            } \]
where $C' = \Sigma_{p : \Sigma(\Delta)} C[\overline{\pi_i(p)/x_i}]$ and $c_i' = c_i[\overline{\pi_i(p)/x_i}]$.
Since $\Gamma \mid \Delta \vdash \Id(c_1,c_2)$ is equivalent to $\Gamma \mid p : \Sigma(\Delta) \vdash \Id(c_1',c_2')$, it follows that $\Id(c_1,c_2)$ is a fibration.
\end{proof}

The following proposition shows that if the class of fibrations is closed under pullbacks and composition, then there is another characterization of the condition that it is closed under identity types.

\begin{prop}[fib-id-comp]
Let $\Fib$ be a class of fibrations in an indexed unary type theory.
If $\Fib$ is closed under pullbacks and composition, then the following conditions are equivalent:
\begin{enumerate}
\item \label{it:fib-pb} For every fibration $p : \Hom(A,B)$, the map $\langle \id_A, \id_A \rangle : \Hom(A, A \times_B A)$ is also a fibration.
\item \label{it:fib-over} For every commutative diagram as below in which $p$ and $q$ are fibrations, $f$ is also a fibration.
\[ \xymatrix{ A \ar[rr]^f \ar@{->>}[dr]_p &   & C \ar@{->>}[dl]^q \\
                                          & B &
            }\]
\end{enumerate}
\end{prop}
\begin{proof}
First, suppose that \eqref{it:fib-pb} holds.
Consider the following diagram:
\[ \xymatrix{ A \ar[r]^f \ar@{->>}[d]_r \pb             & C \ar@{->>}[d]^{\langle \id_C, \id_C \rangle} & \\
              A \times_B C \ar[r]^{f'} \ar@{->>}[d] \pb & C \times_B C \ar[r]^-{q'} \ar@{->>}[d] \pb    & C \ar@{->>}[d]^q \\
              A \ar[r]_f                                & C \ar[r]_q                                    & B
            } \]
The map $q' \circ f'$ is a pullback of $q \circ f = p$, so it is a fibration.
The map $r$ is a fibration since it is a pullback of $\langle \id_C, \id_C \rangle$, which is a fibration by \eqref{it:fib-pb}.
Since fibrations are closed under composition, $q' \circ f' \circ r$ is also a fibration.
Finally, $f$ is a fibration since $f \sim q' \circ \langle \id_C, \id_C \rangle \circ f \sim q' \circ f' \circ r$.

Now, suppose that \eqref{it:fib-over} holds.
Let $f : \Hom(A,B)$ be a fibration.
Let $q$ be the composite $A \times_B A \xrightarrow{\pi_1} A \xrightarrow{f} B$.
Since fibrations are closed under pullbacks, the first map is a fibration.
Since they are closed under composition, $q$ is also a fibration.
Since $q \circ \langle \id_A, \id_A \rangle = f$ is a fibration, $\langle \id_A, \id_A \rangle$ is also a fibration by \eqref{it:fib-over}.
\end{proof}

\begin{lem}[fib-id-idm]
Let $\Fib$ be a class of fibrations in an indexed unary type theory.
If $\Fib$ is closed under pullbacks and contains all identity morphisms, then \eqref{it:fib-over} implies \eqref{it:fib-pb}.
\end{lem}
\begin{proof}
Let $f : \Hom(A,B)$ be a fibration.
Since $\pi_1 : \Hom(A \times_B A, A)$ is a pullback of $f$, it is also a fibration.
Moreover, $\id_A$ is a fibration by assumption.
Now, the claim follows from the fact that $\pi_1 \circ \langle \id_A, \id_A \rangle \sim \id_A$.
\end{proof}

\begin{example}
The class of $n$-truncated maps is closed under composition, pullbacks, $m$-types for $m \leq n$, $\Sigma$-types, and identity types.
This follows from \rlem{trunc-pb}, \rprop{fib-sigma}, \rprop{fib-id}, and \rlem{trunc-id}.
\end{example}

\subsection{Object classifiers}

Let $p : \Hom(\widehat{\mathcal{U}},\mathcal{U})$ be a map in an indexed unary type theory such that its pullbacks along all maps exist.
Then we can define a map from $\Id_{\Hom(D,\mathcal{U})}(f,g)$ to the type of equivalences over $D$ between pullbacks of $p$ along $f$ and $g$ as the transport of the identity map along the homotopy between $f$ and $g$.
An \emph{object classifier} is a map $p : \Hom(\widehat{\mathcal{U}},\mathcal{U})$ such that its pullbacks exist and the map defined above is an equivalence.
We will say that an object classifier \emph{classifies} a class of fibrations if this class is closed under pullbacks, $p$ is a fibration, and every fibration is a pullback of $p$.
A \emph{subobject classifier} is an object classifier for the class of monomorphisms.

\begin{example}
Let $\mathcal{U}$ be a univalent universe in the base theory containing a contractible type $C$ with a point $c : \El(C)$.
Consider the following map of pointed types:
\[ \pi_1 : (\Sigma_{A : \mathcal{U}} \El(A), (C, c)) \to_* (\mathcal{U}, C). \]
Then a straightforward computation shows that the map of indexed types corresponding to $\pi_1$ is an object classifier.
In particular, if the base theory contains a suboject classifier, then so does the theory of pointed types.
\end{example}

\begin{example}
We believe that classes of small fibrations mentioned in \rexample{qcat-fib} should be classified by an object classifier.
Verification of these facts is beyond the scope of this paper.
\end{example}

If an object classifier classifying a class $\Fib$ exists, then it is unique up to a canonical equivalence.
Indeed, let $p : \Hom(\widehat{\mathcal{U}},\mathcal{U})$ and $p' : \Hom(\widehat{\mathcal{U}}',\mathcal{U}')$ be two object classifiers classifying $\Fib$.
Then we have maps $f : \Hom(\mathcal{U},\mathcal{U}')$ and $f' : \Hom(\mathcal{U}',\mathcal{U})$ such that $p$ is a pullback of $p'$ along $f$ and $p'$ is a pullback of $p$ along $f'$.
To construct a homotopy between $f' \circ f$ and $\fs{id}$, it is enough to show that the pullback of $p$ along $f' \circ f$ is equivalent to $p$ over $\mathcal{U}$, but this is obvious by the definition of $f$ and $f'$.
A homotopy between $f \circ f'$ and $\fs{id}$ is constructed similarly.

Let $p : \Hom(\widehat{\mathcal{U}},\mathcal{U})$ be any map.
We can think of such a map as a (non-univalent) universe.
We will say that the universe $\mathcal{U}$ \emph{contains a type $A$} if there is a map $a : \Hom(1,\mathcal{U})$ and an equivalence between $A$ and the pullback of $p$ along $a$.
We will say that $\mathcal{U}$ is \emph{closed under coproducts} if, for all maps $a,b : \Hom(D,\mathcal{U})$, there is a map $a + b : \Hom(D,\mathcal{U})$
and an equivalence over $D$ between the pullback of $p$ along $a + b$ and the sum of pullbacks of $p$ along $a$ and $b$.
Similarly, we will say that $\mathcal{U}$ is \emph{closed under $\Sigma$-types} if, for every map $a : \Hom(D,\mathcal{U})$
and every map $b : \Hom(D \times_\mathcal{U} \widehat{\mathcal{U}}, \mathcal{U})$, there is a map $\Sigma(a,b) : \Hom(D,\mathcal{U})$
together with an equivalence over $D$ between the pullback of $p$ along $\Sigma(a,b)$ and the composition of pullbacks of $p$ along $b$ and $a$.
The closure under other constructions is defined similarly.

In general, being closed under different construction is not a property of a map but additional data on it.
The following proposition shows that it is a property if the map $p$ is an object classifier:

\begin{prop}
Let $p : \Hom(\widehat{\mathcal{U}},\mathcal{U})$ be an object classifier.
Then the types corresponding to the closure conditions listed above are propositions.
\end{prop}
\begin{proof}
Such types consist of a map $c : \Hom(D,\mathcal{U})$ for some fixed type $D$ together with an equivalence between the fiber of $p$ over $c$ and some fixed type over $D$.
Let $(c_1,e_1)$ and $(c_2,e_2)$ be two such pairs.
Since $p$ is an object classifier, to define a homotopy between these pairs, it is enough to define a homotopy $e$ between fibers of $p$ over $c_1$ and $c_2$ together with a homotopy between $e \circ e_1$ and $e_2$.
We can define $e$ as $e_2 \circ e_1^{-1}$.
\end{proof}

The following propositions discuss $n$-truncated object classifiers.

\begin{prop}
If $p : \Hom(\widehat{\mathcal{U}},\mathcal{U})$ is an $n$-truncated map which is also an object classifier, then $\mathcal{U}$ and $\widehat{\mathcal{U}}$ are $(n+1)$-truncated.
\end{prop}
\begin{proof}
First, let us prove that $\mathcal{U}$ is $(n+1)$-truncated.
This is true if and only if $\Hom(B,\mathcal{U})$ is $(n+1)$-truncated for all $B$, which is true if and only if the type $\Id(f,f')$ is $n$-truncated for all $f,f' : \Hom(B,\mathcal{U})$.
Since $p$ is an object classifier, the type $\Id(f,f')$ is equivalent to the type of equivalences over $B$ between pullbacks of $p$ along $f$ and $f'$.
By \rlem{trunc-pb}, pullbacks of $p$ are $n$-truncated.
Since the type of equivalences over $B$ is embedded into the type of maps over $B$, we just need to prove that the type of such maps between $n$-truncated maps is $n$-truncated.

Let $s : \Hom(E,B)$ and $s' : \Hom(E',B)$ be $n$-truncated maps.
The type of maps over $B$ is defined as $\Sigma_{f : \Hom(E,E')} \Id(s' \circ f, s)$.
This type is the fiber of $s' \circ -$ over $s$, which is $n$-truncated since $s' \circ -$ is $n$-truncated.

Finally, since both $p$ and $\mathcal{U}$ are $(n+1)$-truncated, $\widehat{\mathcal{U}}$ is also $(n+1)$-truncated.
\end{proof}

\begin{prop}[mono-classifier]
If $p : \Hom(\widehat{\mathcal{U}},\mathcal{U})$ is an object classifier which is also a monomorphism, then $\widehat{\mathcal{U}}$ is subterminal.
It is terminal if and only if identity morphisms are classified by $p$.
\end{prop}
\begin{proof}
Any commutative square of the form
\[ \xymatrix{ B \ar[r]^s \ar[d]_{\id_B} & \widehat{\mathcal{U}} \ar[d]^p \\
              B \ar[r]_t                & \mathcal{U}
            } \]
is a pullback.
To prove this, we need to show that the canonical map
\[ r : \Hom(X,B) \to \sum_{f : \Hom(X,B)} \sum_{g : \Hom(X,\widehat{\mathcal{U}})} \Id(t \circ f, p \circ g). \]
is an equivalence for all $X$.
Since $t$ is homotopic to $p \circ s$, the type $\Id(t \circ f, p \circ g)$ is equivalent to $\Id(p \circ s \circ f, p \circ g)$.
Since $p$ is a monomorphism, it is equivalent to $\Id(s \circ f, g)$.
Since the type $\Sigma_{g : \Hom(X,\widehat{\mathcal{U}})} \Id(s \circ f, g)$ is contractible, $r$ is indeed an equivalence.

To prove that $\widehat{\mathcal{U}}$ is subterminal, we need to show that any two maps $f_1,f_2 : \Hom(B,\widehat{\mathcal{U}})$ are homotopic.
Since $p$ is a monomorphism, it is enough to construct a homotopy between $p \circ f_1$ and $p \circ f_2$.
We have two pullback squares as above with $s = f_i$ and $t = p \circ f_i$.
Since $p$ is an object classifier, we have an equivalence between $\Id(p \circ f_1, p \circ f_2)$ and the type of equivalences between pullbacks of $p$ along $p \circ f_1$ and $p \circ f_2$.
Since both pullbacks are just $\id_B$, they are equivalent; so, we have a homotopy between $p \circ f_1$ and $p \circ f_2$.

If $p$ classifies identity morphisms, then, for every $B$, the map $\id_B$ is a pullback of $p$.
In particular, there exists a map from $B$ to $\widehat{\mathcal{U}}$.
Thus, $\widehat{\mathcal{U}}$ is terminal.
Conversely, if $\widehat{\mathcal{U}}$ is terminal, then, for every type $B$, we have a commutative square as depicted at the beginning of the proof.
Since this square is a pullback, $p$ classifies $\id_B$.
\end{proof}

Finally, let us prove another simple but useful result.
An analogous result in the context of higher categories was proved in \cite[Theorem~3.28]{rasekh-eht}.

\begin{prop}
Let $p : \Hom(\widehat{\mathcal{U}},\mathcal{U})$ be an object classifier and let $f : \Hom(\mathcal{U}',\mathcal{U})$ be any map.
Then the pullback of $p$ along $f$ is an object classifier if and only if $f$ is a monomorphism.
\end{prop}
\begin{proof}
Let us denote the pullback of $p$ along $f$ by $p' : \Hom(\widehat{\mathcal{U}'},\mathcal{U}')$.
Then the pullback of $p'$ along a map $g : \Hom(\Delta,\mathcal{U}')$ is equivalent to the pullback of $p$ along $f \circ g$.
Thus, we have an equivalence between the type of equivalences between $g_1^*(p')$ and $g_2^*(p')$ and the type of equivalences between $(f \circ g_1)^*(p)$ and $(f \circ g_2)^*(p)$.
Since $p$ is an object classifier, the latter type is equivalent to $\Id(f \circ g_1, f \circ g_2)$.
Thus, $p'$ is an object classifier if and only if the canonical function $\Id(g_1,g_2) \to \Id(f \circ g_1, f \circ g_2)$ is an equivalence.
This function maps $\refl(g)$ to $\refl(f \circ g)$.
This implies that it is homotopic to $\lambda h.\,\pmap(f \circ -, h)$, but this map is an equivalence if and only if $f$ is a monomorphism.
\end{proof}

\subsection{Universes}

Object classifiers can be defined in a more type-theoretic way.
An \emph{(internal) universe} in an indexed dependent type theory is an indexed type $\mathcal{U}$ together with an indexed type $\El(c)$ for all $c : \mathcal{U}$:
\begin{center}
\AxiomC{}
\UnaryInfC{$\Gamma \mid \Delta \vdash \mathcal{U} \ob$}
\DisplayProof
\qquad
\AxiomC{$\Gamma \mid \Delta \vdash c : \mathcal{U}$}
\UnaryInfC{$\Gamma \mid \Delta \vdash \El(c) \ob$}
\DisplayProof
\end{center}
An internal universe is \emph{univalent} if the obvious map from $\Id_{\mathcal{U}}(c,c')$ to the type of equivalences between $\El(c)$ and $\El(c')$ is an equivalence.
Internal universes in a theory with $\Sigma$-types and identity types correspond to maps $\Hom(\widehat{\mathcal{U}},\mathcal{U})$ via the construction $\pi_1 : \Hom(\Sigma_{x : \mathcal{U}} \El(x), \mathcal{U})$.
Such a map is an object classifier if and only if the corresponding universe is univalent.

A universe $\mathcal{U}$ is \emph{weakly (resp., strictly) contains a type $A$} if there is an element $a : \mathcal{U}$ such that $\El(a)$ is propositionally (resp., judgmentally) equivalent to $A$. 
A universe $\mathcal{U}$ is \emph{weakly (resp., strictly) closed under coproducts} if, for all elements $a,b : \mathcal{U}$, there exists an element $a + b : \mathcal{U}$ such that $\El(a + b)$ is propositionally (resp., judgmentally) equivalent to the coproduct of $\El(a)$ and $\El(b)$.
A universe $\mathcal{U}$ is \emph{weakly (resp., strictly) closed under $\Sigma$-types} if, for every element $a : \mathcal{U}$ and every function $b : \El(a) \to \mathcal{U}$, there exists an element $\Sigma(a,x.b(x)) : \mathcal{U}$ such that $\El(\Sigma(a,x.b(x)))$ is propositionally (resp., judgmentally) equivalent to $\Sigma_{x : \El(a)} \El(b(x))$.

\begin{prop}
A universe $\mathcal{U}$ is weakly closed under one of the constructions listed above if and only if the map $\pi_1 : \Hom(\Sigma_{A : \mathcal{U}} \El(A), \mathcal{U})$ is closed under this construction.
\end{prop}
\begin{proof}
This follows from the fact that elements of $\mathcal{U}$ in a context $\Delta$ correspond to maps $\Hom(\Delta, \mathcal{U})$ and
types of the form $\El(c)$ over $\Delta$ correspond to pullbacks of $\pi_1 : \Hom(\Sigma_{A : \mathcal{U}} \El(A), \mathcal{U})$ along $c$.
\end{proof}

We already saw that object classifiers (and hence internal universes) may not exists in some interesting theories.
To fix this problem, we will define the notion of an \emph{external universe}.
In an indexed unary type theory, an external universe consists of a base type $\mathcal{U}_A$ defined for every closed indexed type $A$ together with an indexed type over $A$.
In an indexed dependent type theory, this can be reformulated as the following rules:
\begin{center}
\AxiomC{$\Gamma \mid \cdot \vdash A \ob$}
\UnaryInfC{$\Gamma \vdash \mathcal{U}_A \type$}
\DisplayProof
\qquad
\AxiomC{$\Gamma \vdash c : \mathcal{U}_A$}
\AxiomC{$\Gamma \mid \Delta \vdash a : A$}
\BinaryInfC{$\Gamma \mid \Delta \vdash \El(c,a) \ob$}
\DisplayProof
\end{center}
We have a function from $\Id_{\mathcal{U}_A}(c,c')$ to the type of equivalences between dependent $\Hom$-types $\Hom(x.\,\El(c,x))$ and $\Hom(x.\,\El(c',x))$ defined as the transport of the identity morphism along the homotopy between $c$ and $c'$.
We will say that the universe is \emph{univalent} if this function is an equivalence.
We will say that a universe \emph{classifies} fibrations $\Fib$ if it is univalent, the family $\El(c,a)$ satisfies $\Fib$ for all $c$ and $a$,
the class $\Fib$ is stable under substitution, and every fibration is equivalent to a fibration of the form $\El(c,a)$ for some $c$ and $a$.

Every internal universe $(\mathcal{U},\El)$ gives rise to an external one: $\mathcal{U}_A = \Hom(A,\mathcal{U})$ and $\El(c,a) = \El(c\,a)$.
An internal universe is univalent if and only if the corresponding external one is.
Moreover, a class of fibrations is classified by an internal universe if and only if it is classified by the corresponding external one.
An external universe classifying a given class of fibrations is also unique up to a canonical equivalence.
The proof is the same as for internal universes.

\begin{defn}
We will say that a class $\Fib$ is \emph{locally small} if it is classified by an external universe.
\end{defn}

\begin{defn}
An indexed type theory is called \emph{well-powered} if the class of monomorphisms is locally small.
This definition is analogous to the definition of well-powered indexed categories (see \cite[Example~B1.3.14]{elephant}).
\end{defn}

\begin{example}
If the base theory has a univalent universe $\mathcal{U}$, then we can define the external universe $\mathcal{U}$-small spectra.
If $A$ is an indexed type, then we define $\mathcal{U}_A$ as the type consisting of a sequence of maps $B_n : U_S(A)_n \to \Sigma_{X : \mathcal{U}} \El(X)$
together with a sequence of pointed equivalences $(\Sigma_{a : U_S(A)_n} \El(\pi_1(B_n\,a))) \simeq \Omega(\Sigma_{a : U_S(A)_{n+1}} \El(\pi_1(B_{n+1}\,a)))$ over the structure maps $U_S(A)_n \to \Omega(U_S(A)_{n+1})$.
Every such pair of sequences gives rise to a spectrum over $A$ in the obvious way.
\end{example}

External universes can be used to define function on indexed types.
For example, the operation $\Sigma^n$ on indexed types that we explicitly added in section~\ref{sec:finite-limits} can be defined as a function on $\mathcal{U}$-small indexed types by recursion on $n$.

\section{Locally reflective classes of fibrations}
\label{sec:refl-fib}

In this section, we discuss the notion of modalities in indexed type theories.
Several equivalent definitions of modalities were defined in \cite{modality-hott} in the context of ordinary homotopy type theory.
We define the notion of locally reflective classes of fibrations which is similar to the notion of a reflective subuniverse.
This definition makes sense in an indexed unary type theory.
We also define a dependent version of this notion which is similar to the notion of a higher modality.

\subsection{Locally reflective classes in unary theories}

Let $\Fib$ be a class of fibrations in an indexed unary type theory as defined in the previous section.
We will say that $\Fib$ is \emph{locally reflective} if every map $f : \Hom(A,B)$ factors through a fibration $p : \Hom(C,B)$
such that, for every factorization of $f$ through any fibration $p' : \Hom(C',B)$, the type of lifts in the following square is contractible:
\[ \xymatrix{ A \ar[r] \ar[d]                   & C' \ar@{->>}[d]^{p'} \\
              C \ar@{->>}[r]_p \ar@{-->}[ur]    & B
            } \]
The factorization $A \to C \twoheadrightarrow B$ will be called \emph{the universal factorization} of $f$.
We will say that $\Fib$ is \emph{stably} locally reflective if the universal factorization of any map is stable under pullbacks.

\begin{example}
Suppose that we have a locally reflected class of fibrations in the base theory.
We will say that a map of indexed types in the theory of pointed types is a fibration if the underlying map of base types is a fibration.
If $f : \Hom(A,B)$ is a map of pointed types and $i : \Hom(S^0,A) \to C$, $p : C \to \Hom(S^0,B)$ is the universal factorization of the underlying map,
then it is easy to see that $f$ factors through $R(C, i\,*_A)$, where $*_A$ is the base point of $\Hom(S^0,A)$.
A tedious but straightforward computation shows that the type of lifts in the square depicted above in indexed types is equivalent to the type of lifts in the square underlying it in base types.
It follows that this defines a locally reflected class of fibrations in the theory of pointed types.
Since $\Hom(S^0,-)$ preserves pullbacks, this class is stably locally reflective if and only if this is true for fibrations of base types.
\end{example}

\begin{example}
If the base theory has the $n$-truncation operation, then the class of $n$-truncated maps in the theory of pointed types is stably locally reflective.
This is a spectial case of the previous example.
\end{example}

\begin{example}
We already saw that $n$-truncated maps in the theory of spectra are not interesting.
It is possible to define a more useful notion of $n$-truncatedness.
Let $n$ be an integer.
We will say that a map $f$ of spectra is $n$-truncated if, for every $i : \mathbb{N}$ such that $i + n \geq 0$, the map $U_S(f)_i$ is $(i + n)$-truncated.
Then the class of $n$-truncated maps is locally reflective for every $n : \mathbb{Z}$.
\end{example}

\begin{lem}[fib-refl]
If $A \xrightarrow{i} C \overset{p}\twoheadrightarrow B$ is the universal factorization of $f$, then the following conditions are equivalent:
\begin{enumerate}
\item $i$ is an equivalence.
\item $f$ is a fibration.
\item $i$ has a retraction over $B$.
\end{enumerate}
\end{lem}
\begin{proof}
If $i$ is an equivalence, then $f$ is a fibration since fibrations are closed under equivalences and $p$ is a fibration.
If $f$ is a fibration, then the lift in the following square is a retraction of $i$ over $B$.
\[ \xymatrix{ A \ar@{=}[r] \ar[d]_i             & A \ar@{->>}[d]^{f} \\
              C \ar@{->>}[r]_p \ar@{-->}[ur]^r  & B
            } \]
Let $r$ be a retraction of $i$ over $B$.
Consider the following commutative square:
\[ \xymatrix{ A \ar[r]^i \ar[d]_i                                                           & C \ar@{->>}[d]^{p} \\
              C \ar@{->>}[r]_p \ar@{-->}@<-0.5ex>[ur]_{i \circ r} \ar@{-->}@<0.5ex>[ur]^\id & B
            } \]
It is easy to see that $\id_C$ and $i \circ r$ are lifts in this square.
Since lifts are unique up to a homotopy, these maps are homotopic.
Thus, $i$ is an equivalence.
\end{proof}

\begin{lem}[fib-idm]
Any locally reflective class of fibrations contains all identity morphisms
\end{lem}
\begin{proof}
Let $A \xrightarrow{i} B \overset{p}\twoheadrightarrow A$ be the universal factorization of $\id_A$.
Then $p$ is a retraction of $i$ over $A$.
By \rlem{fib-refl}, $\id_A$ is a fibration.
\end{proof}

\begin{lem}[fib-pullback]
Any stably locally reflective class of fibrations is closed under pullbacks.
\end{lem}
\begin{proof}
Let $f$ be a fibration and let $f = p \circ i$ be its universal factorization.
Since the class of fibrations is stably locally reflective, the pullbacks of $p$ and $i$ constitute the universal factorization of a pullback of $f$.
By \rlem{fib-refl}, $i$ is an equivalence.
Hence, its pullback is also an equivalence.
Since the pullback of $p$ is a fibration and fibrations are closed under equivalences, the pullback of $f$ is also a fibration.
\end{proof}

\begin{lem}[pullback-lift]
Suppose that we have the following diagram, where the right square is a pullback.
\[ \xymatrix{ A \ar[r]^{c} \ar[d]_i & C \ar[r]^e \ar[d]_p \pb   & E \ar[d]^q \\
              B \ar[r]_d            & D \ar[r]_f                & F
            } \]
Then the type of lifts in the left square is equivalent to the type of lifts in the outer rectangle.
\end{lem}
\begin{proof}
Let $H_1$ and $H_2$ be the homotopies witnessing the commutativity of the left and right square, respectively.
By the universal property of pullbacks, the type of lifts in the left square is equivalent to the following type:
\begin{align*}
& \sum_{r_1 : \Hom(B,D)} \sum_{r_2 : \Hom(B,E)} \sum_{r_3 : \Id(f \circ r_1, q \circ r_2)} \sum_{h : \Id(d,r_1)} \\
& \sum_{h_1 : \Id(r_1 \circ i, p \circ c)} \sum_{h_2 : \Id(e \circ c, r_2 \circ i)} \sum_{h_3 : \Id((h_1 * f) \ct (c * H_2) \ct (h_2 * q), i * r_3)} \Id(h_1,h_*(H_1)).
\end{align*}
After reducing $r_1$, $h$, $h_1$, and the last homotopy we get the following equivalent type:
\[ \sum_{r_2 : \Hom(B,E)} \sum_{r_3 : \Id(f \circ d, q \circ r_2)} \sum_{h_2 : \Id(e \circ c, r_2 \circ i)} \Id((H_1 * f) \ct (c * H_2) \ct (h_2 * q), i * r_3). \]
This type is equivalent to the type of lifts in the outer rectangle.
\end{proof}

\begin{lem}[fib-refl-lift]
Let $\Fib$ be a locally reflective class of fibrations closed under pullbacks.
Let $A \xrightarrow{i} C \overset{p}\twoheadrightarrow B$ be the universal factorization of a map $f : \Hom(A,B)$.
Then the type of lifts in every commutative square as below is contractible if $v$ factors through $p$.
\[ \xymatrix{ A \ar[r] \ar[d]_i & D \ar@{->>}[d] \\
              C \ar[r]_v        & E
            } \]
\end{lem}
\begin{proof}
By assumption, $v$ equals to $C \overset{p}\twoheadrightarrow B \xrightarrow{u} E$ for some map $u$.
Consider the following diagram:
\[ \xymatrix{ A \ar[r] \ar[d]_i & C' \ar[r] \ar@{->>}[d] \pb    & D \ar@{->>}[d] \\
              C \ar@{->>}[r]_p  & B \ar[r]_u                    & E
            } \]
The type of lift in the left square is contractible and \rlem{pullback-lift} implies that this type is equivalent to the type of lifts in the original square.
\end{proof}

\begin{prop}[fib-refl-id]
Any locally reflective class of fibrations closed under pullbacks satisfies the equivalent conditions of \rprop{fib-id}.
\end{prop}
\begin{proof}
By \rlem{fib-id-idm} and \rlem{fib-idm}, it is enough to prove condition~\eqref{it:fib-over} of \rprop{fib-id-comp}.
Let $f : \Hom(A,D)$ be a map and let $q : \Hom(D,B)$ be a fibration such that $q \circ f$ is also a fibration.
We need to prove that $f$ is a fibration.
Let $A \xrightarrow{i} C \overset{p}\twoheadrightarrow D$ be the universal factorization of $f$.
By \rlem{fib-refl}, it is enough to show that $i$ has a retraction over $D$.
By \rlem{fib-refl-lift}, we have a lift in the following square:
\[ \xymatrix{ A \ar@{=}[r] \ar[d]_i                 & A \ar@{->>}[d]^{q \circ f} \\
              C \ar[r]_{q \circ p} \ar@{-->}[ur]^r  & B
            } \]
Both maps $p$ and $f \circ r$ are lifts in the following square:
\[ \xymatrix{ A \ar[r]^f \ar[d]_i   & D \ar@{->>}[d]^q \\
              C \ar[r]_{q \circ p}  & B
            } \]
By \rlem{fib-refl-lift}, we have a homotopy $h$ between $p$ and $f \circ r$ such that $i * h$ is homotopic to the canonical homotopy between $p \circ i$ and $f \circ r \circ i$.
This implies that $r$ is a retraction of $i$ over $D$.
\end{proof}

\subsection{Orthogonal factorization systems}

In this subsection, we defined connected maps and orthogonal factorization systems and prove that they are equivalent to locally reflective classes of fibrations.
Similar equivalence for internal factorization systems was proved in \cite{modality-hott}.
We will discuss the relationship between internal and external systems.

\begin{defn}
Let $\Fib$ be a locally reflective class of fibrations in an indexed unary type theory.
A map $f$ is \emph{connected} if the fibration in the universal factorization of $f$ is an equivalence.
\end{defn}

\begin{lem}[uni-conn]
Let $\Fib$ be a locally reflective class of fibrations closed under composition.
If $A \xrightarrow{i} B \overset{p}\twoheadrightarrow C$ is the universal factorization of some map, then $i$ is connected.
\end{lem}
\begin{proof}
Let $A \xrightarrow{j} B' \overset{q}\twoheadrightarrow B$ be the universal factorization of $i$.
We need to prove that $q$ is an equivalence.
Since fibrations are closed under composition, $p \circ q$ is a fibration.
It follows that we have a lift in the following square:
\[ \xymatrix{ A \ar[r]^j \ar[d]_i               & B' \ar@{->>}[d]^{p \circ q} \\
              B \ar@{->>}[r]_p \ar@{-->}[ur]^k  & C
            } \]
Let $h_1 : \Id(j, k \circ i)$ and $h_2 : \Id(p \circ q \circ k, p)$ be the homotopies witnessing the commutativity of triangles in the diagram above.

Let us show that $q \circ k$ is a lift in the following square:
\[ \xymatrix{ A \ar[r]^i \ar[d]_i                   & B \ar@{->>}[d]^p \\
              B \ar@{->>}[r]_p \ar[ur]^{q \circ k}  & C
            } \]
Let $h_0$ be the homotopy between $i$ and $q \circ j$.
The homotopy between $i$ and $q \circ k \circ i$ is defined as $h_0 \ct (h_1 * q)$.
The homotopy between $p \circ q \circ k$ and $p$ is simply $h_2$.
The fact that the combination of these homotopies is homotopic to the trivial homotopy on $p \circ i$ follows from the fact that the combination of $h_1$ and $h_2$ is homotopic to $\sym{h_0} * p$.
Since both $\id_B$ and $q \circ k$ are lifts in the square above, there is a homotopy $h_3$ between them such that $i * h_3$ is homotopic to $h_0 \ct (h_1 * q)$.

Let us show that $k \circ q$ is a lift in the following square:
\[ \xymatrix{ A \ar[r]^j \ar[d]_j                   & B' \ar@{->>}[d]^q \\
              B' \ar@{->>}[r]_q \ar[ur]^{k \circ q} & B
            } \]
The homotopy between $j$ and $k \circ q \circ j$ is defined as $h_1 \ct (h_0 * k)$.
The homotopy between $q \circ k \circ q$ and $q$ is defined as $q * \sym{h_3}$.
The fact that the combination of these homotopies is homotopic to the trivial homotopy on $q \circ j$ follows from the fact that $i * h_3$ is homotopic to $h_0 \ct (h_1 * q)$.
Since both $\id_{B'}$ and $k \circ q$ are lifts in the square above, these maps are homotopic.
It follows that $q$ is an equivalence.
Hence, $i$ is connected.
\end{proof}

\begin{lem}[conn-pullback]
Let $\Fib$ be a stably locally reflective class of fibrations.
Then connected maps are closed under pullbacks.
\end{lem}
\begin{proof}
Let $f : \Hom(A,C)$ be a connected map and let $g : \Hom(D,C)$ be an arbitrary map.
We need to prove that the pullback of $f$ along $g$ is connected.
Let $A \xrightarrow{i} B \overset{p}\twoheadrightarrow C$ be the universal factorization of $f$.
Then we have the following diagram:
\[ \xymatrix{ A' \ar[r] \ar[d] \pb  & A \ar[d]^i \\
              B' \ar[r] \ar[d] \pb  & B \ar@{->>}[d]^p \\
              D  \ar[r]_g           & C
            } \]
Since $\Fib$ is stably locally reflective, $A' \to B' \to D$ is the universal factorization of $A' \to D$.
Since $f$ is connected, $p$ is an equivalence.
Hence, $B' \to D$ is also an equivalence.
Thus, $A' \to D$ is connected.
\end{proof}

\begin{defn}
Let $f : \Hom(A,B)$ and $g : \Hom(C,D)$ be maps in an indexed unary type theory.
We will say that $f$ is \emph{left orthogonal} to $g$ and $g$ is \emph{right orthogonal} to $f$ if the type of lifts in squares of the form depicted below is contractible.
\[ \xymatrix{ A \ar[r] \ar[d]_f         & C \ar[d]^g \\
              B \ar[r] \ar@{-->}[ur]    & D
            } \]
\end{defn}

\begin{lem}[conn-orth]
Let $\Fib$ be a locally reflective class of fibrations closed under pullbacks.
Then connected maps are left orthogonal to fibrations.
\end{lem}
\begin{proof}
Let $i : \Hom(A,B)$ be a connected map and let $p : \Hom(C,D)$ be a fibration.
Consider a commutative square of the following form:
\[ \xymatrix{ A \ar[d]_i \ar[r]^f   & C \ar[d]^p \\
              B \ar[r]_g            & D
            } \]
The type of lifts in this square is
\[ \sum_{r : \Hom(B,C)} \sum_{h_1 : \Id(f, r \circ i)} \sum_{h_2 : \Id(p \circ r, g)} \Id((h_1 * p) \ct (i * h_2), H), \]
where $H$ is the homotopy witnessing the commutativity of the square.
Let us denote this type by $L$.
We need to prove that $L$ is contractible.

Let $A \xrightarrow{i'} B' \overset{q}\twoheadrightarrow B$ be the universal factorization of $i$.
By \rlem{fib-refl-lift}, the type of lifts in the following square is contractible:
\[ \xymatrix{ A \ar[d]_{i'} \ar[rr]^f       &               & C \ar[d]^p \\
              B' \ar[r]_q \ar@{-->}[urr]    & B \ar[r]_g    & D
            } \]
The type of lifts in this square is defined as follows:
\[ \sum_{r' : \Hom(B',C)} \sum_{h_1' : \Id(f, r' \circ i')} \sum_{h_2 : \Id(p \circ r', g \circ q)} \Id((h_1' * p) \ct (i' * h_2'), H \ct (H' * g)), \]
where $H'$ is the homotopy between $i$ and $q \circ i'$.
Let us denote this type by $L'$.
It is enough to prove that $L$ and $L'$ are equivalent.

We have an obvious map $s : L \to L'$ which maps $(r,h_1,h_2,h_3)$ to $(r \circ q, h_1 \ct (H' * r), q * h_2, h_3')$, where $h_3'$ is the following homotopy:
\begin{align*}
((h_1 \ct (H' * r)) * p) \ct (i' * q * h_2) & \sim \\
((h_1 * p) \ct (H' * r * p)) \ct (i' * q * h_2) & \sim \\
(h_1 * p) \ct (H' * h_2) & \sim \\
((h_1 * p) \ct (i * h_2)) \ct (H' * g) & \sim \\
H \ct (H' * g) & ,
\end{align*}
where we use $h_3$ at the last step and other steps are usual interchange laws.

To prove that this map is an equivalence, it is enough to show that it is an equivalence on each component.
Since $i$ is connected, $q$ is an equivalence.
This implies that functions $- \circ q$ and $q * -$ are equivalences and these functions are the first and the third component of $s$, respectively.
The second component of $s$ is $- \ct (H' * r)$, which is also an equivalence.
Finally, the third component of $s$ is an equivalence since it is a function that concatenates its argument with fixed homotopies.
\end{proof}

Let $\mathcal{L}$ and $\mathcal{R}$ be a pair of classes of maps closed under equivalences such that maps in $\mathcal{L}$ are left orthogonal to maps in $\mathcal{R}$.
Then a factorization of a map into a map in $\mathcal{L}$ followed by a map in $\mathcal{R}$ is essentially unique.
The pair $(\mathcal{L},\mathcal{R})$ is called an \emph{orthogonal factorization system} if such a factorization exists for every map.

\begin{lem}[orth-refl]
If $(\mathcal{L},\mathcal{R})$ is an orthogonal factorization system, then $\mathcal{R}$ is a locally reflective class of maps.
Moreover, if $A \xrightarrow{i} B \twoheadrightarrow C$ is the universal factorization of some map, then $i$ belongs to $\mathcal{L}$.
\end{lem}
\begin{proof}
Obviously, any factorization of a map into a map in $\mathcal{L}$ followed by a map in $\mathcal{R}$ is a universal factorization.
The second assertion follows from the facts that the universal factorization is essentially unique and $\mathcal{L}$ is closed under equivalences.
\end{proof}

\begin{prop}
If $(\mathcal{L},\mathcal{R})$ is an orthogonal factorization system, then $\mathcal{L}$ and $\mathcal{R}$ contain all identity morphisms.
\end{prop}
\begin{proof}
By \rlem{orth-refl} and \rlem{fib-idm}, $\mathcal{R}$ contains all identity morphisms.
Since $A \xrightarrow{\id_A} A \xrightarrow{\id_A} A$ is the universal factorization of $\id_A$, \rlem{orth-refl} implies that $\id_A$ belongs to $\mathcal{L}$.
\end{proof}

\begin{prop}[orth-char]
If $(\mathcal{L},\mathcal{R})$ is an orthogonal factorization system, then
$\mathcal{R}$ is precisely the class of maps which are right orthogonal to $\mathcal{L}$ and
$\mathcal{L}$ is precisely the class of maps which are left orthogonal to $\mathcal{R}$.
\end{prop}
\begin{proof}
We prove the first assertion; the other one follows by a dual argument.
Maps in $\mathcal{R}$ are right orthogonal to $\mathcal{L}$ by definition.
Let $f : \Hom(A,C)$ be a map which is right orthogonal to $\mathcal{L}$.
We need to prove that $f$ belongs to $\mathcal{R}$.
By \rlem{orth-refl}, $\mathcal{R}$ is a locally reflective class of maps.
Let $A \xrightarrow{i} B \xrightarrow{p} C$ be the universal factorization of $f$.
By \rlem{orth-refl}, $i$ belongs to $\mathcal{L}$.
Hence, we have a lift in the following square:
\[ \xymatrix{ A \ar@{=}[r] \ar[d]_i     & A \ar[d]^f \\
              B \ar[r]_p \ar@{-->}[ur]  & C
            } \]
By \rlem{fib-refl}, $f$ belongs to $\mathcal{R}$.
\end{proof}

\begin{cor}[orth-unique]
Let $\mathcal{R}$ be a class of maps.
Then a class of maps $\mathcal{L}$ such that $(\mathcal{L},\mathcal{R})$ is an orthogonal factorization system is essentially unique.
That is, if $\mathcal{L}_1$ and $\mathcal{L}_2$ is two such classes, then a map belongs to one of them if and only if it belongs to the other.
Dually, a class of maps $\mathcal{R}$ such that $(\mathcal{L},\mathcal{R})$ is an orthogonal factorization system for a fixed $\mathcal{L}$ is essentially unique.
\end{cor}

\begin{prop}[orth-comp]
If $(\mathcal{L},\mathcal{R})$ is an orthogonal factorization system, then $\mathcal{R}$ and $\mathcal{L}$ are closed under composition.
\end{prop}
\begin{proof}
We prove this for $\mathcal{R}$; the assertion about $\mathcal{L}$ follows by a dual argument.
Let $f : \Hom(A,B)$ and $g : \Hom(B,C)$ be maps in $\mathcal{R}$.
By \rlem{orth-refl}, there exists a universal factorization $A \xrightarrow{i} D \xrightarrow{p} C$ of $g \circ f$ such that $i \in \mathcal{L}$.
By \rlem{fib-refl}, to prove that $g \circ f$ belongs to $\mathcal{R}$, it is enough to show that $i$ has a retraction over $C$.
Since $i \in \mathcal{L}$, we have two lifts in the following diagram:
\[ \xymatrix{ A \ar@{=}[r] \ar[dd]_i                    & A \ar@{->>}[d]^f \\
                                                        & B \ar@{->>}[d]^g \\
              D \ar[r]_p \ar@{-->}[ur] \ar@{-->}[uur]^r & C
            } \]
Then $r$ is a retraction of $i$ over $C$.
\end{proof}

\begin{prop}[orth-pullback]
If $(\mathcal{L},\mathcal{R})$ is an orthogonal factorization system, then $\mathcal{R}$ is closed under pullbacks and $\mathcal{L}$ is closed under pushouts.
\end{prop}
\begin{proof}
We prove this for $\mathcal{R}$; the assertion about $\mathcal{L}$ follows by a dual argument.
Let $f$ be a pullback of a map $g \in \mathcal{R}$.
Since $g$ is right orthogonal to $\mathcal{L}$, \rlem{pullback-lift} implies that $f$ is also right orthogonal to $\mathcal{L}$.
\rprop{orth-char} implies that $f \in \mathcal{R}$.
\end{proof}

Now, we are ready to prove that orthogonal factorization systems are equivalent to locally reflective classes of maps which are closed under composition and pullbacks:

\begin{thm}[refl-orth]
Let $\Fib$ be a class of fibrations.
If $\Fib$ is locally reflective and closed under composition and pullbacks, then $(\mathcal{C},\Fib)$ is an orthogonal factorization system, where $\mathcal{C}$ is the class of connected maps.
The converse is also true in the sense that if $(\mathcal{L},\Fib)$ is an orthogonal factorization system for some class of maps $\mathcal{L}$, then $\Fib$ is locally reflective and closed under composition and pullbacks.
\end{thm}
\begin{proof}
If $\Fib$ is locally reflective and closed under composition and pullbacks, then connected maps are left orthogonal to fibrations by \rlem{conn-orth} and the factorization exists by \rlem{uni-conn}.
Thus, $(\mathcal{C},\Fib)$ is an orthogonal factorization system.
Conversely, if we have an orthogonal factorization system $(\mathcal{L},\Fib)$, then $\Fib$ is locally reflective by \rlem{orth-refl} and it is closed under composition and pullbacks by \rprop{orth-comp} and \rprop{orth-pullback}, respectively.
\end{proof}

\begin{defn}
Let $f : \Hom(A,B)$ and $g : \Hom(C,D)$ be maps in a Cartesian closed indexed unary type theory.
We will say that $f$ is \emph{internally left orthogonal} to $g$ and $g$ is \emph{internally right orthogonal} to $f$ if the type of internal lifts in squares of the form depicted below is contractible.
\[ \xymatrix{ A \ar[r]^u \ar[d]_f       & C \ar[d]^g \\
              B \ar[r]_v \ar@{-->}[ur]  & D
            } \]
That is, the following indexed type should be contractible:
\[ \sum_{l : B \to C} \sum_{s : \Id(u, l \circ f)} \sum_{t : \Id(g \circ l, v)} \Id((s * g) \cdot (f * t), q), \]
where $q$ is the witness of commutativity of the original square.
\end{defn}

Let $f : \Hom(A,B)$ and $g : \Hom(C,D)$ be a pair of maps.
An indexed type is contractible if and only if it is terminal.
Thus, $f$ is internally left orthogonal to $g$ if and only if, for every indexed type $E$, the type following type is contractible:
\[ \Hom(E,\sum_{l : B \to C} \sum_{s : \Id(u, l \circ f)} \sum_{t : \Id(g \circ l, v)} \Id((s * g) \cdot (f * t), q)). \]
By \rprop{hom-sigma-id}, $\Hom$ commutes with $\Sigma$ and $\Id$.
It follows that this type is equivalent to the following one:
\[ \sum_{l : E \times B \to C} \sum_{s : \Id(u', l \circ f')} \sum_{t : \Id(g \circ l, v')} \Id((s * g) \cdot (f' * t), q'), \]
where $u'$ is the composite $E \times A \xrightarrow{\pi_2} A \xrightarrow{u} C$, $v'$ is the composite $E \times B \xrightarrow{\pi_2} B \xrightarrow{v} D$, $f'$ is the map $E \times A \xrightarrow{\id_E \times f} E \times B$,
and $q'$ is the witness of commutativity of the following square:
\[ \xymatrix{ E \times A \ar[r]^-{u'} \ar[d]_{f'}   & C \ar[d]^g \\
              E \times B \ar[r]_-{v'}               & D
            } \]
Thus, $f$ is internally left orthogonal to $g$ if and only if $\id_E \times f$ is left orthogonal to $g$ for all indexed types $E$.

\subsection{Higher modalities}

We can reformulate the definition of locally reflective class of fibrations in indexed dependent type theories.
Let $\Fib$ be a class of fibrations closed under substitutions and identity types.
A factorization of a map in a unary theory can be turned into a factorization of a dependent type $\Gamma \mid x : B \vdash A \ob$, which consists of a dependent type $\Gamma \mid x : B \vdash \| A \| \ob$ and a map $\eta_A : \Hom_B(A, \| A \|)$.
Then $\Fib$ is locally reflective if and only if, for every dependent type $A$, there exists its factorization such that the following function is an equivalence for every fibrant type $\Gamma \mid x : B \vdash C \ob$:
\[ \lambda f.\, f \circ \eta_A : \Hom_B(\| A \|, C) \to \Hom_B(A, C). \]
This condition holds if and only if the type of lifts in the following diagram is contractible for every fibrant type $\Gamma \mid x : B \vdash C \ob$ and every map $\Hom_B(A,C)$:
\[ \xymatrix{ A \ar[r] \ar[d]_{\eta_A} & C \\
              \| A \| \ar@{-->}[ur]
            } \]

The theory of \emph{weak higher modalities} (with respect to $\Fib$) is defined as follows:
\begin{center}
\AxiomC{$\Gamma \mid x : B \vdash A \ob$}
\AxiomC{$\Gamma \mid \Delta \vdash b : B$}
\BinaryInfC{$\Gamma \mid \Delta \vdash \| x.A, b \| \fib$}
\DisplayProof
\end{center}
\medskip

\begin{center}
\AxiomC{$\Gamma \mid x : B \vdash A \ob$}
\AxiomC{$\Gamma \mid \Delta \vdash b : B$}
\AxiomC{$\Gamma \mid \Delta \vdash a : A[b/x]$}
\TrinaryInfC{$\Gamma \mid \Delta \vdash | a, b | : \| x.A, b \|$}
\DisplayProof
\end{center}
\medskip

\begin{center}
\def\extraVskip{1pt}
\Axiom$\fCenter \Gamma \mid x : B, z : \| x.A, x \| \vdash D \fib$
\noLine
\UnaryInf$\fCenter \Gamma \mid x : B, y : A \vdash d : D[|y,x|/z]$
\Axiom$\fCenter \Gamma \mid \Delta \vdash b : B$
\noLine
\UnaryInf$\fCenter \Gamma \mid \Delta \vdash a : \| x.A, b \|$
\def\extraVskip{2pt}
\BinaryInfC{$\Gamma \mid \Delta \vdash \| x.A, b \|\text{-}\fs{elim}(x z. D, x y. d, a) : D[b/x, a/z]$}
\DisplayProof
\end{center}
\medskip

\begin{center}
\def\extraVskip{1pt}
\Axiom$\fCenter \Gamma \mid x : B, z : \| x.A, x \| \vdash D \fib$
\noLine
\UnaryInf$\fCenter \Gamma \mid x : B, y : A \vdash d : D[|y,x|/z]$
\Axiom$\fCenter \Gamma \mid \Delta \vdash b : B$
\noLine
\UnaryInf$\fCenter \Gamma \mid \Delta \vdash a : A[b/x]$
\def\extraVskip{2pt}
\BinaryInfC{$\Gamma \mid \Delta \vdash \| x.A, b \|\text{-}\fs{elim_h}(x z. D, x y. d, a) : \Id(\| x.A, b \|\text{-}\fs{elim}(x z. D, x y. d, |a,b|), d[b/x, a/y])$}
\DisplayProof
\end{center}
\medskip
Of course, we can also define the theory of higher modalities in which the last rule holds judgementally.
The theory of \emph{stable higher modalities} is defined as follows:
\begin{center}
\AxiomC{$\Gamma \mid \Delta \vdash A \ob$}
\UnaryInfC{$\Gamma \mid \Delta \vdash \| A \| \fib$}
\DisplayProof
\qquad
\AxiomC{$\Gamma \mid \Delta \vdash a : A \ob$}
\UnaryInfC{$\Gamma \mid \Delta \vdash |a| : \| A \|$}
\DisplayProof
\end{center}
\medskip

\begin{center}
\AxiomC{$\Gamma \mid \Delta, z : \| A \| \vdash D \fib$}
\AxiomC{$\Gamma \mid \Delta, y : A \vdash d : D[|y|/z]$}
\AxiomC{$\Gamma \mid \Delta \vdash a : \| A \|$}
\TrinaryInfC{$\Gamma \mid \Delta \vdash \| A \|\text{-}\fs{elim}(z.D, y.d, a) : D[a/z]$}
\DisplayProof
\end{center}
\medskip

\[ \| A \|\text{-}\fs{elim}(z.D, y.d, |a|) = d[a/y] \]
Of course, every stable higher modality determines a higher modality: $\| x.A, b \|$ can be defined as $\| A[b/x] \|$.

\begin{example}
We already mentioned that $\qCat$ has several interesting classes of fibrations.
Most of them are locally reflective but not stably locally reflective.
We will see that the existence of a stable higher modality with respect to a class of fibrations implies that this class is stably locally reflective.
Thus, only the unstable version of higher modalities can be defined in $\qCat$ for such classes.
\end{example}

The following proposition shows that weak higher modalities are equivalent to locally reflective classes of fibrations:

\begin{prop}[fib-dep-unst]
Suppose that an indexeded dependent type theory has $\Sigma$-types, unit types, identity types, and extensional $\Hom$-types.
Let $\Fib$ be a class of fibrations closed under substitutions.
Then the following conditions are equivalent:
\begin{enumerate}
\item \label{it:higher-modality} The theory has a weak higher modality with respect to $\Fib$ and $\Fib$ is closed under identity types.
\item \label{it:locally-reflective} $\Fib$ is locally reflective and is closed under $\Sigma$-types.
\end{enumerate}
\end{prop}
\begin{proof}
\eqref{it:higher-modality} $\implies$ \eqref{it:locally-reflective}
Let $\Gamma \mid x : B \vdash A \ob$ be an indexed type over $B$.
Then we can define $\| A \|$ as $\Gamma \mid x : B \vdash \| x.A, x \|$.
Function $\eta_A : \Hom_B(A, \| x.A, x \|)$ is defined as $\lambda a.\,| a, x |$.

Let $C$ be a fibrant type over $B$.
We need to prove that the type of lifts in the following diagram is contractible:
\[ \xymatrix{ A \ar[r]^c \ar[d]_{\eta_A} & C \\
              \| x.A, x \| \ar@{-->}[ur]
            } \]
We can define a lift in this diagram as $\lambda z.\,\| x.A, x \|\text{-}\fs{elim}(x z. C, x y.\,c\,y, z)$.
The commutativity of the triangle is witnessed by the term $\lambda y.\,\| x.A, x \|\text{-}\fs{elim_h}(x z. C, x y.\,c\,y, y)$.

Let $\Gamma \mid x : B, z : \| x.A, x \| \vdash f_i : C$, $\Gamma \mid x : B, y : A \vdash h_i : \Id(f_i[|y,x|/z],\,c\,y)$ be lifts in this diagram for $i \in \{1,2\}$.
We need to construct a homotopy between them.
Since $\Fib$ is closed under identity types, the type $\Id(f_1,f_2)$ is fibrant over $x : B, z : \| x.A, x \|$.
Thus, we have a homotopy between $f_1$ and $f_2$:
\[ \Gamma \mid x : B, z : \| x.A, x \| \vdash \| x.A, x \|\text{-}\fs{elim}(x z.\,\Id(f_1,f_2), x y.\,h_1 \ct \sym{h_2}, z) : \Id(f_1, f_2). \]
Let us denote this homotopy by $H$.
We need to prove that $\sym{H[|y,x|/z]} \ct h_1$ is homotopic to $h_2$.
This homotopy can be constructed from the following one:
\[ \Gamma \mid x : B, y : A \vdash \| x.A, x \|\text{-}\fs{elim_h}(x z.\,\Id(f_1,f_2), x y.\,h_1 \ct \sym{h_2}) : \Id(H[|y,x|/z], h_1 \ct \sym{h_2}). \]

Now, let us prove that $\Fib$ is closed under $\Sigma$-types.
Let $A$ be a fibrant type over $x : B$ and let $C$ be a fibrant type over $x : B, y : A$.
We need to prove that $\Sigma_{y : A} C$ is (equivalent to) a fibrant type over $x : B$.
First, let us define a map $f : \Hom_B(\| \Sigma_{y : A} C \|, A)$ as $\lambda z.\,\| x.\,\Sigma_{y : A} C, x \|\text{-}\fs{elim}(x z. A, x p.\,\pi_1(p), z)$.
Then, for every $z : \| \Sigma_{y : A} C \|$, we define a term $g\,z : C[f\,z/y]$ as follows:
\[ \| x.\,\Sigma_{y : A} C, x \|\text{-}\fs{elim}(x z.\,C[f\,z/y], x p.\,\sym{(\| x.\,\Sigma_{y : A} C, x \|\text{-}\fs{elim_h}(x z. A, x p.\,\pi_1(p), p))}_*(\pi_2(p)), z). \]
Thus, we have a map $r = \lambda z.(f\,z,g\,z) : \Hom_B(\| x.\,\Sigma_{y : A} C, x \|, \Sigma_{y : A} C)$.
It is easy to see that $r \circ \eta_{\Sigma_{y : A} C} \sim \id$ using $\| x.\,\Sigma_{y : A} C, x \|\text{-}\fs{elim_h}$.
To prove that $\eta_{\Sigma_{y : A} C} \circ r$ is also homotopic to the identity map, we can apply $\| x.\,\Sigma_{y : A} C, x \|\text{-}\fs{elim}$.
Then it is enough to prove that $\eta_{\Sigma_{y : A} C} \circ r \circ \eta_{\Sigma_{y : A} C}$ is homotopic to $\eta_{\Sigma_{y : A} C}$ which follows from the fact that $r \circ \eta_{\Sigma_{y : A} C} \sim \id$.

\eqref{it:locally-reflective} $\implies$ \eqref{it:higher-modality}
By \rprop{fib-refl-id}, $\Fib$ is closed under identity types.
Let us define a higher modality.
Let $\Gamma \mid x : B \vdash A \ob$ be an indexed type.
Let $i : \Hom(\Sigma_{x : B} A, \| \Sigma_{x : B} A \|)$, $\pi : \Hom(\| \Sigma_{x : B} A \|, B)$ be the universal factorization of the map $\pi_1 : \Hom(\Sigma_{x : B} A, B)$.
Let $p : \Id(\pi \circ i, \pi_1)$ be the witness of the fact that $i,\pi$ is a factorization of $\pi_1$.
We define $\| x.A, b \|$ as $\Sigma_{z : \| \Sigma_{x : B} A \|} \Id(\pi\,z, b)$ and $| a, b |$ as $(i\,(b,a), \fs{hap}(p,(b,a)))$.

Now, let us define the eliminator.
Let $D$ be a fibrant indexed type over $x : B, z : \| x.A, x \|$.
Then the type $\Gamma \mid w : \| \Sigma_{x : B} A \| \vdash D[\pi\,w/x, (w,\fs{refl})] \ob$ is also fibrant since fibrations are closed under substitution.
Suppose that we have a term $\Gamma \mid x : B, y : A \vdash d : D[|y,x|/z]$.
Then we have the following commutative square:
\[ \xymatrix{ \Sigma_{x : B} A \ar[rrrr]^-{\lambda q.\,(i\,q, d[\pi_1(q)/x, \pi_2(q)/y])} \ar[d]_i  & & & & \Sigma_{w : \| \Sigma_{x : B} A \|} D[\pi\,w/x, (w,\fs{refl})] \ar@{->>}[d]^{\pi_1} \\
              \| \Sigma_{x : B} A \| \ar@{=}[rrrr] \ar@{-->}[urrrr]^h                               & & & & \| \Sigma_{x : B} A \|
            } \]
By \rlem{conn-orth} and \rlem{uni-conn}, there is a lift $h$ in this square.
If $\Gamma \mid \Delta \vdash b : B$ and $\Gamma \mid \Delta \vdash a : \| x.A, b \|$, then $\pi_1(a) : \| \Sigma_{x : B} A \|$ and $\pi_2(a) : \Id(\pi\,(\pi_1(a)), b)$.
We can define $\| x.A, b \|\text{-}\fs{elim}(x z. D, x y. d, a)$ as the transport of $\pi_2(h\,(\pi_1(a)))$ along the path witnessing the commutativity of the bottom triangle and $\pi_2(a)$.
Finally, the existence of $\| x.A, b \|\text{-}\fs{elim_h}(x z. D, x y. d, a)$ follows from the commutativity of the top triangle.
\end{proof}

\begin{remark}
If the theory has stable higher modality and $\Fib$ is closed under identity types, then $\Fib$ is stably locally reflective.
This follows from the fact that $\| - \|$ is stable under substitution.
\end{remark}

\section{Final remarks}

In this paper, we defined the notion of an indexed type theory.
Such theories allow us to work with a larger class of $\infty$-categories and, more generally, indexed $\infty$-categories using the language of type theory.
We also defined several categorical constructions in this setting and proved that they are equivalent to their type-theoretic counterparts.

We believe there is a strong relationship between models of indexed dependent types theories and indexed categories.
It was proved in \cite{kapulkin-szumilo-fin-comp} that the category of models of the dependent type theory with $\Sigma$-types, unit types, and identity types presents an $\infty$-category equivalent to the $\infty$-category of small finitely complete categories.
We conjecture that a similar result should be true for indexed type theories.
Anyway, it is necessary to find more models of indexed type theories.

Another possible line of further research is to consider other variations of indexed type theories such as indexed linear type theories.
Such theories would allow us to reason about monoidal $\infty$-categories such as $\infty$-categories of pointed spaces and spectra.

\begin{thebibliography}{}

\bibitem[Avigad \emph{et al.} 2015]{tt-fibr-cat}
Jeremy Avigad, Krzysztof Kapulkin, and Peter~LeFanu Lumsdaine, \emph{Homotopy
  limits in type theory}, Mathematical Structures in Computer Science
  \textbf{25} (2015), 1040--1070.

\bibitem[Ayala and Francis 2017]{fib-inf-cat}
David {Ayala} and John {Francis}, \emph{{Fibrations of $\infty$-categories}},
  (2017), \href {http://arxiv.org/abs/1702.02681} {\path{arXiv:1702.02681}}.

\bibitem[Cisinski 2010]{cis10b}
Denis-Charles Cisinski, \emph{Invariance de la {K-Th\'eorie} par \'equivalences
  d\'eriv\'ees}, Journal of K-theory: K-theory and its Applications to Algebra,
  Geometry, and Topology \textbf{6} (2010), 505--546.

\bibitem[Isaev 2018b]{indexed-models}
V.~{Isaev}, \emph{{Contextually indexed contextual categories}},
  (2018), \href {http://arxiv.org/abs/1809.03002} {\path{arXiv:1809.03002}}.

\bibitem[Johnstone 2002]{elephant}
Peter~T. Johnstone, \emph{Sketches of an elephant : a topos theory compendium},
  Oxford Logic Guides, Clarendon Press, Oxford, 2002.

\bibitem[Kapulkin and Szumi{\l}o 2017]{kapulkin-szumilo-fin-comp}
C.~{Kapulkin} and K.~{Szumi{\l}o}, \emph{{Internal Language of Finitely
  Complete $(\infty, 1)$-categories}},  (2017), \href
  {http://arxiv.org/abs/1709.09519} {\path{arXiv:1709.09519}}.

\bibitem[Kraus \emph{et al.} 2013]{gen-hedberg}
Nicolai Kraus, Mart{\'i}n Escard{\'o}, Thierry Coquand, and Thorsten
  Altenkirch, \emph{Generalizations of Hedberg's theorem}, Typed Lambda Calculi
  and Applications (Berlin, Heidelberg) (Masahito Hasegawa, ed.), Springer
  Berlin Heidelberg, 2013, pp.~173--188.

\bibitem[Licata \emph{et al.} 2018]{int-univ-hott}
Daniel~R. {Licata}, Ian {Orton}, Andrew~M. {Pitts}, and Bas {Spitters},
  \emph{{Internal Universes in Models of Homotopy Type Theory}},  (2018), \href
  {http://arxiv.org/abs/1801.07664} {\path{arXiv:1801.07664}}.

\bibitem[Lurie 2017]{lurie-algebra}
Jacob Lurie, \emph{Higher algebra}, (2017), Unpublished, \href{http://www.math.harvard.edu/~lurie/papers/HA.pdf} {\path{http://www.math.harvard.edu/~lurie/papers/HA.pdf}}.

\bibitem[Lurie 2009]{lurie-topos}
Jacob Lurie, \emph{Higher topos theory}, Annals of mathematics studies, Princeton
  University Press, Princeton, N.J., Oxford, 2009.

\bibitem[Nguyen \emph{et al.} 2018]{infty-gaft}
H.~K. {Nguyen}, G.~{Raptis}, and C.~{Schrade}, \emph{{Adjoint functor theorems
  for $\infty$-categories}},  (2018), \href {http://arxiv.org/abs/1803.01664}
  {\path{arXiv:1803.01664}}.

\bibitem[Par{\'e} and Schumacher 1978]{indexed-cats}
Robert Par{\'e} and Dietmar Schumacher, \emph{Abstract families and the adjoint
  functor theorems}, Indexed Categories and Their Applications (Berlin,
  Heidelberg), Springer Berlin Heidelberg, 1978, pp.~1--125.

\bibitem[Rasekh 2018]{rasekh-eht}
N.~{Rasekh}, \emph{{A Theory of Elementary Higher Toposes}},  (2018), \href
  {http://arxiv.org/abs/1805.03805} {\path{arXiv:1805.03805}}.

\bibitem[Riehl and Shulman 2017]{riehl-dhott}
E.~{Riehl} and M.~{Shulman}, \emph{A type theory for synthetic
  $\infty$-categories}, Higher Structures \textbf{1} (2017), no.~1, 147--224.

\bibitem[Rijke \emph{et al.} 2017]{modality-hott}
E.~{Rijke}, M.~{Shulman}, and B.~{Spitters}, \emph{{Modalities in homotopy type
  theory}},  (2017), \href {http://arxiv.org/abs/1706.07526}
  {\path{arXiv:1706.07526}}.

\bibitem[Shulman 2017]{split-idemp}
M.~{Shulman}, \emph{Idempotents in intensional type theory}, Logical Methods
  in Computer Science \textbf{12} (2017).

\bibitem[Shulman 2018]{cohesive-hott}
Michael {Shulman}, \emph{Brouwer's fixed-point theorem in real-cohesive
  homotopy type theory}, Mathematical Structures in Computer Science
  \textbf{28} (2018), no.~6, 856–941.

\bibitem[Uemura 2016]{fib-fib-cats}
Taichi {Uemura}, \emph{{Fibred Fibration Categories}},  (2016), \href
  {http://arxiv.org/abs/1602.08206} {\path{arXiv:1602.08206}}.

\bibitem[The {Univalent Foundations Program} 2013]{hottbook}
The {Univalent Foundations Program}, \emph{Homotopy type theory: Univalent
  foundations of mathematics}, \url{https://homotopytypetheory.org/book},
  Institute for Advanced Study, 2013.

\end{thebibliography}

\end{document}
