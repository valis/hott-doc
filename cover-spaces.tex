\documentclass[reqno]{amsart}

\usepackage{amssymb}
\usepackage{hyperref}
\usepackage{ifthen}
\usepackage{xargs}
\usepackage{mathtools}

\hypersetup{colorlinks=true,linkcolor=blue}

\newcommand{\axlabel}[1]{(#1) \phantomsection \label{ax:#1}}
\newcommand{\axitem}[1]{\phantomsection \label{ax:#1}}
\newcommand{\axref}[1]{(\hyperref[ax:#1]{#1})}

\newcommand{\newref}[4][]{
\ifthenelse{\equal{#1}{}}{\newtheorem{h#2}[hthm]{#4}}{\newtheorem{h#2}{#4}[#1]}
\expandafter\newcommand\csname r#2\endcsname[1]{#3~\ref{#2:##1}}
\expandafter\newcommand\csname R#2\endcsname[1]{#4~\ref{#2:##1}}
\expandafter\newcommand\csname n#2\endcsname[1]{\ref{#2:##1}}
\newenvironmentx{#2}[2][1=,2=]{
\ifthenelse{\equal{##2}{}}{\begin{h#2}}{\begin{h#2}[##2]}
\ifthenelse{\equal{##1}{}}{}{\label{#2:##1}}
}{\end{h#2}}
}

\newref[section]{thm}{Theorem}{Theorem}
\newref{lem}{Lemma}{Lemma}
\newref{prop}{Proposition}{Proposition}
\newref{cor}{Corollary}{Corollary}
\newref{cond}{Condition}{Condition}

\theoremstyle{definition}
\newref{defn}{Definition}{Definition}
\newref{example}{Example}{Example}

\theoremstyle{remark}
\newref{remark}{Remark}{Remark}

\numberwithin{figure}{section}

\newcommand{\overlap}[2]{#1 \between #2}
\newcommand{\rb}{\prec}
\newcommand{\cat}[1]{\mathbf{#1}}

\DeclarePairedDelimiter\abs{\lvert}{\rvert}

\makeatletter
\let\oldabs\abs
\def\abs{\@ifstar{\oldabs}{\oldabs*}}
\makeatother

\begin{document}

\title{Completion of Spaces in Constructive Mathematics}

\author{Valery Isaev}

\begin{abstract}
TODO
\end{abstract}

\maketitle

\section{Introduction}

TODO

\section{Topological properties}

In this section, we define cover spaces and and study their topological properties.
First, we need to give a few basic definitions:
\begin{itemize}
\item Two subsets $U,V$ of a set $X$ \emph{intersect}, written $\overlap{U}{V}$, if there exists an element of $X$ which belongs to both subsets.
\item A set $C$ of subsets of a set $X$ is called a \emph{cover} if $\bigcup C = X$. If $C$ is a cover of $X$, we will also say that $C$ \emph{covers} $X$ and that $X$ \emph{is covered by} $C$.
\item A cover $C$ \emph{refines} a cover $D$ if every element of $C$ is a subset of some element of $D$.
\item If $U$ and $V$ are subsets of a topological space $X$, we will say that $V$ is \emph{rather below} $U$, written $V \rb_X U$ or, for short, $V \rb U$, if $X$ is covered by opens $W$ satisfying the implication $\overlap{W}{V} \implies W \subseteq U$.
\item A topological space is \emph{regular} if every open $U$ is covered by opens which are rather below $U$.
\end{itemize}

Now, we are ready to define cover spaces:
\begin{defn}
A \emph{cover space} is a set $X$ together with a set $\mathcal{C}_X$ of covers of $X$, which satisfy the following conditions:
\begin{itemize}
\item[(CT)] \axitem{CT} $\{ X \} \in \mathcal{C}_X$
\item[(CE)] \axitem{CE} If $C \in \mathcal{C}_X$ and $C$ refines $D$, then $D \in \mathcal{C}_X$
\item[(CG)] \axitem{CG} If $C \in \mathcal{C}_X$ and $\{ D_U \}_{U \in C}$ is a collection of covers such that $D_U \in \mathcal{C}_X$ for all $U \in C$,
            then $\{ U \cap V \mid U \in C, V \in D_U \} \in \mathcal{C}_X$.
\item[(CR)] \axitem{CR} If $C \in \mathcal{C}_X$, then $\{ V \subseteq X \mid \exists U \in C, V \rb_X U \} \in \mathcal{C}_X$,
            where $V \rb_X U$ or, for short, $V \rb U$ means $\{ W \subseteq X \mid \overlap{W}{V} \implies W \subseteq U \} \in \mathcal{C}_X$.
\end{itemize}
Elements of $\mathcal{C}_X$ are called \emph{Cauchy covers} of $X$.
\end{defn}

\begin{remark}
Under the assumption of the law of excluded middle, the relation $V \rb U$ can be equivalently expressed as $\{ X \backslash V, U \} \in \mathcal{C}_X$.
Indeed, $\{ X \backslash V, U \}$ is always a subset of $\{ W \mid \overlap{W}{V} \implies W \subseteq U \}$, and, assuming LEM, the latter cover refines the former.
We will discuss the condition $\{ X \backslash V, U \} \in \mathcal{C}_X$ in more detail in Section~\ref{sec:completion}.
\end{remark}

The relation $\rb$ has usual properties of the ``rather below'' relation:

\begin{prop}[rb-props]
The following is true for subsets $U,V,U',V'$ of a cover space $X$:
\begin{enumerate}
\item \label{rb:sub} If $V \rb U$, then $V \subseteq U$.
\item If $V' \subseteq V \rb U \subseteq U'$, then $V' \rb U'$.
\item \label{rb:meet} If $V \rb U$ and $V' \rb U'$, then $V \cap V' \rb U \cap U'$.
\item $V \rb X$.
\item $\varnothing \rb U$.
\end{enumerate}
\end{prop}
\begin{proof}
Item \eqref{rb:sub} follows from the fact that $\{ W \mid \overlap{W}{V} \implies W \subseteq U \}$ is a cover.
Item \eqref{rb:meet} follows from \axref{CG}.
The rest is obvious.
\end{proof}

\begin{example}
Given a set $X$, the \emph{indiscrete} cover space structure on it consists of all covers that contain the whole set: $\mathcal{C}_X = \{ C \mid X \in C \}$.
This is the minimal possible set of covers that makes $X$ into a cover space.
\end{example}

\begin{example}
Given a set $X$, the \emph{discrete} cover space structure on it consists of all covers.
This is the maximal possible set of covers that makes $X$ into a cover space.
\end{example}

\begin{example}
More generally, if $X$ is a regular topological space, we can define a cover to be Cauchy if it contains a neighborhood of every point.
Then the rather below relations defined for $X$ as a topological space and as a cover space coincide.
Thus, the regularity condition on $X$ implies \axref{CR} and the other axioms of cover spaces are obvious.
We will denote this cover space by $T(X)$.
\end{example}

We can define various basic topological concepts for cover spaces:

\begin{defn}[cover-topology]
Let $X$ be a cover space.
\begin{itemize}
\item A \emph{neighborhood} of a point $x \in X$ is a subset $U$ of $X$ such that $\{ x \} \rb U$.
\item A subset $U$ of $X$ is \emph{open} if $U$ is a neighborhood of each $x \in U$.
\item A \emph{limit point} of a subset $U$ of $X$ is a point $x \in X$ such that every neighborhood of $x$ intersects $U$.
\item A subset of $X$ is \emph{closed} if it contains all of its limit points.
\item The \emph{closure} $\overline{U}$ of a subset $U$ of $X$ is the set of its limit points.
\item A subset $D$ of $X$ is \emph{dense} if all points of $X$ are limit points of $D$.
\end{itemize}
\end{defn}

\rprop{rb-props} implies that the set of opens defined above constitutes a topology on every cover space $X$.
We will denote this topological space by $S(X)$.

\begin{example}
Let $X$ be an indiscrete cover space.
A subset $U$ of $X$ is a neighborhood of a point $x \in X$ if and only if $U = X$.
It follows that a subset $U$ of $X$ is open if and only if it equals to $X$ whenever it is inhabited.
Thus, $S(X)$ is an indiscrete topological space.
\end{example}

\begin{example}
Let $X$ be a discrete cover space.
Then a subset $U$ of $X$ is a neighborhood of a point $x \in X$ if and only if $x \in U$.
It follows that all subsets of $X$ are open, so $S(X)$ is a discrete topological space.
\end{example}

\begin{example}
Let $X$ be a regular topological space.
Then regularity of $X$ implies that a subset $U$ of $X$ is a neighborhood of a point $x \in X$ in topological sense if and only if $\{ x \} \rb_{T(X)} U$.
It follows that $S(T(X))$ coincides with $X$.
\end{example}

The following lemma is useful for establishing the relationship between a cover space $X$ and corresponding topological space $S(X)$:

\begin{lem}[int-char]
Let $U$ be a subset of a cover space $X$.
Then the interior of $U$ is equal to $\{ x \mid \{ x \} \rb_X U \}$.
In particular, if $V \rb_X U$, then $V \subseteq \mathrm{int}(U)$.
\end{lem}
\begin{proof}
If $x \in \mathrm{int}(U)$, then $\{ x \} \rb_X \mathrm{int}(U) \subseteq U$, so one direction is clear.
Conversely, it is enough to show that $U' = \{ x \mid \{ x \} \rb_X U \}$ is open.
If $x \in U'$, then $\{ W' \mid \exists W, W' \rb_X W, x \in W \implies W \subseteq U \}$ is a Cauchy cover.
Since this cover refines $\{ W' \mid x \in W' \implies W' \subseteq U' \}$, we get that $\{ x \} \rb_X U'$.
Thus, $U'$ is open.
\end{proof}

The following proposition shows that every Cauchy cover is refined by an open Cauchy cover.
This means we can work exclusively with open covers, but there is no benefit to doing that, so we will continue to use arbitrary covers.

\begin{prop}[cover-int]
If $C$ is a Cauchy cover, then $\{ \mathrm{int}(U) \mid U \in C \}$ is also a Cauchy cover.
\end{prop}
\begin{proof}
If $C$ is Cauchy, then $\{ V \mid \exists U \in C, V \rb U \}$ is also Cauchy and refines $\{ \mathrm{int}(U) \mid U \in C\}$ by \rlem{int-char}.
Thus, the latter cover is also Cauchy.
\end{proof}

Now, we will show how various topological properties of a cover space $X$ are related to corresponding properties of $S(X)$:

\begin{prop}[top-neighborhood]
If $X$ is a cover space, then the notions of a neighborhood, closed subsets, closure, and dense subsets for $X$ coincide with eponymous properties of $S(X)$.
\end{prop}
\begin{proof}
It is enough to show that $U \subseteq X$ is a neighborhood of $x \in X$ in $S(X)$ if and only if $\{ x \} \rb_X U$.
If $V$ is an open set such that $x \in V \subseteq U$, then $\{ x \} \rb_X V \subseteq U$.
Conversely, if $\{ x \} \rb_X U$, then $x \in \mathrm{int}(U) \subseteq U$ by \rlem{int-char}.
\end{proof}

Now, we will show that $S(X)$ is always a regular topological space.
To do this, we first prove a useful lemma:

\begin{lem}[rb-point]
Let $X$ be a cover space.
If $U$ is a neighborhood of a point $x$, then there exists a neighborhood $V$ of $x$ such that $V \rb U$.
\end{lem}
\begin{proof}
If $x \rb U$, then $\{ V' \mid V' \rb V \rb W, x \in W \implies W \subseteq U \}$ covers $X$.
Thus, there exist subsets $V'$, $V$, and $W$ such that $x \in V' \rb V$ and $V \rb W \subseteq U$.
So, $V$ is the required neighborhood.
\end{proof}

\begin{prop}[top-regular]
For every cover space $X$, the topological space $S(X)$ is regular.
\end{prop}
\begin{proof}
First, note that if $V \rb_X U$, then $\{ \mathrm{int}(W) \mid x \in W \implies W \subseteq U \}$ covers $X$ by \rprop{cover-int},
so, for every $x \in X$, we have a set $W$ such that $x \in \mathrm{int}(W)$ and $\overlap{\mathrm{int}(W)}{V} \implies \overlap{W}{V} \implies \mathrm{int}(W) \subseteq W \subseteq U$.
It follows that $V \rb_{S(X)} U$.

To see that $S(X)$ is regular, consider an open $U$.
Then, for every $x \in U$, there exists a neighborhood $V$ of $x$ such that $V \rb_{S(X)} U$ by \rlem{rb-point} and previous observation.
By \rprop{top-neighborhood}, there is an open neighborhood $V'$ of $x$ such that $V' \rb_{S(X)} U$.
Thus, $S(X)$ is regular.
\end{proof}

The following proposition shows that $V \rb U$ implies the usual definition of the rather below relation in terms of limit points:

\begin{prop}[rb-closure]
Let $U,V$ be subsets of a cover space $X$ such that $V \rb U$.
Then $U$ is a neighborhood of every limit point of $V$.
In particular, the closure of $V$ is a subset of $U$.
Moreover, $\overline{V} \rb \mathrm{int}(U)$.
\end{prop}
\begin{proof}
It is enough to prove the last claim.
By \rprop{cover-int}, the cover $\{ \mathrm{int}(W') \mid \exists W, W' \rb W, \overlap{W}{V} \implies W \subseteq U \}$ is Cauchy.
Since this cover refines $\{ W \mid \overlap{W}{\overline{V}} \implies W \subseteq \mathrm{int}(U) \}$, we get that $\overline{V} \rb \mathrm{int}(U)$.
\end{proof}

Many examples of cover spaces come from uniform spaces.
Recall that a uniform space is a set $X$ together with a set of covers $\mathcal{U}_X$ such that the following conditions hold:
\begin{itemize}
\item[(UT)] $\{ X \} \in \mathcal{U}_X$
\item[(UE)] If $C \in \mathcal{U}_X$ and $C$ refines $D$, then $D \in \mathcal{U}_X$
\item[(UI)] \axitem{UI} If $C,D \in \mathcal{U}_X$, then $\{ U \cap V \mid U \in C, V \in D \} \in \mathcal{U}_X$.
\item[(UU)] \axitem{UU} If $C \in \mathcal{U}_X$, then there exists $D \in \mathcal{U}_X$ such that, for every $V \in D$,
there exists $U \in C$ such that every $W \in D$ satisfies the implication $\overlap{W}{V} \implies W \subseteq U$.
\end{itemize}
Elements of $\mathcal{U}_X$ are called \emph{uniform covers} of $X$.

\begin{example}
Every metric space structure on a set $X$ induces a uniform structure on $X$.
A cover $C$ is uniform in this structure if there exists $\varepsilon > 0$ such that every open ball of radius $\varepsilon$ is a subset of some set in $C$.
\end{example}

We will show that every uniform space structure determines a cover space structure.
To do this, it is useful to introduce an auxiliary notion of a precover space:

\begin{defn}
A \emph{precover space} is a set $X$ together with a set $\mathcal{C}_X$ of covers of $X$, which satisfy conditions \axref{CT}, \axref{CE}, and \axref{CG}.
\end{defn}

Thus, a cover space is exactly a precover space that satisfies the regularity condition \axref{CR}.
We will also need the notion of a subbase for a precover structure:

\begin{defn}
Given a set $X$, a \emph{subbase} for a precover structure on $X$ is any set of covers of $X$.
\end{defn}

Given a subbase $\mathcal{B}_X$, we can take its closure under \axref{CT}, \axref{CE}, and \axref{CG}.
We will denote this closure by $\overline{\mathcal{B}_X}$.
Then the following lemma applied to the filter of sets containing a given point implies that all sets in this closure are covers:

\begin{lem}[closure-filter]
Let $\mathcal{B}_X$ be a subbase and $F$ be a filter on $X$.
If $F$ intersects with every set in $\mathcal{B}_X$, then it also intersects with every set in $\overline{\mathcal{B}_X}$.
\end{lem}
\begin{proof}
The proof is an easy induction on the generation of $\overline{\mathcal{B}_X}$.
\end{proof}

Thus, $\overline{\mathcal{B}_X}$ is the minimal precover space structure on $X$ that contains $\mathcal{B}_X$.
In general, $(X,\overline{\mathcal{B}_X})$ is not a cover space, but it is so if $\mathcal{B}_X$ satisfies an additional regularity hypothesis.
We will say that subbase $\mathcal{B}_X$ is \emph{regular} if, for every $C \in \mathcal{B}_X$, the set $\{ V \mid \exists U \in C, V \rb_{(X,\overline{\mathcal{B}_X})} U \}$ belongs to $\overline{\mathcal{B}_X}$.

\begin{prop}[subbase-regular]
If $\mathcal{B}_X$ is a regular subbase, then $(X,\overline{\mathcal{B}_X})$ is a cover space.
\end{prop}
\begin{proof}
We show by induction on generation of $\overline{\mathcal{B}_X}$ that, for every $C \in \overline{\mathcal{B}_X}$, the cover $\{ W' \mid \exists W \in C, W' \rb W \}$ also belongs to $\overline{\mathcal{B}_X}$.
Every element of $\mathcal{B}_X$ satisfies this condition by the regularity assumption.
The only non-trivial remaining case is \axref{CG}.
Suppose that $C = \{ U \cap V \mid U \in D, V \in E_U \} \in \mathcal{C}_X$ for some $D \in \overline{\mathcal{B}_X}$ and $\{ E_U \in \overline{\mathcal{B}_X} \}_{U \in D}$.
Then, by induction hypothesis, we have $D' = \{ U' \mid \exists U \in D, U' \rb U \} \in \overline{\mathcal{B}_X}$ and $E'_{U'} = \{ V' \mid \exists U \in D, V \in E_U, V' \rb V, U' \rb U \} \in \overline{\mathcal{B}_X}$ for every $U' \in D'$.
Since $\{ W' \mid \exists W \in C, W' \rb W \}$ is refined by $\{ U' \cap V' \mid U' \in D', V' \in E'_{U'} \}$, this concludes the proof.
\end{proof}

Clearly, \axref{UU} together with \axref{UE} implies that $\mathcal{U}_X$ is a regular subbase for every uniform space structure $\mathcal{U}_X$.
Thus, every uniform space $(X,\mathcal{U}_X)$ determines a cover space $(X,\overline{\mathcal{U}_X})$.
In particular, every metric spaces determines a cover space.
This is our main source of examples of cover spaces.

\section{Categorical properties}

In this section, we define the category of cover spaces and prove its various properties.
It will also be convenient to define the category of precover spaces.
Thus, we begin with the definition of morphisms of precover spaces:

\begin{defn}
A \emph{cover map} between precover spaces $X$ and $Y$ is a function $f : X \to Y$ such that, for every Cauchy cover $D$ on $Y$, the set $\{ f^{-1}(V) \mid V \in D \}$ is a Cauchy cover on $X$.
\end{defn}

Clearly, the identity function is a cover map and cover maps are closed under composition.
Thus, we get a category of precover spaces, which we denote by $\cat{Precov}$.
The category of cover spaces $\cat{Cov}$ is the full subcategory of $\cat{Precov}$ on cover spaces.
Now, we will show that various constructions we defined before are actually functors.
First, we need a lemma:

\begin{lem}[cover-map-rb]
If $f : X \to Y$ is a cover map and $U$ and $V$ are subsets of $Y$ such that $V \rb_Y U$, then $f^{-1}(V) \rb_X f^{-1}(U)$.
\end{lem}
\begin{proof}
This follows from the fact that $\{ W \mid \overlap{W}{f^{-1}(V)} \implies W \subseteq f^{-1}(U) \}$ is refined by $\{ f^{-1}(W) \mid \overlap{W}{V} \implies W \subseteq U \}$.
\end{proof}

Now, it is easy to see that the function $S : \cat{Cov} \to \cat{Top}_\mathrm{reg}$ from cover spaces to regular topological spaces is functorial.
Indeed, if $f : X \to Y$ is a cover map, $U$ is an open subset of $U$, and $x \in f^{-1}(U)$, then $\{ x \} \subseteq f^{-1}(\{ f(x) \}) \rb f^{-1}(U)$.
So, $f^{-1}(U)$ is also open.
Similarly, the function $T : \cat{Top}_\mathrm{reg} \to \cat{Cov}$ is a functor.
Clearly, both of these functors are faithfull.
Let us show that $T$ is also full.
Indeed, since $S(T(X)) = X$ for every regular topological space $X$, if $f : T(X) \to T(Y)$ is a cover map, then $g = S(f) : X \to Y$ is a continuous map such that $T(g) = f$.

The functor $T$ is also a left adjoint of $S$.
The unit and the counit of the adjunction are both given by the identity function.
The fact that the counit $T(S(X)) \to X$ is a cover map follows from the fact that any Cauchy cover contains a neighborhood of every point.
Thus, $\cat{Top}_\mathrm{reg}$ is equivalent to a full coreflective subcategory of $\cat{Cov}$.

Now, we will show that the category of precover spaces is topological. % TODO: Give a reference for topological categories.
To do this, we will show that the set of precover space structures on a given set $X$ forms a complete lattice under inclusion.
Indeed, if $\{ \mathcal{C}_i \}_{i \in I}$ is a set of precover space structures on $X$, then $\overline{\bigcup_{i \in I} \mathcal{C}_i}$ is the join of these structures.
Also, we need to define the transferred precover space structure.
Given a set $X$, a precover space $Y$, and a function $f : X \to Y$, we can define the smallest precover space structure on $X$ for which this function is a cover map.
A cover of $X$ is Cauchy in this structure if and only if $\{ V \mid \exists U \in C, f^{-1}(V) \subseteq U \}$ is a Cauchy cover of $Y$.
These constructions together imply that $\cat{Precov}$ is a topological category.

Cover spaces are closed under joins in the lattice of precover space structures on a set $X$.
This follows from \rprop{subbase-regular}.
Also, if $Y$ is a cover space and $f : X \to Y$ is a function, then the transferred precover structure on $X$ is actually a cover structure.
It follows that the category of cover spaces is reflective in $\cat{Precov}$:

\begin{prop}[regular-reflective]
The category of cover spaces is reflective in $\cat{Precov}$.
The reflection $X \to R(X)$ is the identity function on $X$.
\end{prop}
\begin{proof}
Given a precover space $X$, we can define its reflection $R(X)$ as the join of all cover space structures on $X$ contained in $\mathcal{C}_X$.
Then the identity function $X \to R(X)$ is a cover map and has the universal property of the reflection.
Indeed, if $Y$ is a cover space and $f : X \to Y$ is a cover map, then the transferred cover structure is a subset of $\mathcal{C}_{R(X)}$.
It follows that $f$ is a cover map from $R(X)$ to $Y$.
\end{proof}

\begin{cor}
The category of cover spaces is topological.
\end{cor}

We finish this section with a discussion of embeddings.
A cover map $f : X \to Y$ is called an \emph{embedding} if, for every $C \in \mathcal{C}_X$,
the set $\{ V \mid \exists U \in C, f^{-1}(V) \subseteq U \}$ is a Cauchy cover in $Y$.
Note that a cover map $f : X \to Y$ is an embedding if and only if the precover space structure on $X$ is the transferred precover space structure.

In general, an embedding might not be injective, but we will show in \rprop{embedding-injective} that this is true if the domain satisfies a separation condition.
For injective embeddings, we have the following characterization:

\begin{prop}
A cover map is a regular monomorphism in $\cat{Precov}$ if and only if it is an injective embedding.
A cover map between cover spaces is a regular monomorphism in $\cat{Cov}$ if and only if it is an injective embedding.
\end{prop}
\begin{proof}
If $f : X \to Y$ is a regular monomorphism in $\cat{Precov}$, then it is injective and it is straightforward to check that the precover space structure on $X$ is transferred.
Coversely, let $f : X \to Y$ be an injective embedding.
Then $f : X \to Y$ is an equalizer of some functions $g,h : Y \to Z$ in $\cat{Set}$.
If we put the indiscrete structure on $Z$, then $g$ and $h$ are cover maps and it is easy to see that $f$ is their equalizer.
The same proof works for cover spaces.
\end{proof}

We will need later the fact that embeddings are closed under products.
For this, we will prove the following lemma first:

\begin{lem}
Let $\mathcal{B}_X$ be a subbase for a precover space structure on $X$ and let $Y$ be a precover space.
Let $f : X \to Y$ be a cover map such that, for every $C \in \mathcal{B}_X$, the set $\{ V \mid \exists U \in C, f^{-1}(V) \subseteq U \}$ is a Cauchy cover of $Y$.
Then $f$ is an embedding.
\end{lem}
\begin{proof}
We will show by induction on the generation of a cover $E \in \overline{\mathcal{B}_X}$ that $\{ U' \mid \exists U \in E, f^{-1}(U') \subseteq U \}$ is a Cauchy cover of $Y$.
The base case holds by assumption.
The cases \axref{CT} and \axref{CE} are obvious.
Let us consider \axref{CG}.
Let $C$ be a Cauchy cover of $X$ and $\{ D_U \}_{U \in C}$ be a collection of Cauchy covers of $X$ such that $E = \{ U \cap V \mid U \in C, V \in D_U \}$.
By induction hypothesis, $C' = \{ U' \mid \exists U \in C, f^{-1}(U') \subseteq U \}$ is a Cauchy cover of $Y$.
Let $D'_{U'} = \{ V' \mid \exists U \in C, \exists V \in D_U, f^{-1}(U') \subseteq U, f^{-1}(V') \subseteq V \}$.
Then $D'_{U'}$ is a Cauchy cover of $Y$ for every $U' \in C'$.
Since $\{ U' \cap V' \mid U' \in C', V' \in D_{U'} \}$ refines $\{ W' \mid \exists W \in E, f^{-1}(W') \subseteq W \}$, this concludes the proof.
\end{proof}

\begin{lem}[prod-embedding]
If $f : X \to Y$ and $g : X' \to Y'$ are embeddings, then so is $f \times g : X \times X' \to Y \times Y'$.
\end{lem}
\begin{proof}
The precover space structure on $X \times X'$ is generted by covers of the form $\{ \pi_1^{-1}(U) \mid U \in C \}$ and $\{ \pi_2^{-1}(U') \mid U' \in C' \}$ for Cauchy covers $C$ on $X$ and $C'$ on $X'$.
Since $f$ and $g$ are embeddings, sets $\{ \pi_1^{-1}(V) \mid \exists U \in C, f^{-1}(V) \subseteq U \}$ and $\{ \pi_2^{-1}(V') \mid \exists U' \in C', f^{-1}(V') \subseteq U' \}$ are Cauchy covers on $Y \times Y'$.
Since these covers refine $\{ V \mid \exists U \in C, (f \times g)^{-1}(V) \subseteq \pi_1^{-1}(U) \}$ and $\{ V' \mid \exists U' \in C', (f \times g)^{-1}(V') \subseteq \pi_2^{-1}(U') \}$, the previous lemma implies that $f \times g$ is an embedding.
\end{proof}

\section{Cauchy filters}

In the next section, we will define complete cover spaces.
To do this, we need to introduce the notion of a Cauchy filter, which we define and study in this section.
We will also show that Cauchy filters on $\mathbb{Q}$ correspond to Dedekind real numbers.

\begin{defn}
A \emph{filter} on a set $X$ is an upward closed set of subsets of $X$ closed under finite intersections.
A \emph{proper filter} is a filter consisting of inhabited sets.
A \emph{Cauchy filter} on a cover space $X$ is a proper filter that intersects with every Cauchy cover.
\end{defn}

\begin{example}
For every point $x$ of a cover space $X$, the set of its neighborhoods forms a Cauchy filter on $X$.
This filter will be denoted by $x^\wedge$.
\end{example}

If a cover space is equipped with a subbase, then it is easier to check when a proper filter is Cauchy:

\begin{prop}[cauchy-filter]
If $\mathcal{B}_X$ is a subbase for a cover space $X$, then a proper filter $F$ is Cauchy if and only if it intersects with every cover from $\mathcal{B}_X$.
\end{prop}
\begin{proof}
The ``only if'' direction is obvious.
The other direction is an easy induction on the consruction of a Cauchy cover.
\end{proof}

\begin{cor}[metric-cauchy-filter]
If $X$ is a metric space, then a proper filter is Cauchy if and only if it contains an open ball of size $\varepsilon$ for every $\varepsilon > 0$.
\end{cor}

The notion of a Cauchy filter makes every cover space into a Cauchy space. % TODO: Give references
We will not use this abstraction and instead give the necessary constructions and definitions directly for cover spaces.

Given a function $f : X \to Y$ and a filter $F$ on $X$, we will write $f(F)$ for the filter $\{ V \mid f^{-1}(V) \in F \}$ on $Y$.
We will say that $f$ is a \emph{Cauchy map} if it preserves Cauchy filters.
Clearly, every cover map is Cauchy.
It is also easy to see that Cauchy maps are continuous:

\begin{prop}[cauchy-continuous]
Every Cauchy map $f : X \to Y$ is a continuous function.
\end{prop}
\begin{proof}
Let $U$ be an open subset of $Y$ and let $x \in X$ be a point such that $f(x) \in U$.
Since $f(x^\wedge)$ is a Cauchy filter and $\{ f(x) \} \rb U$, there exists a set $V \in f(x^\wedge)$ such that the implication $f(x) \in V \implies V \subseteq U$ holds.
The first condition means that $f^{-1}(V)$ is a neighborhood of $x$ and the second condition implies that $f^{-1}(V) \subseteq f^{-1}(U)$.
Thus, $f^{-1}(U)$ is also a neighborhood of $x$.
\end{proof}

There is an equivalence relation of Cauchy filters.
Two Cauchy fitlers are \emph{equivalent} if every Cauchy cover contains an element from both filters.
Note that Cauchy filters are equivalent if and only if their intersection is a Cauchy filter.
In particular, if one Cauchy filter is a subset of another one, they are equivalent.
Now, we will prove a crucial lemma about Cauchy filters:

\begin{lem}[cauchy-filter-rb]
Let $X$ be a cover space, $F$ and $G$ be equivalent Cauchy filters on $X$, and $U$ and $V$ be subsets of $X$ such that $V \rb U$.
If $V$ belongs to $F$, then $U$ belongs to $G$.
\end{lem}
\begin{proof}
Since $V \rb U$, there is a set $W \in F \cap G$ such that impliction $\overlap{W}{V} \implies W \subseteq U$ holds.
Since $W,V \in F$, their intersection also belongs to $F$ and, hence, it is inhabited.
It follows that $W \subseteq U$.
Thus, $U \in G$
\end{proof}

\begin{prop}
The equivalence of Cauchy filters is indeed an equivalence relation.
\end{prop}
\begin{proof}
The only non-trivial part is transitivity.
Let $F,G,H$ be Cauchy filters such that $F$ is equivalent to $G$ and $G$ is equivalent to $H$.
Let $C$ be a Cauchy cover.
By \axref{CR}, there are sets $U$ and $V$ such that $V \rb U$, $U \in C$, and $V \in F \cap G$.
By \rlem{cauchy-filter-rb}, $U$ belongs to both $F$ and $H$.
\end{proof}

A Cauchy filter $F$ is called \emph{regular} if, for every $U \in F$, there is a set $V \in F$ such that $V \rb U$.
Regular Cauchy filters can be used as representatives of their equivalence classes.

\begin{example}
For every point $x \in X$, the neighborhood filter $x^\wedge$ is regular.
This follows from \rlem{rb-point}.
\end{example}

The following propositions show that every Cauchy filter is equivalent to a unique regular one:

\begin{prop}[regular-minimal]
Let $F$ and $G$ be equivalent Cauchy filters.
If $F$ is regular, then $F \subseteq G$.
\end{prop}
\begin{proof}
Let $U$ be an element of $F$.
Since $F$ is regular, there is a set $V \in F$ such that $V \rb U$.
By \rlem{cauchy-filter-rb}, $U$ belongs to $G$.
\end{proof}

\begin{prop}
Let $F$ be a Cauchy filter.
The intersection of all Cauchy subfilters of $F$ is a regular Cauchy filter.
It is a unique regular filter equivalent to $F$.
\end{prop}
\begin{proof}
We will denote the intersection of all Cauchy subfilters of $F$ by $M$.
It is clear that the intersection of a non-empty set of proper filters is a proper filter.
To see that it is a Cauchy filter, let $C$ be a Cauchy cover.
Then by \axref{CR}, there are sets $V \in F$ and $U \in C$ such that $V \rb U$.
By \rlem{cauchy-filter-rb}, every Cauchy filter which is equivalent to $F$ contains $U$.
In particular, it belongs to the intersection of all Cauchy subfilters of $F$.
This shows that $M$ is Cauchy.

Let us prove that $M$ is regular.
Let $G = \{ U \mid \exists W \in F, \exists V, W \rb V, V \rb U \}$.
\rprop{rb-props} implies that $G$ is a proper filter and \axref{CR} implies that it is Cauchy.
Since $G \subseteq F$, we get that $M \subseteq G$.
Thus, if $U$ is a set in $M$, then there exist sets $W \in F$ and $V$ such that $W \rb V$ and $V \rb U$.
Since $F$ and $M$ are equivalent, \rlem{cauchy-filter-rb} implies that $V \in M$, which shows that $M$ is regular.

Since $M$ is a subset of $F$, it is equivalent to it.
The uniqueness follows from \rprop{regular-minimal}.
\end{proof}

Now, we will construct a bijection between regular Cauchy filters and Dedekind reals.
Recall that Dedekind reals are usually defined as Dedekind cuts.
A \emph{Dedekind cut} is a pair $(L,U)$ of subsets of $\mathbb{Q}$ satisfying the following conditions:
\begin{itemize}
\item[(DI)] \axitem{DI} Both $L$ and $U$ are inhabited.
\item[(DL)] \axitem{DL} $a \in L$ if and only if there exists $b > a$ such that $b \in L$.
\item[(DU)] \axitem{DU} $a \in U$ if and only if there exists $b < a$ such that $b \in U$.
\item[(DD)] \axitem{DD} $L$ and $U$ do not intersect.
\item[(DS)] \axitem{DS} If $a < b$, then either $a \in L$ or $b \in U$.
\end{itemize}

Dedekind reals can be alternatively represented by filters.
A proper filter $F$ on $\mathbb{Q}$ will be called a \emph{Dedekind filter} if it satisfies the following conditions:
\begin{itemize}
\item[(DE)] \axitem{DE} For every rational $\varepsilon > 0$, there exists $a \in \mathbb{Q}$ such that the open interval $(a, a + \varepsilon)$ belongs to $F$.
\item[(DR)] \axitem{DR} If $F$ contains an open interval $(a,d)$, then $F$ also contains an open interval $(b,c)$ such that $a < b < c < d$.
\item[(DT)] \axitem{DT} For every $U \in F$, there exists an open interval $(a,b) \in F$ such that $(a,b) \subseteq U$.
\end{itemize}

\begin{prop}[dedekind-cuts-filters]
There is a bijection between the set of Dedekind cuts and the set of Dedekind filters.
\end{prop}
\begin{proof}
If $(L,U)$ is a Dedekind real, then we define $F(L,U)$ as the set of those $V$ for which there exist $a \in L$ and $b \in U$ such that the open interval $(a,b)$ is a subset of $V$.
Clearly, it is a filter that satisfies \axref{DT}, \axref{DI} implies that it is proper, and \axref{DL} and \axref{DU} imply \axref{DR}.
Let us prove \axref{DE}.
By \axref{DI}, \axref{DD}, and \axref{DL}, there exist $a \in L$ and $b \in U$ such that $a < b$.
Let $a' = a + \frac{b - a}{3}$ and $b' = b - \frac{b - a}{3}$.
Then \axref{DS} implies that either $a' \in L$ or $b' \in U$.
So, we reduced the open interval $(a,b) \in F(L,U)$ by a factor of $1.5$.
Thus, we can get an arbitrary small open interval in a finite number of steps, which proves \axref{DE}.

Now, let $F$ be a Dedekind filter.
If we define $L_F = \{ a \in \mathbb{Q} \mid \exists b \in \mathbb{Q}, (a,b) \in F \}$ and $U_F = \{ b \in \mathbb{Q} \mid \exists a \in \mathbb{Q}, (a,b) \in F \}$,
then \axref{DE} implies \axref{DI} and \axref{DS}, \axref{DR} implies \axref{DL} and \axref{DU}, and properness of $F$ implies \axref{DD}.

It is easy to see that $L_{F(L,U)} = L$ and $U_{F(L,U)} = U$ and the fact that $F(L_F,U_F) = F$ follows from \axref{DT}.
\end{proof}

The cover space of rational numbers $\mathbb{Q}$ is defined as the one induced by the usual Euclidean metric space structure on $\mathbb{Q}$.
Now, we can show that regular Cauchy filters on this cover space correspond to Dedekind filters:

\begin{prop}[dedekind-cauchy]
A filter on $\mathbb{Q}$ is a Dedekind filter if and only if it is a regular Cauchy filter.
\end{prop}
\begin{proof}
By \rcor{metric-cauchy-filter}, a proper filter on $\mathbb{Q}$ satisfies \axref{DE} if and only if it is Cauchy.
Thus, we need to show that a Cauchy filter is regular if and only if it satisfies \axref{DR} and \axref{DT}.
If a Cauchy filter $F$ satisfies \axref{DR} and \axref{DT}, then, for every $U \in F$, there exist $a < b < c < d$ such that $(b,c) \subseteq (a,d) \subseteq U$ and $(b,c) \in F$.
Since $(b,c) \rb (a,d)$, this implies that $F$ is regular.

Conversely, suppose that $F$ is regular.
Then $F' = \{ U \in F \mid \exists (a,b) \in F, (a,b) \subseteq U \}$ is a Cauchy filter by \rcor{metric-cauchy-filter}.
Since $F' \subseteq F$ and $F$ is regular, \rprop{regular-minimal} implies that $F = F'$.
Thus, $F$ satisfies \axref{DT}.

Finally, let us prove that $F$ satisfies \axref{DR}.
Since $F$ is regular, if we have $(a,b) \in F$, then there exists a set $U \in F$ such that $U \rb (a,b)$.
By \axref{DT}, we also have an open interval $(c,d) \in F$ such that $(c,d) \subseteq U$.
\rprop{rb-closure} implies that $a < c$ and $d < b$.
This concludes the proof.
\end{proof}

\section{Completion}
\label{sec:completion}

In this section, we define separated and complete cover spaces and prove that complete cover spaces form a reflective subcategory of $\cat{Cov}$.
Separated cover spaces are analogous to $T_0$-spaces or Hausdorff spaces.
These two notions are equivalent for cover spaces since we assume regularity.
We will also prove that a certain subcategory of complete cover spaces and Cauchy maps is equivalent to a certain subcategory of regular topological spaces.

To define separated spaces, we need to define an equivalence relation on points of a cover space.
We will say that two points $x$ and $y$ are \emph{equivalent} if every Cauchy cover contains a set that contains both $x$ and $y$.
The following lemma gives a useful characterization of this relation, which, in particular, implies that it is indeed an equivalence relation:

\begin{lem}[separated-char]
Let $x,y$ be a pair of points in a cover space $X$.
Then the following conditions are equivalent:
\begin{enumerate}
\item \label{sc:sub} $x^\wedge \subseteq y^\wedge$.
\item $x^\wedge$ and $y^\wedge$ are equivalent as Cauchy filters.
\item $x^\wedge = y^\wedge$.
\item \label{sc:con} Every neighborhood of $x$ contains $y$.
\item \label{sc:int} Every neighborhood of $x$ intersects every neighborhood of $y$.
\item \label{sc:neighbor} Every Cauchy cover contains a neighborhood of both $x$ and $y$.
\item \label{sc:cov} $x$ and $y$ are equivalent.
\end{enumerate}
\end{lem}
\begin{proof}
The equivalence of the first three conditions follows from \rprop{regular-minimal} since neighborhood filters are regular.
It is also clear that \eqref{sc:sub} implies \eqref{sc:con}, \eqref{sc:con} implies \eqref{sc:int}, and \eqref{sc:neighbor} implies \eqref{sc:cov}.
Also, \eqref{sc:cov} implies \eqref{sc:neighbor} by \axref{CR}.

Let us prove that \eqref{sc:int} implies \eqref{sc:neighbor}.
Let $C$ be a Cauchy cover.
Then there exist sets $U,W$ such that $U$ is a neighborhood of $x$, $U \rb W$, and $W \in C$.
Since $U \rb W$, there exists a neighborhood $V$ of $y$ such that $\overlap{V}{U} \implies V \subseteq W$.
By \eqref{sc:int}, $U$ and $V$ intersect, which implies that $V \subseteq W$.
Thus, $W$ is a neighborhood of both $x$ and $y$.

Finally, let us prove that \eqref{sc:neighbor} implies \eqref{sc:sub}.
Let $U$ be a neighborhood of $x$.
By \eqref{sc:cov}, there is a neighborhood $V$ of $x$ and $y$ such that $x \in V \implies V \subseteq U$
Thus, $U$ is also a neighborhood of $y$.
\end{proof}

\begin{defn}
A cover space $X$ is called \emph{separated} if any two points of $X$ satisfying equivalent conditions of \rlem{separated-char} are equal.
\end{defn}

\begin{example}
A discrete cover space is separated.
Indeed, since the cover consisting of singleton sets is Cauchy, any two equivalent points must be equal.
\end{example}

\begin{example}
The cover space induced by a metric space is separated.
Indeed, if two points $x$ and $y$ are equivalent, there exists an open ball of any given size containing both $x$ and $y$.
This implies that the distance between $x$ and $y$ is zero and, hence, that they are equal.
\end{example}

A topological space $X$ is \emph{Hausdorff} if, for every pair of points $x,y \in X$, if every neighborhood of $x$ intersects with every neighborhood of $y$, then $x$ and $y$ are equal.
\rlem{separated-char} implies that a cover space is separated if and only if its underlying topological space is Hausdorff.
A subset of a topological space is \emph{dense} if it intersects with every inhabited open set.
A function is \emph{dense} if its image is dense.
We include a proof of the following standard fact to demonstrate that it is constructive:

\begin{prop}[dense-unique]
Let $Y$ be a Hausdorff topological space, $S$ be a dense subset of a topological space $X$, and $f,g : X \to Y$ be continuous functions.
If $f$ and $g$ are equal on every point in $S$, then $f = g$.
\end{prop}
\begin{proof}
Let $x$ be a point in $X$.
Since $Y$ is Hausdorff, to prove that $f(x) = g(x)$, we just need to show that every neighborhood of $f(x)$ intersects with every neighborhood of $g(x)$.
Let $U$ be a neighborhood of $f(x)$ and $V$ be a neighborhood of $g(x)$.
Then $f^{-1}(U) \cap g^{-1}(V)$ is a neighborhood of $x$.
Since $S$ is dense, this neighborhood contains a point $x'$ for some $x' \in S$.
Thus, $U$ and $V$ both contain $f(x') = g(x')$.
\end{proof}

Now, we are ready to define complete cover spaces:

\begin{defn}
A cover space is \emph{complete} if it is separated and every Cauchy filter is equivalent to the neighborhood filter of some point.
\end{defn}

Thus, a cover space is separated if and only if the function $(-)^\wedge$ from points to regular Cauchy filters is injective and it is complete if and only if this function is bijective.
Every Cauchy filter $F$ in a complete cover spaces determines a point, which we will denote by $F^\vee$.
This is the unique point such that $F^{\vee \wedge} \subseteq F$.
The following lemma gives a useful description of the neighborhood filter $F^{\vee \wedge}$:

\begin{lem}[filter-point-char]
Let $F$ be a Cauchy filter in a complete cover space.
Then $F^{\vee \wedge} = \{ U \mid \exists V \in F, V \rb U \}$.
\end{lem}
\begin{proof}
Since $F^{\vee \wedge}$ and $F$ are equivalent, if $V \in F$ and $V \rb U$, then $U \in F^{\vee \wedge}$ by \rlem{cauchy-filter-rb}.
Conversely, suppose that $U \in F^{\vee \wedge}$.
By \rlem{rb-point}, there is a neighborhood $V$ of $F^\vee$ such that $V \rb U$.
Since $F^{\vee \wedge}$, we also have that $V \in F$.
\end{proof}

Now, we will show that cover maps to complete cover spaces extend uniquely along dense embeddings.
First, we prove a lemma that shows how to lift Cauchy filters along dense embeddings:

\begin{lem}[filter-lift]
Let $f : X \to Y$ be a dense embedding between cover spaces and $F$ be a Cauchy filter on $Y$.
Then $G = \{ U \mid \exists V,V' \subseteq Y, f^{-1}(V) \subseteq U, V' \rb V, V' \in F \}$ is a Cauchy filter on $X$ such that $f(G)$ is equivalent to $F$.
\end{lem}
\begin{proof}
Clearly, $G$ is a filter.
To see that it is proper, let $U$ be an element of $G$.
Then we have sets $V$ and $V'$ such that $f^{-1}(V) \subseteq U$, $V' \rb V$, and $V' \in F$.
Since $F$ is proper, there is a point $y \in V'$.
Since $f$ is dense, there is a point $x \in X$ such that $f(x) \in V$.
It follows that $x \in U$.

Let us show that $G$ is Cauchy.
Let $C$ be a Cauchy cover in $X$.
Since $f$ is an embedding, the set $\{ V' \mid \exists V, V' \rb V, \exists U \in C, f^{-1}(V) \subseteq U \}$ is a Cauchy cover in $Y$.
Since $F$ is Cauchy, there exist sets $V',V,U$ such that $V' \in F$, $V' \rb V$, $U \in C$, and $f^{-1}(V) \subseteq U$.
Then $U \in C \cap G$, so $G$ is Cauchy.

Finally, let us prove that $f(G)$ is equivalent to $F$.
Let $C$ be a Cauchy cover in $Y$.
Since $F$ is Cauchy, there exist sets $V' \in F$, $V \in C$ such that $V' \rb V$.
Then $V$ belongs to $C \cap F \cap f(G)$, so $f(G)$ is indeed equivalent to $F$.
\end{proof}

This lemma implies a useful criterion for completeness.
Given a dense embedding $f : X \to Y$, to prove that $Y$ is complete, it is enough to consider only Cauchy filters that come from $X$:

\begin{lem}[complete-part]
Let $X$ be a cover space, $Y$ be a separated cover space, and $f : X \to Y$ be a dense embedding.
Then $Y$ is complete if and only if, for every regular Cauchy filter $F$ on $X$, there is a point $y \in Y$ such that $f(F)$ is equivalent to the neighborhood filter of $y$.
\end{lem}
\begin{proof}
The ``only if'' direction is obvious.
The coverse follows from \rlem{filter-lift}.
\end{proof}

Now, we are ready to prove the extension property:

\begin{thm}[dense-lift]
For every complete cover space $Z$, every dense embedding $f : X \to Y$ between cover spaces, and every Cauchy map $g : X \to Z$,
there exists a unique Cauchy map $\widetilde{g} : Y \to Z$ such that $\widetilde{g} \circ f = g$.
If $g$ is a cover map, then so is $\widetilde{g}$.
\end{thm}
\begin{proof}
The uniqueness follows from \rprop{dense-unique} and \rprop{cauchy-continuous}.
Let us prove the existence.
Let $y$ be a point in $Y$.
Then, by \rlem{filter-lift}, there is a Cauchy filter $G_y$ on $X$ such that $f(G_y)$ is equivalent to $y^\wedge$.
We define $\widetilde{g}(y)$ as $g(G_y)^\vee$.

Let us prove that $\widetilde{g}$ is a Cauchy map.
Let $F$ be a Cauchy filter on $Y$ and $C$ be a Cauchy cover on $Z$.
We need to show that there exists a set $U \in C$ such that $\widetilde{g}^{-1}(U) \in F$.
Let $G_F$ be the filter constructed in \rlem{filter-lift} for $F$.
Since $g$ is Cauchy, $g(G_F)$ is a Cauchy filter.
Thus, there exist sets $U' \in g(G_F)$ and $U \in C$ such that $U' \rb U$.
Unfolding the definition of $G_F$, we get sets $V' \in F$ and $V$ such that $f^{-1}(V) \subseteq g^{-1}(U')$ and $V' \rb V$.
Since $F$ is a filter, it is enough to show that $V' \subseteq \widetilde{g}^{-1}(U)$.
Let $y$ be a point in $V'$.
We will show that $U$ is a neighborhood of $\widetilde{g}(y)$.
By \rlem{filter-point-char}, to do this, it is enough to show that $U' \in g(G_y)$.
This follows from the definition of $G_y$ and \rlem{rb-point} since $f^{-1}(V) \subseteq g^{-1}(U')$ and $V$ is a neighborhood of $y$.

Now, let us show that $\widetilde{g} \circ f = g$.
Let $x$ be a point in $X$.
By \rlem{separated-char}, it is enough to show that every neighborhood of $\widetilde{g}(f(x))$ contains $g(x)$.
Let $W$ be a neighborhood of $\widetilde{g}(f(x))$.
By \rlem{filter-point-char}, there is a set $W' \in g(G_{f(x)})$ such that $W' \rb W$.
From the definition of $G_{f(x)}$, we get sets $V$ and $V'$ such that $f^{-1}(V) \subseteq g^{-1}(W')$, $V' \rb V$, and $\{ f(x) \} \rb V'$.
This implies that $W$ is a neighborhood of $g(x)$.

Finally, assume that $g$ is a cover map and let us prove that $\widetilde{g}$ is also a cover map.
Let $D$ be a Cauchy cover on $Z$.
By \axref{CR}, the set $\{ g^{-1}(W') \mid \exists W \in D, W' \rb W \}$ is a Cauchy cover on $X$.
Since $f$ is an embedding, $\{ V'' \mid \exists V,V',W',W, V'' \rb V', V' \rb V, W \in D, W' \rb W, f^{-1}(V) \subseteq g^{-1}(W') \}$ is a Cauchy cover on $Y$.
We will show that this cover refines $\{ \widetilde{g}^{-1}(W) \mid W \in D \}$.
Let $V,V',V'',W,W'$ be sets such that $V'' \rb V'$, $V' \rb V$, $W \in D$, $W' \rb W$, and $f^{-1}(V) \subseteq g^{-1}(W')$.
Then $V'' \subseteq \widetilde{g}^{-1}(W)$.
Indeed, let $y$ be a point in $V''$.
Then we need to show that $W$ is a neighborhood of $g(G_y)^\vee$, but this follows from \rlem{filter-point-char} and the definition of $G_y$.
\end{proof}

A \emph{completion} of a cover space $X$ is a complete cover space $Y$ together with a dense embedding.
\rthm{dense-lift} implies that completion is unique up to isomorphism.
Now, we can show that every cover space has a completion:

\begin{thm}[completion]
For every cover space $X$, there is a complete cover space $C(X)$ and a dense embedding $\eta_X : X \to C(X)$.
\end{thm}
\begin{proof}
Let $C(X)$ be the set of regular Cauchy filters in $X$.
For every set $U \subseteq X$, let $\widetilde{U} \subseteq C(X)$ be the set of all regular Cauchy filters containing $U$.
The set Cauchy covers of $C(X)$ is defined as the set of those $C$ which are refined by the set $\{ \widetilde{U'} \mid U' \in C' \}$ for some Cauchy cover $C'$ in $X$.
Every such $C$ is a cover.
Indeed, given a regular Cauchy filter $F$, there is a set $U \in F \cap C'$.
Thus, we have $V \in C$ such that $F \in \widetilde{U} \subseteq V$.

Conditions \axref{CT} and \axref{CE} are obvious.
Let us check \axref{CG}.
Let $C$ and $\{ D_U \}_{U \in C}$ be Cauchy covers in $C(X)$.
We need to show that $E = \{ U \cap V \mid U \in C, V \in D_U \}$ is Cauchy.
The set $D'_{U'} = \{ V' \mid \exists U \in C, V \in D_U, \widetilde{U'} \subseteq U, \widetilde{V'} \subseteq V \}$ is a Cauchy cover in $X$ for every $U' \in C'$.
By \axref{CG}, the set $E' = \{ U' \cap V' \mid U' \in C', \exists U \in C, V \in D_U, \widetilde{U'} \subseteq U, \widetilde{V'} \subseteq V \}$ is a Cauchy cover in $X$.
Now, $E$ is Cauchy since it is refined by $\{ \widetilde{W'} \mid W' \in E' \}$.

To prove \axref{CR}, we first need to show that $\widetilde{V} \rb_{C(X)} \widetilde{U}$ whenever $V \rb_X U$.
To prove this, we just need to show that $\{ \widetilde{W} \mid \overlap{W}{V} \implies W \subseteq U \}$ refines $\{ W \mid \overlap{W}{\widetilde{V}} \implies W \subseteq \widetilde{U} \}$.
Indeed, if $\widetilde{W}$ intersects $\widetilde{V}$, then there is a Cauchy filter $F$ containing $W \cap V$.
Since $F$ is proper, $W$ intersects $V$ and, hence $W \subseteq U$.
It follows that $\widetilde{W} \subseteq \widetilde{U}$.
Now, we are ready to check \axref{CR}.
Let $C$ be a Cauchy cover in $C(X)$.
The fact we just proved implies that $\{ \widetilde{V'} \mid \exists U' \in C', V' \rb_X U' \}$ refines $\{ V \mid \exists U \in C, V \rb_{C(X)} U \}$ and, hence, the latter cover is Cauchy.

We define $\eta_X(x)$ as $x^\wedge$.
Let us check tht $\eta_X : X \to C(X)$ is a cover map.
Let $C'$ be a Cauchy cover in $X$.
We need to show that $\{ \eta_X^{-1}(\widetilde{U'}) \mid U' \in C' \}$ is also Cauchy.
This follows from the fact that this cover is refined by $\{ V' \mid \exists U' \in C', V' \rb_X U' \}$, which is Cauchy by \axref{CR}.

It is clear that $\eta_X : X \to C(X)$ is an embedding.
Let us prove that it is dense.
Let $F$ be a regular Cauchy filter and $V$ be a neighborhood of $V$ in $C(X)$.
We need to show that there exists $x \in X$ such that $\eta_X(x) \in V$.
It is enough to show that $\{ x \in X \mid \eta_X(x) \in V \} \in F$ since $F$ is proper.
Since $\{ F \} \rb_{C(X)} V$, there exists a Cauchy cover $C'$ in $X$ sucht that $\{ \widetilde{U'} \mid U' \in C' \}$ refines $\{ W \mid F \in W \implies W \subseteq V \}$.
Since $F$ is a Cauchy filter and $\{ V' \mid \exists U' \in C', V' \rb_X U' \}$ is a Cauchy cover, there exist $V' \in F$ and $U' \in C'$ such that $V' \rb_X U'$.
Then $U' \in F$ and, hence, $F \in \widetilde{U'}$ and $\widetilde{U'} \subseteq V$.
Now, we just need to show that $V' \subseteq \{ x \in X \mid \eta_X(x) \in V \}$.
Let $x$ be a point in $V'$.
Then $U'$ is a neighborhood of $x$, so we have $\eta_X(x) = x^\wedge \in \widetilde{U'}$.
It follows that $\eta_X(x) \in V$.
This complete the proof that $\eta_X$ is dense.

Now, let us check that $C(X)$ it is separated.
Let $F$ and $G$ be regular Cauchy filters on $X$ such that every Cauchy cover in $C(X)$ contains a set which contins both $F$ and $G$.
This implies that, for every Cauchy cover $C'$ in $X$, there is a set $U' \in C'$ such that $F,G \in \widetilde{U'}$.
This means that $F$ and $G$ are equivalent Cauchy filters and, since they are regular, they are equal.

Finally, we just need to show that $C(X)$ is complete.
By \rlem{complete-part}, it is enough to prove that, for every regular Cauchy filter $F$ on $X$, Cauchy filters $\eta_X(F)$ and $F^\wedge$ are equivalent.
We just proved that, for every neighborhood $V$ of $F$, the set $\eta_X^{-1}(V)$ belongs to $F$.
This means that $F^\wedge$ is a subset of $\eta_X(F)$, so they are equivalent.
\end{proof}

\begin{cor}
Complete cover spaces is a reflective subcategory of $\cat{Cov}$.
\end{cor}

The embedding $\eta_X$ might not be injective in general, but it is for separated cover spaces.
In fact, this property characterize separated spaces:

\begin{prop}[embedding-injective]
Given a cover space $X$, the following conditions are equivalent:
\begin{enumerate}
\item \label{ei:sep} $X$ is a separated cover space.
\item \label{ei:any} Every embedding $X \to Y$ is injective.
\item \label{ei:eta} The embedding $\eta_X : X \to C(X)$ is injective.
\item \label{ei:ex} There exists a separated cover space $Y$ and an injective cover map $X \to Y$.
\end{enumerate}
\end{prop}
\begin{proof}
Conditions \eqref{ei:sep} and \eqref{ei:eta} are equivalent by the definition of separated cover spaces.
It is also clear that $\eqref{ei:any}$ implies \eqref{ei:eta} and \eqref{ei:eta} implies \eqref{ei:ex}.
To see that \eqref{ei:ex} implies \eqref{ei:eta}, note that if a cover map $f : X \to Y$ and the map $\eta_Y : Y \to C(Y)$ are injective,
then $\eta_X : X \to C(X)$ is also injective by naturality of $\eta$.

Finally, let us show that \eqref{ei:sep} implies \eqref{ei:any}.
Let $X$ be a separated cover space, $f : X \to Y$ be an embedding, and $x,x'$ be a pair of points in $X$ such that $f(x) = f(x')$.
Since $X$ is separated, to prove that $x = x'$, it is enough to show that every neighborhood $U$ of $x$ contains $x'$.
Since $\{ x \} \rb_X U$ and $f$ is an embedding, $\{ V \mid \exists W \subseteq X, (x \in W \implies W \subseteq U), f^{-1}(V) \subseteq W \}$ is a Cauchy cover in $Y$.
In particular, it is a cover, so there exist sets $V$ and $W$ such that $f(x) = f(x') \in V$, $f^{-1}(V) \subseteq W$, and implication $x \in W \implies W \subseteq U$ holds.
But the first two conditions imply that both $x$ and $x'$ belong to $W$ and the last one implies that $W \subseteq U$, so we have that $x' \in U$.
\end{proof}

Finally, we will discuss Cauchy maps.
\rthm{dense-lift} implies that complete cover spaces form a reflective subcategory in the category of cover spaces and Cauchy maps.
We will show that this subcategory is equivalent to a subcategory of the category of topological spaces.

\begin{prop}[u-fully-faithful]
Let $X$ be a complete cover space and $Y$ be a cover space.
Then a map $f : X \to Y$ is Cauchy if and only if it is continuous.
\end{prop}
\begin{proof}
The ``only if'' direction was proved in \rprop{cauchy-continuous}.
Let us prove the other direction.
Suppose that $f$ is continuous.
Let $F$ be a Cauchy filter on $X$ and $C$ be a Cauchy cover of $Y$.
We need to show that there exists $U \in C$ such that $f^{-1}(U) \in F$.
Since $X$ is complete, there exists a point $x \in X$ such that $x^\wedge \subseteq F$.
Since $C$ is a Cauchy cover, there exists $U \in C$ such that $\{ f(x) \} \rb U$.
Since $f$ is continuous, $f^{-1}(U)$ is a neighborhood of $x$.
Since $x^\wedge \subseteq F$, we get that $f^{-1}(U) \in F$.
\end{proof}

Let $X$ be a topological space.
A filter $F$ on $X$ is called \emph{open} if, for every $U \in F$, there exists an open $V \in F$ such that $V \subseteq U$.
An open filter $F$ is called \emph{completely prime} if, for every set of opens $\{ U_i \}_{i \in I}$, if $\bigcup_{i \in I} U_i \in F$, then $U_i \in F$ for some $i \in I$.
If $x$ is a point in $X$, then the neighborhood filter $x^\wedge$ is a completely prime open filter.
A topological space is called \emph{sober} if every completely prime open filter is of the form $x^\wedge$ for a unique point $x \in X$.

\begin{prop}[u-sober]
If $X$ is a complete cover space, then its underlying topological space is sober.
\end{prop}
\begin{proof}
Let $F$ be a completely prime open filter.
Clearly, it is also Cauchy.
\rlem{rb-point} implies that, for every open $U$, we have $U = \bigcup \{ V \mid V \rb_X U \}$.
It follows that $F$ is regular.
By completeness of $X$, there exists a unique $x \in X$ such that $F = x^\wedge$.
\end{proof}

Recall that if $X$ is a regular topological space, then we write $T(X)$ for the cover space consisting of all \emph{neighborhood covers} of $X$, that is, covers that contain a neighborhood of every point.
Classically, if $X$ is sober, then $T(X)$ is complete, but we do not know if this is true constructively.
To fix this problem, we define another version of regularity, which is classically equivalent to the one we had before, but constructively it is stronger.

If $U$ and $V$ are subsets of a topological space $X$, we will say that $V$ is \emph{strongly rather below} $U$, written $V \rb^s U$, if $X = (X \backslash V) \cup U$.
A topological space is \emph{strongly regular} if every open $U$ is the union of opens which are strongly rather below $U$.
There is also a stronger version of regularity for precover spaces.
Given subsets $V$ and $U$ of a precover space $X$, we will say that $V$ is \emph{strongly rather below} $U$, written $V \rb^s U$, if the set $\{ X \backslash V, U \}$ is a Cauchy cover.
This relation satisfies conditions analogous to conditions in \rprop{rb-props}:

\begin{prop}[rbs-props]
The following is true for subsets $U,V,U',V'$ of a precover space $X$:
\begin{enumerate}
\item If $V \rb^s U$, then $V \rb U$.
\item If $V' \subseteq V \rb^s U \subseteq U'$, then $V' \rb^s U'$.
\item If $V \rb^s U$ and $V' \rb^s U'$, then $V \cap V' \rb^s U \cap U'$.
\item $V \rb^s X$.
\item $\varnothing \rb^s U$.
\end{enumerate}
\end{prop}

A precover space is called \emph{strongly regular} if, for every Cauchy cover $C$, the set $\{ V \mid \exists U \in C, V \rb^s U \}$ is a Cauchy cover.
Since $V \rb^s U$ implies $V \rb U$, every strongly regular precover space is a cover space.
Every metric space is strongly regular.
The completion of a strongly regular cover space is strongly regular.
The proof is the same as in \rthm{completion}.
Also, strongly regular cover spaces are reflective in the category of cover spaces.
The proof is the same as the proof of \rprop{regular-reflective}.
We will also need the following lemma later:

\begin{lem}[rbs-point]
If $U$ is a neighborhood of a point $x$ in a strongly regular cover space, then there exists a neighborhood $V$ of $x$ such that $V \rb^s U$.
\end{lem}
\begin{proof}
The proof is essentially the same as the proof of \rlem{rb-point}.
\end{proof}

Finally, we can show that the category of strongly regular complete cover spaces and Cauchy maps is equivalent to the category of strongly regular sober spaces.

\begin{prop}[u-strongly-regular]
If $X$ is a strongly regular cover space, then its underlying topological space is strongly regular.
\end{prop}
\begin{proof}
The proof is the same as the proof of \rprop{top-regular} using \rlem{rbs-point}.
\end{proof}

\begin{lem}[strongly-regular-filter]
Let $F$ be a regular Cauchy filter in a strongly regular cover space $X$.
Then, for every $U \in F$, there exists $V \in F$ such that $V \rb^s U$.
\end{lem}
\begin{proof}
Consider the set $G = \{ U \mid \exists V \in F, V \rb^s U \}$.
\rprop{rbs-props} implies that it is a proper filter.
Moreover, since $X$ is strongly regular, $G$ is a Cauchy filter.
Since $G \subseteq F$ and $F$ is regular, \rprop{regular-minimal} implies that $F \subseteq G$, which finishes the proof.
\end{proof}

\begin{prop}[u-surjective]
If $X$ is a strongly regular sober space, then $T(X)$ is a strongly regular complete cover space.
\end{prop}
\begin{proof}
Let $C$ be a neighborhood cover of $X$.
Then, for every point $x \in X$, there is a neighborhood $U$ of $x$ in $C$.
By \rlem{rbs-point}, there exists a neighborhood $V$ of $x$ such that $V \rb^s U$.
This means that $\{ V \mid \exists U \in C, V \rb^s U\}$ is a neighborhood cover of $X$.
Thus, $T(X)$ is strongly regular.

To prove that $T(X)$ is complete, it is enough to show that every regular Cauchy filter is completely prime.
Let $F$ be a regular Cauchy filter.
Let $\{ U_i \}_{i \in I}$ be a set of opens such that $\bigcup_{i \in I} U_i \in F$.
By \rlem{strongly-regular-filter}, there exists a set $V \in F$ such that $V \rb^s \bigcup_{i \in I} U_i$.
It follows that $\{ X \backslash V \} \cup \{ U_i \mid i \in I \}$ is a neighborhood cover of $X$.
Since $F$ is a proper filter and $V \in F$, the set $X \backslash V$ does not belong to $F$.
It follows that $U_i$ belongs to $F$ for some $i \in I$ since $F$ is a Cauchy filter.
This completes the proof.
\end{proof}

\begin{thm}[cover-cauchy-top]
The forgetful functor from the category of cover spaces and Cauchy maps to the category of topological spaces restricts to an equivalence between the category of strongly regular complete cover spaces and Cauchy maps and the category of strongly regular sober spaces.
\end{thm}
\begin{proof}
\rprop{top-regular}, \rprop{u-strongly-regular}, \rprop{u-sober}, \rprop{cauchy-continuous} imply that we indeed have such a functor.
\rprop{u-fully-faithful} implies that this functor is fully faithful and \rprop{u-surjective} implies that it is essentially surjective on objects.
\end{proof}

\section{Locales}

In the previous section, we showed that the category of strongly regular complete cover spaces and Cauchy maps is equivalent to the category of strongly regular topological spaces.
In this section, we prove a similar result for the category of cover spaces and cover maps instead of Cauchy maps.
To do this, we need to replace the category of topological spaces with the category of locales.

Recall that a \emph{frame} is a complete distributive lattice.
A map of frames is a function that preserves finite meets and arbitrary joins.
The category of locales is the opposite of the category of frames.
Given a locale $M$, we will write $O_M$ for the underlying frame of this locales.
Elements of $O_M$ are called \emph{opens} of $M$.
Given a map $f : M \to N$ of locales, we will write $f^* : O_N \to O_M$ for the corresponding frame map.

A \emph{filter} in a lattice $M$ is an upward closed subset of $M$ closder under finite meets.
A \emph{completely prime filter} in a complete lattice is a filter $F$ such that the following implication holds: $\bigvee_{j \in J} a_j \in F \implies \exists j \in J, a_j \in F$.
A \emph{point} of a locale $M$ is a complete lattice in $O_M$.
The set of points of a lattice will be denoted by $P(M)$.
There are adjoint functions $\varepsilon^* \dashv \varepsilon_*$ between $O_M$ and the set of subsets of $P(M)$:
$\varepsilon^*(a)$ is the set of points containing $a$ and $\varepsilon_*(U)$ is the join of those $a \in O_M$, for which $\varepsilon^*(a)$ is a subset of $U$.

The set of points of a locale $M$ can be endowed with the structure of a precover space.
We define $\mathcal{C}_{P(M)}$ to be the set of those $C$ for which $\bigvee \{ \varepsilon_*(U) \mid U \in C \} = \top$.

\begin{prop}
For every locale $M$, the pair $(P(M),\mathcal{C}_{P(M)})$ is a precover space.
\end{prop}
\begin{proof}
First, let us check that given every $C$ such that $\bigvee \{ \varepsilon_*(U) \mid U \in C \} = \top$ is indeed a cover.
If $F$ is a point of $M$, then there exist a set $U \in C$ and an open $a \in F$ such that $\varepsilon^*(a) \subseteq U$.
This implies that $F$ belongs to $U$.

Condition \axref{CT} is obvious and \axref{CE} easily follows from the fact that $\varepsilon_*$ is monotone.
Condition \axref{CG} follows from distributivity and the fact that $\varepsilon_*$ preserves meets.
\end{proof}

The following simple lemma gives a useful characterization of Cauchy covers in $P(M)$:

\begin{lem}[locale-cauchy]
If $M$ is a locale, then a set is a Cauchy cover if and only if it is equal to $\{ \varepsilon^*(a) \mid a \in C \}$ for some set of opens $C$ such that $\bigvee C = \top$.
\end{lem}
\begin{proof}
If $\bigvee C = \top$, then $C' = \{ \varepsilon^*(a) \mid a \in C \}$ is a Cauchy cover because $\top = \bigvee C \leq \bigvee \{ \varepsilon_*(U) \mid U \in C' \}$.
Conversely, let $D$ be a Cauchy cover.
If we define $D'$ as $\{ \varepsilon_*(U) \mid U \in D \}$, then $\bigvee D' = \top$ and $D = \{ \varepsilon^*(a) \mid a \in D' \}$.
\end{proof}

Every a map of locales $f : M \to N$ induces a function on points $P(f) : P(M) \to P(N)$ given by $P(f)(F) = \{ a \in O_N \mid f^*(a) \in F \}$.
It is easy to see that this function is a cover map.
Thus, we get a functor $P : \cat{Loc} \to \cat{Precov}$.

Given an open $b$ of a locale $M$, let $\neg b = \bigvee \{ a \in O_M \mid b \wedge a = \bot \}$.
This open is the largest open $a$ for which $b \wedge a = \bot$ holds.
Given two opens $b,a \in O_M$, we will say that $b$ is \emph{rather below} $a$, written $b \rb a$, if $\neg b \vee a = \top$.
A locale $M$ is called \emph{regular} if, for every open $a \in O_M$, we have $a = \bigvee \{ b \in O_M \mid b \rb a \}$.
The following proposition shows that every regular locale gives rise to a strongly regular cover space:

\begin{prop}[locale-regular]
If $M$ is a regular locale, then $P(M)$ is a complete strongly regular cover space.
\end{prop}
\begin{proof}
First, let us show that if $a,b$ is a pair of opens such that $b \rb a$, then $\varepsilon^*(b) \rb^s \varepsilon^*(a)$.
Since $\top \leq \neg b \vee a$, the set $\{ \varepsilon^*(\neg b), \varepsilon^*(a) \}$ is a Cauchy cover.
Thus, we just need to show that $\varepsilon^*(\neg b)$ is a subset of $X \backslash \varepsilon^*(b)$.
This follows from the fact that every point of $M$ is a proper filter: if a point contains $\neg b$ and $b$, then it contains $\bot$, which is impossible.
Now, it is easy to see that $P(M)$ is strongly regular.

Now, let us prove that $P(M)$ is separated.
Let $F$ and $G$ be a pair of equivalent points.
By \rlem{separated-char}, every neighborhood $U$ of $F$ contains $G$ and vice versa.
Let us show that every open $a$ in $F$ belongs to $G$.
Since $M$ is regular, we have $a \leq \bigvee \{ b \mid b \rb a \}$.
Thus, there exists an open $b \in F$ such that $b \rb a$.
Then we have $\varepsilon^*(b) \rb \varepsilon^*(a)$.
Since $F \subseteq \varepsilon^*(b)$, the set $\varepsilon^*(a)$ is a neighborhood of $F$.
It follows that $a$ belongs to $G$.
Thus, $F$ is a subset of $G$ and vice versa, so $P(M)$ is separated.

Finally, let us prove that $P(M)$ is complete.
For every Cauchy filter $F$ in $P(M)$, we define a point $\widetilde{F}$ in $M$.
This point consists of those opens $a$ for which there is an open $b$ such that $b \rb a$ and $F$ contains $\varepsilon^*(b)$.
It is easy to see that this is indeed a filter.
Let us show that it is completely prime.
Suppose that $\widetilde{F}$ contains a join $\bigvee_{j \in J} a_j$.
Then there is an open $b$ such that $b \rb \bigvee_{j \in J} a_j$ and $\varepsilon^*(b) \in F$.
It follows that $C = \{ \varepsilon^*(\neg b) \} \cup \{ \varepsilon^*(b') \mid \exists j \in J, b' \rb a_j \}$ is a Cauchy cover.
If $F$ contains $\varepsilon^*(\neg b)$, then it also contains $\varepsilon^*(\neg b) \cap \varepsilon^*(b) = \varepsilon^*(\bot) = \varnothing$, which is impossible since $F$ is proper.
Thus, there exists an open $b'$ such that $F$ contains $\varepsilon^*(b')$ and $b' \rb a_j$ for some $j \in J$.
This means that $\widetilde{F}$ contains $a_j$, which concludes the proof that $P(M)$ is complete.
\end{proof}

Another property of locales that we will need is the denseness of its set of points.
We will say that a locale \emph{has a strongly dense set of points} if, for every open $a$, $a \leq \bigvee \{ \top \mid \exists x \in \varepsilon^*(a) \}$.
Every spatial locale has a strongly dense set of points, but not all locales with a dense set of points are spatial.
There is also a weaker condition: a locale \emph{has a dense set of points} if $\varepsilon_*(\varnothing) = \bot$.
Classically, these conditions are equivalent, but constructively the former is stronger.

We also have a corresponding property for precover spaces.
We will say that a precover space \emph{has a dense set of points} if, for every Cauchy cover $C$, the subset of $C$ consisting of inhabited sets is also a Cauchy cover.
Classically, this is almost always true: assuming the law of excluded middle, the only precover space that does not have a dense set of points is $(\varnothing, \{ \{ \varnothing \} \})$.
Constructively, this property is more subtle.
We will discuss it in more detail in Section~\ref{sec:reals}.
We also have a weak version of this condition: a precover space \emph{has a weakly dense set of points} if $C$ is a Cauchy cover whenever $C \cup \{ \varnothing \}$ is.
The following proposition shows how these properties are related to corresponding properties for locales:

\begin{prop}[locale-cover-dense]
If $M$ is a locale with a dense set of points, then $P(M)$ has a weakly dense set of points.
If $M$ has a strongly dense set of points, then $P(M)$ has a dense set of points.
\end{prop}
\begin{proof}
Suppose that $M$ has a dense set of points and that $C \cup \{ \varnothing \}$ is a Cauchy cover in $P(M)$.
Then $\top = \bigvee \{ \varepsilon_*(U) \mid U \in C \} \vee \varepsilon_*(\varnothing) = \bigvee \{ \varepsilon_*(U) \mid U \in C \}$, so $C$ is also a Cauchy cover.
Now, suppose that $M$ has a strongly dense set of points and that $C$ is a Cauchy cover in $P(M)$.
To prove that the set of inhabited elements of $C$ is also a Cauchy cover, it is enough to show that $\bigvee \{ \varepsilon_*(U) \mid U \in C \} \leq \bigvee \{ \varepsilon_*(U) \mid U \in C, \exists x \in U \}$.
Thus, we need to show that $\varepsilon_*(U) \leq \bigvee \{ \varepsilon_*(U) \mid U \in C, \exists x \in U \}$ for all $U \in C$.
Since $M$ has a dense set of points, we can assume that there is a point $x$ containing $\varepsilon_*(U)$.
It follows that $x$ belongs to $U$, which completes the proof.
\end{proof}

We will also need the following simple lemma:

\begin{lem}[locale-dense]
A locale has a dense set of points if and only if the only open that does not have points is the bottom one.
\end{lem}
\begin{proof}
Immediate from the definition of $\varepsilon_*(\varnothing)$.
\end{proof}

The following proposition shows that $P : \cat{Loc} \to \cat{Precov}$ restricts to a fully faithful embedding from the category of regular locales with a strongly dense set of points to the category of cover spaces:

\begin{prop}[locale-ff]
If $M$ is a locale with a dense set of points and $N$ is a regular locale, then the function $P : \cat{Loc}(M,N) \to \cat{Cov}(P(M),P(N))$ is a bijection.
\end{prop}
\begin{proof}
Let us prove injectivity first.
Let $f,g : M \to N$ be maps of locales such that $P(f) = P(g)$.
By symmetry, it is enough to show that, for every open $a \in O_N$, we have that $f^*(a) \leq g^*(a)$.
Since $N$ is regular, we have $a = \bigvee \{ b \in O_N \mid b \rb a \}$.
Thus, it is enough to show that $f^*(b) \leq g^*(a)$ for every $b \rb a$.
Since $\top = \neg b \vee a$, we have that $f^*(b) = f^*(b) \wedge (g^*(\neg b) \vee g^*(a))$.
Thus, it is enough to show that $f^*(b) \wedge g^*(\neg b) = \bot$.
By \rlem{locale-dense}, it is enough to show that the open $f^*(b) \wedge g^*(\neg b)$ does not have points.
Indeed, if $F$ is a completely prime filter that contains this open, then $F$ contains $f^*(b)$ and $g^*(\neg b)$.
This implies that $P(f)(F) = P(g)(F)$ contains both $b$ and $\neg b$, which is impossible since this filter is proper.

Now, we will prove surjectivity.
Let $f : P(M) \to P(N)$ be a cover map.
Then we define $g : M \to N$ as follows: $g^*(a) = \bigvee \{ \varepsilon_*(f^{-1}(\varepsilon^*(b))) \mid b \rb a \}$.
It is clear that $g^*$ preserves finite meets since $\varepsilon_*(f^{-1}(\varepsilon^*(-)))$ does.
Let us show that $g^*$ preserves joins.
Since $g^*$ is monotone, it is enough to show that $g^*(\bigvee_{j \in J} a_j) \leq \bigvee_{j \in J} g^*(a_j)$.
Thus, we need to show that $\varepsilon_*(f^{-1}(\varepsilon^*(b))) \leq \bigvee \{ \varepsilon_*(f^{-1}(\varepsilon^*(b'))) \mid \exists j \in J, b' \rb a_j \}$ for every open $b \in O_N$ such that $b \rb \bigvee_{j \in J} a_j$.
Since $N$ is regular, we have $\top = \neg b \vee \bigvee \{ b' \in O_N \mid \exists j \in J, b' \rb a_j \}$.
By \rlem{locale-cauchy}, the set $\{ \varepsilon^*(\neg b) \} \cup \bigcup \{ \varepsilon^*(b') \mid \exists j \in J, b' \rb a_j \}$ is a Cauchy cover in $P(N)$.
Since $f$ is a cover map, $\{ f^{-1}(\varepsilon^*(\neg b)) \} \cup \bigcup \{ f^{-1}(\varepsilon^*(b')) \mid \exists j \in J, b' \rb a_j \}$ is a Cauchy cover in $P(M)$.
It follows that $\varepsilon_*(f^{-1}(\varepsilon^*(\neg b))) \vee \bigvee \{ \varepsilon_*(f^{-1}(\varepsilon^*(b'))) \mid \exists j \in J, b' \rb a_j \} = \top$.
Thus, we just need to show that $\varepsilon_*(f^{-1}(\varepsilon^*(\neg b))) \wedge \varepsilon_*(f^{-1}(\varepsilon^*(b))) = \bot$.
This follows from the facts that $\varepsilon_*(f^{-1}(\varepsilon^*(-)))$ preserves meets and that $M$ has a dense set of points: $\varepsilon_*(f^{-1}(\varepsilon^*(\neg b))) \wedge \varepsilon_*(f^{-1}(\varepsilon^*(b))) = \varepsilon_*(f^{-1}(\varepsilon^*(\neg b \wedge b))) = \varepsilon_*(f^{-1}(\varepsilon^*(\bot))) = \varepsilon_*(f^{-1}(\varnothing)) = \varepsilon_*(\varnothing) = \bot$.

Finally, we need to show that $P(g) = f$.
If $F$ is a point of $M$, then $P(g)(F) = \{ a \in O_N \mid \exists b \rb a, \varepsilon_*(f^{-1}(\varepsilon^*(b))) \in F \}$.
First, let us show that $P(g)(F) \subseteq f(F)$.
Let $b$ and $a$ be opens such that $b \rb a$ and $\varepsilon_*(f^{-1}(\varepsilon^*(b))) \in F$.
Since $\varepsilon^*(\varepsilon_*(f^{-1}(\varepsilon^*(b)))) \subseteq f^{-1}(\varepsilon^*(b))$, we get that $F \in f^{-1}(\varepsilon^*(b))$.
It follows that $a \in f(F)$.
Conversely, let $a \in O_N$ be an open in $f(F)$.
Since $N$ is regular, there opens $c,b \in f(F)$ such that $c \rb b$ and $b \rb a$.
By \rlem{locale-cauchy}, the set $\{ \varepsilon^*(\neg c), \varepsilon^*(b) \}$ is a Cauchy cover.
Since $f$ is a cover map, $\{ f^{-1}(\varepsilon^*(\neg c)), f^{-1}(\varepsilon^*(b)) \}$ is also a Cauchy cover.
Thus, we get that $\top = \varepsilon_*(f^{-1}(\varepsilon^*(\neg c))) \vee \varepsilon_*(f^{-1}(\varepsilon^*(b)))$.
Since $F$ is a completely prime filter, either $\varepsilon_*(f^{-1}(\varepsilon^*(\neg c)))$ or $\varepsilon_*(f^{-1}(\varepsilon^*(b)))$ belongs to $F$.
The former is impossible since then $\neg c$ belongs to $F$ and this contradicts the fact that $c$ belongs to $F$.
In the latter case, we get that $a \in P(g)(F)$, which concludes the proof.
\end{proof}

Next, we need to construct a locale from a precover space.
We define the frame of opens of this locale by a method described in Section~2.11 of \cite{stone-spaces} and Section~0.1.3 of \cite{locales-tychonoff}, which we briefly recall here.
A \emph{coverage} on a meet-semilattice $P$ is a relation $\triangleleft_0$ between elements of $P$ and subsets of $P$ satisfying the following conditions:
\begin{itemize}
\item If $a \triangleleft_0 U$, then $b \leq a$ for every $b \in U$.
\item If $a \triangleleft_0 U$ and $b \leq a$, then there exists a set $V$ such that $b \triangleleft_0 V$ and, for every $c \in V$, it is true that $c \leq d$ for some $d \in U$.
\end{itemize}
Given a coverage $(P,\triangleleft_0)$, a subset $U$ of $P$ is called a \emph{$\triangleleft_0$-ideal} if it is downward closed and $a \in U$ whenever $a \triangleleft_0 V$ and $V \subseteq U$.
Then the set of $\triangleleft_0$-ideals is a frame under the inclusion relation.
We will denote this frame by $F(P,\triangleleft_0)$.
The idea of this definition is that this frame is generated under joins by elements of $P$ under relations given by $\triangleleft_0$.

Given a coverage $(P,\triangleleft_0)$, it is convenient to define a new relation $\triangleleft$ as the closure of $\triangleleft_0$ under the following rules:
\begin{itemize}
\item If $a \in U$, then $a \triangleleft U$.
\item If $a \leq b$, then $a \triangleleft \{ b \}$.
\item If $a \triangleleft U$ and $b \triangleleft V$ for every $b \in U$, then $a \triangleleft V$.
\end{itemize}
Now, if $U$ is a subset of $P$, then $\overline{U} = \{ a \in P \mid a \triangleleft U \}$ is the smallest $\triangleleft_0$-ideal containing $U$.
The join of a set $U$ is given by $\overline{\bigcup U}$.
There is a meet preserving monotone map $[-] : P \to F(P,\triangleleft_0)$ given by $[a] = \overline{ \{ a \} }$.
The frame $F(P,\triangleleft_0)$ is generated under joins by elements of the form $[a]$.
We also have that $[a] \leq \bigvee_{j \in J} [b_j]$ if and only if $a \triangleleft \{ b_j \mid j \in J \}$.

Now, we are ready to define a locale $L(X)$ for every precover space $X$.
The frame of opens of this locale is generated by subsets of $X$ with the following coverage:
\begin{itemize}
\item $U \triangleleft_0 \{ U \cap V \mid V \in C \}$ for every Cauchy cover $C$.
\item $U \triangleleft_0 \{ V \mid V \rb^s_X U \}$ for every set $U \subseteq X$.
\item $\varnothing \triangleleft_0 \varnothing$.
\end{itemize}

This locale is always regular by construction:
\begin{lem}[cover-locale-regular]
For every precover space $X$, the locale $L(X)$ is regular.
\end{lem}
\begin{proof}
It is enough to check regularity only for generating elements.
Since we know that $[U] \leq \bigvee \{ [V] \mid V \rb^s_X U \}$ for all $U$, it is enough to show that $V \rb^s_X U$ implies $[V] \rb [U]$.
Since $\{ X \backslash V, U \}$ is Cauchy, it is true that $\top = [X \backslash V] \vee [U]$.
Thus, we just need to show that $[X \backslash V] \leq \neg [V]$, which follows from the fact that $[X \backslash V] \wedge [V] = [(X \backslash V) \cap V] = [\varnothing] = \bot$.
\end{proof}

The following lemma gives a useful description of the frame of opens of $L(X)$ in terms of Cauchy covers of $X$:

\begin{lem}[locale-cover]
Let $X$ be a strongly regular cover space and $U$, $U'$, and $V_j$ be subsets of $X$ such that $U' \rb^s_X U$ and $[U] \leq \bigvee_{j \in J} [V_j]$.
Then $\{ X \backslash U'\} \cup \{ V_j \mid j \in J \}$ is a Cauchy cover of $X$.
\end{lem}
\begin{proof}
The proof is by induction on the generation of $U \triangleleft \{ V_j \mid j \in J \}$.
First, let us consider the base case $\triangleleft_0$:
\begin{itemize}
\item If $\{ V_j \mid j \in J \} = \{ U \cap V \mid V \in C \}$ for a Cauchy cover $C$, then $\{ (X \backslash U') \cap V \mid V \in C \} \cup \{ U \cap V \mid V \in C \}$ is a Cauchy cover by \axref{CG} and this cover refines $\{ X \backslash U' \} \cup \{ U \cap V \mid V \in C \}$.
\item Let $\{ V_j \mid j \in J \} = \{ V \mid V \rb^s_X U \}$. Then $\{ X \backslash U' \} \cup \{ V \mid V \rb^s_X U \}$ is Cauchy since $X$ is strongly regular.
\item If $U = \varnothing$ and $\{ V_j \mid j \in J \} = \{ V \mid V \rb^s_X U \} = \varnothing$, then $U'$ is also equal to $\varnothing$ and $\{ X \backslash U' \}$ is Cauchy by \axref{CT}.
\end{itemize}
The only non-trivial remaining case is the transitivity rule.
Suppose that $U \triangleleft \{ W_i \mid i \in I \}$ and $W_i \triangleleft \{ V_j \mid j \in J \}$ for every $i \in I$.
By induction hypothesis, $\{ X \backslash U'\} \cup \{ W_i \mid i \in I \}$ is Cauchy.
Since $X$ is strongly regular, $C = \{ X \backslash U'\} \cup \{ W \mid \exists i \in I, W \rb^s_X W_i \}$ is also Cauchy.
For every $W \in C$, we define $D_W$ as the following set:
\[ \{ V \mid W = X \backslash U' \textrm{ or } \exists i \in I, W \rb^s_X W_i, V \in \{ X \backslash W \} \cup \{ V_j \mid j \in J \} \}. \]

This set is a Cauchy cover for every $W \in C$.
Indeed, if $W = X \backslash U'$, then $D_W$ is Cauchy since it refines $\{ X \}$.
Otherwise, there exists $i \in I$ such that $W \rb^s_X W_i$.
Then $D_W$ is Cauchy by induction hypothesis.

By \axref{CG}, the set $\{ W \cap V \mid W \in C, V \in D_W \}$ is a Cauchy cover.
Let us show that it refines $\{ X \backslash U'\} \cup \{ V_j \mid j \in J \}$.
If $W = X \backslash U'$, then $W \cap V \subseteq X \backslash U'$.
Otherwise, there exists $i \in I$ such that $W \rb^s_X W_i$.
Then either $V = X \backslash W$ or $V = V_j$ for some $j \in J$.
In the former case, $W \cap V = \varnothing \subseteq X \backslash U'$.
In the latter case, $W \cap V \subseteq V_j$.
This shows that $\{ X \backslash U'\} \cup \{ V_j \mid j \in J \}$ is a Cauchy cover, which concludes the proof.
\end{proof}

The following two lemmas easily follows from the previous one:
\begin{lem}[locale-top-cover]
If $X$ is a strongly regular cover space with a weakly dense set of points, then a set $C$ is a Cauchy cover of $X$ if and only if $\bigvee \{ [U] \mid U \in C \} = \top$.
\end{lem}
\begin{proof}
The ``only if'' direction is obvious by the definition of $L(X)$.
Conversely, if $\bigvee \{ [U] \mid U \in C \} = \top$, then, by \rlem{locale-cover}, the set $C \cup \{ \varnothing \}$ is a Cauchy cover.
Since $X$ has a weakly dense set of points, $C$ is also a Cauchy cover.
\end{proof}

\begin{lem}[locale-point-cover]
If $U$ is a neighborhood of a point in a strongly regular cover space and $[U] \leq \bigvee_{j \in J} [V_j]$, then $V_j$ is a neighborhood of $x$ for some $j \in J$.
\end{lem}
\begin{proof}
\rlem{rbs-point} implies that $\{ x \} \rb^s U$.
\rlem{locale-cover} and \axref{CR} imply that $\{ X \backslash \{ x \} \} \cup \{ V' \mid \exists j \in J, V' \rb V \}$ is a Cauchy cover.
Since $x$ does not belong to $X \backslash \{ x \}$, it belongs to some $V'$ such that $V' \rb V_j$ for some $j \in J$.
Thus, $V_j$ is a neighborhood of $x$.
\end{proof}

Finally, we are ready to prove the main result of this section:

\begin{thm}[locale-equiv]
The functor $P : \cat{Loc} \to \cat{Precov}$ restricts to an equivalence between the category of regular locales with a strongly dense set of points and the category of complete strongly regular cover spaces with a dense set of points.
\end{thm}
\begin{proof}
By \rprop{locale-ff}, \rprop{locale-regular}, and \rprop{locale-cover-dense}, this functor restricts to a fully faithful functor between the required categories.
We just need to show that it is essentially surjective on objects.
Let $X$ be a strongly regular cover space with a dense set of points.
Then we define a function $\eta_X : X \to P(L(X))$ as follows: $\eta_X(x) = \{ C \mid \exists U, \{ x \} \rb_X U, [U] \leq C \}$.
Clearly, $\eta_X(x)$ is a filter.
To show that it is completely prime, we only need to consider joins of the form $\bigvee_{j \in J} [V_j]$ since elements of the form $[U]$ generate $L(X)$.
Let $U$ be a neighborhood of $x$ such that $[U] \leq \bigvee_{j \in J} [V_j]$ and $\bigvee_{j \in J} [V_j]$ belongs to $\eta_X(x)$.
By \rlem{locale-point-cover}, $V_j$ is a neighborhood of $x$ for some $j \in J$.
Thus, $[V_j]$ belongs to $\eta_X(x)$, so $\eta_X(x)$ is a completely prime filter.

By \rlem{cover-locale-regular}, $L(X)$ is regular.
Now, we can show that it also has a strongly dense set of points.
It is enough to show that $[U] \leq \bigvee \{ \top \mid \exists x \in \varepsilon^*([U]) \}$.
Since $[U] \leq \bigvee \{ [V] \mid V \rb^s U \}$, it is enough to show that $[V] \leq \bigvee \{ \top \mid \exists x \in \varepsilon^*([U]) \}$ for all $V \rb^s U$.
Since $X$ is strongly dense, the set $\{ X \backslash V \} \cup \{ U' \mid U' \rb U, \exists x \in U' \}$ is a Cauchy cover.
It follows that $[V] \leq \bigvee \{ [U] \mid \exists x \in X, \{ x \} \rb U \}$.
Now, if $\{ x \} \rb U$, then $\eta_X(x) \in \varepsilon^*([U])$, which concludes the proof that $L(X)$ has a strongly dense set of points.

Now, we just need to prove that $\eta_X$ is a dense embedding.
First, we need to show that it is a cover map.
Let $C$ be a Cauchy cover of $P(L(X))$.
By definition, this means that $\top = \bigvee \{ \varepsilon_*(W) \mid W \in C \} = \bigvee \{ [U] \mid \exists W \in C, \varepsilon^*([U]) \subseteq W \}$.
\rlem{locale-top-cover} and \axref{CR} imply that the set $\{ V \mid \exists U, V \rb U, \exists W \in C, \varepsilon^*([U]) \subseteq W \}$ is a Cauchy cover of $X$.
We need to show that this cover refines $\{ \eta_X^{-1}(W) \mid W \in C \}$.
Let $V$, $U$, and $W$ be sets such that $V \rb U$, $W \in C$, and $\varepsilon^*([U]) \subseteq W$.
If $x$ is a point in $V$, then $U$ is a neighborhood of $x$, so $[U]$ belongs to $\eta_X(x)$.
It follows that $\eta_X(x)$ belongs to $W$.
Thus, $V \subseteq \eta_X^{-1}(W)$.
This completes the proof that $\eta_X$ is a cover map.

Let us prove that $\eta_X$ is dense.
If $\{ y \} \rb_{P(L(X))} U$, then $\bigvee \{ [V] \mid \exists W, \varepsilon^*([V]) \subseteq W, (y \in W \implies W \subseteq U) \} = \top$.
Since $X$ has a strongly dense set of points, \rlem{locale-top-cover} implies that $\{ V' \mid V' \rb_X V, \exists x \in V', \exists W, \varepsilon^*([V]) \subseteq W, (y \in W \implies W \subseteq U) \}$ is a Cauchy cover of $X$.
Since $y$ is a completely prime filter, there exist sets $V'$, $V$, $W$ and a point $x \in V'$ such that $V' \rb_X V$, $\varepsilon^*([V]) \subseteq W$, and $W \subseteq U$.
Since $V$ is a neighborhood of $x$, we get that $\eta_X(x) \in \varepsilon^*([V])$.
It follows that $\eta_X(x)$ belongs to $U$.
Thus, $\eta_X$ is dense.

Let us prove that $\eta_X$ is an embedding.
We need to show that, for every Cauchy cover $C$ of $X$, the set $\{ V \mid \exists U \in C, \eta_X^{-1}(V) \subseteq U \}$ is a Cauchy cover of $P(L(X))$.
By \rlem{locale-top-cover}, this set is a Cauchy cover if and only if $\{ W \mid \exists U \in C, \exists V, \varepsilon^*([W]) \subseteq V, \eta_X^{-1}(V) \subseteq U \}$ is a Cauchy cover of $X$.
This is true because it is refined by $C$.
Thus, $\eta_X$ is an embedding.

Finally, if $X$ is complete, then \rthm{dense-lift} implies that $\eta_X$ is an isomorphism since it is a dense embedding.
This concludes the proof.
\end{proof}

\section{Real numbers}
\label{sec:reals}

The aim of this section is to study the cover space of real numbers.
We will define this cover space, show that it is complete and corresponds to the locale of real numbers through the equivalence described in the previous section.
We will also show that the field operations on real numbers are cover maps.

The cover space of real numbers $\mathbb{R}$ is defined to be the one induced by the usual Euclidean metric space structure on $\mathbb{R}$.
Since this cover space structure is induced by a metric space structure, its underlying topology is the usual topology on $\mathbb{R}$.
Thus, using the identifiction from \rprop{dedekind-cuts-filters}, a set $U$ is a neighborhood of a Dedekind filter $F$ if and only if there exists an open interval $(a,b) \in F$ such that $(a,b) \subseteq U$.

There is an inclusion $\mathbb{Q} \to \mathbb{R}$ that maps a rational number to its neighborhood filter.
It is easy to see that the cover space structure on $\mathbb{Q}$ induced by the metric space structure is isomorphic to the transferred structure.

\begin{prop}
The cover space of real numbers is the completion of $\mathbb{Q}$.
\end{prop}
\begin{proof}
Since the underlying topology of $\mathbb{R}$ is the usual topological structure induced by the metric, $\mathbb{Q}$ is dense in $\mathbb{R}$.
The inclusion $i : \mathbb{Q} \to \mathbb{R}$ is also an embedding by definition.
Since the underlying cover space of a metric space is always separated, \rlem{complete-part} and \rprop{dedekind-cauchy} imply that $\mathbb{R}$ is complete.
\end{proof}

It is often useful to be able to control of which elements consist Cauchy covers.
To simplify this discussion, we introduce an auxiliary notion.
For a precover space $X$, we will say that a set $B$ of subsets of $X$ \emph{generates} $X$ if every Cauchy cover is refined by a Cauchy cover that consists of elements of $B$.
We can rephrase the property that $X$ has a dense set of points using this new notion.
Indeed, it is clear that $X$ is generated by a set of inhabited sets if and only if it has a dense set of points.
Another useful application of this notion is the following lemma, which also motivates the terminology:

\begin{lem}[locale-gen]
Let $X$ be a precover space generated by a set $B$.
Then $L(X)$ is generated under joins by elements of the form $[b]$ for $b \in B$.
\end{lem}
\begin{proof}
Since $L(X)$ is generated by elements of the form $[U]$ for $U \subseteq X$, it is enough to show that every such element is a join of elements of the form $[b]$ for $b \in B$.
For every $U \subseteq X$, we have $[U] = \bigvee \{ [V] \mid V \rb^s U \}$.
If $V \rb^s U$, then $\{ X \backslash V \} \cup \{ [b] \mid b \in B, b \subseteq U \}$ is a Cauchy cover.
It follows that $[V] \leq \bigvee \{ [b] \mid b \in B, b \subseteq U \}$.
Thus, $[U] = \bigvee \{ [b] \mid b \in B, b \subseteq U \}$, which concludes the proof.
\end{proof}

The following lemma gives us a convenient way of obtaining a generating set for a precover space:

\begin{lem}[subbase-gen]
Let $\mathcal{B}_X$ be a subbase for a precover space $X$.
Then $X$ is generated by finite intersections of sets in $\bigcup \mathcal{B}_X$.
\end{lem}
\begin{proof}
Let $B$ be the set of finite intersections of sets in $\bigcup \mathcal{B}_X$.
We prove by induction on the generation of $\overline{\mathcal{B}_X}$ that every cover $C$ in this set is refined by some Cauchy cover $C' \subseteq B$.
The only non-trivial case is \axref{CG}.
Let $C$ be a set in $\overline{\mathcal{B}_X}$ and $\{ D_U \}_{U \in C}$ be a collection of sets in $\overline{\mathcal{B}_X}$.
We need to show that $E = \{ U \cap V \mid U \in C, V \in D_U \}$ is refined by some Cauchy cover $E' \subseteq B$.
By induction hypothesis, there exists a Cauchy cover $C' \subseteq B$ that refines $C$.
Let $D'_{U'} = \{ V' \mid \exists U \in C, V \in D_U, U' \subseteq U, V' \subseteq V, V' \in B \}$.
Then $D'_{U'}$ is a Cauchy cover for every $U' \in C'$.
If we let $E' = \{ U' \cap V' \mid U' \in C', V' \in D'_{U'} \}$, then $E'$ is a Cauchy cover, $E' \subseteq B$, and $E'$ refines $E$.
Thus, the proof is concluded.
\end{proof}

We can simplify the condition in the previous lemma for completions of $\mathbb{Q}$:

\begin{lem}[rat-gen]
Both $\mathbb{Q}$ and $\mathbb{R}$ are generated by open balls of rational radii with rational centers.
\end{lem}
\begin{proof}
Since $\mathbb{Q}$ is dense in $\mathbb{R}$, \rlem{subbase-gen} implies tht both of these cover spaces are generated by finite intersections of open balls of rational radii with rational centers.
Every such intersection is either empty, the whole space, or an open ball of the specified form.
Thus, every Cauchy cover $C$ is refined by a cover $C'$ that consists of sets of this form.
Then $\{ U \cap B_\varepsilon(x) \mid U \in C', x \in \mathbb{Q}, \overlap{U}{B_\varepsilon(x)} \}$ is a Cauchy cover, refines $C'$, and consists of open balls as required.
\end{proof}

Now, we will show that localic reals correspond to the cover space of real numbers we described.
Recall that the frame of opens of the locale $\mathbb{R}_l$ of real numbers is generated by the set of open intervals of rational numbers together with the empty set under the following coverage:
\begin{itemize}
\item $(a,d) \triangleleft_0 \{ (a,c), (b,d) \}$ for all rational numbers $a < b < c < d$.
\item $(a,d) \triangleleft_0 \{ (b,c) \mid a < b < c < d \}$ for all rational numbers $a < d$.
\item $\varnothing \triangleleft_0 \varnothing$.
\end{itemize}

\begin{prop}
The cover space of real numbers is isomorphic to $P(\mathbb{R}_l)$.
\end{prop}
\begin{proof}
Let us prove that $L(\mathbb{R})$ is isomorphic to $\mathbb{R}_l$.
First, we define $f : L(\mathbb{R}) \to \mathbb{R}_l$ by $f^*((a,b)) = \{ x \in \mathbb{R} \mid a < x < b \}$ and $f^*(\varnothing) = \varnothing$.
It is clear that $f^*$ respects the coverage relation.
To prove that it is an isomorphism, it is enough to show that it is bijective.
It is clearly injective and \rlem{locale-gen} and \rlem{rat-gen} imply that it is surjective.

\rlem{rat-gen} also implies that $\mathbb{R}$ has a dense set of points.
Since $\mathbb{R}$ is also complete and strongly regular, \rthm{locale-equiv} implies that $\mathbb{R} \simeq P(L(\mathbb{R})) \simeq P(\mathbb{R}_l)$.
\end{proof}

This proposition implies that there is a bijection between localic maps $M \to \mathbb{R}_l$ and cover maps $P(M) \to \mathbb{R}$ for regular locales $M$ that have a dense set of points.
Thus, this proposition provides a convenient way of constructing localic real-valued maps.
In general, it might be difficult to check that a function is a cover map, but for uniform spaces there are simpler conditions that we can check.
One of them is \emph{local uniformness}:

\begin{defn}
Let $X$ and $Y$ be uniform spaces.
Then a function $f : X \to Y$ is \emph{locally uniform} if, for every $E \in \mathcal{U}_Y$, there exists $C \in \mathcal{U}_X$ such that, for every $U \in C$, the set $\{ V \mid \exists W \in E, U \cap V \subseteq f^{-1}(W) \}$ is a uniform cover of $X$.
\end{defn}

Unfolding the definition of uniform covers of metric spaces we obtain the following proposition:

\begin{prop}
Let $X$ and $Y$ be metric spaces.
Then a function $f : X \to Y$ is locally uniform if and only if, for every $\varepsilon > 0$, there exists $\delta_1 > 0$ such that,
for every $x_1 \in X$, there exists $\delta_2 > 0$ such that, for every $x_2 \in X$, there exists $y \in Y$ such that $B_{\delta_1}(x_1) \cap B_{\delta_2}(x_2) \subseteq f^{-1}(B_\varepsilon(y))$.
\end{prop}

The following proposition shows that every locally uniform function is a cover map and we will see in the following section that for certain uniform spaces the converse is also true.

\begin{prop}[locally-uniform-cover-map]
Every locally uniform function between metric spaces is a cover map.
\end{prop}
\begin{proof}
Let $f : X \to Y$ be a locally uniform function.
It is enough to show that, for every $E \in \mathcal{U}_Y$, the cover $\{ f^{-1}(W) \mid W \in E \}$ is Cauchy.
Let $E$ be a uniform cover of $Y$.
Then we have a uniform cover $C \in \mathcal{U}_X$ such that $D_U = \{ V \mid \exists W \in E, U \cap V \subseteq f^{-1}(W) \}$ is a uniform cover of $X$ for every $U \in C$.
Since the Cauchy cover $\{ U \cap V \mid U \in C, V \in D_U \}$ refines $\{ f^{-1}(W) \mid W \in E \}$, this completes the proof.
\end{proof}

Finally, we will show that the field operations on $\mathbb{R}$ are cover maps.
First, we need to show that dense embeddings are closed under products:

\begin{lem}
Let $X$ and $X'$ be precover spaces, $Y$ and $Y'$ be cover spaces, and $f : X \to Y$ and $g : X' \to Y'$ be cover maps.
If $f$ and $g$ are dense embeddings, then so is $f \times g : X \times X' \to Y \times Y'$.
\end{lem}
\begin{proof}
Since embeddings are closed under products by \rlem{prod-embedding}, it is enough to show that $f \times g$ is dense whenever $f$ and $g$ are.
Let $(y,y')$ be a point in $Y \times Y'$ and let $U$ be its neighborhood.
By \rlem{rb-point}, there exists a neighborhood $V$ of $(y,y')$ such that $V \rb U$.
By applying \rlem{cover-map-rb} to the map $\langle \mathrm{id}_Y, \mathrm{const}(y') \rangle$, we get that $\{ y_1 \mid (y_1,y') \in V \}$ is a neighborhood of $y$.
Since $f$ is dense, there exists a point $x \in X$ such that $(f(x),y')$ belongs to $V$.
It follows that $U$ is a neighborhood of $(f(x),y')$.
Now, if apply \rlem{cover-map-rb} to the map $\langle \mathrm{const}(f(x)), \mathrm{id}_{Y'} \rangle$ and this neighborhood,
then we get that $\{ y_2 \mid (f(x),y_2) \in U \}$ is a neighborhood of $(f(x),y')$.
Since $g$ is dense, there exists a point $x' \in X'$ such that $(f(x),f(x'))$ belongs to $U$.
Thus, $f \times g$ is indeed dense.
\end{proof}

This lemma implies that we can extend cover maps $\mathbb{Q}^n \to \mathbb{R}$ to cover maps $\mathbb{R}^n \to \mathbb{R}$.
Since the ring operations on $\mathbb{Q}$ are locally uniform, \rprop{locally-uniform-cover-map} implies that they extend to cover maps on $\mathbb{R}$.
By uniqueness of extensions, these maps satisfy the ring axioms.

Now, we just need to show that the inverse function is a cover map.
Let $\mathbb{R}_*$ be the union of negative and positive real numbers.
It is well-known that a real number is invertible if and only if it belongs to $\mathbb{R}_*$.
We want to show that the function $(-)^{-1} : \mathbb{R}_* \to \mathbb{R}$ is a cover map.

If we put the transferred cover space structure on $\mathbb{R}_*$, then this map is certainly not a cover map.
The problem is that $\mathbb{R}_*$ is incomplete with this cover space structure and its completion is the whole $\mathbb{R}$,
so if we could define such a cover map, it would extend to the whole $\mathbb{R}$, which is impossible.
Thus, we need to put another cover space structure on $\mathbb{R}_*$.
We will give a more general construction that applies to any open subset of a complete cover space.
Let $S$ be an open subset of a cover space $X$.
Then we define the following subbase on $S$:
\[ \mathcal{B}_S = \{ \{ V \cap S \mid V \in D \} \mid D \in \mathcal{C}_X \} \cup \{ \{ V \mid V \rb_X V_1 \rb_X \ldots \rb_X V_n \rb_X S \} \mid n \in \mathbb{N} \}. \]

\begin{lem}
Let $S$ be an open subset of a cover space $X$.
Then $\mathcal{B}_S$ is a regular subbase on $S$.
\end{lem}
\begin{proof}
First, note that $\mathcal{B}_S$ consists of covers.
For sets of the form $\{ V \cap S \mid V \in D \}$, this follows from the fact that $D$ is a cover.
For sets of the form $\{ V \mid V = V_1 \rb_X \ldots \rb_X V_n = S \}$, this follows from \rlem{rb-point} and the fact that $S$ is open.
The fact that this subbase is regular is obvious.
\end{proof}

The following lemma shows that neighborhoods in $(S,\overline{\mathcal{B}_X})$ are the same as in $X$:

\begin{lem}[subspace-neighborhood]
Let $S$ be an open subset of a cover space $X$.
If $x$ is a point in $S$ and $U$ is a subset of $S$, then $U$ is a neighborhood of $x$ in $S$ if and only if it is a neighborhood of $x$ in $X$.
\end{lem}
\begin{proof}
Since the inclusion function $S \to X$ is a cover map, \rlem{cover-map-rb} implies the ``if'' direction.
Let us prove the converse.
First, note that $F_x = \{ W \subseteq S \mid \{ x \} \rb_X W \}$ is a filter on $S$.
It is easy to see that this filter intersects every element of $\mathcal{B}_X$ using \rlem{rb-point} and the fact that $S$ is open.
\rlem{closure-filter} implies that $F_x$ also intersects every element of $\overline{\mathcal{B}_X}$.
If $U$ is a neighborhood of $x$ in $S$, then $\{ W \mid x \in W \implies W \subseteq U \}$ belongs to $\overline{\mathcal{B}_X}$.
Thus, there exists a set $W$ such that $\{ x \} \rb_X W$ and $W \subseteq U$ and, hence, $U$ is a neighborhood of $x$ in $X$.
\end{proof}

Now, we can show that $(S,\overline{\mathcal{B}_S})$ is complete whenever $X$ is:

\begin{prop}
Let $S$ be an open subset of a complete cover space $X$.
Then $(S,\overline{\mathcal{B}_S})$ is a complete cover space.
\end{prop}
\begin{proof}
Since the inclusion function $S \to X$ is a cover map, \rprop{embedding-injective} implies that $S$ is separated.
Let us show that $S$ is complete.
If $F$ is a Cauchy filter on $S$, then $G = \{ V \mid V \cap S \in F \}$ is a Cauchy filter on $X$.
By \rlem{subspace-neighborhood}, it is enough to show that $G^\vee$ belongs to $S$.
Since $\{ V \mid V \rb_X S \}$ is a Cauchy cover of $S$, there exists $V \in F$ such that $V \rb_X S$.
Now, \rlem{filter-point-char} implies that $S$ is a neighborhood of $G^\vee$, which concludes the proof.
\end{proof}

Now, we can return to the problem of showing that the inverse function $\mathbb{R}_* \to \mathbb{R}_*$ is a cover map.
To do that, we will assume that $\mathbb{R}_*$ is endowed with the cover space structure as described above.
We can give a simpler description of this structure for $\mathbb{R}_*$:

\begin{lem}
Let $V$ be a subset of $\mathbb{R}$.
Then the following conditions are equivalent:
\begin{enumerate}
\item \label{it:inv-srb} $V \rb^s_\mathbb{R} \mathbb{R}_*$.
\item \label{it:inv-rb} $V \rb_\mathbb{R} \mathbb{R}_*$.
\item \label{it:inv-char} There exists a rational $\varepsilon > 0$ such that $V$ does not intersect $B_\varepsilon(0)$.
\end{enumerate}
\end{lem}
\begin{proof}
Clearly, \eqref{it:inv-srb} implies \eqref{it:inv-rb}.
Let us show that \eqref{it:inv-rb} implies \eqref{it:inv-char}.
If $V \rb_\mathbb{R} \mathbb{R}_*$, then $\{ W' \mid W' \rb_\mathbb{R} W, \overlap{W}{V} \implies W \subseteq \mathbb{R}_* \}$ is a cover.
Thus, there exists a neighborhood $W$ of $0$ such that the implication $\overlap{W}{V} \implies W \subseteq \mathbb{R}_*$ holds.
Since $0$ does not belong to $\mathbb{R}_*$, this neighborhood does not intersect $V$.
So, we showed that there is an open ball with center in $0$ that does not intersect $V$.

Finally, let us show that \eqref{it:inv-char} implies \eqref{it:inv-srb}.
To do this, it is enough to show that the Cauchy cover $\{ B_{\varepsilon/4}(x) \mid x \in \mathbb{R} \}$ refines $\{ \mathbb{R} \backslash V, \mathbb{R}_* \}$.
Let $x$ be a real number.
By \axref{DS}, either $-\varepsilon/2 < x < \varepsilon/2$, or $x < -\varepsilon/4$, or $\varepsilon/4 < x$.
If $-\varepsilon/2 < x < \varepsilon/2$, then $B_{\varepsilon/4}(x) \subseteq \mathbb{R} \backslash V$.
If $x < -\varepsilon/4$ or $\varepsilon/4 < x$, then $B_{\varepsilon/4}(x) \subseteq \mathbb{R}_*$.
This completes the proof.
\end{proof}

\begin{prop}
The cover space structure on $\mathbb{R}_*$ is generated by the following subbase:
\[ \mathcal{B}_{\mathbb{R}_*} = \{ \{ B_\varepsilon(x) \cap \mathbb{R}_* \mid x \in \mathbb{Q}, x \neq 0 \} \mid \varepsilon > 0 \} \cup \{ \{ (- \infty, - \varepsilon) \cup (\varepsilon, \infty) \mid \varepsilon > 0 \} \}. \]
\end{prop}
\begin{proof}
The first set in this union generates the transferred cover space structure on $\mathbb{R}_*$ and the second one generates the same covers as $\{ \{ V \mid V \rb_\mathbb{R} V_1 \rb_\mathbb{R} \ldots \rb_\mathbb{R} V_n \rb_\mathbb{R} \mathbb{R}_* \} \mid n \in \mathbb{N} \}$ by the previous lemma.
Thus, this subbase generates the same set as the one described above for a general open set.
\end{proof}

Now, we are ready to prove that the inverse function is a cover map:

\begin{prop}
The inverse function $\mathbb{R}_* \to \mathbb{R}_*$ is a cover map.
\end{prop}
\begin{proof}
We will denote this function by $e : \mathbb{R}_* \to \mathbb{R}_*$.
First, note that the cover $\{ e^{-1}((- \infty, - \varepsilon) \cup (\varepsilon, \infty)) \mid \varepsilon > 0 \}$ is refined by the uniform cover consisting of open balls of radius $1$.
Thus, we just need to prove that $\{ e^{-1}(B_\varepsilon(y) \cap \mathbb{R}_*) \mid y \in \mathbb{Q}, y \neq 0 \}$ is a Cauchy cover for every $\varepsilon > 0$.
We will show that it is refined by the Cauchy cover $\{ ((- \infty, - \delta) \cup (\delta, \infty)) \cap B_{\frac{\varepsilon \delta^2}{1 + \varepsilon \delta}}(x) \mid \delta > 0, x \in \mathbb{Q}, x \neq 0 \}$.
To do this, it is enough to show that $((- \infty, - \delta) \cup (\delta, \infty)) \cap B_{\frac{\varepsilon \delta^2}{1 + \varepsilon \delta}}(x) \subseteq e^{-1}(B_\varepsilon(\frac{1}{x}) \cap \mathbb{R}_*)$.
Let $z$ be a real number such that $\abs{z} > \delta$ and $\abs{z - x} < \frac{\varepsilon \delta^2}{1 + \varepsilon \delta}$.
We need to show that $\abs{\frac{1}{z} - \frac{1}{x}} < \varepsilon$.
First, note that $\abs{x} \geq \abs{z} - \abs{z - x} > \delta - \frac{\varepsilon \delta^2}{1 + \varepsilon \delta} = \frac{\delta}{1 + \varepsilon \delta}$.
Now, we can finish the proof: $\abs{\frac{1}{z} - \frac{1}{x}} = \frac{\abs{z - x}}{\abs{z} \abs{x}} < \varepsilon$.
\end{proof}

\section{Compactness}

In this section, we define compact and totally bounded spaces and use these notions to give a characterization of cover maps for certain uniform spaces.

We will say that a cover space is \emph{totally bounded} if, for every Cauchy cover $C$, there exists a Kuratowski-finite Cauchy subcover of $C$.
Explicitly, this means that there exist sets $U_1, \ldots U_n \in C$ such that $\{ U_1, \ldots U_n \}$ is still a Cauchy cover.
A cover space is called \emph{compact} if it is complete and totally bounded.

If $X$ is a totally bounded uniform space, it is often true that all Cauchy covers of $X$ are uniform.
To prove this, we need an additional assumption on $X$ which is called \emph{properness}.
We will say that a uniform space $X$ is \emph{proper} if, for every uniform cover $C$ of $X$, the subset of $C$ consisting of inhabitted sets is also a uniform cover of $X$.
Some authors include this condition in the definition of uniform spaces % TODO: Give reference
, but doing so has some unfortunate consequences.
For example, the transferred uniform structure might not be proper even if the original is.
There is always a universal proper transferred structure, but its underlying cover structure might differ from the transferred cover structure.
To avoid these problems, we do not include properness in the definition of uniform spaces.

Before we proceed, we need to give a few definitions and prove a simple lemma.
If $V$ and $U$ are subsets of a uniform space, we will say that $V$ is \emph{uniformly rather below} $U$,
written $V \rb^* U$, if the set $\{ W \mid \overlap{W}{V} \implies W \subseteq U \}$ is a uniform cover of $X$.
Clearly, $V \rb^* U$ implies $V \rb U$.
Also, this relation satisfies the usual properties of the ``rather below'' relation given in \rprop{rb-props}.

\begin{lem}[star-cauchy-reg]
If $X$ is a unifor space and $C$ is a Cauchy cover of $X$, then the set $\{ V \mid \exists U \in C, V \rb^* U \}$ is also a Cauchy cover of $X$.
\end{lem}
\begin{proof}
The proof is the same as the proof of \rprop{subbase-regular} using the fact that the required property holds for uniform covers $C$ by \axref{UU}.
\end{proof}

Now, we are ready to prove that Cauchy covers are uniform in proper totally bounded uniform spaces:

\begin{prop}[tb-cauchy]
Let $X$ be a proper totally bounded uniform space.
Then every Cauchy cover of $X$ is uniform.
\end{prop}
\begin{proof}
Let $C$ be a Cauchy cover.
By \rlem{star-cauchy-reg}, the set $\{ V \mid \exists U \in C, V \rb^* U \}$ is also a Cauchy cover.
Since $X$ is totally bounded, there exist sets $V_1, \ldots V_n$ and sets $U_1, \ldots U_n \in C$ such that $\{ V_1, \ldots V_n \}$ is a Cauchy cover and $V_i \rb^* U_i$ for all $i$.
By \axref{UI} and properness of $X$, the set $E = \{ W_1 \cap \ldots \cap W_n \mid (\forall i, \overlap{W_i}{V_i} \implies W_i \subseteq U_i), \exists x \in \bigcap_i W_i \}$ is a uniform cover.
Let us prove that $E$ refines $C$.

Let $W_1, \ldots W_n$ be sets such that $\overlap{W_i}{V_i} \implies W_i \subseteq U_i$ for all $i$ and there exists a point $x \in \bigcap_i W_i$.
Since $\{ V_1, \ldots V_n \}$ is a cover, there exists $i$ such that $x \in V_i$.
It follows that $W_1 \cap \ldots \cap W_n \subseteq W_i \subseteq U_i \in C$, which shows that $E$ refines $C$.
\end{proof}

Since the properness condition might be difficult to check, we will prove a version of the previous proposition that applies to weakly proper and strongly regular uniform spaces.
We will say that a uniform space $X$ is \emph{weakly proper} if a set $C$ is a uniform cover of $X$ whenever $C \cup \{ \varnothing \}$ is.
Clearly, every proper uniform space is weakly proper.
Also, every inhabitted uniform space is weakly proper since $C \cup \{ \varnothing \}$ refines $C$ in this case.
We will say that $X$ is strongly regular if, for every uniform cover $C$, the set $\{ V \mid \exists U \in C, V \rb^{*s} U \}$ is also a uniform cover,
where $V \rb^{*s} U$ if and only if $\{ X \backslash V, U \}$ is a uniform cover.

\begin{example}
Every metric space is proper strongly regular uniform space.
\end{example}

Now, we can show a version of \rprop{tb-cauchy} for weakly proper strongly regular uniform spaces:

\begin{prop}[tb-ws-cauchy]
Let $X$ be a weakly proper strongly regular totally bounded uniform space.
Then every Cauchy cover of $X$ is uniform.
\end{prop}
\begin{proof}
Let $C$ be a Cauchy cover.
An argument similar to \rlem{star-cauchy-reg} shows that the set $\{ V \mid \exists U \in C, V \rb^{*s} U \}$ is also a Cauchy cover.
Since $X$ is totally bounded, there exist sets $V_1, \ldots V_n$ and sets $U_1, \ldots U_n \in C$ such that $\{ V_1, \ldots V_n \}$ is a Cauchy cover and $V_i \rb^{*s} U_i$ for all $i$.
By \axref{UI}, the set $E = \{ W_1 \cap \ldots \cap W_n \mid \forall i, W_i \in \{ X \backslash V_i, U_i \} \}$ is a uniform cover.
Since $X$ is weakly proper, it is enough to show that $E$ refines $C \cup \{ \varnothing \}$.
If $W_1 \cap \ldots \cap W_n \in E$, then either $W_i = X \backslash V_i$ for every $i$ or there exists $i$ such that $W_i = U_i$.
In the former case, $W_i \cap \ldots \cap W_n = \varnothing$ since $\{ V_1, \ldots V_n \}$ is a cover.
In the latter case, $W_i \cap \ldots \cap W_n \subseteq W_i = U_i \in C$, which completes the proof.
\end{proof}

The following lemma can be used to show that certain uniform spaces are totally bounded:

\begin{lem}[tb-aux]
Let $X$ be a uniform space such that, for every uniform cover $C$ of $X$, there exists a Kuratowski-finite Cauchy subcover of $C$.
Then $X$ is totally bounded.
\end{lem}
\begin{proof}
We prove by induction on construction of a cover $C \in \overline{\mathcal{U}_X}$ that there exists a Kuratowski-finite Cauchy subcover of $C$.
If $C \in \mathcal{U}_X$, then this is true by assumption.
If $C = \{ X \}$, then $C$ is itself a finite Cauchy cover.
If $C$ is refined by some $D \in \overline{\mathcal{U}_X}$, then there exist sets $V_1, \ldots V_n \in D$ such that $\{ V_1, \ldots V_n \}$ is a Cauchy cover.
By the finite axiom of choice, there exist sets $U_1, \ldots U_n \in C$ such that $V_i \subseteq U_i$ for all $i$.
It follows that $\{ U_1, \ldots U_n \}$ is a Cauchy cover since it is refined by $\{ V_1, \ldots V_n \}$.

Finally, suppose that $C = \{ V \cap W \mid V \in D, W \in D_V \}$, where $D \in \overline{\mathcal{U}_X}$ and $E_V \in \overline{\mathcal{U}_X}$ for every $V \in D$.
Then there exist sets $V_1, \ldots V_n \in D$ such that $\{ V_1, \ldots V_n \}$ is a Cauchy cover.
By the finite axiom of choice, we also have sets $W_{i,1}, \ldots W_{i,m_i} \in E_{V_i}$ such that $\{ W_{i,1}, \ldots W_{i,m_i} \}$ is a Cauchy cover.
Then the common refinement of all of these covers together with the cover $\{ V_1, \ldots V_n \}$ is a Cauchy cover that refines $C$.
Thus, $C$ is also a Cauchy cover, which completes the proof.
\end{proof}

The following proposition gives a useful characterization of totally bounded uniform spaces under appropriate assumptions:

\begin{prop}[tb-covers]
Let $X$ be a uniform space.
If $X$ is either proper or weakly proper and strongly regular, then the following conditions are equivalent:
\begin{enumerate}
\item \label{it:tb-tb} $X$ is totally bounded.
\item \label{it:tb-uni-uni} For every uniform cover $C$ of $X$, there exists a Kuratowski-finite uniform subcover of $C$.
\item \label{it:tb-uni-cauchy} For every uniform cover $C$ of $X$, there exists a Kuratowski-finite Cauchy subcover of $C$.
\end{enumerate}
\end{prop}
\begin{proof}
By \rprop{tb-cauchy} and \rprop{tb-ws-cauchy}, if $X$ is totally bounded, then Cauchy covers coincide with uniform covers.
Thus, \eqref{it:tb-tb} implies \eqref{it:tb-uni-uni}.
Also, \eqref{it:tb-uni-uni} clearly implies \eqref{it:tb-uni-cauchy}.
\rlem{tb-aux} shows that \eqref{it:tb-uni-cauchy} implies \eqref{it:tb-tb}.
\end{proof}

Now, we will discuss totally bounded subsets of cover spaces.
A subset $S$ of a cover space is \emph{totally bounded} if the transferred cover space structure on $S$ is totally bounded.
The following proposition gives a characterization for totally bounded sets:

\begin{prop}[tb-set-char]
A subset $S$ of a cover space $X$ is totally bounded if and only if, for every Cauchy cover $C$ of $X$, there exist sets $U_1, \ldots U_n \in C$ such that the set $\{ W \mid \exists i, S \cap W \subseteq U_i \}$ is a Cauchy cover of $X$.
\end{prop}
\begin{proof}
A cover $C$ of $S$ is Cauchy if and only if $\{ W \mid \exists U \in C, S \cap W \subseteq U \}$ is a Cauchy cover of $X$.
First, let us prove the ``if'' direction.
Let $C$ be a Cauchy cover of $S$.
Since $\{ W \mid \exists U \in C, S \cap W \subseteq U \}$ is a Cauchy cover of $X$, there exist sets $U_1, \ldots U_n \in C$ and $V_1, \ldots V_n$ such that $S \cap V_i \subseteq U_i$ and the set $\{ W \mid \exists i, S \cap W \subseteq V_i \}$ is a Cauchy cover of $X$.
Since this set is a subset of $\{ W \mid \exists i, S \cap W \subseteq U_i \}$, the latter set is also a Cauchy cover of $X$, which implies that $\{ U_1, \ldots U_n \}$ is a Cauchy cover of $S$.

Now, let us prove the ``only if'' direction.
Suppose that $S$ is totally bounded and let $C$ be a Cauchy cover of $X$.
Since $C$ is a subset of $\{ W \mid \exists U \in C, S \cap W \subseteq U \}$, the latter set is also a Cauchy cover of $X$.
This implies that $C$ is a Cauchy cover of $S$.
Since $S$ is totally bounded, there exist sets $U_1, \ldots U_n \in C$ such that $\{ U_1, \ldots U_n \}$ is a Cauchy cover of $S$.
It follows that $\{ W \mid \exists i, S \cap W \subseteq U_i \}$ is a Cauchy cover of $X$, which completes the proof.
\end{proof}

For uniform spaces, we can give a further characterization of totally bounded sets:

\begin{prop}
A subset $S$ of a uniform space $X$ is totally bounded if and only if, for every uniform cover $C$ of $X$, there exist sets $U_1, \ldots U_n \in C$ such that the set $\{ W \mid \exists i, S \cap W \subseteq U_i \}$ is a Cauchy cover of $X$.
\end{prop}
\begin{proof}
The ``only if'' direction follows from \rprop{tb-set-char}.
To prove the ``if'' direction, we apply \rlem{tb-aux}.
A cover $C$ of $S$ is uniform if and only if $\{ V \mid \exists U \in C, S \cap V \subseteq U \}$ is a uniform cover of $X$.
Let $C$ be a uniform cover of $S$.
By assumption, there exist sets $V_1, \ldots V_n$ and sets $U_1, \ldots U_n \in C$ such that $S \cap V_i \subseteq U_i$ for all $i$ and the set $\{ W \mid \exists i, S \cap W \subseteq V_i \}$ is a Cauchy cover of $X$.
It follows that $\{ U_1, \ldots U_n \}$ is a Cauchy subcover of $C$.
\end{proof}

\begin{cor}[tb-metric]
A subset $S$ of a metric space $X$ is totally bounded if and only if, for every $\varepsilon > 0$, there exist points $x_1, \ldots x_n \in X$ such that the set $\{ W \mid \exists i, S \cap W \subseteq B_\varepsilon(x_i) \}$ is a Cauchy cover of $X$.
\end{cor}

\begin{example}
It is easy to see that every bounded subset of $\mathbb{R}$ satisfies conditions of \rcor{tb-metric}.
Thus, a subset of $\mathbb{R}$ is totally bounded if and only if it is bounded.
\end{example}

The following proposition gives a useful characterization of Cauchy covers in a uniform space under appropriate assumptions:

\begin{prop}
Let $X$ be a strongly regular uniform space and $C$ be a Cauchy cover of $X$ consisting of totally bounded inhabitted sets.
Then a set $E$ is a Cauchy cover of $X$ if and only if the set $\{ V \mid \exists W \in E, U \cap V \subseteq W \}$ is a uniform cover of $X$ for every $U \in C$.
\end{prop}
\begin{proof}
If $\{ V \mid \exists W \in E, U \cap V \subseteq W \}$ is a uniform cover for every $U \in C$, then \axref{CG} implies tht $E$ is a Cauchy cover.
Coversely, suppose that $E$ is a Cauchy cover and let $U$ be a set in $C$.
Since $U$ is inhabitted and $X$ is strongly regular, the transferred uniform structure on $U$ is weakly proper and strongly regular.
By \rprop{tb-covers}, the set $\{ U \cap W \mid W \in E \}$ is a uniform cover of $U$.
This means that the set $\{ V \mid \exists W \in E, U \cap V \subseteq U \cap W \}$ is a uniform cover of $X$, which implies the required property.
\end{proof}

\begin{cor}
Let $X$ be a strongly regular uniform space such that the set of totally bounded inhabitted sets is a uniform cover of $X$.
Then every cover map $X \to Y$ is locally uniform.
\end{cor}

\begin{cor}
Let $X$ be a metric space in which every bounded set is totally bounded.
Then every cover map $X \to Y$ is locally uniform.
\end{cor}

\begin{cor}
A function $\mathbb{R} \to \mathbb{R}$ is a cover map if and only if it is locally uniform.
\end{cor}

We finish this section with a characterization of compact subspaces.
Recall that a subspace of a cover space is closed if it contains all of its limit points.
The following proposition gives a characterization of complete subspaces of a complete cover space:

\begin{prop}
A subspace of a complete cover space is complete if and only if it is closed.
\end{prop}
\begin{proof}
First, suppose that $S$ is a closed subspace of a complete space $X$.
By \rprop{embedding-injective}, $S$ is separated.
Let us prove that it is complete.
Let $F$ be a Cauchy filter on $S$.
Then $G = \{ U \mid U \cap S \in F \}$ is a Cauchy filter on $X$.
Let us show that $G^\vee$ belongs to $S$.
Since $S$ is closed, it is enough to show that $G^\vee$ is a limit point of $S$.
Let $U$ be a neighborhood of $G^\vee$.
By \rlem{filter-point-char}, there exists $V \in G$ such that $V \rb_X U$.
Since $F$ is proper and $V \cap S \in F$, there exists a point $x \in V \cap S$.
Thus, $U$ and $S$ intersect and, hence, $G^\vee$ is a limit point of $S$.

We also need to show that $\{ G^\vee \} \rb_S V$ implies $V \in F$ for all $V$.
Condition $\{ G^\vee \} \rb_S V$ implies that $\{ W \mid \exists W', S \cap W \subseteq W', (G^\vee \in W' \implies W' \subseteq V) \}$ is a Cauchy cover of $X$.
Thus, there exist sets $W$ and $W'$ such that $\{ G^\vee \} \rb_X W$, $S \cap W \subseteq W'$, and $W' \subseteq V$.
\rlem{filter-point-char} implies that there exists $W'' \in G$ such that $W'' \rb_X W'$.
It follows that $V$ belongs to $F$.

Now, let us prove the converse.
Suppose that $S$ is complete and let $x$ be a limit point of $S$.
Then $F = \{ V \mid \exists W, \{ x \} \rb_X W, U \cap S \subseteq V \}$ is a Cauchy filter on $S$.
The fact that $F$ is proper follows from the fact that $x$ is a limit point of $S$ and other properties of Cauchy filters are obvious.
To prove that $x$ belongs to $S$, it is enough to show that $x = F^\vee$.
By \rlem{separated-char}, it is enough to show that any pair of neighborhoods of $x$ and $F^\vee$ intersect.
Let $U$ be a neighborhood of $x$ and $V$ be a neighborhood of $F^\vee$.
\rlem{filter-point-char} implies that $V \in F$, which means that there exists a set $W$ such that $\{ x \} \rb_X W$ and $W \cap S \subseteq V$.
Since $x$ is a limit point of $S$ and $U \cap W$ is a neighborhood of $x$, there exists a point $y \in U \cap W \cap S$.
Thus, $U$ and $V$ intersect, which completes the proof.
\end{proof}

\begin{cor}
A subspace of a complete cover space is compact if and only if it is closed and totally bounded.
\end{cor}

\section{Limits}

Limits of functions are closely related to lifting problem for cover spaces.
In this section, we show how various kinds of limits and are related and can be expressed in terms of liftings.
In general, a kind of limits can be represented by a dense cover map $X \to \widetilde{X}$.
Then a data of a limit in a precover space $Y$ is given by a function $f : X \to Y$.
Then we can say that the limit of $f$ exists if this function extends to a cover map $\widetilde{f} : \widetilde{X} \to Y$.
The limit itself is given by the image of $\widetilde{f}$ at some point in $\widetilde{X}$.

Let us demonstrate how these definitions apply to the case of sequential limits.
First, we define a uniform space structure on $\mathbb{N}$.
A cover $C$ of $\mathbb{N}$ is uniform if there exist $N \in \mathbb{N}$ and $U \in C$ such that $U$ contains every $n \geq N$.
It is easy to check that this indeed defines a uniform space structure on $\mathbb{N}$.

Cauchy filters correspond to so-called extended natural numbers.
An \emph{extended natural number} is a sequence $x$ of binary digits, that is, elements of $\{ 0, 1 \}$, such that $x_n = 1$ for at most one $n \in \mathbb{N}$.
The set of extended natural numbers will be denoted by $\mathbb{N}_\infty$.
There is a map $\mathbb{N} \amalg \{ \infty \} \to \mathbb{N}_\infty$ that maps $n \in \mathbb{N}$ to the sequence $x$ with $x_n = 1$ and maps $\infty$ to the sequence consisting of $0$.
Classically, this map is bijective, but it is only injective constructively.

\begin{prop}
There is a bijection between the set of regular Cauchy filters on $\mathbb{N}$ and the set of extended natural numbers.
\end{prop}
\begin{proof}
A proper filter $F$ on $\mathbb{N}$ is Cauchy if and only if, for every $N \in \mathbb{N}$, either there exists $n < N$ such that $\{ n \} \in F$ or $[N,\infty) \in F$.
Let us show that the relation $\{ n \} \in F$ is decidable.
We proceed by induction on $n$.
If $\{ k \}$ belongs to $F$ for some $k < n$, then $\{ n \}$ does not belong to $F$ since $F$ is proper.
Otherwise, either $\{ n \} \in F$ or $[N + 1, \infty) \in F$.
The latter case implies that $\{ n \} \notin F$.
Thus, we can decide whether $\{ n \}$ belongs to $F$ or not.

Now, we can define a function $s$ from the set of Cauchy filters to $\mathbb{N}_\infty$ as follows: $s(F)_n = 1$ if and only if $\{ n \} \in F$.
Since $\{ n \}$ belongs to $F$ for at most one $n$, this sequence is indeed an extended natural number.
We also define a function $f$ from $\mathbb{N}_\infty$ to the set of Cauchy filters:
\[ f(x) = \{ U \mid (\exists n \in U, x_n = 1) \lor (\exists N \in \mathbb{N}, [N,\infty) \subseteq U, \forall n < N, x_n = 0) \}. \]
It is easy to check that $f(x)$ is indeed a Cauchy filter.

It is also easy to see that $s(f(x)) = x$ and that $f(s(F)) \subseteq F$.
Thus, $f(s(F))$ is equivalent to $F$.
Also, it is clear that $s(F) = s(G)$ whenever $F \subseteq G$, which implies that $s$ respects the equivalence of Cauchy filters.
Thus, we get a bijection between $\mathbb{N}_\infty$ and equivalence classes of Cauchy filters.
Since regular Cauchy filters are unique representatives of their equivalence classes, we also have the required bijection.
\end{proof}

We will say that a sequence $f : \mathbb{N} \to X$ in a precover space $X$ is \emph{Cauchy} if, for every Cauchy cover $C$ of $X$, there exist $N \in \mathbb{N}$ and $U \in C$ such that, for every $n \geq N$, $f_n \in U$.
If $X$ is a metric space, this is indeed equivalent to the usual definition of a Cauchy sequence.
Thus, we get a generalization of Cauchy sequences for arbitrary precover spaces.
Moreover, $f : \mathbb{N} \to X$ is a Cauchy sequence if and only if it is a cover map for the uniform structure on $\mathbb{N}$ we defined above.
If $X$ is a complete cover space, then \rthm{dense-lift} implies that a sequence $f : \mathbb{N} \to X$ is Cauchy if and only if it extends to a cover map $\widetilde{f} : \mathbb{N}_\infty \to X$.
Then the limit of $f$ is defined as $\widetilde{f}(\infty)$.

We can generalize this discussion to arbitrary directed sets which can be used to give a general notion of a limit. % TODO: Give a reference
A \emph{directed set} $I$ is a preordered set in which every finite set has an upper bound.
If $N$ is an element of $I$, we will write $I_{\geq N}$ for the set $\{ n \in I \mid n \geq N \}$.
There is a precover space structure on $I$ in which a cover $C$ is Cauchy if and only if there exist $U \in C$ and $N \in I$ such that $I_{\geq N} \subseteq U$.
It is straightforward to check that this definition satisfies \axref{CT}, \axref{CE}, and \axref{CG}, but it might not satisfy \axref{CR}.
So, in general, we define the cover space structure on $I$ as the reflection of this precover space structure.

\begin{defn}
Let $I$ be a directed set $X$ be a cover space.
We will say that a function $f : I \to X$ \emph{converges} if it is a cover map.
\end{defn}

A \emph{limit} of a function $f : I \to X$ is a point $x \in X$ such that, for every neighborhood $U$ of $x$, there exists $N \in I$ such that $I_{\geq N} \subseteq f^{-1}(U)$.
If a function has a limit, then it convergence.
Conversely, if $X$ is complete and $f$ converges, then it has a limit.
Indeed, the completion $I_\infty$ of $I$ has a point $\infty$ corresponding to the Cauchy cover $\{ U \mid \exists N \in I, \forall n \geq N, n \in U \}$.
If $X$ is complete and $f$ converges, then it extends to a function from the completion of $I$ to $X$ and the limit of $f$ is given by the image of $\infty$.

The following proposition gives an explicit characterization of convergence:

\begin{prop}
Let $I$ be a directed set $X$ be a cover space.
Then a function $f : I \to X$ converges if and only if, for every Cauchy cover $C$ of $X$, there exist $U \in C$ and $N \in I$ such that $I_{\geq N} \subseteq f^{-1}(U)$.
\end{prop}
\begin{proof}
Since $X$ is a cover space, $f$ converges if and only if, for every Cauchy cover $C$ of $X$, the set $\{ f^{-1}(U) \mid U \in C \}$ is a Cauchy cover in the precover space structure on $I$.
Unfolding the definitions, we get the required condition.
\end{proof}

If $X$ is a uniform space, we can simplify the condition in the previous proposition:

\begin{prop}
Let $I$ be a directed set $X$ be a uniform space.
Then a function $f : I \to X$ converges if and only if, for every uniform cover $C$ of $X$, there exist $U \in C$ and $N \in I$ such that $I_{\geq N} \subseteq f^{-1}(U)$.
\end{prop}
\begin{proof}
If $f$ converges, then the required condition holds by the previous proposition.
Conversely, assume that this condition holds and let us prove that $f$ converges.
Let $F = \{ U \mid \exists N \in I, \forall n \geq N, f(n) \in U \}$.
By assumption, this proper filter intersects with every uniform cover.
By \rprop{cauchy-filter}, it intersects with every Cauchy cover, which implies that $f$ is a cover map.
\end{proof}

This gives the usual definition of convergence for metric spaces:

\begin{cor}
Let $I$ be a directed set $X$ be a metric space.
Then a function $f : I \to X$ converges if and only if, for every $\varepsilon > 0$, there exists $N \in I$ such that $d(f(n),f(N)) < \varepsilon$ for all $n \geq N$.
\end{cor}

Now, we will discuss limits of sequences of functions.
It is well-known that the limit of a pointwise convergent sequence of continuous maps may not be continuous.
The same is true for cover maps.
The problem becomes obvious when expressed in terms of the lifting property.
Indeed, a sequence of functions $X \to Y$ can be represented by a function $f : \mathbb{N} \times X \to Y$.
Now, even if $f(n,-)$ is a cover map for all $n \in \mathbb{N}$ and $f(-,x)$ is a cover map for all $x \in X$, this does not imply that $f$ is a cover map.
In other words, if $f$ represents a pointwise convergent sequence of cover maps, $f$ may not be a cover map itself.
So, it is natural to require the corresponding function $\mathbb{N} \times X \to Y$ to be a cover map.
More generally, we might consider cover maps of the form $I \times X \to Y$ for an arbitrary directed set $I$.
As before, if $Y$ is a complete cover space, then $f$ is a cover map if and only if it extends to a map $\widetilde{f} : I_\infty \times X \to Y$.
Then the limit of $f$ is defined as $\widetilde{f}(\infty,-)$.

It will be convenient to have an explicit description of Cauchy covers of $I \times X$ for the following discussion.
The following lemma provides such a description:

\begin{lem}[dir-prod-char]
Let $I$ be a directed set equipped with the precover space structure and $X$ be a cover space.
Then a set $C$ is a Cauchy cover of $I \times X$ if and only if the following conditions hold:
\begin{enumerate}
\item \label{it:dir-char} The set $\{ U' \mid \exists N \in I, \exists U \in C, I_{\geq N} \times U' \subseteq U \}$ is a Cauchy cover of $X$.
\item \label{it:dir-cover} For every $n \in I$, the set $\{ U' \mid \exists U \in C, \forall x \in U', (n,x) \in U \}$ is a Cauchy cover of $X$.
\end{enumerate}
\end{lem}
\begin{proof}
First, let us show that covers satisfying these two conditions form a precover space structure on $I \times X$.
Axioms \axref{CT} and \axref{CE} are obvious, so let us check \axref{CG}.
Let $C$ be a set and $\{ D_U \}_{U \in C}$ be a collection of sets such that $C$ and $D_U$ satisfy conditions \eqref{it:dir-char} and \eqref{it:dir-cover}.
We need to show that $E = \{ U \cap V \mid U \in C, V \in D_U \}$ also satisfies these conditions.
By assumption, $C' = \{ U' \mid \exists N \in I, \exists U \in C, I_{\geq N} \times U' \subseteq U \}$ is a Cauchy cover of $X$.
Let $D'_{U'} = \{ V' \mid \exists N \in I, \exists U \in C, I_{\geq N} \times U' \subseteq U, \exists V \in D_U, I_{\geq N} \times V' \subseteq V \}$.
Clearly, $D'_{U'}$ is a Cauchy cover of $X$ for every $U' \in C'$.
Since $\{ U' \cap V' \mid U' \in C', V' \in D'_{U'} \}$ is a subset of $\{ W' \mid \exists N \in I, \exists W \in E, I_{\geq N} \times W' \subseteq W \}$, the latter set is also a Cauchy cover of $X$.
Thus, \eqref{it:dir-char} holds for $E$.

Let us show that $E$ satisfies \eqref{it:dir-cover}.
Let $n$ be an element of $I$.
Then $C' = \{ U' \mid \exists U \in C, \forall x \in U', (n,x) \in U \}$ is a Cauchy cover of $X$.
Let $D'_{U'} = \{ V' \mid \exists U \in C, \exists V \in D_U, (\forall x \in U', (n,x) \in U), (\forall x \in V', (n,x) \in V)\}$.
Clearly, $D'_{U'}$ is a Cauchy cover of $X$ for every $U' \in C'$.
Since $\{ U' \cap V' \mid U' \in C', V' \in D'_{U'} \}$ is a subset of $\{ W' \mid \exists W \in E, \forall x \in W', (n,x) \in W \}$, the latter set is also a Cauchy cover of $X$.
Thus, \eqref{it:dir-cover} also holds for $E$.

To prove that every Cauchy cover of $I \times X$ satisfies conditions \eqref{it:dir-char} and \eqref{it:dir-cover}, it is enough to consider only basic Cauchy covers,
that is, covers of the form $\{ \pi_1^{-1}(U) \mid U \in C \}$ and $\{ \pi_2^{-1}(V) \mid V \in D \}$ for Cauchy covers $C$ of $I$ and $D$ of $X$.
This is easy for covers of the form $\{ \pi_1^{-1}(U) \mid U \in C \}$ since both $\{ U' \mid \exists N \in I, \exists U \in C, I_{\geq N} \times U' \subseteq \pi_1^{-1}(U) \}$
and $\{ U' \mid \exists U \in C, \forall x \in U', (n,x) \in \pi_1^{-1}(U) \}$ are refined by $\{ \top \}$.
This is also easy for covers of the form $\{ \pi_2^{-1}(V) \mid V \in D \}$ since both $\{ V' \mid \exists N \in I, \exists V \in D, I_{\geq N} \times V' \subseteq \pi_2^{-1}(V) \}$
and $\{ V' \mid \exists V \in D, \forall x \in V', (n,x) \in \pi_2^{-1}(V) \}$ are refined by $D$.

Now, let us prove that every set $C$ satisfying \eqref{it:dir-char} and \eqref{it:dir-cover} is a Cauchy cover of $I \times X$.
Let $D = \{ \pi_2^{-1}(U') \mid U' \in C' \}$, where $C'$ is the Cauchy cover $\{ U' \mid \exists N \in I, \exists U \in C, I_{\geq N} \times U' \subseteq U \}$.
Let $E_V = \{ W \mid \exists U' \subseteq X, \exists N \in I, \exists U \in C, I_{\geq N} \times U' \subseteq U, V = \pi_2^{-1}(U'), (W \subseteq \pi_1^{-1}(I_{\geq N})) \lor (\exists W_1 \in C, W \subseteq W_1) \}$.
Since $\{ V \cap W \mid V \in D, W \in E_V \}$ refines $C$ and $D$ is a Cauchy cover of $I \times X$, it is enough to check that $E_V$ is a Cauchy cover of $I \times X$ for every $V \in D$.
Let $U' \in C'$, $N \in I$, and $U \in C$ be such that $V = \pi_2^{-1}(U')$ and $I_{\geq N} \times U' \subseteq U$.
Let $F = \{ \pi_1^{-1}(S') \mid S' \in F' \}$, where $F' = \{ I_{\geq N} \} \cup \{ \{ n \} \mid n \in I \}$.
Let $G_S = \{ T \mid \exists S' \subseteq I, S = \pi_1^{-1}(S'), (S' = I_{\geq N}) \lor (\exists n \in I, S' = \{ n \}, \exists U_1' \subseteq X, \exists U_1 \in C, \{ n \} \times U_1' \subseteq U_1, T = \pi_2^{-1}(U_1')) \}$.
Since $\{ S \cap T \mid S \in F, T \in G_S \}$ is a subset of $E_V$ and $F$ is a Cauchy cover of $I \times X$, it is enough to check that $G_S$ is a Cauchy cover of $I \times X$ for every $S \in F$.
Let $S' \in F'$ be such that $S = \pi_1^{-1}(S')$.
Then either $S' = I_{\geq N}$ or $S' = \{ n \}$ for some $n \in I$.
In the former case, $G_S$ is Cauchy since it is refined by $\{ \top \}$.
In the latter case, $G_S$ is Cauchy since it is refined by the Cauchy cover $\{ \pi_2^{-1}(U'') \mid U'' \in C'' \}$,
where $C''$ is the Cauchy cover $\{ U'' \mid \exists U \in C, \forall x \in U'', (n,x) \in U \}$.
This completes the proof.
\end{proof}

The following proposition gives a characterization for convergence of functions of the form $I \times X \to Y$:

\begin{prop}[func-conv-char]
Let $I$ be a directed set and $X$ and $Y$ be cover spaces.
Then $f : I \times X \to Y$ is a cover map if and only if $f(n,-)$ is a cover map for all $n \in I$ and, for every Cauchy cover $D$ of $Y$, the set $\{ U \mid \exists N \in I, \exists V \in D, I_{\geq N} \times U \subseteq f^{-1}(V) \}$ is a Cauchy cover of $X$.
\end{prop}
\begin{proof}
Since the reflector preserves products, a function $f : I \times X \to Y$ is a cover map for the cover space structure on $I$ if and only if it is a cover map for the precover space structure on $I$.
Thus, it is a cover map if and only if, for every Cauchy cover $D$ on $Y$, the set $\{ f^{-1}(V) \mid V \in D \}$ satisfies conditions \eqref{it:dir-char} and \eqref{it:dir-cover} of \rlem{dir-prod-char}.
This completes the proof since these conditions are simply a reformulation of conditions in the statement of this proposition.
\end{proof}

If $Y$ is a uniform space, we can improve this characterization:

\begin{prop}
Let $I$ be a directed set, $X$ be a cover space, and $Y$ be a uniform space.
Then $f : I \times X \to Y$ is a cover map if and only if $f(n,-)$ is a cover map for all $n \in I$ and, for every uniform cover $D$ of $Y$,
the set $\{ U \mid \exists N \in I, \forall x \in U, \exists V \in D, \forall n \geq N, f(n,x) \in V \}$ is a Cauchy cover of $X$.
\end{prop}
\begin{proof}
The ``only if'' direction immediately follows from \rprop{func-conv-char}.
Let us prove the converse.
To do this, we will show that, for every uniform cover $E$ of $Y$, there exists a Cauchy cover $C$ of $X$ such that, for every $U \in C$, there exists $N \in I$ and a Cauchy cover $D$ of $X$
such that, for every $V \in D$, there exists $W \in E$ such that $I_{\geq N} \times (U \cap V) \subseteq f^{-1}(W)$.

By \axref{UU}, there exists a uniform cover $E'$ of $Y$ such that, for every $W' \in E'$,
there exists $W \in E$ such that every $W'' \in E'$ satisfies the implication $\overlap{W''}{W'} \implies W'' \subseteq W$.
By assumption, $C = \{ U \mid \exists N \in I, \forall x \in U, \exists W' \in E', \forall n \geq N, f(n,x) \in W' \}$ is a Cauchy cover of $X$.
Let $U$ be an element of $C$.
Then there exists $N \in I$ such that $\forall x \in U, \exists W' \in E', \forall n \geq N, f(n,x) \in W'$.
The set $D = \{ f(N,-)^{-1}(W') \mid W' \in E' \}$ is a Cauchy cover by assumption.
We need to show that, for every $W'$ be an element of $E'$, there exists $W \in E$ such that $I_{\geq N} \times (U \cap f(N,-)^{-1}(W')) \subseteq f^{-1}(W)$.
Let $W$ be an element of $E$ such that every $W'' \in E'$ satisfies the implication $\overlap{W''}{W'} \implies W'' \subseteq W$.
Let $n$ be an element of $I$ such that $n \geq N$ and $x$ be an element of $U$ such that $f(N,x) \in W'$.
Then there exists $W'' \in E'$ such that, for every $n \geq N$, $f(n,x) \in W''$.
Since $f(N,x) \in W'' \cap W'$, we have that $W'' \subseteq W$.
Since $f(n,x) \in W''$, we get that $f(n,x) \in W$, which completes the proof of the intermediate step.

Finally, let us prove that $f : I \times X \to Y$ is a cover map.
It is enough to show that, for every uniform cover $E$ of $Y$, the set $\{ f^{-1}(W) \mid W \in E \}$ is a Cauchy cover of $I \times X$.
By \rlem{dir-prod-char}, we need to show that the set $C' = \{ U \mid \exists N \in I, \exists W \in E, I_{\geq N} \times U \subseteq f^{-1}(W) \}$ is a Cauchy cover of $X$.
Let $C$ be a Cauchy cover of $X$ such that, for every $U \in C$, there exists $N \in I$ and a Cauchy cover $D$ of $X$
such that, for every $V \in D$, there exists $W \in E$ such that $I_{\geq N} \times (U \cap V) \subseteq f^{-1}(W)$.
Then the set $D_U = \{ V \mid \exists N \in I, \exists W \in E, I_{\geq N} \times (U \cap V) \subseteq f^{-1}(W) \}$ is a Cauchy cover for every $U \in C$.
This completes the proof since $C'$ is refined by the Cauchy cover $\{ U \cap V \mid U \in C, V \in D_U \}$.
\end{proof}

\begin{cor}[func-conv-metric-char]
Let $I$ be a directed set, $X$ be a cover space, and $Y$ be a metric space.
Then $f : I \times X \to Y$ is a cover map if and only if $f(n,-)$ is a cover map for all $n \in I$ and, for every $\varepsilon > 0$,
the set $\{ U \mid \exists N \in I, \forall x \in U, \forall n \geq N, d_Y(f(n,x),f(N,x)) < \varepsilon \}$ is a Cauchy cover of $X$.
\end{cor}

If $a_n$ is a sequence of real numbers, we will say that a series $\sum\limits_{n = 0}^\infty a_n$ \emph{converges} if the sequence $s_k = \sum\limits_{n = 0}^k a_n$ of partial sums converges.
If $\sum\limits_{n = 0}^\infty a_n x^n$ converges for all $x$, then standard techniques imply that it converges uniformly on bounded subsets.
Now, \rcor{func-conv-metric-char} implies that the function $x \mapsto \sum\limits_{n = 0}^\infty a_n x^n$ is a cover map in this case.
This can be used to define various cover maps such as the exponential function, sine, cosine, and so forth.

Finally, we will discuss limits of functions of the form $f : \mathbb{R}_* \to \mathbb{R}$.
As usual, we will say that $y$ is a \emph{limit} of $f$ at $0$ if, for every $\varepsilon > 0$,
there exists $\delta > 0$ such that, for every $x \in \mathbb{R}_*$, if $\abs{x} < \delta$, then $\abs{f(x) - y} < \varepsilon$.
The relationship between this notion and a lifting property is given by the following proposition:

\begin{prop}
Let $f : \mathbb{R}_* \to \mathbb{R}$ be a cover map.
Then $f$ has a limit at $0$ if and only if $f$ extends to a cover map $\widetilde{f} : \mathbb{R} \to \mathbb{R}$.
The limit is then equal to $\widetilde{f}(0)$.
\end{prop}
\begin{proof}
Assume that $f$ extends to $\widetilde{f}$.
\rlem{cover-map-rb} implies that, for every $\varepsilon > 0$, the set $\widetilde{f}^{-1}(B_\varepsilon(\widetilde{f}(0)))$ is a neighborhood of $0$.
It follows that $\widetilde{f}(0)$ is a limit of $f$ at $0$.

Now, assume that $L$ is a limit of $f$ at $0$.
Let $\mathbb{R}'_*$ be the set of invertible real number with the transferred cover space structure.
Since it is densely embeds into $\mathbb{R}$, every cover map $\mathbb{R}'_* \to \mathbb{R}$ extends to a unique map $\mathbb{R} \to \mathbb{R}$.
Thus, we just need to show that $f : \mathbb{R}_* \to \mathbb{R}$ is a cover map from $\mathbb{R}'_*$ to $\mathbb{R}$.
That is, we need to show that, for every $\varepsilon > 0$, the set $\{ f^{-1}(B_\varepsilon(x)) \mid x \in \mathbb{Q} \}$ is a Cauchy cover of $\mathbb{R}'_*$.

For every $\varepsilon > 0$, there exists $\delta > 0$ such that $B_\delta(0) \cap \mathbb{R}_* \subseteq f^{-1}(B_\varepsilon(L))$.
Now, endow $V = \left(- \infty, - \frac{\delta}{3} \right) \cup \left(\frac{\delta}{3}, \infty \right)$ with the transferred cover space structure.
It does not matter if we transfer it from $\mathbb{R}_*$ or $\mathbb{R}$ since the result is the same.
Since the composite $V \to \mathbb{R}_* \overset{f}\to \mathbb{R}$ is a cover map,
there exists a Cauchy cover $C$ of $\mathbb{R}$ such that $\{ U \cap V \mid U \in C \}$ refines $\{ f^{-1}(B_\varepsilon(y)) \mid y \in \mathbb{Q} \}$.

Since $\{ U \cap B_{\delta/3}(x) \cap \mathbb{R}_* \mid U \in C, x \in \mathbb{Q} \}$ is a Cauchy cover of $\mathbb{R}'_*$,
it is enough to show that it refines $\{ f^{-1}(B_\varepsilon(x)) \mid x \in \mathbb{Q} \}$.
Let $U$ be a set in $C$ and $x$ be a rational number.
If $\abs{x} \geq \frac{2}{3} \delta$, then $U \cap B_{\delta/3}(x) \cap \mathbb{R}_*$ is a subset of $U \cap V$ and, hence, it is a subset of $f^{-1}(B_\varepsilon(y))$ for some $y \in \mathbb{Q}$.
If $\abs{x} \leq \frac{2}{3} \delta$, then $U \cap B_{\delta/3}(x) \cap \mathbb{R}_* \subseteq B_\delta(0) \cap \mathbb{R}_* \subseteq f^{-1}(B_\varepsilon(L))$.
This completes the proof.
\end{proof}

\bibliographystyle{amsplain}
\bibliography{ref}

\end{document}
