\documentclass{amsart}

\usepackage{etex}
\usepackage{hyperref}
\usepackage{amssymb}
\usepackage[all]{xy}
\usepackage{ifthen}
\usepackage{xargs}
\usepackage{todonotes}

\hypersetup{colorlinks=true,linkcolor=blue}

\author{Valery Isaev}

\title{Model categories in which all objects are fibrant}

\providecommand\WarningsAreErrors{false}
\ifthenelse{\equal{\WarningsAreErrors}{true}}{\renewcommand{\GenericWarning}[2]{\GenericError{#1}{#2}{}{This warning has been turned into a fatal error.}}}{}

\newcommand{\newref}[4][]{
\ifthenelse{\equal{#1}{}}{\newtheorem{h#2}[hthm]{#4}}{\newtheorem{h#2}{#4}[#1]}
\expandafter\newcommand\csname r#2\endcsname[1]{#4~\ref{#2:##1}}
\newenvironmentx{#2}[2][1=,2=]{
\ifthenelse{\equal{##2}{}}{\begin{h#2}}{\begin{h#2}[##2]}
\ifthenelse{\equal{##1}{}}{}{\label{#2:##1}}
}{\end{h#2}}
}

\newref[section]{thm}{theorem}{Theorem}
\newref{lem}{lemma}{Lemma}
\newref{prop}{proposition}{Proposition}
\newref{cor}{corollary}{Corollary}

\theoremstyle{definition}
\newref{defn}{definition}{Definition}

\newcommand{\we}{\mathcal{W}}
\newcommand{\fib}{\mathcal{F}}
\newcommand{\cof}{\mathcal{C}}
\newcommand{\cat}[1]{\mathbf{#1}}
\newcommand{\C}{\cat{C}}

\newcommand{\I}{\mathrm{I}}
\newcommand{\J}{\mathrm{J}}
\newcommand{\class}[2]{#1\text{-}\mathrm{#2}}
\newcommand{\Iinj}[1][\I]{\class{#1}{inj}}
\newcommand{\Icell}[1][\I]{\class{#1}{cell}}
\newcommand{\Icof}[1][\I]{\class{#1}{cof}}
\newcommand{\Jinj}[1][]{\Iinj[\J#1]}
\newcommand{\Jcell}[1][]{\Icell[\J#1]}
\newcommand{\Jcof}[1][]{\Icof[\J#1]}
\newcommand{\cyli}{i}

\newcommand{\po}[1][dr]{\save*!/#1+1.2pc/#1:(1,-1)@^{|-}\restore}

\begin{document}

\begin{abstract}
In this paper, we study properties of model structures in which all objects are fibrant.
We give a few general ways of constructing such model structures on a complete and cocomplete category.
As an application we define a model structure on the category of algebras for contractible globular operads.
\end{abstract}

\maketitle

 \makeatletter
    \providecommand\@dotsep{5}
  \makeatother
  \listoftodos\relax

\section{Introduction}

Model categories, introduced in \cite{quillen}, are an important tool in homotopy theory.
In this paper, we study model categories in which all objects are fibrant.
We call such model categories \emph{f-model categories}.
If we fix the class of cofibrations, then the class of weak equivalences which defines an f-model structure is unique if it exists.
These class of weak equivalences is the minimum possible among model structures with this class of cofibrations.
Thus every other model structures with this class of cofibrations is a left Bousfield localization of an f-model structure (if it exists).
The main result is \rthm{main}, which gives a necessary and sufficient condition for the existence of an f-model structure.

F-model structures can be compared to left-determinant model structures introduced in \cite{left-det}.
Every f-model structure is left-determinant, but the converse is not necessary true.
For example, the category of simplicial sets (as well as every Grothendieck topos \cite{cisinski})
carries a left-determinant model structure with monomorphisms as cofibrations,
but an f-model structure on this category with this class of cofibrations does not exist.
To see this, consider a cylinder $C$ for the terminal object.
Then $C \amalg_{\Delta[0]} C$ does not have a map from $\Delta[1]$ which
maps faces of this simplex to points of $C$ which were not amalgamated.
Thus $C \amalg_{\Delta[0]} C$ cannot be fibrant.

Thus the class of f-model structures is much narrower than the class of left-determinant model structures.
On the other hand f-model structures have properties not shared by left-determinant ones.
For example, giving a set of generating cofibrations we can often give
a simple description of a set of generating trivial cofibrations (see \rcor{min-cof-gen}).
Using this observation we can give a simple description of the class of weak equivalences.

In section 2, we recall definitions and basic properties of weak factorization systems and model categories.
In section 3, we study properties of f-model categories.
In section 4, we demonstrate a method of constructing f-model structures.
In section 5, we define an f-model structure on the category of algebras for contractible globular operads.

\section{Preliminaries}

In this section, we will introduce a few definitions that we will need later.
We will also recall the definition and basic properties of model categories.

\subsection{Cylinders and homotopy relations}

Let $\C$ be a category and let $V$ be an object of $\C$.
Then we a \emph{cylinder object} for $V$ is an object $C(V)$ together with maps $\cyli_0,\cyli_1 : V \to C(V)$.
If $i : U \to V$ is a morphism of $\C$, then a \emph{relative cylinder object} for $\C$
is a cylinder object $(C_U(V),\cyli_0,\cyli_1)$ for $V$ such that $\cyli_0 \circ i = \cyli_1 \circ i$.
If $\C$ has the initial object $0$, then a relative cylinder object for $0 \to V$ is just a cylinder object for $V$.
A morphism of cylinder objects $C(V_1)$ and $C(V_2)$ is a pair of maps $f : V_1 \to V_2$ and $C(f) : C(V_1) \to C(V_2)$ which commute with $\cyli_0$ and $\cyli_1$.

A \emph{homotopy} (with respect to $C(V)$) between maps $f,g : V \to X$ is a map $h : C(V) \to X$ such that $h \circ \cyli_0 = f$ and $h \circ \cyli_1 = g$.
Maps are \emph{homotopic} if there is a homotopy between them.
Maps are \emph{homotopic relative to $i : U \to V$} (with respect to $C_U(V)$) if there is a homotopy with respect to $C_U(V)$ between them.
Note that maps are homotopic relative to $i$ only if $i \circ f = i \circ g$.
If maps $f$ and $g$ are homotopic, then we write $f \sim g$, and if they are homotopic relative to $i$, then we write $f \sim_i g$.

Let $V,Y$ be objects of a category $\C$ and $R$ some relation on the set $\C(V,Y)$.
Given two morphisms $f : U \to V$ and $g : X \to Y$, we say that $f$ \emph{has the left lifting property (LLP) up to $R$} with respect to $g$,
and $g$ \emph{has the right lifting property (RLP) up to $R$} with respect to $f$ if for every commutative square of the form
\[ \xymatrix{ U \ar[r]^u \ar@{}[dr]|(.7){R} \ar[d]_f & X \ar[d]^g \\
              V \ar[r]_v \ar@{-->}[ur]^h             & Y,
            } \]
there is a dotted arrow $h : V \to X$ such that $h \circ f = u$ and $(g \circ h) R v$.
We say a map $f$ has the right lifting property up to $R$ with respect to
an object $V$ if it has this property with respect to the map $0 \to V$.
Given a morphism $f$ and a set of morphisms $\I$, $f$ has the left (right) lifting property up to $R$
with respect to $\I$ if it has this property with respect to all morphisms in $\I$.
A map has the right (left) lifting property if it has this property up to the equality relation.

Let $R$ be the maximal relation on the set $\C(V,Y)$, that is for every $f_1,f_2 : V \to Y$, we have $f_1\,R\,f_2$.
We will say that $g : X \to Y$ is \emph{pure} with respect to $f : U \to V$ if $g$ has RLP up to $R$ with respect to $f$.
The notion of pure morphism is (formally) similar to the concept of
$\lambda$-pure morphism, used in the theory of accessible categories.

We list a few elementary properties of pure morphisms in the following proposition:

\begin{prop} The following holds in every category $\C$:
\begin{enumerate}
\item If $g$ has RLP up to some relation with respect to $f$, then $g$ is pure with respect to $f$.
\item Pure maps are closed under composition.
\item If $f : X \to Y$ and $g : Y \to Z$ are maps such that $g \circ f$ is pure, then $f$ is also pure.
\item Every split monomorphism is pure with respect to all maps.
\item If a map is pure with respect to itself, then it is a split monomorphism.
\end{enumerate}
\end{prop}

We will say that a relative cylinder object $C(V)$ for $i : U \to V$ is \emph{proper} if the corresponding homotopy relation on every set $\C(V,X)$ is an equivalence relation.
Note that $\sim$ is reflexive if and only if there exists a map $s : C_U(V) \to V$ such that $s \circ \cyli_0 = s \circ \cyli_1 = id_V$.
Moreover, if pushout $V \amalg_U V$ exists and $s$ is pure with respect to $[\cyli_0,\cyli_1] : V \amalg_U V \to C_U(V)$, then the homotopy relation is also symmetric.

\subsection{Model categories}

Model categories were introduced in \cite{quillen}.
For an introduction to the theory of model categories see \cite{hirschhorn,hovey}.

\begin{defn} A weak factorization system on a category $\C$ is a pair $(\mathcal{L},\mathcal{R})$
of full subcategories of the category $\C^\to$ of arrows of $\C$ such that
\begin{itemize}
\item Every morphism factors into a map in $\mathcal{L}$ followed by a map in $\mathcal{R}$.
\item $\mathcal{L}$ is the class of maps that have LLP with respect to $\mathcal{R}$.
\item $\mathcal{R}$ is the class of maps that have RLP with respect to $\mathcal{L}$.
\end{itemize}
\end{defn}

\begin{defn}
A \emph{model structure} on a category $\C$ is a choice of three classes of morphisms $\fib$, $\cof$, and $\we$,
called \emph{fibrations}, \emph{cofibrations}, and \emph{weak equivalences} respectively, satisfying the following axioms:
\begin{itemize}
\item $\we$ has 2-out-of-3 property, that is given composable morphisms $f,g$,
    if two out of three morphisms $f$, $g$, $g \circ f$ are in $\we$, so is the third.
\item $(\cof, \fib \cap \we)$ and $(\cof \cap \we, \fib)$ are weak factorization systems.
\end{itemize}
A \emph{model category} is a complete and cocomplete category equipped with a model structure.
\end{defn}

A map is called a \emph{trivial fibration} if it belongs to $\fib \cap \we$,
and it is called a \emph{trivial cofibration} if it belongs to $\cof \cap \we$.
An object $X$ is called \emph{fibrant} if the morphism $X \to 1$ to the terminal object is a fibration,
and it is called \emph{cofibrant} if the morphism $0 \to X$ from the initial object is a cofibration.

Let $\C$ be a model category.
Then we can define a relative cylinder object $C_U(V)$ for every $i : U \to V$.
Let $[\cyli_0,\cyli_1] : V \amalg_U V \to C(V)$, $s : C(V) \to V$ be a factorization of $[id_V,id_V] : V \amalg_U V \to V$ into a cofibration and a trivial fibration.
The homotopy relation corresponding to this cylinder object is always reflexive and symmetric.
Later, we will see that if $i$ is a cofibration and every object of $\C$ is fibrant, then it is also transitive.

A map $f : X \to Y$ is \emph{an inclusion of a deformation retract} if there is
a map $g : Y \to X$ such that $g \circ f = id_X$ and $f \circ g \sim id_Y$.
A map $f : X \to Y$ is \emph{an inclusion of a strong deformation retract} if the homotopy is relative to $f$.

A \emph{path object} for an object $Z$ is an object $P(Z)$ together with maps $t : Z \to P(Z) \in \we$
and $\langle p_0,p_1 \rangle : P(Z) \to Z \times Z \in \fib$ such that $p_0 \circ t = p_1 \circ t = id_Z$.
A \emph{right homotopy} between morphisms $f_0,f_1 : Y \to Z$ is a map
$h : Y \to P(Z)$ such that $p_0 \circ h = f_0$ and $p_1 \circ h = f_1$.
We say that $f_0$ and $f_1$ are right homotopic and write $f_0 \sim^r f_1$
if there exists a right homotopy $h : Y \to P(Z)$ between them.
We say that a right homotopy $h : Y \to P(Z)$ between $f_0,f_1 : Y \to Z$
is constant on $i : X \to Y$ if $h \circ i = t \circ f_0 \circ i$.
In this case, we write $f_0 \sim^r_i f_1$.

\begin{prop}[left-right-hom]
Let $i : X \to Y$ be a cofibration, $Z$ a fibrant object, and $f_0,f_1 : Y \to Z$ a pair of maps.
Then $f_0 \sim_i f_1$ if and only if $f_0 \sim^r_i f_1$.
\end{prop}

\begin{prop}[hom-ext][Homotopy extension property]
Let $X$ be a fibrant object and let $i : U \to V \in \cof$, $u : U \to X$, and $v : V \to X$ be maps such that $u \sim^r v \circ i$.
Then there exists a map $v' : V \to X$ such that $u = v' \circ i$ and $v' \sim^r v$.
\end{prop}

Let $\C$ be a category and let $\I$ be a class of morphisms of $\C$.
Then we define $\Iinj$ to be the class of morphisms of $\C$ that has RLP with respect to $\I$,
$\Icof$ to be the class of morphisms of $\C$ that has LLP with respect to $\Iinj$, and
$\Icell$ to be the class of transfinite compositions of pushouts of elements of $\I$.
Elements of $\Icell$ are called \emph{relative $\I$-cell complexes}.
Every relative $\I$-cell complex belongs to $\Icof$.

We say that a set $\I$ of maps of a cocomplete category $\C$ \emph{permits the small object argument}
if the domains of maps in $\I$ are small relative to $\Icell$.

\begin{prop}[][The small object argument]
Let $\C$ be a cocomplete category and $\I$ a set of maps of $\C$ that permits the small object argument.
Then $(\Icell,\Iinj)$ is a weak factorization system.
\end{prop}

\begin{defn}
Let $\C$ be a category with a model structure. Then this model category is
\emph{cofibrantly generated} if there are sets $\I$ and $\J$ of maps of $\C$ such
that they permit the small object argument, $\fib = \Jinj$, and $\fib \cap \we = \Iinj$.

Elements of $\I$ are called \emph{generating cofibrations},
and elements of $\J$ are called \emph{generating trivial cofibrations}.
\end{defn}

\begin{prop}[model-cat]
Suppose that $\C$ is a complete and cocomplete category, $\we$ is a class of morphisms of $\C$, and $\I$, $\J$ are sets of morphisms of $\C$.
Then $\C$ is a cofibrantly generated model category with $\I$ as the set of generating cofibrations,
$\J$ as the set of generating trivial cofibrations, and $\we$ as the class of weak equivalences if and only if the following conditions are satisfied:
\begin{description}
\item[(A1)] $\I$ and $\J$ permit the small object argument.
\item[(A2)] $\we$ has 2-out-of-3 property and is closed under retracts.
\item[(A3)] $\Iinj \subseteq \we$.
\item[(A4)] $\Jcell \subseteq \we \cap \Icof$.
\item[(A5)] Either $\Jinj \cap \we \subseteq \Iinj$ or $\Icof \cap \we \subseteq \Jcof$.
\end{description}
\end{prop}

\section{Properties of f-model categories}

An \emph{f-model structure} on a category is a model structure in which every object is fibrant.
In this section, we will prove various properties of f-model categories.
In particular, we will give a characterization of weak equivalences in such categories.

For every $i : U \to V$, let $C_U(V)$ be any cylinder object such that $[\cyli_0,\cyli_1] : V \amalg_U V \to C_U(V) \in \cof$ and $s : C_U(V) \to V \in \fib \cap \we$.
In this section, $\sim_i$ will always stand for the homotopy relation with respect to $C_U(V)$.
The following propositions give useful characterizations of trivial
cofibrations and weak equivalences in f-model categories.

\begin{prop}[min-triv-cof]
Let $\C$ be a category with an f-model structure.
Then a cofibration $f : X \to Y$ is trivial if and only if it is an inclusion of a strong deformation retract.
\end{prop}

This characterization of trivial cofibrations is probably well-known.
The proof of the ``only if'' direction can be found, for example, in \cite{hirschhorn}.
We include a proof of the ``if'' direction for the sake of convenience.

\begin{proof}
Let $f : X \to Y$ be an inclusion of a strong deformation retract.
Factor $f$ into a map $i : X \to Z \in \cof \cap \we$ followed by a map $p : Z \to Y \in \fib$.
Let $g : Y \to X$ be a retraction of $f$ and let $h : C_X(Y) \to Y$ be a homotopy between $f \circ g$ and $id_Y$.
Consider the following diagram:
\[ \xymatrix{ Y \ar[r]^{i \circ g} \ar[d]_{\cyli_0} & Z \ar[d]^p \\
              C_X(Y) \ar[r]_-h \ar@{-->}[ur]^{h'}   & Y.
            } \]
Since $\cyli_0 : Y \to C_X(Y)$ is a trivial cofibration, we have a lift $h' : C_X(Y) \to Z$.
Then $h' \circ \cyli_1 \circ f = i$ and $p \circ h' \circ \cyli_1 = id_Y$, thus $f$ is a retract of $i$, hence a weak equivalence.
\end{proof}

We say that a map in a model category is \emph{pure} if it is pure with respect to cofibrations between cofibrant objects.

Now, we give a characterization of weak equivalences in an f-model category.
Later we will add another equivalent condition to this list.

\begin{prop}[min-we]
Let $\C$ be a category with an f-model structure and let $f : X \to Y$ be a map of $\C$.
Then the following conditions are equivalent:
\begin{enumerate}
\item $f$ is a weak equivalence.
\item $f$ has RLP up to $\sim_i$ with respect to every cofibration $i$.
\item $f$ is pure and has RLP up to $\sim$ with respect to cofibrant objects.
\end{enumerate}
\end{prop}
\begin{proof}
$(1 \Rightarrow 2)$
Let $f : X \to Y$ be a weak equivalence.
Factor it into a trivial cofibration $i : X \to Z$ followed by a trivial fibration $p : Z \to Y$.
Let $c : U \to V$ be a cofibration and let $u : U \to X$, $v : V \to Y$ be maps such that $f \circ u = v \circ c$.
Since $c$ is a cofibration and $p$ is a trivial fibration, we have a lift $q : V \to Z$.
Since $i$ is a trivial cofibration, it has a retraction $r : Z \to X$ such that $i \circ r \sim_i id_Z$.
Then $r \circ q$ is a required lift.

$(2 \Rightarrow 3)$ Obvious.

$(3 \Rightarrow 1)$
Let $f : X \to Y$ be a pure map that has RLP with respect to cofibrant objects.
Let $q_x : X' \to X \in \fib \cap \we$ be a cofibrant replacement for $X$.
Factor $f \circ q_x$ into a map $f' : X' \to Y' \in \cof$ followed by a map $q_y : Y' \to Y \in \fib \cap \we$.
\[ \xymatrix{ X' \ar[r]^{q_x} \ar[d]_{f'} & X \ar[d]^f \\
              Y' \ar[r]_{q_y}             & Y
            } \]
Since $f \circ q_x$ is pure, $f'$ has a retraction $g' : Y' \to X'$.

First, let us show that $f'$ has RLP up to $\sim$ with respect to cofibrant objects.
Indeed, let $U$ be a cofibrant object and $u : U \to Y'$ a map.
Then we have a map $u' : U \to X'$ such that $q_y \circ f' \circ u' = f \circ q_x \circ u' \sim q_y \circ u$.
Since $q_y$ is a trivial fibration, we have a lift in the diagram below, which gives us a homotopy between $f' \circ u'$ and $u$.
\[ \xymatrix{ U \amalg U \ar[rr]^-{[f' \circ u', u]} \ar[d] & & Y' \ar[d]^{q_y} \\
              C(U)       \ar[rr] \ar@{-->}[urr]             & & Y
            } \]

Second, let us show that $g'$ is pure.
Indeed, let $i : U \to V$ be a cofibration between cofibrant objects, and suppose that we have a commutative square of the form
\[ \xymatrix{ U \ar[r]^u \ar[d]_i & Y' \ar[d]^{g'} \\
              V \ar[r]_v          & X'.
            } \]
Then there exists a map $u' : U \to X'$ such that $f' \circ u' \sim u$; hence $v \circ i = g' \circ u \sim u'$.
By \rprop{left-right-hom} and \rprop{hom-ext}, there exists a map $v' : V \to X'$ such that $v' \circ i = u'$ and $v' \sim v$.
Then $f' \circ v' \circ i = f' \circ u' \sim u$, and by the same propositions, there exists a map $v'' : V \to Y'$ such that $v'' \circ i = u$, which shows that $g'$ is pure.

Finally, consider the following diagram:
\[ \xymatrix{ Y' \amalg_{X'} Y' \ar[rr]^-{[f' \circ g', id_{Y'}]} \ar[d] &                & Y' \ar[d]^{g'} \\
              C_{X'}(Y')        \ar[r]                                   & Y' \ar[r]_{g'} & X'.
            } \]
Since $g'$ is pure, we have a homotopy $C_{X'}(Y') \to Y'$ between $f' \circ g'$ and $id_{Y'}$.
By \rprop{min-triv-cof}, $f'$ is a weak equivalence.
Hence $f$ is a weak equivalence by the 2-out-of-3 property.
\end{proof}

\begin{cor}
Let $\C$ be an f-model category.
Then for every $i : U \to V$, the homotopy relation $\sim_i$ is an equivalence.
\end{cor}
\begin{proof}
This relation is reflexive and symmetric in every model category.
To prove that it is transitive, consider the following diagram:
\[ \xymatrix{ V \ar[r]^{i_0} \ar[d]_{i_1} &     C_U(V) \ar[d]^q \\
              C_U(V) \ar[r]  \ar[d]_s     & \po C_U(V) \amalg_V C_U(V) \ar[d]^p \\
              V \ar[r]                    & \po C_U(V)
            } \]
Since $i_0$ is a trivial cofibration, $q$ is also a trivial cofibration.
Hence, $p$ is a weak equivalence by the 2-out-of-3 property.
By the previous proposition, it is pure.

Let $h_1 : C_U(V) \to X$ and $h_2 : C_U(V) \to X$ be homotopies between $f_1,f_2 : V \to X$ and $f_2,f_3 : V \to X$ respectively,
Consider the following diagram:
\[ \xymatrix{ V \amalg_U V \ar[r]^-{i_0 \amalg_i i_1} \ar[d]_{[i_0,i_1]} & C_U(V) \amalg_V C_U(V) \ar[d]^p \\
              C_U(V)       \ar[r]_{id} \ar@{-->}[ur]_h                  & C_U(V)
            } \]
Since $p$ is pure, we have a lift $h$.
Then $[h_1,h_2] \circ h$ is a homotopy between $f_1$ and $f_3$.
\end{proof}

Now, given a set of generating cofibrations we show how to construct a set of generating trivial cofibrations.
Let $\C$ be a cocomplete category and $\I$ a set of maps of $\C$, that permits the small object argument.
Let us denote by $\J_\I$ the set of maps $\cyli_0 : V \to C_U(V)$ for each $U \to V \in \I$.

\begin{prop}[triv-fib-iinj]
Let $\C$ be a cocomplete category and $\I$ a set of maps of $\C$ such that
the domains and the codomains of maps in $\I$ are small relative to $\Icell$.
If $f : X \to Y \in \Jinj[_\I]$ has RLP up to $\sim_i$ with respect to every $i \in \I$, then $f \in \Iinj$.
\end{prop}
\begin{proof}
Suppose we have a commutative square as below.
We need to find a lift $V \to X$ such that both triangles commute.
\[ \xymatrix{ U \ar[d]_{i \in \I} \ar[r] \ar@{}[dr]|(.7){\sim} & X \ar[d]^f \\
              V \ar[r]_v \ar@{-->}[ur]^g              & Y
            } \]
By assumption, we have a lift $g : V \to X$ together with
a homotopy $h : C_U(V) \to Y$ between $f \circ g$ and $v$.
Since $f \in \Jinj[_\I]$, we have a lift $h' : C_U(V) \to X$ in the following diagram:
\[ \xymatrix{ V \ar[r]^g \ar[d]_{\cyli_0}         & X \ar[d]^f \\
              C_U(V) \ar[r]^-h \ar@{-->}[ur]^{h'} & Y.
            } \]
Then $h' \circ \cyli_1$ is a required lift in the original square.
\end{proof}

\begin{cor}[min-cof-gen]
Let $\C$ be an f-model category.
Suppose that the class of cofibrations is generated by a set $\I$ such that
the domains and the codomains of maps in $\I$ are small relative to $\Icell$.
Then the model structure is cofibrantly generated with $\J_\I$ as a set of generating trivial cofibrations.
\end{cor}
\begin{proof}
This follows from the previous proposition and \rprop{model-cat}.
\end{proof}

\section{The existence of an f-model structure}

In this section, we will give a necessary and sufficient
condition for the existence of an f-model structure.
Throughout this section let $\C$ be a fixed cocomplete category and $\I$ a set of maps of $\C$
such that the domains and the codomains of maps in $\I$ are small relative to $\Icell$.
We call a map $f$ a cofibration if $f \in \Icof$, and a trivial fibration if $f \in \Iinj$.
We say that a map is a \emph{weak equivalence} if it has has RLP up to $\sim_i$ with respect to every $i \in \I$.
The class of weak equivalences is denoted by $\we_\I$.
Note that every trivial fibration is a weak equivalence.

\begin{lem}[transitive]
If $\I$ is appropriate, then for every cofibration $f : X \to Y$, relation $\sim_f$ is transitive.
\end{lem}
\begin{proof}
Let $h_1 : Cyl_X Y \to Z$ be a homotopy between $g_1,g_2 : Y \to Z$ and
let $h_2 : Cyl_X Y \to Z$ be a homotopy between $g_2,g_3 : Y \to Z$.
We want to show that $g_1 \sim g_3$.
Let $W$ be the pushout of $\imath_0 : Y \to Cyl_X Y$ and $\imath_1 : Y \to Cyl_X Y$.
Let $q_1 : Cyl_X Y \to W$ be the pushout of $\imath_1$ and $q_0 : Cyl_X Y \to W$ the pushout of $\imath_0$.
Let $t : W \to Cyl_X Y$ be the pushout of $s : Cyl_X Y \to Y$ along $q_1$.
Since $\I$ is appropriate, $t$ is pure.
Consider the following diagram:
\[ \xymatrix{ Y \amalg_X Y \ar[rr]^-{[q_0 \circ \imath_0, q_1 \circ \imath_1]} \ar[d] & & W \ar[d]^t \\
              Cyl_X Y      \ar@{=}[rr]                                                & & Cyl_X Y.
            } \]
Since $t$ is pure, we have a map $h : Cyl_X Y \to W$ such that the top triangle commutes.
Then $[h_1,h_2] \circ h$ is a required homotopy between $g_1$ and $g_3$.
\end{proof}

\begin{lem}[top-2-out-of-3]
Let $f : X \to Y$ and $g : Y \to Z$ be maps such that $g \circ f$ is a weak equivalence and $g$ is pure.
Then $f$ is a weak equivalence.
\end{lem}
\begin{proof}
Suppose that we have a commutative square:
\[ \xymatrix{ U \ar[r]^u \ar[d]^i & X \ar[d]^f \\
              V \ar[r]_v          & Y.
            } \]
Since $g \circ f \in \we_\I$, we have a map $k : V \to X$ such that $k \circ i = u$ and $g \circ v \sim_i g \circ f \circ k$.
Consider the following diagram:
\[ \xymatrix{ V \amalg_U V \ar[rr]^-{[v, f \circ k]} \ar[d] & & Y \ar[d]^g \\
              Cyl_U V      \ar[rr]                          & & Z.
            } \]
Since $g$ is pure, we have a homotopy $h : Cyl_U V \to Y$ between $v$ and $f \circ k$.
\end{proof}

Let $X$ be a cofibrant object.
We say that an object $C_X$ is a cylinder object for $X$ if it is equipped with maps $\imath_0,\imath_1 : X \to C_X$
and $s : C_X \to X$ such that $s$ is pure and $s \circ \imath_0 = s \circ \imath_1 = id_X$.
A pair of maps $f_0,f_1 : X \to Y$ are $C_X$-homotopic if there exists a map
$h : C_X \to Y$ such that $h \circ \imath_0 = f_0$ and $h \circ \imath_1 = f_1$.
Note that if a pair of maps are $C_X$-homotopic, then they are homotopic.
A map of cylinder objects $C_X$ and $C_Y$ is a pair of maps $X \to Y$
and $C_X \to C_Y$ which commute with $\imath_0$, $\imath_1$, and $s$.

We now prove a variant of the homotopy extension property.

\begin{prop}[proper-hom-ext]
Suppose that $\I$ is appropriate.
Let $i : U \to V$ be a cofibration between cofibrant objects and $v : V \to X$ a map.
Let $h : Cyl\,U \to X$ be a homotopy such that $h \circ \imath_1 = v \circ i$.
Then there exists a cylinder object $C$ for $V$ together with a map of cylinder objects
$k : Cyl\,U \to C \in \Icell$ and a homotopy $h' : C \to X$ such that $h' \circ \imath_1 = v$ and $h' \circ k = h$.
\end{prop}
Note, that the same proposition holds, if we replace $\imath_1$ with $\imath_0$.
\begin{proof}
Let $M$ be the pushout of $i$ and $\imath_1 : U \to Cyl\,U$.
Let $M \amalg_U V$ be the pushout of $U \overset{\imath_0}\to Cyl\,U \to M$ and $i$.
Let $t : M \to V$ be the pushout of $s : Cyl\,U \to U$ along $Cyl\,U \to M$.
The maps $t$ and $id_V$ induce a map $M \amalg_U V \to V$.
Factor it into a map $M \amalg_U V \to C \in \Icell$ followed by a map $C \to V \in \Iinj$.
The map $t$ is pure since $\I$ is appropriate; hence $M \to M \amalg_U V \to C$ is pure as well.
It is also a cofibration between cofibrant objects, hence it has a retraction $C \to M$.
The maps $v : V \to X$ and $h : Cyl\,U \to X$ induce a map $M \to X$.
\[ \xymatrix{                               & U \ar[r]^i \ar[d]_{\imath_1} & V \ar[d] \ar[drr]^v     &          &        & \\
              U \ar[r]^-{\imath_0} \ar[d]_i & Cyl\,U \ar[r]                & \po M \ar[d] \ar@{=}[r] & M \ar[r] & X      & \\
              V \ar[rr]                     &      & \po M \amalg_U V \ar[r]^-{\in \Icell} & C \ar[r]^{\in \Iinj} \ar[u] & V
            } \]
Then $C$ is a required cylinder object for $V$, and $C \to M \to X$ is a required homotopy.
\end{proof}

\begin{cor}[proper-hom-ext]
Suppose that $\I$ is appropriate.
Let $i : U \to V$ be a cofibration between cofibrant objects.
Let $u : U \to X$ and $v : V \to X$ be maps such that $u \sim v \circ i$.
Then there exists a map $v' : V \to X$ such that $u = v' \circ i$ and $v' \sim v$.
\end{cor}

Using the previous proposition, we can prove the following lemma, which will be useful later.

\begin{lem}[bot-pure]
Suppose that $\I$ is appropriate.
Let $i : U \to V \in \Icof$, $f : X \to Y$, and $g : Y \to Z$ be maps such that $U$ is cofibrant, $g \circ f$
is pure with respect to $i : U \to V$, and $f$ has RLP up to $\sim$ with respect to $U$.
Then $g$ is pure with respect to $i$.
\end{lem}
\begin{proof}
Suppose that we have a commutative square as below.
Then there exists a map $u' : U \to X$ such that $f \circ u' \sim u$.
\[ \xymatrix{   \ar@{}[dr]|(.7){\sim}                & X \ar[d]^f \\
              U \ar@{-->}[ur]^{u'} \ar[d]_i \ar[r]_u & Y \ar[d]^g \\
              V \ar[r]_v                             & Z
            } \]
Since $g \circ f \circ u' \sim v \circ i$, by \rcor{proper-hom-ext}, there
exists a map $v_1 : V \to Z$ such that $g \circ f \circ u' = v_1 \circ i$.
Since $g \circ f$ is pure with respect to $i$, there exists a map $v_2 : V \to X$ such that $u' = v_2 \circ i$.
Since $f \circ v_2 \circ i = f \circ u' \sim u$, by \rcor{proper-hom-ext},
there exists a map $v_3 : V \to Y$ such that $v_3 \circ i = u$.
Thus $g$ is pure with respect to $i$.
\end{proof}

The following proposition is a crucial step in the proof of the 2-out-of-3 property for $\we_\I$.

\begin{prop}[hom-pushouts]
Suppose that $\I$ is appropriate.
Let $Z \leftarrow X \to Y$ and $Z' \leftarrow X' \to Y'$ be diagrams in which
all objects are cofibrant, and $X \to Y$ and $X' \to Z'$ are cofibrations.
Suppose that we have a morphism between these diagrams such that $X \to X'$, $Y \to Y'$, and $Z \to Z'$ are trivial fibrations.
Then the induced map $Y \amalg_X Z \to Y' \amalg_{X'} Z'$ is pure.
\end{prop}
\begin{proof}
First, note that if $A,B,C,E$ are cofibrant objects, $A \to B$ is a cofibration, $A \to C$ is an arbitrary map, and
$C \to E$ is a pure map, then the induced map $B \amalg_A C \to B \amalg_A E$ is pure.
Indeed, factor $C \to E$ into a map $C \to D \in \Icell$ followed by a map $D \to E \in \Iinj$.
\[ \xymatrix{ A \ar[r] \ar[d] &     C \ar[d]     \ar[r] &     D \ar[d]     \ar[r] &     E \ar[d] \\
              B \ar[r]        & \po B \amalg_A C \ar[r] & \po B \amalg_A D \ar[r] & \po B \amalg_A E
            } \]
Since $C \to D$ is a split monomorphism, $B \amalg_A C \to B \amalg_A D$ is a split monomorphism as well, hence it is pure.
The map $B \amalg_A D \to B \amalg_A E$ is pure since $\I$ is appropriate.
Thus $B \amalg_A C \to B \amalg_A E$ is pure.

Now, note that the map $Y \amalg_X Z \to Y' \amalg_{X'} Z'$ factors into the following maps
$Y \amalg_X Z \to Y \amalg_X Z' \simeq (X' \amalg_X Y) \amalg_{X'} Z' \to Y' \amalg_{X'} Z'$.
Since $\I$ is appropriate, $Y \to X' \amalg_X Y$ has RLP up to $\sim$ with respect to cofibrant objects.
Since $Y \to Y'$ is pure, by \rlem{bot-pure}, $X' \amalg_X Y \to Y'$ is pure.
Then $(X' \amalg_X Y) \amalg_{X'} Z' \to Y' \amalg_{X'} Z'$ and $Y \amalg_X Z \to Y \amalg_X Z'$ are pure, as we noted before.
Thus $Y \amalg_X Z \to Y' \amalg_{X'} Z'$ is pure as well.
\end{proof}

\begin{cor}[cyl-pushouts]
Suppose that $\I$ is appropriate.
Let $X,Y,Z$ be cofibrant objects and let $C_X,C_Y,C_Z$ be cylinder objects
such that $C_X \to X$, $C_Y \to Y$, and $C_Z \to Z$ are trivial fibrations.
Let $C_X \to C_Y \in \Icof$ and $C_X \to C_Z$ be maps of cylinder objects such that $X \to Z \in \Icof$.
Then $C_Y \amalg_{C_X} C_Z$ is a cylinder object for $Y \amalg_X Z$.
\end{cor}

Using the previous proposition, we can prove the last part of the 2-out-of-3 property.

\begin{lem}[bot-2-out-of-3]
Suppose that $\I$ is appropriate.
Let $f : X \to Y$ and $g : Y \to Z$ be maps such that $g \circ f$ is a weak equivalence
and $f$ has RLP up to $\sim$ with respect to the domains of maps in $\I$.
Then $g$ is a weak equivalence.
\end{lem}
\begin{proof}
Suppose that we have a commutative square as below.
Then there exists a map $u_x : U \to X$ and a homotopy $h : Cyl\,U \to Y$ between $f \circ u_x$ and $u$.
\[ \xymatrix{   \ar@{}[dr]|(.7){\sim}                & X \ar[d]^f \\
              U \ar@{-->}[ur]^{u_x} \ar[d]_i \ar[r]_u & Y \ar[d]^g \\
              V \ar[r]_v                             & Z
            } \]
The map $g \circ h$ is a homotopy between $g \circ f \circ u_x$ and $v \circ i$.
Let $C_V$ be a cylinder object for $V$ and $h_z : C_V \to Z$ a homotopy constructed in \rprop{proper-hom-ext}.
Let $v_z : V \to Z$ be equal to $h_z \circ \imath_0$.
Consider the following diagram:
\[ \xymatrix{ U \ar[r]^{u_x} \ar[d]_i \ar@{}[dr]|(.7){\sim_i} & X \ar[d]^{g \circ f} \\
              V \ar[r]_{v_z} \ar@{-->}[ur]^{v_x}              & Z.
            } \]
Since $g \circ f$ is a weak equivalence, we have a lift $v_x : V \to X$ together
with a homotopy $h_1 : Cyl_U V \to Z$ between $v_z$ and $g \circ f \circ v_x$.

Note that $h \circ \imath_0 = f \circ v_x \circ i$.
Let $C'_V$ be a cylinder object for $V$ and $h_y : C'_V \to Y$ a homotopy constructed in \rprop{proper-hom-ext}.
Let $v_y : V \to Y$ be equal to $h_y \circ \imath_1$.
By \rcor{cyl-pushouts}, $C_V \amalg_{Cyl\,U} C'_V$ is a cylinder object for $V \amalg_U V$.
Let $h_2 : C_V \amalg_{Cyl\,U} C'_V \to Z$ be equal to $[h_z, g \circ h_y]$.
Then $h_2$ is a homotopy between $V \amalg_U V \to Cyl_U V \overset{h_1}\to Z$ and $[v, g \circ v_y]$.
By \rcor{proper-hom-ext}, there exists a homotopy $h_3 : Cyl_U V \to Z$ between $v$ and $g \circ v_y$.
Thus $v_y$ is a required lift; hence $g$ is a weak equivalence.
\end{proof}

Now, we are ready to prove one of the main results of this paper.

\begin{thm}[main]
Let $\C$ be a complete and cocomplete category and $\I$ a set of maps of $\C$ such that
the domains and the codomains of maps in $\I$ are cofibrant and small relative to $\Icell$.
Then the following conditions are equivalent:
\begin{enumerate}
\item There is an f-model structure on $\C$ with $\Icof$ as the class of cofibrations.
\item \label{item:main:snd} $\I$ is appropriate, and $\Jcell[_\I]$ has RLP up to $\sim$ with respect to the domains of maps in $\I$.
\end{enumerate}
In this case, a map is a weak equivalence if and only if it has RLP up to $\sim_i$ with respect to every $i \in \I$.
\end{thm}
\begin{proof}
Suppose that the model structure on $\C$ exists.
% Then the required conditions hold by \rprop{cof-proper} and \rprop{min-we}.
Now, suppose that $(\ref{item:main:snd})$ holds.
Let us verify the conditions of \rprop{model-cat}.
\begin{description}
\item[(A1)] The domains of maps in $\J_\I$ are the codomains of maps in $\I$, and $\Jcell[_\I] \subseteq \Icell$.
Hence the domains of maps in $\J_\I$ are small relative to $\Jcell[_\I]$.
\item[(A2)] The closure of $\we_\I$ under retracts is obvious.
Let us prove the 2-out-of-3 property.
Let $f : X \to Z$ be a weak equivalence.
Factor $f$ into a map $g : X \to Y \in \Jcell[_\I]$ followed by $h : Y \to Z \in \Jinj[_\I]$.
By \rlem{bot-2-out-of-3}, $h$ is a weak equivalence, and by \rprop{triv-fib-iinj}, $h$ is a trivial fibration.
Thus every weak equivalence is pure and has RLP up to $\sim$ with respect to the domains of maps in $\I$.
Now, the 2-out-of-3 property follows from \rlem{transitive}, \rlem{top-2-out-of-3}, and \rlem{bot-2-out-of-3}.
\item[(A3)] This is obvious.
\item[(A4)] Since maps in $\J_\I$ are split monomorphisms, every map
$f \in \Jcell[_\I]$ has a retraction, which is a weak equivalence by
\rlem{bot-2-out-of-3}; hence $f$ is a weak equivalence by 2-out-of-3 property.
\item[(A5)] This holds by \rprop{triv-fib-iinj}.
\end{description}
\end{proof}

\section{Properness}

The conditions of \rthm{main} may still be difficult to verify.
In this section, we show that in a certain case, given a set of generating cofibrations,
left properness implies the existence of an f-model structure.

\begin{lem}[j-transfinite]
Suppose that $U$ is finite (that is $\omega$-small) relative to $\Icell$.
If $\sim$ is transitive and pushouts of maps in $\J_\I$ has RLP up to $\sim$ with respect to $U$, then every map in $\Jcell[_\I]$ has this property.
\end{lem}
\begin{proof}
Let $P$ be the class of maps that has RLP up to $\sim$ with respect to $U$.
We will show that $P$ is closed under transfinite compositions.
Let $\lambda$ be an ordinal and $X : \lambda \to \C$ a $\lambda$-sequence such that
for every $\alpha < \lambda$, the map $X_\alpha \to X_{\alpha + 1}$ belongs to $P$.
For each $\alpha < \beta \leq \lambda$, let us denote by $x^\alpha_\beta$ the map $X_\alpha \to X_\beta$.
We show by transfinite induction on $\lambda$ that $x_\lambda \in P$.

The case $\lambda = 0$ is trivial.

Suppose that $\lambda = \alpha + 1$.
Let $u : U \to X_{\alpha + 1}$ be a map.
Then there exists a map $u' : U \to X_\alpha$ such that $x^\alpha_{\alpha + 1} \circ u' \sim u$.
By induction hypothesis, there exists a map $u'' : U \to X_0$ such that $x^0_\alpha \circ u'' \sim u'$.
Hence $x^0_{\alpha + 1} \circ u'' \sim u$.

Suppose that $\lambda$ is a limit ordinal. 
Suppose that we have a map $u : U \to X_\lambda$.
Since $U$ is finite, there exists a map $u' : U \to X_\alpha$ for
some $\alpha < \lambda$ such that $u = x^\alpha_\lambda \circ u'$.
By induction hypothesis, there exists a map $u'' : U \to X_0$ such that $x^0_\alpha \circ u'' \sim u'$.
Hence $x^0_\lambda \circ u'' \sim u$.
\end{proof}

First, let us give a characterization of weak equivalences.

\begin{prop}[pure-we]
Suppose that $\I$ is appropriate.
Let $f : X \to Y$ be a pure map that has RLP up to $\sim$ with respect to $V \amalg_U V$ for every $i : U \to V \in \I$.
Then $f$ is a weak equivalence.
\end{prop}
\begin{proof}
Let $u : X' \to X \in \Iinj$ be a cofibrant replacement for $X$.
Factor $f \circ u$ into a map $f' : X' \to Y' \in \Icell$ followed by a map $Y' \to Y \in \Iinj$.
Since every map from $i \in \I$ to $f$ factors through $f'$, it is enough to show that $f'$ is a weak equivalence.
Suppose that we have a commutative square
\[ \xymatrix{ U \ar[r]^u \ar[d]_i & X' \ar[d]^{f'} \\
              V \ar[r]_v          & Y'.
            } \]
Since $f$ is pure, $f'$ is pure as well.
Let $g' : Y' \to X'$ be a retraction of $f'$.
Consider the following diagram:
\[ \xymatrix{ V \amalg_U V \ar[rr]^-{[f' \circ g' \circ v, v]} \ar[d] & & Y' \ar[d]^{g'} \\
              Cyl_U V      \ar[rr]_-{g' \circ v \circ s}              & & X'.
            } \]
Since $f$ has RLP up to $\sim$ with respect to $V \amalg_U V$, $f'$ has this property as well.
Hence by \rlem{bot-pure}, $g'$ is pure with respect to $V \amalg_U V \to Cyl_U V$.
Hence we have a homotopy $h' : Cyl_U V \to Y'$ between $f' \circ g' \circ v$ and $v$.
Thus $f'$ is a weak equivalence.
\end{proof}

\begin{prop}[we-pure]
Suppose that $\I$ is appropriate, trivial fibrations between cofibrant objects are closed under coproducts,
and the domains of maps in $\I$ are finite relative to $\Icell$.
Then weak equivalences are pure and have RLP up to $\sim$ with respect to finite cofibrant objects.
\end{prop}
\begin{proof}
Let $f : X \to Y$ be a weak equivalence.
First, let us show that $f$ has RLP up to $\sim$ with respect to
cofibrations between cofibrant objects, which are pushouts of coproducts of maps in $\I$.
Consider the following diagram:
\[ \xymatrix{ \coprod U_a \ar[r]^u \ar[d] &     U' \ar[r]^{u'} \ar[d] & X \ar[d]^f \\
              \coprod V_a \ar[r]_v        & \po V' \ar[r]_{v'}        & Y.
            } \]
Since $f$ is a weak equivalence, we have a lift $v_1 : \coprod V_a \to X$ together
with a homotopy $h : \coprod Cyl_{U_a} V_a \to Y$ between $f \circ v_1$ and $v' \circ v$.
Since trivial fibrations are closed under coproducts, $\coprod Cyl_{U_a} V_a$ is a cylinder object for $\coprod V_a$.
The maps $v_1$ and $u'$ induce a map $v_2 : V' \to X$.
Let $C_{U'}$ be a cylinder object for $V'$ such that there exists a map of cylinder objects $Cyl\,U \to C_{U'} \in \Icell$.
Let $C_{V'}$ be the pushout of $\coprod Cyl\,U_a \overset{s}\to \coprod U_a \to \coprod Cyl_{U_a} V_a$ and $\coprod Cyl\,U_a \to C_{U'}$.
By \rcor{cyl-pushouts}, $C_{V'}$ is a cylinder object for $V'$.
The maps $h : \coprod Cyl_{U_a} V_a \to Y$ and $C_{U'} \to U' \overset{u'}\to X \overset{f}\to Y$
induce a homotopy $C_{V'} \to Y$ between $f \circ v_2$ and $v'$.
Thus $v_2 : V' \to X$ is a required lift.

Now, let us show that $f$ is pure.
Since every cofibration is a retract of relative $\I$-cell complex, we only need to prove the following fact.
Let $U_0 \xrightarrow{u^0} U_1 \xrightarrow{u^1} \ldots U_\omega$ be a
sequence of pushouts of coproducts of maps in $\I$ such that $U_0$ is cofibrant.
Let us denote by $u^i_\omega : U_i \to U_\omega$ the map $U_i \to U_\omega$.
We need to show that $f$ is pure with respect to $u^0_\omega$.
Suppose that we have a commutative square
\[ \xymatrix{ U_0      \ar[r]^{v_0} \ar[d]_{u^0_\omega} & X \ar[d]^f \\
              U_\omega \ar[r]_v                         & Y.
            } \]
Then we construct maps $v_n : U_n \to X$ by induction on $n$ in such a
way that $v_{n+1} \circ u^n = v_n$ and $f \circ v_n \sim v \circ u^n_\omega$.
Map $v_0$ is already defined.
Suppose that $n > 0$.
By induction hypothesis, there is a map $v_{n-1} : U_{n-1} \to X$
such that $f \circ v_{n-1} \sim v \circ u^n_\omega$.
By \rcor{proper-hom-ext}, there exists a map $v' : U_\omega \to Y$
such that $f \circ v_{n-1} = v' \circ u^n_\omega$ and $v' \sim v$.
Then there exists a map $v_n : U_n \to X$ such that $v_n \circ u^{n-1} = v_{n-1}$
and $f \circ v_n \sim v' \circ u^n_\omega \sim v \circ u^n_\omega$.
The maps $v_n : U_n \to X$ induce a map $v_\omega : U_\omega \to X$ such that $v_\omega \circ u^0_\omega = v_0$.
Thus $f$ is pure with respect to $u^0_\omega$.

We already saw that $f$ has RLP up to $\sim$ with respect to every cofibration between
cofibrant objects, which is a finite composition of pushouts of coproducts of maps in $\I$.
Since for every finite cofibrant object $X$, map $0 \to X$ is a retract of
such cofibration, it follows that $f$ has RLP up to $\sim$ with respect to $X$.
\end{proof}

Now we are ready to prove that left properness implies the existence of an f-model structure.

\begin{thm}[properness]
Let $\C$ be a complete and cocomplete category and $\I$ a set of maps of $\C$ such that
trivial fibrations between cofibrant objects are closed under coproducts, and the domains
and the codomains of maps in $\I$ are cofibrant and finite relative to $\Icell$.
Then the following conditions are equivalent:
\begin{enumerate}
\item There is a left proper f-model structure on $\C$ with $\Icof$ as the class of cofibrations.
\item $\I$ is appropriate, and for every $i : U \to V \in \I$, $u : U \to X$, and
$X \to Y \in \Iinj$, the map $X \amalg_U V \to Y \amalg_U V$ is a weak equivalence.
\end{enumerate}
\end{thm}
\begin{proof}
If a left proper f-model structure exists, then the required conditions hold by \rthm{main}.
To prove the converse, we need to show that $\Jcell[_\I]$ has
RLP up to $\sim$ with respect to the domains of maps in $\I$.
By \rlem{j-transfinite}, it is enough to show that pushouts of maps in
$\J_\I$ has RLP up to $\sim$ with respect to the domains of maps in $\I$.
Let $g : X \to Y$ be the pushout of a map $j : V \to Cyl_U V \in \J_\I$ along a map $f : V \to X$.
Factor $f$ into a map $V \to X' \in \Icell$ followed by a map $X' \to X \in \Iinj$.
\[ \xymatrix{ V       \ar[r] \ar[d]_j &     X' \ar[r] \ar[d] &     X \ar[d]^g \\
              Cyl_U V \ar[r] \ar[d]_s & \po Y' \ar[r] \ar[d] & \po Y \\
              V       \ar[r] & \po X'
            } \]
The map $Y' \to X'$ is pure since $\I$ is appropriate, hence $X' \to Y'$ is a weak equivalence by \rlem{top-2-out-of-3}.
Thus we only need to show that weak equivalences with cofibrant domains are closed
under pushouts along relative $\I$-cell complexes between cofibrant objects.
Then $Y' \to Y$ is a weak equivalence, and by \rprop{we-pure},
$g$ has RLP up to $\sim$ with respect to the domains of maps in $\I$.

Let $f : X \to Z$ be a weak equivalence with cofibrant domain, $i : U \to V \in \I$, and $u : U \to X$.
Let us show that $X \amalg_U V \to Z \amalg_U V$ is a weak equivalence.
Factor $f$ into a map $g : X \to Y \in \Icell$ followed by a map $Y \to Z \in \Iinj$.
Then $Y \amalg_U V \to Z \amalg_U V$ is a weak equivalence by \rprop{pure-we}.
The map $g$ is a weak equivalence, hence by \rprop{we-pure}, it is pure.
Let $h : Y \to X$ be a retraction of $g$.
By \rlem{bot-2-out-of-3}, $h$ is a weak equivalence.
Factor $h$ into a map $h' : Y \to Y' \in \Icell$ followed by a map $k : Y' \to X \in \Iinj$.
\[ \xymatrix{ U \ar[d]_i \ar[r]^u &     X \ar[d]     \ar[r]^g &     Y \ar[d]     \ar[r]^{h'} &     Y' \ar[d]     \ar[r]^k &     X \ar[d] \\
              V          \ar[r]   & \po X \amalg_U V \ar[r]   & \po Y \amalg_U V \ar[r]      & \po Y' \amalg_U V \ar[r]   & \po X \amalg_U V
            } \]
By \rprop{pure-we}, $h'$ is pure, hence it is a split monomorphism.
Hence $Y \amalg_U V \to Y' \amalg_U V$ is a split monomorphism.
Since $\I$ is appropriate, $Y' \amalg_U V \to X \amalg_U V$ is pure.
Hence $Y \amalg_U V \to X \amalg_U V$ is pure as well.
By \rlem{top-2-out-of-3}, $X \amalg_U V \to Y \amalg_U V$ is a weak equivalence.
Hence $X \amalg_U V \to Z \amalg_U V$ is a weak equivalence.

Now, it is easy to show by transfinite induction that weak equivalences with cofibrant domains are closed
under pushouts along relative $\I$-cell complexes between cofibrant objects.
\end{proof}

\bibliographystyle{amsplain}
\bibliography{ref}

\end{document}
