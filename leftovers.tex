% 1. Это более общее определение морфизмов алгебраических теорий.
% Оно мне пока не нужно, а определение более сложное, чем для теорий с фиксированными сортами.
% Поэтому я решил его пока не включать, но потом может пригодиться.

There are several equivalent ways of defining essentially algebraic theories (\cite{LPC}, \cite{GAT}, \cite{PHL}, \cite[D 1.3.4]{elephant}).
We use approach introduced in \cite{PHL} under the name of partial Horn theories since it is the most convenient one.
There is a structure of a category on partial Horn theories.
A \emph{generalized morphism} between theories $\mathbb{T}$ and $\mathbb{T}'$ is a model of $\mathbb{T}$ in $\mathcal{C}_{\mathbb{T}'}$,
where $\mathcal{C}_{\mathbb{T}'}$ is the classifying category for $\mathbb{T}'$.
We will define another notion of morphisms between theories, which is more explicit.

\subsection{Morphisms of partial Horn theories}

Let $\mathbb{T}$ be a partial Horn theory.
A \emph{restricted term} of $\mathbb{T}$ is a term $t$ together with a formula $\varphi$.
We denote such a restricted term by $t|_\varphi$.
A \emph{derived sort} of $\mathbb{T}$ is a sequence of sorts $s_1, \ldots s_k$ together with a formula $\varphi$ succh that $FV(\varphi) \subseteq \{ x_1 : s_1, \ldots x_k : s_k \}$.

We will use the following abbreviations:
\begin{align*}
R(t_1|_{\varphi_1}, \ldots t_k|_{\varphi_k}) & \Longleftrightarrow R(t_1, \ldots t_k) \land \varphi_1 \land \ldots \land \varphi_k \\
t|_\varphi = s|_\psi & \Longleftrightarrow t = s \land \varphi \land \psi \\
t|_\varphi\!\downarrow & \Longleftrightarrow t\!\downarrow\!\land \varphi \\
\chi \sststile{}{V} t|_\varphi \cong s|_\psi & \Longleftrightarrow \chi \land t|_\varphi\!\downarrow\,\sststile{}{V} t = s \land \psi \text{ and } \chi \land s|_\psi\!\downarrow\,\sststile{}{V} t = s \land \varphi
\end{align*}

We define morphisms of theories $\mathbb{T}$ and $\mathbb{T}'$ as equivalence classes of functions $h$ satisfying the following conditions:
\begin{enumerate}
\item For every sort $s$ of $\mathbb{T}$, it determines a derived sort $h(s)$ of $\mathbb{T}'$,
\item For every predicate symbol $P : s_1 \times \ldots \times s_k$ of $\mathbb{T}$, it determines a formula $h(P)$ of $\mathbb{T}'$
such that $FV(h(P)) = \{ x^1_1 : s^1_1, \ldots x^{n_1}_1 : s^{n_1}_1, \ldots x^1_k : s^1_k, \ldots x^{n_k}_k : s^{n_k}_k \}$
and the sequent $h(P) \sststile{}{FV(h(P))} \varphi_1 \land \ldots \land \varphi_k$ is derivable,
where $\varphi_i$ are the formulas that correspond to the derived sorts $h(s_i)$ and $FV(\varphi_i) = \{ x^{n_i}_i, \ldots x^{n_i}_i \}$.
\item For every function symbol $\sigma : s_1 \times \ldots \times s_k \to s$ of $\mathbb{T}$,
it determines a sequence of restricted terms $h(\sigma)_1$, \ldots $h(\sigma)_m$ of $\mathbb{T}'$
such that $FV(h(\sigma)_i) = \{ x^1_1 : s^1_1, \ldots x^{n_1}_1 : s^{n_1}_1, \ldots x^1_k : s^1_k, \ldots x^{n_k}_k : s^{n_k}_k \}$
and the sequents $h(\sigma)_i\!\downarrow\ \sststile{}{FV(h(\sigma)_i)} \varphi^i_1 \land \ldots \land \varphi^i_k$
and $\bigwedge_{1 \leq i \leq m} h(\sigma)_i\!\downarrow\ \sststile{}{FV(h(\sigma)_i)} \psi[h(\sigma)_1/y_1, \ldots h(\sigma)_m/y_m]$ are derivable,
where $\varphi_i$ are the formulas that correspond to the derived sorts $h(s_i)$ and $FV(\varphi_i) = \{ x^{n_i}_i, \ldots x^{n_i}_i \}$
and $\psi$ is the formula that correspond to the derived sort $h(s)$ and $FV(\psi) = \{ y_1, \ldots y_m \}$.
\item For every axiom $S$ of $\mathbb{T}$, the sequent $h(S)$ is derivable in $\mathbb{T}'$.
\end{enumerate}

We will say that functions $h$ and $h'$ as described above are equivalent if the following conditions hold:
\begin{enumerate}
\item For every sort $s$ of $\mathbb{T}$, the sequences of sorts that correspond to $h(s)$ and $h'(s)$ are equal
and the formulas $\varphi$ and $\psi$ that correspond to $h(s)$ and $h'(s)$ are equivalent;
that is, the sequents $\varphi \ssststile{}{FV(\varphi) \cup FV(\psi)} \psi$ are derivable.
\item For every predicate symbol $P : s_1 \times \ldots \times s_k$ of $\mathbb{T}$, the formulas $h(P)$ and $h'(P)$ are equivalent.
\item For every function symbol $\sigma : s_1 \times \ldots \times s_k \to s$ of $\mathbb{T}$, restricted terms $h(\sigma)_i$ and $h'(\sigma)_i$ are equivalent for every $i$;
that is, the sequents $\sststile{}{FV(h(\sigma)_i)} h(\sigma)_i \cong h'(\sigma)_i$ are derivable.
\end{enumerate}

The identity morphism is defined in the obvious way.
To define the composition of morphisms, we need to extend the definition of a function $h : \mathbb{T} \to \mathbb{T}'$ to terms and formulas.
Let $t$ be a term of $\mathbb{T}$ of sort $s$ with free variables $x_1 : s_1$, \ldots $x_k : s_k$.
Suppose that $h(s)$ is a sequence $s_1$, \ldots $s_m$ together with a formula $\psi$
and $h(s_i)$ is a sequence $s^1_i$, \ldots $s^{n_i}_i$ together with a formula $\varphi_i$.
Then, by induction on $t$, we define a sequence of restricted terms $h(t)_1$, \ldots $h(t)_m$ of $\mathbb{T}'$
with free variables $x^1_1 : s^1_1, \ldots x^{n_1}_1 : s^{n_1}_1, \ldots x^1_k : s^1_k, \ldots x^{n_k}_k : s^{n_k}_k$.
If $t = x$ is a variable, then let $h(x)_i = x^i$.
% Это я не закончил.

% 2. Это раздел про partial Horn theories из моей старой версии статьи про алгебраические теории типов.
% В основном это шлак, но там есть пара интересных лемм, например про мономорфизмы.

\section{Partial Horn theories}
\label{sec:PHT}

There are several equivalent ways of defining essentially algebraic theories (\cite{LPC}, \cite{GAT}, \cite{PHL}, \cite[D 1.3.4]{elephant}).
We use approach introduced in \cite{PHL} under the name of partial Horn theories since it is the most convenient one.
We define morphisms of partial Horn theories in terms of morphisms of monads and left modules over them.
In this section we review necessary for our development parts of the theory of monads, left modules over them and partial Horn theories.
We also define algebraic dependent type theories as certain partial Horn theories.

\subsection{Monads and left modules over them}

We recall definitions of monads and left modules over a monad.
For our purposes the following definitions (see \cite{manes-algebraic-theories}) will be more convenient than the ordinary ones.
\begin{defn}
A \emph{monad} $(T,\eta,(-)^*)$ on a category $\C$ consists of a function $T : Ob(\C) \to Ob(\C)$,
a function $\eta$ that to each $A \in Ob(\C)$ assign a morphism $\eta_A : A \to T(A)$,
and a function that to each $A,B \in Ob(\C)$ assigns a function $(-)^* : Hom_\C(A,T(B)) \to Hom_\C(T(A),T(B))$, satisfying the following conditions:
\begin{itemize}
\item $\eta_A^* = id_{T(A)}$.
\item For every $\rho : A \to T(B)$, $\rho^* \circ \eta_A = \rho$.
\item For every $\rho : A \to T(B)$, $\sigma : B \to T(C)$, $\sigma^* \circ \rho^* = (\sigma^* \circ \rho)^*$.
\end{itemize}

A \emph{left module} $(M,(-)^\circ)$ over a monad $(T,\eta,(-)^*)$ with values in a category $\D$ consists of a function $M : Ob(\C) \to Ob(\D)$
and a function that to each $A,B \in Ob(\C)$ assigns a function $(-)^\circ : Hom_\C(A,T(B)) \to Hom_\D(M(A),M(B))$, satisfying the following conditions:
\begin{itemize}
\item $\eta_A^\circ = id_{M(A)}$.
\item For every $\rho : A \to T(B)$, $\sigma : B \to T(C)$, $\sigma^\circ \circ \rho^\circ = (\sigma^* \circ \rho)^\circ$.
\end{itemize}
\end{defn}
These data and axioms imply that $T$ and $M$ are functorial: if $f : A \to B$, then we can define $T(f)$ as $(\eta_B \circ f)^*$ and $M(f)$ as $(\eta_B \circ f)^\circ$.
Moreover, $\eta$, $(-)^*$ and $(-)^\circ$ are natural.

\begin{defn}
A morphism of monads $(T,\eta,(-)^*)$ and $(T',\eta',(-)^{*'})$ on $\C$ is a function $\alpha$ that to each $A \in Ob(\C)$ assigns a morphism $\alpha_A : T(A) \to T'(A)$,
satisfying the following conditions:
\begin{itemize}
\item $\alpha_A \circ \eta_A = \eta'_A$.
\item For every $\rho : A \to T(B)$, $\alpha_B \circ \rho^* = (\alpha_B \circ \rho)^{*'} \circ \alpha_A$.
\end{itemize}

Let $(M,(-)^\circ)$ and $(M',(-)^{\circ'})$ be left modules with values in $\D$ over monads $(T,\eta,(-)^*)$ and $(T',\eta',(-)^{*'})$ respectively.
A morphism between them is a pair of functions $(\alpha,\beta)$, where $\alpha$ is a morphism of monads $T$ and $T'$,
and $\beta$ assigns to each $A \in Ob(\C)$ a morphism $\beta_A : M(A) \to M'(A)$,
such that, for every $\rho : A \to T(B)$, $\beta_B \circ \rho^\circ = (\alpha_B \circ \rho)^{\circ'} \circ \beta_A$.
\end{defn}
These data and axioms imply that $\alpha$ and $\beta$ are natural.

Let $\mathcal{S}$ be a set of sorts, and let $(T,\eta,(-)^*)$ be a monad on the category of $\mathcal{S}$-sets.
We think of elements of $T(V)_s$ as terms of sort $s$ with free variables in $V$.
Given $t \in T(V)_s$ and $\rho : V \to T(V')$, we will write $t[\rho] \in T(V')_s$ for $\rho^*(t)$.
Let $(F,(-)^\circ)$ be a left module over $T$ with values in $\Set$.
We think of elements of $F(V)$ as formulas with free variables in $V$.
Given $\varphi \in F(V)$ and $\rho : V \to T(V')$, we will write $\varphi[\rho] \in F(V')$ for $\rho^\circ(\varphi)$.

Let $T : \Set^\mathcal{S} \to \Set^\mathcal{S}$ be a monad.
Then \emph{a free variables structure} on $T$ is a function $FV$ that to each $t \in T(V)_s$ assigns a subset of $V$, that is $FV(t) \subseteq V$, called the set of free variables of $t$.
This function must satisfy the following conditions:
\begin{align*}
FV(\eta(x)) & = x \\
FV(t[\rho]) & = \bigcup_{x \in FV(t)} FV(\rho(x))
\end{align*}

Let $F : \Set^\mathcal{S} \to \Set$ be a left module over $T$.
Then \emph{a free variables structure} on $F$ is a function $FV$ that to each $\varphi \in F(V)$ assigns a subset of $V$, that is $FV(\varphi)$, called the set of free variables of $\varphi$.
This function must satisfy the following condition:
\[ FV(\varphi[\rho]) = \bigcup_{x \in FV(\varphi)} FV(\rho(x)) \]

\emph{A module of formulas} over $T$ is a left module $F$ over $T$ together with a function
    $\land : F(V) \times F(V) \to F(V)$ and a constant $\top \in F(V)$ for every $V \in \Set^\mathcal{S}$, satisfying the following conditions:
\begin{itemize}
\item For every $\rho : V \to T(V')$, $\top[\rho] = \top$.
\item For every $\rho : V \to T(V')$, $(\varphi \land \psi)[\rho] = \varphi[\rho] \land \psi[\rho]$.
\end{itemize}

For every monad $T$ on $\Set^\mathcal{S}$ we define a left module $E$ with values in $\Set$.
For every $V \in \Set^\mathcal{S}$, let $E(V)$ be the set of triples $(s,t,t')$, where $s \in \mathcal{S}$, and $t,t' \in T(V)_s$.
For every $\rho : V \to T(V')$ and $(s,t,t') \in E(V)$, we let $(s,t,t')[\rho] = (s,t[\rho],t'[\rho])$.
We think of $(s,t,t')$ as a formula asserting the equality of terms $t$ and $t'$.
We write $t =_s t'$ (or simply $t = t'$) for $(s,t,t')$.
\emph{A module of formulas with equality} over $T$ is a module $F$ of formulas over $T$ together with a morphism $e : E \to F$.

\begin{defn}[mon-pres]
A \emph{monadic presentation of a partial Horn theory} is a triple $(T,F,\mu)$, where
    $T : \Set^\mathcal{S} \to \Set^\mathcal{S}$ is a finitary monad with a free variables structure,
    $F : \Set^\mathcal{S} \to \Set$ is a finitary module of formulas with equality and a free variables structure, and
    $\mu_V : T(V) \times F(V) \to T(V)$ is a function such that the following conditions hold:
\begin{itemize}
\item For every $\rho : V \to T(V')$, $\mu_V(t,\varphi)[\rho] = \mu_{V'}(t[\rho],\varphi[\rho])$.
\item $\mu_V(t, \top) = t$.
\item $\mu_V(t, \varphi \land \psi) = \mu_V(\mu_V(t, \varphi), \psi)$.
\end{itemize}
A morphism of triples $(T,F,\mu)$ and $(T',F',\mu')$ is a morphism $f$ of left modules $(T,F)$ and $(T',F')$ such that $f$ preserves free variables, equality, $\top$, $\land$ and $\mu$.
The category of monadic presentations of partial Horn theories with $\mathcal{S}$ as the set of sorts is denoted by $\PMnd_\mathcal{S}$.
\end{defn}

\subsection{The category of partial Horn theories}
\label{sec:PHT-rules}

Let $\mathcal{S}$ be a set of sorts, $T : \Set^\mathcal{S} \to \Set^\mathcal{S}$ a monad with a free variables structure,
    and $\mathcal{P}$ a set of predicate symbols together with a function that to each $R \in \mathcal{P}$
    assigns its signature $R : s_1 \times \ldots \times s_n$, where $s_1, \ldots s_n \in \mathcal{S}$.

Let $\mathcal{F}$ be a set of function symbols together with a function that to each $\sigma \in \mathcal{F}$ assigns its signature $\sigma : s_1 \times \ldots \times s_n \to s$, where $s_1, \ldots s_n, s \in \mathcal{S}$.
Then we can define an example of a monad over $\Set^\mathcal{S}$.
For each $V \in \Set^\mathcal{S}$ we can define a set $Term_\mathcal{F}(V)_s$ of terms of sort $s$ inductively:
\begin{itemize}
\item If $x \in V_s$, then $x \in Term_\mathcal{F}(V)_s$.
\item If $\sigma : s_1 \times \ldots \times s_n \to s$ and $t_i \in Term_\mathcal{F}(V)_{s_i}$, then $\sigma(t_1, \ldots t_n) \in Term_\mathcal{F}(V)_s$.
\end{itemize}
If $\rho : V \to Term_\mathcal{F}(V')$, then substitution is defined as follows:
\begin{align*}
x[\rho] & = \rho(x) \\
\sigma(a_1, \ldots a_k)[\rho] & = \sigma(a_1[\rho], \ldots a_k[\rho])
\end{align*}
Thus $Term_\mathcal{F} : \Set^\mathcal{S} \to \Set^\mathcal{S}$ is a monad, which we call the standard monad (over $\mathcal{F}$).

An \emph{atomic formula} with free variables in $V$ is an expression either of the form $t_1 =_s t_2$ (we will usually omit $s$ in the notation),
    where $s \in \mathcal{S}$ and $t_1, t_2 \in T(V)_s$, or of the form $R(t_1, \ldots t_n)$, where $R \in \mathcal{P}$, $R : s_1 \times \ldots \times s_n$ and $t_i \in T(V)_{s_i}$.
A \emph{Horn formula} (over $\mathcal{P}$) with free variables in $V$ is an expression of the form $\varphi_1 \land \ldots \land \varphi_n$ where $\varphi_i$ are atomic formulas.
If $n = 0$, then we write such a formula as $\top$.
The set of Horn formulas with free variables in $V$ is denoted by $Form_\mathcal{P}(V)$.
If $\varphi \in Form_\mathcal{P}(V)$ and $\rho : V \to T(V')$, then we will write $\varphi[\rho]$ for a formula defined as follows:
\begin{align*}
(t = t')[\rho] & = (t[\rho] = t'[\rho]) \\
R(t_1, \ldots t_k)[\rho] & = R(t_1[\rho], \ldots t_k[\rho]) \\
(\varphi_1 \land \ldots \land \varphi_n)[\rho] & = \varphi_1[\rho] \land \ldots \land \varphi_n[\rho]
\end{align*}
Thus $Form_\mathcal{P}$ is a left module over $T$.
Moreover, a free variables structure on $Form_\mathcal{P}$ is defined as follows:
\begin{align*}
FV(t = t') & = FV(t) \cup FV(t') \\
FV(R(t_1, \ldots t_k)) & = FV(t_1) \cup \ldots \cup FV(t_k) \\
FV(\varphi_1 \land \ldots \land \varphi_n) & = FV(\varphi_1) \cup \ldots \cup FV(\varphi_n)
\end{align*}

A \emph{Horn sequent} is an expression of the form $\varphi \sststile{}{V} \psi$, where $\varphi$ and $\psi$ are Horn formulas with free variables in $V$.
We will often write $\varphi_1, \ldots \varphi_n \sststile{}{V} \psi_1, \ldots \psi_k$ instead of $\varphi_1 \land \ldots \land \varphi_n \sststile{}{V} \psi_1 \land \ldots \land \psi_k$.
A \emph{partial Horn theory} is a set of Horn sequents.
The rules of \emph{partial Horn logic} are listed below.
If $\mathcal{A}$ is a partial Horn theory, then a \emph{theorem} of $\mathcal{A}$ is a sequent derivable from $\mathcal{A}$ in this logic.
\begin{center}
$\varphi \sststile{}{V} \varphi$ \axlabel{b1}
\qquad
\AxiomC{$\varphi \sststile{}{V} \psi$}
\AxiomC{$\psi \sststile{}{V} \chi$}
\RightLabel{\axlabel{b2}}
\BinaryInfC{$\varphi \sststile{}{V} \chi$}
\DisplayProof
\qquad
$\varphi \sststile{}{V} \top$ \axlabel{b3}
\end{center}

\medskip
\begin{center}
$\varphi \land \psi \sststile{}{V} \varphi$ \axlabel{b4}
\qquad
$\varphi \land \psi \sststile{}{V} \psi$ \axlabel{b5}
\qquad
\AxiomC{$\varphi \sststile{}{V} \psi$}
\AxiomC{$\varphi \sststile{}{V} \chi$}
\RightLabel{\axlabel{b6}}
\BinaryInfC{$\varphi \sststile{}{V} \psi \land \chi$}
\DisplayProof
\end{center}

\medskip
\begin{center}
$\sststile{}{x} x\!\downarrow$ \axlabel{a1}
\qquad
$x = y \land \varphi \sststile{}{V,x,y} \varphi[y/x]$ \axlabel{a2}
\end{center}

\medskip
\begin{center}
\AxiomC{$\varphi \sststile{}{V} \psi$}
\RightLabel{, $x \in FV(\varphi)$, $t \in T(V')$ \axlabel{a3}}
\UnaryInfC{$\varphi[t/x] \sststile{}{V,V'} \psi[t/x]$}
\DisplayProof
\end{center}
\medskip
Here, $t/x$ denotes a function $\rho : V \to T(V \cup V')$ such that $\rho(x) = t$ and $\rho(y) = y$ if $y \neq x$.

Note that this set of rules is a generalization of the one described in \cite{PHL}.
If $T$ is the standard monad $Term_\mathcal{F}$, then these rules are equivalent to the rules from \cite{PHL}.
In particular, the following sequents are derivable if $x \in FV(t)$:
\begin{align*}
R(t_1, \ldots t_k) & \sststile{}{V} t_i = t_i \axtag{a4} \\
t_1 = t_2 & \sststile{}{V} t_i = t_i \axtag{a4'} \\
t[t'/x]\!\downarrow & \sststile{}{V} t' = t' \axtag{a5}
\end{align*}

We will need the following lemmas from \cite{PHL}:
\begin{lem}[cong-a]
For every $u_i,v_i \in T(V)_{s_i}$ and $t \in T(\{ x_1 : s_1, \ldots x_n : s_n\})_s$,
sequents $u_1 = v_1 \land \ldots \land u_n = v_n \sststile{}{V} t[x_i \mapsto u_i] \cong t[x_i \mapsto v_i]$ are theorems of any theory.
\end{lem}

\begin{lem}
Sequent $y = x \land \varphi[y/x] \sststile{}{V} \varphi$ is a theorem of any theory.
\end{lem}

Using the previous lemma we prove the following fact:

\begin{lem}[cong-b]
For every $u_i,v_i \in T(V)_{s_i}$ and $\varphi \in Form_\mathcal{P}(\{ x_1 : s_1, \ldots x_n : s_n\})$,
sequent $u_1 = v_1 \land \ldots \land u_n = v_n \land \varphi[x_i \mapsto u_i] \sststile{}{V} \varphi[x_i \mapsto v_i]$ is a theorem of any theory.
\end{lem}
\begin{proof}
By the previous lemma we have $y_n = x_n \land \varphi[y_n/x_n] \sststile{}{x_1 : s_1, \ldots x_n : s_n, y_n : s_n} \varphi$ is provable.
If we take $\varphi$ to be equal to $y_n = x_n \land \varphi[y_n/x_n]$, then we get sequent
$y_{n-1} = x_{n-1} \land y_n = x_n \land \varphi[y_n/x_n,y_{n-1}/x_{n-1}] \sststile{}{x_1 : s_1, \ldots x_n : s_n, y_{n-1} : s_{n-1}, y_n : s_n} y_n = x_n \land \varphi[y_n/x_n]$.
By \axref{b2} we get sequent
\[ y_{n-1} = x_{n-1} \land y_n = x_n \land \varphi[y_n/x_n,y_{n-1}/x_{n-1}] \sststile{}{x_1 : s_1, \ldots x_n : s_n, y_{n-1} : s_{n-1}, y_n : s_n} \varphi. \]
Repeating this argument we can conclude that
\[ y_1 = x_1 \land \ldots \land y_n = x_n \land \varphi[y_1/x_1, \ldots y_n/x_n] \sststile{}{x_1 : s_1, \ldots x_n : s_n, y_1 : s_1, y_n : s_n} \varphi. \]
By \axref{a3} we conclude that the required sequent is derivable.
\end{proof}

Now we define a functor $PT : \Set^\mathcal{S} \to \Set^\mathcal{S}$ of restricted terms.
We let $PT(V)_s$ to be the set of expressions $t|_\varphi$ where $t \in T(V)_s$ and $\varphi \in Form_\mathcal{P}(V)$.
If $\varphi = \top$, then we will write $t|_\varphi$ simply as $t$.
If $p \in PT(V)_s$, $p = t|_\varphi$ and $\psi \in Form_\mathcal{P}(V)$, then we will write $p|_\psi$ for $t|_{\varphi \land \psi}$.

We will use the following abbreviations:
\begin{align*}
t\!\downarrow & \Longleftrightarrow t = t \\
\varphi \sststile{}{V} t \leftrightharpoons s & \Longleftrightarrow \varphi \land t\!\downarrow \land s\!\downarrow\,\sststile{}{V} t = s \\
\varphi \sststile{}{V} t \cong s & \Longleftrightarrow \varphi \land t\!\downarrow\,\sststile{}{V} t = s \text{ and } \varphi \land s\!\downarrow\,\sststile{}{V} t = s \\
\varphi \ssststile{}{V} \psi & \Longleftrightarrow \varphi \sststile{}{V} \psi \text{ and } \psi \sststile{}{V} \varphi \\
R(t_1|_{\varphi_1}, \ldots t_k|_{\varphi_k}) & \Longleftrightarrow R(t_1, \ldots t_k) \land \varphi_1 \land \ldots \land \varphi_k \\
t|_\varphi = s|_\psi & \Longleftrightarrow t = s \land \varphi \land \psi \\
t|_\varphi\!\downarrow & \Longleftrightarrow t\!\downarrow\!\land \varphi \\
\chi \sststile{}{V} t|_\varphi \leftrightharpoons s|_\psi & \Longleftrightarrow \chi \land t|_\varphi\!\downarrow, s|_\psi\!\downarrow\,\sststile{}{V} t = s \\
\chi \sststile{}{V} t|_\varphi \cong s|_\psi & \Longleftrightarrow \chi \land t|_\varphi\!\downarrow\,\sststile{}{V} t = s \land \psi \text{ and } \chi \land s|_\psi\!\downarrow\,\sststile{}{V} t = s \land \varphi
\end{align*}

Now we define substitution functions for restricted terms.
For every $\rho : V \to PT(V')$, $t \in T(V)_s$ and $\varphi \in Form_\mathcal{P}(V)$,
we define $t[\rho] \in PT(V')_s$, $\varphi[\rho] \in Form_\mathcal{P}(V')$ and $t_\varphi[\rho] \in PT(V')_s$ as follows:
\begin{align*}
t[\rho] & = t[\rho_1]|_{\bigcup_{x \in FV(t)} \rho_2(x)} \\
R(t_1, \ldots t_k)[\rho] & = R(t_1[\rho], \ldots t_k[\rho]) \\
(\varphi_1 \land \ldots \land \varphi_n)[\rho] & = \varphi_1[\rho] \land \ldots \land \varphi_n[\rho] \\
t|_\varphi[\rho] & = t[\rho]|_{\varphi[\rho]}
\end{align*}
where if $\rho(x) = t|_\varphi$, then $\rho_1(x) = t$ and $\rho_2(x) = \varphi$.
Free variables of $t|_\varphi$ is defined as follows: $FV(t|_\varphi) = FV(t) \cup FV(\varphi)$.

Note that $PT$ is not a monad in general since this substitution does not satisfy axioms.
To fix this we introduce an equivalence relation on sets $PT(V)_s$ and $Form_\mathcal{P}(V)$.
Let $\mathbb{T}$ be a partial Horn theory.
For every $t, t' \in PT(V)_s$, $t \sim t'$ if and only if $FV(t) = FV(t')$ and $\sststile{}{V} t \cong t'$ is a theorem of $\mathbb{T}$.
For every $\varphi, \psi \in Form_\mathcal{P}(V)$, $\varphi \sim \psi$ if and only if $FV(\varphi) = FV(\psi)$ and $\varphi \ssststile{}{V} \psi$ is a theorem of $\mathbb{T}$.
Then let $P(V)_s = PT(V)_s/\!\!\sim$ and $F(V) = Form_\mathcal{P}(V)/\!\!\sim$.
For every $x \in V_s$, $\eta_V(x)$ is the equivalence class of $x|_\top$.
Substitution functions respect equivalence relations, and it is easy to see that they define a structure of a monad and of a left module over it on $T$ and $F$.
For every $t,t' \in T(V)_s$, $e(s,t,t')$ is the equivalence class of $t = t'$.
For every $t \in T(V)_s$ and $\varphi \in F(V)$, let $\mu_V(t,\varphi) = t|_\varphi$.
It is easy to see that $(P,F,\mu)$ satisfies axioms of monadic presentations.
We will call it the monadic presentation of partial Horn theory $\mathbb{T}$ and denote by $P(\mathbb{T})$.

The category of partial Horn theories over $\mathcal{S}$ has tuples $(T,\mathcal{P},\mathcal{A})$ as objects,
    where $T$ is a finitary monad with a free variables structure, $\mathcal{P}$ is a set of predicate symbols and $\mathcal{A}$ is a set of axioms.
Morphisms of partial Horn theories $\mathbb{T}$ and $\mathbb{T}'$ are morphisms of their monadic presentations.
The category of partial Horn theories over $\mathcal{S}$ is denoted by $\Th^T_\mathcal{S}$.

\begin{prop}[mor-def]
Let $\mathbb{T} = (T,\mathcal{P},\mathcal{A})$ and $\mathbb{T}' = (T',\mathcal{P}',\mathcal{A}')$ be partial Horn theories,
    and let $P(\mathbb{T}) = (P,F,\mu)$ and $P(\mathbb{T}') = (P',F',\mu')$ be their monadic presentations.
To construct a morphism of these theories, it is enough to specify the following data:
\begin{itemize}
\item A morphism of monads $\alpha : T \to P'$ that preserves free variables.
\item For every $R \in \mathcal{P}$, $R : s_1 \times \ldots \times s_k$,
    a formula $\beta(R) \in F'(\{ x_1 : s_1, \ldots x_k : s_k \})$ such that $FV(\beta(R)) = \{ x_1, \ldots x_k \}$.
\end{itemize}
Then there is a morphism of left modules $f : (T,Form_\mathcal{P}) \to (T',F')$
    such that $f(\sigma(x_1, \ldots x_k)) = \alpha(\sigma)$ and $f(R(x_1, \ldots x_k)) = \beta(R)$.
If $f$ preserves axioms of $\mathbb{T}$, then it extends to a morphism of theories.
Moreover, there is at most one morphism with these properties.
\end{prop}
\begin{proof}
Morphism $f$ is already defined on terms, and we can define it on formulas as follows:
\begin{align*}
f(a = b) & = f(a) = f(b) \\
f(R(a_1, \ldots a_k)) & = \beta(R)[x_i \mapsto f(a_i)] \\
f(\varphi_1 \land \ldots \land \varphi_n) & = f(\varphi_1) \land \ldots \land f(\varphi_n)
\end{align*}
We also can define $f$ on restricted terms:
\[ f(t|_\varphi) = f(t)|_{f(\varphi)} \]
It is easy to see that $f$ preserves substitution.
Thus to prove that $f$ extends to a morphism of theories, we only need to show that it preserves theorems of $\mathbb{T}$.
By assumption, it preserves axioms, thus we only need to check that application of $f$ preserves inference rules.
This is obvious for \axref{b1}-\axref{b6} and \axref{a1}.
For \axref{a2} and \axref{a3} it follows from the facts that $f(\varphi[t/x]) = f(\varphi)[f(t)/x]$ and $FV(f(\varphi)) = FV(\varphi)$.

Now, let us prove that $f$ is unique.
Let $f$ and $f'$ be morphisms of theories such that $f(t) = f'(t)$ for every $t \in T(V)_s$, and
    $f(R(x_1, \ldots x_k)) = f'(R(x_1, \ldots x_k))$ for every $R \in \mathcal{P}$.
Then we prove that $f = f'$.

Let us prove that $f(\varphi) = f'(\varphi)$ for every $\varphi \in Form_\mathcal{P}(V)$.
It is enough to prove this for atomic formulas $\varphi$.
If $\varphi$ equals to $t = t'$, then $f(\varphi)$ equals to $f(t) = f(t')$ and $f'(\varphi)$ equals to $f'(t) = f'(t')$.
We know that $\sststile{}{V} f(t) \cong f'(t)$ and $\sststile{}{V} f(t') \cong f'(t')$.
Thus by transitivity and symmetry we can conclude that $f(t) = f(t)' \sststile{}{V} f'(t) = f'(t')$.

If $\varphi = R(t_1, \ldots t_k)$, then $f(\varphi) = f(R(x_1, \ldots x_k))[x_i \mapsto f(t_i)]$
    and $f'(\varphi) = f'(R(x_1, \ldots x_k))[x_i \mapsto f'(t_i)]$.
We know that $f(R(x_1, \ldots x_k)) \sststile{}{x_1, \ldots x_k}$ \linebreak $f'(R(x_1, \ldots x_k))$.
Since $FV(f(R(x_1, \ldots x_k))) = \{ x_1, \ldots x_k \}$, by \axref{a3} we can conclude that $f(\varphi) \sststile{}{V} f'(R(x_1, \ldots x_k))[x_i \mapsto f(t_i)]$.
Since $f'(R(x_1, \ldots x_k))[x_i \mapsto f(t_i)] \sststile{}{V} f(t_i)\!\downarrow$, \rlem{cong-b} implies that
    $f'(R(x_1, \ldots x_k))[x_i \mapsto f(t_i)] \sststile{}{V} f'(\varphi)$.
By \axref{b2} we conclude that $f(\varphi) \sststile{}{V} f'(\varphi)$.
The same argument shows that $f'(\varphi) \sststile{}{V} f(\varphi)$.

Finally, it is easy to see that $f(t) = f'(t)$ for every $t \in PT(V)_s$.
Thus $f = f'$.
\end{proof}

Note that if $T$ is the standard monad $Term_\mathcal{F}$, then to define a morphism of monads $T \to T'$,
it is enough to specify for every $\sigma \in \mathcal{F}$, $\sigma : s_1 \times \ldots \times s_k \to s$,
a restricted term $\alpha(\sigma) \in T'(\{ x_1 : s_1, \ldots x_k : s_k \})$ such that $FV(\alpha(\sigma)) = \{ x_1, \ldots x_k \}$.
Then there is a unique morphism of monads $f : T \to T'$ such that $f(\sigma(x_1, \ldots x_k)) = \alpha(\sigma)$.

Now, let us define a category $\Th_\mathcal{S}$ of standard partial Horn theories.
Its objects are tuples $((\mathcal{S},\mathcal{F},\mathcal{P}),\mathcal{A})$, where $\mathcal{F}$ is a set of function symbols,
    $\mathcal{P}$ is a set of relation symbols, and $\mathcal{A}$ is a set of axioms over $(Term_\mathcal{F},Form_\mathcal{P})$.
Morphisms of standard partial Horn theories are morphisms of corresponding partial Horn theories.
Thus $\Th_\mathcal{S}$ is (equivalent to) a full subcategory of $\Th^T_\mathcal{S}$.

\subsection{Models of partial Horn theories}

Given a monad $T : \Set^\mathcal{S} \to \Set^\mathcal{S}$, we define a category of its partial algebras.
A \emph{partial algebra} over $T$ is a pair $(A,\alpha)$, where $A$ is an $\mathcal{S}$-set and $\alpha_V : Hom_{\PSet^\mathcal{S}}(V,A) \to Hom_{\PSet^\mathcal{S}}(T(V),A)$,
    where $\PSet$ is the category of sets and partial functions between them.
This pair must satisfy the following conditions:
\begin{itemize}
\item For every partial function $f : V \to A$, $\alpha_V(f) \circ \eta_V = f$.
\item For every total function $\rho : V \to T(V')$ and every partial function $f : V' \to A$, $\alpha_V(\alpha_{V'}(f) \circ \rho) = \alpha_{V'}(f) \circ \rho^*$.
\end{itemize}
A morphism of partial algebras $(A,\alpha)$ and $(A',\alpha')$ is a total morphism $h : A \to A'$ of $\mathcal{S}$-sets
    such that, for every partial function $f : V \to A$ and every $t \in T(V)_s$, if $\alpha_V(f)(t)$ is defined,
    then $\alpha'_V(h \circ f)(t)$ is also defined and $h(\alpha_V(f)(t)) = \alpha'_V(h \circ f)(t)$.

\begin{lem}[par-alg-str]
If $Term_\mathcal{F}$ is the standard monad, then categories of partial algebras over $Term_\mathcal{F}$
    and partial structures for signature $(\mathcal{S},\mathcal{F},\varnothing)$ as defined in \cite{PHL} are isomorphic.
\end{lem}
\begin{proof}
A partial structure for signature $(\mathcal{S},\mathcal{F},\varnothing)$ is an $\mathcal{S}$-set $A$ together with a collection of partial functions
    $A(\sigma) : A_{s_1} \times \ldots \times A_{s_n} \to A_s$ for every $\sigma \in \mathcal{F}$, $\sigma : s_1 \times \ldots \times s_n \to s$.
Given such partial structure, we define a partial algebra $F(A)$ over $Term_\mathcal{F}$ as $(A,\alpha)$, where $\alpha$ is defined as follows:
\begin{align*}
\alpha_V(f)(x) & = f(x) \\
\alpha_V(f)(\sigma(t_1, \ldots t_n)) & = A(\sigma)(\alpha_V(f)(t_1), \ldots \alpha_V(f)(t_n))
\end{align*}
For every morphism $h : A \to A'$ of partial structures, let $F(h) = h$.

For every partial algebra $(A,\alpha)$, we define a partial structure $G(A,\alpha)$.
Let $G(A,\alpha) = A$ and $G(A,\alpha)(\sigma)(a_1, \ldots a_n) = \alpha_{x_1, \ldots x_n}(x_i \mapsto a_i)(\sigma(x_1, \ldots x_n))$.
For every morphism $h : (A,\alpha) \to (A',\alpha')$ of partial algebras, let $G(h) = h$.
It is easy to see that functors $F$ and $G$ determine isomorphisms of categories.
\end{proof}

If $F : \Set^\mathcal{S} \to \Set$ is a left module of formulas over $T$, then we define a category of its partial algebras.
A \emph{partial algebra} over $(T,F)$ is a partial algebra $(A,\alpha)$ over $T$ together with a function $\beta_V : Hom_{\PSet^\mathcal{S}}(V,A) \to Hom_\Set(F(V),\Omega)$,
    where $\Omega = \{ \top, \bot \}$ is the set of truth-values.
This function must satisfy the following conditions:
\begin{itemize}
\item For every total function $\rho : V \to T(V')$ and every partial function $f : V' \to A$, $\beta_V(\alpha_{V'}(f) \circ \rho) = \beta_{V'}(f) \circ \rho^\circ$.
\item For every partial function $f : V \to A$, $\beta_V(f)(\top) = \top$.
\item For every partial function $f : V \to A$, $\beta_V(f)(\varphi \land \psi) = \beta_V(f)(\varphi) \land \beta_V(f)(\psi)$,
    where $P \land Q = \top$ if and only if $P = \top$ and $Q = \top$.
\end{itemize}
A morphism of partial algebras $(A,\alpha,\beta)$ and $(A',\alpha',\beta')$ is a morphism $h$ of partial algebras $(A,\alpha)$ and $(A',\alpha')$
    such that, for every partial function $f : V \to A$ and every $\varphi \in F(V)$, if $\beta_V(f)(\varphi) = \top$, then $\beta'_V(h \circ f)(\varphi) = \top$.

We define a function $\epsilon_V : Hom_{\PSet^\mathcal{S}}(V,A) \to Hom_\Set(E(V),\Omega)$ for the left module $E$ of equality.
Let $\epsilon_V(e(s,t,t')) = \top$ if and only if $\alpha_V(f)(t)$ and $\alpha_V(f)(t')$ are defined and equal.
If $F$ is a left module of formulas with equality over $T$, then we say that a partial algebra $(A,\alpha,\beta)$ is standard
    if, for every partial function $f : V \to A$, $e_V \circ \beta_V(f) = \epsilon_V(f)$, where $e_V : E(V) \to F(V)$.

\begin{lem}[par-alg-pred]
If $Term_\mathcal{F}$ is the standard monad and $Form_\mathcal{P}$ is the left module of Horn formulas,
    then categories of partial algebras over $(Term_\mathcal{F},Form_\mathcal{P})$ and partial structures for signature $(\mathcal{S},\mathcal{F},\mathcal{P})$ are isomorphic.
\end{lem}
\begin{proof}
A partial structure for signature $(\mathcal{S},\mathcal{F},\mathcal{P})$ is a partial structure $A$ for signature $(\mathcal{S},\mathcal{F},\varnothing)$
    together with a relation $A(R) \subseteq A_{s_1} \times \ldots \times A_{s_n}$ for every $R \in \mathcal{P}$, $R : s_1 \times \ldots \times s_n$.
Given such partial structure, we define a partial algebra $F(A)$ over $(Term_\mathcal{F},Form_\mathcal{P})$ as $(A,\alpha,\beta)$,
    where $(A,\alpha)$ is the partial algebra defined in \rlem{par-alg-str}, and $\beta$ defined as follows:
\begin{align*}
\beta_V(f)(t =_s t') & = \epsilon_V(e(s,t,t')) \\
\beta_V(f)(R(t_1, \ldots t_n)) & = \top \text{ if and only if } (\alpha_V(f)(t_1), \ldots \alpha_V(f)(t_n)) \in A(R) \\
\beta_V(f)(\varphi_1 \land \ldots \land \varphi_n) & = \beta_V(f)(\varphi_1) \land \ldots \land \beta_V(f)(\varphi_n)
\end{align*}
For every morphism $h : A \to A'$ of partial structures, let $F(h) = h$.

For every partial algebra $(A,\alpha,\beta)$, we define a partial structure $G(A,\alpha,\beta)$.
We already defined interpretation of function symbols in \rlem{par-alg-str}.
For every $R \in \mathcal{P}$, let $G(A,\alpha,\beta)(R) = \{ (a_1, \ldots a_n)\ |\ \beta_{x_1, \ldots x_n}(x_i \mapsto a_i)(R(x_1, \ldots x_n)) = \top \}$.
For every morphism $h : (A,\alpha,\beta) \to (A',\alpha',\beta')$ of partial algebras, let $G(h) = h$.
It is easy to see that functors $F$ and $G$ determine isomorphisms of categories.
\end{proof}

If $(T,F,\mu)$ is a monadic presentation, then we define a category of its partial algebras as a full subcategory of partial algebras over $(T,F)$.
A partial algebra $(A,\alpha,\beta)$ over $(T,F)$ is a partial algebra over $(T,F,\mu)$ if, for every partial function $f : V \to A$,
    every $t \in T(V)_s$ and every $\varphi \in F(V)$, $\alpha_V(f)(\mu_V(t,\varphi))$ is defined if and only if $\alpha_V(f)(t)$ is defined and $\beta_V(f)(\varphi) = \top$,
    and $\alpha_V(f)(\mu_V(t,\varphi))$ equals to $\alpha_V(f)(t)$ when it is defined.
The category of partial algebras over $(T,F,\mu)$ will be denoted by $\PAlg{(T,F,\mu)}$.

\begin{lem}
If $Term_\mathcal{F}$ is the standard monad and $\mathbb{T} = (Term_\mathcal{F},\mathcal{P},\mathcal{A})$ is a partial Horn theory,
    then categories of partial algebras over $P(\mathbb{T})$ and models of $\mathbb{T}$ as defined in \cite{PHL} are isomorphic.
\end{lem}
\begin{proof}
Using \rlem{par-alg-pred}, models of $\mathbb{T}$ can be described as partial algebras $(A,\alpha',\beta')$ over $(Term_\mathcal{F},Form_\mathcal{P})$
    such that, for every derivable sequent $\varphi \ssststile{}{V} \psi$ of $\mathbb{T}$ and every partial function $f : V \to A$, $\beta'_V(f)(\varphi) = \beta'_V(f)(\psi)$.

Let $(A,\alpha,\beta)$ be a partial algebra over $P(\mathbb{T})$.
Then we define a partial algebra $F(A,\alpha,\beta)$ over $(Term_\mathcal{F},Form_\mathcal{P})$.
Let $F(A,\alpha,\beta) = (A,\alpha',\beta')$, where $\alpha'_V(f)(t) = \alpha_V(f)([t|_\top]_\sim)$ and $\beta'_V(f)(\varphi) = \alpha_V(f)([\varphi]_\sim)$,
    where $[t|_\top]_\sim$ and $[\varphi]_\sim$ are equivalence classes of $t_\top$ and $\varphi$ in $P(V)$ and $F(V)$ respectively.
Then $F(A,\alpha,\beta)$ is a model of $\mathbb{T}$.
Indeed, if $\varphi \ssststile{}{V} \psi$ is a theorem of $\mathbb{T}$, then $\varphi' \ssststile{}{V} \psi'$
    is also a theorem of $\mathbb{T}$, where $\varphi' = \varphi \land x_1 \land \ldots \land x_n$, $\psi' = \psi \land y_1 \land \ldots \land y_k$,
    $x_1, \ldots x_n$ is the set of free variables of $\psi$, and $y_1, \ldots y_k$ is the set of free variables of $\varphi$.
It follows that $[\varphi']_\sim = [\psi']_\sim$; hence $\beta'_V(f)(\varphi') = \beta'_V(f)(\varphi')$.
But $\beta'_V(f)(\varphi) = \beta'_V(f)(\varphi')$ and $\beta'_V(f)(\psi) = \beta'_V(f)(\psi')$; hence $F(A,\alpha,\beta)$ is a model of $\mathbb{T}$.
If $h$ is a morphism of partial algebras over $P(\mathbb{T})$, then let $F(h) = h$.

Let $(A,\alpha',\beta')$ be a model of $\mathbb{T}$.
Then we define a partial algebra $G(A,\alpha',\beta')$ over $P(\mathbb{T})$.
Let $G(A,\alpha',\beta') = (A,\alpha,\beta)$, where $\beta_V(f)([\varphi]_\sim) = \beta'_V(f)(\varphi)$, and $\alpha_V(f)([t|_\varphi]_\sim)$ is defined
    if and only if $\alpha'_V(f)(t)$ is defined and $\beta'_V(f)(\varphi) = \top$, and in this case $\alpha_V(f)([t|_\varphi]_\sim) = \alpha'_V(f)(t)$.
These definitions do not depend on the choice of a representative of the equivalence classes.
Indeed, if $\varphi \sim \psi$, then $\varphi \ssststile{}{V} \psi$ is a theorem of $\mathbb{T}$,
    and in this case $\beta'_V(f)(\varphi) = \beta'_V(f)(\psi)$ since $A$ is a model of $\mathbb{T}$.
The same argument shows that the definition of $\alpha$ does not depend on the choice of a representative of $[t|_\varphi]_\sim$.
If $h$ is a morphism of models, then let $G(h) = h$.
It is easy to see that functors $F$ and $G$ determine isomorphisms of categories.
\end{proof}

Finally, we prove a proposition which shows that if $\mathbb{T}'$ is a partial Horn theory under $\mathbb{T}$,
    then we can think of models of $\mathbb{T}'$ as models of $\mathbb{T}$ with additional structure.

\begin{prop}[func-mod]
For every morphism of monadic presentations $f : (P,F,\mu) \to (P',F',\mu')$, there is a faithful functor $f^* : \PAlg{(P',F',\mu')} \to \PAlg{(P,F,\mu)}$
    such that $id_{(P,F,\mu)}^*$ is the identity functor and $(g \circ f)^* = f^* \circ g^*$.
\end{prop}
\begin{proof}
If $(A,\alpha,\beta)$ is a partial algebra over $(P',F',\mu')$, then let $f^*(A,\alpha,\beta) = (A, e \mapsto \alpha_V(e) \circ f_V, e \mapsto \beta_V(e) \circ f_V)$.
If $h : (A,\alpha,\beta) \to (A',\alpha',\beta')$ is a morphism of partial algebras, then let $f^*(h) = h$.
It is easy to see that these definitions satisfy all required conditions.
\end{proof}

\subsection{Properties of the category of theories}
\label{sec:prop}

Now we prove a few properties of the category of theories.
We begin with a proof of the existence of colimits.

\begin{prop}[th-cocomplete]
Category $\Th_\mathcal{S}$ is cocomplete.
\end{prop}
\begin{proof}
First, let $\{ \mathbb{T}_i \}_{i \in S} = \{ ((\mathcal{S},\mathcal{F}_i,\mathcal{P}_i),\mathcal{A}_i) \}_{i \in S}$ be a set of theories.
Then we can define its coproduct $\coprod\limits_{i \in S} \mathbb{T}_i$ as the theory with $\coprod\limits_{i \in S} \mathcal{F}_i$ as the set of function symbols and $\coprod\limits_{i \in S} \mathcal{A}_i$ as the set of axioms.
Morphisms $f_i : \mathbb{T}_i \to \coprod\limits_{i \in S} \mathbb{T}_i$ are defined in the obvious way.
If $g_i : \mathbb{T}_i \to X$ is a collection of morphisms, then \rprop{mor-def} implies that there is a unique morphism $g : \coprod\limits_{i \in S} \mathbb{T}_i \to X$
    satisfying $g(\sigma(x_1, \ldots x_n)) = g_i(\sigma(x_1, \ldots x_n))$ and $f(R(x_1, \ldots x_n)) = f_i(R(x_1, \ldots x_n))$
    for every $\sigma \in \mathcal{F}_i$ and $R \in \mathcal{P}_i$.

Now, let $f,g : \mathbb{T}_1 \to \mathbb{T}_2$ be a pair of morphisms of theories.
Then we can define their coequalizer $\mathbb{T}$ as the theory with the same set of function and predicate symbols as $\mathbb{T}_2$ and the set of axioms which consists of the axioms of $\mathbb{T}_2$
together with $\sststile{}{x_1, \ldots x_n} f(\sigma(x_1, \ldots x_n)) \cong g(\sigma(x_1, \ldots x_n))$ for each function symbols $\sigma$ of $\mathbb{T}_1$
and $f(R(x_1, \ldots x_n)) \ssststile{}{x_1, \ldots x_n} g(R(x_1, \ldots x_n))$ for each predicate symbols $R$ of $\mathbb{T}_1$.
Then we can define $e : \mathbb{T}_2 \to \mathbb{T}$ as identity function on terms and formulas.
By \rprop{mor-def}, $e \circ f = e \circ g$.
If $h : \mathbb{T}_2 \to X$ is such that $h \circ f = h \circ g$, then it extends to a morphism $\mathbb{T} \to X$ since additional axioms are preserved by the assumption on $h$.
This extension is unique since $e$ is an epimorphism.
\end{proof}

Now we give a characterization of monomorphisms.

\begin{prop}[mono]
A morphism of theories $f : \mathbb{T}_1 \to \mathbb{T}_2$ is a monomorphism if and only if, for every sequent $\varphi \sststile{}{V} \psi$ of $\mathbb{T}_1$,
if $f(\varphi) \sststile{}{V} f(\psi)$ is a theorem of $\mathbb{T}_2$, then $\varphi \sststile{}{V} \psi$ is a theorem of $\mathbb{T}_1$.
\end{prop}
\begin{proof}
First, let us prove the ``if'' part.
Let $g,h : \mathbb{T} \to \mathbb{T}_1$ be a pair of morphisms such that $f \circ g = f \circ h$.
If $t \in RTerm_\Sigma(V)_s$, then $\sststile{}{V} f(g(t)) \cong f(h(t))$; hence $\sststile{}{V} g(t) \cong h(t)$.
If $\varphi \in Form_\mathcal{P}(V)$, then $f(g(\varphi)) \ssststile{}{V} f(h(\varphi))$; hence $g(\varphi) \ssststile{}{V} h(\varphi)$.
Thus $g = h$.

Now, let us prove the ``only if'' part.
Suppose that $f$ is a monomorphism.
Let $\varphi \sststile{}{V} \psi$ be a sequent of $\mathbb{T}_1$ such that $f(\varphi) \sststile{}{V} f(\psi)$ is a theorem of $\mathbb{T}_2$.
Let $\mathbb{T}$ be a theory which consists of a single predicate symbol $R : s_1 \times \ldots \times s_n \times s'_1 \times \ldots \times s'_k$
where $s_1, \ldots s_n$ are sorts of variables in $FV(\varphi)$ and $s'_1, \ldots s'_k$ are sorts of variables in $FV(\psi)$.
Let $g : \mathbb{T} \to \mathbb{T}_1$ be a morphism defined by $g(R(x_1, \ldots x_n, y_1, \ldots y_k)) = \varphi \land y_1\!\downarrow \land \ldots \land y_k\!\downarrow$ and
let $h : \mathbb{T} \to \mathbb{T}_1$ be a morphism defined by $h(R(x_1, \ldots x_n, y_1, \ldots y_k)) = \varphi \land \psi$.
By \rprop{mor-def}, $f \circ g = f \circ h$, hence $g = h$ which implies that $\varphi \sststile{}{V} \psi$.
\end{proof}

Let $\mathbb{T} = ((\mathcal{S},\mathcal{F},\mathcal{P}),\mathcal{A})$ and $\mathbb{T}' = ((\mathcal{S}',\mathcal{F}',\mathcal{P}'),\mathcal{A}')$ be a pair of theories.
Then we say that $\mathbb{T}'$ is a \emph{subtheory} of $\mathbb{T}$ if $\mathcal{S}' \subseteq \mathcal{S}$, $\mathcal{F}' \subseteq \mathcal{F}$, $\mathcal{P}' \subseteq \mathcal{P}$ and $\mathcal{A}' \subseteq \mathcal{A}$.
If $\mathbb{T}'$ is a subtheory of a theory $\mathbb{T}$, then we often need to know when a theorem of $\mathbb{T}$ is a theorem of $\mathbb{T}'$.
The lemma below gives us a simple criterion for this.
First, we need to introduce a bit of notation.
Let $t$ is a term over the signature of $\mathbb{T}$ such that there is no subterm of a sort that does not belong to $\mathcal{S}'$.
Then we define a term $Ret(t)$ over the signature of $\mathbb{T}'$ as follows:
\begin{align*}
Ret(x) & = x \\
Ret(\sigma(t_1, \ldots t_n)) & = \sigma(Ret(t_1), \ldots Ret(t_n)) \text{, if $\sigma \in \mathcal{F}'$} \\
Ret(\sigma(t_1, \ldots t_n)) & = x_s \text{, if $\sigma \notin \mathcal{F}'$ and $\sigma : s_1 \times \ldots \times s_n \to s$}
\end{align*}
where $x_s$ is a variable of sort $s$ that is not a free variable of $t$.

If $\varphi$ is an atomic formula over the signature of $\mathbb{T}$, then we define a formula $Ret(\varphi)$ over the signature of $\mathbb{T}'$ as follows:
\begin{align*}
Ret(t = t') & = (Ret(t) = Ret(t')) \text{, if $Ret(t)$ and $Ret(t')$ are defined} \\
Ret(R(t_1, \ldots t_n)) & = R(Ret(t_1), \ldots Ret(t_n)) \text{, if $Ret(t_i)$ is defined for every $i$} \\
Ret(\varphi) & = \top \text{, otherwise}
\end{align*}
For an arbitrary Horn formula $\varphi$ we define $Ret(\varphi)$ as follows:
\[ Ret(\varphi_1 \land \ldots \land \varphi_n) = Ret(\varphi_1) \land \ldots \land Ret(\varphi_n) \]
For every restricted term $t|_\varphi$, let $Ret(t|_\varphi) = Ret(t)|_{Ret(\varphi)}$.
If $S$ is sequent $\varphi \sststile{}{V} \psi$ in the signature of $\mathbb{T}$,
then we define sequent $Ret(S)$ in the signature of $\mathbb{T}'$ as $Ret(\varphi) \sststile{}{V \cup FV(Ret(\varphi)) \cup FV(Ret(\psi))} Ret(\psi)$.

\begin{lem}[subtheory]
Let $\mathbb{T}'$ be a subtheory of $\mathbb{T}$.
Suppose that, for every axiom $S$ of $\mathbb{T}$, $Ret(S)$ is a theorem of $\mathbb{T}'$.
Then if a sequent in the signature of $\mathbb{T}'$ is provable in $\mathbb{T}$, then it is also provable in $\mathbb{T}'$.
\end{lem}
\begin{proof}
If $S$ is a sequent in the signature of $\mathbb{T}'$, then $Ret(S) = S$.
Thus we only need to prove that if $S$ is a theorem of $\mathbb{T}$, then $Ret(S)$ is a theorem of $\mathbb{T}'$.
For axioms this is true by assumption.
We need to check that $Ret(-)$ preserves inference rules.
This is clearly true for rules \axref{b1}-\axref{b6} and \axref{a1}.

Let us consider rule \axref{a2}.
Let $S$ equals $x = y \land \varphi \sststile{}{x:s,y:s,V} \varphi[y/x]$.
Note that $Ret(\varphi[y/x])$ is defined if and only if $Ret(\varphi)$ is defined, and in this case $Ret(\varphi[y/x]) = Ret(\varphi)[y/x]$.
Thus $Ret(S)$ is either of the form $x = y \land Ret(\varphi) \sststile{}{x:s,y:s,V,FV(Ret(\varphi))} Ret(\varphi)[y/x]$,
or of the form $x = y \sststile{}{x:s,y:s,V} \top$, or of the form $\top \sststile{}{x:s,y:s,V} \top$.
In all of these cases $Ret(S)$ is a theorem of $\mathbb{T}'$.

Finally, let us consider rule \axref{a3}.
To prove that it preserves the required property, it is enough to show that $\varphi$ is a formula of $(\mathcal{S}',\mathcal{F}',\mathcal{P}')$ if and only if $\varphi[t/x]$ is.
If $x \notin FV(\varphi)$, then $\varphi = \varphi[t/x]$.
Suppose that $x \in FV(\varphi)$ and $\varphi$ is a formula of $(\mathcal{S}',\mathcal{F}',\mathcal{P}')$.
If $x$ has sort $s$, then $s \in \mathcal{S}'$.
We need to show that a term of sort $s$ is a term of $(\mathcal{S}',\mathcal{F}',\mathcal{P}')$.
But this follows from the assumption on the set of function symbols.
\end{proof}

Sometimes it is convenient to have a sort which consists of a single element.
Let $\mathcal{S}$ be a set of sorts and let $s_0$ be a sort in $\mathcal{S}$.
Then we define a theory $\mathbb{T}_{s_0}$ which consists of a single function symbol $\emptyCtx : s_0$
    and two axioms: $\sststile{}{} \emptyCtx\!\downarrow$ and $\sststile{}{x} x = \emptyCtx$.
Then, for every theory $\mathbb{T} \in \Th_\mathcal{S}$, there is at most one morphism from $\mathbb{T}_{s_0}$ to $\mathbb{T}$.
If such morphism exists, we will say that $s_0$ is \emph{trivial} in $\mathbb{T}$.
Thus $\mathbb{T}_{s_0}/\Th_\mathcal{S}$ is (equivalent to) a full subcategory of $\Th_\mathcal{S}$.

As an application of the previous results we will prove that adding a trivial sort does not change the category of theories.
Every theory $\mathbb{T} \in \Th_\mathcal{S}$ is naturally a theory in $\Th_{\mathcal{S} \amalg \{ s_0 \}}$.
Thus we have a functor $i : \Th_\mathcal{S} \to \mathbb{T}_{s_0}/\Th_{\mathcal{S} \amalg \{ s_0 \}}$ such that $i(\mathbb{T}) = \mathbb{T} \amalg \mathbb{T}_{s_0}$.
\begin{prop}[triv-sort]
Functor $i : \Th_\mathcal{S} \to \mathbb{T}_{s_0}/\Th_{\mathcal{S} \amalg \{ s_0 \}}$ is an equivalence of categories.
\end{prop}
\begin{proof}
Let $\mathbb{T}_1,\mathbb{T}_2 \in \Th_\mathcal{S}$ be theories with $P(\mathbb{T}_i) = (T_i,F_i,\mu_i)$, $i = 1,2$.
Let $\alpha,\beta : \mathbb{T}_1 \to \mathbb{T}_2$ be morphisms such that $i(\alpha) = i(\beta)$.
Then, for every $t \in T_1$, sequent $\sststile{}{V} i(\alpha)(t) \cong i(\beta)(t)$ is a theorem of $i(\mathbb{T}_2)$.
Since $\mathbb{T}_2$ is (isomorphic to) a subtheory of $i(\mathbb{T}_2)$, by \rlem{subtheory}, sequent $\sststile{}{V} \alpha(t) \cong \beta(t)$ is a theorem of $\mathbb{T}_2$.
Analogously, we can show that $\alpha(\varphi) \ssststile{}{V} \beta(\varphi)$ is a theorem of $\mathbb{T}_2$ for every $\varphi \in F_1$.
Thus $i$ is faithful.

Let $\alpha : i(\mathbb{T}_1) \to i(\mathbb{T}_2)$ be a morphism.
For every $t \in T_1(V)_s$, let $\beta(t) = Ret(\alpha(t))$ and, for every $\varphi \in F_1(V)$, let $\beta(\varphi) = Ret(\alpha(\varphi))$.
Since $Ret$ preserves substitution, $\land$ and $\top$, this defines a morphism $\beta : \mathbb{T}_1 \to \mathbb{T}_2$.
Since $s_0$ is trivial in $i(\mathbb{T}_2)$, $Ret(t) = t$ and $Ret(\varphi) = \varphi$ for every restricted term $t$ and every formula $\varphi$.
Thus $i(\beta) = \alpha$; hence $i$ is full.

Let $\mathbb{T} \in \Th_{\mathcal{S} \amalg \{ s_0 \}}$ be a theory with trivial $s_0$.
Then we define a theory $\mathbb{T}' \in \Th_\mathcal{S}$.
It has a predicate symbol $R : s'_1 \times \ldots \times s'_n$ for every predicate symbol $R : s_1 \times \ldots \times s_n$ of $\mathbb{T}$,
    where $s'_1, \ldots s'_n$ is the subsequence of $s_1, \ldots s_n$ consisting of sorts from $\mathcal{S}$.
It has a function symbol $\sigma : s'_1 \times \ldots \times s'_n \to s$ for every function symbol
    $\sigma : s_1 \times \ldots \times s_n \to s$ of $\mathbb{T}$ such that $s \in \mathcal{S}$.
Also, for every function symbol $\sigma : s_1 \times \ldots \times s_n \to s_0$ of $\mathbb{T}$,
    there is a predicate symbol $R_\sigma : s_1' \times \ldots \times s'_n$ in $\mathbb{T}'$.

For every term $t$ of $\mathbb{T}$ of a sort from $\mathcal{S}$, we can define a term $r(t)$ of $\mathbb{T}'$.
Term $r(t)$ is obtained from $t$ by omitting subterms of sort $s_0$.
For every formula $\varphi$ of $\mathbb{T}$, we can define a formula $r(\varphi)$ of $\mathbb{T}'$:
\begin{align*}
r(t =_{s_0} t') & = \top \\
r(t =_s t') & = (r(t) =_s r(t')) \\
r(R(t_1, \ldots t_n)) & = R(r(t'_1), \ldots r(t'_n)) \\
r(\varphi_1 \land \ldots \land \varphi_n) & = r(\varphi_1) \land \ldots \land r(\varphi_n)
\end{align*}
where $t'_1, \ldots t'_n$ is the subsequence of $t_1, \ldots t_n$ consisting of the terms of sorts from $\mathcal{S}$.
Axioms of $\mathbb{T}'$ are sequents of the form $r(\varphi) \sststile{}{FV(r(\varphi)) \cup FV(r(\psi))} r(\psi)$ for every axiom $\varphi \sststile{}{V} \psi$ of $\mathbb{T}$.
It is easy to see that $i(\mathbb{T}')$ is isomorphic to $\mathbb{T}$.
Thus $i$ is essentially surjective on objects.
\end{proof}
