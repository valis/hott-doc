\documentclass[reqno]{amsart}

\usepackage{amssymb}
\usepackage{hyperref}
\usepackage{mathtools}
\usepackage[all]{xy}
\usepackage{verbatim}
\usepackage{ifthen}
\usepackage{xargs}
\usepackage{bussproofs}
\usepackage{turnstile}
\usepackage{etex}

\hypersetup{colorlinks=true,linkcolor=blue}

\renewcommand{\turnstile}[6][s]
    {\ifthenelse{\equal{#1}{d}}
        {\sbox{\first}{$\displaystyle{#4}$}
        \sbox{\second}{$\displaystyle{#5}$}}{}
    \ifthenelse{\equal{#1}{t}}
        {\sbox{\first}{$\textstyle{#4}$}
        \sbox{\second}{$\textstyle{#5}$}}{}
    \ifthenelse{\equal{#1}{s}}
        {\sbox{\first}{$\scriptstyle{#4}$}
        \sbox{\second}{$\scriptstyle{#5}$}}{}
    \ifthenelse{\equal{#1}{ss}}
        {\sbox{\first}{$\scriptscriptstyle{#4}$}
        \sbox{\second}{$\scriptscriptstyle{#5}$}}{}
    \setlength{\dashthickness}{0.111ex}
    \setlength{\ddashthickness}{0.35ex}
    \setlength{\leasturnstilewidth}{2em}
    \setlength{\extrawidth}{0.2em}
    \ifthenelse{%
      \equal{#3}{n}}{\setlength{\tinyverdistance}{0ex}}{}
    \ifthenelse{%
      \equal{#3}{s}}{\setlength{\tinyverdistance}{0.5\dashthickness}}{}
    \ifthenelse{%
      \equal{#3}{d}}{\setlength{\tinyverdistance}{0.5\ddashthickness}
        \addtolength{\tinyverdistance}{\dashthickness}}{}
    \ifthenelse{%
      \equal{#3}{t}}{\setlength{\tinyverdistance}{1.5\dashthickness}
        \addtolength{\tinyverdistance}{\ddashthickness}}{}
        \setlength{\verdistance}{0.4ex}
        \settoheight{\lengthvar}{\usebox{\first}}
        \setlength{\raisedown}{-\lengthvar}
        \addtolength{\raisedown}{-\tinyverdistance}
        \addtolength{\raisedown}{-\verdistance}
        \settodepth{\raiseup}{\usebox{\second}}
        \addtolength{\raiseup}{\tinyverdistance}
        \addtolength{\raiseup}{\verdistance}
        \setlength{\lift}{0.8ex}
        \settowidth{\firstwidth}{\usebox{\first}}
        \settowidth{\secondwidth}{\usebox{\second}}
        \ifthenelse{\lengthtest{\firstwidth = 0ex}
            \and
            \lengthtest{\secondwidth = 0ex}}
                {\setlength{\turnstilewidth}{\leasturnstilewidth}}
                {\setlength{\turnstilewidth}{2\extrawidth}
        \ifthenelse{\lengthtest{\firstwidth < \secondwidth}}
            {\addtolength{\turnstilewidth}{\secondwidth}}
            {\addtolength{\turnstilewidth}{\firstwidth}}}
        \ifthenelse{\lengthtest{\turnstilewidth < \leasturnstilewidth}}{\setlength{\turnstilewidth}{\leasturnstilewidth}}{}
    \setlength{\turnstileheight}{1.5ex}
    \sbox{\turnstilebox}
    {\raisebox{\lift}{\ensuremath{
        \makever{#2}{\dashthickness}{\turnstileheight}{\ddashthickness}
        \makehor{#3}{\dashthickness}{\turnstilewidth}{\ddashthickness}
        \hspace{-\turnstilewidth}
        \raisebox{\raisedown}
        {\makebox[\turnstilewidth]{\usebox{\first}}}
            \hspace{-\turnstilewidth}
            \raisebox{\raiseup}
            {\makebox[\turnstilewidth]{\usebox{\second}}}
        \makever{#6}{\dashthickness}{\turnstileheight}{\ddashthickness}}}}
        \mathrel{\usebox{\turnstilebox}}}

\newcommand{\axlabel}[1]{(#1) \phantomsection \label{ax:#1}}
\newcommand{\axtag}[1]{\label{ax:#1} \tag{#1}}
\newcommand{\axref}[1]{(\hyperref[ax:#1]{#1})}

\newcommand{\newref}[4][]{
\ifthenelse{\equal{#1}{}}{\newtheorem{h#2}[hthm]{#4}}{\newtheorem{h#2}{#4}[#1]}
\expandafter\newcommand\csname r#2\endcsname[1]{#3~\ref{#2:##1}}
\expandafter\newcommand\csname R#2\endcsname[1]{#4~\ref{#2:##1}}
\expandafter\newcommand\csname n#2\endcsname[1]{\ref{#2:##1}}
\newenvironmentx{#2}[2][1=,2=]{
\ifthenelse{\equal{##2}{}}{\begin{h#2}}{\begin{h#2}[##2]}
\ifthenelse{\equal{##1}{}}{}{\label{#2:##1}}
}{\end{h#2}}
}

\newref[section]{thm}{Theorem}{Theorem}
\newref{lem}{Lemma}{Lemma}
\newref{prop}{Proposition}{Proposition}
\newref{cor}{Corollary}{Corollary}
\newref{cond}{Condition}{Condition}

\theoremstyle{definition}
\newref{defn}{Definition}{Definition}
\newref{example}{Example}{Example}

\theoremstyle{remark}
\newref{remark}{Remark}{Remark}

\newcommand{\fs}[1]{\mathrm{#1}}
\newcommand{\Term}{\fs{Term}}
\newcommand{\FV}{\fs{FV}}
\newcommand{\subst}{\fs{subst}}
\newcommand{\Id}{\fs{Id}}
\newcommand{\refl}{\fs{refl}}
\newcommand{\El}{\fs{El}}
\newcommand{\emptyCtx}{\mathbf{1}}
\newcommand{\ft}{\fs{ft}}
\newcommand{\ty}{\fs{ty}}
\newcommand{\ctx}{\fs{ctx}}
\newcommand{\tm}{\fs{tm}}

\newcommand{\cat}[1]{\mathbf{#1}}
\newcommand{\Th}{\cat{Th}}
\newcommand{\algtt}{\cat{TT}}

\numberwithin{figure}{section}

\newcommand{\pb}[1][dr]{\save*!/#1-1.2pc/#1:(-1,1)@^{|-}\restore}
\newcommand{\po}[1][dr]{\save*!/#1+1.2pc/#1:(1,-1)@^{|-}\restore}

\begin{document}

\title{Syntax of Algebraic Dependent Type Theories}

\author{Valery Isaev}

\begin{abstract}
TODO
\end{abstract}

\maketitle

\section{Introduction}

TODO

\section{Syntax}

In this section, we define a syntax for algebraic dependent type theories which is closer to the usual presentation.

\subsection{Preliminaries}

Let us recall definitions from \cite{PHL} and \cite{alg-tt}.
A many sorted first-order signature $(\mathcal{S},\mathcal{F},\mathcal{P})$ consists of a set $\mathcal{S}$ of sorts,
a set $\mathcal{F}$ of function symbols and a set $\mathcal{P}$ of predicate symbols.
Each function symbol $\sigma$ is equipped with a signature of the form $\sigma : s_1 \times \ldots \times s_k \to s$, where $s_1$, \ldots $s_k$, $s$ are sorts.
Each predicate symbol $R$ is equipped with a signature of the form $R : s_1 \times \ldots \times s_k$.
If $V$ is an $\mathcal{S}$-set, then the $\mathcal{S}$-set of terms of $T$ with free variables in $V$ will be denoted by $\Term^e_T(V)$.

An atomic formula is an expression either of the form $t_1 = t_2$ or of the form $R(t_1, \ldots t_n)$,
where $R$ is a predicate symbol and $t_1$, \ldots $t_n$ are terms.
We abbreviate $t = t$ to $t\!\downarrow$.
A Horn formula is an expression of the form $\varphi_1 \land \ldots \land \varphi_n$, where $\varphi_1$, \ldots $\varphi_n$ are atomic formulas.
The conjunction of the empty set of atomic formulas is denoted by $\top$.
A sequent is an expression of the form $\varphi \sststile{}{x_1, \ldots x_n} \psi$, where $x_1$, \ldots $x_n$ are variables
and $\varphi$ and $\psi$ are Horn formulas such that $\FV(\varphi) \cup \FV(\psi) \subseteq \{ x_1, \ldots x_n \}$.
A \emph{partial Horn theory} consists of a signature and a set of Horn sequents in this signature.

The rules of \emph{partial Horn logic} are listed below.
A \emph{theorem} of a partial Horn theory $T$ is a sequent derivable from $T$ in this logic.
We will write $\varphi \sststile{T}{V} \psi$ to denote the fact that sequent $\varphi \sststile{}{V} \psi$ is derivable in $T$.

\begin{center}
\AxiomC{}
\RightLabel{\axlabel{nv}}
\UnaryInfC{$\varphi \sststile{}{V} x\!\downarrow$}
\DisplayProof
\qquad
\AxiomC{$\varphi \sststile{}{V} a = b$}
\RightLabel{\axlabel{ns}}
\UnaryInfC{$\varphi \sststile{}{V} b = a$}
\DisplayProof
\end{center}

\begin{center}
\AxiomC{}
\RightLabel{\axlabel{nh}}
\UnaryInfC{$\varphi_1 \land \ldots \land \varphi_n \sststile{}{V} \varphi_i$}
\DisplayProof
\qquad
\AxiomC{$\varphi \sststile{}{V} a = b$}
\AxiomC{$\varphi \sststile{}{V} \psi[a/x]$}
\RightLabel{\axlabel{nl}}
\BinaryInfC{$\varphi \sststile{}{V} \psi[b/x]$}
\DisplayProof
\end{center}

\begin{center}
\AxiomC{$\varphi \sststile{}{V} R(t_1, \ldots t_n)$}
\RightLabel{\axlabel{np}}
\UnaryInfC{$\varphi \sststile{}{V} t_i\!\downarrow$}
\DisplayProof
\qquad
\AxiomC{$\varphi \sststile{}{V} \sigma(t_1, \ldots t_n)\!\downarrow$}
\RightLabel{\axlabel{nf}}
\UnaryInfC{$\varphi \sststile{}{V} t_i\!\downarrow$}
\DisplayProof
\end{center}
where $R$ is a predicate symbol of the theory and $\sigma$ is its function symbol.

Finally, for every axiom $\psi_1 \land \ldots \land \psi_n \sststile{}{x_1 : s_1, \ldots x_k : s_k} \chi_1 \land \ldots \land \chi_m$
and for all terms $t_1 : s_1$, \ldots $t_k : s_k$, we have the following rules for all $1 \leq j \leq m$:
\smallskip
\begin{center}
\AxiomC{$\varphi \sststile{}{V} t_i\!\downarrow$, $1 \leq i \leq k$}
\AxiomC{$\varphi \sststile{}{V} \psi_i[t_1/x_1, \ldots t_k/x_k]$, $1 \leq i \leq n$}
\RightLabel{\axlabel{na}}
\BinaryInfC{$\varphi \sststile{}{V} \chi_j[t_1/x_1, \ldots t_k/x_k]$}
\DisplayProof
\end{center}

The \emph{theory of substitutions} is the theory with $\mathcal{S} = \{ \ctx, \tm \} \times \mathbb{N}$ as the set of sorts, function symbols given below, and axioms listed in \cite[Section~3.1]{alg-tt}.
\begin{align*}
\emptyCtx      & : (\ctx,0) \\
\ft_n          & : (\ty,n) \to (\ctx,n) \\
\ty_n          & : (\tm,n) \to (\ty,n) \\
v_{n,i}        & : (\ctx,n) \to (\tm,n) \text{, } 0 \leq i < n \\
\subst_{p,n,k} & : (\ctx,n) \times (p,k) \times (\tm,n)^k \to (p,n) \text{, } p \in \{ \tm, \ty \}
\end{align*}

We let $\ft^i_n : (\ctx,n+i) \to (\ctx,n)$ and $\ctx_{p,n} : (p,n) \to (\ctx,n)$ be the following derived operations:
\begin{align*}
\ft^0_n(A)      & = A \\
\ft^{i+1}_n(A)  & = \ft^i_n(\ft_{n+i}(A)) \\
\ctx_{\ty,n}(t) & = \ft_n(t) \\
\ctx_{\tm,n}(t) & = \ft_n(\ty_n(t)) \\
\ctx^i_{p,n}(t) & = \ft^i_n(\ctx_{p,n+i}(t))
\end{align*}
We write $(\ty,n)$ for $(\ctx,n+1)$.
We also define the following notations: $d_\ty = \ctx$, $d_\tm = \ty$, $e_\ty(a) = \ft(a)$, $e_\tm(a) = \ty(a)$,

Let $\mathcal{F}_0$ be a set of function symbols and let $\mathcal{P}_0$ be a set of predicate symbols.
We call elements of these sets basic function symbols and basic predicate symbols, respectively.
Then we define the full sets of function and predicate symbols:
\begin{align*}
\mathcal{F} = \{ & \sigma_m : (\ctx,m) \times (p_1,m+n_1) \times \ldots \times (p_k,m+n_k) \to (p,m+n) \mid \\
                 & m \in \mathbb{N}, \sigma \in \mathcal{F}_0, \sigma : (p_1,n_1) \times \ldots \times (p_k,n_k) \to (p,n) \} \\
\mathcal{P} = \{ & R_m : (\ctx,m) \times (p_1,m+n_1) \times \ldots \times (p_k,m+n_k) \mid \\
                 & m \in \mathbb{N}, R \in \mathcal{P}_0, R : (p_1,n_1) \times \ldots \times (p_k,n_k) \}
\end{align*}

An \emph{algebraic dependent type theory} is a theory of the form $(\mathcal{S}, \mathcal{F}_s \cup \mathcal{F}, \mathcal{P}, \mathcal{A}_s \cup \mathcal{A})$, where $\mathcal{S}$, $\mathcal{F}$, and $\mathcal{P}$ are defined above,
$\mathcal{F}_s$ is the set of function symbols of the theory of substitutions, $\mathcal{A}_s$ is the set of its axioms, and $\mathcal{A}$ is an arbitrary set of axioms such that the following sequents are derivable for every $\sigma_m \in \mathcal{F}$ and $R_m \in \mathcal{P}$:
\begin{align*}
\sigma_m(\Gamma, x_1, \ldots x_k)\!\downarrow\ & \sststile{}{\Gamma, x_1, \ldots x_k} \ctx^n_{p,m}(\sigma_m(\Gamma, x_1, \ldots x_k)) = \Gamma \\
\sigma_m(\Gamma, x_1, \ldots x_k)\!\downarrow\ & \sststile{}{\Gamma, x_1, \ldots x_k} \bigwedge_{1 \leq i \leq k} \ctx^{n_i}_{p_i,m}(x_i) = \Gamma \\
R_m(\Gamma, x_1, \ldots x_k) & \sststile{}{\Gamma, x_1, \ldots x_k} \bigwedge_{1 \leq i \leq k} \ctx^{n_i}_{p_i,m}(x_i) = \Gamma
\end{align*}
Note that \cite[Definition~4.5]{alg-tt} has an additional condition which just says that $\subst$ commutes with all function symbols, but we do not assume this condition in general.
Theories satisfying it will be called \emph{regular}.

Let $T$ be an algebraic dependent type theory.
We define $P_M$ as the set of pairs $V,\varphi$ such that $V = \{ x_1, \ldots x_k \}$ and $\varphi = \varphi_1 \land \ldots \land \varphi_k$, where $\varphi_i$ equals to $e_p(x_i) = t_i$,
where $t_i$ is a term of $T$ with free variables in $\{ x_1, \ldots x_{i-1} \}$ such that for every $1 \leq i \leq k$,
sequent $\varphi_1 \land \ldots \land \varphi_{i-1} \sststile{}{x_1, \ldots x_{i-1}} t_i\!\downarrow$ is derivable in $T$.
We will be mostly interested in theorems of the form $\varphi \sststile{}{V} \psi$ where $(\varphi,V) \in P_M$ for reasons explained in \cite{morita-equiv}.

\begin{defn}
We will say that a partial Horn theory has \emph{separated axioms} if the set of axioms of this theory consists of three disjoint subsets $\mathcal{A}_d$, $\mathcal{A}'_d$, and $\mathcal{A}_e$ such that the following conditions hold:
\begin{enumerate}
\item The set $\mathcal{A}_d$ consists of axioms of the form $\varphi \sststile{}{x_1, \ldots x_k} \sigma(x_1, \ldots x_k)\!\downarrow$ and $\varphi \sststile{}{x_1, \ldots x_k} R(x_1, \ldots x_k)$ and, for every function symbol $\sigma$ and every predicate symbol $R$, there is exactly one axiom of this form.
We will denote the left hand side of these axioms by $\varphi_\sigma$ and $\varphi_R$.
\item The set $\mathcal{A}'_d$ consists of (not necessary all) sequents of the form $\sigma(x_1, \ldots x_k)\!\downarrow\ \sststile{}{x_1, \ldots x_k} \varphi_\sigma$ and $R(x_1, \ldots x_k) \sststile{}{x_1, \ldots x_k} \varphi_R$.
\item For every axiom $\varphi \sststile{}{V} \psi$ in $\mathcal{A}_e$ and every subterm $\sigma(t_1, \ldots t_k)$ of $\psi$,
the sequent $\varphi \sststile{}{V} \varphi_\sigma[t_1/x_1, \ldots t_k/x_k]$ is derivable from $\mathcal{A}_d \cup \mathcal{A}_e$.
Also, for every subformula $R(t_1, \ldots t_k)$ of $\psi$, the sequent $\varphi \sststile{}{V} \varphi_R[t_1/x_1, \ldots t_k/x_k]$ is derivable from $\mathcal{A}_d \cup \mathcal{A}_e$.
\end{enumerate}
We will say that a theory with separated axioms is \emph{minimal} (resp., \emph{maximal}) if $\mathcal{A}'_d$ is empty (resp., consists of all sequents of the specified form).
\end{defn}

\begin{remark}
Theories with separated axioms were defined in \cite[Section~5.1]{morita-equiv}.
There we do not allow axioms of the form $\varphi \sststile{}{x_1, \ldots x_k} R(x_1, \ldots x_k)$ in $\mathcal{A}_d$, but the main property of such theories (\rprop{sep-main}) still holds for this generalization.
\end{remark}

\begin{remark}
Every theory is isomorphic to a theory with a separated axioms.
We can take $\mathcal{A}_d$ to be the set of axioms of the form
\[ \sigma(x_1, \ldots x_k)\!\downarrow\ \sststile{}{x_1, \ldots x_k} \sigma(x_1, \ldots x_k)\!\downarrow \]
and $\mathcal{A}_e$ to be the set of axioms of the original theory (if some axiom $\varphi \sststile{}{V} \psi$ clashes with an axiom in $\mathcal{A}_d$, we can replace it with $\varphi \sststile{}{V} \psi \land \psi$).
\end{remark}

The main property of theories with separated axioms is that axioms in $\mathcal{A}_d'$ are redundant in a certain sense:

\begin{prop}[sep-main]
Let $T$ be an algebraic dependent type theory with separated axioms and let $(\varphi,V)$ be a pair in $P_M$.
Then a sequent $\varphi \sststile{}{V} \psi$ is derivable in $T$ if and only if it is derivable from $\mathcal{A}_d \cup \mathcal{A}_e$.
\end{prop}
\begin{proof}
It is easy to see that the proof of \cite[Proposition~5.3]{morita-equiv} holds for the generalized definition of theories with separated axioms.
\end{proof}

\subsection{Contexts}

The syntax we described in the previous subection is too verbose.
We can make it closer to the usual presentation of type theories by removing some redundant information.
We define another $\mathcal{S}$-set of terms $\Term_T(V)$ of a theory $T$ inductively as follows:
\begin{itemize}
\item Every element of $V_s$ is a term of sort $s$.
\item If $t$ is a term of sort $(\tm,n)$, then $\ty(t)$ is a term of sort $(\ty,n)$.
\item $v_i$ is a term of sort $(\tm,n)$ for every $0 \leq i < n$.
\item If $t$ is a term of sort $(p,k)$, where $p \in \{ \ty, \tm \}$, and $t_1, \ldots t_k$ are terms of sort $(\tm,n)$, then $\subst(t, t_1, \ldots t_k)$ is a term of sort $(p,n)$.
\item If $\sigma : (p_1,n_1) \times \ldots \times (p_k,n_k) \to (p,n)$ is a basic function symbol of $T$ and $t_i$ is a term of sort $(p_i,n_i)$, then $\sigma(t_1, \ldots t_k)$ is a term of sort $(p,n)$.
\end{itemize}

Thus, these terms define a syntax, called the \emph{$\ft$-free syntax}, which has metavariables (elements of $V$), de Bruijn indices ($v_i$), explicit substitutions ($\subst$), and constructions depending on the theory (basic function symbols).
We also have an explicit typing operation ($\ty$).
We will see in section~\ref{sec:types} that it is possible to get rid of it too, but this makes the theory more awkward to work with.

We want to be able to reconstruct a term in $\Term^e_T(V)$ from a term in $\Term_T(V)$ and a context.
This can be done under very mild assumptions on $T$.
Suppose that $T$ has separated axioms.
Moreover, suppose that, for every function or predicate symbol $S$, there are 

\subsection{Types}
\label{sec:types}

\section{Examples}

\section{Strong normalization}

\section{Confluence}

\bibliographystyle{amsplain}
\bibliography{ref}

\end{document}
