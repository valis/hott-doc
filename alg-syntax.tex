\documentclass[reqno]{amsart}

\usepackage{amssymb}
\usepackage{hyperref}
\usepackage{mathtools}
\usepackage[all]{xy}
\usepackage{verbatim}
\usepackage{ifthen}
\usepackage{xargs}
\usepackage{bussproofs}
\usepackage{turnstile}
\usepackage{etex}

\hypersetup{colorlinks=true,linkcolor=blue}

\renewcommand{\turnstile}[6][s]
    {\ifthenelse{\equal{#1}{d}}
        {\sbox{\first}{$\displaystyle{#4}$}
        \sbox{\second}{$\displaystyle{#5}$}}{}
    \ifthenelse{\equal{#1}{t}}
        {\sbox{\first}{$\textstyle{#4}$}
        \sbox{\second}{$\textstyle{#5}$}}{}
    \ifthenelse{\equal{#1}{s}}
        {\sbox{\first}{$\scriptstyle{#4}$}
        \sbox{\second}{$\scriptstyle{#5}$}}{}
    \ifthenelse{\equal{#1}{ss}}
        {\sbox{\first}{$\scriptscriptstyle{#4}$}
        \sbox{\second}{$\scriptscriptstyle{#5}$}}{}
    \setlength{\dashthickness}{0.111ex}
    \setlength{\ddashthickness}{0.35ex}
    \setlength{\leasturnstilewidth}{2em}
    \setlength{\extrawidth}{0.2em}
    \ifthenelse{%
      \equal{#3}{n}}{\setlength{\tinyverdistance}{0ex}}{}
    \ifthenelse{%
      \equal{#3}{s}}{\setlength{\tinyverdistance}{0.5\dashthickness}}{}
    \ifthenelse{%
      \equal{#3}{d}}{\setlength{\tinyverdistance}{0.5\ddashthickness}
        \addtolength{\tinyverdistance}{\dashthickness}}{}
    \ifthenelse{%
      \equal{#3}{t}}{\setlength{\tinyverdistance}{1.5\dashthickness}
        \addtolength{\tinyverdistance}{\ddashthickness}}{}
        \setlength{\verdistance}{0.4ex}
        \settoheight{\lengthvar}{\usebox{\first}}
        \setlength{\raisedown}{-\lengthvar}
        \addtolength{\raisedown}{-\tinyverdistance}
        \addtolength{\raisedown}{-\verdistance}
        \settodepth{\raiseup}{\usebox{\second}}
        \addtolength{\raiseup}{\tinyverdistance}
        \addtolength{\raiseup}{\verdistance}
        \setlength{\lift}{0.8ex}
        \settowidth{\firstwidth}{\usebox{\first}}
        \settowidth{\secondwidth}{\usebox{\second}}
        \ifthenelse{\lengthtest{\firstwidth = 0ex}
            \and
            \lengthtest{\secondwidth = 0ex}}
                {\setlength{\turnstilewidth}{\leasturnstilewidth}}
                {\setlength{\turnstilewidth}{2\extrawidth}
        \ifthenelse{\lengthtest{\firstwidth < \secondwidth}}
            {\addtolength{\turnstilewidth}{\secondwidth}}
            {\addtolength{\turnstilewidth}{\firstwidth}}}
        \ifthenelse{\lengthtest{\turnstilewidth < \leasturnstilewidth}}{\setlength{\turnstilewidth}{\leasturnstilewidth}}{}
    \setlength{\turnstileheight}{1.5ex}
    \sbox{\turnstilebox}
    {\raisebox{\lift}{\ensuremath{
        \makever{#2}{\dashthickness}{\turnstileheight}{\ddashthickness}
        \makehor{#3}{\dashthickness}{\turnstilewidth}{\ddashthickness}
        \hspace{-\turnstilewidth}
        \raisebox{\raisedown}
        {\makebox[\turnstilewidth]{\usebox{\first}}}
            \hspace{-\turnstilewidth}
            \raisebox{\raiseup}
            {\makebox[\turnstilewidth]{\usebox{\second}}}
        \makever{#6}{\dashthickness}{\turnstileheight}{\ddashthickness}}}}
        \mathrel{\usebox{\turnstilebox}}}

\newcommand{\axlabel}[1]{(#1) \phantomsection \label{ax:#1}}
\newcommand{\axtag}[1]{\label{ax:#1} \tag{#1}}
\newcommand{\axref}[1]{(\hyperref[ax:#1]{#1})}

\newcommand{\newref}[4][]{
\ifthenelse{\equal{#1}{}}{\newtheorem{h#2}[hthm]{#4}}{\newtheorem{h#2}{#4}[#1]}
\expandafter\newcommand\csname r#2\endcsname[1]{#3~\ref{#2:##1}}
\expandafter\newcommand\csname R#2\endcsname[1]{#4~\ref{#2:##1}}
\expandafter\newcommand\csname n#2\endcsname[1]{\ref{#2:##1}}
\newenvironmentx{#2}[2][1=,2=]{
\ifthenelse{\equal{##2}{}}{\begin{h#2}}{\begin{h#2}[##2]}
\ifthenelse{\equal{##1}{}}{}{\label{#2:##1}}
}{\end{h#2}}
}

\newref[section]{thm}{Theorem}{Theorem}
\newref{lem}{Lemma}{Lemma}
\newref{prop}{Proposition}{Proposition}
\newref{cor}{Corollary}{Corollary}
\newref{cond}{Condition}{Condition}

\theoremstyle{definition}
\newref{defn}{Definition}{Definition}
\newref{example}{Example}{Example}

\theoremstyle{remark}
\newref{remark}{Remark}{Remark}

\newcommand{\fs}[1]{\mathrm{#1}}
\newcommand{\Term}{\fs{Term}}
\newcommand{\FV}{\fs{FV}}
\newcommand{\subst}{\fs{subst}}
\newcommand{\Id}{\fs{Id}}
\newcommand{\refl}{\fs{refl}}
\newcommand{\El}{\fs{El}}
\newcommand{\emptyCtx}{\cdot}
\newcommand{\ft}{\fs{ft}}
\newcommand{\ty}{\fs{ty}}
\newcommand{\ctx}{\fs{ctx}}
\newcommand{\tm}{\fs{tm}}
\newcommand{\sub}{\fs{Sub}}

\newcommand{\cat}[1]{\mathbf{#1}}
\newcommand{\Th}{\cat{Th}}
\newcommand{\algtt}{\cat{TT}}

\numberwithin{figure}{section}

\newcommand{\pb}[1][dr]{\save*!/#1-1.2pc/#1:(-1,1)@^{|-}\restore}
\newcommand{\po}[1][dr]{\save*!/#1+1.2pc/#1:(1,-1)@^{|-}\restore}

\begin{document}

\title{Syntax of Algebraic Dependent Type Theories}

\author{Valery Isaev}

\begin{abstract}
TODO
\end{abstract}

\maketitle

\section{Introduction}

TODO

\section{Syntax}

In this section, we define a syntax for algebraic dependent type theories which is closer to the usual presentation.

\subsection{Preliminaries}

Let us recall definitions from \cite{PHL} and \cite{alg-tt}.
A many sorted first-order signature $(\mathcal{S},\mathcal{F},\mathcal{P})$ consists of a set $\mathcal{S}$ of sorts,
a set $\mathcal{F}$ of function symbols and a set $\mathcal{P}$ of predicate symbols.
Each function symbol $\sigma$ is equipped with a signature of the form $\sigma : s_1 \times \ldots \times s_k \to s$, where $s_1$, \ldots $s_k$, $s$ are sorts.
Each predicate symbol $R$ is equipped with a signature of the form $R : s_1 \times \ldots \times s_k$.
If $V$ is an $\mathcal{S}$-set, then the $\mathcal{S}$-set of terms of $T$ with free variables in $V$ will be denoted by $\Term^e_\mathcal{F}(V)$ or by $\Term^e_T(V)$.

An atomic formula is an expression either of the form $t_1 = t_2$ or of the form $R(t_1, \ldots t_n)$,
where $R$ is a predicate symbol and $t_1$, \ldots $t_n$ are terms.
We abbreviate $t = t$ to $t\!\downarrow$.
A Horn formula is an expression of the form $\varphi_1 \land \ldots \land \varphi_n$, where $\varphi_1$, \ldots $\varphi_n$ are atomic formulas.
The conjunction of the empty set of atomic formulas is denoted by $\top$.
A sequent is an expression of the form $\varphi \sststile{}{x_1, \ldots x_n} \psi$, where $x_1$, \ldots $x_n$ are variables
and $\varphi$ and $\psi$ are Horn formulas such that $\FV(\varphi) \cup \FV(\psi) \subseteq \{ x_1, \ldots x_n \}$.
We also write $\varphi \ssststile{}{V} \psi$ to denote the pair of sequents $\varphi \sststile{}{V} \psi$ and $\psi \sststile{}{V} \varphi$.
A \emph{partial Horn theory} consists of a signature and a set of Horn sequents in this signature.

The rules of \emph{partial Horn logic} are listed below.
A \emph{theorem} of a partial Horn theory $T$ is a sequent derivable from $T$ in this logic.
We will write $\varphi \sststile{T}{V} \psi$ to denote the fact that sequent $\varphi \sststile{}{V} \psi$ is derivable in $T$.

\begin{center}
\AxiomC{}
\RightLabel{\axlabel{nv}}
\UnaryInfC{$\varphi \sststile{}{V} x\!\downarrow$}
\DisplayProof
\qquad
\AxiomC{$\varphi \sststile{}{V} a = b$}
\RightLabel{\axlabel{ns}}
\UnaryInfC{$\varphi \sststile{}{V} b = a$}
\DisplayProof
\end{center}

\begin{center}
\AxiomC{}
\RightLabel{\axlabel{nh}}
\UnaryInfC{$\varphi_1 \land \ldots \land \varphi_n \sststile{}{V} \varphi_i$}
\DisplayProof
\qquad
\AxiomC{$\varphi \sststile{}{V} a = b$}
\AxiomC{$\varphi \sststile{}{V} \psi[a/x]$}
\RightLabel{\axlabel{nl}}
\BinaryInfC{$\varphi \sststile{}{V} \psi[b/x]$}
\DisplayProof
\end{center}

\begin{center}
\AxiomC{$\varphi \sststile{}{V} R(t_1, \ldots t_n)$}
\RightLabel{\axlabel{np}}
\UnaryInfC{$\varphi \sststile{}{V} t_i\!\downarrow$}
\DisplayProof
\qquad
\AxiomC{$\varphi \sststile{}{V} \sigma(t_1, \ldots t_n)\!\downarrow$}
\RightLabel{\axlabel{nf}}
\UnaryInfC{$\varphi \sststile{}{V} t_i\!\downarrow$}
\DisplayProof
\end{center}
where $R$ is a predicate symbol of the theory and $\sigma$ is its function symbol.
Note that this system derives only sequents in which the conclusion is atomic.
For this reason, we will consider only such sequents.

Finally, for every axiom $\psi_1 \land \ldots \land \psi_n \sststile{}{x_1 : s_1, \ldots x_k : s_k} \chi_1 \land \ldots \land \chi_m$
and for all terms $t_1 : s_1$, \ldots $t_k : s_k$, we have the following rules for all $1 \leq j \leq m$:
\smallskip
\begin{center}
\AxiomC{$\varphi \sststile{}{V} t_i\!\downarrow$, $1 \leq i \leq k$}
\AxiomC{$\varphi \sststile{}{V} \psi_i[t_1/x_1, \ldots t_k/x_k]$, $1 \leq i \leq n$}
\RightLabel{\axlabel{na}}
\BinaryInfC{$\varphi \sststile{}{V} \chi_j[t_1/x_1, \ldots t_k/x_k]$}
\DisplayProof
\end{center}

The \emph{theory of substitutions} is the theory with $\mathcal{S} = \{ \ctx, \tm \} \times \mathbb{N}$ as the set of sorts, function symbols given below, and axioms listed in \cite[Section~3.1]{alg-tt} (also, see the section~\ref{sec:contexts}).
\begin{align*}
\emptyCtx      & : (\ctx,0) \\
\ft_n          & : (\ty,n) \to (\ctx,n) \\
\ty_n          & : (\tm,n) \to (\ty,n) \\
v_{n,i}        & : (\ctx,n) \to (\tm,n) \text{, } 0 \leq i < n \\
\subst_{p,n,k} & : (\ctx,n) \times (p,k) \times (\tm,n)^k \to (p,n) \text{, } p \in \{ \tm, \ty \}
\end{align*}

We let $\ft^i_n : (\ctx,n+i) \to (\ctx,n)$ and $\ctx_{p,n} : (p,n) \to (\ctx,n)$ be the following derived operations:
\begin{align*}
\ft^0_n(A)      & = A \\
\ft^{i+1}_n(A)  & = \ft^i_n(\ft_{n+i}(A)) \\
\ctx_{\ty,n}(t) & = \ft_n(t) \\
\ctx_{\tm,n}(t) & = \ft_n(\ty_n(t)) \\
\ctx^i_{p,n}(t) & = \ft^i_n(\ctx_{p,n+i}(t))
\end{align*}
We write $(\ty,n)$ for $(\ctx,n+1)$.
We also define the following notations: $d_\ty = \ctx$, $d_\tm = \ty$, $e_\ty(a) = \ft(a)$, $e_\tm(a) = \ty(a)$,

Let $\mathcal{F}_0$ be a set of function symbols and let $\mathcal{P}_0$ be a set of predicate symbols.
We call elements of these sets basic function symbols and basic predicate symbols, respectively.
Then we define the full sets of function and predicate symbols:
\begin{align*}
\mathcal{F} = \{ & \sigma_m : (\ctx,m) \times (p_1,m+n_1) \times \ldots \times (p_k,m+n_k) \to (p,m+n) \mid \\
                 & m \in \mathbb{N}, \sigma \in \mathcal{F}_0, \sigma : (p_1,n_1) \times \ldots \times (p_k,n_k) \to (p,n) \} \\
\mathcal{P} = \{ & R_m : (\ctx,m) \times (p_1,m+n_1) \times \ldots \times (p_k,m+n_k) \mid \\
                 & m \in \mathbb{N}, R \in \mathcal{P}_0, R : (p_1,n_1) \times \ldots \times (p_k,n_k) \}
\end{align*}

An \emph{algebraic dependent type theory} is a theory of the form $(\mathcal{S}, \mathcal{F}_s \cup \mathcal{F}, \mathcal{P}, \mathcal{A}_s \cup \mathcal{A})$, where $\mathcal{S}$, $\mathcal{F}$, and $\mathcal{P}$ are defined above,
$\mathcal{F}_s$ is the set of function symbols of the theory of substitutions, $\mathcal{A}_s$ is the set of its axioms, and $\mathcal{A}$ is an arbitrary set of axioms such that the following sequents are derivable for every $\sigma_m \in \mathcal{F}$ and $R_m \in \mathcal{P}$:
\begin{align*}
\sigma_m(\Gamma, x_1, \ldots x_k)\!\downarrow\ & \sststile{}{\Gamma, x_1, \ldots x_k} \ctx^n_{p,m}(\sigma_m(\Gamma, x_1, \ldots x_k)) = \Gamma \\
\sigma_m(\Gamma, x_1, \ldots x_k)\!\downarrow\ & \sststile{}{\Gamma, x_1, \ldots x_k} \bigwedge_{1 \leq i \leq k} \ctx^{n_i}_{p_i,m}(x_i) = \Gamma \\
R_m(\Gamma, x_1, \ldots x_k) & \sststile{}{\Gamma, x_1, \ldots x_k} \bigwedge_{1 \leq i \leq k} \ctx^{n_i}_{p_i,m}(x_i) = \Gamma
\end{align*}
Note that \cite[Definition~4.5]{alg-tt} has an additional condition which just says that $\subst$ commutes with all function symbols, but we do not assume this condition in general.
Theories satisfying it will be called \emph{regular}.

Let $T$ be an algebraic dependent type theory.
We define $P_M$ as the set of pairs $V,\varphi$ such that $V = \{ x_1, \ldots x_k \}$ and $\varphi = \varphi_1 \land \ldots \land \varphi_k$, where $\varphi_i$ equals to $e_p(x_i) = t_i$,
where $t_i$ is a term of $T$ with free variables in $\{ x_1, \ldots x_{i-1} \}$ such that for every $1 \leq i \leq k$,
sequent $\varphi_1 \land \ldots \land \varphi_{i-1} \sststile{}{x_1, \ldots x_{i-1}} t_i\!\downarrow$ is derivable in $T$.
We will be mostly interested in theorems of the form $\varphi \sststile{}{V} \psi$ where $(\varphi,V) \in P_M$ for reasons explained in \cite{morita-equiv}.

\begin{defn}
We will say that a partial Horn theory has \emph{separated axioms} if the set of axioms of this theory consists of three disjoint subsets $\mathcal{A}_d$, $\mathcal{A}'_d$, and $\mathcal{A}_e$ such that the following conditions hold:
\begin{enumerate}
\item The set $\mathcal{A}_d$ consists of axioms of the form $\varphi \sststile{}{x_1, \ldots x_k} \sigma(x_1, \ldots x_k)\!\downarrow$ and $\varphi \sststile{}{x_1, \ldots x_k} R(x_1, \ldots x_k)$ and, for every function symbol $\sigma$ and every predicate symbol $R$, there is exactly one axiom of this form.
We will denote the left hand side of these axioms by $\varphi_\sigma$ and $\varphi_R$.
\item The set $\mathcal{A}'_d$ consists of (not necessary all) sequents of the form $\sigma(x_1, \ldots x_k)\!\downarrow\ \sststile{}{x_1, \ldots x_k} \varphi_\sigma$ and $R(x_1, \ldots x_k) \sststile{}{x_1, \ldots x_k} \varphi_R$.
\item For every axiom $\varphi \sststile{}{V} \psi$ in $\mathcal{A}_e$ and every subterm $\sigma(t_1, \ldots t_k)$ of $\psi$,
the sequent $\varphi \sststile{}{V} \varphi_\sigma[t_1/x_1, \ldots t_k/x_k]$ is derivable from $\mathcal{A}_d \cup \mathcal{A}_e$.
Also, for every subformula $R(t_1, \ldots t_k)$ of $\psi$, the sequent $\varphi \sststile{}{V} \varphi_R[t_1/x_1, \ldots t_k/x_k]$ is derivable from $\mathcal{A}_d \cup \mathcal{A}_e$.
\end{enumerate}
We will say that a theory with separated axioms is \emph{minimal} (resp., \emph{maximal}) if $\mathcal{A}'_d$ is empty (resp., consists of all sequents of the specified form).
\end{defn}

\begin{remark}
Theories with separated axioms were defined in \cite[Section~5.1]{morita-equiv}.
There we do not allow axioms of the form $\varphi \sststile{}{x_1, \ldots x_k} R(x_1, \ldots x_k)$ in $\mathcal{A}_d$, but the main property of such theories (\rprop{sep-main}) still holds for this generalization.
\end{remark}

\begin{remark}
Every theory is isomorphic to a theory with a separated axioms.
We can take $\mathcal{A}_d$ to be the set of axioms of the form
\[ \sigma(x_1, \ldots x_k)\!\downarrow\ \sststile{}{x_1, \ldots x_k} \sigma(x_1, \ldots x_k)\!\downarrow \]
and $\mathcal{A}_e$ to be the set of axioms of the original theory (if some axiom $\varphi \sststile{}{V} \psi$ clashes with an axiom in $\mathcal{A}_d$, we can replace it with $\varphi \sststile{}{V} \psi \land \psi$).
\end{remark}

The main property of theories with separated axioms is that axioms in $\mathcal{A}_d'$ are redundant in a certain sense:

\begin{prop}[sep-main]
Let $T$ be an algebraic dependent type theory with separated axioms and let $(\varphi,V)$ be a pair in $P_M$.
Then a sequent $\varphi \sststile{}{V} \psi$ is derivable in $T$ if and only if it is derivable from $\mathcal{A}_d \cup \mathcal{A}_e$.
\end{prop}
\begin{proof}
It is easy to see that the proof of \cite[Proposition~5.3]{morita-equiv} holds for the generalized definition of theories with separated axioms.
\end{proof}

\subsection{The $\ft$-free syntax}
\label{sec:contexts}

The syntax we described in the previous subection is too verbose.
We can make it closer to the usual presentation of type theories by removing some redundant information.
We define another $\mathcal{S}$-set of terms $\Term_\mathcal{F}(V)$ inductively as follows:
\begin{itemize}
\item Every element of $V_s$ is a term of sort $s$.
\item If $t$ is a term of sort $(\tm,n)$, then $\ty(t)$ is a term of sort $(\ty,n)$.
\item $v_i$ is a term of sort $(\tm,n)$ for every $0 \leq i < n$.
\item If $t$ is a term of sort $(p,k)$, where $p \in \{ \ty, \tm \}$, and $t_1, \ldots t_k$ are terms of sort $(\tm,n)$, then $\subst(t, t_1, \ldots t_k)$ is a term of sort $(p,n)$.
\item If $\sigma : (p_1,n_1) \times \ldots \times (p_k,n_k) \to (p,n)$ is a function symbol of $T$ and $t_i$ is a term of sort $(p_i,m+n_i)$, then $\sigma(t_1, \ldots t_k)$ is a term of sort $(p,m+n)$.
\end{itemize}
These terms will be called \emph{$\ft$-free} terms.
They define a syntax, called the \emph{$\ft$-free syntax}, which has metavariables (elements of $V$), de Bruijn indices ($v_i$), explicit substitutions ($\subst$), and constructions depending on the theory (basic function symbols).
We also have an explicit typing operation ($\ty$).
We will see in section~\ref{sec:types} that it is possible to get rid of it too, but this makes the theory more awkward to work with.

A \emph{context} of length $n$ a sequence of terms of sorts $(\ty,0)$, \ldots $(\ty,n-1)$.
An ($\ft$-free) \emph{judgement} is an expression of one of the following forms:
\[ \Gamma \vdash t \qquad \Gamma \vdash t \equiv t' \qquad \Gamma \vdash R(t_1, \ldots t_k) \]
where $\Gamma$ is a context of length $m$, $t$ and $t'$ are terms of sort $(p,m)$ (where $p \in \{ \ty, \tm \}$), $R : (p_1,n_1) \times \ldots \times (p_k,n_k)$ is a predicate symbol, and $t_1$, \ldots $t_k$ are terms of sorts $(p_1,m+n_1)$, \ldots $(p_k,m+n_k)$, respectively.
All terms above are written in the $\ft$-free syntax.
We will use judgements of the form $\Gamma \vdash$.
If $\Gamma$ is the empty context, this judgement denotes $\top$.
If $\Gamma = (\Gamma', A)$, this judgement denotes $\Gamma' \vdash A$.

Jusgements play the role of atomic formulas in the $\ft$-free syntax.
An ($\ft$-free) \emph{formula} is a finite conjunction of judgements.
Sequents are defined as before.
We will often write sequents in the form of derivation rules.
Thus, $\varphi_1 \land \ldots \land \varphi_n \sststile{}{V} \psi$ can be written as
\begin{center}
\AxiomC{$\varphi_1$}
\AxiomC{\ldots}
\AxiomC{$\varphi_n$}
\TrinaryInfC{$\psi$}
\DisplayProof
\end{center}
The set of variables $V$ is implicit in this notation and we let it to be the union of all variables that appear in the premise and in the conclusion.
Also, $\varphi_1 \land \ldots \land \varphi_n \ssststile{}{V} \psi$ will be represented by the following rule:
\begin{center}
\AxiomC{$\varphi_1$}
\AxiomC{\ldots}
\AxiomC{$\varphi_n$}
\doubleLine
\TrinaryInfC{$\psi$}
\DisplayProof
\end{center}

If $\Gamma$ and $\Delta$ are two contexts of the same length, we can define formula $\Gamma \equiv \Delta$ by induction on their length.
If they are empty, then this formula is defined as $\top$.
If $\Gamma = (\Gamma',A)$ and $\Delta = (\Delta',B)$, then this formula is defined as $(\Gamma' \equiv \Delta') \land (\Gamma' \vdash A \equiv B)$.

An ($\ft$-free) \emph{theory} $(\mathcal{F},\mathcal{P},\mathcal{A})$ consits of a set of function symbols $\mathcal{F}$, a set of predicate symbols $\mathcal{P}$, and a set of axioms $\mathcal{A}$.
These sets must satisfy a few conditions that we list below.
First, we assume that, for every function or predicate symbol $S$ and every $n \in \mathbb{N}$, there is a unique axiom of the form $\varphi \sststile{}{A_1, \ldots A_n, x_1, \ldots x_k} A_1, \ldots A_n \vdash S(x_1, \ldots x_k)$.
Moreover, we assume that, for every $1 \leq i \leq k$, the formula $\varphi$ contains a unique judgement of the form $\Gamma \vdash x_i$ and the first $n$ terms of $\Gamma$ are $A_1$, \ldots $A_n$.
The rest of the terms in $\Gamma$ will be denoted by $\Gamma^S_i$.
Note that $\Gamma^S_i$ implicitly depends on $n$.
This abuse of notation will not cause any problems because sorts of terms in $C_{S,i}$ also depend on $n$, hence $n$ can be inferred from the context.
For example, if we write $\Gamma, \Gamma^S_i \vdash t$, then $n$ is equal to the length of $\Gamma$.

A \emph{contexted term} is a pair $(\Gamma,t)$, where $\Gamma$ is a context of length $n$ and $t$ is a term of sort $(p,n)$ for some $p \in \{ \ty, \tm \}$.
For every contexted term $(\Gamma,t)$, we can define the set $\sub(\Gamma,t)$ of its contxeted subterms:
\begin{align*}
\sub(\Gamma,t) & = \{ (\Gamma,t) \} \text{ if $t$ is a variable or $v_i$} \\
\sub(\Gamma,\ty(t')) & = \{ (\Gamma,t) \} \cup \sub(\Gamma,t') \\
\sub(\Gamma,\subst(t', t_1, \ldots t_k)) & = \{ (\Gamma,t) \} \cup \sub((\ty(t_1), \ldots \ty(t_k)), t') \cup \bigcup_{1 \leq i \leq k} \sub(\Gamma,t_i) \\
\sub(\Gamma,\sigma(t_1, \ldots t_k)) & = \{ (\Gamma,t) \} \cup \bigcup_{1 \leq i \leq k} \sub((\Gamma_{\leq n},\Gamma^\sigma_i),t_i)
\end{align*}
where $\sigma : (p_1,n_1) \times \ldots \times (p_k,n_k) \to (p,n)$ and $\Gamma_{\leq n}$ is the first $n$ terms of $\Gamma$.
In each of the cases, $(\Gamma,t)$ is the contexted term itself.
So, every contexted term is always a contexted subterm of itself.
Note that the definition of $\sub$ depends on the theory.

We can also define the set of contexted subterms of a formula:
\begin{align*}
\sub(\Gamma \vdash t) & = \sub(\Gamma, t) \\
\sub(\Gamma \vdash t \equiv t') & = \sub(\Gamma,t) \cup \sub(\Gamma,t') \\
\sub(\Gamma, R(t_1, \ldots t_k)) & = \bigcup_{1 \leq i \leq k} \sub((\Gamma,\Gamma^R_i),t_i) \\
\sub(\varphi_1 \land \ldots \land \varphi_n) & = \bigcup_{1 \leq i \leq n} \sub(\varphi_i)
\end{align*}

Now, we can describe the derivation system for the $\ft$-free syntax:
\medskip
\begin{center}
\AxiomC{}
\RightLabel{\axlabel{ch}}
\UnaryInfC{$\varphi_1 \land \ldots \land \varphi_n \sststile{}{V} \varphi_i$}
\DisplayProof
\qquad
\AxiomC{}
\RightLabel{, $(\Gamma,t) \in \sub(\varphi)$ \axlabel{cd}}
\UnaryInfC{$\varphi \sststile{}{V} \Gamma \vdash t$}
\DisplayProof
\end{center}

\medskip
\begin{center}
\AxiomC{}
\RightLabel{, $(\Gamma,x),(\Delta,x) \in \sub(\varphi)$ \axlabel{cc}}
\UnaryInfC{$\varphi \sststile{}{V} \Gamma \equiv \Delta$}
\DisplayProof
\qquad
\AxiomC{$\varphi \sststile{}{V} \Gamma \vdash t$}
\RightLabel{\axlabel{cr}}
\UnaryInfC{$\varphi \sststile{}{V} \Gamma \vdash t \equiv t$}
\DisplayProof
\end{center}

\medskip
\begin{center}
\AxiomC{$\varphi \sststile{}{V} \Gamma \vdash a \equiv b$}
\RightLabel{\axlabel{cs}}
\UnaryInfC{$\varphi \sststile{}{V} \Gamma \vdash b \equiv a$}
\DisplayProof
\qquad
\AxiomC{$\varphi \sststile{}{V} \Gamma \vdash a \equiv b$}
\AxiomC{$\varphi \sststile{}{V} \Gamma \vdash b \equiv c$}
\RightLabel{\axlabel{ct}}
\BinaryInfC{$\varphi \sststile{}{V} \Gamma \vdash a \equiv c$}
\DisplayProof
\end{center}

\medskip
\begin{center}
\AxiomC{$\varphi \sststile{}{V} \Gamma \vdash R(t_1, \ldots t_k)$}
\AxiomC{$\varphi \sststile{}{V} \Gamma, \Gamma^R_i \vdash t_i \equiv t_i'$}
\RightLabel{\axlabel{cp}}
\BinaryInfC{$\varphi \sststile{}{V} \Gamma \vdash R(t_1, \ldots t_{i-1}, t_i', t_{i+1}, \ldots t_k)$}
\DisplayProof
\end{center}

\medskip
\begin{center}
\AxiomC{$\varphi \sststile{}{V} \Gamma \vdash \sigma(t_1, \ldots t_k)$}
\AxiomC{$\varphi \sststile{}{V} \Gamma, \Gamma^\sigma_i \vdash t_i \equiv t_i'$}
\RightLabel{\axlabel{cf}}
\BinaryInfC{$\varphi \sststile{}{V} \Gamma \vdash \sigma(t_1, \ldots t_k) = \sigma (t_1, \ldots t_{i-1}, t_i', t_{i+1}, \ldots t_k)$}
\DisplayProof
\end{center}

% \medskip
% \begin{center}
% \AxiomC{$\varphi \sststile{}{V} \Gamma \vdash t \equiv t$}
% \RightLabel{\axlabel{cde}}
% \UnaryInfC{$\varphi \sststile{}{V} \Gamma \vdash t$}
% \DisplayProof
% \qquad
% \AxiomC{$\varphi \sststile{}{V} \Gamma \vdash R(t_1, \ldots t_k)$}
% \RightLabel{\axlabel{cdp}}
% \UnaryInfC{$\varphi \sststile{}{V} \Gamma, \Gamma^R_i \vdash t_i$}
% \DisplayProof
% \end{center}

% \medskip
% \begin{center}
% \AxiomC{$\varphi \sststile{}{V} \Gamma \vdash \sigma(t_1, \ldots t_n)$}
% \RightLabel{\axlabel{cdf}}
% \UnaryInfC{$\varphi \sststile{}{V} \Gamma, \Gamma^\sigma_i \vdash t_i$}
% \DisplayProof
% \end{center}

Finally, for every axiom $\psi_1 \land \ldots \land \psi_n \sststile{}{x_1 : s_1, \ldots x_k : s_k} \chi_1 \land \ldots \land \chi_m$
and for all terms $t_1$, \ldots $t_k$, we have the following rules for all $1 \leq j \leq m$:
\smallskip
\begin{center}
\AxiomC{$\varphi \sststile{}{V} \psi_i[t_1/x_1, \ldots t_k/x_k]$, $1 \leq i \leq n$}
\RightLabel{\axlabel{ca}}
\UnaryInfC{$\varphi \sststile{}{V} \chi_j[t_1/x_1, \ldots t_k/x_k]$}
\DisplayProof
\end{center}

% We will later see that if a theory satisfies some mild additional hypothesis, then we can replace rules \axref{cde}, \axref{cdp}, and \axref{cdf} with simpler ones.
% Thus, we will denote this set of rules by (cd*) and other rules by (c*).

We will say that a term or a formula $E$ is \emph{valid} with respect to $(\varphi,V)$ if the following conditions hold:
\begin{enumerate}
\item For every contexted subterm $(\Gamma,t)$ of $E$, the sequent $\varphi \sststile{}{V} \Gamma \vdash t$ is derivable.
\item For every pair $(\Gamma,x)$ and $(\Delta,x)$ of contexted subterms of $E$ (where $x$ is any variable), the sequent $\varphi \sststile{}{V} \Gamma \equiv \Delta$ is derivable.
\end{enumerate}
We will say that a sequent $\varphi \sststile{}{V} \psi$ is \emph{valid} if $\psi$ is valid with respect to $(\varphi,V)$.

\begin{remark}
The first condition expresses the idea that everything that occurs in $E$ should be defined.
The second condition says that every variable should have a unique context, that is if a variable is used in two different contexts, they must be equivalent.
\end{remark}

\begin{remark}
Rules \axref{cd} and \axref{cc} say that every formula is valid with respect to itself.
\end{remark}

\begin{defn}
An ($\ft$-free) \emph{theory} $(\mathcal{F},\mathcal{P},\mathcal{A})$ consits of a set of function symbols $\mathcal{F}$, a set of predicate symbols $\mathcal{P}$, and a set of axioms $\mathcal{A}$.
For every function or predicate symbol $S$, there must a unique axiom of the form
\[ \varphi \land \bigwedge_{1 \leq i \leq k} \Gamma, \Gamma^S_i \vdash x_i \sststile{}{x_1, \ldots x_k} S(x_1, \ldots x_k), \]
where $\varphi$ does not contain judgements of the form $\Delta \vdash x_i$.
All axioms must be valid and the set of free variables of the conclusion of every axiom must be contained in the set of free variables of its premise.
\end{defn}

% We want to be able to reconstruct a term in $\Term^e_T(V)$ from a term in $\Term_T(V)$ and a context.
% This can be done under very mild assumptions on $T$.
% Suppose that $T$ has separated axioms.
% Then we define a binary relation $\prec$ on the set of basic function symbols.
% Let $\tau \prec \sigma$ if and only if $\tau$ appears in $\varphi_\sigma$.
% We will say that (basic) function symbols are \emph{well-founded} if this relation is.

\subsection{The $\ty$-free syntax}
\label{sec:types}

\section{Examples}

\section{Strong normalization}

\section{Confluence}

\bibliographystyle{amsplain}
\bibliography{ref}

\end{document}
