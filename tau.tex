\documentclass{amsart}

\usepackage{amssymb}
\usepackage[all]{xy}
\usepackage{verbatim}
\usepackage{ifthen}
\usepackage{xargs}
\usepackage{bussproofs}

\providecommand\WarningsAreErrors{false}
\ifthenelse{\equal{\WarningsAreErrors}{true}}{\renewcommand{\GenericWarning}[2]{\GenericError{#1}{#2}{}{This warning has been turned into a fatal error.}}}{}

\newcommand{\newref}[4][]{
\ifthenelse{\equal{#1}{}}{\newtheorem{h#2}[hthm]{#4}}{\newtheorem{h#2}{#4}[#1]}
\expandafter\newcommand\csname r#2\endcsname[1]{#3~\ref{#2:##1}}
\expandafter\newcommand\csname R#2\endcsname[1]{#4~\ref{#2:##1}}
\newenvironmentx{#2}[2][1=,2=]{
\ifthenelse{\equal{##2}{}}{\begin{h#2}}{\begin{h#2}[##2]}
\ifthenelse{\equal{##1}{}}{}{\label{#2:##1}}
}{\end{h#2}}
}

\newref[section]{thm}{theorem}{Theorem}
\newref{lem}{lemma}{Lemma}
\newref{prop}{proposition}{Proposition}
\newref{cor}{corollary}{Corollary}

\theoremstyle{definition}
\newref{defn}{definition}{Definition}
\newref{example}{example}{Example}

\theoremstyle{remark}
\newref{remark}{remark}{Remark}

\newcommand{\cat}[1]{\mathbf{#1}}
% \newcommand{\C}{\cat{C}}
\newcommand{\bs}{\beta\sigma}
\newcommand{\ebs}{=_{\bs}}
\newcommand{\rbs}{\to_{\bs}}
\newcommand{\bst}{\bs\tau}
\newcommand{\ebst}{=_{\bst}}
\newcommand{\rbst}{\to_{\bst}}
\newcommand{\sSet}{\cat{sSet}}

\numberwithin{figure}{section}

\begin{document}

\title{Canonicity for homotopy type theory with an interval type}

\author{Valery Isaev}

\begin{abstract}
In this paper, we prove canonicity for a version of homotopy type theory, which we call \emph{homotopy type theory with an interval type}.
We show that this theory essentially equivalent to Martin-L\"{o}f type theory with the univalence axiom.
In particular, this implies that every closed term of Martin-L\"{o}f type theory with one univalent universe is propositionally equivalent to a term in canonical form.
\end{abstract}

\maketitle

\section{Introduction}

\section{Syntax}

In this section, we will give inference rules for MLTT-U and HoTT-I.
First, let us describe inference and reduction rules for a basic system, which is a common part of MLTT-U and HoTT-I.
It has $\Pi$-types, $\Sigma$-types, natural numbers, and a universe, and it can be extended with other (higher) inductive types.
% It also has an interval type $I$, which has two constructors ($left$ and $right$) and one eliminator ($coe$).

\centerAlignProof

\begin{table}

\medskip
\begin{center}
\AxiomC{}
\UnaryInfC{$\varnothing \vdash$}
\DisplayProof
\quad
\AxiomC{$\Gamma \vdash A$}
\RightLabel{, $x \notin \Gamma$}
\UnaryInfC{$\Gamma, x : A \vdash$}
\DisplayProof
\quad
\AxiomC{$\Gamma \vdash$}
\RightLabel{, $x : A \in \Gamma$}
\UnaryInfC{$\Gamma \vdash x : A$}
\DisplayProof
\end{center}

\medskip
\begin{center}
\AxiomC{$\Gamma \vdash a : A$}
\AxiomC{$\Gamma \vdash B$}
\RightLabel{, $A =_{\beta \sigma \tau} B$}
\BinaryInfC{$\Gamma \vdash a : B$}
\DisplayProof
\end{center}

\medskip
\begin{center}
\AxiomC{$\Gamma \vdash A$}
\AxiomC{$\Gamma, x : A \vdash B$}
\BinaryInfC{$\Gamma \vdash \Pi (x : A) B$}
\DisplayProof
\quad
\AxiomC{$\Gamma \vdash A$}
\AxiomC{$\Gamma, x : A \vdash B$}
\BinaryInfC{$\Gamma \vdash \Sigma (x : A) B$}
\DisplayProof
\end{center}

\medskip
\begin{center}
\AxiomC{$\Gamma, x : A \vdash b : B$}
\UnaryInfC{$\Gamma \vdash \lambda x. b : \Pi (x : A) B$}
\DisplayProof
\quad
\AxiomC{$\Gamma \vdash f : \Pi (x : A) B$}
\AxiomC{$\Gamma \vdash a : A$}
\BinaryInfC{$\Gamma \vdash f\ a : B[x := a]$}
\DisplayProof
\end{center}

\medskip
\begin{center}
\AxiomC{$\Gamma \vdash a : A$}
\AxiomC{$\Gamma \vdash b : B[x : = a]$}
\AxiomC{$\Gamma \vdash B$}
\TrinaryInfC{$\Gamma \vdash (a,b) : \Sigma (x : A) B$}
\DisplayProof
\end{center}

\medskip
\begin{center}
\AxiomC{$\Gamma \vdash p : \Sigma (x : A) B$}
\UnaryInfC{$\Gamma \vdash proj_1\ p : A$}
\DisplayProof
\quad
\AxiomC{$\Gamma \vdash p : \Sigma (x : A) B$}
\UnaryInfC{$\Gamma \vdash proj_2\ p : B[x := proj_1\ p]$}
\DisplayProof
\end{center}

\medskip
\begin{center}
\AxiomC{$\Gamma \vdash$}
\UnaryInfC{$\Gamma \vdash \mathbb{N}$}
\DisplayProof
\quad
\AxiomC{$\Gamma \vdash$}
\UnaryInfC{$\Gamma \vdash 0 : \mathbb{N}$}
\DisplayProof
\quad
\AxiomC{$\Gamma \vdash n : \mathbb{N}$}
\UnaryInfC{$\Gamma \vdash S\ n : \mathbb{N}$}
\DisplayProof
\end{center}

\medskip
\begin{center}
\AxiomC{$\Gamma, x : \mathbb{N} \vdash A$}
\AxiomC{$\Gamma \vdash b : A[x := 0]$}
\AxiomC{$\Gamma, x : \mathbb{N}, r : A \vdash s : A[x := S\ x]$}
\AxiomC{$\Gamma \vdash n : \mathbb{N}$}
\QuaternaryInfC{$\Gamma \vdash R_{\lambda x. A}\ b\ (\lambda x r. s)\ n : A[x := n]$}
\DisplayProof
\end{center}

\medskip
\begin{center}
\AxiomC{$\Gamma \vdash$}
\UnaryInfC{$\Gamma \vdash Type$}
\DisplayProof
\quad
\AxiomC{$\Gamma \vdash A : Type$}
\UnaryInfC{$\Gamma \vdash A$}
\DisplayProof
\quad
\AxiomC{$\Gamma \vdash$}
\UnaryInfC{$\Gamma \vdash \mathbb{N} : Type$}
\DisplayProof
\end{center}

\medskip
\begin{center}
\AxiomC{$\Gamma \vdash A : Type$}
\AxiomC{$\Gamma, x : A \vdash B : Type$}
\BinaryInfC{$\Gamma \vdash \Pi (x : A) B : Type$}
\DisplayProof
\quad
\AxiomC{$\Gamma \vdash A : Type$}
\AxiomC{$\Gamma, x : A \vdash B : Type$}
\BinaryInfC{$\Gamma \vdash \Sigma (x : A) B : Type$}
\DisplayProof
\end{center}

\bigskip
\caption{Inference rules.}
\label{table:inf-rules}
\end{table}

Reduction rules:
\begin{itemize}
\item $(\lambda x.b)\ a \to_\beta b[x := a]$
\item $path\ (\lambda x. a)\ @\ i \to_\beta a[x := i]$
\item $coe_{\lambda k.A}\ i\ a\ i \to_\beta a$
\item $coe_{\lambda k.A}\ i\ a\ j \to_\sigma a$ if $k \notin FV(A)$
\end{itemize}

% \bibliographystyle{amsplain}
% \bibliography{ref}

\end{document}
