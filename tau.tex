\documentclass{amsart}

\usepackage{amssymb}
\usepackage{amsmath}
\usepackage[all]{xy}
\usepackage{verbatim}
\usepackage{ifthen}
\usepackage{xargs}
\usepackage{bussproofs}

\providecommand\WarningsAreErrors{false}
\ifthenelse{\equal{\WarningsAreErrors}{true}}{\renewcommand{\GenericWarning}[2]{\GenericError{#1}{#2}{}{This warning has been turned into a fatal error.}}}{}

\newcommand{\newref}[4][]{
\ifthenelse{\equal{#1}{}}{\newtheorem{h#2}[hthm]{#4}}{\newtheorem{h#2}{#4}[#1]}
\expandafter\newcommand\csname r#2\endcsname[1]{#3~\ref{#2:##1}}
\expandafter\newcommand\csname R#2\endcsname[1]{#4~\ref{#2:##1}}
\newenvironmentx{#2}[2][1=,2=]{
\ifthenelse{\equal{##2}{}}{\begin{h#2}}{\begin{h#2}[##2]}
\ifthenelse{\equal{##1}{}}{}{\label{#2:##1}}
}{\end{h#2}}
}

\newref[section]{thm}{theorem}{Theorem}
\newref{lem}{lemma}{Lemma}
\newref{prop}{proposition}{Proposition}
\newref{cor}{corollary}{Corollary}

\theoremstyle{definition}
\newref{defn}{definition}{Definition}
\newref{example}{example}{Example}

\theoremstyle{remark}
\newref{remark}{remark}{Remark}

\newcommand{\cat}[1]{\mathbf{#1}}
% \newcommand{\C}{\cat{C}}
\newcommand{\red}{\Rightarrow}
\newcommand{\bs}{\beta\sigma}
\newcommand{\ebs}{=_{\bs}}
\newcommand{\rbs}{\red_{\bs}}
\newcommand{\bst}{\bs\tau}
\newcommand{\ebst}{=_{\bst}}
\newcommand{\rbst}{\red_{\bst}}
\newcommand{\sSet}{\cat{sSet}}

\numberwithin{figure}{section}

\begin{document}

\title{Canonicity for homotopy type theory with one univalent universe}

\author{Valery Isaev}

\begin{abstract}
We prove that a version of homotopy type theory, which we call \emph{homotopy type theory with an interval type}, has the canonicity property.
\end{abstract}

\maketitle

\section{Syntax}

We will work in a version of Martin-L\"{o}f's type theory (MLTT), which we call \emph{homotopy type theory with an interval type} (HoTT-I).
It has all of the usual constructions of MLTT except for the identity types.

\subsection{Basic theory}

The inference and reduction rules for (some of) these construction are listed below.

\medskip
\begin{center}
\AxiomC{}
\UnaryInfC{$\vdash$}
\DisplayProof
\quad
\AxiomC{$\Gamma \vdash A$}
\RightLabel{, $x \notin \Gamma$}
\UnaryInfC{$\Gamma, x : A \vdash$}
\DisplayProof
\quad
\AxiomC{$\Gamma \vdash$}
\RightLabel{, $x : A \in \Gamma$}
\UnaryInfC{$\Gamma \vdash x : A$}
\DisplayProof
\end{center}

\medskip
\begin{center}
\AxiomC{$\Gamma \vdash a : A$}
\AxiomC{$\Gamma \vdash B$}
\RightLabel{, $A \equiv B$}
\BinaryInfC{$\Gamma \vdash a : B$}
\DisplayProof
\end{center}

\medskip
\begin{center}
\AxiomC{$\Gamma \vdash A$}
\AxiomC{$\Gamma, x : A \vdash B$}
\BinaryInfC{$\Gamma \vdash \Pi (x : A) B$}
\DisplayProof
\quad
\AxiomC{$\Gamma \vdash A$}
\AxiomC{$\Gamma, x : A \vdash B$}
\BinaryInfC{$\Gamma \vdash \Sigma (x : A) B$}
\DisplayProof
\end{center}

\medskip
\begin{center}
\AxiomC{$\Gamma, x : A \vdash b : B$}
\UnaryInfC{$\Gamma \vdash \lambda x.\ b : \Pi (x : A) B$}
\DisplayProof
\quad
\AxiomC{$\Gamma \vdash f : \Pi (x : A) B$}
\AxiomC{$\Gamma \vdash a : A$}
\BinaryInfC{$\Gamma \vdash f\ a : B[x := a]$}
\DisplayProof
\end{center}

\medskip
\begin{center}
\AxiomC{$\Gamma \vdash a : A$}
\AxiomC{$\Gamma \vdash b : B[x : = a]$}
\AxiomC{$\Gamma \vdash B$}
\TrinaryInfC{$\Gamma \vdash (a,b) : \Sigma (x : A) B$}
\DisplayProof
\end{center}

\medskip
\begin{center}
\AxiomC{$\Gamma \vdash p : \Sigma (x : A) B$}
\UnaryInfC{$\Gamma \vdash proj_1\ p : A$}
\DisplayProof
\quad
\AxiomC{$\Gamma \vdash p : \Sigma (x : A) B$}
\UnaryInfC{$\Gamma \vdash proj_2\ p : B[x := proj_1\ p]$}
\DisplayProof
\end{center}

\medskip
\begin{center}
\AxiomC{$\Gamma \vdash$}
\UnaryInfC{$\Gamma \vdash \mathbb{N}$}
\DisplayProof
\quad
\AxiomC{$\Gamma \vdash$}
\UnaryInfC{$\Gamma \vdash 0 : \mathbb{N}$}
\DisplayProof
\quad
\AxiomC{$\Gamma \vdash n : \mathbb{N}$}
\UnaryInfC{$\Gamma \vdash S\ n : \mathbb{N}$}
\DisplayProof
\end{center}

\medskip
\begin{center}
\AxiomC{$\Gamma, x : \mathbb{N} \vdash A$}
\AxiomC{$\Gamma \vdash b : A[x := 0]$}
\AxiomC{$\Gamma, x : \mathbb{N}, r : A \vdash s : A[x := S\ x]$}
\AxiomC{$\Gamma \vdash n : \mathbb{N}$}
\QuaternaryInfC{$\Gamma \vdash R_{\lambda x.\ A}\ b\ (\lambda x r.\ s)\ n : A[x := n]$}
\DisplayProof
\end{center}

\medskip
\begin{center}
\AxiomC{$\Gamma \vdash$}
\UnaryInfC{$\Gamma \vdash U_i$}
\DisplayProof
\quad
\AxiomC{$\Gamma \vdash A : U_i$}
\UnaryInfC{$\Gamma \vdash A$}
\DisplayProof
\quad
\AxiomC{$\Gamma \vdash A : U_i$}
\UnaryInfC{$\Gamma \vdash A : U_{i+1}$}
\DisplayProof
\quad
\AxiomC{$\Gamma \vdash$}
\UnaryInfC{$\Gamma \vdash \mathbb{N} : U_i$}
\DisplayProof
\end{center}

\medskip
\begin{center}
\AxiomC{$\Gamma \vdash A : U_i$}
\AxiomC{$\Gamma, x : A \vdash B : U_i$}
\BinaryInfC{$\Gamma \vdash \Pi (x : A) B : U_i$}
\DisplayProof
\quad
\AxiomC{$\Gamma \vdash A : U_i$}
\AxiomC{$\Gamma, x : A \vdash B : U_i$}
\BinaryInfC{$\Gamma \vdash \Sigma (x : A) B : U_i$}
\DisplayProof
\end{center}

\medskip
\begin{align*}
& (\lambda x.\ b)\ a \red_\beta b[x := a] \\
& proj_1\ (a, b) \red_\beta a \\
& proj_2\ (a, b) \red_\beta b \\
& R_{\lambda x.\ A}\ b\ (\lambda x r.\ s)\ 0 \red_\beta b \\
& R_{\lambda x.\ A}\ b\ (\lambda x r.\ s)\ (S\ n) \red_\beta s[x := n, r := R_{\lambda x.\ A}\ b\ (\lambda x r.\ s)\ n]
\end{align*}

We will write $(x_1 \ldots x_n : A) \to B$ for $\Pi (x_1 : A) \ldots \Pi (x_n : A) B$.

Apart from the usual types, HoTT-I also has an interval type $I$, which has two constructors $left$ and $right$ and one eliminator $coe$.
The inference rules for $I$, $left$ and $right$ are given below, and the inference rule for $coe$ will be described later.
\medskip
\begin{center}
\AxiomC{$\Gamma \vdash$}
\UnaryInfC{$\Gamma \vdash I : U_i$}
\DisplayProof
\quad
\AxiomC{$\Gamma \vdash$}
\UnaryInfC{$\Gamma \vdash left : I$}
\DisplayProof
\quad
\AxiomC{$\Gamma \vdash$}
\UnaryInfC{$\Gamma \vdash right : I$}
\DisplayProof
\end{center}

Using the interval type, we can define path types as functions from the interval type.
Path types have one constructor $path$ and one eliminator $@$.

\medskip
\begin{center}
\AxiomC{$\Gamma, x : I \vdash A$}
\AxiomC{$\Gamma \vdash a  : A[x := left]$}
\AxiomC{$\Gamma \vdash a' : A[x := right]$}
\TrinaryInfC{$\Gamma \vdash Path\ (\lambda x.\ A)\ a\ a'$}
\DisplayProof
\end{center}

\medskip
\begin{center}
\AxiomC{$\Gamma, x : I \vdash a : A$}
\UnaryInfC{$\Gamma \vdash path\ (\lambda x.\ a) : Path\ (\lambda x.\ A)\ a[x := left]\ a[x := right]$}
\DisplayProof
\end{center}

\medskip
\begin{center}
\AxiomC{$\Gamma \vdash p : Path\ (\lambda x.\ A)\ a\ a'$}
\AxiomC{$\Gamma \vdash i : I$}
\BinaryInfC{$\Gamma \vdash p\ @_{a,a'}\ i : A[x := i]$}
\DisplayProof
\end{center}

\medskip
\begin{center}
\AxiomC{$\Gamma, x : I \vdash A : U_i$}
\AxiomC{$\Gamma \vdash a  : A[x := left]$}
\AxiomC{$\Gamma \vdash a' : A[x := right]$}
\TrinaryInfC{$\Gamma \vdash Path\ (\lambda x.\ A)\ a\ a' : U_i$}
\DisplayProof
\end{center}

\begin{align*}
& path\ (\lambda x.\ t)\ @_{a,a'}\ i \red_\beta t[x := i] \\
& p\ @_{a,a'}\ left \red_\beta a \\
& p\ @_{a,a'}\ right \red_\beta a' \\
& path\ (\lambda x.\ p @_{a,a'} x) \equiv_\eta p
\end{align*}

We write $a =_A a'$ (or simply $a = a'$) for $Path\ (\lambda x.\ A)\ a\ a'$ if $x \notin FV(A)$.

There are several equivalent to define an elimination rule for the interval type, but the idea is the same:
we have a dependent type $A$ over $I$ and a point in some fibre of $A$, then we can extend this point to the whole interval.
First, we describe the basic version of $coe$:

\medskip
\begin{center}
\AxiomC{$\Gamma, x : I \vdash A$}
\AxiomC{$\Gamma \vdash a : A[x := left]$}
\AxiomC{$\Gamma \vdash j : I$}
\TrinaryInfC{$\Gamma \vdash coe_1\ (\lambda x.\ A)\ a\ j : A[x := j]$}
\DisplayProof
\end{center}

\[ coe_1\ (\lambda x.\ A)\ a\ left \red_\beta a \]

Using $coe_1$, we can show that path types satisfy the usual $J$ rule.
Reflexivity is defined simply as the constant path:
\begin{align*}
& refl : (x : A) \to x = x \\
& refl = \lambda x.\ path\ (\lambda i.\ x)
\end{align*}
To construct $J$, first we need to define function $squeeze$, which satisfies the following rules:
\begin{align*}
& squeeze : I \to I \to I \\
& squeeze\ left\ j \equiv left \\
& squeeze\ right\ j \equiv j \\
& squeeze\ i\ left \equiv left \\
& squeeze\ i\ right \equiv i
\end{align*}

To define $squeeze$, first we define function $squeeze'$ which satisfies all of the equations above except for the last one:
\[ squeeze' = \lambda i j.\ coe_1\ (\lambda x.\ left = x)\ (path\ (\lambda x.\ left))\ j\ @\ i \]
Now we can define $squeeze$ as follows:
\begin{align*}
& squeeze = \lambda i j.\ coe_1\ (\lambda i.\ Path\ (\lambda j.\ left = squeeze'\ i\ j)\ (path\ (\lambda x.\ left)) \\
& \qquad (path\ (\lambda j.\ squeeze'\ i\ j)))\ (path\ (\lambda x.\ path\ (\lambda x.\ left)))\ right\ @\ i\ @\ j
\end{align*}
The idea is that the right hand sides of the definitions of $squeeze'$ and $squeeze$ describe fillers of certain cubical horns of dimension two and three respectively.

Using $squeeze$ we can defined function $psqueeze$:
\begin{align*}
& psqueeze : (x\ y : A) \to (p : x = y) \to (i : I) \to x = p\ @\ i \\
& psqueeze = \lambda x\ y\ p\ i.\ path\ (\lambda j.\ p\ @\ squeeze\ i\ j)
\end{align*}
Using $psqueeze$ we can define $J$ which satisfies the following inference rule:
\begin{center}
\AxiomC{$\Gamma, x : A, x' : A, z : x = x' \vdash D$}
\AxiomC{$\Gamma, x : A \vdash d : D[x' := x, z := refl\ x]$}
\AxiomC{$\Gamma \vdash p : a = a'$}
\TrinaryInfC{$\Gamma \vdash J\ (\lambda x x' z.\ D)\ (\lambda x.\ d)\ a\ a'\ p : D[x := a, x' := a', z := p]$}
\DisplayProof
\end{center}
\medskip
$J\ (\lambda x x' z.\ D)\ (\lambda x.\ d)\ a\ a'\ p$ is defined as follows:
\[ coe_1\ (\lambda i.\ D[x := a, x' := p\ @\ i, z := psqueeze\ a\ a'\ p\ i])\ (d[x := a])\ right \]

The usual computation rule does not hold for this version of $J$.
We can fix this by adding the following additional reduction rule for $coe_1$:
\[ coe_1\ (\lambda x.\ A)\ a\ i \red_\sigma a \text{, if } x \notin FV(A) \]
But it is not necessary, since this rule holds propositionally.

\subsection{Other versions of $coe$}

We can define $coe$ in a different way:
\medskip
\begin{center}
\AxiomC{$\Gamma, x : I \vdash A$}
\AxiomC{$\Gamma \vdash i : I$}
\AxiomC{$\Gamma \vdash a : A[x := i]$}
\AxiomC{$\Gamma \vdash j : I$}
\QuaternaryInfC{$\Gamma \vdash coe_2\ (\lambda x.\ A)\ i\ a\ j : A[x := j]$}
\DisplayProof
\end{center}

Obiously, $coe_1$ is a special case of $coe_2$, but $coe_2$ can be constructed from $coe_1$.
Indeed, first let us prove that $I$ is contractible:
\[ (left, \lambda x.\ path\ (\lambda i.\ squeeze\ x\ i) ) \]
It follows that for every $i, j : I$ we have term $pp\ i\ j$ of type $i = j$.
Now we can define $coe_2$ as follows:
\[ coe_2\ (\lambda x.\ A)\ i\ a\ j = coe_1\ (\lambda y.\ A[x := pp\ i\ j\ @\ y])\ a\ right \]

Yet another version of $coe$ is the following one:
\medskip
\begin{center}
\AxiomC{$\Gamma, x : I \vdash A$}
\AxiomC{$\Gamma \vdash a : A[x := left]$}
\BinaryInfC{$\Gamma \vdash coe_0\ (\lambda x.\ A)\ a : A[x := right]$}
\DisplayProof
\end{center}
This version seems to be weaker then $coe_1$, but if we have $squeeze$, then we can define $coe_1$ as follows:
\[ coe_1\ (\lambda x.\ A)\ a\ i = coe_0\ (\lambda y.\ A[x := squeeze\ y\ i])\ a \]
In the rest of this document we will use this version of $coe$.
We will assume that $squeeze$ is defined as a primitive operation which satisfies four rules that we described.

We will also need function $inv : I \to I$ which satisfies the following rules:
\begin{align*}
& inv\ left \equiv right \\
& inv\ right \equiv left
\end{align*}
To define it, first note that we have function $sym : a = a' \to a' = a$.
Now we can define $inv$ as follows:
\[ \lambda x.\ sym\ (path\ (\lambda y.\ y))\ @\ x \]

\section{Canonicity}

In this section we will add more construction and reduction rules to the theory so that it will have the canonicity property.
The problem with the current version of the theory is that terms of the form $coe\ (\lambda x.\ A)\ a$ are never reduce.
To fix this we define new reduction rules by recursion on $A$.

\subsection{Basic theory}

We already saw that we can define fillers using $coe$.
It is convenient to add fillers as separate constructions.
But before we do this let us introduce a bit of notation.
We will write $\overline{x}$ for a sequence of variables or terms $x_1\ \ldots\ x_n$,
    and we will write $\overline{x}_{\hat{i}}$ for the sequence $x_1\ \ldots\ x_{i - 1}\ x_{i + 1}\ \ldots\ x_n$.
We use the same notation in other cases.
For example, $\overline{x : I}$ denotes $x_1 : I, \ldots x_n : I$, and $A \overline{[x := j]}$ denotes $A[x_1 := j_1] \ldots [x_n := j_n]$.
Now we can define the inference rules for fillers:
\begin{center}
\Axiom$\fCenter \Gamma, \overline{x : I} \vdash A$
\def\extraVskip{1pt}
\noLine
\UnaryInf$\fCenter \Gamma, \overline{x : I}_{\hat{k}} \vdash a_k : A[x_k := left] \text{, } 1 \leq k \leq n$
\noLine
\UnaryInf$\fCenter \Gamma, \overline{x : I}_{\hat{k}} \vdash a'_k : A[x_k := right] \text{, } 2 \leq k \leq n$
\noLine
\UnaryInf$\fCenter \Gamma \vdash j_k : I \text{, } 2 \leq k \leq n$
\noLine
\UnaryInf$\fCenter a_{i_1}[x_{i_2} := left] \equiv a_{i_2}[x_{i_1} := left] \text{, } 1 \leq i_1 < i_2 \leq n$
\noLine
\UnaryInf$\fCenter a'_{i_1}[x_{i_2} := left] \equiv a_{i_2}[x_{i_1} := right] \text{, } 2 \leq i_1 < i_2 \leq n$
\noLine
\UnaryInf$\fCenter a_{i_1}[x_{i_2} := right] \equiv a'_{i_2}[x_{i_1} := left] \text{, } 1 \leq i_1 < i_2 \leq n$
\noLine
\UnaryInf$\fCenter a'_{i_1}[x_{i_2} := right] \equiv a'_{i_2}[x_{i_1} := right] \text{, } 2 \leq i_1 < i_2 \leq n$
\def\extraVskip{2pt}
\UnaryInf$\fCenter \Gamma \vdash Fill^n\ (\lambda \overline{x}.\ A)\ (\lambda \overline{x}_{\hat{n}}.\ a_n)\ (\lambda \overline{x}_{\hat{n}}.\ a'_n)\ \ldots\ (\lambda \overline{x}_{\hat{2}}.\ a_2)\ (\lambda \overline{x}_{\hat{2}}.\ a'_2)\ (\lambda \overline{x}_{\hat{1}}.\ a_1)\ \overline{j}_{\hat{1}} : A [x_1 := right] \overline{[x := j]}_{\hat{1}}$
\DisplayProof
\end{center}
\medskip
These terms are defined for every $n \geq 2$.
It is actually possible to define these terms using $coe$, but it is convenient to introduce them as primitive constructions.
Note that the rule for $Fill^1$ is the same as for $coe_0$.
So we will write $Fill^1$ for $coe_0$.

As in the case of $coe_0$, there are versions of $Fill^n$ that define filler for the whole box and not only for the lid.
We can define them as follows:
\[ Fill^n_1\ (\lambda \overline{x}.\ A)\ (\lambda \overline{x}_{\hat{n}}.\ a_n)\ (\lambda \overline{x}_{\hat{n}}.\ a'_n) \ldots (\lambda \overline{x}_{\hat{2}}.\ a_2)\ (\lambda \overline{x}_{\hat{2}}.\ a'_2)\ (\lambda \overline{x}_{\hat{1}}.\ a_1)\ \overline{j} = \]
\[ Fill^n\ (\lambda \overline{x}.\ A[x_1 := squeeze\ x_1\ j_1])\ (\lambda \overline{x}_{\hat{n}}.\ a_n[x_1 := squeeze\ x_1\ j_1])\ (\lambda \overline{x}_{\hat{n}}.\ a'_n[x_1 := squeeze\ x_1\ j_1]) \ldots \]
\[ (\lambda \overline{x}_{\hat{2}}.\ a_2[x_1 := squeeze\ x_1\ j_1])\ (\lambda \overline{x}_{\hat{2}}.\ a'_2[x_1 := squeeze\ x_1\ j_1])\ (\lambda \overline{x}_{\hat{1}}.\ a_1)\ \overline{j}_{\hat{1}} \]

There are a few $\beta$ rules for $Fill^n$:
\[ Fill^n\ (\lambda \overline{x}.\ A)\ (\lambda \overline{x}_{\hat{n}}.\ a_n)\ (\lambda \overline{x}_{\hat{n}}.\ a'_n)\ \ldots\ (\lambda \overline{x}_{\hat{2}}.\ a_2)\ (\lambda \overline{x}_{\hat{2}}.\ a'_2)\ (\lambda \overline{x}_{\hat{1}}.\ a_1)\ j_2\ \ldots j_{i-1}\ left\ j_{i+1}\ \ldots\ j_n \red_\beta \]
\[ a_i[x_1 := right][x_2 := j_2] \ldots [x_{i-1} := j_{i-1}] [x_{i+1} := j_{i+1}] \ldots [x_n := j_n] \]
\[ Fill^n\ (\lambda \overline{x}.\ A)\ (\lambda \overline{x}_{\hat{n}}.\ a_n)\ (\lambda \overline{x}_{\hat{n}}.\ a'_n)\ \ldots\ (\lambda \overline{x}_{\hat{2}}.\ a_2)\ (\lambda \overline{x}_{\hat{2}}.\ a'_2)\ (\lambda \overline{x}_{\hat{1}}.\ a_1)\ j_2\ \ldots j_{i-1}\ right\ j_{i+1}\ \ldots\ j_n \red_\beta \]
\[ a'_i[x_1 := right][x_2 := j_2] \ldots [x_{i-1} := j_{i-1}] [x_{i+1} := j_{i+1}] \ldots [x_n := j_n] \]

We can also define the sigma rule for every $Fill^n$:
\[ Fill^n\ (\lambda \overline{x}.\ A)\ (\lambda \overline{x}_{\hat{n}}.\ a_n)\ (\lambda \overline{x}_{\hat{n}}.\ a'_n)\ \ldots\ (\lambda \overline{x}_{\hat{2}}.\ a_2)\ (\lambda \overline{x}_{\hat{2}}.\ a'_2)\ (\lambda \overline{x}_{\hat{1}}.\ a_1)\ j_2\ \ldots\ j_n \red_\sigma \]
\[ a_1[x_2 := j_2] \ldots [x_n := j_n] \text{ if } x_1 \notin FV(A) \cup FV(a_n) \cup FV(a'_n) \cup \ldots \cup FV(a_2) \cup FV(a'_2) \]

But as in the case of $coe_0$ this rule is not necessary.
To make the theory computational, we need to add new rules, which we call $\tau$, for every type construction that we have: $Path$, $\Sigma$, $\Pi$, $\mathbb{N}$ and $U_i$.
It is easy to define $\tau$ rules for $\mathbb{N}$, path and sigma types.
For every $n \geq 1$, we add the following rules:
\[ Fill^n\ (\lambda \overline{x}.\ \mathbb{N})\ (\lambda \overline{x}_{\hat{n}}.\ 0)\ (\lambda \overline{x}_{\hat{n}}.\ 0)\ \ldots\ (\lambda \overline{x}_{\hat{2}}.\ 0)\ (\lambda \overline{x}_{\hat{2}}.\ 0)\ (\lambda \overline{x}_{\hat{1}}.\ 0)\ \overline{j}_{\hat{1}}\ \red_\tau 0 \]
\[ Fill^n\ (\lambda \overline{x}.\ \mathbb{N})\ (\lambda \overline{x}_{\hat{n}}.\ S\ a_n)\ (\lambda \overline{x}_{\hat{n}}.\ S\ a'_n)\ \ldots\ (\lambda \overline{x}_{\hat{2}}.\ S\ a_2)\ (\lambda \overline{x}_{\hat{2}}.\ S\ a'_2)\ (\lambda \overline{x}_{\hat{1}}.\ S\ a_1)\ \overline{j}_{\hat{1}}\ \red_\tau \]
\[ S\ (Fill^n\ (\lambda \overline{x}.\ \mathbb{N})\ (\lambda \overline{x}_{\hat{n}}.\ a_n)\ (\lambda \overline{x}_{\hat{n}}.\ a'_n)\ \ldots\ (\lambda \overline{x}_{\hat{2}}.\ a_2)\ (\lambda \overline{x}_{\hat{2}}.\ a'_2)\ (\lambda \overline{x}_{\hat{1}}.\ a_1)\ \overline{j}_{\hat{1}}) \]

\[ Fill^n\ (\lambda \overline{x}.\ Path\ (\lambda x_{n+1}. A)\ a_{n+1}\ a'_{n+1})\ (\lambda \overline{x}_{\hat{n}}.\ path\ (\lambda x_{n+1}.\ a_n))\ (\lambda \overline{x}_{\hat{n}}.\ path\ (\lambda x_{n+1}.\ a'_n))\ \ldots \]
\[ (\lambda \overline{x}_{\hat{2}}.\ path\ (\lambda x_{n+1}.\ a_2))\ (\lambda \overline{x}_{\hat{2}}.\ path\ (\lambda x_{n+1}.\ a'_2))\ (\lambda \overline{x}_{\hat{1}}.\ path\ (\lambda x_{n+1}.\ a_1))\ j_2\ \ldots\ j_n \red_\tau \]
\[ path\ (\lambda j_{n+1}.\ Fill^{n+1}\ (\lambda \overline{x}.\ \lambda x_{n+1}.\ A)\ (\lambda \overline{x}.\ a_{n+1})\ (\lambda \overline{x}.\ a'_{n+1})\ \ldots \]
\[ (\lambda \overline{x}_{\hat{2}}.\ \lambda x_{n+1}.\ a_2)\ (\lambda \overline{x}_{\hat{2}}.\ \lambda x_{n+1}.\ a'_2)\ \ldots\ (\lambda \overline{x}_{\hat{1}}.\ \lambda x_{n+1}.\ a_1)\ j_2\ \ldots\ j_{n+1}) \]

\[ Fill^n\ (\lambda \overline{x}.\ \Sigma (y : A) B)\ (\lambda \overline{x}_{\hat{n}}.\ (a_n, b_n))\ (\lambda \overline{x}_{\hat{n}}.\ (a'_n, b'_n))\ \ldots\ (\lambda \overline{x}_{\hat{2}}.\ (a_2, b_2))\ (\lambda \overline{x}_{\hat{2}}.\ (a'_2, b'_2))\ (\lambda \overline{x}_{\hat{1}}.\ (a_1, b_1))\ \overline{j}_{\hat{1}}\ \red_\tau \]
\[ (Fill^n\ (\lambda \overline{x}.\ A)\ (\lambda \overline{x}_{\hat{n}}.\ a_n)\ (\lambda \overline{x}_{\hat{n}}.\ a'_n)\ \ldots\ (\lambda \overline{x}_{\hat{2}}.\ a_2)\ (\lambda \overline{x}_{\hat{2}}.\ a'_2)\ (\lambda \overline{x}_{\hat{1}}.\ a_1)\ \overline{j}_{\hat{1}}, \]
\[ Fill^n\ (\lambda \overline{x}.\ B[y := Fill^n_1\ (\lambda \overline{x}.\ A)\ (\lambda \overline{x}_{\hat{n}}.\ a_n)\ (\lambda \overline{x}_{\hat{n}}.\ a'_n)\ \ldots\ (\lambda \overline{x}_{\hat{2}}.\ a_2)\ (\lambda \overline{x}_{\hat{2}}.\ a'_2)\ (\lambda \overline{x}_{\hat{1}}.\ a_1)\ \overline{x}]) \]
\[ (\lambda \overline{x}_{\hat{n}}.\ b_n)\ (\lambda \overline{x}_{\hat{n}}.\ b'_n)\ \ldots\ (\lambda \overline{x}_{\hat{2}}.\ b_2)\ (\lambda \overline{x}_{\hat{2}}.\ b'_2)\ (\lambda \overline{x}_{\hat{1}}.\ b_1)\ \overline{j}_{\hat{1}}) \]

To define $\tau$ rules for $\Pi$, we need an operation that transports a point in a fibre of a type over a cube to any other fibre.
We already have such operation for the interval, namely $coe_2$.
We can define the general one in terms of $coe_2$:
\begin{align*}
& coe^0\ A\ a = a \\
& coe^{n+1}\ (\lambda \overline{x} x_{n+1}.\ A)\ \overline{i}\ i_{n+1}\ a\ \overline{j}\ j_{n+1} = \\
& \qquad coe_2\ (\lambda x_{n+1}.\ A \overline{[x := j]})\ i_{n+1}\ (coe^n\ (\lambda \overline{x}.\ A[x_{n+1} := i_{n+1}])\ \overline{i}\ a\ \overline{j})\ j_{n+1}
\end{align*}

Now, for every $n \geq 1$, we can define $\tau$ rules for $\Pi$:
\[ Fill^n\ (\lambda \overline{x}.\ \Pi (y : A) B)\ (\lambda \overline{x}_{\hat{n}}.\ \lambda y.\ b_n)\ (\lambda \overline{x}_{\hat{n}}.\ \lambda y.\ b'_n)\ \ldots\ (\lambda \overline{x}_{\hat{2}}.\ \lambda y.\ b_2)\ (\lambda \overline{x}_{\hat{2}}.\ \lambda y.\ b'_2)\ (\lambda \overline{x}_{\hat{1}}.\ \lambda y.\ b_1)\ \overline{j}_{\hat{1}} \red_\tau \]
\[ \lambda y'.\ Fill^n\ (\lambda \overline{x}.\ B[y := coe^n\ (\lambda \overline{x}.\ A)\ right\ j_2\ \ldots\ j_n\ y'\ x_1\ \ldots\ x_n]) \]
\[ (\lambda \overline{x}_{\hat{n}}.\ b_n[y := coe^n\ (\lambda \overline{x}.\ A)\ right\ j_2\ \ldots\ j_n\ y'\ x_1\ \ldots\ x_{n-1}\ left]) \]
\[ (\lambda \overline{x}_{\hat{n}}.\ b'_n[y := coe^n\ (\lambda \overline{x}.\ A)\ right\ j_2\ \ldots\ j_n\ y'\ x_1\ \ldots\ x_{n-1}\ right])\ \ldots \]
\[ (\lambda \overline{x}_{\hat{2}}.\ b_2[y := coe^n\ (\lambda \overline{x}.\ A)\ right\ j_2\ \ldots\ j_n\ y'\ x_1\ left\ x_3\ \ldots\ x_n]) \]
\[ (\lambda \overline{x}_{\hat{2}}.\ b'_2[y := coe^n\ (\lambda \overline{x}.\ A)\ right\ j_2\ \ldots\ j_n\ y'\ x_1\ right\ x_3\ \ldots\ x_n]) \]
\[ (\lambda \overline{x}_{\hat{1}}.\ b_1[y := coe^n\ (\lambda \overline{x}.\ A)\ right\ j_2\ \ldots\ j_n\ y'\ left\ x_2\ \ldots\ x_n])\ j_1\ \ldots\ j_n \]

\subsection{Universes}

\[ Fill^1_{\lambda x. iso\,A\,B\,f\,g\,p\,q\,i}\ i'\ a\ j' \red_\tau Fill^1_{\lambda x. A}\ i'\ a\ j' \text{ if } i[x := i'] \ebst left, i[x := j'] \ebst left \]
\[ Fill^1_{\lambda x. iso\,A\,B\,f\,g\,p\,q\,i}\ i'\ a\ j' \red_\tau f[x := j'] (Fill^1_{\lambda x. A}\ i'\ a\ j') \text{ if } i[x := i'] \ebst left, i[x := j'] \ebst right \]
\[ Fill^1_{\lambda x. iso\,A\,B\,f\,g\,p\,q\,i}\ i'\ a\ j' \red_\tau g[x := j'] (Fill^1_{\lambda x. B}\ i'\ a\ j') \text{ if } i[x := i'] \ebst right, i[x := j'] \ebst left \]
\[ Fill^1_{\lambda x. iso\,A\,B\,f\,g\,p\,q\,i}\ i'\ a\ j' \red_\tau Fill^1_{\lambda x. B}\ i'\ a\ j' \text{ if } i[x := i'] \ebst right, i[x := j'] \ebst right \]

\begin{table}

\begin{center}
\Axiom$\fCenter \Gamma \vdash A : Type$
\noLine
\UnaryInf$\fCenter \Gamma \vdash B : Type$
\def\extraVskip{1pt}
\Axiom$\fCenter \Gamma \vdash f : A \to B$
\noLine
\UnaryInf$\fCenter \Gamma \vdash g : B \to A$
\Axiom$\fCenter \Gamma \vdash p : (a : A) \to g\ (f\ a) =_A a$
\noLine
\UnaryInf$\fCenter \Gamma \vdash q : (b : B) \to f\ (g\ b) =_B b$
\def\extraVskip{2pt}
\AxiomC{$\Gamma \vdash i : I$}
\QuaternaryInfC{$\Gamma \vdash iso\ A\ B\ f\ g\ p\ q\ i : Type$}
\DisplayProof
\end{center}
\medskip

\begin{align*}
& iso\ A\ B\ f\ g\ p\ q\ left \red_\beta A \\
& iso\ A\ B\ f\ g\ p\ q\ right \red_\beta B \\
& coe_{\lambda k. iso\,A\,B\,f\,g\,p\,q\,k}\ left\ a\ right \red_\beta f\ a \text{ if } k \notin FV(A\ B\ f\ g\ p\ q) \\
& coe_{\lambda k. iso\,A\,B\,f\,g\,p\,q\,k}\ right\ b\ left \red_\beta g\ b \text{ if } k \notin FV(A\ B\ f\ g\ p\ q)
\end{align*}

\bigskip
\caption{Inference rules for univalence.}
\label{table:Univalence-rules}
\end{table}

% \bibliographystyle{amsplain}
% \bibliography{ref}

\end{document}
