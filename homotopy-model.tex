\documentclass{amsart}

\usepackage[english,russian]{babel}
\usepackage[utf8]{inputenc}
\usepackage{amssymb}
\usepackage[all]{xy}
\usepackage{verbatim}
\usepackage{ifthen}
\usepackage{xargs}
\usepackage{bussproofs}
\usepackage{type1ec}
\usepackage{stmaryrd}
% \usepackage[T2A]{fontenc}

\providecommand\WarningsAreErrors{false}
\ifthenelse{\equal{\WarningsAreErrors}{true}}{\renewcommand{\GenericWarning}[2]{\GenericError{#1}{#2}{}{This warning has been turned into a fatal error.}}}{}

\newcommand{\newref}[4][]{
\ifthenelse{\equal{#1}{}}{\newtheorem{h#2}[hthm]{#4}}{\newtheorem{h#2}{#4}[#1]}
\expandafter\newcommand\csname r#2\endcsname[1]{#3~\ref{#2:##1}}
\expandafter\newcommand\csname R#2\endcsname[1]{#4~\ref{#2:##1}}
\newenvironmentx{#2}[2][1=,2=]{
\ifthenelse{\equal{##2}{}}{\begin{h#2}}{\begin{h#2}[##2]}
\ifthenelse{\equal{##1}{}}{}{\label{#2:##1}}
}{\end{h#2}}
}

\newref[section]{thm}{theorem}{Theorem}
\newref{lem}{lemma}{Lemma}
\newref{prop}{proposition}{Proposition}
\newref{cor}{corollary}{Corollary}

\theoremstyle{definition}
\newref{defn}{definition}{Definition}
\newref{example}{example}{Example}

\theoremstyle{remark}
\newref{remark}{remark}{Remark}

\newcommand{\red}{\Rightarrow}
\newcommand{\deq}{\Leftrightarrow}
\renewcommand{\ll}{\llbracket}
\newcommand{\rr}{\rrbracket}

\newcommand{\pb}[1][dr]{\save*!/#1-1.2pc/#1:(-1,1)@^{|-}\restore}
\newcommand{\po}[1][dr]{\save*!/#1+1.2pc/#1:(1,-1)@^{|-}\restore}

\numberwithin{figure}{section}

\begin{document}

\title{An interpretation of homotopy type theory in categories with attributes}

\author{Valery Isaev}

% \begin{abstract}
% Abstract
% \end{abstract}

\maketitle

\section{Правила вывода}

\begin{comment}
\begin{table}

\medskip
\begin{center}
\AxiomC{}
\UnaryInfC{$\varnothing \vdash$}
\DisplayProof
\quad
\AxiomC{$\Gamma \vdash A$}
\RightLabel{, $x \notin \Gamma$}
\UnaryInfC{$\Gamma, x : A \vdash$}
\DisplayProof
\end{center}

\medskip
\begin{center}
\AxiomC{$\Gamma \vdash A$}
\RightLabel{, $x \notin \Gamma$}
\UnaryInfC{$\Gamma, x : A \vdash x : A$}
\DisplayProof
\quad
\AxiomC{$\Gamma \vdash x : A$}
\AxiomC{$\Gamma \vdash B$}
\RightLabel{, $y \notin \Gamma$}
\BinaryInfC{$\Gamma, y : B \vdash x : A$}
\DisplayProof
\end{center}

\medskip
\begin{center}
\AxiomC{$\Gamma \vdash a : A$}
\AxiomC{$\Gamma \vdash B$}
\RightLabel{, $A \deq B$}
\BinaryInfC{$\Gamma \vdash a : B$}
\DisplayProof
\end{center}

\medskip
\begin{center}
\AxiomC{$\Gamma \vdash A \red B$}
\UnaryInfC{$\Gamma \vdash A \deq B$}
\DisplayProof
\quad
\AxiomC{$\Gamma \vdash A \deq B$}
\UnaryInfC{$\Gamma \vdash B \deq A$}
\DisplayProof
\end{center}

\medskip
\begin{center}
\AxiomC{$\Gamma \vdash A \deq B$}
\AxiomC{$\Gamma \vdash B \deq C$}
\BinaryInfC{$\Gamma \vdash A \deq C$}
\DisplayProof
\end{center}

\medskip
\begin{center}
\AxiomC{$\Gamma \vdash a \red a' : A$}
\UnaryInfC{$\Gamma \vdash a \deq a' : A$}
\DisplayProof
\quad
\AxiomC{$\Gamma \vdash a \deq a' : A$}
\UnaryInfC{$\Gamma \vdash a' \deq a : A$}
\DisplayProof
\end{center}

\medskip
\begin{center}
\AxiomC{$\Gamma \vdash a \deq a' : A$}
\AxiomC{$\Gamma \vdash a' \deq a'' : A$}
\BinaryInfC{$\Gamma \vdash a \deq a'' : A$}
\DisplayProof
\end{center}

\medskip
\caption{Правила вывода.}
\label{table:inf-rules}
\end{table}
\end{comment}

\begin{table}

\medskip
\begin{center}
\AxiomC{}
\UnaryInfC{$\varnothing \vdash$}
\DisplayProof
\quad
\AxiomC{$\Gamma \vdash A$}
\RightLabel{, $x \notin \Gamma$}
\UnaryInfC{$\Gamma, x : A \vdash$}
\DisplayProof
\end{center}

\medskip
\begin{center}
\AxiomC{$\Gamma \vdash A$}
\RightLabel{, $x \notin \Gamma$}
\UnaryInfC{$\Gamma, x : A \vdash x \Uparrow A$}
\DisplayProof
\quad
\AxiomC{$\Gamma \vdash x \Uparrow A$}
\AxiomC{$\Gamma \vdash B$}
\RightLabel{, $y \notin \Gamma$}
\BinaryInfC{$\Gamma, y : B \vdash x \Uparrow A$}
\DisplayProof
\end{center}

\medskip
\begin{center}
\AxiomC{$\Gamma \vdash a \Uparrow A$}
\AxiomC{$\Gamma \vdash B$}
\RightLabel{, $A \deq B$}
\BinaryInfC{$\Gamma \vdash a \Downarrow B$}
\DisplayProof
\end{center}

\begin{comment}
\medskip
\begin{center}
\AxiomC{$\Gamma \vdash A \red B$}
\UnaryInfC{$\Gamma \vdash A \deq B$}
\DisplayProof
\quad
\AxiomC{$\Gamma \vdash A \deq B$}
\UnaryInfC{$\Gamma \vdash B \deq A$}
\DisplayProof
\end{center}

\medskip
\begin{center}
\AxiomC{$\Gamma \vdash A \deq B$}
\AxiomC{$\Gamma \vdash B \deq C$}
\BinaryInfC{$\Gamma \vdash A \deq C$}
\DisplayProof
\end{center}

\medskip
\begin{center}
\AxiomC{$\Gamma \vdash a \red a' : A$}
\UnaryInfC{$\Gamma \vdash a \deq a' : A$}
\DisplayProof
\quad
\AxiomC{$\Gamma \vdash a \deq a' : A$}
\UnaryInfC{$\Gamma \vdash a' \deq a : A$}
\DisplayProof
\end{center}

\medskip
\begin{center}
\AxiomC{$\Gamma \vdash a \deq a' : A$}
\AxiomC{$\Gamma \vdash a' \deq a'' : A$}
\BinaryInfC{$\Gamma \vdash a \deq a'' : A$}
\DisplayProof
\end{center}
\end{comment}

\medskip
\caption{Правила вывода.}
\label{table:bi-inf-rules}
\end{table}

\newpage

\section{Интерпретация}

\subsection{Определение интерпретации}

Интерпретация контекстов:

\medskip
\begin{center}
\AxiomC{}
\UnaryInfC{$\ll \varnothing \vdash \rr = 1$}
\DisplayProof
\quad
\AxiomC{$\ll \Gamma \vdash A \rr = A \dotsb \Gamma$}
\RightLabel{, $x \notin \Gamma$}
\UnaryInfC{$\ll \Gamma, x : A \vdash \rr = \Gamma.A$}
\DisplayProof
\end{center}
\bigskip

Интерпретация переменных:

\medskip
\begin{center}
\AxiomC{$\ll \Gamma \vdash A \rr = A \dotsb \Gamma$}
\RightLabel{, $x \notin \Gamma$}
\UnaryInfC{$\ll \Gamma, x : A \vdash x \Uparrow A \rr = a : \pi_A^*(A) \dotsb \Gamma.A$}
\DisplayProof
\end{center}

где $a$ определяется как в следующей диаграме:
\[ \xymatrix@-1pc{ \Gamma.A \ar[rddd]_{id_{\Gamma.A}} \ar[rrrd]^{id_{\Gamma.A}} \ar@{-->}[rd]_a & & & \\
                        & \Gamma.A.\pi_A^*(A) \ar[rr]_-{\pi_A.A} \ar[dd]^{\pi_{\pi_A^*(A)}} \pb & & \Gamma.A \ar[dd]^{\pi_A} \\
                        &                                                                       & & \\
                        & \Gamma.A \ar[rr]_{\pi_A}                                              & & \Gamma
                 }\]

\begin{center}
\AxiomC{$\ll \Gamma \vdash x \Uparrow A \rr = a : A \dotsb \Gamma$}
\AxiomC{$\ll \Gamma \vdash B \rr = B \dotsb \Gamma$}
\RightLabel{, $y \notin \Gamma$}
\BinaryInfC{$\ll \Gamma, y : B \vdash x \Uparrow A \rr = a' : \pi_B^*(A) \dotsb \Gamma.B$}
\DisplayProof
\end{center}

где $a'$ определяется как в следующей диаграме:
\[ \xymatrix@-1pc{ \Gamma.B \ar[rddd]_{id_{\Gamma.B}} \ar[rrrd]^{a \circ \pi_B} \ar@{-->}[rd]_{a'} & & & \\
                        & \Gamma.B.\pi_B^*(A) \ar[rr]_-{\pi_B.A} \ar[dd]^{\pi_{\pi_B^*(A)}} \pb & & \Gamma.A \ar[dd]^{\pi_A} \\
                        &                                                                       & & \\
                        & \Gamma.B \ar[rr]_{\pi_B}                                              & & \Gamma
                 }\]

Интерпретация правила проверки типа:

\medskip
\begin{center}
\AxiomC{$\ll \Gamma \vdash a \Uparrow A_1 \rr = a : A_1 \dotsb \Gamma$}
\AxiomC{$\ll \Gamma \vdash A_i \red^* B \rr = f_i : A_i \to B \dotsb \Gamma$, $i \in \{1, 2\}$}
\RightLabel{, $B \in NF$}
\BinaryInfC{$\ll \Gamma \vdash a \Downarrow A_2 \rr = f_2^{-1} \circ f_1 \circ a : A_2 \dotsb \Gamma$}
\DisplayProof
\end{center}
\bigskip

Интерпретация правил редукций типов:

\medskip
\begin{center}
\AxiomC{$\ll \Gamma \vdash A \rr = A \dotsb \Gamma$}
\UnaryInfC{$\ll \Gamma \vdash A \red^* A \rr = id : A \to A \dotsb \Gamma$}
\DisplayProof
\end{center}

где $id : \Gamma.A \to \Gamma.A$ - тождественное отображение.

\medskip
\begin{center}
\AxiomC{$\ll \Gamma \vdash A \red B \rr = f_1 : A \to B \dotsb \Gamma$}
\AxiomC{$\ll \Gamma \vdash B \red^* C \rr = f_2 : B \to C \dotsb \Gamma$}
\BinaryInfC{$\ll \Gamma \vdash A \red^* C \rr = f_2 \circ f_1 : A \to C \dotsb \Gamma$}
\DisplayProof
\end{center}
\bigskip

Интерпретация правил редукций термов:

\medskip
\begin{center}
\AxiomC{$\ll \Gamma \vdash a \Downarrow A \rr = a : A \dotsb \Gamma$}
\UnaryInfC{$\ll \Gamma \vdash a \red^* a \Downarrow A \rr = h : I \to A \dotsb \Gamma$}
\DisplayProof
\end{center}

где $h : \Gamma \times I \to \Gamma.A$ определяется как композиция $\Gamma \times I \xrightarrow{\pi_1} \Gamma \xrightarrow{a} \Gamma.A$.

\medskip
\begin{center}
\AxiomC{$\ll \Gamma \vdash a \red a' \Downarrow A \rr = h_1 : I \to A \dotsb \Gamma$}
\AxiomC{$\ll \Gamma \vdash a' \red^* a'' \Downarrow A \rr = h_2 : I \to A \dotsb \Gamma$}
\BinaryInfC{$\ll \Gamma \vdash a \red^* a'' \Downarrow A \rr = h_3 : I \to A \dotsb \Gamma$}
\DisplayProof
\end{center}
\bigskip

Данное правило применимо при условии, что $h_1 \circ (id_\Gamma \times i_1) = h_2 \circ (id_\Gamma \times i_0)$.
Тогда $h_3$ определяется как отображение, соответствующее композиции $\Gamma \xrightarrow{\langle id_\Gamma, i_1 \rangle} \Gamma \times I \xrightarrow{s} \Gamma.A^I$, где $s$ определяется следующим образом:
\[ \xymatrix{ \Gamma \ar[rrr]^{\overline{h_1}} \ar[d]_{\langle id_\Gamma, i_0 \rangle}                                        & & & \Gamma.A^I \ar[d] \\
              \Gamma \times I \ar[rrr]_-{\langle h_1 \circ (id_\Gamma \times i_0) \circ \pi_1, h_2 \rangle} \ar@{-->}[urrr]^s & & & \Gamma.A \times \Gamma.A
            }\]

\begin{comment}
Пусть $\Gamma.A \xrightarrow{t} P(\Gamma.A) \xrightarrow{\langle p_1, p_2 \rangle} \Gamma.A \times \Gamma.A$ - объект путей для $\Gamma.A$.

Во-первых, определим $h'_1 : \Gamma \to P(\Gamma.A)$ как композицию $p \circ \langle id_\Gamma, i_1 \rangle$, где $p$ определяется следующим образом:
\[ \xymatrix{ \Gamma \ar[r]^-a \ar[d]_{\langle id_\Gamma, i_0 \rangle} & \Gamma.A \ar[r]^-t & P(\Gamma.A) \ar[d]^{\langle p_1, p_2 \rangle} \\
              \Gamma \times I \ar[rr]_-{\langle a \circ \pi_1, h_1 \rangle} \ar@{-->}[urr]^p & & \Gamma.A \times \Gamma.A
            }\]
Во-вторых, определим $h'_3 : \Gamma \to P(\Gamma.A)$ как композицию $q \circ \langle id_\Gamma, i_1 \rangle$, где $q$ определяется следующим образом:
\[ \xymatrix{ \Gamma \ar[rr]^{h'_1} \ar[d]_{\langle id_\Gamma, i_0 \rangle}                  & & P(\Gamma.A) \ar[d]^{\langle p_1, p_2 \rangle} \\
              \Gamma \times I \ar[rr]_-{\langle a \circ \pi_1, h_2 \rangle} \ar@{-->}[urr]^q & & \Gamma.A \times \Gamma.A
            }\]
Теперь определим $h_3 : \Gamma \times I \to \Gamma.A$ как композицию $p_2 \circ s$, где $s$ определяется следующим образом:
\[ \xymatrix{ \Gamma \amalg \Gamma \ar[rr]^{[t \circ a, h'_3]} \ar[d] & & P(\Gamma.A) \ar[d]^{p_1} \\
              \Gamma \times I \ar[r]_-{\pi_1} \ar@{-->}[urr]^s & \Gamma \ar[r]_-a & \Gamma.A
            }\]
\end{comment}

\subsection{Свойства интерпретации}

\begin{lem}[completeness][Completeness of the interpretation]
Верны следующие утверждения:
\begin{enumerate}
\item Если $\ll \Gamma \vdash \rr = \Gamma'$, то $\Gamma \vdash$.
\item Если $\ll \Gamma \vdash A \rr = A' \dotsb \Gamma'$, то $\Gamma \vdash A$.
\item Если $\ll \Gamma \vdash a \Downarrow A \rr = a' : A' \dotsb \Gamma'$, то $\Gamma \vdash a \Downarrow A$.
\item Если $\ll \Gamma \vdash a \Uparrow A \rr = a' : A' \dotsb \Gamma'$, то $\Gamma \vdash a \Uparrow A$.
\item Если $\ll \Gamma \vdash A \red B \rr = A' \to B' \dotsb \Gamma'$, то $\Gamma \vdash A$, $\Gamma \vdash B$, и $A \red B$.
\item Если $\ll \Gamma \vdash A \red^* B \rr = A' \to B' \dotsb \Gamma'$, то $\Gamma \vdash A$, $\Gamma \vdash B$, и $A \red^* B$.
\item Если $\ll \Gamma \vdash a \red b \Downarrow A \rr = h : I \to A' \dotsb \Gamma'$, то $\Gamma \vdash a \Downarrow A$, $\Gamma \vdash b \Downarrow A$, и $a \red b$.
\item Если $\ll \Gamma \vdash a \red^* b \Downarrow A \rr = h : I \to A' \dotsb \Gamma'$, то $\Gamma \vdash a \Downarrow A$, $\Gamma \vdash b \Downarrow A$, и $a \red^* b$.
\end{enumerate}
\end{lem}
\begin{proof}
Индукцией по определению отношения интерпретации.
\end{proof}

\begin{lem}[functionality][Functionality of the interpretation]
Верны следующие утверждения:
\begin{enumerate}
\item Если $\ll \Gamma \vdash \rr = \Gamma_1$ и $\ll \Gamma \vdash \rr = \Gamma_2$, то $\Gamma_1 = \Gamma_2$.
\item Если $\ll \Gamma \vdash A \rr = A_1 \dotsb \Gamma_1$ и $\ll \Gamma \vdash A \rr = A_2 \dotsb \Gamma_2$, то $\Gamma_1 = \Gamma_2$, и $A_1 = A_2$.
\item Если $\ll \Gamma \vdash a \Downarrow A \rr = a_1 : A_1 \dotsb \Gamma_1$ и $\ll \Gamma \vdash a \Downarrow A \rr = a_2 : A_2 \dotsb \Gamma_2$, то $\Gamma_1 = \Gamma_2$, $A_1 = A_2$, и $a_1 = a_2$.
\item Если $\ll \Gamma \vdash a \Uparrow A_1 \rr = a_1 : A'_1 \dotsb \Gamma_1$ и $\ll \Gamma \vdash a \Uparrow A_2 \rr = a_2 : A'_2 \dotsb \Gamma_2$, то $A_1 = A_2$, $\Gamma_1 = \Gamma_2$, $A'_1 = A'_2$, и $a_1 = a_2$.
\item Если $\ll \Gamma \vdash A \red B_1 \rr = A_1 \to B'_1 \dotsb \Gamma_1$ и $\ll \Gamma \vdash A \red B_2 \rr = A_2 \to B'_2 \dotsb \Gamma_2$, то $B_1 = B_2$, $\Gamma_1 = \Gamma_2$, и $A_1 = A_2$, и $B'_1 = B'_2$.
\item Если $\ll \Gamma \vdash A \red^* B_1 \rr = A_1 \to B'_1 \dotsb \Gamma_1$, $\ll \Gamma \vdash A \red^* B_2 \rr = A_2 \to B'_2 \dotsb \Gamma_2$, и $B_1. B_2 \in NF$, то $B_1 = B_2$, $\Gamma_1 = \Gamma_2$, и $A_1 = A_2$, и $B'_1 = B'_2$.
\item Если $\ll \Gamma \vdash a \red b_1 \Downarrow A \rr = h_1 : I \to A_1 \dotsb \Gamma_1$ и $\ll \Gamma \vdash a \red b_2 \rr = h_2 : I \to A_2 \dotsb \Gamma_2$, то $b_1 = b_2$, $\Gamma_1 = \Gamma_2$, и $A_1 = A_2$, и $h_1 = h_2$.
\item Если $\ll \Gamma \vdash a \red^* b_1 \Downarrow A \rr = h_1 : I \to A_1 \dotsb \Gamma_1$ и $\ll \Gamma \vdash a \red^* b_2 \rr = h_2 : I \to A_2 \dotsb \Gamma_2$, и $b_1, b_2 \in NF$, то $b_1 = b_2$, $\Gamma_1 = \Gamma_2$, и $A_1 = A_2$, и $h_1 = h_2$.
\end{enumerate}
\end{lem}
\begin{proof}
Индукцией по определению отношения интерпретации.

В шестом и восьмом пунктах используется полонота интерпретации и тот факт, что отношение $\red$ иррефлексивно, что следует из сильной нормализуемости.
\end{proof}

Таким образом отношения $\ll - \rr = -$ определяют частичные функции, которые мы будем обозначать $\ll - \rr$.

\begin{comment}
Пусть $\Gamma, x_1 : A_1, \ldots x_n : A_n$ - некоторые контексты, $\Gamma'$ - некоторый объект, а $A'_1, \ldots A'_n$ - последовательность такая, что $A'_1 \dotsb \Gamma'$ и $A'_{i+1} \dotsb \Gamma.A'_1.A'_2 \ldots A'_i$.
Тогда мы будем писать, что $\ll \Gamma \backslash x_1 : A_1, \ldots x_n : A_n \rr = (A'_1, \ldots A'_n) \dotsb \Gamma'$, если $\ll \Gamma \vdash \rr = \Gamma'$ и для любого $i$ верно, что $\ll \Gamma, x_1 : A_1, \ldots x_{i-1} : A_{i-1} \vdash A_i \rr = A'_i \dotsb \Gamma'.A'_1.A'_2 \ldots A'_{i-1}$.

\begin{lem}[correctness-subst][Correctness of the substitution]
Пусть $\ll \Gamma \vdash a : A \rr = a : A \dotsb \Gamma$.
Тогда верны следующие утверждения:
\begin{enumerate}
\item Если $\ll \Gamma, x : A, \Delta \vdash \rr = D$, то 
\item Если $\ll \Gamma, x : A, \Delta \vdash B \rr = B \dotsb D$
\end{enumerate}
\end{lem}
\end{comment}

\begin{lem}[weakening][Weakening lemma]
Пусть $\ll \Gamma \vdash B\rr = B \dotsb \Gamma$ и $x \notin \Gamma$.
Тогда верны следующие утверждения:
\begin{enumerate}
\item Если $\ll \Gamma \vdash A \rr = A \dotsb \Gamma$, то $\ll \Gamma, x : B \vdash A \rr = \pi_B^*(A) \dotsb \Gamma.B$
\item Если $\ll \Gamma \vdash a : A \rr = a : A \dotsb \Gamma$, то $\ll \Gamma, x : B \vdash A \rr = \langle id_{\Gamma.B}, a \circ \pi_B \rangle : \pi_B^*(A) \dotsb \Gamma.B$
\end{enumerate}
\end{lem}

\begin{lem}[correctness-typing][Corectness of the typing]
Верны следующие утверждения:
\begin{enumerate}
\item Если $\ll \Gamma \vdash A \rr = A' \dotsb \Gamma'$, то $\ll \Gamma \vdash \rr = \Gamma'$.
\item Если $\ll \Gamma \vdash a \Uparrow A \rr = a : A' \dotsb \Gamma'$, то $\ll \Gamma \vdash A \rr = A' \dotsb \Gamma'$.
\item Если $\ll \Gamma \vdash a \Downarrow A \rr = a : A' \dotsb \Gamma'$, то $\ll \Gamma \vdash A \rr = A' \dotsb \Gamma'$.
\item Если $\ll \Gamma \vdash A \red B \rr = f : A' \to B' \dotsb \Gamma'$, то $\ll \Gamma \vdash A \rr = A' \dotsb \Gamma'$ и $\ll \Gamma \vdash B \rr = B' \dotsb \Gamma'$.
\item Если $\ll \Gamma \vdash A \red^* B \rr = f : A' \to B' \dotsb \Gamma'$, то $\ll \Gamma \vdash A \rr = A' \dotsb \Gamma'$ и $\ll \Gamma \vdash B \rr = B' \dotsb \Gamma'$.
\item Если $\ll \Gamma \vdash a \red b \Downarrow A \rr = h : I \to A' \dotsb \Gamma'$, то $\ll \Gamma \vdash a \Downarrow A \rr = h \circ \langle id_{\Gamma'}, i_0 \rangle : A' \dotsb \Gamma'$ и $\ll \Gamma \vdash b \Downarrow A \rr = h \circ \langle id_{\Gamma'}, i_1 \rangle : A' \dotsb \Gamma'$.
\item Если $\ll \Gamma \vdash a \red^* b \Downarrow A \rr = h : I \to A' \dotsb \Gamma'$, то $\ll \Gamma \vdash a \Downarrow A \rr = h \circ \langle id_{\Gamma'}, i_0 \rangle : A' \dotsb \Gamma'$ и $\ll \Gamma \vdash b \Downarrow A \rr = h \circ \langle id_{\Gamma'}, i_1 \rangle : A' \dotsb \Gamma'$.
\end{enumerate}
\end{lem}

\begin{lem}[correctness-reductions][Corectness of the interpretation of reductions]
Верны следующие утверждения:
\begin{enumerate}
\item Если $\ll \Gamma \vdash A \rr = A' \dotsb \Gamma'$, и $A \red B$, то $\ll \Gamma \vdash A \red B \rr$ определенно.
\item Если $\ll \Gamma \vdash A \rr = A' \dotsb \Gamma'$, и $A \red^* B$, то $\ll \Gamma \vdash A \red^* B \rr$ определенно.
\item Если $\ll \Gamma \vdash a \Downarrow A \rr = a' : A' \dotsb \Gamma'$, и $a \red b$, то $\ll \Gamma \vdash a \red b \Downarrow A \rr$ определенно.
\item Если $\ll \Gamma \vdash a \Downarrow A \rr = a' : A' \dotsb \Gamma'$, и $a \red^* b$, то $\ll \Gamma \vdash a \red^* b \Downarrow A \rr$ определенно.
\end{enumerate}
\end{lem}
\begin{proof}
Второй и четвертый пункты элементарно следуют из первого и третьего соответственно.
Первый и третий пункты на данный момент тривиальны, т.к. мы не ввели пока никаких правил редукций.
\end{proof}

\begin{lem}[correctness][Corectness of the interpretation]
Верны следующие утверждения:
\begin{enumerate}
\item Если $\Gamma \vdash$, то $\ll \Gamma \vdash \rr$ определенно.
\item Если $\Gamma \vdash A$, то $\ll \Gamma \vdash A \rr$ определенно.
\item Если $\Gamma \vdash a \Downarrow A$, то $\ll \Gamma \vdash a \Downarrow A \rr$ определенно.
\item Если $\Gamma \vdash a \Uparrow A$, то $\ll \Gamma \vdash a \Uparrow A \rr$ определенно.
\end{enumerate}
\end{lem}

\bibliographystyle{amsplain}
\bibliography{ref}

\end{document}
